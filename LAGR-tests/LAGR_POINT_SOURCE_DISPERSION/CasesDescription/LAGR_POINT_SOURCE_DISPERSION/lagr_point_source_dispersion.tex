\renewcommand{\IMAGES}{../LAGR_POINT_SOURCE_DISPERSION/IMAGES}

\newcommand{\lra}[1]{\langle #1 \rangle }
\chapter{Lagrangian Point Source Dispersion}
\sousti{Christophe Henry, (INRIA Sophia Antipolis), Martin Ferrand, RENUDA}{25/09/20}
\keywords{3D, Lagrangian, Dispersion}
%--------------------------------------------------------------------------------------------------
\label{LAGR:case:lagr_point_source_dispersion}
%--------------------------------------------------------------------------------------------------
%%%%%%%%%%%%%%%%%%%%%%%%%%%%%%%%%%%%%%%%%%%%%%%%%%%%%%%%%
%                                                       %
%            Start of test case description             %
%                                                       %
%%%%%%%%%%%%%%%%%%%%%%%%%%%%%%%%%%%%%%%%%%%%%%%%%%%%%%%%%
%-----------------------------------

This test case corresponds to a verification case for the Lagrangian module in \CS.
It consists of point source dispersion.

\section{Isotropic test case}
%-----------------------------------
\subsection{Introduction}
%........................
This test case consists in a fluid at rest in a cubic mesh (no flow) inside which particles are initially injected  in the centre of the domain.

To verify the present algorithm for the treatment of the equations of particle motion, numerical results are compared to analytical results in ideal cases that correspond to limiting cases (with constant coefficients).

This test case consists in an isotropic case as the one described in \cite{Rep1} and \cite{Rep2}: constant coefficients $C_i(t,\mathbf{x}_p) = 0$, $\mathbf{A}_i(t,\mathbf{x}_p) = 0$ and initial conditions $x_p(0)=U_p(0)=U_s(0)=0$.
In that case, the system simplifies to
\begin{equation} \label{eq:lagr:num_schem:syst}
\left\{\begin{aligned}
& dx_p = U_p\, dt \\
& dU_p = \frac{1}{\tau_p}(U_s-U_p)\, dt \\
& dU_s = -\frac{1}{T}U_s\,dt + \sigma\, dW(t),
\end{aligned}\right.
\end{equation}
and the solution of the system is
\begin{equation} \label{eq:lagr:num_schem:solved}
\left\{\begin{aligned}
& x_p = \Omega(t) \\
& U_p = \Gamma(t) \\
& U_s = \gamma(t),
\end{aligned}\right.
\end{equation}

The present test case was updated compared to the one detailed in \cite{Rep2}. Previously, the particles were injected at an inlet face and then displaced to the centre.  This led to having a combination of 5 symmetry boundary conditions and one inlet boundary condition.  Here, the particles are injected directly in the cell, which makes it possible to impose symmetry boundary conditions on all the cell faces and leads to slightly better results.

\subsection{Theory}
The dynamics of discrete particles and the corresponding system of SDEs has been described in details in \cite{Rep3} and only the main features are recalled here. The system of SDEs describing the dynamics of the discrete particles reads
\begin{equation}
\left\{
\begin{aligned} \label{eq:lagr:num_scheme:sysEDS}
dx_{p,i}(t) & = U_{p,i}\, dt, \\
dU_{p,i}(t) & =\frac{1}{\tau_p}(U_{s,i}-U_{p,i})\, dt + \mathcal{A}_i\,dt,\\
dU_{s,i}(t) & =-\frac{1}{T_{L,i}^*}U_{s,i}\,dt+C_i\, dt + \sum_j
B_{ij}\, dW_j(t),
\end{aligned} \right .
\end{equation}
where $C_i$ is a term that includes all mean contributions: the mean pressure gradient, $-(\partial \lra{P}/\partial x_i)/\rho_f$, the mean drift term, $(\lra{U_{p,j}}-\lra{U_j})(\partial \lra{U_i}/\partial x_j)$, and the mean part of the return-to-equilibrium term, $\lra{U_i}/T^*_{L,i}$. $\mathcal{A}_i$ is an acceleration (gravity in the present work, but it can be extended for practical reasons to the case of other external force fields).

The weak numerical schemes, with the required features, are developed based on the analytical solution to Eqs.~(\ref{eq:lagr:num_scheme:sysEDS}) {\it with constant coefficients} (independent of time), the main idea being to derive a numerical scheme by freezing the coefficients on the integration intervals. This methodology ensures \textit{stability} and \textit{consistency with all limit systems}:
\begin{enumerate}
\item[-] stability because the form of the equations gives analytical solutions with exponentials of the type $\exp(-\Delta t/T)$ where $T$ is one of the characteristic timescales ($\tau_p$ and $T_{L,i}^*$),
\item[-] consistency with all limit systems by construction, since the schemes are based on an analytical solution.
\end{enumerate}
Different techniques shall be used to derive first and second-order (in time) schemes from the analytical solutions with constant coefficients. A first-order scheme can be obtained by computing, at each time step, the variables on the basis of the analytical solutions (all coefficients are frozen at the beginning of the integration interval), i.e. a numerical scheme of the {\it Euler} kind is obtained. A second-order scheme can be derived by resorting to a predictor-corrector technique where the prediction step is the first-order scheme.

\paragraph{Analytical solution} Before presenting the weak numerical schemes, it is a prerequisite to give the analytical solutions to system (\ref{eq:lagr:num_scheme:sysEDS}), with constant coefficients (in time). These solutions are obtained by resorting to It\^o's calculus in combination with the method of the variation of the constant. For instance, for the fluid velocity seen, one seeks a solution of the form $U_{s,i}(t)=H_i(t)\exp(-t/T_i)$, where $H_i(t)$ is a stochastic process defined by (\textit{from now on the notation is slightly changed: $T_{L,i}^*$ is noted $T_i$ for the sake of clarity in the complex formulae to come})
\begin{equation} \label{eq:H}
dH_i(t)=\exp(t/T_i)[C_i\,dt + \Check{B}_i\,dW_i(t)],
\end{equation}
that is, by integration on a time interval $[t_0,t]$ ($\Delta t=t-t_0$),
\begin{equation}
\begin{split}
U_{s,i}(t) = U_{s,i}(t_0)& \exp(-\Delta t/T_i)+ C_i
             \,T_i\,[1-\exp(-\Delta t/T_i)] \\
+ & \Check{B}_i \exp(-t/T_i)
    \int_{t_0}^{t}\exp(s/T_i)\,dW_i(s),
\end{split}
\end{equation}
where $\Check{B}_i=B_{ii}$ since $B_{ij}$ is a diagonal matrix). By proceeding in the same way for the other equations (position and velocity), the analytical solution is obtained for the entire system, cf. Table \ref{tab:lagrangian:exa}. The three stochastic integrals, Eqs. (\ref{eq:lagr:num_scheme:gamma_exa}) to (\ref{eq:lagr:num_scheme:Omega_exa}) in Table \ref{tab:lagrangian:exa}, are centred Gaussian processes. These integrals are defined implicitly, but they can be simplified by integration by parts, cf. Table \ref{tab:lagrangian:exa}.

\begin{table}[htbp]
\caption{Analytical solutions to system (\ref{eq:lagr:num_scheme:sysEDS}) for time-independent coefficients.}
\hrule
\begin{align}
& x_{p,i}(t) = x_{p,i}(t_0)
  + U_{p,i}(t_0)\tau_p  [1-\exp(-\Delta t/\tau_p)]
  + U_{s,i}(t_0)\,\theta_i \{T_i[1-\exp(-\Delta t/T_i)] \notag \\
&  \hspace*{5mm} + \tau_p[\exp(-\Delta t/\tau_p)-1]\}
  + [C_i\,T_i]
    \{\Delta t-\tau_p[1-\exp(-\Delta t/\tau_p)] \notag \\
&  \hspace*{5mm} - \theta_i (T_i[1-\exp(-\Delta t/T_i)]+
      \tau_p[\exp(-\Delta t/\tau_p)-1])\}
     + \Omega_i(t) \label{eq:lagr:num_scheme:xpa_exa} \\
&  \hspace*{5mm} \text{\quad with \quad} \theta_i = T_i/(T_i-\tau_p)
  \notag \\
& U_{p,i}(t) = U_{p,i}(t_0)\exp(-\Delta t/\tau_p)
  + U_{s,i}(t_0)\,\theta_i
    [\exp(-\Delta t/T_i)-\exp(-\Delta t/\tau_p)] \notag \\
& \hspace*{5mm} + [C_i\,T_i]
    \{[1-\exp(-\Delta t/\tau_p)]-\theta_i
      [\exp(-\Delta t/T_i)-\exp(-\Delta t/\tau_p)]\} \notag \\
& \hspace*{5mm} + \Gamma_i(t) \label{eq:lagr:num_scheme:Upa_exa} \\
& U_{s,i}(t) = U_{s,i}(t_0)\exp(-\Delta t/T_i)
  + C_i\,T_i[1-\exp(-\Delta t/T_i)]
  + \gamma _i(t) \label{eq:lagr:num_scheme:Ufa_exa} \\ \notag \\
%------------
& \text{\underline{The stochastic integrals $\gamma _i(t),\;\Gamma
_i(t),\;\Omega _i(t)$ are given by:}}\notag \\
& \quad \gamma _i(t) = \Check{B}_i\exp(-t/T_i)
  \int _{t_0}^{t} \exp(s/T_i)\,dW_i(s), \label{eq:lagr:num_scheme:gamma_exa}\\
& \quad \Gamma _i(t) = \frac{1}{\tau_p}\exp(-t/\tau_p)
  \int _{t_0}^{t}\exp(s/\tau_p)\,\gamma _i(s)\,ds, \label{eq:lagr:num_scheme:Gamma_exa}\\
& \quad \Omega _i(t) = \int _{t_0}^{t}\Gamma
_i(s)\,ds. \label{eq:lagr:num_scheme:Omega_exa} \\ \notag \\
%------------
& \text{\underline{By resorting to stochastic integration by parts, $\gamma
_i(t),\;\Gamma _i(t),\;\Omega _i(t)$ can be written:}}\notag \\
& \quad \gamma _i(t) = \Check{B}_i\,\exp(-t/T_i)
\,I_{1,i}, \label{eq:lagr:num_scheme:gammaN_exa}\\
& \quad \Gamma _i(t) = \theta_i \,\Check{B}_i\,
 [\exp(-t/T_i)\,I_{1,i} -\exp(-t/\tau_p)\,I_{2,i}], \label{eq:lagr:num_scheme:GammaN_exa}\\
& \quad \Omega _i(t) = \theta_i \,\Check{B}_i\,
   \{ (T_i-\tau_p)\,I_{3,i} \notag \\
& \hskip 4.0cm -[T_i \exp(-t/T_i)\,I_{1,i} -
   \tau_p \exp(-t/\tau_p) \,I_{2,i}]\}, \label{eq:lagr:num_scheme:OmegaN_exa} \\
& \text{with} \quad I_{1,i} = \int_{t_0}^{t}\exp(s/T_i)\,dW_i(s),
\quad I_{2,i}= \int_{t_0}^{t}\exp(s/\tau_p)\,dW_i(s) \notag \\
& \hskip 7.5cm \text{and} \quad I_{3,i}=\int_{t_0}^{t}dW_i(s).\notag
\end{align}
\hrule
\label{tab:lagrangian:exa}
\end{table}

\paragraph{Weak first-order scheme} From the analytical solutions of this system assuming constant coefficients, the weak-first order scheme is extracted. The equations are described in Table~\ref{tab:lagrangian:sch1}.

\begin{table}[htbp]
\caption{Weak first-order scheme (Euler scheme)}
\hrule
\begin{align}
& \text{\underline{Numerical integration of the system:}}\notag \\
& \quad x_{p,i}^{n+1} = x_{p,i}^n + A_1\,U_{p,i}^n + B_1\,U_{s,i}^n
  + C_1\,[T_i^n C_i^n] + \Gamma _i^n,
  \notag \\
& \quad U_{p,i}^{n+1} = U_{p,i}^n\, \exp(-\Delta t/\tau_p^n)
              + D_1\,U_{s,i}^n + [T_i^n C_i^n](E_1-D_1)
              + \Omega _i^n, \notag \\
& \quad U_{s,i}^{n+1} = U_{s,i}^n\, \exp(-\Delta t/T_i^n)
              + [T_i^n C_i^n] [1-\exp(-\Delta t/T_i^n)]
              + \gamma _i^n.\notag \\ \notag \\
%----------------
& \text{\underline{The coefficients $A_1,\;B_1,\;C_1,\;D_1$ and $E_1$ are
  given by:}}\notag \\
& \quad A_1 = \tau_p^n\,[1-\exp(-\Delta t/\tau_p^n)],\notag\\
& \quad B_1 = \theta_i ^n\,[T_i^n(1-\exp(-\Delta t/T_i^n)-A_1]
      \quad \text{with}\quad \theta_i ^n = T_i^n/(T_i^n-\tau_p^n),\notag\\
& \quad C_1 = \Delta t - A_1 - B_1, \notag \\
& \quad D_1 = \theta_i ^n [\exp(-\Delta t/T_i^n)-\exp(-\Delta
  t/\tau_p^n)],\notag\\
& \quad E_1 = 1 - \exp(-\Delta t/\tau_p^n).\notag \\ \notag \\
%----------------
& \text{\underline{The stochastic integrals $\gamma _i^n,\;\Omega
_i^n,\;\Gamma_i^n$ are simulated by:}}\notag \\
& \quad \gamma_i^n = P_{11}\,{\mathcal G}_{1,i},\notag \\
& \quad \Omega_i^n = P_{21}\,{\mathcal G}_{1,i}+ P_{22}\,{\mathcal G}_{2,i} \notag \\
& \quad \Gamma_i^n = P_{31}\,{\mathcal G}_{1,i}+ P_{32}\,{\mathcal G}_{2,i}+
                     P_{33}\,{\mathcal G}_{3,i}, \notag\\
& \quad \text{where ${\mathcal G}_{1,i},\;{\mathcal G}_{2,i},\;{\mathcal G}_{3,i}$ are
  independent ${\cal N}(0,1)$ random variables.} \notag\\ \notag \\
%----------------
& \text{\underline{The coefficients
 $P_{11},\;P_{21},\;P_{22},\;P_{31},\;P_{32},\;P_{33}$ are defined
 as:}}\notag\\
& \quad P_{11} = \sqrt{\lra{(\gamma_i^n)^2}}, \notag \\
& \quad P_{21} = \frac{\lra{\Gamma_i^n\gamma_i^n}}{\sqrt{\lra{(\gamma_i^n)^2}}},
  \quad P_{22} = \sqrt{\lra{(\Gamma_i^n)^2}-
  \frac{\lra{\Gamma_i^n\gamma_i^n}^2}{\lra{(\gamma_i^n)^2}}},\notag \\
& \quad P_{31} = \frac{\lra{\Omega_i^n\gamma_i^n}}{\sqrt{\lra{(\gamma_i^n)^2}}},
  \quad P_{32} = \frac{1}{P_{22}}(\lra{\Omega_i^n\Gamma_i^n}-P_{21}P_{31}),
  \quad P_{33} = \sqrt{\lra{(\Omega_i^n)^2}-P_{31}^2-P_{32}^2)}.\notag
\end{align}
\hrule
\label{tab:lagrangian:sch1}
\end{table}

\paragraph{Weak second-order scheme} The weak second-order scheme consists in a corretion step for the particle velocity and the velocity of the fluid seen as described in Table~\ref{tab:lagrangian:sch2}:
\begin{table}[htbp]
\caption{Weak second-order scheme}
\hrule
\begin{align}
& \text{\underline{Prediction step:}}
\quad \text{Euler scheme, see Table \ref{tab:lagrangian:sch1} (predicted values noted with $\tilde{ }$ )} .\notag \\
%----------------
& \text{\underline{Correction step:}} \notag \\
& \quad U_{p,i}^{n+1} =
    \frac{1}{2}\,U_{p,i}^n\, \exp(-\Delta t/\tau_p^n)
  + \frac{1}{2}\,U_{p,i}^n\,\exp(-\Delta t/\tilde{\tau }_p^{n+1})\notag \\
& \qquad \qquad \qquad \qquad + \frac{1}{2}\,U_{s,i}^n\,C_{2c}(\tau_p^n,\,T_i^n)
  + \frac{1}{2}\,U_{s,i}^n\,C_{2c}(\tilde{\tau}_p^{n+1},\,\tilde{T_i}^{n+1})
\notag \\
& \qquad \qquad \qquad \qquad + A_{2c}(\tau_p^n,\,T_i^n) \,[T_i^n C_i^n]
  + B_{2c}(\tilde{\tau}_p^{n+1},\,\tilde{T}_i^{n+1})\,[\tilde{T}_i^{n+1} C_i^{n+1}]\notag \\
& \qquad \qquad \qquad \qquad + A_2(\Delta t,\tau_p^n)[\tau_p^n\,{\cal A}_i^n]
  + B_2(\Delta t,\tilde{\tau_p}^{n+1})[\tilde{\tau_p}^{n+1}\,{\cal A}_i^{n+1}]
  + \tilde{\Gamma}_i^{n+1}, \notag \\
& \quad U_{s,i}^{n+1} = \frac{1}{2}\,U_{s,i}^n\,\exp(-\Delta t/T_i^n)
+ \frac{1}{2}\,U_{s,i}^{n}\,\exp(-\Delta t/\tilde{T}_i^{n+1})
+ A_2(\Delta t,\,T_i^n) \,[T_i^n C_i^n]\notag \\
& \qquad \qquad \qquad \qquad
+ B_2(\Delta t,\,\tilde{T}_i^{n+1}) \,[\tilde{T}_i^{n+1} C_i^{n+1}]
+ \tilde{\gamma}_i^{n+1}.\notag\\
%----------------
& \text{\underline{The coefficients $A_2,\;B_2,\;A_{2c},\;B_{2c}$ et $C_{2c}$ are
  defined as:}}\notag \\
& \quad A_2(\Delta t,x) = -\exp(-\Delta t/x)
+ [1-\exp(-\Delta t/x)][x/\Delta t], \notag\\
& \quad B_2(\Delta t,x) = 1-[1-\exp(-\Delta t/x)][x/\Delta t],\notag \\
& \quad A_{2c}(x,y) =
- \exp(-\Delta t/x) + [(x+y)/\Delta t][1-\exp(-\Delta t/x)]
- (1+y/\Delta t)\,C_{2c}(x,y), \notag \\
& \quad B_{2c}(x,y) =  1 - [(x+y)/\Delta t][1-\exp(-\Delta t/x)]
+ (y/\Delta t)\,C_{2c}(x,y), \notag\\
& \quad C_{2c}(x,y) = [y/(y-x)][\exp(-\Delta t/y)-\exp(-\Delta
  t/x)].\notag \\
%----------------
& \text{\underline{The stochastic integrals
        $\tilde{\gamma}_i^{n+1}\;$ \text{and} $\;\tilde{\Gamma}_i^{n+1}$
        are simulated as follows:}}\notag \\
& \quad \tilde{\gamma}_i^{n+1}= \sqrt {\frac{[B_i^{*}]^2\tilde{T}_i^{n+1}}{2}
               [1-\exp(-2\Delta t/\tilde{T}_i^{n+1})]}\; {\mathcal G}_{1,i},\notag \\
& \quad \text{with} \quad \left[1-\exp(-2\,\Delta t/\tilde{T}_i^{n+1})\right]\,B_i^{*} =
    A_2(2\,\Delta t,\,\tilde{T}_i^{n+1})\,\sqrt{(\Check{B}_i^n)^2} + \notag \\
& \hskip 8.0cm B_2(2\,\Delta t,\,\tilde{T}_i^{n+1})\,\sqrt{(\tilde{\Check{B}}_i^{n+1})^2}. \notag \\
& \quad \tilde{\Gamma}_i^{n+1} =
  \frac{\lra{ \tilde{\Gamma}_i^{n+1}\tilde{\gamma}_i^{n+1} } }
     {\lra{(\tilde{\gamma}_i^{n+1})^2}}\,\tilde{\gamma}_i^{n+1}+
  \sqrt{\lra{(\tilde{\Gamma}_i^{n+1})^2}-
      \frac{\lra{\tilde{\Gamma}_i^{n+1}\tilde{\gamma}_i^{n+1}}^2}
           {\lra{(\tilde{\gamma}_i^{n+1})^2}} }\;{\mathcal G}_{2,i} \notag \\ \notag \\
& \quad \text{with}\quad \lra{ \tilde{\Gamma}_i^{n+1}\tilde{\gamma}_i^{n+1} } =
  \lra{\Gamma_i\gamma_i}(\tau_p^n,\,\tilde{T}_i^{n+1},\,B^{*}_i)
  \quad \text{and}\quad \lra{(\tilde{\Gamma}_i^{n+1})^2} =
  \lra{\Gamma_i^2}(\tau_p^n,\,\tilde{T}_i^{n+1},\,B^{*}_i). \notag
\end{align}
\hrule
\label{tab:lagrangian:sch2}
\end{table}

\clearpage

\subsection{References}

\begin{thebibliography}{3}

   \bibitem{Rep1} J.-P. Minier, E. Peirano, and S. Chibbaro. Weak first
   and second order numerical schemes for stochastic differential equations
   appearing in lagrangian two-phase flow modelling.
   {\it Monte Carlo Methods and Applications}, 9(2):93–133, 2003.

   \bibitem{Rep2} C. HENRY, J. POZORSKI
   Modelling of agglomeration and deposition of colloidal particles carried by a flow: Fourth progress report.
   {\it EDF Report}, Report-EDF-09-2016, 2016.

   \bibitem{Rep3} E. Peirano, S. Chibbaro, J. Pozorski, and J.-P. Minier.
   Mean-field/PDF numerical approach for polydispersed turbulent two-phase flows.
   {\it Progress in Energy and Combustion Science}, 32(3):315–371, 2006.

\end{thebibliography}

\section{Numerical set up}
%-----------------------------------
\subsection{Computational domain}
%..............................
To perform a numerical simulation in such an ideal case, a very simple computational domain consisting of a single cubic cell (with a size of \SI{1000}{m}) has been retained.

\begin{figure}[H]
 \centering
 \includegraphics[scale=0.25, trim = 0cm 0cm 0cm 0cm, clip]{\IMAGES/Fig_NUM_SCHEME_mesh.pdf}
 \caption{3D view of the single cell used for the verification of the numerical scheme in ideal cases.}
 \label{Fig_NUM_SCHEME_Verif_mesh}
\end{figure}

%
\subsection{Physical modeling for the fluid flow}
%..............................
The fluid is at rest (no fluid motion).
Since all parameters are fixed manually in the Lagrangian module of \CS{}, the simulation of the fluid phase is not primordial. Nevertheless, the following parameters have been retained to be able to run a two-phase flow simulation. The flow is isothermal and incompressible. Gravity has been neglected. The properties used for the liquid in the flow domain are summarised in Table~\ref{Tab_NUM_SCHEME_fluid}:
\begin{table}[H]
\begin{center}
\begin{tabular}{|c|c|c|}
\hline
$U_{inlet}$ & $\mu_f$ 			& $\rho_f$ 		\\
\hline
\SI{0}{m.s^{-1}} & \SI{1.002e-3}{kg.m^{-1}.s^{-1}} 	& \SI{998}{kg.m^{-3}}	\\
\hline
\end{tabular}
 \caption{Fluid properties used in the simulation: inlet velocity $U_{inlet}$, Reynolds number $Re$, dynamic viscosity $\mu_f$ and density $\rho_f$.}
\label{Tab_NUM_SCHEME_fluid}
\end{center}
\end{table}

A turbulence model has been activated to be able to use the particle turbulent dispersion model:
\begin{itemize}
 \item Turbulence model: $k-\varepsilon$ (with initial values to get the desired Lagrangian times)
 \item Frozen dynamic field
\end{itemize}

\paragraph{Initial conditions}
\subparagraph{} The initial conditions are the following:
\begin{itemize}
 \item reference and initial pressure: \SI{101325}{Pa}
 \item fluid velocity: \SI{0.0}{m.s^{-1}}
 \item temperature: \SI{293.15}{K}
\end{itemize}

\paragraph{Boundary conditions}

\subparagraph{} All the faces of the computational domain are defined as symmetry boundary conditions. The particles are injected directly at the centre of the computational domain.

\subsection{Physical modelling for the particles}

\subparagraph{Injection} Particles are injected in the domain. The properties used for the injected particles are:

\begin{itemize}
 \item Particle diameter: \SI{1e-3}{m}
 \item Monodispersed particles
 \item Particle density: \SI{998}{kg.m^{-3}}
 \item Frequency of injection: 0 (only initially)
 \item Number of particles in class: \SI{20000}{}
\end{itemize}

\subparagraph{Boundary condition} The symmetry boundary conditions for particles have been set to

\begin{itemize}
 \item symmetry (particles zero-flux)
\end{itemize}

\subparagraph{Model for transport} The CFD simulation with the injection of particles has been performed using the Lagrangian module in \textit{Code$\_$Saturne} and the following properties
\begin{itemize}
 \item Eulerian-Lagrangian multi-phase treatment: one way coupling
 \item The continuous phase is a steady flow
 \item No additional models associated with the particles
 \item Integration for the stochastic differential equations: first-order scheme
 \item Particle turbulent dispersion model: activated
\end{itemize}
\subparagraph{Numerical parameters} The two-phase flow simulation has been performed using the following numerical parameters:
\begin{itemize}
 \item Calculation restart from the fluid flow simulation
 \item Constant time step: \SI{0.001}{s}
 \item Number of iterations: \SI{4000}{}
\end{itemize}

The subroutines cs\_user\_initialization.c, cs\_user\_lagr\_particle.c, cs\_user\_lagr\_volume\_conditions.c, cs\_user\_mesh-modify.c are coded to set-up the test case.

\begin{description}

   \item[-] \textbf{cs\_user\_initialization.c} is used to initialize the turbulence $k$ and $eps$ from $T_L$ and $\sigma$.
   \item[-] \textbf{cs\_user\_lagr\_volume\_conditions.c} is used to set-up the particle class.
   \item[-] \textbf{cs\_user\_lagr\_particle.c} is used to set-up the parameter $\tau_p$ and to put all the particles at the middle of the domain (see also Figure~\ref{Fig_NUM_SCHEME_Verif_mesh}).
   \item[-] \textbf{cs\_user\_mesh-modify.c} is used to center and scale the domain to \SI{1000}{m}.

\end{description}

The various cases studied are summarised in Table~\ref{Tab_NUM_SCHEME_cases}:

\begin{table}[H]
\begin{center}
\begin{tabular}{|c|c|c|c|}
\hline
Case 		& $\tau_p$	 & $T_L$	& $\sigma$ 		\\
		& (s)		 & (s)		& (\SI{}{m.s^{-3/2}}) 	\\
\hline
 General case	& \SI{e-1}{} 	& \SI{2e-1}{}	& \SI{e1}{}	\\
\hline
 Limit case I	& \SI{e-5}{} 	& \SI{e-1}{}	& \SI{e1}{}	\\
\hline
 Limit case II	& \SI{e-1}{} 	& \SI{e-5}{}	& \SI{e3}{}	\\
\hline
 Limit case III	& \SI{2e-5}{} 	& \SI{e-5}{}	& \SI{e3}{}	\\
\hline
\end{tabular}
 \caption{Values of the fixed parameters in the various cases studied. General case: $\Delta t \ll T_L, \tau_p$. Limit case I: $\tau_p \ll \Delta t \ll T_L$. Limit case II: $T_L \ll \Delta t \ll \tau_p$. Limit case III: $\tau_p, T_L \ll \Delta t $.}
\label{Tab_NUM_SCHEME_cases}
\end{center}
\end{table}

%------------------------------------------
\section{Results}
%------------------------------------------

Two-phase flow simulations of system~\ref{eq:lagr:num_schem:syst} have been performed in the various cases previously mentioned. The system considered consists in purely diffusive motion of particles, as seen in Figure~\ref{Fig_NUM_SCHEME_Verif_part}.
\begin{figure}[H]
  \centering
  \subfigure[Zoom on the initial configuration]{
    \includegraphics[width=0.48\textwidth, trim = 0cm 0cm 15cm 0cm, clip]{\IMAGES/Fig_NUM_SCHEME_part_init.pdf}
  }
  \subfigure[Zoom on the final configuration]{
    \includegraphics[width=0.48\textwidth,trim = 0cm 0cm 15cm 0cm, clip]{\IMAGES/Fig_NUM_SCHEME_part_end.pdf}
  }
  \caption{3D view of the particles inside single cell (zoom) at the initial and final time step.}
  \label{Fig_NUM_SCHEME_Verif_part}
\end{figure}

Results are displayed in Figures~\ref{Fig_NUM_SCHEME_GENERAL_CASE}-\ref{Fig_NUM_SCHEME_LIMIT_CASE_III} for the second-order moments (first-order moments are omitted since the solutions of system~\ref{eq:lagr:num_schem:syst} are Gaussian random variables with a mean equal to zero) in the various cases. It can be seen that all numerical results are close to the analytical solution.
\begin{figure}[H]
    \centering
    \includegraphics[width=1.0\textwidth]{\IMAGES/VNV/Fig_Num_Scheme_General_CASE.pdf}
  \caption{Comparison of numerical results for system~\ref{eq:lagr:num_schem:syst} and analytical solutions (lines) in the general case.}
    \label{Fig_NUM_SCHEME_GENERAL_CASE}
\end{figure}

\begin{figure}[H]
    \centering
    \includegraphics[width=1.0\textwidth]{\IMAGES/VNV/Fig_Num_Scheme_Limit_CASE1.pdf}
  \caption{Comparison of numerical results for system~\ref{eq:lagr:num_schem:syst} and analytical solutions (lines) in the limit case I.}
  \label{Fig_NUM_SCHEME_LIMIT_CASE_I}
\end{figure}

\begin{figure}[H]
    \centering
    \includegraphics[width=1.0\textwidth]{\IMAGES/VNV/Fig_Num_Scheme_Limit_CASE2.pdf}
  \caption{Comparison of numerical results for system~\ref{eq:lagr:num_schem:syst} and analytical solutions (lines) in the limit case II.}
  \label{Fig_NUM_SCHEME_LIMIT_CASE_II}
\end{figure}

\begin{figure}[H]
    \centering
    \includegraphics[width=1.0\textwidth]{\IMAGES/VNV/Fig_Num_Scheme_Limit_CASE3.pdf}
  \caption{Comparison of numerical results for system~\ref{eq:lagr:num_schem:syst} and analytical solutions (lines) in the limit case III.}
  \label{Fig_NUM_SCHEME_LIMIT_CASE_III}
\end{figure}

\section{Anisotropic test case}
%........................
\subsection{Description}
%........................
A variant of this test case imposes anisotropic time values which means that each spatial coordinate has a different value of $T_L$. For this test we use the values:

\begin{table}[H]
    \begin{center}
        \begin{tabular}{|c|c|c|c|}
            \hline
            Case 		& $T_Lx$  & $T_Ly$  & $T_Lz$ \\
                		& (s)	  & (s)     & (s)    \\
            \hline
            Case I	    & 0.4    & 0.2    & 0.2	     \\
            \hline
            Case II	    & 0.2    & 0.4    & 0.2	     \\
            \hline
            Case III	& 0.2    & 0.2    & 0.4	     \\
            \hline
        \end{tabular}
        \caption{Values of the fixed $T_L$ along the three spatial axes ($T_Lx$, $T_Ly$, $T_Lz$).}
        \label{Tab_NUM_SCHEME_ANISO_cases}
    \end{center}
\end{table}

For each one of these test cases all other parameters are set equal to those of the isotropic General Case.

\subsection{User coding}
%........................
Based on this table we see that Case I has a high $T_L$ value for the X axis, Case II has a high value for the Y axis and Case III has a high value for the Z axis. Note that imposing such values to the solver requires the addition of more user coding that forces the code to use these values instead of computing them.

The purpose of this set of variations of the original isotropic test case is to validate the algorithm that performs a rotation of the SDEs to a local system of coordinates before their solution as well as their rotation back to the global system of coordinates. This process occurs in function \textbf{\_lages1} and file \textbf{cs\_lagr\_sde.c} which updates the particle position, velocity and velocity seen in the case of a 1st order scheme for all particles in the simulation. The new particle position is computed with 5 terms entering the SDEs and 2 additional terms which take account of the Brownian motion. The new particle velocity is computed with 5 terms entering the SDEs and 1 additional term for Brownian motion while the new particle velocity seen is computed with 3 terms entering the SDEs.

A rotation is applied for spherical particles based on the mean relative velocity but for these particular test cases the rotation process is modified so that the main axis along which the SDEs are expressed coincide with the axis with the high value of $T_L$. So, for Case I the rotation is performed so that the main axis is the global X axis, for Case II the rotation is such that Y axis becomes the main axis and in Case III the rotation makes the Z axis the main axis for the computations. This is done by a modification of function \textbf{\_lages1} which requires file \textbf{cs\_lagr\_sde.c} to become part of the user coding files.

The enforcement of the specific values of $T_Lx$, $T_Ly$ and $T_Lz$ requires a modification of the part of the solver which computes these terms. These computations appear in function \textbf{cs\_lagr\_car} in file \textbf{cs\_lagr\_car.c} and the modifications to the code are such that these timing parameters are no longer computed but instead set to the specified values. Normally, a local rotation is performed at this stage based on the mean relative velocity but in this case the rotation is performed similarly to the one in \textbf{\_lages1} described previously so that the main axis is always the one with the high value of $T_L$ i.e. the X axis for Case I, the Y axis for Case II and the Z axis for Case III. In this manner we are consistent with the rotation performed in \textbf{\_lages1}. After that, all the other terms are computed accordingly so that they are consistent with the enforced timing values. Obviously, because of these modifications, file \textbf{cs\_lagr\_car.c} becomes an additional user coding file.

The expected result from these test cases is to observe an anisotropic distribution of the particles in the form of an ellipsoid with the long diameter aligned with the axis that is set to have the high value of $T_L$. An analytical solution is also possib conle by using the same equations as for the isotropic case but this time for each spatial component in separate and with the appropriate value of $T_L$.

\subsection{Distribution of particles}
%........................
In the following we will present the distribution of the particles at the end of the computations in order to verify that they have the expected ellipsoid shape and axis alignment.

\begin{figure}[H]
    \centering
    \includegraphics[scale=0.3, trim = 0cm 0cm 0cm 0cm, clip]{\IMAGES/ANISO_X-Fig1.png}
    \caption{The distribution of particles for the anisotropic Case I (X-Y plane).}
    \label{Fig_ANISO_X_Fig1}
\end{figure}

In figure~\ref{Fig_ANISO_X_Fig1} we observe the ellipsoid shape aligned with the X axis as expected.

\begin{figure}[H]
    \centering
    \includegraphics[scale=0.3, trim = 0cm 0cm 0cm 0cm, clip]{\IMAGES/ANISO_X-Fig2.png}
    \caption{The distribution of particles for the anisotropic Case I (Y-Z plane).}
    \label{Fig_ANISO_X_Fig2}
\end{figure}

In figure~\ref{Fig_ANISO_X_Fig2} we observe the circular shape of the ellipsoid on the Y-Z plane as expected.

\begin{figure}[H]
    \centering
    \includegraphics[scale=0.3, trim = 0cm 0cm 0cm 0cm, clip]{\IMAGES/ANISO_Y-Fig1.png}
    \caption{The distribution of particles for the anisotropic Case II (Y-Z plane).}
    \label{Fig_ANISO_Y_Fig1}
\end{figure}

In figure~\ref{Fig_ANISO_Y_Fig1} we observe the ellipsoid shape aligned with the Y axis as expected.

\begin{figure}[H]
    \centering
    \includegraphics[scale=0.3, trim = 0cm 0cm 0cm 0cm, clip]{\IMAGES/ANISO_Y-Fig2.png}
    \caption{The distribution of particles for the anisotropic Case II (X-Z plane).}
    \label{Fig_ANISO_Y_Fig2}
\end{figure}

In figure~\ref{Fig_ANISO_Y_Fig2} we observe the circular shape of the ellipsoid on the X-Z plane as expected.

\begin{figure}[H]
    \centering
    \includegraphics[scale=0.3, trim = 0cm 0cm 0cm 0cm, clip]{\IMAGES/ANISO_Z-Fig1.png}
    \caption{The distribution of particles for the anisotropic Case III (X-Z plane).}
    \label{Fig_ANISO_Z_Fig1}
\end{figure}

In figure~\ref{Fig_ANISO_Z_Fig1} we observe the ellipsoid shape aligned with the Z axis as expected.

\begin{figure}[H]
    \centering
    \includegraphics[scale=0.3, trim = 0cm 0cm 0cm 0cm, clip]{\IMAGES/ANISO_Z-Fig2.png}
    \caption{The distribution of particles for the anisotropic Case III (X-Y plane).}
    \label{Fig_ANISO_Z_Fig2}
\end{figure}

In figure~\ref{Fig_ANISO_Z_Fig2} we observe the circular shape of the ellipsoid on the X-Z plane as expected.

We can therefore conclude that the distribution of the particles in all cases is as expected.

\subsection{Analytical solution}
%........................
In this section we will compare the analytical solution for the anisotropic test cases with the solution obtained by the solver in the same manner as for the isotropic cases. It can be seen that all numerical results are close to the analytical solution.

\begin{figure}[H]
    \centering
    \subfigure[{Case I - X Components}]{
    \includegraphics[width=1.0\textwidth]{\IMAGES/VNV/Fig_Num_Scheme_ANISO_X_X_dir.pdf}
    }
    \caption{Comparison of numerical results and analytical solutions (lines) for the X spatial components of Case I.}
    \label{Fig_NUM_SCHEME_CASE_1X}
\end{figure}

\begin{figure}[H]
    \centering
    \subfigure[{Case I - Y Components}]{
    \includegraphics[width=1.0\textwidth]{\IMAGES/VNV/Fig_Num_Scheme_ANISO_X_Y_dir.pdf}
    }
    \caption{Comparison of numerical results and analytical solutions (lines) for the Y spatial components of Case I.}
    \label{Fig_NUM_SCHEME_CASE_1Y}
\end{figure}

\begin{figure}[H]
    \centering
    \subfigure[{Case I - Z Components}]{
    \includegraphics[width=1.0\textwidth]{\IMAGES/VNV/Fig_Num_Scheme_ANISO_X_Z_dir.pdf}
    }
    \caption{Comparison of numerical results and analytical solutions (lines) for the Z spatial components of Case I.}
    \label{Fig_NUM_SCHEME_CASE_1Z}
\end{figure}

\begin{figure}[H]
    \centering
    \subfigure[{Case II - X Components}]{
    \includegraphics[width=1.0\textwidth]{\IMAGES/VNV/Fig_Num_Scheme_ANISO_Y_X_dir.pdf}
    }
    \caption{Comparison of numerical results and analytical solutions (lines) for the X spatial components of Case II.}
    \label{Fig_NUM_SCHEME_CASE_2X}
\end{figure}

\begin{figure}[H]
    \centering
    \subfigure[{Case II - Y Components}]{
    \includegraphics[width=1.0\textwidth]{\IMAGES/VNV/Fig_Num_Scheme_ANISO_Y_Y_dir.pdf}
    }
    \caption{Comparison of numerical results and analytical solutions (lines) for the Y spatial components of Case II.}
    \label{Fig_NUM_SCHEME_CASE_2Y}
\end{figure}

\begin{figure}[H]
    \centering
    \subfigure[{Case II - Z Components}]{
    \includegraphics[width=1.0\textwidth]{\IMAGES/VNV/Fig_Num_Scheme_ANISO_Y_Z_dir.pdf}
    }
    \caption{Comparison of numerical results and analytical solutions (lines) for the Z spatial components of Case II.}
    \label{Fig_NUM_SCHEME_CASE_2Z}
\end{figure}

\begin{figure}[H]
    \centering
    \subfigure[{Case III - X Components}]{
    \includegraphics[width=1.0\textwidth]{\IMAGES/VNV/Fig_Num_Scheme_ANISO_Z_X_dir.pdf}
    }
    \caption{Comparison of numerical results and analytical solutions (lines) for the X spatial components of Case III.}
    \label{Fig_NUM_SCHEME_CASE_3X}
\end{figure}

\begin{figure}[H]
    \centering
    \subfigure[{Case III - Y Components}]{
    \includegraphics[width=1.0\textwidth]{\IMAGES/VNV/Fig_Num_Scheme_ANISO_Z_Y_dir.pdf}
    }
    \caption{Comparison of numerical results and analytical solutions (lines) for the Y spatial components of Case III.}
    \label{Fig_NUM_SCHEME_CASE_3Y}
\end{figure}

\begin{figure}[H]
    \centering
    \subfigure[{Case III - Z Components}]{
    \includegraphics[width=1.0\textwidth]{\IMAGES/VNV/Fig_Num_Scheme_ANISO_Z_Z_dir.pdf}
    }
    \caption{Comparison of numerical results and analytical solutions (lines) for the Z spatial components of Case III.}
    \label{Fig_NUM_SCHEME_CASE_3Z}
\end{figure}

\section{Conclusions}
Overall, this algorithm for the numerical scheme is precise and robust in both the isotropic and anisotropic cases considered.

\clearpage

%%%%%%%%%%%%%%%%%%%%%%%%%%%%%%%%%%%%%%%%%%%%%%%%%%%%%%%%%
%                                                       %
%            End of test case desciption                %
%                                                       %
%%%%%%%%%%%%%%%%%%%%%%%%%%%%%%%%%%%%%%%%%%%%%%%%%%%%%%%%%

