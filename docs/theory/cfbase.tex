%-------------------------------------------------------------------------------

% This file is part of code_saturne, a general-purpose CFD tool.
%
% Copyright (C) 1998-2024 EDF S.A.
%
% This program is free software; you can redistribute it and/or modify it under
% the terms of the GNU General Public License as published by the Free Software
% Foundation; either version 2 of the License, or (at your option) any later
% version.
%
% This program is distributed in the hope that it will be useful, but WITHOUT
% ANY WARRANTY; without even the implied warranty of MERCHANTABILITY or FITNESS
% FOR A PARTICULAR PURPOSE.  See the GNU General Public License for more
% details.
%
% You should have received a copy of the GNU General Public License along with
% this program; if not, write to the Free Software Foundation, Inc., 51 Franklin
% Street, Fifth Floor, Boston, MA 02110-1301, USA.

%-------------------------------------------------------------------------------

\programme{cfbl**}\label{ap:cfbase}

%%%%%%%%%%%%%%%%%%%%%%%%%%%%%%%%%%
%%%%%%%%%%%%%%%%%%%%%%%%%%%%%%%%%%
\section*{Fonction}
%%%%%%%%%%%%%%%%%%%%%%%%%%%%%%%%%%
%%%%%%%%%%%%%%%%%%%%%%%%%%%%%%%%%%


On s'intéresse à la résolution des équations de Navier-Stokes en compressible,
en particulier pour des configurations sans choc. Le schéma global correspond à une
extension des algorithmes volumes finis mis en \oe uvre pour simuler les
équations de Navier-Stokes en incompressible.

Dans les grandes lignes, le sch\'ema est constitu\'e d'une \'etape
``acoustique'' fournissant la masse volumique (ainsi qu'une pr\'ediction de
pression et un d\'ebit acoustique), suivie de la r\'esolution de l'\'equation de
la quantit\'e de mouvement~; on r\'esout ensuite l'\'equation de l'\'energie
et, pour terminer, la pression est mise \`a jour.
Moyennant une contrainte sur la valeur du pas de temps, le sch\'ema permet
d'assurer la positivit\'e de la masse volumique.

La thermodynamique prise en compte \`a ce jour est celle des gaz parfaits, mais
l'organisation du code \`a \'et\'e  pr\'evue pour permettre \`a l'utilisateur de
fournir ses propres lois.

Pour compléter la présentation, on pourra se reporter à la référence
suivante~: \\
\textbf{[Mathon]} P. Mathon, F. Archambeau, J.-M. Hérard : "Implantation d'un
algorithme compressible dans \CS", HI-83/03/016/A

Le cas de validation "tube \`a choc" de la version 1.2 de \CS permettra
\'egalement d'apporter quelques compl\'ements (tube \`a choc de Sod,
discontinuit\'e de contact instationnaire, double d\'etente sym\'etrique,
double choc sym\'etrique).

\newpage
%=================================
\subsection*{Notations}
%=================================

\begin{table}[h!]
\begin{tabular}{ccp{10,5cm}}

{\bf Symbole} & {\bf Unit\'e} & {\bf Signification}\\

\phantom{$C_v$, ${C_v}_i$} & \phantom{$\lbrack f\rbrack.\,kg/(m^3.\,s)$} & \\

$C_p$, ${C_p}_i$ & $J/(kg.\,K)$        & capacité calorifique \`a pression constante
        $C_p = \left.\frac{\partial h}{\partial T}\right)_P$\\
$C_v$, ${C_v}_i$ & $J/(kg.\,K)$        & capacité calorifique \`a volume constant
        $C_v = \left.\frac{\partial \varepsilon}{\partial T}\right)_\rho$\\
$\mathcal{D}_{f/b}$ & $m^2/s$         & diffusivit\'e mol\'eculaire du composant $f$
                                        dans le bain\\
$E$                 & $J/m^3$        & \'energie totale volumique $E = \rho e$\\
$F$                 &                  & centre de gravit\'e d'une face\\
$H$                 & $J/kg$         & enthalpie totale massique
                                        $H = \frac{E+P}{\rho}$\\
$I$                 &                  & point de co-location de la cellule $i$\\
$I'$                 &                  & pour une face $ij$ partag\'ee entre les
                                        cellules $i$ et $j$, $I'$
                                        est le projet\'e de $I$ sur la
                                        normale \`a la $ij$ passant
                                        par $F$, centre de $ij$\\
$K$                 & $kg/(m.\,s)$         & diffusivit\'e thermique\\
$M$, $M_i$         & $kg/mol$         & masse molaire ($M_i$ pour le constituant $i$)\\
$P$                 & $Pa$                 & pression\\
$\vect{Q}$         & $kg/(m^2.\,s)$& vecteur quantit\'e de mouvement
                                        $\vect{Q} = \rho\vect{u}$\\
$\vect{Q}_{ac}$ & $kg/(m^2.\,s)$& vecteur quantit\'e de mouvement issu
                                        de l'\'etape acoustique\\
$Q$                 & $kg/(m^2.\,s)$& norme de $\vect{Q}$\\
$R$                 & $J/(mol.\,K)$ & constante universelle des gaz parfaits\\
$S$                 & $J/(K.\,m^3)$        & entropie volumique\\
$\mathcal{S}$         & $\lbrack f\rbrack.\,kg/(m^3.\,s)$
                                & Terme de production/dissipation volumique
                                        pour le scalaire $f$\\
$T$                 & $K$                 & temp\'erature ($>0$)\\
$Y_i$                 &                 & fraction massique du compos\'e $i$
                                        ($0 \leqslant Y_i \leqslant 1$)\\
\end{tabular}
\end{table}

\clearpage

\begin{table}[htp]
\begin{tabular}{ccp{10,5cm}}

{\bf Symbole} & {\bf Unit\'e} & {\bf Signification}\\

\phantom{$C_v$, ${C_v}_i$} & \phantom{$\lbrack f\rbrack.\,kg/(m^3.\,s)$} & \\

$c^2$                 & $(m/s)^2$         & carr\'e de la vitesse du son
                $c^2 = \left.\frac{\partial P}{\partial \rho}\right)_s$\\
$e$                 & $J/kg$         & \'energie totale massique
                                        $e = \varepsilon + \frac{1}{2}u^2$\\
$\vect{f}_v$         & $N/kg$         & $\rho\vect{f}_v$ repr\'esente le terme
                                        source volumique pour la quantit\'e
                                        de mouvement~: gravit\'e, pertes
                                        de charges, tenseurs des contraintes
                                        turbulentes, forces de Laplace...\\
$\vect{g}$         & $m/s^2$         & acc\'el\'eration de la pesanteur\\
$h$                 & $J/kg$         & enthalpie massique
                                        $h=\varepsilon + \frac{P}{\rho}$\\
$i$                 &                  & indice faisant r\'ef\'erence \`a la
                                        cellule $i$~; $f_i$ est la valeur
                                        de la variable $f$ associ\'ee
                                        au point de co-location $I$\\
$I'$                 &                  & indice faisant r\'ef\'erence \`a la
                                        cellule $i$~; $f_I'$ est la valeur
                                        de la variable $f$ associ\'ee
                                        au point $I'$\\
$\vect{j}\wedge\vect{B}$
                & $N/m^3$         & forces de Laplace\\
$r$, $r_i$         & $J/(kg.\,K)$         & constante massique des gaz parfaits
                                        $r = \frac{R}{M}$
                                        (pour le constituant $i$, on a $r_i=\frac{R}{M_i}$)\\
$s$                 & $J/(K.\,kg)$         & entropie massique\\
$t$                 & $s$                 & temps\\
$\vect{u}$         & $m/s$         & vecteur vitesse\\
$u$                 & $m/s$         & norme de $\vect{u}$\\

\end{tabular}
\end{table}


\newpage

\begin{table}[htp]
\begin{tabular}{ccp{10,5cm}}

{\bf Symbole} & {\bf Unit\'e} & {\bf Signification}\\

\phantom{$C_v$, ${C_v}_i$} & \phantom{$\lbrack f\rbrack.\,kg/(m^3.\,s)$} & \\

$\beta$         & $kg/(m^3.\,K)$ &
        $\beta = \left.\frac{\partial P}{\partial s}\right)_\rho$\\
$\gamma$         & $kg/(m^3.\,K)$ & constante caract\'eristique
                                        d'un gaz parfait
                                        $\gamma = \frac{C_p}{C_v}$\\
$\varepsilon$         & $J/kg$         & \'energie interne massique\\
$\kappa$         & $kg/(m.\,s)$         & viscosit\'e dynamique en volume\\
$\lambda$         & $W/(m.\,K)$         & conductivit\'e thermique\\
$\mu$                 & $kg/(m.\,s)$         & viscosit\'e dynamique ordinaire\\
$\rho$                 & $kg/m^3$         & densit\'e\\
$\vect{\varphi}_f$
                & $\lbrack f\rbrack.\,kg/(m^2.\,s)$
                                & vecteur flux diffusif du compos\'e $f$\\
$\varphi_f$         & $\lbrack f\rbrack.\,kg/(m^2.\,s)$
                                & norme de $\vect{\varphi}_f$\\

\phantom{ouden}        &                 & \\

$\tens{\Sigma}^v$ &$kg/(m^2.\,s^2)$& tenseur des contraintes visqueuses\\
$\vect{\Phi}_s$ & $W/m^2$        & vecteur flux conductif de chaleur\\
$\Phi_s$         & $W/m^2$         & norme de $\vect{\Phi}_s$\\
$\Phi_v$         & $W/kg$         & $\rho\Phi_v$ repr\'esente le terme
                                        source volumique d'\'energie,
                                        comprenant par exemple
                                        l'effet Joule $\vect{j}\cdot\vect{E}$,
                                        le rayonnement...\\
\end{tabular}
\end{table}
\clearpage

%=================================
\subsection*{Syst\`eme d'\'equations laminaires de r\'ef\'erence}
%=================================

L'algorithme d\'evelopp\'e propose de r\'esoudre
l'\'equation de continuit\'e, les \'equations de Navier-Stokes
ainsi que l'\'equation d'\'energie totale de mani\`ere conservative,
pour des \'ecoulements compressibles.

\begin{equation}\label{Cfbl_Cfbase_eq_ref_laminaire_cfbase}
\left\{\begin{array}{l}

\displaystyle\frac{\partial\rho}{\partial t} + \divs(\vect{Q}) = 0 \\
\\
\displaystyle\frac{\partial\vect{Q}}{\partial t}
+ \divv(\vect{u} \otimes \vect{Q}) + \gradv{P}
= \rho \vect{f}_v + \divv(\tens{\Sigma}^v) \\
\\
\displaystyle\frac{\partial E}{\partial t} + \divs( \vect{u} (E+P) )
= \rho\vect{f}_v\cdot\vect{u} + \divs(\tens{\Sigma}^v \vect{u})
- \divs{\,\vect{\Phi}_s} + \rho\Phi_v

\end{array}\right.
\end{equation}

Nous avons présenté ici le système d'équations laminaires, mais il faut préciser
que la turbulence ne pose pas de problème particulier dans la mesure où les
équations suppl\'ementaires sont découplées du syst\`eme~(\ref{Cfbl_Cfbase_eq_ref_laminaire_cfbase}).

%=================================
\subsection*{Expression des termes intervenant dans les \'equations}
%=================================

\begin{itemize}

\item{\'Energie totale volumique :
        \begin{equation}
        E = \rho e = \rho\varepsilon + \frac{1}{2} \rho u^2
        \end{equation}
        avec l'\'energie interne $\varepsilon(P,\rho)$ donn\'ee par l'\'equation d'\'etat}
\\
\item{Forces volumiques : $\rho\vect{f}_v$ (dans la plupart des cas
                                            $\rho\vect{f}_v= \rho\vect{g}$)}
\\
\item{Tenseur des contraintes visqueuses pour un fluide Newtonien :
        \begin{equation}
        \tens{\Sigma}^v = \mu (\gradt{\vect{u}} +\ ^t\gradt{\vect{u}})
        + (\kappa - \frac{2}{3}\mu) \divs\,{\vect{u}}\ \tens{Id}
        \end{equation}
        avec $\mu(T,\ldots)$ et $\kappa(T,\ldots)$ mais souvent $\kappa =0$}
\\
\item{Flux de conduction de la chaleur : loi de Fourier
        \begin{equation}
        \vect{\Phi}_s = -\lambda \gradv{T}
        \end{equation}
        avec $\lambda(T,\ldots)$}
\\
\item{Source de chaleur volumique : $\rho\Phi_v$}

\end{itemize}


%=================================
\subsection*{\'Equations d'\'etat et expressions de l'\'energie interne}
\label{Cfbl_Cfbase_equations_etat_cfbase}
%=================================

%---------------------------------
\subsubsection*{Gaz parfait}
%---------------------------------

\'Equation d'\'etat : $P = \rho r T$\\


\'Energie interne massique :
$\varepsilon = \displaystyle\frac{P}{(\gamma -1) \rho}$

Soit~:
\begin{equation}\label{Cfbl_Cfbase_eq_pression_gp_cfbase}
P = (\gamma -1) \rho (e - \frac{1}{2} u^2)
\end{equation}


%---------------------------------
\subsubsection*{M\'elange de gaz parfaits}
%---------------------------------

On consid\`ere un m\'elange de $N$ constituants de fractions massiques
$(Y_i)_{i=1 \ldots N}$\\

\'Equation d'\'etat : $P = \rho\ r_{m\acute elange}\ T$\\

\'Energie interne massique :
$\varepsilon = \displaystyle\frac{P}{(\gamma_{m\acute elange} -1)\rho}$

Soit~:
\begin{equation}\label{Cfbl_Cfbase_eq_pression_melange_gp_cfbase}
P = (\gamma_{m\acute elange} -1) \rho (e - \frac{1}{2} u^2)
\end{equation}

avec $\gamma_{m\acute elange}
= \displaystyle\frac{\sum\limits_{i=1}^{N} {Y_i C_{pi}}}
{\sum\limits_{i=1}^{N} {Y_i C_{vi}}}$\ \
et\ \ $r_{m\acute elange} = \displaystyle\sum\limits_{i=1}^{N} {Y_i r_i}$


%---------------------------------
\subsubsection*{Equation d'\'etat de Van der Waals}
%---------------------------------

Cette \'equation est une correction de l'\'equation d'\'etat
des gaz parfaits pour tenir compte des forces intermol\'eculaires
et du volume des mol\'ecules constitutives du gaz.
On introduit deux coefficients correctifs~:
$a$ [$Pa.\,m^6 / kg^2$] est li\'e aux forces intermol\'eculaires
et $b$ [$m^3/kg$] est le covolume (volume occup\'e par les mol\'ecules).\\

\'Equation d'\'etat : $(P+a\rho^2)(1-b\rho) = \rho r T$\\

\'Energie interne massique :
$\varepsilon = \displaystyle\frac{(P+a\rho^2)(1-b\rho)}
{(\hat{\gamma} -1)\rho} - a \rho$

Soit~:
\begin{equation}\label{Cfbl_Cfbase_eq_pression_vdw_cfbase}
P = (\hat{\gamma} -1) \displaystyle\frac{\rho}{(1-b\rho)}
(e - \frac{1}{2} u^2 + a\rho) - a \rho^2
\end{equation}

avec $\hat{\gamma} = 1 + \displaystyle\frac{r}{C_v}
= \displaystyle\frac{C_p}{C_v}
\displaystyle\left(\frac{P-a\rho^2 (1-2b\rho)}{P+a\rho^2}\right)
+ \displaystyle\frac{2a\rho^2 (1-b\rho)}{P+a\rho^2}$



%=================================
\subsection*{Calcul des grandeurs thermodymamiques}
%=================================

%---------------------------------
\subsubsection*{Pour un gaz parfait \`a $\gamma$ constant}
%---------------------------------

%`````````````````````````````````
\paragraph{Equation d'\'etat~:}
%,,,,,,,,,,,,,,,,,,,,,,,,,,,,,,,,,

$P = \rho r T$

On suppose connues la chaleur massique \`a pression constante $C_p$
et la masse molaire $M$ du gaz, ainsi que les variables d'\'etat.

%`````````````````````````````````
\paragraph{Chaleur massique \`a volume constant~:}
%,,,,,,,,,,,,,,,,,,,,,,,,,,,,,,,,,

$C_v = C_p - \displaystyle\frac{R}{M} = C_p - r$


%`````````````````````````````````
\paragraph{Constante caract\'eristique du gaz~:}
%,,,,,,,,,,,,,,,,,,,,,,,,,,,,,,,,,

$\gamma = \displaystyle\frac{C_p}{C_v} = \displaystyle\frac{C_p}{C_p - r}$


%`````````````````````````````````
\paragraph{Vitesse du son~:}
%,,,,,,,,,,,,,,,,,,,,,,,,,,,,,,,,,

$c^2 = \gamma \displaystyle\frac{P}{\rho}$


%`````````````````````````````````
\paragraph{Entropie~:}
%,,,,,,,,,,,,,,,,,,,,,,,,,,,,,,,,,

$s = \displaystyle\frac{P}{\rho^{\gamma}}$
\quad et
$\beta = \left.\displaystyle\frac{\partial P}{\partial s}\right)_{\rho}
= \rho^{\gamma}$

\noindent\textit{Remarque~:} L'entropie choisie ici n'est pas l'entropie
physique, mais une entropie math\'ematique qui v\'erifie \quad
$c^2 \left.\displaystyle\frac{\partial s}{\partial P}\right)_{\rho}
+ \left.\displaystyle\frac{\partial s}{\partial \rho}\right)_{P} = 0$


%`````````````````````````````````
\paragraph{Pression~:}
%,,,,,,,,,,,,,,,,,,,,,,,,,,,,,,,,,

$P = (\gamma-1) \rho \varepsilon$


%`````````````````````````````````
\paragraph{Energie interne~:}
%,,,,,,,,,,,,,,,,,,,,,,,,,,,,,,,,,

$\varepsilon = C_v T
= \displaystyle\frac{1}{\gamma-1} \displaystyle\frac{P}{\rho}$\qquad\text{\ \ avec\ \ }
$\varepsilon_{sup} = 0$

%`````````````````````````````````
\paragraph{Enthalpie~:}
%,,,,,,,,,,,,,,,,,,,,,,,,,,,,,,,,,

$h = C_p T
= \displaystyle\frac{\gamma}{\gamma-1} \displaystyle\frac{P}{\rho}$


%---------------------------------
\subsubsection*{Pour un m\'elange de gaz parfaits}
%---------------------------------

Une intervention de l'utilisateur dans le sous-programme utilisateur
\fort{uscfth} est nécessaire pour pouvoir utiliser ces lois.

%`````````````````````````````````
\paragraph{Equation d'\'etat~:}
%,,,,,,,,,,,,,,,,,,,,,,,,,,,,,,,,,

$P = \rho\ r_{m\acute el}\ T$
\quad avec $r_{m\acute el} = \displaystyle\sum\limits_{i=1}^{N} {Y_i r_i}
= \displaystyle\sum\limits_{i=1}^{N} Y_i \displaystyle\frac{R}{M_i}$


On suppose connues la chaleur massique \`a pression constante
des diff\'erents constituants ${C_p}_i$,
la masse molaire $M_i$ des constituants du gaz,
ainsi que les variables d'\'etat (dont les fractions massiques $Y_i$).

%`````````````````````````````````
\paragraph{Masse molaire du m\'elange~:}
%,,,,,,,,,,,,,,,,,,,,,,,,,,,,,,,,,

$M_{m\acute el} = \left(\displaystyle\sum\limits_{i=1}^{N}
\displaystyle\frac{Y_i}{M_i} \right)^{-1}$

%`````````````````````````````````
\paragraph{Chaleur massique \`a pression constante du m\'elange~:}
%,,,,,,,,,,,,,,,,,,,,,,,,,,,,,,,,,
$\\$
${C_p}_{m\acute el} = \displaystyle\sum\limits_{i=1}^{N} Y_i {C_p}_i$


%`````````````````````````````````
\paragraph{Chaleur massique \`a volume constant du m\'elange~:}
%,,,,,,,,,,,,,,,,,,,,,,,,,,,,,,,,,
$\\$
${C_v}_{m\acute el} = \displaystyle\sum\limits_{i=1}^{N} Y_i {C_v}_i
= {C_p}_{m\acute el} - \displaystyle\frac{R}{M_{m\acute el}}
= {C_p}_{m\acute el} - r_{m\acute el}$


%`````````````````````````````````
\paragraph{Constante caract\'eristique du gaz~:}
%,,,,,,,,,,,,,,,,,,,,,,,,,,,,,,,,,

$\gamma_{m\acute el} = \displaystyle\frac{{C_p}_{m\acute el}}
{{C_v}_{m\acute el}}
= \displaystyle\frac{{C_p}_{m\acute el}}{{C_p}_{m\acute el} - r_{m\acute el}}$


%`````````````````````````````````
\paragraph{Vitesse du son~:}
%,,,,,,,,,,,,,,,,,,,,,,,,,,,,,,,,,

$c^2 = \gamma_{m\acute el} \displaystyle\frac{P}{\rho}$


%`````````````````````````````````
\paragraph{Entropie~:}
%,,,,,,,,,,,,,,,,,,,,,,,,,,,,,,,,,

$s = \displaystyle\frac{P}{\rho^{\gamma_{m\acute el}}}$
\quad et
$\beta = \left.\displaystyle\frac{\partial P}{\partial s}\right)_{\rho}
= \rho^{\gamma_{m\acute el}}$


%`````````````````````````````````
\paragraph{Pression~:}
%,,,,,,,,,,,,,,,,,,,,,,,,,,,,,,,,,

$P = (\gamma_{m\acute el}-1) \rho \varepsilon$


%`````````````````````````````````
\paragraph{Energie interne~:}
%,,,,,,,,,,,,,,,,,,,,,,,,,,,,,,,,,

$\varepsilon = {C_v}_{m\acute el}\ T$\qquad\text{\ \ avec\ \ }
$\varepsilon_{sup} = 0$


%`````````````````````````````````
\paragraph{Enthalpie~:}
%,,,,,,,,,,,,,,,,,,,,,,,,,,,,,,,,,

$h = {C_p}_{m\acute el}\  T
= \displaystyle\frac{\gamma_{m\acute el}}{\gamma_{m\acute el}-1}
\displaystyle\frac{P}{\rho}$

%---------------------------------
\subsubsection*{Pour un gaz de Van der Waals}
%---------------------------------

Ces lois n'ont pas été programmées, mais l'utilisateur peut intervenir
dans le sous-programme utilisateur \fort{uscfth} s'il souhaite le faire.

%`````````````````````````````````
\paragraph{Equation d'\'etat~:}
%,,,,,,,,,,,,,,,,,,,,,,,,,,,,,,,,,

$(P+a\rho^2)(1-b\rho) = \rho r T$

avec $a$ [$Pa.\,m^6 / kg^2$] li\'e aux forces intermol\'eculaires
et $b$ [$m^3/kg$] le covolume (volume occup\'e par les mol\'ecules).

On suppose connus les coefficients $a$ et $b$,
la chaleur massique \`a pression constante $C_p$,
la masse molaire $M$ du gaz et
les variables d'\'etat.


%`````````````````````````````````
\paragraph{Chaleur massique \`a volume constant~:}
%,,,,,,,,,,,,,,,,,,,,,,,,,,,,,,,,,

$C_v = C_p - r
\displaystyle\frac{P+a\rho^2}{P-a\rho^2 (1-2b\rho)}$


%`````````````````````````````````
\paragraph{Constante ``\'equivalente'' du gaz~:}
%,,,,,,,,,,,,,,,,,,,,,,,,,,,,,,,,,

$\hat{\gamma} = 1 + \displaystyle\frac{r}{C_v}
= \displaystyle\frac{C_p}{C_v}
\displaystyle\left(\frac{P-a\rho^2 (1-2b\rho)}{P+a\rho^2}\right)
+ \displaystyle\frac{2a\rho^2 (1-b\rho)}{P+a\rho^2}$

%`````````````````````````````````
\paragraph{Vitesse du son~:}
%,,,,,,,,,,,,,,,,,,,,,,,,,,,,,,,,,

$c^2 = \hat{\gamma} \displaystyle\frac{P+a\rho^2}{\rho(1-b\rho)} - 2a\rho$


%`````````````````````````````````
\paragraph{Entropie~:}
%,,,,,,,,,,,,,,,,,,,,,,,,,,,,,,,,,

$s = (P+a\rho^2)
\left(\displaystyle\frac{1-b\rho}{\rho}\right)^{\hat{\gamma}}$
\quad et
$\beta = \left.\displaystyle\frac{\partial P}{\partial s}\right)_{\rho}
= \left(\displaystyle\frac{\rho}{1-b\rho}\right)^{\hat{\gamma}}$


%`````````````````````````````````
\paragraph{Pression~:}
%,,,,,,,,,,,,,,,,,,,,,,,,,,,,,,,,,

$P = (\hat{\gamma} -1) \displaystyle\frac{\rho}{(1-b\rho)}
(\varepsilon + a\rho) - a \rho^2$

%`````````````````````````````````
\paragraph{Energie interne~:}
%,,,,,,,,,,,,,,,,,,,,,,,,,,,,,,,,,

$\varepsilon = C_v T - a \rho$\qquad\text{\ \ avec\ \ }
$\varepsilon_{sup} = - a \rho$


%`````````````````````````````````
\paragraph{Enthalpie~:}
%,,,,,,,,,,,,,,,,,,,,,,,,,,,,,,,,,

$h = \displaystyle\frac{\hat{\gamma}-b\rho}{\hat{\gamma}-1}
 \displaystyle\frac{P+a\rho^2}{\rho} - 2a\rho$


%=================================
\subsection*{Algorithme de base}
%=================================

On suppose connues toutes les variables au temps $t^n$ et on cherche
\`a les d\'eterminer \`a l'instant $t^{n+1}$.
On r\'esout en deux blocs principaux~: d'une part le syst\`eme masse-quantit\'e
de mouvement, de l'autre l'\'equation portant sur l'\'energie et les scalaires
transport\'es.
Dans le premier bloc, on distingue le traitement du syst\`eme (coupl\'e)
acoustique et le traitement de l'\'equation de la quantit\'e de mouvement
compl\`ete.

Au d\'ebut du pas de temps, on commence par mettre \`a jour
les propri\'et\'es physiques variables (par exemple $\mu(T)$, $\kappa(T)$,
$C_p(Y_1,\ldots ,Y_N)$ ou $\lambda(T)$), puis on
r\'esout les \'etapes suivantes~:

\begin{enumerate}

  \item {\bf Acoustique~: sous-programme \fort{cfmsvl}} \\
     Résolution d'une équation de convection-diffusion portant sur $\rho^{n+1}$.\\
     On obtient à la fin de l'étape $\rho^{n+1}$, $Q^{n+1}_{ac}$ et
éventuellement une
pr\'ediction de la pression $P^{pred}(\rho^{n+1},e^{n})$.\\

  \item {\bf Quantit\'e de mouvement~: sous-programme \fort{cfqdmv}}\\
     Résolution d'une équation de convection-diffusion portant sur $u^{n+1}$ qui
     fait intervenir  $Q^{n+1}_{ac}$ et $P^{pred}$.\\
     On obtient à la fin de l'étape $u^{n+1}$.\\

  \item {\bf \'Energie totale~: sous-programme \fort{cfener}}\\
     Résolution d'une équation de convection-diffusion portant sur $e^{n+1}$ qui
     fait intervenir  $Q^{n+1}_{ac}$, $P^{pred}$ et  $u^{n+1}$.\\
     On obtient \`a la fin de l'étape $e^{n+1}$ et une valeur actualisée de la
pression $P(\rho^{n+1},e^{n+1})$.\\

  \item {\bf Scalaires passifs}\\
     Résolution d'une équation de convection-diffusion standard par
     scalaire, avec  $Q^{n+1}_{ac}$ pour flux convectif.
\end{enumerate}

%%%%%%%%%%%%%%%%%%%%%%%%%%%%%%%%%%
%%%%%%%%%%%%%%%%%%%%%%%%%%%%%%%%%%
\section*{Discr\'etisation}
%%%%%%%%%%%%%%%%%%%%%%%%%%%%%%%%%%
%%%%%%%%%%%%%%%%%%%%%%%%%%%%%%%%%%

On se reportera aux sections relatives aux sous-programmes
\fort{cfmsvl} (masse volumique), \fort{cfqdmv}
(quantit\'e de mouvement) et \fort{cfener} (\'energie).
La documentation du sous-programme
\fort{cfxtcl} fournit des \'el\'ements relatifs aux
conditions
aux limites.

%%%%%%%%%%%%%%%%%%%%%%%%%%%%%%%%%%
%%%%%%%%%%%%%%%%%%%%%%%%%%%%%%%%%%
\section*{Mise en \oe uvre}
%%%%%%%%%%%%%%%%%%%%%%%%%%%%%%%%%%
%%%%%%%%%%%%%%%%%%%%%%%%%%%%%%%%%%

Le module compressible est une ``physique particuli\`ere'' activ\'ee lorsque le
mot-cl\'e \var{IPPMOD(ICOMPF)} est positif ou nul.

Dans ce qui suit, on pr\'ecise les inconnues et les propri\'et\'es
principales utilis\'ees dans le module.
On fournit \'egalement un arbre d'appel simplifi\'e des sous-programmes du
module~: initialisation avec \fort{initi1} puis (\fort{iniva0} et) \fort{inivar} et
enfin, boucle en temps avec \fort{tridim}.


\subsection*{Inconnues et propri\'et\'es}

Les \var{NSCAPP} inconnues scalaires associ\'ees \`a la physique
particuli\`ere sont d\'efinies dans \fort{cfvarp} dans l'ordre
suivant~:
\begin{itemize}
\item la masse volumique \var{RTP(*,ISCA(IRHO))},
\item l'énergie totale   \var{RTP(*,ISCA(IENERG))},
\item la température     \var{RTP(*,ISCA(ITEMPK))}
\end{itemize}

On souligne que la temp\'erature est d\'efinie en tant que variable ``\var{RTP}'' et
non pas en tant que propri\'et\'e physique ``\var{PROPCE}''. Ce choix a \'et\'e
motiv\'e par la volont\'e de simplifier la gestion des conditions aux limites,
au prix cependant d'un encombrement m\'emoire l\'eg\`erement sup\'erieur (une
grandeur \var{RTP} consomme plus qu'une grandeur \var{PROPCE}).

La pression et la vitesse sont classiquement associ\'ees aux tableaux suivants~:
\begin{itemize}
\item pression~: \var{RTP(*,IPR)}
\item vitesse~: \var{RTP(*,IU)}, \var{RTP(*,IV)}, \var{RTP(*,IW)}.
\end{itemize}


\bigskip
Outre les propri\'et\'es associ\'ees en standard aux variables
identifi\'ees ci-dessus, le
tableau \var{PROPCE} contient \'egalement~:
 \begin{itemize}
\item la chaleur massique à volume constant $C_v$, stock\'ee dans
\var{PROPCE(*,IPPROC(ICV))},
      si  l'utilisateur a indiqué dans \fort{uscfth} qu'elle \'etait variable.
\item la viscosité en volume \var{PROPCE(*,IPPROC(IVISCV))}
      si  l'utilisateur a indiqué dans \fort{uscfx2} qu'elle \'etait variable.
\end{itemize}


\bigskip
Pour la gestion des conditions aux limites et en particulier pour le calcul du
flux convectif par le sch\'ema de Rusanov
aux entr\'ees et sorties (hormis en sortie supersonique), on
dispose des tableaux suivants dans  \var{PROPFB}~:
\begin{itemize}
\item flux convectif de quantit\'e de mouvement au bord pour les trois
composantes dans les tableaux
\var{PROPFB(*,IPPROB(IFBRHU))} (composante $x$),
\var{PROPFB(*,IPPROB(IFBRHV))} (composante $y$) et
\var{PROPFB(*,IPPROB(IFBRHW))} (composante $z$)
\item flux convectif d'\'energie au bord
\var{PROPFB(*,IPPROB(IFBENE))}
\end{itemize}
et on dispose \'egalement dans \var{IA}~:
\begin{itemize}
\item d'un tableau d'entiers dont la premi\`ere ``case'' est \var{IA(IIFBRU)}, dimensionn\'e au nombre de faces de bord
et permettant de rep\'erer les faces de bord pour lesquelles on calcule
le flux convectif par le sch\'ema de Rusanov,
\item d'un tableau  d'entiers dont la premi\`ere ``case'' est \var{IA(IIFBET)}, dimensionn\'e au nombre de faces de bord
et permettant de rep\'erer les faces de paroi \`a temp\'erature ou \`a flux
thermique impos\'e.
\end{itemize}



\newpage

\subsection*{Arbre d'appel simplifi\'e}
\nopagebreak
\begin{table}[h!]
\begin{center}
\begin{tabular}{lllllp{8cm}}
\fort{usini1}         &                 &                &                &
        & Initialisation des mots-cl\'es utilisateur g\'en\'eraux et positionnement des variables\\
                &\fort{usppmo}         &                &                &
        & D\'efinition du module ``physique particuli\`ere'' employ\'e\\
                &\fort{varpos}         &                &                &
        & Positionnement des variables \\
                &                 & \fort{pplecd} &                &
        & Branchement des physiques particuli\`eres pour la lecture du fichier de donn\'ees \'eventuel \\
                &                 & \fort{ppvarp} &                &
        & Branchement des physiques particuli\`eres pour le positionnement des inconnues \\
                &                 &                 & \fort{cfvarp} &
        & Positionnement des inconnues sp\'ecifiques au module compressible \\
                &                 &                 &               & \fort{uscfth}
        & Appelé avec ICCFTH=-1, pour indiquer que $C_p$ et $C_v$ sont constants ou variables\\
                &                 &                 &               & \fort{uscfx2}
        & Conductivité thermique mol\'eculaire constante ou variable et viscosité en volume
           constante ou variable (ainsi que leur valeur, si elles sont constantes)\\
                &                 & \fort{ppprop} &                &
        & Branchement des physiques particuli\`eres pour le positionnement des propri\'et\'es\\
                &                 &                 & \fort{cfprop} &
        & Positionnement des propri\'et\'es sp\'ecifiques au module compressible \\
%
\fort{ppini1}         &                &                &                &
        & Branchement des physiques particuli\`eres pour l'initialisation des
mots-cl\'es sp\'ecifiques \\
                &\fort{cfini1}         &                &                &
        & Initialisation des mots-cl\'es sp\'ecifiques au module compressible\\
                &\fort{uscfi1}         &                &                &
        & Initialisation des mots-cl\'es utilisateur sp\'ecifiques au module compressible\\
\end{tabular}
\caption{Sous-programme \fort{initi1}~: initialisation des mots-cl\'es et
positionnement des variables}
\end{center}
\end{table}

\newpage

\begin{table}[h!]
\begin{center}
\begin{tabular}{llllp{10cm}}
\fort{ppiniv}         &                &                &
        & Branchement des physiques particuli\`eres pour l'initialisation des variables \\
                & \fort{cfiniv} &                &
        & Initialisation des variables sp\'ecifiques au module compressible \\
                 &                 & \fort{memcfv} &
        & R\'eservation de tableaux de travail locaux   \\
                 &                 & \fort{uscfth} &
        & Initialisation des variables par défaut (en calcul suite~: seulement
$C_v$~; si le calcul n'est pas une suite~: $C_v$, la masse volumique et l'\'energie) \\
                 &                 & \fort{uscfxi} &
        & Initialisation des variables par l'utilisateur (seulement si le calcul
n'est pas une suite)  \\
\end{tabular}
\caption{Sous-programme \fort{inivar}~: initialisation des variables}
\end{center}
\end{table}


\begin{table}[h!]
\begin{center}
\begin{tabular}{llllp{10cm}}
\fort{phyvar}         &                &                &
        & Calcul des propri\'et\'es physiques variables \\
                & \fort{ppphyv} &                &
        & Branchement des physiques particuli\`eres pour le calcul des
                propri\'et\'es physiques variables \\
                &               & \fort{cfphyv} &
        & Calcul des propri\'et\'es physiques variables pour le module
                compressible \\
                 &                 &               & \fort{usphyv}
        & Calcul par l'utilisateur des propri\'et\'es physiques variables pour
                le module
                compressible ($C_v$ est calcul\'e dans \fort{uscfth} qui est
                appel\'e par \fort{usphyv}) \\
\end{tabular}
\caption{Sous-programme \fort{tridim}~: partie 1 (propri\'et\'es physiques)}
\end{center}
\end{table}

\newpage

\begin{table}[h!]
\begin{center}
\begin{tabular}{llllp{10cm}}
\fort{dttvar}         &                &                &
        & Calcul du pas de temps variable  \\
                & \fort{cfdttv} &                &
        & Calcul de la contrainte liée au CFL en compressible \\
                &                    &\fort{memcft}         &
        & Gestion de la m\'emoire pour le calcul de la contrainte en CFL \\
                &                    &\fort{cfmsfl}         &
        & Calcul du flux associ\'e \`a la contrainte en CFL \\

\fort{precli}         &                  &                &
        & Initialisation des tableaux avant calcul des conditions aux
                limites (\var{IITYPF}, \var{ICODCL}, \var{RCODCL})\\
                & \fort{ppprcl} &                &
        & Initialisations sp\'ecifiques aux diff\'erentes physiques
                particuli\`eres avant calcul des conditions aux limites
                (pour le module compressible~: \var{IZFPPP}, \var{IA(IIFBRU)},
                \var{IA(IIFBET)}, \var{RCODCL}, flux convectifs pour la
                quantit\'e de mouvement et l'\'energie)\\

\fort{ppclim}         &                  &                &
        & Branchement des physiques particuli\`eres pour les conditions aux limites (en lieu et place de \fort{usclim})\\
                & \fort{uscfcl} &                &
        & Intervention de l'utilisateur pour les conditions aux limites (en lieu
                et place de \fort{usclim}, m\^eme pour les variables qui ne sont
                pas sp\'ecifiques au module compressible) \\

\fort{condli}         &                  &                &
        & Traitement des conditions aux limites\\
                & \fort{pptycl} &                &
        & Branchement des physiques particuli\`eres pour le traitement des conditions aux limites \\
                &                 &\fort{cfxtcl}         &
        & Traitement des conditions aux limites pour le compressible \\
                &                 &                &\fort{uscfth}
        & Calculs de thermodynamique pour le calcul des conditions aux limites \\
                &                 &                &\fort{cfrusb}
        & Flux de Rusanov (entr\'ees ou sorties sauf sortie supersonique) \\
\end{tabular}
\caption{Sous-programme \fort{tridim}~: partie 2 (pas de temps variable et conditions
                                                  aux limites)}
\end{center}
\end{table}

\newpage

\begin{table}[h!]
\begin{center}
\begin{tabular}{llllp{10cm}}
\fort{memcfm}        &                  &                &
        & Gestion de la m\'emoire pour la r\'esolution de l'\'etape ``acoustique'' \\
\fort{cfmsvl}         &                  &                &
        & R\'esolution de l'étape ``acoustique'' \\
                 & \fort{cfmsfl}  &                &
        & Calcul du "flux de masse" aux faces
                (not\'e $\rho\,\vect{w}\cdot\vect{n}\,S$ dans la documentation
                du sous-programme \fort{cfmsvl}) \\
                 &                  & \fort{cfdivs}&
        & Calcul du terme en divergence du tenseur des contraintes visqueuses
                  (trois appels), éventuellement \\
                 &                  &                &
        & Apr\`es \fort{cfmsfl}, on impose le flux de masse aux faces de bord
                \`a partir des conditions aux limites \\
                 & \fort{cfmsvs}  &                &
        & Calcul de la "viscosité" aux faces
                (not\'ee $\Delta\,t\,c^2\frac{S}{d}$ dans la documentation
                du sous-programme \fort{cfmsvl}) \\
                 &                  &                &
        & Apr\`es \fort{cfmsvs}, on annule la viscosit\'e aux faces de bord
                pour que le flux de masse soit bien celui souhait\'e \\
                 & \fort{codits}  &                &
        & R\'esolution du syst\`eme portant sur la masse volumique \\
                 & \fort{clpsca}  &                &
        & Impression des bornes et clipping \'eventuel (pas de clipping en standard)  \\
                 & \fort{uscfth}  &                &
        & Gestion  \'eventuelle des bornes par l'utilisateur  \\
                 & \fort{cfbsc3}  &                &
        & Calcul du flux de masse acoustique aux faces
                (not\'e $\vect{Q}_{ac}\cdot\vect{n}$ dans la documentation
                du sous-programme \fort{cfmsvl}) \\
                 & \fort{uscfth}  &                &
        & Actualisation de la pression, éventuellement  \\
\fort{cfqdmv}         &                  &                &
        & R\'esolution de la quantité de mouvement\\
                & \fort{cfcdts}         &                &
        & Résolution du syst\`eme\\
                &                  & \fort{cfbsc2}&
        & Calcul des termes de convection et de diffusion au second membre\\
\end{tabular}
\caption{Sous-programme \fort{tridim}~: partie 3 (Navier-Stokes)}
\end{center}
\end{table}

\newpage

\begin{table}[h!]
\begin{center}
\begin{tabular}{llllp{10cm}}
\fort{scalai}          &                  &                &
        & R\'esolution des \'equations sur les scalaires  \\
                & \fort{cfener}         &                &
        & R\'esolution de l'équation sur l'énergie totale\\
                &                  & \fort{memcfe}&
        & Gestion de la m\'emoire locale\\
                &                  & \fort{cfdivs}&
        & Calcul du terme en divergence du produit
           ``tenseur des contraintes par vitesse''\\
                &                  & \fort{uscfth}&
        & Calcul de l'\'ecart  ``\'energie interne - $C_v\,T$''
                ($\varepsilon_{sup}$)\\
                &                  & \fort{cfcdts}&
        & Résolution du syst\`eme\\
                &                  &                  &\fort{cfbsc2}
        & Calcul des termes de convection et de diffusion au second membre\\
                 &                  & \fort{clpsca}&
        & Impression des bornes et clipping \'eventuel (pas de clipping en standard)  \\
                 &                  & \fort{uscfth}&
        & Gestion \'eventuelle des bornes par l'utilisateur  \\
                 &                  & \fort{uscfth}&
        & Mise \`a jour de la pression  \\
\end{tabular}
\caption{Sous-programme \fort{tridim}~: partie 4 (scalaires)}
\end{center}
\end{table}

%\newpage

Le sous-programme \fort{cfbsc3} est similaire \`a \fort{bilsc2}, mais il produit
des flux aux faces et n'est \'ecrit que pour un sch\'ema upwind, \`a l'ordre 1
en temps (ce qui est coh\'erent avec les choix faits dans l'algorithme compressible).

Le sous-programme \fort{cfbsc2} est similaire \`a \fort{bilsc2}, mais
n'est \'ecrit que pour un sch\'ema d'ordre 1 en
temps.
%et fait encore appara\^itre la variable IITURB au lieu de IITYTU (il
%faudrait corriger ce dernier point).
Le sous-programme \fort{cfbsc2} permet d'effectuer un traitement
sp\'ecifique aux faces de bord pour lesquelles on a appliqu\'e
un sch\'ema de Rusanov pour calculer le flux convectif total.
Ce sous-programme est appel\'e pour la r\'esolution de l'\'equation de
la quantit\'e de mouvement et de l'\'equation de l'\'energie.
On pourra se reporter \`a la documentation du sous-programme \fort{cfxtcl}.

Le sous-programme \fort{cfcdts} est similaire \`a \fort{codits} mais fait appel
\`a \fort{cfbsc2} et non pas \`a \fort{bilsc2}.
Il diff\`ere de \fort{codits} par quelques autres d\'etails qui ne sont pas
g\^enants dans l'imm\'ediat~:
initialisation de PVARA et de SMBINI,
%nombre d'it\'erations pour le second membre (NSWRSM-1 au lieu de NSWRSM),
%mode de d\'etermination du solveur (IRESLP),
%test de convergence sur RNORM (compar\'e \`a 0.D0 au lieu de EPZERO),
ordre en temps (ordre 2 non pris en compte).
%Mis \`a part pour l'ordre en temps (l'algorithme
%compressible est \`a l'ordre 1), il serait bon de modifier \fort{cfcdts} pour
%qu'il soit conforme \`a  \fort{codits}.

\newpage
%%%%%%%%%%%%%%%%%%%%%%%%%%%%%%%%%%
%%%%%%%%%%%%%%%%%%%%%%%%%%%%%%%%%%
\section*{Points \`a traiter}
%%%%%%%%%%%%%%%%%%%%%%%%%%%%%%%%%%
%%%%%%%%%%%%%%%%%%%%%%%%%%%%%%%%%%

Des actions compl\'ementaires sont identifi\'ees ci-apr\`es, dans l'ordre
d'urgence d\'ecroissante (on se reportera
\'egalement \`a la section "Points \`a traiter" de la documentation
des autres sous-programmes du module compressible).

\begin{itemize}
\item Assurer la coh\'erence des sous-programmes suivants (ou, \'eventuellement,
les fusionner pour \'eviter qu'ils ne divergent)~:
        \begin{itemize}
        \item \fort{cfcdts} et \fort{codits},
%(actuellement pour PVARA et
%                SMBINI, mais \`a plus long terme pour \'eviter que les
%                deux sous-programmes ne divergent),
% propose en patch 1.2.1
%        \item \fort{cfcdts} et \fort{codits}
%                (au moins pour PVARA, SMBINI, NSWRSM,
%                IRESLP, RNORM),
        \item \fort{cfbsc2} et \fort{bilsc2},
        \item \fort{cfbsc3} et \fort{bilsc2}.
        \end{itemize}
% propose en patch 1.2.1
%        \item Remplacer la valeur 100 par 90 pour ICCFTH dans \fort{uscfth}
%                (plus grande coh\'erence avec les choix faits dans le reste de ce
%                sous-programme).
% propose en patch 1.2.1
%        \item \'Eliminer \fort{memcff} qui ne sert plus.
\item Permettre les suites de calcul incompressible/compressible et
        compressible/incompressible.
\item Apporter un compl\'ement de validation (exemple~: IPHYDR).
\item Assurer la compatibilit\'e avec certaines physiques particuli\`eres, selon
        les besoins. Par exemple~: arc \'electrique, rayonnement, combustion.
\item Identifier les causes des difficult\'es rencontr\'ees sur certains cas
acad\'emiques, en particulier~:
        \begin{itemize}
        \item canal subsonique (comment s'affranchir des effets ind\'esirables
        associ\'es aux conditions d'entr\'ee et de sortie, comment r\'ealiser un
        calcul p\'eriodique, en particulier pour la temp\'erature dont le
        gradient dans la direction de l'\'ecoulement n'est pas nul, si
        les parois sont adiabatiques),
        \item cavit\'e ferm\'ee sans vitesse ni effets de gravit\'e,
        avec temp\'erature ou flux thermique impos\'e en paroi (il pourrait
        \^etre utile d'extrapoler le gradient de pression au bord~:
        la pression d\'epend de la temp\'erature et une simple condition de
        Neumann homog\`ene est susceptible de cr\'eer un terme source de
        quantit\'e de mouvement parasite),
        \item maillage non conforme (non conformit\'e dans la direction
        transverse d'un canal),
      \item ``tube à choc'' avec terme source d'énergie.
        \end{itemize}
\item Compl\'eter certains points de documentation, en particulier les
        conditions aux limites thermiques pour le couplage avec \syrthes.
\item Am\'eliorer la rapidit\'e \`a faible nombre de Mach (est-il
possible de lever la limite
actuelle sur la valeur du pas de temps~?).
\item Enrichir, au besoin~:
        \begin{itemize}
        \item les thermodynamiques prises en compte (multiconstituant,
        gamma variable, Van der Waals...),
        \item la gamme des conditions aux limites d'entr\'ee
        disponibles (condition \`a d\'ebit massique et d\'ebit enthalpique
        impos\'es par exemple).
        \end{itemize}
\item Tester des variantes de l'algorithme~:
        \begin{itemize}
        \item prise en compte des termes sources de l'\'equation de la
        quantit\'e de mouvement autres que la gravit\'e dans l'\'equation de la
        masse r\'esolue lors de l'\'etape ``acoustique'' (les tests r\'ealis\'es
        avec cette variante de l'algorithme devront \^etre repris dans la
        mesure o\`u, dans \fort{cfmsfl}, IIROM et IIROMB n'\'etaient pas
        initialis\'es),
        \item implicitation du terme de convection dans
        l'\'equation de la masse (\'eliminer cette possibilit\'e si
        elle n'apporte rien),
        \item \'etape de pr\'ediction de la pression,
        \item non reconstruction de la masse volumique pour le terme convectif
        (actuellement, les termes convectifs sont trait\'es avec
        d\'ecentrement amont, d'ordre 1 en espace~;
        pour l'\'equation de la quantit\'e de mouvement et l'\'equation de
        l'\'energie, on utilise les valeurs prises au centre des cellules
        sans reconstruction~: c'est l'approche standard de \CS, traduite
        dans \fort{cfbsc2}~; par contre, dans \fort{cfmsvl}, on reconstruit
        les valeurs de la masse volumique utilis\'ees pour le terme
        convectif~; il n'y a pas de raison d'adopter des strat\'egies
        diff\'erentes, d'autant plus que la reconstruction de la masse
        volumique ne permet pas de monter en ordre et augmente le risque
        de d\'epassement des bornes physiques),
        \item mont\'ee en ordre en espace (en v\'erifier l'utilit\'e et
        la robustesse, en particulier relativement au principe du
        maximum pour la masse volumique),
        \item mont\'ee en ordre en temps (en v\'erifier l'utilit\'e et
        la robustesse).
        \end{itemize}
\item Optimiser l'encombrement m\'emoire.
\end{itemize}

