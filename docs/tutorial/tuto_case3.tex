%-------------------------------------------------------------------------------

% This file is part of Code_Saturne, a general-purpose CFD tool.
%
% Copyright (C) 1998-2011 EDF S.A.
%
% This program is free software; you can redistribute it and/or modify it under
% the terms of the GNU General Public License as published by the Free Software
% Foundation; either version 2 of the License, or (at your option) any later
% version.
%
% This program is distributed in the hope that it will be useful, but WITHOUT
% ANY WARRANTY; without even the implied warranty of MERCHANTABILITY or FITNESS
% FOR A PARTICULAR PURPOSE.  See the GNU General Public License for more
% details.
%
% You should have received a copy of the GNU General Public License along with
% this program; if not, write to the Free Software Foundation, Inc., 51 Franklin
% Street, Fifth Floor, Boston, MA 02110-1301, USA.

%-------------------------------------------------------------------------------

\section{Solution for case3}

Only a few elements are different from \texttt{case2}.

In this case the density becomes variable.
Go to the {\itshape Fluid properties} item under the heading, {\itshape Physical properties}
and change the nature of the density from \texttt{constant} to \texttt{ user law}.

The user law of the density is defined as following in the \CS (GUI):\\
\fbox{\begin{minipage}{\textwidth}\texttt{    \\
rho = TempC * ( -4.668E-03*TempC - 5.0754E-02 ) + 1000.9 ;
}\end{minipage} }

Click on the highlighted icon and define the user law in the window that pops up.
Follow the format used in the {\itshape Examples} tab.

\begin{figure}[h!]
\begin{center}
\begin{tabular}{c}
\includegraphics[width=9cm]{case3-V1}
\end{tabular}
\caption{Fluid properties - Variable density}
\end{center}
\end{figure}

\newpage
\begin{figure}[h!]
\begin{center}
\begin{tabular}{c}
\includegraphics[width=9cm]{case3-V2}
\end{tabular}
\caption{Fluid properties - Variable density - User expression}
\label{fig1_e3}
\end{center}
\end{figure}

\newpage
As the density is variable, the influence of gravity has to be considered. In the
heading {\itshape Physical properties} go to
{\itshape Gravity} and set the value of each component of the gravity vector.

\fbox{\begin{minipage}{\textwidth}\texttt{    \\
g\_x = 0.0 ; g\_y = -9.81 ; g\_z = 0.0
}\end{minipage} }

\begin{figure}[h!]
\begin{center}
\includegraphics[width=12cm]{case3-V3}
\caption{Fluid properties - Gravity}
\label{fig2_e3}
\end{center}
\end{figure}


\newpage
Add a monitoring point close to the entry boundary condition in the
{\itshape Output control} item.

\begin{center}
\begin{tabular}{|c|c|c|c|}
\hline
Probe & x (m) & y (m) & z (m)\\
\hline
\hline
9 & -0.5 & 2.25 & 0.0 \\
\hline
\end{tabular}
\end{center}

\begin{figure}[h!]
\begin{center}
\includegraphics[width=12cm]{case3-V4}
\caption{New monitoring probe}
\label{fig3_e3}
\end{center}
\end{figure}


\newpage
After completing the interface, before running the calculation,
some Fortran user routines need to be modified.

Go to the folder SRC/REFERENCE/base and copy {\texttt cs\_user\_boundary\_conditions.f90} in the SRC directory.

$\bullet$ \textbf{\texttt{cs\_user\_boundary\_conditions.f90}}:\\
In this case, {\texttt cs\_user\_boundary\_conditions.f90} is used to specify the time dependent boundary
condition for the temperature. Refer to the comments in the routine or to the \CS user manual
for more information on this routine.\\

In our case, you need to identify the boundary faces of color \textbf{'1'}.\\
The command \texttt{call getfbr('1',nlelt,lstelt)} will return an integer \texttt{nlelt}, corresponding
to the number of boundary faces of color 1, and an integer array \texttt{lstelt} containing the list
of the \texttt{nlelt} boundary faces of color 1.

$\bullet$ {\bf Remark}: Note that the string '1' \textbf{can be more complex and combine
different colors, group references or geometrical criteria}, with the same syntax
as in the Graphical Interface.

For each boundary face \texttt{ifac} in the list, the Dirichlet value is given in the
multi-dimension array \texttt{rcodcl} as follows:

\fbox{\begin{minipage}{\textwidth}\texttt{\\
if (ttcabs.lt.3.8d0) then \\
  do ielt = 1, nlelt      \\
    ifac = lstelt(ielt)   \\
    rcodcl(ifac,isca(1),1) = 20.d0 + 100.d0*ttcabs \\
  enddo\\
else \\
  do ielt = 1, nlelt                \\
    ifac = lstelt(ielt)             \\
    rcodcl(ifac,isca(1),1) = 400.d0 \\
  enddo \\
endif \\
}\end{minipage} }

%\begin{verbatim}
%\end{verbatim}}


\textbf{\texttt{isca(1)}} refers to the first scalar and {\textbf\texttt{ttcabs}}
is the current physical time.

See the example \texttt{cs\_user\_boundary\_conditions-base.f90} file in the subdirectory
\texttt{SRC/EXAMPLES} to complet correctly your boundary conditions for this \texttt{case3}.

$\bullet$ Remark: Note that, although the inlet boundary conditions for temperature are
specified in the \texttt{cs\_user\_boundary\_conditions.f90} file, it is necessary to
specify them also in the Graphical Interface.\\

\textbf{The value given in the Interface can be anything, it will be overwritten by the
Fortran routine}.

After updating the Fortran file, run the calculation as explained in
\texttt{case2}.

\newpage

When a calculation is finished, \CS stores all the necessary elements to
continue the computation in another execution, with total continuity. These
elements are stored in several files, grouped in
a \texttt{yyyymmdd-hhmm/checkpoint} subdirectory, in the \texttt{RESU} directory.

In this case, after the first calculation is finished, a second calculation will
be run, starting from the results of the first one.

Go directly on the {\itshape Start/Restart} item under the heading
{\itshape Calculation management}.  Activate the {\itshape Calculation restart}
by clicking the \textbf{``on''} box.

Then click on the folder icon next to it to specify the restart files to use.

\begin{figure}[h!]
\begin{center}
\includegraphics[width=12cm]{case3-V5}
\caption{Start / Restart}
\label{fig4_e3}
\end{center}
\end{figure}


\newpage
A window opens, with the architecture of the study sub-directories. Open the
\texttt{RESU} folder and click on the folder \texttt{yyyymmdd-hhmm/checkpoint}
(where \texttt{yyyymmdd-hhmm} corresponds to the reference of the first calculation results).
Then click on {\itshape Validate}.

\begin{figure}[h!]
\begin{center}
\includegraphics[width=10cm]{case3-V6}
\caption{Start / Restart - Selection of the restart directory}
\label{fig5_e3}
\end{center}
\end{figure}


\newpage
Go to the {\itshape Time step} item under the heading {\itshape Numerical parameters}
and change the number of iterations. It must be the total number of
iterations, from the beginning of the first calculation.\\

The first calculation was done with 300 iterations and another 400 iterations
are needed for the present case. Therefore the value 700 must be entered.

\begin{figure}[h!]
\begin{center}
\includegraphics[width=12cm]{case3-V7}
\caption{Time step}
\label{fig6_e3}
\end{center}
\end{figure}

Eventually, run the calculation.
