%-----------------------------------------------------------------------
%
%     This file is part of the Code_Saturne Kernel, element of the
%     Code_Saturne CFD tool.
%
%     Copyright (C) 1998-2008 EDF S.A., France
%
%     contact: saturne-support@edf.fr
%
%     The Code_Saturne Kernel is free software; you can redistribute it
%     and/or modify it under the terms of the GNU General Public License
%     as published by the Free Software Foundation; either version 2 of
%     the License, or (at your option) any later version.
%
%     The Code_Saturne Kernel is distributed in the hope that it will be
%     useful, but WITHOUT ANY WARRANTY; without even the implied warranty
%     of MERCHANTABILITY or FITNESS FOR A PARTICULAR PURPOSE.  See the
%     GNU General Public License for more details.
%
%     You should have received a copy of the GNU General Public License
%     along with the Code_Saturne Kernel; if not, write to the
%     Free Software Foundation, Inc.,
%     51 Franklin St, Fifth Floor,
%     Boston, MA  02110-1301  USA
%
%-----------------------------------------------------------------------
\documentclass[a4paper,10pt,twoside]{report}
%
%%%%%%%%%%%%%%%%%%%%%%%%%%%%%%%%%%%%%%%%%%%%%%%%%%%%%%%%%%%%%%%%%%%%%%
% PACKAGES OBLIGATOIRES
\usepackage{csdoc}
% MACROS SUPPLEMENTAIRES
\usepackage{csmacros}
%
%%%%%%%%%%%%%%%%%%%%%%%%%%%%%%%%%%%%%%%%%%%%%%%%%%%%%%%%%%%%%%%%%%%%%%

%
%%%%%%%%%%%%%%%%%%%%%%%%%%%%%%%%%%%%%%%%%%%%%%%%%%%%%%%%%%%%%%%%%%%%%%
% PACKAGES ET COMMANDES POUR LE DOCUMENTS PDF ET LES HYPERLIENS
\hypersetup{%
  pdftitle = {CodeSaturne Theory and Programmer's Guide},
  pdfauthor = {MFEE},
  pdfpagemode = UseOutlines
}
\pdfinfo{/CreationDate (D:20030429000000-01 00 )}
%
% Pour avoir les Thumbnails a l'ouverture du document sous ACROREAD :
% pdfpagemode = UseThumbs
%
%%%%%%%%%%%%%%%%%%%%%%%%%%%%%%%%%%%%%%%%%%%%%%%%%%%%%%%%%%%%%%%%%%%%%%
% MACROS SUPPLEMENTAIRES
% \newcommand{/...}{...}
%
\setcounter{tocdepth}{0}
%Compteur de ``programme'' remis a jour dans les part.
\newcounter{prog}[part]
\renewcommand{\theprog}{\arabic{prog}}
\renewcommand{\thesection}{\theprog.\arabic{section}}
\renewcommand{\theequation}{\thepart.\theprog.\arabic{equation}}
\renewcommand{\thefigure}{\thepart.\theprog.\arabic{figure}}
\newcommand{\programme}[1]{%
\passepage
\refstepcounter{prog}
\stepcounter{chapter}
\setcounter{section}{0}
\setcounter{equation}{0}
\setcounter{figure}{0}
\begin{center}
\Huge \bf \theprog - \underline{Sous-programme \fort{#1}}
\end{center}
\addcontentsline{toc}{chapter}{\theprog- Sous-programme #1}}
%
%%%%%%%%%%%%%%%%%%%%%%%%%%%%%%%%%%%%%%%%%%%%%%%%%%%%%%%%%%%%%%%%%%%%%%
% INFO POUR PAGES DE GARDES
\titreCS{\CS \verscs Theory and Programmer's Guide}
\docassociesCS{}
\resumeCS{Ce document constitue la documentation th\'eorique et
informatique des parties centrales du noyau de \CS~\verscs.
La documentation est attach\'ee \`a la version du code
correspondante pour favoriser les mises \`a jour. En pratique, les
utilisateurs de \CS
peuvent acc\'eder \`a la documentation sous
\texttt{\$CS\_HOME/doc/NOYAU/}, information qui leur est rappel\'ee par la
commande d'information g\'en\'erale \texttt{info\_cs [theory]}.}
%
%%%%%%%%%%%%%%%%%%%%%%%%%%%%%%%%%%%%%%%%%%%%%%%%%%%%%%%%%%%%%%%%%%%%%%

%
%%%%%%%%%%%%%%%%%%%%%%%%%%%%%%%%%%%%%%%%%%%%%%%%%%%%%%%%%%%%%%%%%%%%%%
% DEBUT DU DOCUMENT
\begin{document}

%\def\bibname{R�f�rences}
\def\contentsname{\textbf{\normalsize SOMMAIRE}\pdfbookmark[1]{Sommaire}{contents}}
\def\indexname{Index des variables principales et des mots cl\'es}

\pdfbookmark[1]{Pages de garde}{pdg}
\large
\makepdgCS
\normalsize

\passepage
\input summary

\passepart
\begin{center}\begin{singlespace}
\tableofcontents
\end{singlespace}\end{center}


%\passepage
%\printsymliste\
%
%%%%%%%%%%%%%%%%%%%%%%%%%%%%%%%%%%%%%%%%%%%%%%%%%%%%%%%%%%%%%%%%%%%%%%
% CORPS DU DOCUMENT
%
\passepage
\part{Introduction}
\stepcounter{prog}
%-------------------------------------------------------------------------------

% This file is part of code_saturne, a general-purpose CFD tool.
%
% Copyright (C) 1998-2022 EDF S.A.
%
% This program is free software; you can redistribute it and/or modify it under
% the terms of the GNU General Public License as published by the Free Software
% Foundation; either version 2 of the License, or (at your option) any later
% version.
%
% This program is distributed in the hope that it will be useful, but WITHOUT
% ANY WARRANTY; without even the implied warranty of MERCHANTABILITY or FITNESS
% FOR A PARTICULAR PURPOSE.  See the GNU General Public License for more
% details.
%
% You should have received a copy of the GNU General Public License along with
% this program; if not, write to the Free Software Foundation, Inc., 51 Franklin
% Street, Fifth Floor, Boston, MA 02110-1301, USA.

%-------------------------------------------------------------------------------

%-------------------------------------------------------------------------------
\section*{Disclaimer}
%-------------------------------------------------------------------------------
\CS is free software; you can redistribute it
and/or modify it under the terms of the GNU General Public License
as published by the Free Software Foundation; either version 2 of
the License, or (at your option) any later version.

\CS is distributed in the hope that it will be
useful, but WITHOUT ANY WARRANTY; without even the implied warranty
of MERCHANTABILITY or FITNESS FOR A PARTICULAR PURPOSE.  See the
GNU General Public License for more details.\footnote{You should have
received a copy of the GNU General Public License
along with \CS; if not, write to the
Free Software Foundation, Inc.,
51 Franklin St, Fifth Floor,
Boston, MA  02110-1301  USA}

%-------------------------------------------------------------------------------
\section{Aims of the document}
%-------------------------------------------------------------------------------

This chapter constitutes an introduction to the theory guide
associated with the kernel of \CS.
The system of equations considered consists of the
Navier-Stokes equations, with turbulence and passive scalars. Firstly, the
continuous equations for mass, momentum, turbulence and passive scalars are
presented. Secondly, information related to the time scheme is supplied.
Thirdly, the spatial discretisation is detailed: it is based on a co-located%
\footnote{%
All the variables are located at the centres of the cells.} finite volume
scheme for unstructured meshes. Fourthly, the different source terms are
described. Fifthly, boundary conditions are detailed. And finally, some algebrae
such as how to solve a non-linear convection diffusion equation and some
linear algebrae algorithms are presented.

In a second part, advanced modellings are presented with their particular
treatments.

To make the documentation suitable to the developers' needs, the appendix
has been organized into sub-sections corresponding to the major steps of the
algorithm and to some important subroutines of the code.

During the development process of the code, the documentation is naturally
updated as and when required by the evolution of the source code itself.
Suggestions for improvement are \textbf{more than} welcome. In particular,
it will be necessary to deal with some transverse subjects
(such as parallelism, periodicity) which were voluntarily left out of
the first versions, to focus on the algorithms and their implementation.

To make it easier for the developers to keep the documentation up to date
during the development process, the choice is made not to base this document
on the implementation (except in the appendix) but to keep as much as possible
a general formulation. For developers who are interested in the way theory is
implemented, please refer to the \texttt{doxygen} documentation (see \doxygenfile{index.html}{local html documentation}).
A special effort will be made to link this theory guide to the \texttt{doxygen}
documentation.

\clearpage



\part{Module de base}
%-------------------------------------------------------------------------------

% This file is part of Code_Saturne, a general-purpose CFD tool.
%
% Copyright (C) 1998-2020 EDF S.A.
%
% This program is free software; you can redistribute it and/or modify it under
% the terms of the GNU General Public License as published by the Free Software
% Foundation; either version 2 of the License, or (at your option) any later
% version.
%
% This program is distributed in the hope that it will be useful, but WITHOUT
% ANY WARRANTY; without even the implied warranty of MERCHANTABILITY or FITNESS
% FOR A PARTICULAR PURPOSE.  See the GNU General Public License for more
% details.
%
% You should have received a copy of the GNU General Public License along with
% this program; if not, write to the Free Software Foundation, Inc., 51 Franklin
% Street, Fifth Floor, Boston, MA 02110-1301, USA.

%-------------------------------------------------------------------------------

\programme{bilsc2}

\hypertarget{bilsc2}{}

\vspace{1cm}
%%%%%%%%%%%%%%%%%%%%%%%%%%%%%%%%%%
%%%%%%%%%%%%%%%%%%%%%%%%%%%%%%%%%%
\section*{Function}
%%%%%%%%%%%%%%%%%%%%%%%%%%%%%%%%%%
%%%%%%%%%%%%%%%%%%%%%%%%%%%%%%%%%%

In this subroutine, called by \texttt{codits} and \texttt{turbke}, the
contributions to the explicit budget of the reconstructed (on
non-orthogonal meshes and if the user chooses to) convective and
diffusive terms of the right-hand side of a convection/diffusion
equation for a scalar $a$ are computed. These terms
write \footnote{They appear on the right-hand side of the incremental
system for cell $I$ of the momentum prediction step:
$\mathcal{EM}(\delta\vect{u}^{k+1},I) =
\mathcal{E}(\vect{u}^{n+1/2,k},I)$  (see \fort{navstv} for more details)}:

\begin{equation}\label{Base_Bilsc2_eq_continue}
\begin{array}{ll}
\mathcal{B_{\mathcal{\beta}}}((\rho\,\vect{u})^{n},a)
&=\underbrace{-\dive(\,(\rho \vect{u})^{n}a)}_{\text {convective part}}
+\underbrace{\dive(\,\beta\,\grad a)}_{\text{diffusive part}}\\
\end{array}
\end{equation}

with $\rho$, $\vect{u}$, $\beta$ and $a$ the variables at time  ${t^n}$.

See the \doxygenanchor{cs__convection__diffusion_8c.html\#cs_convection_diffusion_scalar}{programmers
reference of the dedicated reference} for further details.

%%%%%%%%%%%%%%%%%%%%%%%%%%%%%%%%%%
%%%%%%%%%%%%%%%%%%%%%%%%%%%%%%%%%%
\section*{Discretization}
%%%%%%%%%%%%%%%%%%%%%%%%%%%%%%%%%%
%%%%%%%%%%%%%%%%%%%%%%%%%%%%%%%%%%
\subsection*{\bf Convective Part}
Using the notations adopted in the subroutine \fort{navstv}, the
explicit budget corresponding to the integration over a cell
$\Omega_i$ of the convective part $-{\dive(\,(\rho\,\vect{u})^n a)}$
of $\mathcal{B_{\mathcal{\beta}}}$ can be written as a sum of the
numerical fluxes $F_{\,ij}$ calculated at the faces of the internal
cells, and the numerical fluxes $F_{\,b_{ik}}$ calculated at the
boundary faces of the computational domain $\Omega$. Let's take
$\Neigh{i}$ the set of the centres of the neighbouring cells of
${\Omega_i}$ and $\gamma_b(i)$ the set of the centres of the boundary
faces of ${\Omega_i}$ (if they exist). Thus we can write

\begin{equation}\notag
\int_{\Omega_i}{\dive( (\rho \vect{u})^n  a )\, d\Omega} =
\sum_{j\in \Neigh{i}}{F_{\,ij}((\rho \vect{u})^n, a)}
+\sum_{k\in {\gamma_b(i)}} {F_{\,{b}_{ik}}((\rho \vect{u})^n,a)}
\end{equation}

with :
\begin{equation}
F_{\,ij}((\rho \vect{u})^n,a) = \left[{(\rho \vect{u})_{\,ij}^n} \text{.}\, \vect{S}_{\,ij}\right]\ a_{\,f,ij}
\end{equation}

\begin{equation}
F_{\,{b}_{ik}}((\rho \vect{u})^n, a) =  \left[{(\rho \vect{u})_{\,{b}_{ik}}^n} \text{.}\, \vect{S}_{\,{b}_{ik}}\right]\ {a_f}_{\,{b}_{ik}}
\end{equation}
where $a_{\,f,ij}$ and ${a_f}_{\,{b}_{ik}}$ represent the values of
$a$ at the internal and boundary faces of ${\Omega_i}$, respectively.\\

Before presenting the different convection schemes available in \CS, we define:\\
\begin{figure}[h]
\hspace*{1cm}\parbox{8cm}{%
\centerline{\includegraphics[height=4cm]{facette}}}
\parbox{8cm}{%
\centerline{\includegraphics[height=4cm]{facebord}}}
\caption{\label{Base_Bilsc2_fig_geom}Definition of the geometric entities for internal (left) and a boundary faces (right).}
\end{figure}
\begin{equation}\notag
\displaystyle\alpha_{ij}=\frac{\overline{FJ'}}{\overline{I'J'}} \text{\ defined at the internal faces only and}
\end{equation}
\begin{equation}\notag
\vect{u}_{K'} = \vect{u}_{K}+(\ggrad{\vect{u}})_K\text{.}\, \vect{KK'}\, \text{\ at the first order in space, for ${K = I \,\text{or}\, J}$}
\end{equation}\\
The value of the convective flux ${F_{\,ij}}$ depends on the numerical scheme. Three different types of convection schemes are available in this subroutine:


\renewcommand{\arraystretch}{2.}
\begin{tabular}{ll}
\multicolumn{2}{l}{$\bullet\ $a $1^{st}$ order upwind scheme:}\\
&$F_{\,ij}((\rho \vect{u})^n,a)=
F^{\it{ upstream}}_{\,ij}((\rho \vect{u})^n,a)$\\
o\`u :&
$a_{\,f,ij}= \left\lbrace\begin{array}{l}
a_I \text{ si }(\rho \vect{u})_{\,ij}^n\,.\,\vect{S}_{\,ij}\geqslant 0\\
a_J \text{ si }(\rho \vect{u})_{\,ij}^n\,.\,\vect{S}_{\,ij} < 0
\end{array}\right.$\\

\multicolumn{2}{l}{$\bullet\ $a centered scheme:}\\
&$F_{\,ij}((\rho \vect{u})^n,a)=
F^{\text{\it{ centered}}}_{\,ij}((\rho \vect{u})^n,a)$\\
with :&$a_{\,f,ij}= \alpha_{ij} a_{I'}+(1-\alpha_{ij}) a_{J'}$\\
\multicolumn{2}{l}{$\bullet\ $a Second Order Linear Upwind scheme (SOLU):}\\
&$F_{\,ij}((\rho \vect{u})^n,a)=
F^{\text{\it { SOLU}}}_{\,ij}((\rho \vect{u})^n,a)$ \\
with :&
$a_{\,f,ij}=\left\lbrace\begin{array}{l}
a_I +\vect{IF}\,.\,(\grad a)_{\,I}
\text{\ \  si }(\rho \vect{u})_{\,ij}^n.\ \vect{S}_{\,ij}\geqslant 0\\
a_J + \vect{JF}\,.\,(\grad a)_{\,J}
\text{\ \  si }(\rho \vect{u})_{\,ij}^n.\ \vect{S}_{\,ij} < 0
\end{array}\right.$\\
\end{tabular}\\
\renewcommand{\arraystretch}{1.}
\footnotetext{Extrapolation  of the upwind value at the faces centre.}

The value of $F_{\,b_{ik}}$ is calculated as :
\begin{equation}\notag
{a_f}_{\,{b}_{ik}}=\left\lbrace\begin{array}{l}
a_I \text{ \ \ \ \ if }(\rho \vect{u})_{\,{b}_{ik}}^n
\text{.}\, \vect{S}_{\,{b}_{ik}}\geqslant 0\\
a_{\,{b}_{ik}}\text{ \ \ if } (\rho \vect{u})_{\,{b}_{ik}}^n
\text{.}\, \vect{S}_{\,{b}_{ik}} < 0
\end{array}\right.
\end{equation}
$a_{\,{b}_{ik}}$ is the boundary value directly computed from the prescribed boundary conditions.

\minititre{Remark 1}

When a centered scheme is used, we actually write (to ensure first order discretization in space for $a$)

\begin{equation}\notag
a_{\,f,ij} = \alpha_{ij} a_I +  (1-\alpha_{ij}) a_J  + \displaystyle
\frac{1}{2}\left[(\grad a)_I+(\grad a)_J\right] \text{.}\, \vect{OF}
\end{equation}

A factor $\displaystyle \frac{1}{2}$ is used for numerical stability reasons.

\minititre{Remark 2}
A slope test (which may introduce non-linearities in the convection operator) allows to switch from
the centered or SOLU scheme to the first order upwind scheme (without blending). Additionally, in standard mode $a_{\,f,ij}$ is computed as a weighted average between the upstream value and the centered value (blending), according to users' choice (variable $\var{BLENCV}$ in the subroutine  \fort{usini1}).


\subsection*{\bf Diffusive Part}

Similarly, the diffusive part writes :

\begin{equation}\notag
\int_{\Omega_i}{\dive(\,\beta\ \grad a)\  d\Omega} =
\sum_{j\in \Neigh{i}}{D_{\,ij}(\,\beta, a)}
+\sum_{k\in {\gamma_b(i)}} {D_{\,{b}_{ik}}(\beta, a)}
\end{equation}
with:
\begin{equation}
D_{\,ij}(\,\beta, a) = \beta_{\,ij}
\frac{a_{\,J'}- a_{\,I'}}{\overline{I'J'}} S_{\,ij}
\end{equation}
and :
\begin{equation}
D_{\,b_{ik}}(\,\beta, a) = \beta_{\,b_{ik}}
\frac{a_{\,b_{ik}}-a_{\,I'}}{\overline{I'F}} S_{\,b_{ik}}
\end{equation}
using the same notations as before, and with $S_{\,ij}$ and $S_{\,b_{ik}}$ being the norms of  vectors
$\vect{S}_{\,ij}$, and $\vect{S}_{\,b_{ik}}$ respectively.

%%%%%%%%%%%%%%%%%%%%%%%%%%%%%%%%%%
%%%%%%%%%%%%%%%%%%%%%%%%%%%%%%%%%%
\section*{Implementation}
%%%%%%%%%%%%%%%%%%%%%%%%%%%%%%%%%%
%%%%%%%%%%%%%%%%%%%%%%%%%%%%%%%%%%

In the following, the reader is reminded of the role of the variables used in the different tests:
$\bullet \ $ \var{IRCFLP}, from array \var{IRCFLU} ; indicates for the considered variables wether
or not the convective and diffusive fluxes are reconstructed \\
\hspace*{1cm}$ = 0$ : no reconstruction\\
\hspace*{1cm}$ = 1$ : reconstruction\\
$\bullet \ $ \var{ICONVP}, from array \var{ICONV} ; indicates if the considered variables is convected or not.\\
\hspace*{1cm}$ = 0$ : no convection\\
\hspace*{1cm}$ = 1$ : convection\\
$\bullet \ $ \var{IDIFFP}, from array \var{IDIFF} ; indicates if the diffusion of the considered variables is
taken into account or not.\\
\hspace*{1cm}$ = 0$ : no diffusion\\
\hspace*{1cm}$ = 1$ : diffusion\\
 $\bullet \ $ \var{IUPWIN} indicates locally, in \fort{bilsc2} (to avoid unnecessary calculations) whether a pure upwind scheme is chosen  or not for the considered variables to be convected. \\
\hspace*{1cm}$ = 0$ : no pure upwind\\
\hspace*{1cm}$ = 1$ : pure upwind is used\\
$\bullet \ $ \var{ISCHCP}, from array \var{ISCHCV} ; indicates which type of second order convection scheme is used on orthogonal meshes for the considered variable to convect  (only useful if $\var{BLENCP}>0\ $).\\
\hspace*{1cm}$ = 0$ : we use the  SOLU scheme (Second Order Linear
Upwind ) \\
\hspace*{1cm}$ = 1$ : we use a centered scheme\\
In both cases the blending coefficient \var{BLENCP} needs to be given in \fort{usini1}.\\
$\bullet \ $ \var{BLENCP}, from array \var{BLENCV} ; indicates the percentage of centered or SOLU convection scheme that one wants to use. This weighting coefficient is between $0$ and $1$.\\
$\bullet \ $ \var{ISSTPP}, from array \var{ISSTPC} ; indicates if one wants to remove the slope test that switches the convection scheme from second order to upwind if the test is positive.\\
\hspace*{1cm}$ = 0$ : a slope test is systematically used\\
\hspace*{1cm}$ = 1$ : no slope test
\subsection*{\bf Computation of the gradient $\vect{G}_{\,c,i}$ of variable $a$}
The computation of the gradient of variable $a$ is necessary for the computation of the explicit budget.   \fort{grdcel} is called everytime this gradient is needed, and it is stored in the array (\var{DPDX}, \var{DPDY}, \var{DPDZ}). The computation of the gradient is necessary in the following situations:
\hspace*{0.5cm}$\bullet \ $ if the convection is activated with a non pure upwind scheme
 ($\var{ICONVP} \ne {0}$ and $\var{IUPWIN} =~0$) {\bf and},\\
\hspace*{1cm} if we want to reconstruct the fluxes  ($\var{IRCFLP} = {1}$),\\
\hspace*{1cm}{\bf or} if we want to use the SOLU scheme
($\var{ISCHCP} = {0}$),\\
\hspace*{1cm}{\bf or} if we use the slope test ($\var{ISSTPP} = {0}$),\\
or :\\
\hspace*{0.5cm}$\bullet \ $ if there is diffusion and we want to reconstruct the fluxes ($\var{IDIFFP} \ne {0}$ and $\var{IRCFLP} =~1$).\\\\
In all other cases, the array (\var{DPDX}, \var{DPDY}, \var{DPDZ}) is set to zero.

\subsection*{\bf Computation of the upwind gradient $\vect{G}^{\,amont}_{\,c,i}$ of variable $a$}

$\vect{G}^{\,amont}_{\,c,i}$ refers to the upwind gradient of variable $a$, for cell $\Omega_i$.
It is stored in the array ($\var{DPDXA}, \var{DPDYA}, \var{DPDZA}$).\\
We also define the scalars $a^{\,amont}_{\,ij}$
and $a^{\,amont}_{\,b_{ik}}$ as:

\begin{equation}\label{Base_Bilsc2_Eq_grad_decentre}
\begin{array}{ll}
|\Omega_i|\,\vect{G}^{\,upwind}_{\,c,i}&\overset{\text{\it\small def}}{=}
\sum\limits_{j\in \Neigh{i}}a^{\,upwind}_{\,ij}\,{\vect S_{\,ij}} + \sum\limits_{k\in {\gamma_b(i)}}a^{\,upwind}_{\,b_{ik}}\,\vect{S}_{\,{b}_{ik}} \\
\end{array}
\end{equation}

After initializing it to zero, {\bf $\vect{G}^{\,amont}_{\,c,i}$ is only computed} when
the user wishes to compute {\bf a convection term with a centered or SOLU method, and a slope test}.\\
$\bullet \ $ For each cell $\Omega_i$, the face values $a_{IF}$ (variable \var{PIF}) and $a_{JF}$~ (variable \var{PJF}), are computed as:\\
\begin{equation}\notag
\begin{array}{ll}
a_{IF}= a_I + \vect {IF}\,.\,(\grad a)_{\,I} \\
a_{JF}= a_J + \vect {JF}\,.\,(\grad a)_{\,J} \\
\end{array}
\end{equation}

Depending on the sign $s^n_{ij}$ of the mass flux $(\rho \vect{u})_{\,ij}^n.\ \vect{S}_{\,ij}$,
we give $a_{IF}$ or $a_{JF}$ the value $a^{\,upwind}_{\,ij}$ of the expression
 $\sum\limits_{j\in \Neigh{i}}a^{\,upwind}_{\,ij}\,{\vect S_{\,ij}}$.\\
\begin{equation}\notag
a^{\,upwind}_{\,ij}=\left\lbrace\begin{array}{l}
a_I +\vect{IF}\,.\,(\grad a)_{\,I}
\text{\ \  si } s^n_{ij} = 1\\
a_J + \vect{JF}\,.\,(\grad a)_{\,J}
\text{\ \  si } s^n_{ij} = - 1
\end{array}\right.
\end{equation}

$\bullet \ $The boundary terms are computed in a classic manner as follows (keeping the same notations as in the other subroutines):
\begin{equation}\notag
\begin{array}{ll}
\sum\limits_{k\in {\gamma_b(i)}}a^{\,upwind}_{\,b_{ik}}\,\vect{S}_{\,{b}_{ik}}
& = \sum\limits_{k\in {\gamma_b(i)}}(\var{INC}\,A_{\,b,ik} + B_{\,b,ik}\,a_{I'})\,\vect{S}_{\,{b}_{ik}}\\
&=\sum\limits_{k\in {\gamma_b(i)}}\left[ \var{INC}\,A_{\,b,ik} +
B_{\,b,ik}\,a_{I} + B_{\,b,ik}\,\vect {II'}\,.\,\vect{G}_{\,c,i}
\right]\,\vect{S}_{\,{b}_{ik}}
\end{array}
\end{equation}
$(A_{\,b,ik},
B_{\,b,ik})_{k\in {\gamma_b(i)}}$ are stored in the arrays (\var{COEFAP}, \var{COEFBP}). The vector  $\vect{II'}$ is stored in the array (\var{DIIPBX}, \var{DIIPBY},
\var{DIIPBZ}). The surfaces  $(\vect{S}_{\,{b}_{ik}})_{k\in {\gamma_b(i)}}$ are stored in the array \var{SURFBO} .

\subsection*{\bf Summation of the numerical convective and diffusive fluxes}
The contributions to the explicit budget $[-\dive(\,(\rho \vect{u})^{n}a)
+\dive(\,\beta\,\grad a)]$ are computed and added to the right-hand side array \var{SMBR},
which has already been initialized before the call to
\fort{BILSC2} (with the explicit source terms for instance, etc.).\\
The variable \var{FLUX} gathers the convective and diffusive parts of the numerical fluxes. It is computed in a classic manner, first on the internal faces, and then on the boundary faces.
The indices $i$ and $j$ are represented by \var{II} and \var{JJ}, respectively.\\
In order to take into account (when necessary) the sign $s^n_{ij}$ of
the mass flux $(\rho\vect{u})_{\,ij}^n.\ \vect{S}_{\,ij}$, the following equations are used :

For any real $b$, we have :
\begin{equation}\notag
\left\{\begin{array}{lll}
 b &= b^{+} + b^{-} \text{  with  } b^{+} = max\  (b,0),\ \ b^{-} = min\  (b,0)\\
|\,b |&= b^{+} - b^{-}\\
b^{+}& =\displaystyle\frac{1}{2}\,[\ b + |\,b |\ ]\\
b^{-}& =\displaystyle\frac{1}{2}\,[\ b - |\,b\ |\ ]\\
\end{array}\right.
\end{equation}
In this subroutine, $b$ represents the mass flux $\var{FLUMAS(IFAC)}$
on an internal face $\var{IFAC}$ ($\var{FLUMAB(IFAC)}$ for a boundary face \var{IFAC})
; $b^{+}$ is stored in $\var{FLUI}$ and $b^{-}$ in $\var{FLUJ}$.\\\\
\hspace*{1cm}{\tiny$\blacksquare$}\, for an internal face $ij$ (\var{IFAC})\\
We calculate :
\begin{equation}\notag
\sum_{j\in \Neigh{i}}{F_{\,ij}((\rho \vect{u})^n, a)}
- \sum_{j\in \Neigh{i}}{D_{\,ij}(\,\beta, a)}\\
= \sum_{j\in \Neigh{i}}\displaystyle\left({ \left[{(\rho \vect{u})_{\,ij}^n} \text{.}\,
\vect{S}_{\,ij}\right]\ a_{\,f,ij}
- \beta_{\,ij}\frac{a_{\,J'}- a_{\,I'}}{\overline{I'J'}} S_{\,ij}}\right)
\end{equation}
The above sum corresponds to the numerical operation:
\begin{equation}\label{Base_Bilsc2_eq_flux_interne}
\begin{array}{ll}
\var{FLUX}& = \var{ICONVP} \,.\,[\ \var{FLUI}\,.\, \var{PIF} + \var{FLUJ}\,.\, \var{PJF}\ ]\\
&+\,\var{IDIFFP}\,.\,\var{VISCF(IFAC)}\,.\,[\ \var{PIP} - \var{PJP}\ ]
\end{array}
\end{equation}
The above equation does not depend on the chosen convective scheme, since the latter only affects the quantities  \var{PIF} (face value of $a$ used
when $b$ is positive) and \var{PJF} (face value of $a$ used
when $b$ is n\'egative). \var{PIP} represents $a_{I'}$, \var{PJP} $a_{J'}$ and \var{VISCF(IFAC)}
 $ \beta_{\,ij} \displaystyle \frac{S_{\,ij}}{\overline{I'J'}}$  .\\
The treatment of diffusive part is identical (either with or without reconstruction). Consequently, only the numerical scheme relative to the convection differs.\\\\
\hspace*{1cm}{\tiny$\blacksquare$}\, for a boundary face $ik$ (\var{IFAC})\\
We compute the terms :
\begin{equation}\notag
\sum_{k\in {\gamma_b(i)}} {F_{\,{b}_{ik}}((\rho \vect{u})^n,a)}
- \sum_{k\in {\gamma_b(i)}} {D_{\,{b}_{ik}}(\beta, a)}
=\sum_{k\in {\gamma_b(i)}}\displaystyle\left(\left[{(\rho
\vect{u})_{\,{b}_{ik}}^n} \text{.}\, \vect{S}_{\,{b}_{ik}}\right]\
{a_f}_{\,{b}_{ik}}- \beta_{\,b_{ik}}
\frac{a_{\,b_{ik}}-a_{\,I'}}{\overline{I'F}} S_{\,b_{ik}}\right)
\end{equation}
with:
\begin{equation}\notag
\begin{array}{lll}
a_{I'}& = a_I + \vect{II'}\,.\,\vect{G}_{\,c,i}\\
a_{\,{b1}_{ik}} &= \var{INC}\,A_{\,b,ik} + B_{\,b,ik}\,a_{I'}\\
a_{\,b_{ik}}& = \var{INC}\,A^{diff}_{\,b,ik} + B^{diff}_{\,b,ik}\,a_{I'}\\
\end{array}
\end{equation}
The coefficients $( A_{\,b,ik}, B_{\,b,ik} )_{k\in {\gamma_b(i)}}$ $\left(\ \text{resp.} ( A^{diff}_{\,b,ik}, B^{diff}_{\,b,ik} )_{k\in
{\gamma_b(i)}}\ \right)$ represent the boundary conditions associated with
$a$ (resp. the diffusive fluxes \footnote { see
\var{clptur} for more details. The difference is actually only effective when the $k-\epsilon$ model is used, and for the velocity only.} of $a$).\\
The above sum corresponds to the numerical operation:
\begin{equation}\label{Base_Bilsc2_eq_flux_bord}
\begin{array}{ll}
\var{FLUX}& = \var{ICONVP} \,.\,[\ \var{FLUI}\,.\, \var{PVAR(II)} + \var{FLUJ}\,.\, \var{PFAC}\ ]\\
&+\,\var{IDIFFP}\,.\,\var{VISCB(IFAC)}\,.\,[\ \var{PIP} - \var{PFACD}\ ]
\end{array}
\end{equation}
where \var{PFAC} represents $a_{\,{b1}_{ik}}$, \var{PIP} $a_{I'}$, $\var{PFACD}$ $a_{\,b_{ik}}$ and $\var{VISCB(IFAC)}$
$ \beta_{\,b_{ik}} \displaystyle\frac{S_{\,b_{ik}}}{\overline{I'F}} $.\\
This treatment is common to all schemes, because boundary values only depend on boundary conditions, and because a very simplified expression of $F_{\,{b}_{ik}}$ is used (upwind)
\footnote{Actually, ${a_f}_{\,{b}_{ik}}$ is $a_I$ if  ${(\rho
\vect{u})_{\,{b}_{ik}}^n} \text{.}\, \vect{S}_{\,{b}_{ik}}\,\geqslant \,0$, $a_{\,{b1}_{ik}}$ otherwise.}.\\\\
We still have to compute, when the convection option is activated ($\var{ICONVP} = 1$),
the values of variables \var{PIF} and
\var{PJF}, for any internal face \var{IFAC} between cell
$\Omega_i$  and $\Omega_j$.
\subsubsection*{\bf Calculation of the flux in pure upwind $\var{IUPWIN} = 1$}

In this case, there is no reconstruction since only
the values \var{PVAR(II)} and \var{PVAR(JJ)} at the cell centres are needed.
\begin{equation}
\begin{array}{ll}
\var{PIF} &= \var{PVAR(II)} \\
\var{PJF} &= \var{PVAR(JJ)} \\
\end{array}
\end{equation}
The variable \var{INFAC} counts the number of calculations in pure upwind,
in order to be printed in the listing file.
In order to obtain the global numerical flux \var{FLUX} (convective + diffusive)
associated, the following operations are performed :\\
$\bullet$ calculation of vectors \vect{II'} and \vect{JJ'},\\
$\bullet$ calculation of the face gradient (\var{DPXF}, \var{DPYF}, \var{DPZF}) with the half-sum of the cell gradients $\vect{G}_{\,c,i}$ et $\vect{G}_{\,c,j}$,\\
$\bullet$ calculation of the reconstructed (if necessary) values $a_{I'}$ and  $a_{J'}$ (variables \var {PIP} and \var{PJP}, respectively) given by :
\begin{equation}\label{Base_Bilsc2_Eq_Rec_Dif1}
\begin{array}{lll}
a_{K'}&= a_K +  \var{IRCFLP}\,.\,\vect {KK'}\,.\,\displaystyle\frac{1}{2}\,(\,\vect{G}_{\,c,i}\,+\,\vect{G}_{\,c,j}\,)&\text{ K = I et J}\\
\end{array}
\end{equation}
$\bullet$ calculation of the quantities \var{FLUI} and \var{FLUJ},\\
$\bullet$ calculation of the flux \var{FLUX} using (\ref{Base_Bilsc2_eq_flux_interne}).\\
The computation of the sum in \var{SMBR} is straight-forward, following (\ref{Base_Bilsc2_eq_continue})
\footnote{ taking into account the negative sign of $\mathcal{B_{\mathcal{\beta}}}$.} .
\subsubsection*{\bf Calculation of the flux with a centered or SOLU scheme ($\var{IUPWIN} = 0$)}

The two available second order schemes on orthogonal meshes are the centered scheme and the SOLU scheme.
\\ In both cases, the following operations are performed:\\
$\bullet$ calculation of the vector \vect{II'}, the array (\var{DIIPFX}, \var{DIIPFY}, \var{DIIPFZ}) and the vector \vect{JJ'}, the array (\var{DJJPFX}, \var{DJJPFY}, \var{DJJPFZ})\\
$\bullet$ calculation of the face gradient (\var{DPXF}, \var{DPYF}, \var{DPZF})
haff-sum of the cell gradients $\vect{G}_{\,c,i}$ and $\vect{G}_{\,c,j}$,\\
$\bullet$ calculation of the possibly reconstructed (if \var{IRCFLP} = 1) values $a_{I'}$ and  $a_{J'}$ (variables \var {PIP} and \var{PJP}, respectively) given by :
\begin{equation}\label{Base_Bilsc2_Eq_Rec_Dif2}
\begin{array}{lll}
a_{K'}&= a_K +  \var{IRCFLP}\,.\,\vect {KK'}\,.\,\displaystyle\frac{1}{2}\,(\,\vect{G}_{\,c,i}\,+\,\vect{G}_{\,c,j}\,)&\text{ K = I and J}\\
\end{array}
\end{equation}
$\bullet$ calculation of \var{FLUI} and \var{FLUJ}.\\\\
%\hspace*{2.5cm}{\tiny$\clubsuit$} en centr\'e ($\var{ISCHCP} = 1$)\\
\hspace*{2cm}{\tiny$\blacksquare$} \underline{ without slope test ($\var{ISSTPP} = 1$)}\\\\
\hspace*{2.5cm}{\tiny$\bigstar$} with a centered scheme ($\var{ISCHCP} = 1$)\\
The values of the variables  \var{PIF} and \var{PJF} are equal, and calculated
using the weighting coefficient $\displaystyle\alpha_{ij}$ as follows:
\begin{equation}
\begin{array}{ll}
P_{IF} &=\displaystyle\alpha_{ij}\,.\, P_{I'} + (1 - \displaystyle\alpha_{ij})\,.\, P_{J'}\\
P_{JF} &= P_{IF}
\end{array}
\end{equation}
\hspace*{2.5cm}{\tiny$\bigstar$} with a SOLU scheme ($\var{ISCHCP} = 0$)\\\\
After calculating the vectors $\vect{IF}$ and $\vect{JF}$, the values of the variables \var{PIF} and \var{PJF} are computed as follows:
\begin{equation}
\begin{array}{ll}
P_{IF} &= P_I + \vect{IF}\,.\,\vect{G}_{\,c,i}\\
P_{JF} &= P_J + \vect{JF}\,.\,\vect{G}_{\,c,j}\\
\end{array}
\end{equation}
 \var{PIF} and \var{PJF} are systematically reconstructed in order to avoid
using pure upwind, {\it i.e.} this formulae is applied even when the user
chooses not to reconstruct ($\var{IRCFLP} = 0$).\\\\
\hspace*{2cm}{\tiny$\blacksquare$} \underline{ with slope test ($\var{ISSTPP}
= 0$)}\\\\
The procedure is quite similar to the one described in the previous paragraph.
There is, in addition to the previous procedure, a slope test that makes under
certain conditions the scheme switch locally (but systematically) from the chosen
 centered or SOLU scheme to a pure upwind scheme. \\\\
\hspace*{2.5cm}$\rightsquigarrow$\ \ calculation of the slope test\\
Equation (\ref{Base_Bilsc2_Eq_grad_decentre}) writes on an internal cell
$\Omega_i$, with
$ s^n_{ij} = sgn \left[(\rho\vect{u})_{\,ij}^n\,.\,\vect{S}_{\,ij}\right]$ :\\
\begin{equation}\notag
\begin{array}{lll}
|\Omega_i|\,\vect{G}^{\,upwind}_{\,c,i}&=
\sum\limits_{j\in \Neigh{i}}a^{\,upwind}_{\,ij}\,{\vect S_{\,ij}} \\
&=\sum\limits_{j\in
\Neigh{i}}\left[\displaystyle\frac{1}{2}(\ s^n_{ij} + 1\ )\right.&a_{IF} +
\left.\displaystyle\frac{1}{2}(\ s^n_{ij} - 1\ )\,a_{JF}\right]\ \vect{S}_{\,ij}\\
&=\sum\limits_{j\in
\Neigh{i}}\left[\displaystyle\frac{1}{2}(\ s^n_{ij} + 1\ )\right.&(\ a_I + \vect {IF}\,.\,(\grad a)_{\,I}\ ) \\
& &
+\displaystyle\frac{1}{2}\ \left.(\ s^n_{ij} - 1\ )\,(\ a_I + \vect{JF}\,.\,(\grad a)_{\,J}\ )\right]\ \vect{S}_{\,ij}
\end{array}
\end{equation}\\
On a cell $\Omega_i$ with neighbours $(\Omega_j)_{j\in \Neigh{i}}$, the classic slope test
consists in locating where a variable $a$ is non-monotonic by studying the sign of
the scalar product of the cell gradients of $\vect{G}_{\,c,i}$
and $\vect{G}_{\,c,j}$. If this product is negative, we switch to an upwind scheme, if it is
positive, we use a centered or SOLU scheme.\\
Another technique which also ensures the monotonicity of the solution is
to apply this criterion to the upwind gradients
$\,\vect{G}^{\,amont}_{\,c,k}\,$ or to their normal projection on
face $(\,\vect{G}^{\,amont}_{\,c,k}\,.\,\vect{S}_{\,kl}\,)$.\\
We then study the sign of the product
$\,\vect{G}^{\,amont}_{\,c,i}\,.\,\vect{G}^{\,amont}_{\,c,j}\,$ or of the product
$(\,\vect{G}^{\,amont}_{\,c,i}\,.\,\vect{S}_{\,ij}\,)\,.\,(\,\vect{G}^{\,amont}_{\,c,j}\,.\,\vect{S}_{\,ij}\,)\,$.
The slope test implemented is based on the first quantity,
$\,\vect{G}^{\,amont}_{\,c,i}\,.\,\vect{G}^{\,amont}_{\,c,j}\,$
(the second one was abandonned because it was found to be less general). The
choice of a slope test based on
$\,\vect{G}^{\,amont}_{\,c,i}\,.\,\vect{G}^{\,amont}_{\,c,j}\,$
comes from the following line of argument in one-dimension \footnote{Information
on the second derivative would permit to study more finely the behaviour and the strong
variations of $a$.}:

Let's take  $p$ a second order in $x$ polynomial function. Its value at points $I-1$, $I$, $I+1$ of coordinates  $x_{I-1}$, $x_I$ and $x_{I+1}$ are $p_{I-1}$, $p_I$, and
$p_{I+1}$, respectively. To simplify, we
suppose that $I$ is the origin $O$ (~$x_I = 0$~), and that the grid spacing $h$ is constant,
which results in $ x_{I+1} = - x_{I-1} = h $. Additionally, we suppose that the
velocity is orientated from point $I$ towards point $I+1$, {\it i.e.} $s^n_{ij} =
1$. Therefore we consider the points $I-1$, $I$ and $I+1$ for  the face
$ij$ which is located between $I$ and $I+1$.\\
The sign of the product $ p'(x_{I-1})\,.\,p'(x_{I+1}) $ indicates the monotonicity
of function $p$. If this product is positive, the function is monotonic and
we use a centered or a SOLU scheme, otherwise, we switch to an upwind scheme. By identifying
the polynomial coefficients using the equations $\  p\,(x_{I-1}) = p_{I-1}\
$, $\ p\,(x_I) = p_I\ $,  $\ p\,(x_{I+1}) = p_{I+1}\ $, we obtain :\\
\begin{equation}
\begin{array}{lll}
p'(x_{I-1})& = + \displaystyle \frac{p_{I+1} -  p_{I-1}}{2h} & +
\left[\displaystyle {\ \frac{p_I - p_{I-1}}{h} - \frac{p_{I+1} -  p_I}{h} }\right]\\
p'(x_{I+1})& = + \displaystyle \frac{p_{I+1} -  p_{I-1}}{2h} & -
\left[\displaystyle {\ \frac{p_I - p_{I-1}}{h} - \frac{p_{I+1} -  p_I}{h} }\right] \\
\end{array}
\end{equation}
or after simplification :
\begin{equation}
\begin{array}{lll}
p'(x_{I-1}) = G_{\,c,i} + \left(\ G^{\,amont}_{\,c,i} - \displaystyle \frac{p_{I+1} -  p_I}{h}\right)\\
p'(x_{I+1}) = G_{\,c,i} - \left(\ G^{\,amont}_{\,c,i} - \displaystyle \frac{p_{I+1} -  p_I}{h}\right)\\
\end{array}
\end{equation}
We know that :\\
{\tiny $\clubsuit$} $\displaystyle \frac{p_{I+1} -  p_I}{h}$ representes the
upwind derivative at point $I+1$, directly accessible by the
values of $p$ in the neighbouring cells of face $ij$,\\
{\tiny $\clubsuit$} $\displaystyle \frac{p_{I+1} -  p_{I-1}}{2h}$ represents the
centered derivative (in finite volume) at point $I$, namely $ G_{\,c,i}$,\\
{\tiny $\clubsuit$} $\displaystyle\frac{p_I - p_{I-1}}{h}$ represents the value of the
upwind derivative (in finite volume) at point $I$, namely $ G^{\,amont}_{\,c,i}$.\\
The slope test relative to $ p'(x_{I-1})\,.\,p'(x_{I+1}) $ reduces
to studying the sign of $\mathcal{TP}_{1d}$ :\\
\begin{equation}
\begin{array}{ll}
\mathcal{TP}_{1d}
&= \left(G_{\,c,i}\ +\ [\ G^{\,amont}_{\,c,i}
-\displaystyle \frac{p_{I+1} -  p_I}{h}]\right).\left(G_{\,c,i}\ -\ [\ G^{\,amont}_{\,c,i}
-\displaystyle \frac{p_{I+1} -  p_I}{h}]\right) \\
 &= |G_{\,c,i}|^2 - (\ G^{\,amont}_{\,c,i}
-\displaystyle \frac{p_{I+1} -  p_I}{h})^2\\
\end{array}
\end{equation}
Using a similar line of argument, a possible extension to higher dimensions
 consists in replacing the values $ G_{\,c,k} $  and  $ G^{\,amont}_{\,c,k} $ by
$(\,\vect{G}_{\,c,k}\,.\,\vect{S}_{\,kl}\,)$ eand
$(\,\vect{G}^{\,amont}_{\,c,k}\,.\,\vect{S}_{\,kl}\,)$ respectively. After simplifications, this leads us
to the formulae $\mathcal{TP}^{+}_{3d}$ :
\begin{equation}
\mathcal{TP}^{+}_{3d} = (\vect{G}_{\,c,i}\,.\, \vect{S}_{\,ij})^2 -
(\vect{G}^{\,amont}_{\,c,i}\,.\,\vect{S}_{\,ij} - \displaystyle\frac{a_{\,J}- a_{\,I}}{\overline{I'J'}} S_{\,ij})^2
\end{equation}
for $(\rho \vect{u})_{\,ij}^n.\ \vect{S}_{\,ij} > 0$.\\
Similarly, we can deduce a  $\mathcal{TP}^{-}_{3d}$
associated with $(\rho \vect{u})_{\,ij}^n.\ \vect{S}_{\,ij} < 0$, defined by
:\\
\begin{equation}
\mathcal{TP}^{-}_{3d} = (\vect{G}_{\,c,j}\,.\, \vect{S}_{\,ij})^2 -
(\vect{G}^{\,amont}_{\,c,j}\,.\,\vect{S}_{\,ij} - \displaystyle\frac{a_{\,J}- a_{\,I}}{\overline{I'J'}} S_{\,ij})^2
\end{equation}
\\
We introduce the variables  \var{TESTI}, \var{TESTJ} and \var{TESTIJ} computed as:
\begin{equation}
\begin{array}{lll}
\var{TESTI}&=\vect{G}^{\,amont}_{\,c,i}\,.\, \vect{S}_{\,ij}\\
\var{TESTJ}&=\vect{G}^{\,amont}_{\,c,j}\,.\, \vect{S}_{\,ij}\\
\var{TESTIJ}&=\vect{G}^{\,amont}_{\,c,i}\,.\, \vect{G}^{\,amont}_{\,c,j}\\
\end{array}
\end{equation}
The quantity \var{TESQCK} corresponding to $\mathcal{TP}_{3d}$, is
computed dynamically, depending on the sign of the mass flux  $s^n_{ij}$.\\
\hspace*{2.5cm}$\rightsquigarrow$\ \ consequently  :\\\\
\hspace*{1.5cm}{\tiny$\diamond$} if $(\rho \vect{u})_{\,ij}^n.\ \vect{S}_{\,ij} > 0$ and \\
\hspace*{2cm} if $\underbrace{(\vect{G}_{\,c,i}\,.\, \vect{S}_{\,ij})^2 - (\vect{G}^{\,amont}_{\,c,i}\,.\,\vect{S}_{\,ij} - \displaystyle\frac{a_{\,J}-
a_{\,I}}{\overline{I'J'}} S_{\,ij})^2}_{\var{TESQCK}}  < 0 \text{ or } (\vect{G}^{\,amont}_{\,c,i}\,.\,\vect{G}^{\,amont}_{\,c,j}) < 0$,\\\\
%$
\hspace*{1.5cm} or~:\\
\hspace*{1.5cm} if $(\rho \vect{u})_{\,ij}^n.\ \vect{S}_{\,ij} < 0$  and \\
\hspace*{2cm} if $\underbrace{(\vect{G}_{\,c,j}\,.\, \vect{S}_{\,ij})^2 -
(\vect{G}^{\,amont}_{\,c,j}\,.\,\vect{S}_{\,ij} - \displaystyle\frac{a_{\,J}-
a_{\,I}}{\overline{I'J'}} S_{\,ij})^2}_{\var{TESQCK}} < 0 \text{ or }
(\vect{G}^{\,amont}_{\,c,i}\,.\,\vect{G}^{\,amont}_{\,c,j}) < 0$,\\\\
%$
\hspace*{1.5cm}then we switch to a pure upwind scheme:
\begin{equation}
\begin{array}{ll}
\var{PIF} &= \var{PVAR(II)} \\
\var{PJF} &= \var{PVAR(JJ)} \\
\end{array}
\end{equation}
\hspace*{1.5cm} and \var{INFAC} is incremented.\\\\
\hspace*{1.5cm}{\tiny$\diamond$} otherwise :\\
\hspace*{1.5cm} the centered or the SOLU scheme values values are used as before :\\\\
\hspace*{2.5cm}{\tiny$\bigstar$} with a centered scheme ($\var{ISCHCP} = 1$)\\
\hspace*{1.5cm} The values of the variables  \var{PIF} and \var{PJF} are equal and calculated
using the weighting coefficient $\displaystyle\alpha_{ij}$ :
\begin{equation}
\begin{array}{ll}
P_{IF} &=\displaystyle\alpha_{ij}\,.\, P_{I'} + (1 - \displaystyle\alpha_{ij})\,.\, P_{J'}\\
P_{JF} &= P_{IF}
\end{array}
\end{equation}
\hspace*{2.5cm}{\tiny$\bigstar$} with a SOLU scheme ($\var{ISCHCP} = 0$)\\
\hspace*{1.5cm} After calculating the vectors $\vect{IF}$ and $\vect{JF}$, the values of the variables \var{PIF} and \var{PJF} are computed as follows:
\begin{equation}
\begin{array}{ll}
P_{IF} &= P_I + \vect{IF}\,.\,\vect{G}_{\,c,i}\\
P_{JF} &= P_J + \vect{JF}\,.\,\vect{G}_{\,c,j}\\
\end{array}
\end{equation}
\hspace*{1.5cm} \var{PIF} and \var{PJF} are systematically reconstructed in order to avoid
using pure upwind, {\it i.e.} this formulae is applied even when the user
chooses not to reconstruct ($\var{IRCFLP} = 0$).\\\\

Wether the slope test is activated or not, when the centered or the SOLU schemes are activated,
a blending coefficient (\var{BLENCP}) between 0 and 1, provided by the user, enables to blend, if desired, the chosen scheme and the pure upwind scheme following the formulae:
\begin{equation}
\begin{array}{ll}
P_{IF} &= \var{BLENCP} P^{\,\it (centre\  ou\  SOLU) }_{\,IF} + (1 - \var{BLENCP})\  P_{II}\\
P_{JF} &= \var{BLENCP} P^{\,\it (centre\  ou\  SOLU) }_{\,JF} + (1 - \var{BLENCP})\  P_{JJ}\\
\end{array}
\end{equation}
$\bullet$ calculation of \var{FLUI} and \var{FLUJ},\\
$\bullet$ calculation of the flux \var{FLUX} using equation (\ref{Base_Bilsc2_eq_flux_interne}).\\
The computation of the sum in \var{SMBR} is straight-forward, following (\ref{Base_Bilsc2_eq_continue})\footnote{ taking into account the negative sign of $\mathcal{B_{\mathcal{\beta}}}$.}

\noindent\minititre{Remark}
For more information on the convection schemes and the slope test in \CS
(version 1.1), the reader is referred to EDF internal report EDF HI-83/04/020 (F. Archambeau,
2004).

\newpage
%%%%%%%%%%%%%%%%%%%%%%%%%%%%%%%%%%
%%%%%%%%%%%%%%%%%%%%%%%%%%%%%%%%%%
\section*{Points to treat}
%%%%%%%%%%%%%%%%%%%%%%%%%%%%%%%%%%
%%%%%%%%%%%%%%%%%%%%%%%%%%%%%%%%%%
\etape{Convection scheme}
\hspace*{1cm}$\rightsquigarrow$\  \underline {Upwind scheme}\\
As all first-order schemes, it is robust, but introduces severe numerical diffusion. \\
\hspace*{1cm}$\rightsquigarrow$\ \underline {Centered or SOLU scheme}\\
This type of schemes can generate numerical oscillations, that can cause the calculation
to blow up. It can also lead to physical scalars taking unphysical values.\\
Considering these limitations, other schemes are currently being tested and implemented
in order to improve the quality of the schemes  available to the users.\\\\
\etape{Diffusion scheme}
The formulae :
\begin{equation}
D_{\,ij}(\,\beta, a) = \beta_{\,ij} \frac{a_{\,J'}- a_{\,I'}}{\overline{I'J'}} S_{\,ij}
\end{equation}
is second-order accurate only for $\alpha_{ij}\ = \displaystyle\frac{1}{2}$.
A possible correction may be to write :\\
\begin{equation}
\vect{G}_{\,f,ij}\,.\,\vect{S}_{\,ij} = (\grad a)_{\,ij} = \frac{a_{\,J'}- a_{\,I'}}{\overline{I'J'}}\,.\,\vect{S}_{\,ij} + (\displaystyle\frac{1}{2} -
\alpha_{ij}\ )\left[(\grad a)_{I'} - (\grad a)_{J'}\right]\,.\,\vect{S}_{\,ij}
\end{equation}
with a gradient limiter and a computation of $\beta_{ij}$ which does not
alter the order of accuracy.\\\\
\etape{Implementation}
In order to improve the CPU time, an effort on loops can be done.
More particularly, there is a test \var{IF} inside of a loop on variable \var{IFAC} that needs
to be checked.\\\\
\etape{Calculation of the gradient used during the reconstruction of the diffusive fluxes}
Why do we use $\displaystyle\frac{1}{2}\,(\,\vect{G}_{\,c,i}\,+\,\vect{G}_{\,c,j}\,)$ instead of $\,\vect{G}_{\,c,k}\,$, for $k=i$ or for $k=j$ in the reconstructed values  $a_{I'}$ or $a_{J'}$ of (\ref{Base_Bilsc2_Eq_Rec_Dif1}) and (\ref{Base_Bilsc2_Eq_Rec_Dif2})?

%-------------------------------------------------------------------------------

% This file is part of Code_Saturne, a general-purpose CFD tool.
%
% Copyright (C) 1998-2019 EDF S.A.
%
% This program is free software; you can redistribute it and/or modify it under
% the terms of the GNU General Public License as published by the Free Software
% Foundation; either version 2 of the License, or (at your option) any later
% version.
%
% This program is distributed in the hope that it will be useful, but WITHOUT
% ANY WARRANTY; without even the implied warranty of MERCHANTABILITY or FITNESS
% FOR A PARTICULAR PURPOSE.  See the GNU General Public License for more
% details.
%
% You should have received a copy of the GNU General Public License along with
% this program; if not, write to the Free Software Foundation, Inc., 51 Franklin
% Street, Fifth Floor, Boston, MA 02110-1301, USA.

%-------------------------------------------------------------------------------

\programme{clptur}

\hypertarget{clptur}{}

\vspace{1cm}
%%%%%%%%%%%%%%%%%%%%%%%%%%%%%%%%%%
%%%%%%%%%%%%%%%%%%%%%%%%%%%%%%%%%%
\section*{Function}
%%%%%%%%%%%%%%%%%%%%%%%%%%%%%%%%%%
%%%%%%%%%%%%%%%%%%%%%%%%%%%%%%%%%%
This subroutine is dedicated to the calculation of the wall boundary conditions. The notations introduced in \var{CONDLI} for the general boundary conditions will be used.

The wall boundary conditions refer to all the boundary conditions for the velocity, the turbulent variables ($k$, $\varepsilon$, $R_{ij}$), the temperature when it has a prescribed value at the wall (or the enthalpy and more generally the {\it VarScalaires}\footnote{As in \fort{condli} the {\it VarScalaire} are any solution of a convection-diffusion equation apart from the velocity, the pressure and the turbulent variables  $k$, $\varepsilon$ and $R_{ij}$. More specifically, the name {\it VarScalaire} can refer to the temperature, the enthalpy or a passive scalar.  } to treat at the wall by using a similarity law for the associated boundary layer). For the {\it VarScalaire} in particular, when the boundary conditions at the wall are of Neumann type (homogeneous or not), they are treated in  \fort{condli} and don't present them here. In particular, the boundary conditions of the {\it VarScalaires} are not treated here because their treatment at the wall is of homogeneous Neumann type.

We present the calculation of the pair of coefficients $A_b$ and $B_b$ which are used during the computation of certain discretized terms of the equations to solve, and which allow in particular to determine a value associated with the boundary faces $f_{b,int}$ (at a point located at the "centre" of the boundary face, the barycentre of its vertices) using the formulae   $f_{b,int} = A_b+B_b\,f_{I'}$ ($f_{I'}$ is the value of the variable at point $I'$, the projection of the centre of the boundary cell onto the line normal to the boundary face and passing through its centre~: see figure~\ref{Base_Clptur_fig_flux_clptur}).

See the \doxygenfile{clptur_8f90.html}{programmers reference of the dedicated subroutine} for further details.

\begin{figure}[h]
\centerline{\includegraphics[height=7cm]{fluxbord}}
\caption{\label{Base_Clptur_fig_flux_clptur}Boundary cell.}
\end{figure}

%%%%%%%%%%%%%%%%%%%%%%%%%%%%%%%%%%
%%%%%%%%%%%%%%%%%%%%%%%%%%%%%%%%%%
\section*{Discretisation}
%%%%%%%%%%%%%%%%%%%%%%%%%%%%%%%%%%
%%%%%%%%%%%%%%%%%%%%%%%%%%%%%%%%%%

\etape{Notations\vspace{0,3cm}}
%%%%%%%%%%%%%%%%%%%%%%%%%%%%%%%%%%%%%%%%%%%%%%%%%%%%%%%%%%%%%%%%%%%%%%%%%%%%%%%
The velocity of the wall is noted
$\vect{v}_p$. We assume it is projected onto the plane tangent to the wall (if it is not, then the code projects it).


The velocity of the fluid is noted $\vect{u}$. Index $I$, $I'$ or $F$ denotes the point at which the velocity is estimated. The component tangent to the wall writes $u_\tau$. The fluid velocity in the coordinate system attached to the wall ("relative" velocity) writes $\vect{u}^r=\vect{u} - \vect{v}_p$.



The orthonormal coordinate system attached to the wall writes
$\hat {\mathcal R}=(\vect{\tau},\vect{\tilde{n}},\vect{b})$.
\begin{itemize}
\item [$\bullet$] $\vect{\tilde{n}}=-\vect{n}$ is the unit vector orthogonal to the wall and directed towards the interior of the computational domain.
\item [$\bullet$] $\vect{\tau} = \displaystyle\frac{1}{\|\vect{u}^r_{I'}-(\vect{u}^r_{I'}\,.\,\vect{\tilde{n}})\|}\left[\vect{u}^r_{I'}-(\vect{u}^r_{I'}\,.\,\vect{\tilde{n}})\right]$ is the unit vector parallel to the projection of the relative velocity at $I'$, $\vect{u}^r_{I'}$, in the plane tangent to the wall
 ({\it i.e.} orthogonal to $\vect{\tilde{n}}$)~: see
figure~\ref{Base_Clptur_fig_flux_clptur}.
\item [$\bullet$] $\vect{b}$ is the unit vector which completes the positively oriented coordinate system.
\end{itemize}
\vspace{0.2cm}

The dimensionless limit distance which separates the viscous sublayer from the logarithmic region writes $y^+_{lim}$. Its value is $1/\kappa$ (with $\kappa = 0,42$) in general (to ensure the continuity of the velocity gradient) and 10.88 in LES (to ensure the continuity of the velocity).

In the case of the {\bf two velocity scale model},
\begin{itemize}
\item [-] $u_k$ is the friction velocity at the wall obtained from the turbulent kinetic energy.
We write $u^*$ the friction velocity at the wall calculated from the equation
 $ \displaystyle\frac{u^r_{\tau,I'}}{u^*} = f(y^+_k)$.

\item [-]
$y^+_k$  represents a dimensionless wall distance,
$y^+_k= \displaystyle\frac{u_k\,I'F}{\nu}$ ($\nu$ is the molecular kinematic viscosity
taken at the centre $I$ of the boundary cell).
The function $f$ gives the ideal shape of the velocity profile.
It is piecewisely approximated by the logarithmic law
 $f(z)=f_1(z)= \displaystyle\frac{1}{\kappa}ln(z)+5,2$ for
$z> y^+_{lim}$
and by the linear law $f(z)=f_2(z)=z$ otherwise.

\item [-] The two velocity scale $u_k$ and $u^*$ are simple to compute
but their computation requires the knowledge of the turbulent kinetic energy $k_I$
at the centre of cell adjoint to the boundary face  (with the $R_{ij}-\varepsilon$ model, we use half the trace of the Reynolds stress tensor).


\item [-] The two velocity scale model is the default model in \CS.
It often permits, and in particular in cases with heat transfer, to reduce the
effects of certain flaws associated to the $k-\varepsilon$ model.
\end{itemize}


Later on, we will use $u^*$ and $u_k$ for the boundary conditions of the velocity and scalars (in particular the temperature).


\begin{equation}\label{Base_Clptur_Eq_Mod_'2ech_Vit}
\begin{array}{l}
\text{\bf Two velocity scale model}\\
\left\{\begin{array}{l}
u_k = C_\mu^\frac{1}{4}k_I^\frac{1}{2}\\
u^* \text{is solution of }
\left\{\begin{array}{lll}
\displaystyle\frac{u^r_{\tau,I'}}{u^*} &=
\displaystyle\frac{1}{\kappa}ln(y^+_k)+5,2 &\text { for }y^+_k>y^+_{lim}\\
\displaystyle\frac{u^r_{\tau,I'}}{u^*} &= y^+_k                &\text { for }
y^+_k \leqslant y^+_{lim}
\end{array}\right.    \\
\qquad\qquad
\text{   with   } C_\mu =0,09\qquad y^+_k=\displaystyle\frac{u_k\,I'F}{\nu}
                                                      \text{  and }\kappa = 0,42
\end{array}\right.
\end{array}
\end{equation}



In the case of the {\bf one velocity scale model},

we write $u^*$ the only
friction velocity at the wall solution of the equation
$\displaystyle\frac{u^r_{\tau,I'}}{u^*} = f(y^+)$.
$y^+$  represents a dimensionless wall distance
$y^+=\displaystyle\frac{u^*\,I'F}{\nu}$ ($\nu$ is the molecular kinematic viscosity
taken at the centre $I$ of the boundary cell).
The function $f$ gives the ideal shape of the velocity profile, as in the case of the two velocity scale model. One can note that this friction velocity, calculated using a more complicated approach (Newton method), can however be obtained without making any reference to the turbulent variables ($k$, $\varepsilon$, $R_{ij}$). For convenience in the case of the one velocity scale model, we write $u_k=u^*$.

Later on, we will use $u^*$ and $u_k$ for the boundary conditions of the velocity and scalars (in particular the temperature).

\begin{equation}
\begin{array}{l}
\text{\bf Mod\`ele \`a une \'echelle de vitesse}\\
\left\{\begin{array}{l}
u_k = u^*\\
u^* \text{ solution de } \left\{\begin{array}{lll}
\displaystyle\frac{u^r_{\tau,I'}}{u^*} &=
\displaystyle\frac{1}{\kappa}ln(y^+)+5,2 &\text { pour }y^+>y^+_{lim}\\
\displaystyle\frac{u^r_{\tau,I'}}{u^*} &= y^+                         &\text
{pour } y^+\leqslant y^+_{lim}
\end{array}\right.\\
\qquad\qquad\text{   avec   } y^+=\displaystyle\frac{u^*\,I'F}{\nu}
                                                      \text{  et }\kappa=0,42
\end{array}\right.
\end{array}
\end{equation}


{\bf Remark~:} Hereafter, we provide three exemples
based on the two velocity scale model.
\begin{itemize}
\item In this way, one can implement a specific wall function~:
$$\displaystyle\frac{u_{\tau,I'}}{u^*}=g(y^+)$$
by simply imposing
$\displaystyle{u^*}=u_{\tau,I'}/g(y^+)$.
\item It is also possible to use a rough-wall wall function such as~:
$$\displaystyle\frac{u_{\tau,I'}}{u^*}=\displaystyle\frac{1}{\kappa}\,ln(\frac{y}{\xi})+8,5$$
where $\xi$ is the height of the roughness elements at the wall~: one just has to impose
$\displaystyle u^*=u_{\tau,I'}/\left[\frac{1}{\kappa}ln(\frac{y}{\xi})+8,5\right]$,
 $y$ being deduced from $y^+$, available as an argument, by the equation
$\displaystyle y=y^+\frac{\nu}{u_k}$.
\item Even a more general correlation could be used of
Colebrook type~:
$$u^*=u_{deb}/\left[-4\sqrt{2}log_{10}\left(\displaystyle\frac{2,51}{2\sqrt{2}D_H^+}+\frac{\xi}{3,7\,D_H}\right)\right]$$
where $D_H^+$ is the hydraulic diameter made dimensionless using $u_k$, $\nu$,
$u_{deb}$ the mean streamwise velocity
and $\displaystyle\frac{\xi}{D_H}$ the relative roughness.
\end{itemize}



\etape{Boundary conditions for the velocity in $k-\varepsilon$\vspace{0,3cm}}
%%%%%%%%%%%%%%%%%%%%%%%%%%%%%%%%%%%%%%%%%%%%%%%%%%%%%%%%%%%%%%%%%%%%%%%%%%%%%%%
We first consider the boundary conditions used in the case of calculation using the
 $k-\varepsilon$ model. Indeed these cases are the most complex and general.

The boundary conditions are necessary to prescribe at the boundary
 the correct tangential stress $\sigma_\tau=\rho_Iu^*u_k$ in the momentum
equation\footnote{Proposition de modification des conditions aux limites de
paroi turbulente pour le Solveur Commun dans le cadre du mod\`ele
$k-\varepsilon$ standard, rapport EDF HI-81/00/019/A, 2000, M. Boucker, J.-D. Mattei.}
($\rho_I$ is the density at the centre of cell $I$).
The term which requires boundary conditions is the one containing the
velocity derivative in the normal direction to the wall\footnote{The transpose gradient term is treated in \fort{visecv}
and thus will not be considered here.}~:
$(\mu_I+\mu_{t,I})\ggrad{\vect{u}}\,\vect{n}$. It appears on the
right-hand side of the usual momentum equation (see \fort{bilsc2} and \fort{predvv}).

In the case where the  $k-\varepsilon$  model tends to surestimate
the production of turbulent kinetic energy, the length scale of the model,
$L_{k-\varepsilon}$,
can become significantly larger than the maximum theoretical length scale
of the turbulent boundary layer eddies  $L_{\text{theo}}$. We write~:
\begin{equation}
\left\{\begin{array}{l}
L_{k-\varepsilon} = C_{\mu}\displaystyle\frac{k^\frac{3}{2}}{\varepsilon}\\
L_{\text{theo}} = \kappa\, I'F
\end{array}\right.
\end{equation}

In the case where $L_{k-\varepsilon}>L_{\text{th\'eo}}$, we thus have
$\mu_{t,I}>\mu_{t}^{lm}$ with $\mu_{t,I}$ the turbulent viscosity of the
$k-\varepsilon$ model at point $I$ and $\mu_{t}^{lm}=\rho_I L_{\text{theo}}u_k$
the turbulent viscosity of the mixing length model. Additionally, the
tangential stress can write by making the turbulent viscosity appear~:
\begin{equation}
\sigma_\tau = \rho_Iu^*u_k = \displaystyle\frac{u^*}{\kappa\, I'F}\underbrace{\rho_I\kappa\, I'F\, u_k}_{\mu^{lm}_t}
\end{equation}
The viscosity scale introduced in the stress thus contradicts the one deduced
from the neighbouring turbulence calculated by the model.
Consequently we prefer to write the stress, by using the velocity scale of the $k-\varepsilon$ model when it is lower than
the limit $L_{\text{th\'eo}}$~:
\begin{equation}
\sigma_\tau = \displaystyle\frac{u^*}{\kappa\, I'F} max(\mu_{t}^{lm},\mu_{t,I})
\end{equation}

One can then use this value to calculate the diffusive flux
which depends upon it in the Navier-Stokes equations~:
\begin{equation}\label{Base_Clptur_eq_grad_sigma_clptur}
(\mu_I+\mu_{t,I})\ggrad{\vect{u}}\,\vect{n}=-\sigma_\tau \vect{\tau}.
\end{equation}

But the velocity gradient (face gradient) is computed in the code as~:
\begin{equation}\label{Base_Clptur_eq_grad_uf_clptur}
(\mu_I+\mu_{t,I})\ggrad{\vect{u}}\,\vect{n}=
\displaystyle\frac{(\mu_I+\mu_{t,I})}{\overline{I'F}}(\vect{u}_F-\vect{u}_{I'})
\end{equation}

Using (\ref{Base_Clptur_eq_grad_sigma_clptur}) and
(\ref{Base_Clptur_eq_grad_uf_clptur}) we obtain the value of $\vect{u}_F$
to be prescribed, referred to as $\vect{u}_{F,flux}$~(conservation of the momentum flux)~:
\begin{equation}\label{Base_Clptur_eq_uf_flux_clptur}
\begin{array}{ll}
\vect{u}_{F,flux}&=\vect{u}_{I'}-\displaystyle\frac{\overline{I'F}}{\mu_I+\mu_{t,I}}\sigma_\tau \vect{\tau}\\
                 &=\vect{u}_{I'}-\displaystyle\frac{u^*}{\kappa\, (\mu_I+\mu_{t,I})} max(\mu_{t}^{lm},\mu_{t,I}) \vect{\tau}
\end{array}
\end{equation}

In reality, an extra approximation is made. It consists in imposing a zero
 normal velocity at the wall and in using equation (\ref{Base_Clptur_eq_uf_flux_clptur})
 projected on the plane parallel to the wall~:
\begin{equation}
\vect{u}_{F,flux}=\left[u_{\tau,I'}-\displaystyle\frac{u^*}{\kappa\,
(\mu_I+\mu_{t,I})} max(\mu_{t}^{lm},\mu_{t,I}) \right]\vect{\tau}
\end{equation}

Moreover, if the value obtained for $y^+$ is
lower than  $y^+_{lim}$ a no-slip condition is applied.
Finally, one can also make the wall velocity appear in the final expression~:
\begin{equation}
\begin{array}{l}
\text{\bf "Flux" boundary conditions of the velocity}\,(k-\varepsilon)\\
\left\{\begin{array}{llr}
\vect{u}_{F,flux}&=\vect{v}_p& \text{ if\ }  y^+\leqslant
                           y^+_{lim} \\
\vect{u}_{F,flux}&=\vect{v}_p+&\left[u^r_{\tau,I'}-\displaystyle\frac{u^*}{\kappa\,
(\mu_I+\mu_{t,I})} max(\mu_{t}^{lm},\mu_{t,I}) \right]\vect{\tau}
\text{ otherwise }
\end{array}\right.
\end{array}
\end{equation}

A first pair of coefficients $A_{flux}$ and $B_{flux}$ can then be deduced
(for each component of the velocity separately) and it is used only
 to compute the tangential stress dependent term  (see \fort{bilsc2})~:
\begin{equation}
\begin{array}{l}
\text{\bf Coefficients associated with  the "flux" boundary conditions of the velocity } (k-\varepsilon)\\
\left\{\begin{array}{l}
\left\{\begin{array}{llr}
\vect{A}_{flux}&=\vect{v}_p& \text{ if\ } y^+\leqslant y^+_{lim} \\
\vect{A}_{flux}&=\vect{v}_p+&\left[u^r_{\tau,I'}-\displaystyle\frac{u^*}{\kappa\,
(\mu_I+\mu_{t,I})} max(\mu_{t}^{lm},\mu_{t,I}) \right]\vect{\tau} \text{ otherwise }
\end{array}\right.  \\
\vect{B}_{flux} = \vect{0}
\end{array}\right.
\end{array}
\end{equation}

We saw above how to impose a boundary condition to compute directly the stress term.
Further analysis is necessary to calculate correctly the velocity gradients. We
want to find a boundary face value which permits to obtain,
with the chosen expression for the gradient,
 the value the turbulent production as close as possible to its theoretical value
(determined by using the logarithmic law), in order to evaluate the normal
derivative the tangential velocity.
Thus, we define (at point $I$)~:
\begin{equation}\label{Base_Clptur_eq_ptheo_clptur}
P_{\text{th\'eo}} = \rho_I u^* u_k
\|\displaystyle\frac{\partial u_\tau}{\partial\vect{n}}\|_{I} =
\rho_I \displaystyle\frac{u_k(u^*)^2}{\kappa\, I'F}
\end{equation}

Morevoer, the dominant term of the production computed in cell $I$ is,
in classical situations ($y$ is the coordinate on the axis
whose direction vector is $\vect{\tilde{n}}$),
\begin{equation}
P_{\text{calc}} =
\mu_{t,I}\left(\displaystyle\frac{\partial u_\tau}{\partial y}\right)^2_{I}
\end{equation}

The normal gradient of the tangential velocity (cell gradient)
is calculated in the code using finite volume, and its expression
on regular orthogonal meshes is (see the notations on
figure \ref{Base_Clptur_fig_bord_ortho_clptur})~:
\begin{equation}
P_{\text{calc}} =
\mu_{t,I}\left(\displaystyle\frac{u_{\tau,G}-u_{\tau,F}}{2d}\right)^2 =
\mu_{t,I}\left(\displaystyle\frac{u_{\tau,I}+u_{\tau,J}-2u_{\tau,F}}{4d}\right)^2
\end{equation}
We then assume that $u_{\tau,J}$ can be obtained from $u_{\tau,I}$
and from the normal gradient of $u_{\tau}$ calculated in G
from the logarithmic law~:
\begin{equation}
\label{Base_Clptur_eq_dvp_lim_utau}
u_{\tau,J}=u_{\tau,I}+ IJ\,.\,(\partial_y u_{\tau})_G+\mathcal{O} (IJ^{\,2}) \approx
u_{\tau,I}+ IJ\,.\,\left[\partial_y \left(\displaystyle
\frac{u^*}{\kappa}\,ln{ (y^+)} + 5,2 \right)\right]_G=
u_{\tau,I}+2d\displaystyle\frac{u^*}{\kappa\, 2d}
\end{equation}
and thus we obtain~:
\begin{equation}\label{Base_Clptur_eq_pcalc_clptur}
\begin{array}{lll}
P_{\text{calc}} &=&
\mu_{t,I}\left(\displaystyle\frac{u_{\tau,I}+u_{\tau,I}+2d\frac{u^*}{\kappa\, 2d}-2u_{\tau,F}}{4d}\right)^2 \\
&=&\mu_{t,I}\left(\displaystyle\frac{2u_{\tau,I}+2\frac{u^*}{2\kappa}-2u_{\tau,F}}{4d}\right)^2 =
\mu_{t,I}\left(\displaystyle\frac{u_{\tau,I}+\frac{u^*}{2\kappa}-u_{\tau,F}}{2d}\right)^2
\end{array}
\end{equation}

\begin{figure}[h]
\centerline{\includegraphics[height=7cm]{bordortho}}
\caption{\label{Base_Clptur_fig_bord_ortho_clptur}Cellule de bord - Maillage orthogonal.}
\end{figure}

We then use (\ref{Base_Clptur_eq_ptheo_clptur}) and
(\ref{Base_Clptur_eq_pcalc_clptur}) to impose that the calculated
production is equal to the theoretical production. The preceeding
formulae are extended with no precaution to non-orthogonal meshes (the velocity
at $I$ is then simply computed at $I'$).
The following expression for $u_{\tau,F}$ is then obtained~:
\begin{equation}
u_{\tau,F,grad} =u_{\tau,I'}-\displaystyle\frac{u^*}{\kappa}\left(
2\sqrt{\displaystyle\frac{\rho_I\kappa\, u_k I'F}{\mu_{t,I}} }-\displaystyle\frac{1}{2}\right)
\end{equation}

Additionally, we force the gradient to remain as stiff as the one
given by the normal derivative of the theoretical velocity profile
(logarithmic) at $I'$~:\\
$\partial_y u_{\tau} = \partial_y (\displaystyle
\frac{u^*}{\kappa}\,ln{ (y^+)} + 5,2 ) =\displaystyle\frac{u^*}{\kappa\, \overline{I'F}}$, thus~:
\begin{equation}
u_{\tau,F,grad} =u_{\tau,I'}-\displaystyle\frac{u^*}{\kappa}max\left(1,
2\sqrt{\displaystyle\frac{\rho_I\kappa\, u_k I'F}{\mu_{t,I}} }-\displaystyle\frac{1}{2}\right)
\end{equation}

Finally, we clip the velocity at the wall with a minimum value calculated
assuming that we are in the logarithmic layer~:
\begin{equation}\label{Base_Clptur_eq_ugrad_clptur}
u_{\tau,F,grad} =
max\left(u^*\left(\displaystyle\frac{1}{\kappa}ln(y^+_{lim})+5,2\right),
u_{\tau,I'}-\displaystyle\frac{u^*}{\kappa}\left[max\left(1,
2\sqrt{\displaystyle\frac{\rho_I\kappa\, u_k I'F}{\mu_{t,I}}
}-\displaystyle\frac{1}{2}\right)\right]\right)
\end{equation}


The normal derivative at the wall is prescribed to zero.
If the $y^+$ value at the wall is lower than  $y^+_{lim}$,
a no-slip condition is prescribed. Finally, one can also
make explicit the velocity of the wall in the final expression~:
\begin{equation}
\begin{array}{l}
\text{\bf "Gradient" boundary conditions of the velocity} (k-\varepsilon)\\
\left\{\begin{array}{l}
\vect{u}_{F,grad}=\vect{v}_p
          \qquad\qquad\text{ if }  y^+\leqslant y^+_{lim} \\
\vect{u}_{F,grad}=\vect{v}_p+\\
          \left\{
max\left(u^*\left(\displaystyle\frac{1}{\kappa}ln(y^+_{lim})+5,2\right),
u^r_{\tau,I'}-\displaystyle\frac{u^*}{\kappa}\left[max\left(1,
2\sqrt{\displaystyle\frac{\rho_I\kappa\, u_k I'F}{\mu_{t,I}}
}-\displaystyle\frac{1}{2}\right)\right]\right)
\right\}\vect{\tau}
          \text{ otherwise }
\end{array}\right.
\end{array}
\end{equation}

A second pair of coefficients $A_{grad}$ and $B_{grad}$ can then be deduced
(for each velocity component separately). It is used when the velocity
gradient is necessary (except for the terms depending on the tangential shear,
those being treated in \fort{bilsc2} using $A_{flux}$ and $B_{flux}$)~:
\begin{equation}
\begin{array}{l}
\text{\bf Coefficients associated to the "gradient" boundary conditions of }\\
\qquad\qquad\qquad\qquad\text{\bf the velocity}  (k-\varepsilon)\\
\left\{\begin{array}{l}
\left\{\begin{array}{l}
\vect{A}_{grad}=\vect{v}_p
                    \qquad\qquad\text{ \ if\ } y^+\leqslant y^+_{lim} \\
\vect{A}_{grad}=\vect{v}_p+\\
\left\{
max\left(u^*\left(\displaystyle\frac{1}{\kappa}ln(y^+_{lim})+5,2\right),
u^r_{\tau,I'}-\displaystyle\frac{u^*}{\kappa}\left[max\left(1,
2\sqrt{\displaystyle\frac{\rho_I\kappa\, u_k I'F}{\mu_{t,I}}
}-\displaystyle\frac{1}{2}\right)\right]\right)
\right\}\vect{\tau}
    \text{ otherwise }
\end{array}\right.  \\
\vect{B}_{grad} = \vect{0}
\end{array}\right.
\end{array}
\end{equation}


\etape{Boundary conditions of the velocity in $R_{ij}-\varepsilon$\vspace{0,3cm}}
%%%%%%%%%%%%%%%%%%%%%%%%%%%%%%%%%%%%%%%%%%%%%%%%%%%%%%%%%%%%%%%%%%%%%%%%%%%%%%%
The boundary conditions of the velocity with the  $R_{ij}-\varepsilon$ model are
more simple, since there are only of one type.
Keeping the same notations as above, we want the tangential velocity gradient
(calculated at $I$, and to be used to evaluate the turbulent production)
to be consistent with the logarithmic law giving the ideal tangential
velocity profile. The theoretical gradient is~:

\begin{equation}\label{Base_Clptur_eq_grad_theo_clptur}
G_{\text{theo}} = \left(\displaystyle\frac{\partial u_\tau}{\partial y}\right)_{I'}=\frac{u^*}{\kappa\, I'F}
\end{equation}

The normal gradient of the tangential velocity (cell gradient) is
calculated in the code using finite volumes, and its expression in the case
of regular orthogonal meshes is (see notations in figure
\ref{Base_Clptur_fig_bord_ortho_clptur})~:
\begin{equation}
G_{\text{calc}}=\displaystyle\frac{u_{\tau,G}-u_{\tau,F}}{2d} =
\displaystyle\frac{u_{\tau,I}+u_{\tau,J}-2u_{\tau,F}}{4d}
\end{equation}
We then assume that $u_{\tau,J}$ can be obtained from $u_{\tau,I}$
and from the normal gradient of $u_{\tau}$ calculated in G
from the logarithmic law (see equation (\ref{Base_Clptur_eq_dvp_lim_utau}))
$u_{\tau,J}=u_{\tau,I}+2d\displaystyle\frac{u^*}{\kappa\, 2d}$ and we thus
obtain~:
\begin{equation}\label{Base_Clptur_eq_grad_calc_clptur}
G_{\text{calc}}=\displaystyle\frac{u_{\tau,I}+u_{\tau,I}+2d\displaystyle\frac{u^*}{\kappa\, 2d}-2u_{\tau,F}}{4d}=
\displaystyle\frac{2u_{\tau,I}+2\displaystyle\frac{u^*}{2\kappa}-2u_{\tau,F}}{4d}=
\displaystyle\frac{u_{\tau,I}+\displaystyle\frac{u^*}{2\kappa}-u_{\tau,F}}{2d}
\end{equation}
We then use the equations (\ref{Base_Clptur_eq_grad_theo_clptur}) and
(\ref{Base_Clptur_eq_grad_calc_clptur}) to derive an expression for
$u_{\tau,F}$ (the preceeding
formulae are extended with no precaution to the case non-orthogonal meshes,
the velocity at $I$ being simply computed at $I'$)~:
\begin{equation}
u_{\tau,F}= u_{\tau,I'}-\displaystyle\frac{3u^*}{2\kappa }
\end{equation}
The normal derivative at the wall is prescribed to zero.
If the value obtained for $y^+$ at the wall is lower than  $y^+_{lim}$,
a no-slip condition is prescribed. Finally, one can also
make explicit the velocity of the wall in the final expression~:
\begin{equation}\label{Base_Clptur_eq_CL_vitesse_rij_clptur}
\begin{array}{l}
\text{\bf Boundary conditions of the velocity }(R_{ij}-\varepsilon)\\
\left\{\begin{array}{lll}
\vect{u}_{F}&=\vect{v}_p& \text{ \ if\ } y^+\leqslant
                           y^+_{lim} \\
\vect{u}_{F}&=\left[u^r_{\tau,I'}-\displaystyle\frac{3u^*}{2\kappa } \right]\vect{\tau} +\vect{v}_p &\text{ otherwise }
\end{array}\right.
\end{array}
\end{equation}


Un couple de coefficients $A$ et $B$ s'en d\'eduit (pour
chaque composante de vitesse s\'epar\'ement)~:
\begin{equation}\label{Base_Clptur_eq_AB_vitesse_rij_clptur}
\begin{array}{l}
\text{\bf Coefficients associ\'es aux conditions aux limites sur la vitesse }(R_{ij}-\varepsilon)\\
\left\{\begin{array}{l}
\left\{\begin{array}{lll}
\vect{A}&=\vect{v}_p& \text{ \ si\ } y^+\leqslant y^+_{lim} \\
\vect{A}&=\left[u^r_{\tau,I'}-\displaystyle\frac{3u^*}{2\kappa } \right]\vect{\tau}
+\vect{v}_p &\text{ sinon }
\end{array}\right.\\
\vect{B}= \vect{0}
\end{array}\right.
\end{array}
\end{equation}
A pair of coefficients $A_{grad}$ and $B_{grad}$ can be deduced
from the above equation
(for each velocity component separately).


\etape{Boundary conditions of the velocity in laminar\vspace{0,3cm}}
When no turbulence model is activated, we implicitly use a one velocity
scale model (there is no turbulent variables to compute  $u_k$), and the same
conditions
\footnote{In other words; the boundary conditions are given by
 (\ref{Base_Clptur_eq_CL_vitesse_rij_clptur}) and
(\ref{Base_Clptur_eq_AB_vitesse_rij_clptur}).}
 as in $R_{ij}-\varepsilon$ are used~: the model degenerates automatically.



\newpage

\etape{Boundary conditions for $k$ and $\varepsilon$ (standard
$k-\varepsilon$ model)\vspace{0,3cm}}

We impose $k$ with a Dirichlet condition using the friction velocity
$u_k$ (see equation~(\ref{Base_Clptur_Eq_Mod_'2ech_Vit})) :
\begin{equation}
k= \displaystyle\frac{u_k^2}{C_\mu^\frac{1}{2}}
\end{equation}

We want to impose the normal derivative of  $\varepsilon$ from
of the following theoretical law
 (see the notations in figure \ref{Base_Clptur_fig_bord_ortho_clptur})~:
\begin{equation}\label{Base_Clptur_eq_partialep_theo_clptur}
G_{\text{theo},\varepsilon} = \displaystyle\frac{\partial \left(u_k^3/(\kappa\, y)\right)}{\partial y}
\end{equation}



We use point $M$ to impose a boundary condition with a higher order of
accuracy in space.
 Indeed, using the simple expression
$\varepsilon_F=\varepsilon_I+d\partial_y\varepsilon_I + O(d^2)$ leads to
first order accuracy.
 A second order accuracy can be reached
 using the following Taylor series expansion:
\begin{equation}
\left\{\begin{array}{ll}
\varepsilon_M&=\varepsilon_I-\displaystyle\frac{d}{2}\partial_y\varepsilon_I+\displaystyle\frac{d^2}{8}\partial^2_y\varepsilon_I+O(d^3)\\
\varepsilon_M&=\varepsilon_F+\displaystyle\frac{d}{2}\partial_y\varepsilon_F+\displaystyle\frac{d^2}{8}\partial^2_y\varepsilon_F+O(d^3)
\end{array}\right.
\end{equation}
By substracting these twxo expression, we obtain
\begin{equation}\label{Base_Clptur_eq_epsf_clptur}
\varepsilon_F=\varepsilon_I-\displaystyle\frac{d}{2}(\partial_y\varepsilon_I+\partial_y\varepsilon_F)+O(d^3)
\end{equation}
Additionally, we have
\begin{equation}
\left\{\begin{array}{ll}
\partial_y\varepsilon_I&=\partial_y\varepsilon_M+d\partial^2_y\varepsilon_M+O(d^2)\\
\partial_y\varepsilon_F&=\partial_y\varepsilon_M-d\partial^2_y\varepsilon_M+O(d^2)
\end{array}\right.
\end{equation}
The sum of these last two expressions gives
$\partial_y\varepsilon_I+\partial_y\varepsilon_F=2\partial_y\varepsilon_M+O(d^2)$ and,
using equation
(\ref{Base_Clptur_eq_epsf_clptur}), we finally obtain a second order
approximation for $\varepsilon_F$~:
\begin{equation}
\varepsilon_F=\varepsilon_I-d\partial_y\varepsilon_M+O(d^3)
\end{equation}
The theoretical value (see equation \ref{Base_Clptur_eq_partialep_theo_clptur})
is then used in order to evaluate
 $\partial_y\varepsilon_M$ and thus the value to prescribe at
the boundary is obtained ($d=I'F$)~:
\begin{equation}
\varepsilon_F=\varepsilon_I+d\displaystyle\frac{ u_k^3}{\kappa\, (d/2)^2}
\end{equation}

This expression is extended to non-orthogonal mesh without any precautions
(which is bound to deteriorate the spatial accuracy ot the method).

Additionally, the velocity $u_k$ is set to zero for $y^+\leqslant y^+_{lim}$.
Consequently, the value of $k$ and the flux of $\varepsilon$ are both zero.

Finally we have~:

\begin{equation}
\begin{array}{l}
\text{\bf Boundary conditions for } k \text { \bf and } \varepsilon \\
\left\{\begin{array}{ll}
k_F&= \displaystyle\frac{u_k^2}{C_\mu^\frac{1}{2}}\\
\varepsilon_F&=\varepsilon_{I'}+I'F\displaystyle\frac{ u_k^3}{\kappa\, (I'F/2)^2}
\end{array}\right. \\
\text{with } u_k = 0 \text { if } y^+\leqslant y^+_{lim}
\end{array}
\end{equation}
and the associated pair of coefficients
\begin{equation}
\begin{array}{l}
\text{\bf Coefficients associated to the boundary conditions of }
k \text { \bf et } \varepsilon \\
\left\{\begin{array}{llll}
A_k&= \displaystyle\frac{u_k^2}{C_\mu^\frac{1}{2}} &\text{ and } B_k&= 0 \\
A_\varepsilon&=I'F\displaystyle\frac{ u_k^3}{\kappa\, (I'F/2)^2}&\text{ and } B_\varepsilon&= 1
\end{array}\right.\\
\text{with } u_k = 0 \text { if } y^+\leqslant y^+_{lim}
\end{array}
\end{equation}







\etape{Boundary conditions for $R_{ij}$ and $\varepsilon$
(standard $R_{ij}-\varepsilon$ model)\vspace{0,3cm}}

The boundary conditions for the Reynolds stresses in the coordinate system
attached to the wall write ($\hat R$ refers to the local coordinate system)~:
\begin{equation}
\begin{array}{lll}
\partial_{\tilde{n}} \hat R_{\tau\tau} = \partial_{\tilde{n}} \hat R_{\tilde{n}\tilde{n}}=\partial_{\tilde{n}} \hat R_{bb}=0  &
\text { et } \hat R_{\tau\tilde{n}} = -u^*u_k  &\text { et  } \hat R_{\tau b} = \hat R_{\tilde{n} b}
= 0
\end{array}
\end{equation}


Additionally, if the value obtained for $y^+$ is lower
than  $y^+_{lim}$, all Reynolds stresses are set to zero
(we assume that the turbulent stresses are negligible
compared to the viscous stresses).

Although it is done systematically in the code,
expressing the above boundary conditions in the computation coordinate
system is relatively complex (rotation of a tensor):~
the reader is referred to the documentation of \fort{clsyvt}
where more details are provided. In what follows,
the boundary conditions will only be presented in the local
coordinate system.

Thus we want to impose the boundary values~:
\begin{equation}
\begin{array}{l}
\text{\bf Boundary conditions of } R_{ij} \\
\left\{\begin{array}{lll}
\text{ if }y^+\leqslant y^+_{lim}&\hat R_{\alpha\alpha,F} = \hat R_{\alpha\beta,F} = 0 \\
\text { otherwise }                         &\left\{\begin{array}{l}
\hat R_{\alpha\alpha,F} = \hat R_{\alpha\alpha,I'}  \text{ with }\alpha \in \{\tau,\tilde{n},b\}\text{ (without summation)}\\
\hat R_{\tau\tilde{n}} = -u^*u_k  \text { and } \hat R_{\tau b} = \hat
R_{\tilde{n} b} =0
\end{array}\right.
\end{array}\right.
\end{array}
\end{equation}

For the dissipation, the boundary condition applied is identical to the
one applied with the $k-\varepsilon$ model~:
\begin{equation}
\begin{array}{l}
\text{\bf Boundary conditions of } \varepsilon  \text{ }(R_{ij}-\varepsilon)  \\
\left\{
\begin{array}{l}
\varepsilon_F=\varepsilon_{I'}+I'F\displaystyle\frac{ u_k^3}{\kappa\, (I'F/2)^2}\\
\text{with } u_k = 0 \text { if } y^+\leqslant y^+_{lim}
\end{array}\right.
\end{array}
\end{equation}


These boundary conditions can be imposed explicitly (by default,  ICLPTR=0)
or (semi-)implicitly ( ICLPTR=1). The standard option (explicit) leads to
the following values\footnote{It can be noticed
that the value of $\varepsilon$ is not reconstructed at $I'$. We thus wish
to improve the ''stability'' since $\varepsilon$ has a very steep gradient
at the wall ($\varepsilon \approx \displaystyle\frac{1}{y}$), and thus
only weak recontruction errors at $I'$ could lead to important
deterioration of the results. However, it would be necessary to
check if stability is altered with the gradient reconstruction of
\fort{gradrc}.}
of the coefficients  $A$ and $B$~:


\begin{equation}
\begin{array}{l}
\text{\bf Coefficients associated to the explicit boundary conditions of }
R_{ij} \text{\bf et } \varepsilon \\
\left\{\begin{array}{l}
\begin{array}{l}
\text{If }y^+\leqslant y^+_{lim}\text{~:}\\
\qquad\begin{array}{lll}
      A_{\hat R_{\alpha\alpha}} = A_{\hat R_{\alpha\beta}} = 0  &\text{ and } B_{\hat R_{\alpha\alpha}} =B_{\hat R_{\alpha\beta}}= 0 &
      \end{array}\\
\text{Otherwise~:}\\
\qquad\left\{\begin{array}{lll}
      A_{\hat R_{\alpha\alpha}} = (R_{\alpha\alpha})_I  &\text{ and } B_{\hat R_{\alpha\alpha}} = 0
         &\text{ with }\alpha \in \{\tau,\tilde{n},b\}\text{ (without summation)}\\
      A_{\hat R_{\tau \tilde{n}}} = -u^*u_k   &\text{ and } B_{\hat R_{\tau \tilde{n}}} = 0 &\\
      A_{\hat R_{\tau  b}} = A_{\hat R_{\tilde{n} b}} = 0   &\text{ and } B_{\hat R_{\tau b}} =B_{\hat R_{\tilde{n} b}} = 0 &\\
      \end{array}\right.
\end{array}\\
\text{And for all cases~:}\\
\qquad A_\varepsilon=\varepsilon_{I}+I'F\displaystyle\frac{ u_k^3}{\kappa\, (I'F/2)^2} \text{ and } B_\varepsilon= 0
\end{array}\right.\\
\text{with } u_k = 0 \text { if } y^+\leqslant y^+_{lim}
\end{array}
\end{equation}

The semi-implicit option leads to the following values for the
coefficients $A$ and $B$. They differ from the preceeding ones, only
as regards as the diagonal Reynolds stresses and dissipation.
In the general case, impliciting of some components of the tensor
in the local coordinate system leads to partially impliciting
all the components in the global computation coordinate system~:
\begin{equation}
\begin{array}{l}
\text{\bf Coefficients associated to the semi-implicit boundary conditions of}\\
\qquad\qquad\qquad\qquad\text {\bf sur les variables } R_{ij} \text{\bf et } \varepsilon \\
\left\{\begin{array}{l}
\begin{array}{l}
\text{If }y^+\leqslant y^+_{lim}\text{~:}\\
\qquad\begin{array}{lll}
      A_{\hat R_{\alpha\alpha}} = A_{\hat R_{\alpha\beta}} = 0  &\text{ and } B_{\hat R_{\alpha\alpha}} =B_{\hat R_{\alpha\beta}}= 0 &
      \end{array}\\
\text{Sinon~:}\\
\qquad\left\{\begin{array}{lll}
      A_{\hat R_{\alpha\alpha}} = 0  &\text{ and } B_{\hat R_{\alpha\alpha}} = 1 &\text{ with }\alpha \in \{\tau,\tilde{n},b\}\text{ (without summation)}\\
      A_{\hat R_{\tau \tilde{n}}} = -u^*u_k   &\text{ and } B_{\hat R_{\tau \tilde{n}}} = 0 &\\
      A_{\hat R_{\tau  b}} = A_{\hat R_{\tilde{n} b}} = 0   &\text{ and } B_{\hat R_{\tau b}} =B_{\hat R_{\tilde{n} b}} = 0 &\\
      \end{array}\right.
\end{array}\\
\text{And for all cases~:}\\
\qquad A_\varepsilon=I'F\displaystyle\frac{ u_k^3}{\kappa\, (I'F/2)^2}\text{ and } B_\varepsilon= 1
\end{array}\right.\\
\text{with } u_k = 0 \text { if } y^+\leqslant y^+_{lim}
\end{array}
\end{equation}




\newpage
\etape{Boundary conditions of the {\it VarScalaires}\vspace{0,3cm}}
Only the boundary conditions when a boundary value is imposed
(at the wall or away from the wall with a possible external exchange coefficient)
are treated here.
The reader is referred to the notations in figure
\ref{Base_Clptur_fig_flux_clptur} and to the general presentation provided in
\fort{condli} (in what follows only the most essential part of the
presentation is repeated).

The conservation of the normal flux at the boundary for variable $f$ writes~:
\begin{equation}\label{Base_Clptur_eq_flux_clptur}
\begin{array}{l}
    \underbrace{h_{int}(f_{b,int}-f_{I'})}_{\phi_{int}}
  = \underbrace{h_{b}(f_{b,ext}-f_{I'})}_{\phi_{b}}
  = \left\{\begin{array}{ll}
    \underbrace{h_{imp,ext}(f_{imp,ext}-f_{b,ext})}_{\phi_{\text{\it real
imposed}}} &\text{(Dirichlet condition)}\\
    \underbrace{\phi_{\text{\it imp,ext}}}_{\phi_{\text{\it real imposed}}}
            &\text{(Neumann condition)}
           \end{array}\right.
\end{array}
\end{equation}

The above two equation are rearranged in order to obtain the value of the
numerical flux $f_{b,int}=f_{F}$ to impose at the wall boundary face,
according to the values of $f_{imp,ext}$ and $h_{imp,ext}$ set by the user,
and to the value of  $h_{b}$ set by the similarity laws detailed hereafter.
The coefficients $A$ and $B$  can then be readily derived, and are presented here.

\begin{equation}\label{Base_Clptur_eq_fbint_clptur}
\begin{array}{l}
\text{\bf Boundary conditions of the {\it VarScalaires} }\\
f_{b,int} =
\underbrace{\displaystyle\frac{h_{imp,ext}}{h_{int}+h_r h_{imp,ext} } f_{imp,ext}}_{A} +
\underbrace{\displaystyle\frac{h_{int}+h_{imp,ext}(h_r-1)}{h_{int}+h_r h_{imp,ext} }}_{B} f_{I'}
\text{  with } h_r=\displaystyle\frac{h_{int}}{h_{b}}
\end{array}
\end{equation}


\newpage
{\bf Similarity principle~: calculation of } $h_b$.

The only remaining unknown in expression (\ref{Base_Clptur_eq_fbint_clptur})
is the value of $h_{b}$, since  $h_{int}$ has a numerical value which
is coherent with the face gradient computation options detailed in
 \fort{condli} ($h_{int}=\displaystyle\frac{\alpha}{\overline{I'F}}$).
The value of  $h_{b}$ must relate the flux to the values
$f_{I'}$ and $f_{b,ext}$ by taking into account the boundary layer
(the profile of $f$ is not always linear)~:
\begin{equation}
\phi_b=h_b\,(f_{b,ext}-f_{I'})
\end{equation}


The following considerations are presented using the general notations.
In particular, the Prandtl-Schmidt number writes
$\sigma=\displaystyle\frac{\nu\,\rho\,C}{\alpha}$.
When the considered scalar $f$ is the temperature,
we have (see \fort{condli})
\begin{list}{$\bullet$}{}
\item $C=C_p$ (specific heat capacity),
\item $\alpha=\lambda$ (molecular conductivity),
\item $\sigma = \displaystyle\frac{\nu\,\rho\,C_p}{\lambda} = Pr$
       (Prandtl number),
\item $\sigma_t = Pr_t$ (turbulent Prandtl number),
\item $\phi=\left(\lambda+\displaystyle\frac{C_p\mu_t}{\sigma_t}\right)
        \displaystyle\frac{\partial T}{\partial y}$ (flux in $Wm^{-2}$).
\end{list}

The reference "Convection Heat Transfer",
Vedat S. Arpaci and Poul S. Larsen, Prentice-Hall, Inc was used.

The flux at the wall writes for the scalar $f$ (the flux is positive
if it enters the fluid domain, as shown by the orientation of the
$y$ axis)~:
\begin{equation}\label{Base_Clptur_Eq_Flux_scalaire}
\phi = -\left(\alpha+C\,\frac{\mu_t}{\sigma_t}\right)
                  \frac{\partial f}{\partial y}
     = -\rho\,C \left(\displaystyle\frac{\alpha}{\rho\,C}+
                                \frac{\mu_t}{\rho\sigma_t}\right)
                  \frac{\partial f}{\partial y}
\end{equation}

Similarly for the temperature, with
 $a=\displaystyle\frac{\lambda}{\rho\,C_p}$ and
$a_t=\displaystyle\frac{\mu_t}{\rho\,\sigma_t}$,
we have~:
\begin{equation}
\phi = -\rho\,C_p(a+a_t)\frac{\partial T}{\partial y}
\end{equation}

In order to make $f$ dimensionless, we introduce $f^*$ defined using
the flux at the boundary  $\phi_b$~:
\begin{equation}
f^* = -\displaystyle\frac{\phi_b}{\rho\,C\,u_k}
\end{equation}
For the temperature, we thus have~:
\begin{equation}
T^* = -\displaystyle\frac{\phi_b}{\rho\,C_p\,u_k}
\end{equation}

We then divide both sides of equation~(\ref{Base_Clptur_Eq_Flux_scalaire})
by  $\phi_b$. For the left-hand side, we simplify using the conservation
of the flux (and thus the fact that $\phi=\phi_b$). For the right-hand
side, we replace $\phi_b$ by its value $-\rho\,C\,u_k\,f^*$.
With the notations~:
\begin{equation}
       \nu=\displaystyle\frac{\mu}{\rho}
\qquad \nu_t=\displaystyle\frac{\mu_t}{\rho}
\qquad y^+=\displaystyle\frac{y\,u_k}{\nu}
\qquad f^+=\displaystyle\frac{f-f_{b,ext}}{f^*}
\end{equation}
we have~:
\begin{equation}\label{Base_Clptur_Eq_Flux_scalaire_adim}
1 =  \left(\displaystyle\frac{1}{\sigma}+
              \displaystyle\frac{1}{\sigma_t}\frac{\nu_t}{\nu}\right)
                  \displaystyle\frac{\partial f^+}{\partial y^+}
\end{equation}

One can remark at this stage that with the notations used in the
preceeding,  $h_b$ can be expressed as a function of $f^+_{I'}$~:
\begin{equation}
h_b=\displaystyle\frac{\phi_b}{f_{b,ext}-f_{I'}}=\frac{\rho\,C\,u_k}{f^+_{I'}}
\end{equation}

In order to compute $h_b$, we integrate
equation ~(\ref{Base_Clptur_Eq_Flux_scalaire_adim}) to obtain
$f^+_{I'}$.
The only difficulty then consists in prescribing a variation law
$\mathcal{K}=\displaystyle\frac{1}{\sigma}+
              \displaystyle\frac{1}{\sigma_t}\frac{\nu_t}{\nu}$.


In the fully developed turbulent region
(far enough from the wall, for $y^+\geqslant y^+_2$),
a mixing length hypothesis models the variations of
$\nu_t$~:
\begin{equation}
\nu_t = l^2 \arrowvert \frac{\partial U}{\partial y} \arrowvert =
\kappa \,y\,u^*
\end{equation}
Additionally, the molecular diffusion of $f$
(or the conduction when $f$ represents the temperature)
is negligible compared to its turbulent diffusion~: therefore
we neglect
$\displaystyle\frac{1}{\sigma}$ compared to
$\displaystyle\frac{1}{\sigma_t}\frac{\nu_t}{\nu}$.
Finally we have
\footnote{We make the approximation that the definitions of $y^+$
from $u^*$ and $u_k$ are equivalent.}~:
\begin{equation}
\mathcal{K}= \displaystyle\frac{\kappa \,y^+}{\sigma_t}
\end{equation}

On the contrary, in the near-wall region (for $y^+ < y^+_1$)
the turbulent contribution becomes negligible
compared to the molecular contribution and we thus neglect
$\displaystyle\frac{1}{\sigma_t}\frac{\nu_t}{\nu}$ compared to
$\displaystyle\frac{1}{\sigma}$.

It would be possible to restrict ourselves to these
two regions, but Arpaci and Larsen suggest the model
can be improved by introducing an intermediate
region ($y^+_1 \leqslant y^+ < y^+_2$)
in which the following hypothesis is made~:
\begin{equation}
\frac{\nu_t}{\nu} = a_1 (y^+)^3
\end{equation}
where $a_1$ is a constant whose value is obtained from
experimental correlations~:
\begin{equation}
a_1 =\displaystyle\frac{\sigma_t}{1000}
\end{equation}

Thus the following model is used for $\mathcal{K}$
(see a sketch
in figure~\ref{Base_Clptur_Fig_a_plus_at_fonction_yplus})~:
\begin{equation}
\mathcal{K}=\left\{
\begin{array}{ll}
\displaystyle\frac{1}{\sigma}
             &\text{if } y^+ < y^+_1\\
\displaystyle\frac{1}{\sigma}
+\displaystyle\frac{a_1 (y^+)^3}{\sigma_t}
             &\text{if } y^+_1 \leqslant y^+ < y^+_2\\
\displaystyle\frac{\kappa \,y^+}{\sigma_t}
             &\text{if } y^+_2\leqslant y^+\\
\end{array}
\right.
\end{equation}

\begin{figure}[htp]\label{Base_Clptur_Fig_a_plus_at_fonction_yplus}
\centerline{\includegraphics[height=8cm]{clthermique}}
\caption{$(a+a_t)/\nu$ as a function of $y^+$ obtained
                       for $\sigma=1$ and $\sigma_t=1$.}
\end{figure}


The values of $y^+_1$ and $y^+_2$ are obtained by calculating
the intersection points of the variations laws used
for $\mathcal{K}$.

The existence of an intermediate region depends upon the
values of $\sigma$.
Let's first consider the case where $\sigma$ cannot be neglected
compared to 1. In practise we consider  $\sigma > 0,1$
(this is the common case when scalar $f$ represents
the air or the water temperature in normal temperature
and pressure conditions). It is assumed that
$\displaystyle\frac{1}{\sigma}$ can be neglected compared to
$\displaystyle\frac{a_1 (y^+)^3}{\sigma_t}$ in the
intermediate region.
We thus obtain~:
\begin{equation}
  y^+_1 =\left(\displaystyle\frac{1000}{\sigma}\right)^\frac{1}{3} \qquad\qquad
  y^+_2 = \sqrt{\displaystyle\frac{1000\kappa}{\sigma_t}}
\end{equation}
The dimensionless equation~(\ref{Base_Clptur_Eq_Flux_scalaire_adim})
is integrated under the same hypothesis and we obtain the law of $f^+$~:
\begin{equation}
\left\{
\begin{array}{ll}
f^+ = \sigma \,y^+ & \text{if } y^+ < y^+_1 \\
f^+ = a_2 -\displaystyle\frac{\sigma_t}{2\,a_1\,(y^+)^2}& \text{if } y_1^+ \leqslant y^+ < y_2^+ \\
f^+ = \displaystyle\frac{\sigma_t}{\kappa}\,ln(y^+)+a_3& \text{if } y^+_2 \leqslant y^+\\
\end{array}
\right.
\end{equation}
where $a_2$ and $a_3$ are integration constants,
which have been chosen to obtain
a continuous  profile of $f^+$~:
\begin{equation}
a_2=15\sigma^{\frac{2}{3}}\qquad\qquad
a_3=15\sigma^{\frac{2}{3}}-\displaystyle\frac{\sigma_t}{2\kappa}
\left(1+
ln\left(\displaystyle\frac{1000\kappa}{\sigma_t}\right)\right)
\end{equation}

Let's now study the case where  $\sigma$ is much smaller than 1.
In practise it is assumed that $\sigma \leqslant 0,1$ (this is for
instance the case for liquid metals whose thermal conductivity is very
large, and who have Prandtl number of values of the order of 0.01).
The intermediate region then disappears and the coordinate of the
interface between the law used in the near-wall region and the one
used away from the wall is given by~:
\begin{equation}
y^+_0= \displaystyle\frac{\sigma_t}{\kappa\sigma}
\end{equation}

The dimensionless equation~(\ref{Base_Clptur_Eq_Flux_scalaire_adim})
is then integrated under the same hypothesis, and the law of
 $f^+$ is obtained~:
\begin{equation}
\left\{
\begin{array}{ll}
f^+ = \sigma \,y^+ & \text{if } y^+ \leqslant y^+_0 \\
f^+ = \displaystyle\frac{\sigma_t}{\kappa}\,
        ln\left(\displaystyle\frac{y^+}{y^+_0}\right)+\sigma \,y^+_0
                   & \text{if } y^+_0 < y^+\\
\end{array}
\right.
\end{equation}


\newpage
To summarize, the computation of $h_b$
\begin{equation}
h_b=\displaystyle\frac{\phi_b}{f_{b,ext}-f_{I'}}=\frac{\rho\,C\,u_k}{f^+_{I'}}
\end{equation}
is performed by calculating  $f^+_{I'}$ from $y^+=y^+_{I'}$
using the following laws.

If $\sigma\leqslant 0,1$, a two-layer model is used~:
\begin{equation}
\left\{
\begin{array}{ll}
f^+ = \sigma \,y^+ & \text{if } y^+ \leqslant y^+_0 \\
f^+ = \displaystyle\frac{\sigma_t}{\kappa}\,
        ln\left(\displaystyle\frac{y^+}{y^+_0}\right)+\sigma \,y^+_0
                   & \text{if } y^+_0 < y^+\\
\end{array}
\right.
\end{equation}
with
\begin{equation}
y^+_0= \displaystyle\frac{\sigma_t}{\kappa\sigma}
\end{equation}


If $\sigma > 0,1$, a three-layer model is used~:
\begin{equation}
\left\{
\begin{array}{ll}
f^+ = \sigma \,y^+ & \text{if } y^+ < y^+_1 \\
f^+ = a_2 -\displaystyle\frac{\sigma_t}{2\,a_1\,(y^+)^2}& \text{if } y_1^+ \leqslant y^+ < y_2^+ \\
f^+ = \displaystyle\frac{\sigma_t}{\kappa}\,ln(y^+)+a_3& \text{if } y^+_2 \leqslant y^+\\
\end{array}
\right.
\end{equation}
with
\begin{equation}
  y^+_1 =\left(\displaystyle\frac{1000}{\sigma}\right)^\frac{1}{3} \qquad\qquad
  y^+_2 = \sqrt{\displaystyle\frac{1000\kappa}{\sigma_t}}
\end{equation}
and
\begin{equation}
a_2=15\sigma^{\frac{2}{3}}\qquad\qquad
a_3=15\sigma^{\frac{2}{3}}-\displaystyle\frac{\sigma_t}{2\kappa}
\left(1+
ln\left(\displaystyle\frac{1000\kappa}{\sigma_t}\right)\right)
\end{equation}

\newpage
%%%%%%%%%%%%%%%%%%%%%%%%%%%%%%%%%%
%%%%%%%%%%%%%%%%%%%%%%%%%%%%%%%%%%
\section*{Points to treat}
%%%%%%%%%%%%%%%%%%%%%%%%%%%%%%%%%%
%%%%%%%%%%%%%%%%%%%%%%%%%%%%%%%%%%


The use of \var{HFLUI/CPP} when \var{ISCSTH} is 2 (case with
radiation) needs to be checked (\var{CPP} is actually 1 in this case).

The boundary conditions of the velocity are based on derivations
focusing on only one term of the tangential stress
$(\mu_I+\mu_{t,I})(\ggrad{\vect{u}})\,\vect{n}$ without taking
into account the tranpose gradient.

In order to establish the boundary conditions of the velocity in
$k-\varepsilon$ based on the constraint , a projection onto the plane
tangent to the wall and an arbitrary zero normal velocity
are introduced.

The hypothesis made in order to establish formulae for the different types
of boundary conditions (dissipation, velocities) are based on the assumption
that the mesh is orthogonal at the wall. This assumption is extended
without any caution to the case of non-orthogonal meshes.

The one velocity scale (\fort{cs\_wall\_functions.c}) wall function requires
solving an equation using a Newton algorithm.
The computational cost of the latter is low. One can also used
a  $1/7$ power law (Werner et Wengle) which yields results which are as
accurate as the logarithmic law in the logarithmic region, and which permits
analytical resolutions (chosen option in LES mode). Be careful however,
since with this law, the intersection with the linear law is
slightly different, which thus requires some adaptations (intersection
around 11.81 instead of 10.88 for the law adopted here
$U^+=8,3\,(y^+)^\frac{1}{7}$).


The values of all the physical properties are taken at the cell centres,
without any reconstruction. Without modifying this approach, it would be
useful to keep this in mind.

%
%
%
%Pb de continuite si YPLULI.NE.10.88
%
% Le mode de r\'esolution permettant d'obtenir $u^*$ est particulier. Avec le
%mod\`ele \`a une \'echelle de vitesse, on \'evalue
%tout d'abord la vitesse de frottement $u^*$ issue de la loi logarithmique. On
%pose $u_k=u^*$, puis on calcule la valeur de $y^+$.  Avec le
%mod\`ele \`a deux \'echelles, on calcule tout d'abord $u_k$, on en d\'eduit
%$y^+$ puis $u^*$. Dans les deux cas, si
%$y^+\leqslant y^+_{lim}=\displaystyle\frac{2}{\kappa}$, on applique une condition
%d'adh\'erence (vitesse impos\'ee nulle \`a la paroi, \'energie turbulente et
%tensions de Reynolds impos\'ees nulles, flux nul pour la dissipation).
%Il serait bon de v\'erifier que cette m\'ethode, qui utilise une loi
%logarithmique jusqu'\`a de tr\`es petites
%valeurs de $y^+$, ne conduit pas \`a des valeurs trop faibles de la
%vitesse de frottement lorsqu'on s'approche de la paroi ($y^+ \leqslant 10$).
%La figure \ref{Base_Clptur_fig_loi_log_clptur} propose une illustration~: supposons qu'en un
%point on obtienne, avec la m\'ethode actuelle, $u+\approx 8,6$ et
%$y^+\approx 4,3$. On en d\'eduit alors que la vitesse de frottement est
%$u^*\approx 1$ (courbe logarithmique en trait plein (noire) repr\'esentant
%$ln(y^+)/0,42+5,2$).
%Toutes choses \'egales par ailleurs (ce qui constitue une
%hypoth\`ese en soi), avec une m\'ethode prenant en compte une loi lin\'eaire en
%dessous de $y^+\approx 10$, on aurait obtenu $u^*\approx 2$
%(courbe lin\'eaire en trait plein (rouge) repr\'esentant $2y^+$). Bien
%entendu, cette analyse est relativement na\"\i ve et ne prend pas en compte le
%caract\`ere implicite des r\'esolutions ainsi que le fait qu'il est d'ordinaire
%peu recommand\'e de placer la premi\`ere maille \`a des valeurs aussi faibles de
%$y^+$ avec les mod\`eles de type haut Reynolds.
%
%\begin{figure}[h]
%\centerline{\includegraphics[height=7cm]{loilog}}
%\caption{\label{Base_Clptur_fig_loi_log_clptur}D\'etermination de $y^+$.}
%\end{figure}



% Plus d'actualite a priori, mais pistes de reflexion quand meme
%La limitation par valeur minimale de la vitesse dans (\ref{Base_Clptur_eq_ugrad_clptur})
%a \'et\'e corrig\'ee dans la version 1.1.0.q.
%Auparavant, la formulation \'etait susceptible de conduire
%\`a une valeur trop faible du gradient de vitesse et donc de la production
%turbulente en paroi. Elle s'\'ecrivait~:
%\begin{equation}
%u_{\tau,F,grad} =u_{\tau,I'}-\displaystyle\frac{u^*}{\kappa}min\left[max\left(1,
%2\sqrt{\displaystyle\frac{\rho_I\kappa\, u_k I'F}{\mu_{t,I}}
%}-\displaystyle\frac{1}{2}\right),ln\frac{2}{\kappa}+5,2\kappa\right]
%\end{equation}
%La  formulation a \'et\'e modifi\'ee~:
%\begin{equation}\notag
%u_{\tau,F,grad} =max\left(u^*(\frac{1}{\kappa}ln\displaystyle\frac{2}{\kappa}+5,2),
%u_{\tau,I'}-\displaystyle\frac{u^*}{\kappa}\left[max\left(1,
%2\sqrt{\displaystyle\frac{\rho_I\kappa\, u_k I'F}{\mu_{t,I}}
%}-\displaystyle\frac{1}{2}\right)\right]\right)
%\end{equation}
%Elle a \'et\'e adopt\'ee apr\`es des tests sur des
%configurations de validation (canal, marche descendante,  jet impactant, dune,
%echo, rra) qui n'ont montr\'e aucune influence de la modification.
%\`A partir de la version 1.1.0.t, on a utilis\'e la valeur
%$y^+_\text{lim}=10,88$ et non plus $\frac{2}{\kappa}$ pour caract\'eriser le
%passage de la loi lin\'eaire \`a la loi logarithmique et la
%relation a donc \'et\'e modifi\'ee comme suit~:
%\begin{equation}\notag
%u_{\tau,F,grad} =max\left(u^*(\frac{1}{\kappa}ln(y^+_\text{lim})+5,2),
%u_{\tau,I'}-\displaystyle\frac{u^*}{\kappa}\left[max\left(1,
%2\sqrt{\displaystyle\frac{\rho_I\kappa\, u_k I'F}{\mu_{t,I}}
%}-\displaystyle\frac{1}{2}\right)\right]\right)
%\end{equation}
%Comme, pour des $y^+$ inf\'erieurs \`a $y^+_\text{lim}$, on applique une
%condition d'adh\'erence, cette approche n'est pas continue au voisinage de
%$y^+_\text{lim}$. Il serait utile de se pencher sur la question.
%Il faut cependant garder \`a l'esprit que la condition
%de Dirichlet pour $k$ est prise nulle quand  $y^+$ est inf\'erieur \`a
%$y^+_\text{lim}$, ce qui tend \'egalement  \`a annuler la production,
%quelle que soit la condition \`a la limite utilis\'ee pour la vitesse.


For the thermal law with very small Prandtl numbers compared to 1,
Arpaci and Larsen suggest $y_0^+ \simeq 5/Pr$ (with proof from
experimental data) rather than $Pr_t/(Pr\,\kappa)$  (current value,
computed as the analytical intersection of the linear and logarithmic
laws considered). One should address this question.



%-------------------------------------------------------------------------------

% This file is part of Code_Saturne, a general-purpose CFD tool.
%
% Copyright (C) 1998-2013 EDF S.A.
%
% This program is free software; you can redistribute it and/or modify it under
% the terms of the GNU General Public License as published by the Free Software
% Foundation; either version 2 of the License, or (at your option) any later
% version.
%
% This program is distributed in the hope that it will be useful, but WITHOUT
% ANY WARRANTY; without even the implied warranty of MERCHANTABILITY or FITNESS
% FOR A PARTICULAR PURPOSE.  See the GNU General Public License for more
% details.
%
% You should have received a copy of the GNU General Public License along with
% this program; if not, write to the Free Software Foundation, Inc., 51 Franklin
% Street, Fifth Floor, Boston, MA 02110-1301, USA.

%-------------------------------------------------------------------------------

\programme{clptrg}


\vspace{1cm}
%%%%%%%%%%%%%%%%%%%%%%%%%%%%%%%%%%
%%%%%%%%%%%%%%%%%%%%%%%%%%%%%%%%%%
\section*{Fonction}
%%%%%%%%%%%%%%%%%%%%%%%%%%%%%%%%%%
%%%%%%%%%%%%%%%%%%%%%%%%%%%%%%%%%%
Ce sous-programme est d�di� au calcul des conditions aux limites en paroi
rugueuse. On utilise le
formalisme introduit dans \var{CONDLI} pour les conditions 
aux limites g\'en\'erales. 

Par conditions aux limites en paroi, on entend ici l'ensemble des conditions aux
limites pour la vitesse, les grandeurs turbulentes ($k$, $\varepsilon$),
la temp\'erature lorsqu'elle a une valeur de paroi impos\'ee   
(ou l'enthalpie et plus g\'en\'eralement les 
{\it VarScalaires}\footnote{Comme dans \fort{condli}, on d\'esignera ici par 
{\it VarScalaire} toute variable solution
d'une \'equation de convection-diffusion autre que la 
vitesse, la pression et les grandeurs turbulentes $k$, $\varepsilon$. La
d\'enomination {\it VarScalaire} pourra en particulier se rapporter 
\`a la temp\'erature, \`a l'enthalpie ou \`a un scalaire passif.} 
\`a traiter en paroi en prenant en compte une loi de similitude
pour la couche limite associ\'ee). Pour les {\it VarScalaires}, en particulier,
lorsque les conditions aux limites de paroi sont du type Neumann (homog\`ene ou non),
elles sont trait\'ees dans \fort{condli} et on ne s'y int\'eresse donc pas
ici. En particulier, les conditions aux limites des  {\it VarScalaires}
repr\'esentant la variance de fluctuations d'autres  {\it VarScalaires} ne
sont pas trait\'ees ici car leur traitement en paroi est de type Neumann homog\`ene. 

On indique comment sont calcul\'es les couples de coefficients 
$A_b$ et $B_b$ qui sont utilis\'es pour le calcul de certains 
termes discrets des \'equations \`a r\'esoudre et qui
permettent  en particulier de d\'eterminer une valeur associ\'ee aux faces 
de bord $f_{b,int}$ (en un point localis\'e au ``centre'' de la face de bord, 
barycentre de ses sommets) par la
relation $f_{b,int} = A_b+B_b\,f_{I'}$ ($f_{I'}$ est la valeur de 
la variable au point
$I'$, projet\'e du centre de la cellule jouxtant le bord sur la droite 
normale \`a 
la face de bord et passant par son centre~: voir la figure~\ref{fig_flux_clptur}). 

\begin{figure}[h]
\centerline{\includegraphics[height=7cm]{fluxbord}}
\caption{\label{fig_flux_clptur}Cellule de bord.}
\end{figure}

%%%%%%%%%%%%%%%%%%%%%%%%%%%%%%%%%%
%%%%%%%%%%%%%%%%%%%%%%%%%%%%%%%%%%
\section*{Discr\'etisation}
%%%%%%%%%%%%%%%%%%%%%%%%%%%%%%%%%%
%%%%%%%%%%%%%%%%%%%%%%%%%%%%%%%%%%

\etape{Notations\vspace{0,3cm}}
%%%%%%%%%%%%%%%%%%%%%%%%%%%%%%%%%%%%%%%%%%%%%%%%%%%%%%%%%%%%%%%%%%%%%%%%%%%%%%%
La vitesse de la paroi est not\'ee
$\vect{v}_p$. On la suppose projet\'ee dans le plan tangent \`a la paroi (si
elle ne l'est pas, le code la projette).

La vitesse du fluide est not\'ee $\vect{u}$. L'indice $I$, $I'$ ou $F$ d\'esigne le
point auquel elle est estim\'ee. La composante tangentielle par rapport \`a la
paroi est not\'ee $u_\tau$. 
 La vitesse du fluide dans le rep\`ere li\'e \`a la paroi (vitesse
``relative'') est not\'ee $\vect{u}^r=\vect{u} - \vect{v}_p$. 

Le rep\`ere orthonorm\'e li\'e \`a la paroi est not\'e 
$\hat {\mathcal R}=(\vect{\tau},\vect{\tilde{n}},\vect{b})$. 
\begin{itemize}
\item [$\bullet$] $\vect{\tilde{n}}=-\vect{n}$ est le vecteur norm\'e
orthogonal \`a la paroi et dirig\'e vers l'int\'erieur du domaine de calcul.
\item [$\bullet$] $\vect{\tau} = \displaystyle\frac{1}{\|\vect{u}^r_{I'}-(\vect{u}^r_{I'}\,.\,\vect{\tilde{n}})\|}\left[\vect{u}^r_{I'}-(\vect{u}^r_{I'}\,.\,\vect{\tilde{n}})\right]$ est le vecteur norm\'e port\'e par la projection de la vitesse 
relative en $I'$, $\vect{u}^r_{I'}$, dans le plan tangent \`a la
paroi ({\it i.e.} orthogonal \`a $\vect{\tilde{n}}$)~: voir la
figure~\ref{fig_flux_clptur}. 
\item [$\bullet$] $\vect{b}$ est le vecteur norm\'e compl\'etant le rep\`ere direct. 
\end{itemize}

\vspace{0.5cm}

Dans le cas du {\bf mod\`ele \`a deux  \'echelles de vitesse}, on note~:
\begin{itemize}
\item [-] $u_k$ la
vitesse de frottement en paroi obtenue \`a partir de l'\'energie turbulente.

\item [-] $u^*$ la vitesse de frottement en paroi d\'etermin\'ee \`a
partir de la relation $ \displaystyle\frac{u^r_{\tau,I'}}{u^*} = f(z_p)$. 

\item [-] $z_p$  repr�sente une distance � la paroi
      (c'est � dire la distance depuis le bord du domaine de calcul),  soit 
$z_p= I'F$ (voir la figure~\ref{fig_flux_clptur}). La fonction $f$ traduit la forme id�ale du profil de
      vitesse. Dans l'atmosph�re, cette fonction est donn�e par
      une loi de type logarithmique faisant intervenir la rugosit� dynamique de
      la paroi $z_0$~:

$f(z_p)= \displaystyle\frac{1}{\kappa} ln \left ( \displaystyle \frac
      {z_p+z_0}{z_0} \right ) $


\item [-] Les deux \'echelles de vitesse $u_k$ et $u^*$ sont simples \`a
      calculer mais leur obtention  
n\'ecessite la connaissance de l'\'energie turbulente $k_I$ au centre de la
maille jouxtant la face de bord. 

\item [-] Le mod\`ele \`a deux \'echelles 
est le mod\`ele par d\'efaut dans \CS. Il permet souvent, et en particulier 
dans les cas avec transfert thermique, de diminuer les effets de certains 
d\'efaut li\'es au mod\`ele $k-\varepsilon$ (exemple classique du jet impactant).  
\end{itemize}

On se sert plus bas de $u^*$ et $u_k$ pour les conditions aux limites portant
sur la vitesse et les scalaires (temp\'erature en particulier). 


\begin{equation}
\label{Eq_Mod_'2ech_Vit}
\begin{array}{l}
\text{\bf Mod\`ele \`a deux \'echelles de vitesse}\\
\left\{\begin{array}{l}
u_k = C_\mu^\frac{1}{4}k_I^\frac{1}{2}\\
u^* \text{solution de }  \displaystyle\frac{u^r_{\tau,I'}}{u^*} =
\displaystyle\frac{1}{\kappa}ln(z_k)\\
z_k=\displaystyle\frac{I'F+z_0}{z_0} = \displaystyle\frac{z_p+z_0}{z_0}\\
\text{   avec   } C_\mu =0,09 \text{  et }  \kappa = 0,42 
\end{array}\right.\\ 
\end{array}
\end{equation}


\vspace{0.5cm}

Dans le cas du {\bf mod\`ele \`a une \'echelle de vitesse}, on note $u^*$ l'unique vitesse
de frottement en paroi solution de l'\'equation 
$\displaystyle\frac{u^r_{\tau,I'}}{u^*} = f(z_p)$. La grandeur
$z_p$  repr\'esente une distance \`a la paroi, soit
$z_p=I'F$. La fonction $f$ traduit la forme id\'eale du profil de vitesse comme
pour le mod\`ele \`a deux \'echelles de vitesses. On peut 
noter que cette vitesse de frottement, d'un calcul plus d\'elicat (point fixe), 
s'obtient  cependant sans faire r\'ef\'erence aux
variables turbulentes ($k$, $\varepsilon$). Par commodit\'e, on posera
dans le cas du mod\`ele \`a une \'echelle $u_k=u^*$.

On se sert plus bas de $u^*$ et $u_k$ pour les conditions aux limites portant
sur la vitesse et les scalaires (temp\'erature en particulier). 

\begin{equation}
\begin{array}{l}
\text{\bf Mod\`ele \`a une \'echelle de vitesse}\\
\left\{\begin{array}{l}
u_k = u^*\\
u^* \text{solution de }  \displaystyle\frac{u^r_{\tau,I'}}{u^*} =
\displaystyle\frac{1}{\kappa}ln(z_k)\\
z_k=\displaystyle\frac{I'F+z_0}{z_0} = \displaystyle\frac{z_p+z_0}{z_0}\\
\text{   avec   } C_\mu =0,09 \text{  et }  \kappa = 0,42 
\end{array}\right.\\ 
\end{array}
\end{equation}

\etape{Conditions aux limites pour la vitesse en $k-\varepsilon$\vspace{0,3cm}}
%%%%%%%%%%%%%%%%%%%%%%%%%%%%%%%%%%%%%%%%%%%%%%%%%%%%%%%%%%%%%%%%%%%%%%%%%%%%%%%
On consid\`ere tout d'abord les conditions utilis\'ees dans le cas d'un calcul
r\'ealis\'e avec le mod\`ele $k-\varepsilon$. Ce sont en effet les plus
complexes et les plus g\'en\'erales. 

Les conditions aux limites sont n�cessaires pour imposer au bord la contrainte
tangentielle $\sigma_\tau=\rho_Iu^*u_k$ ad�quate dans l'\'equation de  quantit\'e de
mouvement\footnote{Proposition de modification des conditions aux limites de
paroi turbulente pour le Solveur Commun dans le cadre du mod\`ele
$k-\varepsilon$ standard, rapport EDF HI-81/00/019/A, 2000, M. Boucker, J.-D. Mattei.} 
($\rho_I$ est la masse volumique au centre de la
cellule $I$). Le terme qui n\'ecessite des conditions aux limites est celui qui contient la
d\'eriv\'ee de la vitesse dans la direction normale \`a la paroi, 
soit\footnote{Le terme en gradient transpos\'e est trait\'e dans \fort{visecv} 
et ne sera pas consid\'er\'e ici.}~:
$(\mu_I+\mu_{t,I})\ggrad{\vect{u}}\,\vect{n}$. Il appara\^\i t au second membre
de l'\'equation de quantit\'e de mouvement usuelle (voir \fort{bilsc2} et \fort{predvv}). 

Dans le cas o\`u le mod\`ele $k-\varepsilon$ a tendance \`a surestimer la
production de l'\'energie turbulente, l'\'echelle de longueur du mod\`ele,
$L_{k-\varepsilon}$, 
peut devenir significativement plus grande que l'\'echelle th\'eorique maximale
des tourbillons de la couche limite turbulente $L_{\text{th\'eo}}$. On note : 
\begin{equation}
\left\{\begin{array}{l}
L_{k-\varepsilon} = C_{\mu}\displaystyle\frac{k^\frac{3}{2}}{\varepsilon}\\
L_{\text{th\'eo}} = \kappa\, \left( I'F+z_0 \right) = \kappa\, \left(z_p+z_0 \right)
\end{array}\right.
\end{equation}

Dans le cas o\`u $L_{k-\varepsilon}>L_{\text{th\'eo}}$, on a donc
$\mu_{t,I}>\mu_{t}^{lm}$ avec $\mu_{t,I}$ la viscosit\'e turbulente du mod\`ele
$k-\varepsilon$ au point $I$ et $\mu_{t}^{lm}=\rho_I L_{\text{th\'eo}}u_k$ la
viscosit\'e turbulente du mod\`ele de longueur de m\'elange. En outre, la
contrainte tangentielle peut s'\'ecrire en faisant appara\^\i tre la viscosit\'e
turbulente, soit~: 
\begin{equation}
\sigma_\tau = \rho_Iu^*u_k = \displaystyle\frac{u^*}{\kappa\,
 \left(I'F+z_0 \right)}
\underbrace{\rho_I\kappa\, \left( I'F+z_0 \right)  u_k}_{\mu^{lm}_t}
\end{equation}
L'\'echelle de viscosit\'e introduite dans la contrainte est alors en
contradiction avec celle d\'eduite de la turbulence calcul\'ee alentour par le
mod\`ele. On pr\'ef\`ere d\`es lors \'ecrire, en utilisant l'\'echelle de
longueur du $k-\varepsilon$ chaque fois qu'elle est inf\'erieure \`a la limite
$L_{\text{th\'eo}}$~: 
\begin{equation}
\sigma_\tau = \displaystyle\frac{u^*}{\kappa\, \left(I'F+z_0 \right)} max(\mu_{t}^{lm},\mu_{t,I})
\end{equation}

On peut alors utiliser cette valeur  pour le calcul du flux
diffusif qui en d\'epend dans l'\'equation de Navier-Stokes~: 
\begin{equation}\label{eq_grad_sigma_clptur}
(\mu_I+\mu_{t,I})\ggrad{\vect{u}}\,\vect{n}=-\sigma_\tau \vect{\tau}
\end{equation} 

Or, le gradient de vitesse (gradient \`a la face de bord) est calcul\'e dans le
code sous la forme suivante~: 
\begin{equation}\label{eq_grad_uf_clptur}
(\mu_I+\mu_{t,I})\ggrad{\vect{u}}\,\vect{n}=
\displaystyle\frac{(\mu_I+\mu_{t,I})}{\overline{I'F}}(\vect{u}_F-\vect{u}_{I'})
\end{equation} 

Du rapprochement de (\ref{eq_grad_sigma_clptur}) et de
(\ref{eq_grad_uf_clptur}) on tire alors la valeur de $\vect{u}_F$ \`a
imposer, soit $\vect{u}_{F,flux}$~(respect du flux de quantit\'e de mouvement)~:
\begin{equation}\label{eq_uf_flux_clptur}
\begin{array}{ll}
\vect{u}_{F,flux}&=\vect{u}_{I'}-\displaystyle\frac{\overline{I'F}}{\mu_I+\mu_{t,I}}\sigma_\tau \vect{\tau}\\
                 &=\vect{u}_{I'}-\displaystyle\frac{u^*}{\kappa\,
		  (\mu_I+\mu_{t,I})} max(\mu_{t}^{lm},\mu_{t,I})
		  \, \displaystyle\frac {I'F} {\left(I'F+z_0 \right)} \vect{\tau}
\end{array} 
\end{equation} 

En r\'ealit\'e, une approximation suppl\'ementaire est r\'ealis\'ee, qui
consiste \`a imposer la vitesse normale nulle \`a la paroi et \`a utiliser
l'\'equation (\ref{eq_uf_flux_clptur}) projet\'ee sur le plan tangent \`a la
paroi, soit~:
\begin{equation}
\vect{u}_{F,flux}=\left[u_{\tau,I'}-\displaystyle\frac{u^*}{\kappa\,
(\mu_I+\mu_{t,I})} max(\mu_{t}^{lm},\mu_{t,I}) \, \displaystyle\frac {I'F} {\left(I'F+z_0 \right)} \right]\vect{\tau}
\end{equation} 

Enfin, on peut \'egalement faire appara\^\i tre la vitesse de la paroi
dans l'expression finale~:
\begin{equation}
\begin{array}{l}
\text{\bf Conditions aux limites sur la vitesse de type ``flux''}\,(k-\varepsilon)\\
\left\{\begin{array}{l}
\vect{u}_{F,flux}=\vect{v}_p+\left[u^r_{\tau,I'}-\displaystyle\frac{u^*}{\kappa\,
(\mu_I+\mu_{t,I})} max(\mu_{t}^{lm},\mu_{t,I})  \, \displaystyle\frac {I'F} {\left(I'F+z_0 \right)}\right]\vect{\tau}
\end{array}\right.\\ 
\end{array} 
\end{equation} 

Un premier couple de coefficients $A_{flux}$ et $B_{flux}$ s'en d\'eduit (pour
chaque composante de vitesse s\'epar\'ement) et il n'est utilis\'e que pour le
calcul du terme d\'ependant de la contrainte tangentielle (voir \fort{bilsc2})~: 
\begin{equation}
\begin{array}{l}
\text{\bf Coefficients associ\'es aux conditions aux limites sur la vitesse de
type ``flux''} (k-\varepsilon)\\
\left\{\begin{array}{l}
\vect{A}_{flux}=\vect{v}_p+\left[u^r_{\tau,I'}-\displaystyle\frac{u^*}{\kappa\,
(\mu_I+\mu_{t,I})} max(\mu_{t}^{lm},\mu_{t,I}) \, \displaystyle\frac {I'F} {\left(I'F+z_0 \right)} \right]\vect{\tau} \\
\vect{B}_{flux} = \vect{0}
\end{array}\right.
\end{array}
\end{equation} 

On a vu ci-dessus comment imposer une condition \`a la limite permettant de
calculer correctement le terme en contrainte. Une analyse suppl\'ementaire est
n\'ecessaire pour le calcul des gradients de vitesse. On cherche \`a trouver une
valeur en face de bord qui permette d'obtenir, avec la formulation adopt\'ee pour le gradient, la valeur de la production turbulente la
plus proche possible de la valeur th\'eorique, elle-m\^eme d\'etermin\'ee
en utilisant la loi
logarithmique, pour \'evaluer la d\'eriv\'ee normale de la vitesse tangentielle.
Ainsi, on d\'efinit (au point $I$)~: 
\begin{equation}\label{eq_ptheo_clptur}
P_{\text{th\'eo}} = \rho_I u^* u_k
\|\displaystyle\frac{\partial u_\tau}{\partial\vect{n}}\|_{I} = 
\rho_I \displaystyle\frac{u_k(u^*)^2}{\kappa\, \left(I'F+z_0 \right)}
\end{equation}

Par ailleurs, le terme pr\'epond\'erant de la production calcul\'ee dans la
cellule $I$ est, pour les situations classiques ($z$ est l'ordonn\'ee sur l'axe
de vecteur directeur $\vect{\tilde{n}}$), 
\begin{equation}
P_{\text{calc}} =
\mu_{t,I}\left(\displaystyle\frac{\partial u_\tau}{\partial z}\right)^2_{I}
\end{equation}

\begin{figure}[h]
\centerline{\includegraphics[height=7cm]{bordortho}}
\caption{\label{fig_bord_ortho_clptur}Cellule de bord - Maillage orthogonal.}
\end{figure}
 
Or, le gradient normal de la vitesse tangentielle (gradient cellule) est
calcul\'e dans le code en volumes finis et son expression dans le cas d'un
maillage orthogonal et r\'egulier est la suivante (voir les notations sur la figure~\ref{fig_bord_ortho_clptur})~:
\begin{equation}
P_{\text{calc}} =
\mu_{t,I}\left(\displaystyle\frac{u_{\tau,G}-u_{\tau,F}}{2d}\right)^2 = 
\mu_{t,I}\left(\displaystyle\frac{u_{\tau,I}+u_{\tau,J}-2u_{\tau,F}}{4d}\right)^2 
\end{equation}
On suppose alors que $u_{\tau,J}$ peut \^etre obtenu \`a partir de $u_{\tau,I}$
et du gradient normal de $u_{\tau}$ \'evalu\'e en G \`a partir de la loi
logarithmique, soit~:
\begin{equation}
\label{eq_dvp_lim_utau}
u_{\tau,J}=u_{\tau,I}+ IJ\,.\,(\partial_z u_{\tau})_G+\mathcal{O} (IJ^{\,2}) \approx
u_{\tau,I}+ IJ\,.\,\left[\partial_z \left(\displaystyle
\frac{u^*}{\kappa}\,ln{ (z)} \right)\right]_G=
u_{\tau,I}+2d \, \displaystyle\frac{u^*}{\kappa \left(2d + z_0\right)}
\end{equation}
et l'on obtient alors~:
\begin{equation}\label{eq_pcalc_clptur}
\begin{array}{lll}
P_{\text{calc}} 
&=&\mu_{t,I}\left(\displaystyle\frac{2u_{\tau,I}+2d  \, \displaystyle\frac{\,u^*}{\kappa
	     \left(2d + z_0\right) } -2u_{\tau,F}}{4d}\right)^2 \\
&=&
\mu_{t,I}\left(\displaystyle\frac{u_{\tau,I}+d \,\displaystyle\frac{\,u^*}{\kappa
	  \left(2d + z_0\right)} -u_{\tau,F}}{2d}\right)^2 
\end{array}
\end{equation}

On rapproche alors les \'equations (\ref{eq_ptheo_clptur}) et
(\ref{eq_pcalc_clptur}) pour imposer que la production calcul\'ee soit \'egale
\`a la la production th\'eorique. On \'etend sans pr\'ecaution les formules
pr\'ec\'edentes aux maillages non orthogonaux (la vitesse en $I$ est 
alors simplement prise en $I'$). 
On obtient alors l'expression suivante pour $u_{\tau,F}$~: 
\begin{equation}
u_{\tau,F,grad} =u_{\tau,I'}-\displaystyle\frac{u^*}{\kappa}\left(
2d\sqrt{\displaystyle\frac{\rho_I\kappa\, u_k }{\mu_{t,I} \left(I'F
							   +z_0\right) }
}-\displaystyle\frac{1}{2 + z_0/I'F}\right)
\end{equation}

On impose d'autre part que le gradient reste au moins aussi raide que celui
donn\'e par la d\'eriv\'ee normale du profil de vitesse th\'eorique
(logarithmique) en $I'$~:\\
$\partial_z u_{\tau} = \partial_z (\displaystyle
\frac{u^*}{\kappa}\,ln{ (z)} ) =\displaystyle\frac{u^*}{\kappa\,\left(I'F + z_0\right) }$, soit
donc~:
\begin{equation}
u_{\tau,F,grad} =u_{\tau,I'}-\displaystyle\frac{u^*}{\kappa}max\left(1,
2d\sqrt{\displaystyle\frac{\rho_I\kappa\, u_k }{\mu_{t,I} \left(I'F + z_0\right)}}-\displaystyle\frac{1}{2 + z_0/I'F}\right) 
\end{equation}


La vitesse normale \`a la paroi est impos\'ee nulle. 
De plus, si la vitesse tangentielle en $I'$ est
nulle (de valeur absolue inf\'erieure \`a une limite num\'erique arbitraire de
$10^{-12}$) une condition d'adh\'erence est appliqu\'ee. Enfin, on peut
\'egalement faire appara\^\i tre la vitesse de la paroi dans l'expression finale~:
\begin{equation}
\begin{array}{l}
\text{\bf Conditions aux limites sur la vitesse de type ``gradient''} (k-\varepsilon)\\
\left\{\begin{array}{l}
\vect{u}_{F,grad}=\vect{v}_p 
          \qquad\qquad\text{ si } u^r_{\tau,I'} < 10^{-12}  \\ 
\vect{u}_{F,grad}=\vect{v}_p +
u^r_{\tau,I'}-\displaystyle\frac{u^*}{\kappa}\left[max\left(1,2d\sqrt{\displaystyle\frac{\rho_I\kappa\, u_k }{\mu_{t,I}
\left(I'F + z_0\right)}}-\displaystyle\frac{1}{2 + z_0/I'F}\right)\right] \vect{\tau}
\end{array}\right.
\end{array}
\end{equation} 

Un second couple de coefficients $A_{grad}$ et $B_{grad}$ s'en d\'eduit (pour
chaque composante de vitesse s\'epar\'ement) et est utilis\'e chaque fois que le
gradient de la vitesse est n\'ecessaire (hormis pour les termes d\'ependant de
la contrainte tangentielle, trait\'es dans \fort{bilsc2} au moyen des
coefficients $A_{flux}$ et $B_{flux}$)~: 
\begin{equation}
\begin{array}{l}
\text{\bf Coefficients associ\'es aux conditions aux limites sur la vitesse }\\
\qquad\qquad\qquad\qquad\text{\bf de type ``gradient''} (k-\varepsilon)\\
\left\{\begin{array}{l}
\vect{A}_{grad}=\vect{v}_p 
                    \qquad\qquad\text{ \ si\ } u^r_{\tau,I'} < 10^{-12}  \\
\vect{A}_{grad}=\vect{v}_p+
 u^r_{\tau,I'}-\displaystyle\frac{u^*}{\kappa}\left[max\left(1,2d\sqrt{\displaystyle\frac{\rho_I\kappa\, u_k }{\mu_{t,I} 
\left(I'F + z_0\right)}}-\displaystyle\frac{1}{2 + z_0/I'F}\right)\right] \vect{\tau}\\ 
\vect{B}_{grad} = \vect{0}
\end{array}\right.
\end{array}
\end{equation} 

\newpage

\etape{Conditions aux limites pour les variables $k$ et $\varepsilon$ (mod\`ele
$k-\varepsilon$ standard)\vspace{0,3cm}}

On impose sur $k$ une condition de Dirichlet tir\'ee de la vitesse de frottement
$u_k$ (se reporter \`a l'\'equation~(\ref{Eq_Mod_'2ech_Vit})), soit : 
\begin{equation}
k= \displaystyle\frac{u_k^2}{C_\mu^\frac{1}{2}}
\end{equation} 


On cherche \`a imposer la d\'eriv\'ee normale de $\varepsilon$ \`a partir de la
loi th\'eorique suivante (voir les notations sur la figure \ref{fig_bord_ortho_clptur})~:
\begin{equation}\label{eq_partialep_theo_clptur}
G_{\text{th\'eo},\varepsilon} = \displaystyle\frac{\partial}{\partial z}
 \left[ \displaystyle\frac{u_k^3}{\kappa\, \left(z + z_0\right)}\right]
\end{equation} 
 


On utilise le point $M$ pour imposer une condition \`a la limite avec un ordre plus
\'elev\'e en espace. En effet, la simple utilisation de la relation
$\varepsilon_F=\varepsilon_I+d\partial_z\varepsilon_I + O(d^2)$ conduit \`a une
pr\'ecision d'ordre 1. 
En utilisant les d\'eveloppements limit\'es suivants, on peut
obtenir une pr\'ecision \`a l'ordre 2~:
\begin{equation}
\left\{\begin{array}{ll}
\varepsilon_M&=\varepsilon_I-\displaystyle\frac{d}{2}\partial_z\varepsilon_I+\displaystyle\frac{d^2}{8}\partial^2_z\varepsilon_I+O(d^3)\\
\varepsilon_M&=\varepsilon_F+\displaystyle\frac{d}{2}\partial_z\varepsilon_F+\displaystyle\frac{d^2}{8}\partial^2_z\varepsilon_F+O(d^3)
\end{array}\right.
\end{equation}
Par diff\'erence, ces relations conduisent \`a 
\begin{equation}\label{eq_epsf_clptur}
\varepsilon_F=\varepsilon_I-\displaystyle\frac{d}{2}(\partial_z\varepsilon_I+\partial_z\varepsilon_F)+O(d^3)
\end{equation}
De plus, on a 
\begin{equation}
\left\{\begin{array}{ll}
\partial_z\varepsilon_I&=\partial_z\varepsilon_M+d\partial^2_z\varepsilon_M+O(d^2)\\
\partial_z\varepsilon_F&=\partial_z\varepsilon_M-d\partial^2_z\varepsilon_M+O(d^2)
\end{array}\right.
\end{equation}
La somme de ces deux derni\`eres relations permet d'\'etablir
$\partial_z\varepsilon_I+\partial_z\varepsilon_F=2\partial_z\varepsilon_M+O(d^2)$ et, en reportant dans
(\ref{eq_epsf_clptur}), on obtient alors une expression de $\varepsilon_F$ \`a
l'ordre 2, comme souhait\'e~:
\begin{equation}
\varepsilon_F=\varepsilon_I-d\partial_z\varepsilon_M+O(d^3)
\end{equation}
On utilise alors la valeur th\'eorique (\ref{eq_partialep_theo_clptur}) pour
\'evaluer $\partial_z\varepsilon_M$ et on obtient alors la valeur \`a imposer au bord ($d=I'F$)~:
\begin{equation}
\varepsilon_F=\varepsilon_I+d\displaystyle\frac{ u_k^3}{\kappa\, (d/2+ z_0)^2}
\end{equation}


Cette relation est \'etendue au cas de maillages non orthogonaux sans
pr\'ecaution (ce qui doit d\'egrader l'ordre en espace).

On a finalement~:
 
\begin{equation}
\begin{array}{l}
\text{\bf Conditions aux limites sur les variables } k \text { \bf et } \varepsilon \\
\left\{\begin{array}{ll}
k_F&= \displaystyle\frac{u_k^2}{C_\mu^\frac{1}{2}}\\
\varepsilon_F&=\varepsilon_{I'}+I'F\displaystyle\frac{ u_k^3}{\kappa\,
 (I'F/2 + z_0)^2}
\end{array}\right. \\
\end{array}
\end{equation}
et les coefficients associ\'es 
\begin{equation}
\begin{array}{l}
\text{\bf Coefficients associ\'es aux conditions aux limites sur les variables }
k \text { \bf et } \varepsilon \\
\left\{\begin{array}{llll}
A_k&= \displaystyle\frac{u_k^2}{C_\mu^\frac{1}{2}} &\text{ et } B_k&= 0 \\
A_\varepsilon&=I'F\displaystyle\frac{ u_k^3}{\kappa\, (I'F/2 + z_0)^2}&\text{ et } B_\varepsilon&= 1 
\end{array}\right.\\
\end{array}
\end{equation}

\newpage
\etape{Conditions aux limites pour les {\it VarScalaires}\vspace{0,3cm}}
On ne traite ici que les conditions se pr\'esentant sous la forme d'une valeur
impos\'ee (\`a la paroi ou en retrait de celle-ci avec un coefficient
d'\'echange externe \'eventuel). On se reporte aux notations de la figure
\ref{fig_flux_clptur} et \`a la pr\'esentation g\'en\'erale disponible dans
\fort{condli} dont on ne reprend que la partie essentielle ci-dessous. 

La conservation du flux normal au bord pour la variable $f$ s'\'ecrit sous la forme~:
\begin{equation}\label{eq_flux_clptur}
\begin{array}{l}
    \underbrace{h_{int}(f_{b,int}-f_{I'})}_{\phi_{int}}
  = \underbrace{h_{b}(f_{b,ext}-f_{I'})}_{\phi_{b}}
  = \left\{\begin{array}{ll}
    \underbrace{h_{imp,ext}(f_{imp,ext}-f_{b,ext})}_{\phi_{\text{\it r\'eel
impos\'e}}} &\text{(condition de Dirichlet)}\\
    \underbrace{\phi_{\text{\it imp,ext}}}_{\phi_{\text{\it r\'eel impos\'e}}}
            &\text{(condition de Neumann)}
           \end{array}\right.
\end{array}
\end{equation}


On r\'earrange ces deux \'equations afin d'obtenir la valeur num\'erique
$f_{b,int}=f_{F}$ \`a imposer en face de paroi, \'etant donn\'ees les valeurs de
$f_{imp,ext}$ et de $h_{imp,ext}$ fix\'ees par l'utilisateur et la valeur $h_{b}$
dict\'ee par les lois de similitude qui seront d\'etaill\'ees plus bas. On
pr\'ecise les coefficients $A$ et $B$ qui s'en d\'eduisent naturellement. 

\begin{equation}\label{eq_fbint_clptur}
\begin{array}{l}
\text{\bf Conditions aux limites sur les {\it VarScalaires} }\\
f_{b,int} = 
\underbrace{\displaystyle\frac{h_{imp,ext}}{h_{int}+h_r h_{imp,ext} } f_{imp,ext}}_{A} +
\underbrace{\displaystyle\frac{h_{int}+h_{imp,ext}(h_r-1)}{h_{int}+h_r h_{imp,ext} }}_{B} f_{I'} 
\text{  avec } h_r=\displaystyle\frac{h_{int}}{h_{b}}
\end{array}
\end{equation}


\newpage
{\bf Principe de similitude~: calcul de } $h_b$. 

Dans l'expression (\ref{eq_fbint_clptur}), seule reste �
d\'eterminer la valeur de $h_{b}$, celle de $h_{int}$ \'etant une valeur
num\'erique coh\'erente avec le mode de calcul des gradients aux faces et
pr\'ecis\'ee dans \fort{condli} ($h_{int}=\displaystyle\frac{\alpha}{\overline{I'F}}$). La
valeur de  $h_{b}$ doit permettre de 
relier le flux \`a l'\'ecart entre les valeurs $f_{I'}$ et $f_{b,ext}$ en 
prenant en compte la couche limite (le profil de $f$ n'est pas toujours 
lin\'eaire)~:
\begin{equation}
\phi_b=h_b\,(f_{b,ext}-f_{I'})
\end{equation} 

Les consid\'erations suivantes sont pr\'esent\'ees en adoptant des notations
g\'en\'erales. En particulier, le nombre de Prandtl-Schmidt est not\'e 
$\sigma=\displaystyle\frac{\nu\,\rho\,C}{\alpha}$. 
Lorsque le scalaire $f$ consid\'er\'e est la temp\'erature,  
on a (voir \fort{condli})~: 
\begin{list}{$\bullet$}{}
\item $C=C_p$ (chaleur massique), 
\item $\alpha=\lambda$ (conductivit\'e mol\'eculaire), 
\item $\sigma = \displaystyle\frac{\nu\,\rho\,C_p}{\lambda} = Pr$ 
       (nombre de Prandtl), 
\item $\sigma_t = Pr_t$ (nombre de Prandtl turbulent), 
\item $\phi=\left(\lambda+\displaystyle\frac{C_p\mu_t}{\sigma_t}\right)
        \displaystyle\frac{\partial T}{\partial z}$ (flux en $Wm^{-2}$). 
\end{list}

On s'est appuy� sur la r\'ef\'erence "The atmospheric boundary layer", 
J. R. Garratt, Cambridge University Press. 

Le flux en paroi s'�crit, pour le scalaire $f$ (le flux est positif s'il est
entrant dans le domaine fluide, comme l'indique l'orientation de l'axe $z$)~: 
\begin{equation}\label{Eq_Flux_scalaire}
\phi = -\left(\alpha+C\,\frac{\mu_t}{\sigma_t}\right)
                  \frac{\partial f}{\partial z} 
     = -\rho\,C \left(\displaystyle\frac{\alpha}{\rho\,C}+
                                \frac{\mu_t}{\rho\sigma_t}\right)
                  \frac{\partial f}{\partial z}
\end{equation}

\noindent Pour la temp�rature, avec $a=\displaystyle\frac{\lambda}{\rho\,C_p}$ et 
$a_t=\displaystyle\frac{\mu_t}{\rho\,\sigma_t}$, 
on a donc, de mani�re �quivalente~:
\begin{equation}
\phi = -\rho\,C_p(a+a_t)\frac{\partial T}{\partial z}
\end{equation}

\noindent On introduit $f^*$ afin d'adimensionner $f$, en utilisant la valeur du flux 
au bord $\phi_b$~:
\begin{equation}
f^* = -\displaystyle\frac{\phi_b}{\rho\,C\,u_k}
\end{equation}
Pour la temp\'erature, on a donc~:
\begin{equation}
T^* = -\displaystyle\frac{\phi_b}{\rho\,C_p\,u_k}
\end{equation}

\noindent On rappelle que dans le cas du  mod\`ele \`a deux  \'echelles de vitesse,  $u_k$ est la
vitesse de frottement en paroi obtenue \`a partir de l'\'energie
cin�tique moyenne du mouvement turbulent\footnote{$u_k = C_\mu^\frac{1}{4}k_I^\frac{1}{2}$}. Dans le
cas du  mod�le � une �chelle de vitesse, on pose $u_k=u^*$ avec $u^*$
la vitesse de frottement en paroi d�termin�e � partir de la loi logarithmique. 

On divise alors les membres de l'\'equation~(\ref{Eq_Flux_scalaire}) 
par $\phi_b$. Pour le membre de gauche, on simplifie en utilisant le fait 
que le flux se conserve et donc que $\phi=\phi_b$. Pour le membre de droite, 
on remplace $\phi_b$ par sa valeur $-\rho\,C\,u_k\,f^*$. Avec les notations~: 
\begin{equation}
       \nu=\displaystyle\frac{\mu}{\rho}
\qquad \nu_t=\displaystyle\frac{\mu_t}{\rho}
\qquad f^+=\displaystyle\frac{f-f_{b,ext}}{f^*}
\end{equation}
on a~:
\begin{equation}\label{Eq_Flux_scalaire_adim}
1 =  \left(\displaystyle\frac{\nu}{\sigma}+
              \displaystyle\frac{\nu_t} {\sigma_t}\right)
                  \displaystyle\frac{\partial f^+}{\partial z} \displaystyle\frac{1}{u_k}
\end{equation}

Remarquons d\`es \`a pr\'esent qu'avec les notations pr\'ec\'edentes, 
$h_b$ s'exprime en fonction de $f^+_{I'}$~:
\begin{equation}
h_b=\displaystyle\frac{\phi_b}{f_{b,ext}-f_{I'}}=\frac{\rho\,C\,u_k}{f^+_{I'}}
\end{equation}

Pour d\'eterminer $h_b$, on int\`egre alors 
l'\'equation~(\ref{Eq_Flux_scalaire_adim}) afin de disposer de $f^+_{I'}$. 
L'unique difficult\'e consiste alors \`a prescrire une loi de variation de 
$\mathcal{K}=\displaystyle\frac{\nu}{\sigma}+
              \displaystyle\frac{\nu_t} {\sigma_t}$


Dans la zone turbulente pleinement d�velopp\'ee, une hypoth\`ese de 
longueur de m\'elange permet de mod\'eliser les variations de $\nu_t$~:
\begin{equation}
\nu_t = l^2 \arrowvert \frac{\partial U}{\partial z} \arrowvert = 
\kappa\,u^* \left(z + z_0\right)
\end{equation}
De plus, les effets de diffusion de $f$ 
(ou effets "conductifs" lorsque $f$ repr\'esente la temp\'erature) 
sont n\'egligeables devant les effets turbulents~: on n\'eglige alors  
$\displaystyle\frac{\nu}{\sigma}$ devant 
$\displaystyle\frac{\nu_t}{\sigma_t}$. On a donc 
finalement~:
\begin{equation}
\mathcal{K}= \displaystyle\frac{\kappa \,u_k}{\sigma_t}  \left(z+z_0\right)
\end{equation}


On int\`egre l'�quation adimensionnelle~(\ref{Eq_Flux_scalaire_adim}) 
sous la m\^eme hypoth\`ese et on obtient alors la loi donnant $f^+$~:
\begin{equation}
f^+ = \displaystyle\frac{\sigma_t}{\kappa}\,
        ln\left(\displaystyle\frac{z+z_0}{z_{o_T}}\right)
\end{equation}
o� $z_{o_T}$ est la longueur de rugosit� thermique. Son ordre de
grandeur compar� � la rugosit� dynamique  est donn� par la
relation $ln\left(\displaystyle\frac{z_0}{z_{o_T}}\right) \approx 2$ (r�f�rence J. R. Garratt).

Pour r\'esumer, le calcul de $h_b$ est r\'ealis\'e en d\'eterminant~:  
\begin{eqnarray}
f^+_{I'}&=& \displaystyle\frac{\sigma_t}{\kappa}\,
 ln\left(\displaystyle\frac{I'F+z_0}{z_{o_T}}\right) \\
h_b&=&\displaystyle\frac{\phi_b}{f_{b,ext}-f_{I'}}=\frac{\rho\,C\,u_k}{f^+_{I'}}
\end{eqnarray}


%%%%%%%%%%%%%%%%%%%%%%%%%%%%%%%%%%
%%%%%%%%%%%%%%%%%%%%%%%%%%%%%%%%%%
\section*{Mise en \oe uvre}
%%%%%%%%%%%%%%%%%%%%%%%%%%%%%%%%%%
%%%%%%%%%%%%%%%%%%%%%%%%%%%%%%%%%%

%\etape{Mode de prescription des conditions aux limites\vspace{0,3cm}}
%%%%%%%%%%%%%%%%%%%%%%%%%%%%%%%%%%%%%%%%%%%%%%%%%%%%%%%%%%%%%%%%%%%%
On traite ici les variables \var{IVAR} sur les faces \var{IFAC} 
telles que \var{ICODCL(IFAC,IVAR)}=5. 

La vitesse de d\'efilement (\'eventuellement nulle) de la paroi est tout d'abord
projet\'ee dans le plan tangent \`a la paroi. Ses trois composantes dans le
rep\`ere de calcul sont stock\'ees dans les
tableaux\\  \var{RCODCL(IFAC,IUIPH,1),RCODCL(IFAC,IVIPH,1)RCODCL(IFAC,IWIPH,1)}. 

On d\'etermine ensuite le
rep\`ere local $\hat{\mathcal R}$. Pour chaque face, il est disponible dans les
vecteurs $\vect{\tau}$ \var{(TX, TY, TZ)}, $\vect{\tilde{n}}$ \var{(-RNX, -RNY,
-RNZ)}, et $\vect{b}$ \var{(T2X, -T2Y,-T2Z)}. Par ailleurs, si la norme de la vitesse
tangentielle est inf\'erieure \`a la valeur arbitraire \var{EPZERO}
($10^{-12}$), l'indicateur \var{TXN0} est positionn\'e \`a 0 (il vaut 1 sinon)
et le vecteur $\vect{\tau}$ est pris 
\begin{itemize}
\item [-] arbitraire, dans le plan perpendiculaire \`a $\vect{\tilde{n}}$ en
$R_{ij}-\varepsilon$  (on utilise les composantes de $\vect{\tilde{n}}$ pour
construire  $\vect{\tau}$~; si  $\vect{\tilde{n}}$ est identiquement nul, le
code s'arr\^ete)~;
\item [-] identiquement nul sinon.
\end{itemize}

On calcule ensuite les vitesses de frottement qui sont stock\'ees dans
\var{UET} ($=u^*$) et dans \var{UK} ($=u_k$). Le sous-programme \fort{causta}
 permet de calculer la vitesse de 
frottement pour le mod\`ele \`a une \'echelle
de vitesse (\var{IDEUCH}=0). Pour le mod\`ele \`a deux \'echelles
(\var{IDEUCH}=1), le calcul (plus simple) de \var{UET} et de \var{UK} est fait directement
dans \fort{clptur}.

Dans le cas ou le mod\`ele \`a une \'echelle de vitesse est actif, on impose
\var{UK=UET} afin de conserver la coh\'erence du codage dans la suite du
sous-programme.

Les conditions aux limites pour la vitesse sont ensuite compl\'et\'ees.
\begin{itemize}

\item [-] En $k-\varepsilon$, on affecte, pour les trois composantes de vitesse
respectivement, \`a $$\var{COEFA(IFAC,ICLU)}, \var{COEFA(IFAC,ICLV)} \,\text{ et
}\,\var{COEFA(IFAC,ICLW)}$$ les coefficients $A_{flux}$  
issus de l'analyse relative \`a la contrainte tangentielle.\\
 De m\^eme, les coefficients $B_{flux}$ sont affect\'es \`a
$$\var{COEFB(IFAC,ICLU)}, \var{COEFB(IFAC,ICLV)} \,\text{ et
}\,\var{COEFB(IFAC,ICLW)}.$$ Les conditions aux limites
issues de l'analyse portant directement sur le gradient de vitesse $A_{grad}$ et
$B_{grad}$ sont affect\'ees \`a  $$\var{COEFA(IFAC,ICLUF)},
\var{COEFA(IFAC,ICLVF)}, \var{COEFA(IFAC,ICLWF)}$$ et
$$\var{COEFB(IFAC,ICLUF)}, \var{COEFB(IFAC,ICLVF)}, \var{COEFB(IFAC,ICLWF)}.$$ 

\item [-] Lorsque la vitesse tangentielle en $I'$ est inf\'erieure \`a
\var{EPZERO}, l'indicateur \var{TXN0} est annul\'e (sinon,  \var{TXN0} vaut 1). 
 Cet  indicateur est utilis\'e pour annuler les coefficients $A$ et
      imposer des conditions d'adh\'erence. 

\item [-] la vitesse de d\'efilement de la paroi est prise en compte dans les
conditions aux limites (coefficients \var{COEFA}).

\end{itemize}

Les conditions sur les grandeurs turbulentes sont ensuite
compl\'et\'ees. 


Les conditions aux limites pour les scalaires sont ensuite compl\'et\'ees. Les
coefficients \var{COEFA} et \var{COEFB} sont simplement renseign\'es en
utilisant les conditions aux limites d\'ecrites pr\'ec\'edemment. La seule
difficult\'e consiste \`a g\'erer correctement les diff\'erentes grandeurs
permettant de  calculer le coefficient d'\'echange $h_b$ sans erreur.

L'indicateur \var{ISCSTH} sert, pour chaque {\it VarScalaires} \`a indiquer
quelle valeur de $C$ utiliser 
au moment du traitement des conditions aux limites. Ainsi, pour \var{ISCSTH}=1,
la variable doit \^etre trait\'ee comme une temp\'erature, avec $C=C_p$. Pour
\var{ISCSTH}=0 ou 2, la variable doit \^etre trait\'ee comme un scalaire passif
ou une enthalpie respectivement, avec  $C=1$ (constante sans dimension) dans les deux cas. 

Par ailleurs, une valeur strictement positive de l'entier \var{IPCCP}
indique que  $C_p$ est variable en espace
et disponible dans le tableau \var{PROPCE(IEL,IPCCP )} (renseign\'e dans
\var{USPHYV}). Lorsque \var{IPCCP} est
nul, $C_p$ est constant et disponible sous
forme du  r\'eel \var{CP0(IPHAS)}.

L'indicateur \var{IHCP} permet de rassembler ces informations~: 
\begin{itemize}
\item [-] \var{IHCP} = 0 : \var{CPP} = $C=1$ 
\item [-] \var{IHCP} = 1 : \var{CPP} = $C=C_p$ uniforme en espace 
\item [-] \var{IHCP} = 2 : \var{CPP} = $C=C_p$ variable en espace 
\end{itemize}

Pour la {\it VarScalaire} \var{LL}, l'indicateur \var{IVISLS(LL)} 
permet \'egalement de rep\'erer si $\displaystyle\frac{\alpha_m}{C}$  
est variable en espace et disponible dans le tableau \var{PROPCE(IEL,IPCVSL)}
(\var{IVISLS} $> 0$) ou uniforme en espace et disponible sous forme du r\'eel
\var{VISLS0(LL)} (\var{IVISLS} $= 0$). On pose \var{RKL}$ =
\displaystyle\frac{\alpha_m}{C}$ 

Le nombre de Prandtl local est alors calcul\'e \var{PRDTL} $ =
\displaystyle\frac{\mu}{\alpha_m/C}$ ($\mu$ est la viscosit\'e dynamique mol\'eculaire
disponible dans  \var{VISCLC}).  

Le coefficient $h_{int}=\displaystyle\frac{\alpha}{\overline{I'F}}$  est ensuite
d\'etermin\'e et conserv\'e 
dans \var{HINT}. 

Lorsqu'un mod\`ele de turbulence est activ\'e, 
on calcule  \var{HFLUI} $ = h_b = \displaystyle\frac{\rho C u_k}{T^+}$.
Si le calcul est r\'ealis\'e en laminaire, on a 
simplement \var{HFLUI} = \var{HINT} ($ = h_b = h_{int}=\displaystyle\frac{\alpha}{\overline{I'F}}$)

Dans les cas o\`u l'on souhaite stocker le coefficient d'\'echange (couplage
avec SYRTHES), \var{HFLUI} est conserv\'e dans le tableau \var{HBORD}.
Dans les cas o\`u l'on utilise le module de rayonnement, \var{HFLUI} est
stock\'e  dans le tableau de travail \var{RA(IHCONV)}. 

On dispose alors de tous les \'elements pour calculer les coefficients $A$ et
$B$ (\var{COEFA} et \var{COEFB}) relatif \`a la variable trait\'ee. Noter pour
terminer les 
correspondances suivantes qui permettent de rapprocher le code source de la
relation (\ref{eq_fbint_clptur})~: 
\begin{eqnarray}
\var{HEXT} &=& h_{imp,ext}
\nonumber \\
\var{PIMP} &=& f_{imp,ext}
\nonumber \\
\var{HREDUI} &=&  h_r
\nonumber \\
 \var{HINT} &=&  h_{int}
\nonumber \\
\var{HFLUI}  &=& h_b 
\nonumber
\end{eqnarray}

%-------------------------------------------------------------------------------

% This file is part of Code_Saturne, a general-purpose CFD tool.
%
% Copyright (C) 1998-2020 EDF S.A.
%
% This program is free software; you can redistribute it and/or modify it under
% the terms of the GNU General Public License as published by the Free Software
% Foundation; either version 2 of the License, or (at your option) any later
% version.
%
% This program is distributed in the hope that it will be useful, but WITHOUT
% ANY WARRANTY; without even the implied warranty of MERCHANTABILITY or FITNESS
% FOR A PARTICULAR PURPOSE.  See the GNU General Public License for more
% details.
%
% You should have received a copy of the GNU General Public License along with
% this program; if not, write to the Free Software Foundation, Inc., 51 Franklin
% Street, Fifth Floor, Boston, MA 02110-1301, USA.

%-------------------------------------------------------------------------------

\programme{clsyvt}

\hypertarget{clsyvt}{}

\vspace{1cm}
%%%%%%%%%%%%%%%%%%%%%%%%%%%%%%%%%%
%%%%%%%%%%%%%%%%%%%%%%%%%%%%%%%%%%
\section*{Function}
%%%%%%%%%%%%%%%%%%%%%%%%%%%%%%%%%%
%%%%%%%%%%%%%%%%%%%%%%%%%%%%%%%%%%

The aim of this subroutine is to fill the arrays of boundary conditions
 (\var{COEFA} and \var{COEFB}) of the velocity and of the Reynolds stress tensor,
for the symmetry boundary faces.
These conditions are express relatively naturally in the local coordinate system
of the boundary face. The function of \fort{clsyvt} is then to transform these
natural boundary conditions (expressed in the local coordinate sytem) in the
general coordinate sytem, and then to possibly partly implicit them.

It should be noted that the part of the subroutine  \fort{clptur} (for the wall
boundary conditions) relative to the writing in the local coordinate system and to
the rotation is totally identical.

See the \doxygenfile{clsyvt_8f90.html}{programmers reference of the dedicated subroutine}
for further details.

%%%%%%%%%%%%%%%%%%%%%%%%%%%%%%%%%%
%%%%%%%%%%%%%%%%%%%%%%%%%%%%%%%%%%
\section*{Discretisation}
%%%%%%%%%%%%%%%%%%%%%%%%%%%%%%%%%%
%%%%%%%%%%%%%%%%%%%%%%%%%%%%%%%%%%

Figure \ref{Base_Clsyvt_fig_facesym} presents the notations used at the face.
The local coordinate system is defined from the normal at the face and
the velocity at $I'$:\\
$\bullet\ \displaystyle\vect{t}
=\frac{1}{|\vect{u}_{I',\tau}|}\vect{u}_{I',\tau}$ is the first
vector of the local coordinate system.\\
$\bullet\ \vect{\tilde{n}}=-\vect{n}$ is the second
vector of the local coordinate system.\\
$\bullet\ \vect{b}=\vect{t}\wedge\vect{\tilde{n}}=\vect{n}\wedge\vect{t}$
is the third
vector of the local coordinate system.\\

\begin{figure}[h]
\centerline{\includegraphics[width=8cm]{facesym}}
\caption{\label{Base_Clsyvt_fig_facesym}Definition of the vectors forming the local coordinate system.}
\end{figure}

Here, $\vect{n}$ is the normalized normal vector to the boundary face
in the sense of  \CS ({\em i.e.}
directed towards the outside of the computational domain)
and $\vect{u}_{I',\tau}$ is the projection of the velocity at $I'$
in the plane of the face:
$\vect{u}_{I',\tau}=\vect{u}_{I'}-(\vect{u}_{I'}.\vect{n})\vect{n}$.\\
If $\vect{u}_{I',\tau}=\vect{0}$, the direction of $\vect{t}$ in the plane
normal to $\vect{n}$ is irrelevant. thus it is defined as:
$\displaystyle\vect{t}=\frac{1}{\sqrt{n_y^2+n_z^2}}(n_z\vect{e}_y-n_y\vect{e}_z)$
where
$\displaystyle\vect{t}=\frac{1}{\sqrt{n_x^2+n_z^2}}(n_z\vect{e}_x-n_x\vect{e}_z)$
along the non-zero components of $\vect{n}$ (components in the global
coordinate system $(\vect{e}_x,\vect{e}_y,\vect{e}_z)$).


For sake of clarity, the following notations will be used:\\
$\bullet\ $The general coordinate system will write
$\mathcal{R}=(\vect{e}_x,\vect{e}_y,\vect{e}_z)$.\\
$\bullet\ $The local coordinate system will write
$\hat{\mathcal{R}}=(\vect{t},-\vect{n},\vect{b})=(\vect{t},\vect{\tilde{n}},\vect{b})$.\\
$\bullet\ $The matrices of the components of a vector $\vect{u}$
in the coordinate systems
$\mathcal{R}$ and $\hat{\mathcal{R}}$ will write
$\mat{U}$ and $\hat{\mat{U}}$, respectively.\\
$\bullet\ $The matrices of the components of a tensor $\tens{R}$ (2$^{nd}$ order)
in the coordinate systems $\mathcal{R}$ and $\hat{\mathcal{R}}$ will write
 $\matt{R}$ and $\hat{\matt{R}}$, respectively.\\
$\bullet\ $ $\matt{P}$ refers to the (orthogonal) matrix transforming
 $\mathcal{R}$ into $\hat{\mathcal{R}}$.
\begin{equation}
\matt{P}=\left[
\begin{array}{ccc}
t_x & -n_x & b_x\\
t_y & -n_y & b_y\\
t_z & -n_z & b_z
\end{array}\right]
\end{equation}

($\matt{P}$ being orthogonal, $\matt{P}^{-1}=\,^t\matt{P}$).

In particular, we have for any vector $\vect{u}$
and for any second order tensor $\tens{R}$:\\
\begin{equation}
\left\{\begin{array}{l}
\mat{U} = \matt{P}\,.\,\hat{\mat{U}}\\
\matt{R}= \matt{P}\,.\,\hat{\matt{R}}\,.\,^t\matt{P}
\end{array}\right.
\end{equation}

\minititre{Treatment of the velocity}
In the local coordinate system, the boundary conditions for $\vect{u}$
naturally write:\\
\begin{equation}
\left\{\begin{array}{lcl}
u_{F,t} & = & u_{I',t}\\
u_{F,\tilde{n}} & = & 0\\
u_{F,b} & = & u_{I',b}
\end{array}\right.
\end{equation}
or
\begin{equation}
\mat{U}_F = \matt{P}\,.\,\hat{\mat{U}}_F
= \matt{P}\,.\,\left[
\begin{array}{ccc}
1 & 0 & 0\\
0 & 0 & 0\\
0 & 0 & 1
\end{array}
\right]\,.\,\hat{\mat{U}}_{I'}
=\matt{P}\,.\,\left[
\begin{array}{ccc}
1 & 0 & 0\\
0 & 0 & 0\\
0 & 0 & 1
\end{array}
\right]\,.\,^t\matt{P}\,.\,\mat{U}_{I'}
\end{equation}


Let's take
$\matt{A}=\matt{P}\,.\,\left[
\begin{array}{ccc}
1 & 0 & 0\\
0 & 0 & 0\\
0 & 0 & 1
\end{array}
\right]\,.\,^t\matt{P}\qquad$ (matrix in the coordinate system $\mathcal{R}$
of the projector orthogonal to the face).

The boundary conditions for  $\vect{u}$ thus write:
\begin{equation}
\mat{U}_F = \matt{A}\,.\,\mat{U}_{I'}
\end{equation}

Since the matrix $\matt{P}$ is orthogonal, it can be shown that
\begin{equation}
\matt{A}=\left[
\begin{array}{ccc}
1-\tilde{n}_x^2 & -\tilde{n}_x\tilde{n}_y & -\tilde{n}_x\tilde{n}_z\\
-\tilde{n}_x\tilde{n}_y & 1-\tilde{n}_y^2 & -\tilde{n}_y\tilde{n}_z\\
-\tilde{n}_x\tilde{n}_z & -\tilde{n}_y\tilde{n}_z & 1-\tilde{n}_z^2
\end{array}\right]
\end{equation}

The boundary conditions can then be partially implicited:
\begin{equation}
\label{Base_Clsyvt_eq_clU}
u_{F,x}^{(n+1)} = \underbrace{1-\tilde{n}_x^2}_{\var{COEFB}}u_{I',x}^{(n+1)}
\underbrace{-\tilde{n}_x\tilde{n}_y u_{I',y}^{(n)}
-\tilde{n}_x\tilde{n}_z u_{I',z}^{(n)}}_{\var{COEFA}}
\end{equation}

The other components have a similar treatment. Since only the coordinates
of $\vect{n}$ are useful, we do not need (for $\vect{u}$) to define
explicitly the vectors  $\vect{t}$ and $\vect{b}$.

\vspace{1cm}
\minititre{Treatment of the Reynolds stress tensor}
We saw that we have the following relation:
\begin{equation}
\label{Base_Clsyvt_eq_chgtrepR}
\matt{R}= \matt{P}\,.\,\hat{\matt{R}}\,.\,^t\matt{P}
\end{equation}

The boundary conditions we want to write are relations of the type:
\begin{equation}
\comp{R}_{F,ij}=\sum_{k,l}\alpha_{ijkl}\comp{R}_{I',kl}
\end{equation}
We are then naturally brought to introduce the column matrices of the
components of $\tens{R}$ in the different coordinate systems.

We write
\begin{equation}
\mat{S}=\,^t[\comp{R}_{11},\comp{R}_{12},\comp{R}_{13},
\comp{R}_{21},\comp{R}_{22},\comp{R}_{23},
\comp{R}_{31},\comp{R}_{32},\comp{R}_{33}]
\end{equation}
and
\begin{equation}
\hat{\mat{S}}=\,^t[\hat{\comp{R}}_{11},\hat{\comp{R}}_{12},\hat{\comp{R}}_{13},
\hat{\comp{R}}_{21},\hat{\comp{R}}_{22},\hat{\comp{R}}_{23},
\hat{\comp{R}}_{31},\hat{\comp{R}}_{32},\hat{\comp{R}}_{33}]
\end{equation}

Two functions $q$ and $r$ from $\{1,2,3,4,5,6,7,8,9\}$ to
$\{1,2,3\}$ are defined. Their values are given in the following table:
\begin{center}
\begin{tabular}{|c|c|c|c|c|c|c|c|c|c|}
\hline
$i$&1&2&3&4&5&6&7&8&9\\
\hline
$q(i)$&1&1&1&2&2&2&3&3&3\\
\hline
$r(i)$&1&2&3&1&2&3&1&2&3\\
\hline
\end{tabular}
\end{center}
$i\longmapsto (q(i),r(i))$ is then a bijection from $\{1,2,3,4,5,6,7,8,9\}$
to $\{1,2,3\}^2$, and we have:
\begin{equation}
\left\{\begin{array}{l}
\comp{R}_{ij}=\comp{S}_{3(i-1)+j}\\
\comp{S}_i=\comp{R}_{q(i)r(i)}
\end{array}\right.
\end{equation}

Using equation \ref{Base_Clsyvt_eq_chgtrepR}, we thus have:
\begin{eqnarray}
\comp{S}_{F,i} & = & \comp{R}_{F,q(i)r(i)} =
\sum_{(m,n)\in\{1,2,3\}^2}\comp{P}_{q(i)m}\hat{\comp{R}}_{F,mn}\comp{P}_{r(i)n}\nonumber\\
&=&\sum_{j=1}^9\comp{P}_{q(i)q(j)}\hat{\comp{R}}_{F,q(j)r(j)}\comp{P}_{r(i)r(j)}
\quad\text{(because $i->(q(i),r(i))$ is bijective)}\nonumber\\
&=&\sum_{j=1}^9\comp{P}_{q(i)q(j)}\comp{P}_{r(i)r(j)}\hat{\comp{S}}_{F,j}
\end{eqnarray}

Or
\begin{equation}
\mat{S}_{F}=\matt{A}\,.\,\hat{\mat{S}}_F\quad\text{ with }
\comp{A}_{ij}=\comp{P}_{q(i)q(j)}\comp{P}_{r(i)r(j)}
\end{equation}

It can be shown that $\matt{A}$ is an orthogonal matrix (see Annexe A).

In the local coordinate system, the boundary conditions of $\tens{R}$
write naturally\footnote{cf. Davroux A., Archambeau F., {\em Le
$R_{ij}-\varepsilon$ dans \CS (version $\beta$)}, HI-83/00/030/A}.
\begin{equation}
\label{Base_Clsyvt_eq_clRij}%
\left\{\begin{array}{lll}
\hat{\comp{R}}_{F,11}=\hat{\comp{R}}_{I',11} \qquad\qquad&
\hat{\comp{R}}_{F,21}=0 \qquad\qquad&
\hat{\comp{R}}_{F,31}=B\hat{\comp{R}}_{I',31} \\
\hat{\comp{R}}_{F,12}=0 \qquad\qquad&
\hat{\comp{R}}_{F,22}=\hat{\comp{R}}_{I',22} \qquad\qquad&
\hat{\comp{R}}_{F,32}=0 \\
\hat{\comp{R}}_{F,13}=B\hat{\comp{R}}_{I',13} \qquad\qquad&
\hat{\comp{R}}_{F,23}=0 \qquad\qquad&
\hat{\comp{R}}_{F,33}=\hat{\comp{R}}_{I',33}
\end{array}\right.
\end{equation}

or
\renewcommand{\arraystretch}{0.5}
\begin{equation}
\hat{\mat{S}}_F=\matt{B}\,.\,\hat{\mat{S}}_{I'}
\qquad\text{avec }\matt{B}=
\left[\begin{array}{ccccccccc}
1&0&\cdots&\cdots&\cdots&\cdots&\cdots&\cdots&0\\
0&0&\ddots&&&&&&\vdots\\
\vdots&\ddots&B&\ddots&&&&&\vdots\\
\vdots&&\ddots&0&\ddots&&&&\vdots\\
\vdots&&&\ddots&1&\ddots&&&\vdots\\
\vdots&&&&\ddots&0&\ddots&&\vdots\\
\vdots&&&&&\ddots&B&\ddots&\vdots\\
\vdots&&&&&&\ddots&0&0\\
0&\cdots&\cdots&\cdots&\cdots&\cdots&\cdots&0&1
\end{array}\right]
\end{equation}
\renewcommand{\arraystretch}{1.}


For the symmetry faces which are treated by \fort{clsyvt}, the
coefficient $B$ is 1. However a similar treatment is applied in
\fort{clptur} for the wall faces, and in this $B$ is zero. This
parameter has to be specified when \code{cs\_turbulence\_bc\_rij\_transform} is called
(see \S\ref{Base_Clsyvt_prg_meo}).

Back in the global coordinate system, the following formulae is
finally obtained:
\begin{equation}
\label{eq:base:clsyvt:cl_s}
\mat{S}_F=\matt{C}\,.\,\mat{S}_{I'}\qquad
\text{avec }\matt{C}=\matt{A}\,.\,\matt{B}\,.\,^t\matt{A}
\end{equation}

It can be shown that the components of the matrix $\matt{C}$ are:
\begin{equation}
\comp{C}_{ij}=\sum_{k=1}^9
\comp{P}_{q(i)q(k)}\comp{P}_{r(i)r(k)}\comp{P}_{q(j)q(k)}\comp{P}_{r(j)r(k)}
(\delta_{k1}+B\delta_{k3}+\delta_{k5}+B\delta_{k7}+\delta_{k9})
\end{equation}

To conclude, it can be noted thatthe tensor  $\tens{R}$ is symmetric.
Thus only the simplified matrices (stored by increasing diagonal)
$\mat{S}^\prime$ and $\hat{\mat{S}}^\prime$ will be used:
\begin{equation}
\mat{S}^\prime=\,^t[\comp{R}_{11},\comp{R}_{22},\comp{R}_{33},
\comp{R}_{12},\comp{R}_{23},\comp{R}_{13}]
\end{equation}
\begin{equation}
\hat{\mat{S}}^\prime=\,^t[\hat{\comp{R}}_{11},\hat{\comp{R}}_{22},\hat{\comp{R}}_{33},
\hat{\comp{R}}_{12},\hat{\comp{R}}_{23},\hat{\comp{R}}_{13}]
\end{equation}

By gathering different lines of matrix $\matt{C}$, equation
\eqref{eq:base:clsyvt:cl_s} is transformed into the final equation:
\begin{equation}
\mat{S}_F^\prime=\matt{D}\,.\,\mat{S}_{I'}^\prime
\end{equation}

The computation of the matrix $\matt{D}$ is performed in the subroutine
\code{cs\_turbulence\_bc\_rij\_transform}. The methodology is described in annexe B.

From $\matt{D}$, the coefficients of the boundary conditions can be
 partially implicited ($\var{ICLSYR}=1$) or totally explicited
($\var{ICLSYR}=0$).

$\bullet\ ${\sc Partial implicitation}\\
\begin{equation}
\label{Base_Clsyvt_eq_clRimp}
{\comp{S}_{F,i}^\prime}^{(n+1)} =
\underbrace{\comp{D}_{ii}}_{\var{COEFB}}{\comp{S}_{I',i}^\prime}^{(n+1)}
+\underbrace{\sum_{j\ne i}\comp{D}_{ij}{\comp{S}_{I',j}^\prime}^{(n)}}_{\var{COEFA}}
\end{equation}

$\bullet\ ${\sc Total explicitation}\\
\begin{equation}
\label{Base_Clsyvt_eq_clRexp}
{\comp{S}_{F,i}^\prime}^{(n+1)} =
\underbrace{\sum_j\comp{D}_{ij}{\comp{S}_{I',j}^\prime}^{(n)}}_{\var{COEFA}}
\qquad(\var{COEFB}=0)
\end{equation}

%%%%%%%%%%%%%%%%%%%%%%%%%%%%%%%%%%
%%%%%%%%%%%%%%%%%%%%%%%%%%%%%%%%%%
\section*{Implementation}
%%%%%%%%%%%%%%%%%%%%%%%%%%%%%%%%%%
%%%%%%%%%%%%%%%%%%%%%%%%%%%%%%%%%%
\label{Base_Clsyvt_prg_meo}%
\etape{Beginning of the loop}

Beginnning of the loop on all the boundary faces  \var{IFAC} with
symmetry conditions.
A face is considered as a symmetry face if
\var{ICODCL(IFAC,IU)} is 4. The tests in \fort{vericl} are designed
 for  \var{ICODCL} to be equal to 4 for \var{IU} if and only if it is equal
to 4 for the other components of the velocity and for the components of $\tens{R}$
(if necessary)\\.

The value 0 is given to  \var{ISYMPA}, which identifies the face as a
wall or symmetry face (a face where the mass flux will be set to zero as
explained in \fort{inimas}).

\etape{Calculation of the basis vectors}
The normal vector $\vect{n}$ is stored in \var{(RNX,RNY,RNZ)}.\\
$\vect{u}_{I'}$ is calculated in \fort{CONDLI}, passed {\em via} \var{COEFU},
and stored in \var{(UPX,UPY,UPZ)}.

\etape{Case with $R_{ij}-\varepsilon$}
With the $R_{ij}-\varepsilon$ model (\var{ITURB}=30 or 31), the vectors
 $\vect{t}$ and $\vect{b}$ must be calculated explicitly
(we use $\matt{P}$, not simply $\matt{A}$).
They are stored in \var{(TX,TY,TZ)} and
\var{(T2X,T2Y,T2Z)}, respectively.\\
The transform matrix $\matt{P}$ is then calculated and stored in the array \var{ELOGLO}.\\
The subroutine \code{cs\_turbulence\_bc\_rij\_transform} is then called to calculate the reduced
matrix $\matt{D}$. It is stored in \var{ALPHA}. \code{cs\_turbulence\_bc\_rij\_transform} is called
with a parameter \var{CLSYME} which is 1, and which corresponds to the parameter
$B$ of equation \ref{Base_Clsyvt_eq_clRij}.


\etape{Filling the arrays \var{COEFA} and \var{COEFB}}
The arrays \var{COEFA} and \var{COEFB} are filled following directly
equations \ref{Base_Clsyvt_eq_clU}, \ref{Base_Clsyvt_eq_clRimp} and \ref{Base_Clsyvt_eq_clRexp}.\\
\var{RIJIPB(IFAC,.)} corresponds to the vector $\mat{S}_{I'}^\prime$, computed in
\fort{condli}, and passed as an argument to \var{clsyvt}.

\etape{Filling the arrays \var{COEFAF} and \var{COEFBF}}
If they are defined, the arrays \var{COEFAF} and \var{COEFBF}
are filled. They contain the same values as \var{COEFA} and
\var{COEFB}, respectively.


%%%%%%%%%%%%%%%%%%%%%%%%%%%%%%%%%%
%%%%%%%%%%%%%%%%%%%%%%%%%%%%%%%%%%
\section*{Annexe A}
%%%%%%%%%%%%%%%%%%%%%%%%%%%%%%%%%%
%%%%%%%%%%%%%%%%%%%%%%%%%%%%%%%%%%
\minititre{Proof of the orthogonality of matrix $\matt{A}$}

All the notations used in paragraphe 2 are kept. We have:
\begin{eqnarray}
(^t\matt{A}\,.\,\matt{A})_{ij}
& = & \sum_{k=1}^9\comp{A}_{ki}\comp{A}_{kj}\nonumber\\
& = & \sum_{k=1}^9
\comp{P}_{q(k)q(i)}\comp{P}_{r(k)r(i)}\comp{P}_{q(k)q(j)}\comp{P}_{r(k)r(j)}
\end{eqnarray}
When $k$ varies from 1 to 3, $q(k)$ remains equal to 1 and $r(k)$ varies from 1
to 3. We thus have:
\begin{eqnarray}
\sum_{k=1}^3
\comp{P}_{q(k)q(i)}\comp{P}_{r(k)r(i)}\comp{P}_{q(k)q(j)}\comp{P}_{r(k)r(j)}
&=&\comp{P}_{1q(i)}\comp{P}_{1q(j)}\sum_{k=1}^3
\comp{P}_{r(k)r(i)}\comp{P}_{r(k)r(j)}\nonumber\\
&=&\comp{P}_{1q(i)}\comp{P}_{1q(j)}\sum_{k=1}^3
\comp{P}_{kr(i)}\comp{P}_{kr(j)}\\
&=&\comp{P}_{1q(i)}\comp{P}_{1q(j)}\delta_{r(i)r(j)}\qquad\text{(by
orthogonality of $\matt{P}$)}\nonumber
\end{eqnarray}
Likewise for $k$ varying from 4 to 6 or from 7 to 9, $q(k)$ being 2 or 3, respectively,
 we obtain:
\begin{eqnarray}
(^t\matt{A}\,.\,\matt{A})_{ij}
&=&
\sum_{k=1}^9
\comp{P}_{q(k)q(i)}\comp{P}_{r(k)r(i)}\comp{P}_{q(k)q(j)}\comp{P}_{r(k)r(j)}
\nonumber\\
&=&
\sum_{k=1}^3\comp{P}_{kq(i)}\comp{P}_{kq(j)}\delta_{r(i)r(j)}\\
&=&\delta_{q(i)q(j)}\delta_{r(i)r(j)}\nonumber\\
&=&\delta_{ij}\qquad
\text{(by the bijectivity of $(q,r)$)}\nonumber
\end{eqnarray}

Thus $^t\matt{A}\,.\,\matt{A}=\matt{Id}$. Similarly, it can be shown that
$\matt{A}\,.\,^t\matt{A}=\matt{Id}$. Thus $\matt{A}$ is an orthogonal matrix.


%%%%%%%%%%%%%%%%%%%%%%%%%%%%%%%%%%
%%%%%%%%%%%%%%%%%%%%%%%%%%%%%%%%%%
\section*{Annexe B}
%%%%%%%%%%%%%%%%%%%%%%%%%%%%%%%%%%
%%%%%%%%%%%%%%%%%%%%%%%%%%%%%%%%%%
\minititre{Calculation of the matrix $\matt{D}$}

The relation between the matrices of dimension $9\times1$
of the components of $\tens{R}$ in the coordinate system $\mathcal{R}$ at $F$ and at $I'$
(matrices $\mat{S}_F$ and $\mat{S}_{I'}$):
\begin{equation}
\mat{S}_F=\matt{C}\,.\,\mat{S}_{I'}
\end{equation}
with
\begin{equation}
\comp{C}_{ij}=\sum_{k=1}^9
\comp{P}_{q(i)q(k)}\comp{P}_{r(i)r(k)}\comp{P}_{q(j)q(k)}\comp{P}_{r(j)r(k)}
(\delta_{k1}+B \delta_{k3}+\delta_{k5}+B \delta_{k7}+\delta_{k9})
\end{equation}

To transform $\mat{S}$ into the matrix $6\times 1$ $\mat{S}^\prime$,
the function $s$ from $\{1,2,3,4,5,6,7,8,9\}$ to
$\{1,2,3,4,5,6\}$ is defined. It takes the following values:
\begin{center}
\begin{tabular}{|c|c|c|c|c|c|c|c|c|c|}
\hline
$i$&1&2&3&4&5&6&7&8&9\\
\hline
$s(i)$&1&4&6&4&2&5&6&5&3\\
\hline
\end{tabular}
\end{center}
By construction, we have $\comp{S}_i=\comp{S}^\prime_{s(i)}$ for all $i$ between 1
and 9.

To compute $\comp{D}_{ij}$, we can choose a value of $m$ to satisfy
$s(m)=i$ and sum all the $\comp{C}_{mn}$ to have $s(n)=j$. The choice of $m$
is irrelevent. We can also compute the sum over all $m$ so that  $s(m)=i$ and then
divide by the number of such values of $m$. We will use this method.

We define $N(i)$ the number of integers $m$ between 1 and 9 so that $s(m)=i$ ($N(i)$ is equal to $1$ or $2$).
According to what preceeds we have

\begin{eqnarray}
\comp{D}_{ij}&=&\frac{1}{N(i)}\sum_{s(m)=i \atop s(n)=j}\comp{C}_{mn}\\
&=&\frac{1}{N(i)}\sum_{{s(m)=i \atop s(n)=j}\atop 1\leqslant k\leqslant 9}
\comp{P}_{q(m)q(k)}\comp{P}_{r(m)r(k)}\comp{P}_{q(n)q(k)}\comp{P}_{r(n)r(k)}
(\delta_{k1}+B \delta_{k3}+\delta_{k5}+B\delta_{k7}+\delta_{k9})\nonumber
\end{eqnarray}

\vspace{1cm}
$\bullet\ ${\sc First case}: $i\leqslant 3$ and $j\leqslant 3$\\
In this case, we have $N(i)=N(j)=1$. Additionally, if $s(m)=i$ and $s(n)=j$,
then $q(m)=r(m)=i$ and $q(n)=r(n)=j$. Thus \\
\begin{equation}
\comp{D}_{ij}=\sum_{k=1}^9
\comp{P}_{iq(k)}\comp{P}_{ir(k)}\comp{P}_{jq(k)}\comp{P}_{jr(k)}
(\delta_{k1}+ B\delta_{k3}+\delta_{k5}+ B\delta_{k7}+\delta_{k9})
\end{equation}
When $k$ belongs to $\{1,5,9\}$, $q(k)=r(k)$ belongs to $\{1,2,3\}$. And for $k=3$ or
$k=7$, $q(k)=1$ and $r(k)=3$, or the inverse (and for $k$ even the
the sum of Kronecker symbol is zero). Finally we have:
\begin{equation}
\comp{D}_{ij}=\sum_{k=1}^3\comp{P}_{ik}^2\comp{P}_{jk}^2
+2 B\comp{P}_{j1}\comp{P}_{i3}\comp{P}_{i1}\comp{P}_{j3}
\end{equation}

\vspace{1cm}
$\bullet\ ${\sc Second case}: $i\leqslant 3$ and $j\geqslant 4$\\
Again we have $N(i)=1$, and if $s(m)=i$ then $q(m)=r(m)=i$.\\
On the contrary, we have $N(j)=2$, the two possibilities being $m_1$ and $m_2$.\\
\begin{itemize}
\item[-] if $j=4$, then $m_1=2$ and $m_2=4$,
$q(m_1)=r(m_2)=1$ and $r(m_1)=q(m_2)=2$. We then have
$m=1$ and $n=2$.

\item[-] if $j=5$, then $m_1=6$ and $m_2=8$,
$q(m_1)=r(m_2)=1$ and $r(m_1)=q(m_2)=3$. We then have
$m=1$ and $n=3$.

\item[-] if $j=6$, then $m_1=3$ and $m_2=7$,
$q(m_1)=r(m_2)=2$ and $r(m_1)=q(m_2)=3$. We then have
$m=2$ and $n=3$.
\end{itemize}

And we have:
\begin{equation}
\comp{D}_{ij}=\sum_{k=1}^9
\comp{P}_{iq(k)}\comp{P}_{ir(k)}\left[
\comp{P}_{mq(k)}\comp{P}_{nr(k)}+\comp{P}_{nq(k)}\comp{P}_{mr(k)}\right]
(\delta_{k1}+B \delta_{k3}+\delta_{k5}+B \delta_{k7}+\delta_{k9})
\end{equation}

But when $k$ is in $\{1,5,9\}$, $q(k)=r(k)$ is in $\{1,2,3\}$. Thus:
\begin{equation}
\comp{D}_{ij}=2\sum_{k=1}^3
\comp{P}_{ik}^2\comp{P}_{mk}\comp{P}_{nk}
+B \sum_{k=1}^9
\comp{P}_{iq(k)}\comp{P}_{ir(k)}\left[
\comp{P}_{mq(k)}\comp{P}_{nr(k)}+\comp{P}_{nq(k)}\comp{P}_{mr(k)}\right]
(\delta_{k3}+\delta_{k7})
\end{equation}

And for $k=3$ or $k=7$, $q(k)=1$ and $r(k)=3$, or the opposite. We finally have:
\begin{equation}
\comp{D}_{ij}=2\left[\sum_{k=1}^3
\comp{P}_{ik}^2\comp{P}_{mk}\comp{P}_{nk}
+B\comp{P}_{i1}\comp{P}_{i3}\left(
\comp{P}_{m1}\comp{P}_{n3}+\comp{P}_{n1}\comp{P}_{m3}\right)
\right]
\end{equation}
with $(m,n)=(1,2)$ if $j=4$, $(m,n)=(1,3)$ if $j=5$ and $(m,n)=(2,3)$ if
$j=6$.

\vspace{1cm}
$\bullet\ ${\sc Third case}: $i\geqslant 4$ and $j\leqslant 3$\\
By symmetry of $\matt{C}$, we obtain a result which is the symmetric of
the second case, except that $N(i)$ is now 2. Thus:
\begin{equation}
\comp{D}_{ij}=\sum_{k=1}^3
\comp{P}_{jk}^2\comp{P}_{mk}\comp{P}_{nk}
+B\comp{P}_{j1}\comp{P}_{j3}\left(
\comp{P}_{m1}\comp{P}_{n3}+\comp{P}_{n1}\comp{P}_{m3}\right)
\end{equation}
with $(m,n)=(1,2)$ if $i=4$, $(m,n)=(2,3)$ if $i=5$ and $(m,n)=(1,3)$ if
$i=6$.

\vspace{1cm}
$\bullet\ ${\sc Fourth case}: $i\geqslant 4$ and $j\geqslant 4$\\
Then $N(i)=N(j)=2$.\\
We have $(m,n)=(1,2)$ if $i=4$, $(m,n)=(2,3)$ if $i=5$ and
$(m,n)=(1,3)$ if $i=6$. Likewise we define $m^\prime$ and
$n^\prime$ as a function of $j$. We then obtain:
\begin{eqnarray}
\comp{D}_{ij}&=&\frac{1}{2}\sum_{k=1}^9
\left(\comp{P}_{mq(k)}\comp{P}_{nr(k)}+\comp{P}_{nq(k)}\comp{P}_{mr(k)}\right)
\left(\comp{P}_{m^\prime q(k)}\comp{P}_{n^\prime r(k)}
+\comp{P}_{n^\prime q(k)}\comp{P}_{m^\prime r(k)}\right)\nonumber\\
&&\qquad\qquad\qquad\qquad\qquad\qquad\qquad\qquad\qquad\times
(\delta_{k1}+B \delta_{k3}+\delta_{k5}+B \delta_{k7}+\delta_{k9})\nonumber\\
&=&\frac{1}{2}\left[
\sum_{k=1}^3 4\comp{P}_{mk}\comp{P}_{nk}
\comp{P}_{m^\prime k}\comp{P}_{n^\prime k}
+2 B \left(\comp{P}_{m1}\comp{P}_{n3}+\comp{P}_{n1}\comp{P}_{m3}\right)
\left(\comp{P}_{m^\prime 1}\comp{P}_{n^\prime 3}
+\comp{P}_{n^\prime 1}\comp{P}_{m^\prime 3}\right)\right]
\end{eqnarray}

and finally:
\begin{equation}
\comp{D}_{ij}=
2\sum_{k=1}^3  \comp{P}_{mk}\comp{P}_{nk}
\comp{P}_{m^\prime k}\comp{P}_{n^\prime k}
+B \left(\comp{P}_{m1}\comp{P}_{n3}+\comp{P}_{n1}\comp{P}_{m3}\right)
\left(\comp{P}_{m^\prime 1}\comp{P}_{n^\prime 3}
+\comp{P}_{n^\prime 1}\comp{P}_{m^\prime 3}\right)
\end{equation}
with $(m,n)=(1,2)$ if $i=4$, $(m,n)=(2,3)$ if $i=5$ and $(m,n)=(1,3)$ if
$i=6$\\
and $(m^\prime ,n^\prime )=(1,2)$ if $j=4$, $(m^\prime ,n^\prime )=(2,3)$
if $j=5$ and $(m^\prime ,n^\prime )=(1,3)$ if $j=6$.

%-------------------------------------------------------------------------------

% This file is part of Code_Saturne, a general-purpose CFD tool.
%
% Copyright (C) 1998-2013 EDF S.A.
%
% This program is free software; you can redistribute it and/or modify it under
% the terms of the GNU General Public License as published by the Free Software
% Foundation; either version 2 of the License, or (at your option) any later
% version.
%
% This program is distributed in the hope that it will be useful, but WITHOUT
% ANY WARRANTY; without even the implied warranty of MERCHANTABILITY or FITNESS
% FOR A PARTICULAR PURPOSE.  See the GNU General Public License for more
% details.
%
% You should have received a copy of the GNU General Public License along with
% this program; if not, write to the Free Software Foundation, Inc., 51 Franklin
% Street, Fifth Floor, Boston, MA 02110-1301, USA.

%-------------------------------------------------------------------------------

\programme{codits}\label{ap:codits}
%

\vspace{1cm}
%-------------------------------------------------------------------------------
\section*{Fonction}
%-------------------------------------------------------------------------------
Ce sous-programme, appel\'{e} entre autre par \fort{predvv}, \fort{turbke}, \fort{covofi},
\fort{resrij}, \fort{reseps}, ..., r\'{e}sout les \'{e}quations de convection-diffusion
d'un scalaire $a$ avec termes sources du type :
\begin{equation}\label{Base_Codits_eq_ref}
\begin{array}{c}
\displaystyle f_s^{\,imp} (a^{n+1} - a^{n}) +
\theta \ \underbrace{\dive((\rho \underline{u})\,a^{n+1})}_{\text{convection implicite}}
-\theta \ \underbrace{\dive(\mu_{\,tot}\,\grad a^{n+1})}_{\text{diffusion implicite}}
\\\\
= f_s^{\,exp}-(1-\theta) \ \underbrace{\dive((\rho \underline{u})\,a^{n})}_{\text{convection explicite}}
 + (1-\theta) \ \underbrace{\dive(\mu_{\,tot}\,\grad a^{n})}_{\text{diffusion explicite}}
\end{array}
\end{equation}
o\`{u} $\rho \underline{u}$, $f_s^{exp}$ et $f_s^{imp}$ d\'{e}signent respectivement le flux de masse, les termes sources explicites et les termes lin\'{e}aris\'{e}s en $a^{n+1}$.
$a$ est un scalaire d\'{e}fini sur toutes les cellules\footnote{$a$, sous forme discr\`ete en espace, correspond \`a un vecteur dimensionn\'e \`a \var{NCELET} de composante $a_I$, I d\'ecrivant l'ensemble des cellules.}.
Par souci de clart\'{e} on suppose, en l'absence d'indication, les propri\'{e}tes
physiques $\Phi$ (viscosit\'{e} totale $\mu_{tot}$,...) et le flux de masse $(\rho
\underline{u})$ pris respectivement aux instants $n+\theta_\Phi$ et
$n+\theta_F$, o\`{u} $\theta_\Phi$ et $\theta_F$ d\'{e}pendent des sch\'{e}mas en temps
sp\'{e}cifiquement utilis\'{e}s pour ces grandeurs\footnote{cf. \fort{introd}}.
\\
L'\'{e}criture des termes de convection et diffusion en maillage non orthogonal
engendre des difficult\'{e}s (termes de reconstruction et test de pente) qui sont
contourn\'{e}es en utilisant une m\'ethode it\'erative dont la limite, si elle
existe, est la solution de l'\'{e}quation pr\'{e}c\'{e}dente.

%%%%%%%%%%%%%%%%%%%%%%%%%%%%%%%%%%
%%%%%%%%%%%%%%%%%%%%%%%%%%%%%%%%%%
\section*{Discr\'etisation}
%%%%%%%%%%%%%%%%%%%%%%%%%%%%%%%%%%
%%%%%%%%%%%%%%%%%%%%%%%%%%%%%%%%%%
Afin d'expliquer la proc\'{e}dure utilis\'{e}e pour traiter les difficult\'{e}s dues aux
termes de reconstruction et de test de pente dans les termes de
convection-diffusion, on note, de fa\c con analogue \`a ce qui est d\'efini dans
\fort{navstv} mais sans discr\'etisation spatiale associ\'ee, $\mathcal{E}_{n}$ l'op\'erateur :
\begin{equation}\label{Base_Codits_Eq_ref_small}
\begin{array}{c}
\mathcal{E}_{n}(a) = f_s^{\,imp}\,a + \theta\,\, \dive((\rho
\underline{u})\,a) - \theta\,\, \dive(\mu_{\,tot}\,\grad a)\\
- f_s^{\,exp} -  f_s^{\,imp}\,a^{n} +(1-\theta)\,\,\dive((\rho
\underline{u})\, a^n) - (1-\theta)\,\, \dive(\mu_{\,tot}\,\grad a^n)
\end{array}
\end{equation}
L'\'equation (\ref{Base_Codits_eq_ref}) s'\'ecrit donc :
\begin{equation}
\mathcal{E}_{n}(a^{n+1}) = 0
\end{equation}
La quantit\'e  $\mathcal{E}_{n}(a^{n+1})$ comprend donc :\\
\hspace*{1.cm} $\rightsquigarrow$ $f_s^{\,imp}\,a^{n+1}$, contribution des
termes diff\'erentiels d'ordre $0$ lin\'eaire en $a^{n+1}$,\\
\hspace*{1.cm} $\rightsquigarrow$ $\theta\,\,
\dive((\rho\underline{u})\,a^{n+1})
- \theta\,\, \dive(\mu_{\,tot}\,\grad a^{n+1})$, termes de convection-diffusion
implicites complets (termes non reconstruits + termes de reconstruction)
lin\'eaires\footnote{Lors de la discr\'{e}tisation en espace, le caract\`{e}re lin\'{e}aire
de ces termes pourra cependant �tre perdu, notamment \`{a} cause du test de pente.}
en $a^{n+1}$,\\
\hspace*{1.cm} $\rightsquigarrow$ $f_s^{\,exp}- f_s^{\,imp}\,a^n$ et
$(1-\theta)\,\,\dive((\rho
\underline{u})\,a^n) - (1-\theta)\,\, \dive(\mu_{\,tot}\,\grad a^n)$ l'ensemble
des termes explicites (y compris la partie explicite provenant du sch\'ema en
temps appliqu\'e \`a la convection diffusion).\\\\

De m\^eme, on introduit un op\'erateur $\mathcal{EM}_{n}$ approch\'e de
$\mathcal{E}_{n}$, lin\'eaire et simplement inversible, tel que son
expression contient :\\
\hspace*{1.cm}$\rightsquigarrow$ la prise en compte des termes lin\'eaires en $a$,\\
\hspace*{1.cm}$\rightsquigarrow$ la convection int\'{e}gr\'{e}e par un sch\'{e}ma d\'{e}centr\'{e} amont
(upwind) du premier ordre en espace,\\
\hspace*{1.cm}$\rightsquigarrow$ les flux diffusifs non reconstruits.\\
\begin{equation}
\mathcal{EM}_{n}(a) = f_s^{\,imp}\,a + \theta\,\,[\dive((\rho
\underline{u})\,a)]^{\textit{amont}} - \theta\,\, [\dive(\mu_{\,tot}\,\grad a)]^{\textit{N Rec}}
\end{equation}
Cet op\'erateur permet donc de contourner la difficult\'e induite par la pr\'esence d'\'eventuelles non lin\'earit\'es introduites par l'activation du test de pente lors du sch\'ema convectif, et par le remplissage important de la structure de la matrice d\'ecoulant de la pr\'esence des gradients propres \`a la reconstruction.\\
On a la relation\footnote{On pourra se reporter au sous-programme
\fort{matrix} pour plus de d\'etails relativement \`a
$\mathcal{EM_{\it{disc}}}$, op\'erateur discret agissant sur un scalaire $a$.}, pour toute cellule $\Omega_I$ de centre $I$  :
\begin{equation}\notag
\mathcal{EM_{\it{disc}}}(a,I) = \int_{\Omega_i}\mathcal{EM}_{n}(a)  \, d\Omega
\end{equation}
On cherche \`{a} r\'{e}soudre :
\begin{equation}
0 =\mathcal{E}_{n}(a^{n+1}) =  \mathcal{EM}_{n}(a^{n+1}) +  \mathcal{E}_{n}(a^{n+1}) - \mathcal{EM}_{n}(a^{n+1})
\end{equation}
Soit :
\begin{equation}
\mathcal{EM}_{n}(a^{n+1}) =  \mathcal{EM}_{n}(a^{n+1}) -  \mathcal{E}_{n}(a^{n+1})
\end{equation}
On va pour cela utiliser un algorithme de type point fixe en d\'{e}finissant la
suite $(a^{n+1,\,k})_{k\in \mathbb{N}}$\footnote{Dans le cas ou le point fixe en
vitesse-pression est utilis\'{e} (\var{NTERUP}$>$ 1) $a^{n+1,0}$ est initialis\'{e} par
la derni\`{e}re valeur obtenue de $a^{n+1}$.}:
\begin{equation}\notag
\left\{\begin{array}{l}
a^{n+1,\,0} = a^{n}\\
a^{n+1,\,k+1} = a^{n+1,\,k} + \delta a^{n+1,\,k+1}
\end{array}\right.
\end{equation}
o\`{u} $\delta a^{n+1,\,k+1}$ est solution de :
\begin{equation}
\mathcal{EM}_{n}(a^{n+1,\,k} + \delta a^{n+1,\,k+1}) = \mathcal{EM}_{n}(a^{n+1,\,k}) - \mathcal{E}_{n}(a^{n+1,\,k})
\end{equation}
Soit encore, par lin\'{e}arit\'{e} de $\mathcal{EM}_{n}$ :
\begin{equation}
\mathcal{EM}_{n}(\delta a^{n+1,\,k+1}) = - \mathcal{E}_{n}(a^{n+1,\,k})
\label{Base_Codits_Eq_Codits}
\end{equation}

Cette suite, coupl\'ee avec le choix de l'op\'erateur $\mathcal{E}_{n}$, permet donc de lever la difficult\'{e} induite par la
pr\'esence de la convection (discr\'etis\'ee \`a l'aide de sch\'emas num\'eriques
qui peuvent introduire des non lin\'earit\'es) et les termes de
reconstruction. Le sch\'ema r\'eellement choisi par l'utilisateur pour la
convection (donc \'eventuellement non lin\'eaire si le test de pente est activ\'e) ainsi que les termes de
reconstruction vont \^etre pris \`a l'it\'{e}ration $k$ et trait\'es au second membre {\it via} le sous-programme \fort{bilsc2},  alors que les termes
non reconstruits sont pris \`{a} l'it\'{e}ration $k+1$ et repr\'esentent donc les
inconnues du syst\`eme lin\'eaire r\'esolu par \fort{codits}\footnote{cf. le sous-programme
\fort{navstv}.}.\\

On suppose de plus que cette suite $(a^{n+1,\,k})_k$ converge vers la solution
$a^{n+1}$ de l'\'equation (\ref{Base_Codits_Eq_ref_small}), {\it i.e.}
$\lim\limits_{k\rightarrow\infty} \delta a^{n+1,\,k}\,=\,0$, ceci pour tout $n$ donn\'e.\\
(\ref{Base_Codits_Eq_Codits}) correspond \`a l'\'equation r\'esolue par \fort{codits}. La
matrice $\tens{EM}_{\,n}$, matrice associ\'ee \`a $\mathcal{EM}_{n}$  est
 \`a inverser, les termes non lin\'eaires sont mis au second membre mais sous forme
 explicite (indice $k$ de $a^{n+1,\,k}$) et ne posent donc plus de probl\`eme.

\minititre{Remarque 1}
La viscosit\'{e} $\mu_{\,tot}$ prise dans $\mathcal{EM}_{n}$ et dans
$\mathcal{E}_{n}$  d\'{e}pend du mod\`{e}le de turbulence utilis\'{e}. Ainsi on a
 $\mu_{\,tot}=\mu_{\,laminaire} + \mu_{\,turbulent}$
dans $\mathcal{EM}_{n}$ et dans $\mathcal{E}_{n}$ sauf lorsque l'on
utilise un mod\`{e}le $R_{ij}-\varepsilon$, auquel cas on a
$\mu_{\,tot}=\mu_{\,laminaire}$.\\
Le choix de $\mathcal{EM}_{n}$ \'{e}tant  {\it a
priori} arbitraire ($\mathcal{EM}_{n}$ doit \^etre lin\'eaire et la suite
 $(a^{n+1,\,k})_{k\in\mathbb{N}}$ doit converger pour tout $n$ donn\'e), une option des mod\`{e}les
$R_{ij}-\varepsilon$ ($\var{IRIJNU}=1$) consiste \`a for\c cer  $\mu_{\,tot}^n$
dans l'expression de $\mathcal{EM}_{n}$ \`a la
valeur $\mu_{\,laminaire}^n + \mu_{\,turbulent}^n$ lors de l'appel \`a
\fort{codits} dans le sous-programme \fort{navstv}, pour l'\'etape de
pr\'ediction de la vitesse. Ceci n'a pas de sens
physique (seul $\mu_{\,laminaire}^n$ \'{e}tant cens\'{e} intervenir), mais cela peut
dans certains cas avoir un effet stabilisateur, sans que cela modifie pour
autant les valeurs de la limite de la suite $(a^{n+1,\,k})_k$.\\

\minititre{Remarque 2}
Quand \fort{codits} est utilis\'e pour le couplage instationnaire renforc\'{e}
vitesse-pression (\var{IPUCOU}=1), on fait une seule it\'{e}ration $k$ en initialisant la suite $(a^{n+1,\,k})_{k\in\mathbb{N}}$ \`{a} z\'{e}ro. Les conditions de type Dirichlet sont
annul\'{e}es (on a $\var{INC}\,=\,0$) et le second membre est \'{e}gal \`{a} $\rho |\Omega_i|$.
Ce qui permet d'obtenir une approximation de type diagonal de
$\tens{EM}_{n}$
n\'ecessaire lors de l'\'{e}tape de correction de la vitesse\footnote{cf. le sous-programme \fort{resopv}.}.

%%%%%%%%%%%%%%%%%%%%%%%%%%%%%%%%%%
%%%%%%%%%%%%%%%%%%%%%%%%%%%%%%%%%%
\section*{Mise en \oe uvre}
%%%%%%%%%%%%%%%%%%%%%%%%%%%%%%%%%%
%%%%%%%%%%%%%%%%%%%%%%%%%%%%%%%%%%
L'algorithme de ce sous-programme est le suivant :\\
- d\'{e}termination des propri\'{e}t\'{e}s de la matrice $\tens{EM}_{n}$ (sym\'etrique
si pas de convection, non sym\'etrique sinon)\\
- choix automatique de la m\'{e}thode de r\'{e}solution pour l'inverser si l'utilisateur
ne l'a pas sp\'ecifi\'e pour la variable trait\'ee. La m\'ethode de Jacobi est utilis\'ee par d\'efaut pour toute variable scalaire $a$ convect\'ee. Les m\'ethodes
disponibles sont la m\'ethode du gradient conjugu\'e, celle de Jacobi, et le
bi-gradient conjugu\'e stabilis\'e ($BICGStab$) pour les matrices non
sym\'etriques. Un pr\'econditionnement diagonal est possible et utilis\'e par
d\'efaut pour tous ces solveurs except\'{e} Jacobi.\\
- prise en compte de la p\'eriodicit\'e (translation ou rotation d'un scalaire, vecteur ou tenseur),\\
- construction de la matrice $\tens{EM}_{n}$ correspondant \`{a} l'op\'erateur
lin\'eaire $\mathcal{EM}_{n}$ par appel au sous-programme
\fort{matrix}\footnote{ On rappelle que dans \fort{matrix}, la convection  est
trait\'{e}e, quelque soit le choix de l'utilisateur, avec un sch\'{e}ma d\'{e}centr\'{e} amont d'ordre 1 en espace et qu'il n'y a pas de reconstruction
pour la diffusion. Le choix de l'utilisateur quant au
sch\'{e}ma num\'{e}rique pour la convection intervient uniquement lors de l'int\'{e}gration
des termes de convection de $\mathcal{E}_{n}$, au second membre de
(\ref{Base_Codits_Eq_Codits}) dans le sous-programme \fort{bilsc2}.}. Les termes implicites correspondant \`{a}
la partie diagonale de la matrice et donc aux contributions diff\'erentielles
d'ordre $0$ lin\'eaires en $a^{n+1}$,({\it i.e} $f_s^{imp}$), sont stock\'{e}s dans le tableau \var{ROVSDT} (r\'ealis\'e en amont du sous-programme appelant \fort{codits}).\\
- cr\'{e}ation de la hi\'{e}rarchie de maillage si on utilise le multigrille
($ \var{IMGRP}\,>0 $).\\
- appel \`{a} \fort{bilsc2} pour une \'{e}ventuelle prise en compte de la
convection-diffusion explicite lorsque $\theta \ne 0$.\\
- boucle sur le nombre d'it\'{e}rations de 1 \`a $\var{NSWRSM}$ (appel\'{e} $\var{NSWRSP}$ dans \fort{codits}).
 Les it\'{e}rations sont repr\'{e}sent\'{e}es par $k$ appel\'{e}
\var{ISWEEP} dans le code et d\'efinissent les indices de la suite $(a^{n+1,\,k})_k$
et de $(\delta a^{n+1,\,k})_k$.\\
Le second membre est scind\'e en deux parties :\\
\hspace*{1cm}{\tiny$\blacksquare$}\ un terme, affine
en  $a^{n+1,\,k-1}$, facile \`a mettre \`a jour dans le cadre de la r\'esolution
par incr\'ement, et qui s'\'ecrit :
\begin{equation}\notag
 -f_s^{\,imp} \left(\,a^{n+1,\,k-1} - a^{n+1,0}\right) + f_s^{\,exp}- (1-\theta)\,\left[\,\dive((\rho \underline{u})\,a^{n+1,0}) - \dive(\mu_{\,tot}\,\grad a^{n+1,0})\,\right]\\
\end{equation}
\\
\hspace*{1cm}{\tiny$\blacksquare$}\ les termes issus de la
convection/diffusion (avec reconstruction) calcul\'{e}e par \fort{bilsc2}.\\
\begin{equation}\notag
- \theta\,\left[\,\dive\left((\rho \underline{u})\,a^{n+1,\,k-1}\right)- \dive\left(\mu_{\,tot}\,\grad a^{n+1,\,k-1}\right)\right]
\end{equation}

La boucle en $k$ est alors la suivante :
\begin{itemize}
\item Calcul du second membre, hors contribution des termes de
convection-diffusion explicite $\var{SMBINI}$; le second membre complet correspondant
\`a $\mathcal{E}_{n}(a^{n+1,\,k-1})$ est, quant \`a lui, stock\'e dans le
tableau $\var{SMBRP}$, initialis\'e par $\var{SMBINI}$ et compl\'et\'e par les
termes reconstruits de convection-diffusion par appel au sous-programme
\fort{bilsc2}.\\
\`A l'it\'eration $k$, $\var{SMBINI}$ not\'e  $\var{SMBINI}^{\,k}$ vaut :\\
\begin{equation}\notag
\var{SMBINI}^{\,k}\  = f_s^{\,exp}-(1-\theta)\,\left[\,\dive((\rho \underline{u})\,a^n) - \dive(\mu_{\,tot}\,\grad a^n)\,\right]-\,f_s^{\,imp}(\,a^{n+1,\,k-1} - a^n\,) \\
\end{equation}
\\
$\bullet$ Avant de boucler sur $k$, un premier appel au sous-programme \fort{bilsc2} avec $\var{THETAP}=1-\theta$ permet de prendre
en compte la partie explicite des termes de convection-diffusion provenant du sch\'ema en temps.
\begin{equation}\notag
\displaystyle
\var{SMBRP}^{\,0} = f_s^{\,exp} -(1-\theta)\,[\,\dive((\rho \underline{u})\,a^n) - \dive(\mu_{\,tot}\,\grad a^n)\,]\\
\end{equation}
Avant de boucler sur $k$, le second membre $\var{SMBRP}^{\,0}$ est stock\'e dans le tableau $\var{SMBINI}^{\,0}$ et sert pour l'initialisation du reste du calcul.
\begin{equation}\notag
\var{SMBINI}^{\,0} =\var{SMBRP}^{\,0}
\end{equation}
\\
$\bullet$ pour $k = 1$,
\begin{equation}\notag
\begin{array}{ll}
\var{SMBINI}^{\,1}\ &=f_s^{\,exp}-(1-\theta)\,\left[\,\dive((\rho \underline{u})\,a^n) - \dive(\mu_{\,tot}\,\grad a^n)\,\right]-\,f_s^{\,imp}\,(\,a^{n+1,\,0} - a^n\,)\\
&=f_s^{\,exp}- (1-\theta)\,\left[\,\dive((\rho \underline{u})\,a^{n+1,\,0}) - \dive(\mu_{\,tot}\,\grad a^{n+1,\,0})\,\right]-f_s^{\,imp}\,\delta a^{n+1,\,0} \\
\end{array}
\end{equation}
On a donc :
\begin{equation}\notag
\var{SMBINI}^{\,1}\ =\ \var{SMBINI}^{\,0} - \var{ROVSDT}\,*(\,\var{PVAR}-\,\var{PVARA})
\end{equation}
et $\var{SMBRP}^{\,1}$ est compl\'et\'e par un second appel au sous-programme \fort{bilsc2} avec $\var{THETAP}=\theta$, de mani\`ere \`a ajouter dans le second membre la partie de la convection-diffusion implicite.
\begin{equation}\notag
\begin{array}{ll}
\var{SMBRP}^{\,1} & = \var{SMBINI}^{\,1} -\theta\,\left[\,\dive((\rho \underline{u})\,a^{n+1,\,0}) - \dive(\mu_{\,tot}\,\grad a^{n+1,\,0})\,\right]\\
& = f_s^{\,exp}\ - (1-\theta)\,\left[\,\dive((\rho \underline{u})\,a^{n}) - \dive(\mu_{\,tot}\,\grad a^{n})\,\right]- f_s^{\,imp}\,(a^{n+1,\,0} -a^{n}) \\
& -\theta\,\left[\,\dive((\rho \underline{u})\,a^{n+1,\,0}) - \dive(\mu_{\,tot}\,\grad a^{n+1,\,0})\,\right]\\
\end{array}
\end{equation}
$\bullet$ pour $k = 2$,\\
de fa\c con analogue, on obtient :
\begin{equation}\notag
\begin{array}{ll}
\var{SMBINI}^{\,2}\ &=f_s^{\,exp}-(1-\theta)\,\left[\,\dive((\rho \underline{u})\,a^n) - \dive(\mu_{\,tot}\,\grad a^n)\,\right]-\,f_s^{\,imp}\,(\,a^{n+1,\,1} - a^n\,)\\
\end{array}
\end{equation}
Soit :
\begin{equation}\notag
\var{SMBINI}^{\,2}\ =\ \var{SMBINI}^{\,1} - \var{ROVSDT}\,*\,\var{DPVAR}^{\,1}
\end{equation}
l'appel au sous-programme \fort{bilsc2}, \'etant syst\'ematiquement fait par la suite avec $\var{THETAP}=\theta$, on obtient de m\^eme :
\begin{equation}\notag
\begin{array}{ll}
\var{SMBRP}^{\,2}\ &=\ \var{SMBINI}^{\,2}-\theta\left[\dive\left((\rho \underline{u})\,a^{n+1,\,1}\right)- \dive\left(\mu_{\,tot}\,\grad \,a^{n+1,\,1}\right)\right]\\
\end{array}
\end{equation}
o\`u
\begin{equation}\notag
a^{n+1,\,1}=\var{PVAR}^{\,1}=\var{PVAR}^{\,0}+\var{DPVAR}^{\,1}=a^{n+1,\,0}+\delta a^{n+1,\,1}
\end{equation}
$\bullet$ pour l'it\'eration $k+1$,\\
Le tableau $\var{SMBINI}^{\,k+1}$ initialise le second membre complet
$\var{SMBRP}^{\,k+1}$ auquel vont \^etre rajout\'ees les contributions
convectives et diffusives {\it via} le sous-programme \fort{bilsc2}.\\
on a la formule :
\begin{equation}\notag
\begin{array}{ll}
\var{SMBINI}^{\,k+1}\ &= \var{SMBINI}^{\,k} - \var{ROVSDT}\,*\,\var{DPVAR}^{\,k}\\
\end{array}
\end{equation}
Puis suit le calcul et l'ajout des termes de convection-diffusion reconstruits de $-\  \mathcal{E}_{n}(a^{n+1,\,k})$, par appel au sous-programme
\fort{bilsc2}. On rappelle que la convection est prise en compte \`{a} cette \'{e}tape
par le sch\'{e}ma num\'{e}rique choisi par l'utilisateur (sch\'{e}ma d\'{e}centr\'{e} amont du
premier ordre en espace, sch\'{e}ma centr\'{e} du second ordre en espace, sch\'{e}ma
d\'{e}centr\'{e} amont du second ordre S.O.L.U. ou une
pond\'{e}ration (blending) des sch\'{e}mas dits du second ordre (centr\'{e}  ou S.O.L.U.) avec le sch\'{e}ma
amont du premier ordre, utilisation \'eventuelle d'un test de pente).\\
Cette contribution (convection-diffusion) est alors
ajout\'{e}e dans le second membre  $\var{SMBRP}^{\,k+1}$ (initialis\'e par $\var{SMBINI}^{\,k+1}$).
\begin{equation}\notag
\begin{array}{ll}
\var{SMBRP}^{\,k+1}\ &= \var{SMBINI}^{\,k+1} - \theta\,\left[\,\dive\left((\rho \underline{u})\,a^{n+1,\,k}\right)- \dive\left(\mu_{\,tot}\,\grad a^{n+1,\,k}\right)\right]\\
& = f_s^{\,exp}-(1-\theta)\,\left[\,\dive((\rho \underline{u})\,a^n) - \dive(\mu_{\,tot}\,\grad a^n)\,\right]- f_s^{\,imp}\,(a^{n+1,\,k} -a^{n}) \\
&-\theta\,\left[\,\dive((\rho \underline{u})\,a^{n+1,k}) - \dive(\mu_{\,tot}\,\grad a^{n+1,k})\,\right]\\
\end{array}
\end{equation}

\item R\'esolution du syst\`{e}me lin\'{e}aire en $\delta a^{n+1,\,k+1}$ correspondant
\`a l'\'equation (\ref{Base_Codits_Eq_Codits}) par inversion de la matrice
$\tens{EM}_{n}$, en appelant le sous programme \fort{invers}.
On calcule $a^{n+1,\,k+1}$ gr\^ace \`a la formule :
\begin{equation}\notag
a^{n+1,\,k+1} =  a^{n+1,\,k} + \delta a^{n+1,\,k+1}
\end{equation}
Soit :
\begin{equation}\notag
\var{PVAR}^{\,k+1} =  \var{PVAR}^{\,k} + \var{DPVAR}^{\,k+1}
\end{equation}

\item Traitement de la p\'eriodicit\'e et du parall\'{e}lisme.
\item Test de convergence :\\
Il porte sur la quantit\'e  $||\var{SMBRP}^{\,k+1}|| < \varepsilon
||\tens{EM}_{n}(a^{n}) + \var{SMBRP}^{\,1}|| $, o\`u $||\,.\,||$ repr\'esente la
norme euclidienne.
Si le test est v\'erifi\'e, la convergence est atteinte et on sort de la
boucle sur les it\'{e}rations. La solution recherch\'{e}e est  $a^{\,n+1} = a^{n+1,\,k+1}$.\\
Sinon, on continue d'it\'erer dans la limite des it\'{e}rations impos\'{e}es par $\var{NSWRSM}$ dans \fort{usini1}.\\
En ``continu'' ce test de convergence s'\'{e}crit aussi :
\begin{equation}\notag
\begin{array}{ll}
||\var{SMBRP}^{\,k+1}||& < \varepsilon ||f_s^{\,exp}\ - \dive((\rho \underline{u})\,a^{n}) + \dive(\mu_{\,tot}\,\grad a^{n}) \\
& +[\dive((\rho \underline{u})\,a^{n})]^{\textit{amont}} + [\dive(\mu_{\,tot}\,\grad a^{n})]^{\textit{N Rec}}||\\
\end{array}
\end{equation}
Si bien que sur maillage orthogonal avec sch\'{e}ma de convection upwind et en l'absence de terme source, la suite converge en th\'{e}orie en une unique it\'{e}ration puisque par construction~:
\begin{equation}\notag
\begin{array}{ll}
||\var{SMBRP}^{\,2}||=\,0\,& < \varepsilon ||f_s^{\,exp}||
\end{array}
\end{equation}
\end{itemize}
Fin de la boucle.
\\

%%%%%%%%%%%%%%%%%%%%%%%%%%%%%%%%%%
%%%%%%%%%%%%%%%%%%%%%%%%%%%%%%%%%%
\section*{Points \`a traiter}\label{Base_Codits_section4}
%%%%%%%%%%%%%%%%%%%%%%%%%%%%%%%%%%
%%%%%%%%%%%%%%%%%%%%%%%%%%%%%%%%%%
\etape{Approximation $\mathcal{EM}_{n}$ de l'op\'erateur
$\mathcal{E}_{n}$}
D'autres approches visant soit \`a modifier la d\'efinition de l'approxim\'ee, prise en compte du sch\'ema centr\'e sans reconstruction par exemple,
soit \`a abandonner cette voie seraient \`a \'etudier.\\

\etape{Test de convergence}
La quantit\'e d\'efinissant le  test de convergence est \'{e}galement \`{a} revoir, \'{e}ventuellement \`{a} simplifier.

\etape{Prise en compte de $T_s^{imp}$}
Lors de la r\'{e}solution de l'\'{e}quation par \fort{codits}, le tableau \var{ROVSDT} a
deux fonctions : il sert \`{a} calculer la diagonale de la matrice (par appel de
\fort{matrix}) et il sert \`{a} mettre \`{a} jour le second membre \`{a} chaque
sous-it\'{e}ration de la r\'{e}solution en incr\'{e}ments. Or, dans le cas o\`{u} $T_s^{imp}$
est positif, on ne l'int\`{e}gre pas dans \var{ROVSDT}, afin de ne pas affaiblir la
diagonale de la matrice. De ce fait, on ne l'utilise pas pour mettre \`{a} jour le
second membre, alors que ce serait tout \`{a} fait possible. Au final, on obtient
donc un terme source trait\'{e} totalement en explicite ($T_s^{exp}+T_s^{imp}a^n$),
alors que la r\'{e}solution en incr\'{e}ments nous permettrait justement de l'impliciter
quasiment totalement ($T_s^{exp}+T_s^{imp}a^{n+1,k_{fin}-1}$, o\`{u} $k_{fin}$ est
la derni\`{e}re sous-it\'{e}ration effectu\'{e}e).\\
Pour ce faire, il faudrait d\'{e}finir deux tableaux \var{ROVSDT} dans \fort{codits}.

%-------------------------------------------------------------------------------

% This file is part of code_saturne, a general-purpose CFD tool.
%
% Copyright (C) 1998-2022 EDF S.A.
%
% This program is free software; you can redistribute it and/or modify it under
% the terms of the GNU General Public License as published by the Free Software
% Foundation; either version 2 of the License, or (at your option) any later
% version.
%
% This program is distributed in the hope that it will be useful, but WITHOUT
% ANY WARRANTY; without even the implied warranty of MERCHANTABILITY or FITNESS
% FOR A PARTICULAR PURPOSE.  See the GNU General Public License for more
% details.
%
% You should have received a copy of the GNU General Public License along with
% this program; if not, write to the Free Software Foundation, Inc., 51 Franklin
% Street, Fifth Floor, Boston, MA 02110-1301, USA.

%-------------------------------------------------------------------------------

\programme{condli}

\hypertarget{condli}{}

\vspace{1cm}
%%%%%%%%%%%%%%%%%%%%%%%%%%%%%%%%%%
%%%%%%%%%%%%%%%%%%%%%%%%%%%%%%%%%%
\section*{Function}
%%%%%%%%%%%%%%%%%%%%%%%%%%%%%%%%%%
%%%%%%%%%%%%%%%%%%%%%%%%%%%%%%%%%%
Boundary conditions are required in at least three principal cases:
\begin{itemize}
\item calculation of the convection terms (first order derivative in space) at
the boundary: the calculation uses a flux at the boundary and requires the
value of the convected variable at the boundary when the latter is entering
the domain in the sens of the characteristic curves of the system (in the sens
of the velocity, in the case of the single equation of a simple scalar:
 sufficient interpretation in the current framework of
\CS\footnote{except with the compressible module, see. \fort{cfxtcl}})~;
\item calculation of the diffusion terms (second order derivative
in space):
a method to determine the value of the first order spatial derivatives
at the boundary is then required
 (more exactly, the terms that depend upon it are required,
 such as the stresses of the thermal fluxes at the wall):
\item calculation of the cell  gradients: the variable at the boundary faces
 are required (more generally, the discrete terms of the equations which depend
upon the gradient inside boundary cells are required, such as the transpose
gradient terms in the Navier-Stokes equations).
\end{itemize}
These considerations only concern the computational variables
(velocity, pressure, Reynolds tensor,
scalars solution of a convection-diffussion equation). For these variables
\footnote{
The other variables
(the physical properties for instance) have a different treatment which will
not be detailed here (for instance, for the density, the user defines
directly the values at the boundary. This information is then stored~; one
is referred to \fort{usphyv} or \fort{phyvar} for more information).
},
the user has to define the boundary conditions at every boundary face
(\fort{usclim}).


The \fort{condli} subroutine transforms the data provided by the user
(in \fort{usclim}) into an internal format of representation of the boundary
 conditions. Verifications of the completeness and coherence are also
performed (in \fort{vericl}). More particularly, the smooth-wall boundary conditions
 (\fort{clptur}), the rough-wall boundary conditions (\fort{clptrg})
and the symmetry boundary conditions for the velocities and
the Reynolds stress tensor (\fort{clsyvt}) are treated in dedicated subroutines.

The \fort{condli} subroutine
provides as an output pairs of coefficients
$A_b$ and $B_b$
for each variable~$f$ and for each boudary face.
These are used for the calculation of the discrete terms in the equations
to be solved. More specifically, these coefficients are used to calculate
a value at the boundary faces $f_{b,int}$ (localised at the centre of the
boundary face, barycentre of its vertices) by the relation
 $f_{b,int} = A_b+B_b\,f_{I'}$, where $f_{I'}$ is the value of the variable
at point $I'$. $I'$ is the projection onto the centre of the cell
adjoin to the boundary on the line normal to the boundary face and passing
by its centre
(see figure~\ref{Base_Condli_fig_flux_condli}).

See the \doxygenfile{condli_8f90.html}{programmers reference of the dedicated subroutine} for further details.

\begin{figure}[h]
\centerline{\includegraphics[height=8cm]{fluxbord}}
\caption{\label{Base_Condli_fig_flux_condli}Boundary cell.}
\end{figure}

%%%%%%%%%%%%%%%%%%%%%%%%%%%%%%%%%%
%%%%%%%%%%%%%%%%%%%%%%%%%%%%%%%%%%
\section*{Discretization}
%%%%%%%%%%%%%%%%%%%%%%%%%%%%%%%%%%
%%%%%%%%%%%%%%%%%%%%%%%%%%%%%%%%%%

\etape{Notation}
%%%%%%%%%%%%%%%%
In the following, we will denote by {\it VarScalaire} any variable
\begin{itemize}
\item [-] other than the
velocity, pressure, turbulent quantities $k$, $\varepsilon$,
$R_{ij}$, $\varphi$, $\bar{f}$ and $\omega$,

\item [-] solution of a convection-diffusion equation.
\end{itemize}
The denomination {\it VarScalaire} may in particular designate
the temperature, a passive scalar,
a mass fraction or (unless explicitly stated otherwise) the variance of fluctuations
of another {\it VarScalar}. The inferred state variables (mass
volumetric, viscosity...) will not be denoted by {\it VarScalar}.


\etape{Representation of standard boundary conditions in \fort{cs\_user\_boundary\_conditions} \vspace{0,3cm}}
%%%%%%%%%%%%%%%%%%%%%%%%%%%%%%%%%%%%%%%%%%%%%%%%%%%%%%%%%%%%%%%%%%%%%%%%%%%%%%%
Standard boundary conditions can be provided by
the user in \fort{cs\_user\_boundary\_conditions}. For this, it is necessary to assign a
type to boundary faces \footnote{The assignment of a type is done
by filling in the table
\var{BC\_TYPE}.}. The default conditions are as follows:

\begin{itemize}
\item [$\bullet$] {\bf CS\_INLET}: corresponds to a Dirichlet condition on
all the variables transported (velocity, turbulent variables,
{\it VarScalaires}...), and
a homogeneous Neumann condition (zero flow) on the pressure.

\item [$\bullet$] {\bf CS\_OUTLET}:
        \begin{itemize}
        \item [-] when the mass flow is effectively directed outwards
of the domain, this choice corresponds to a homogeneous Neumann condition
on all the transported variables and\\
$\displaystyle\frac{\partial^2P}{\partial \vect{n}\partial\vect{\tau}_i}=0$, taken into account as
of Dirichlet for the pressure
($\vect{n}$ and $(\vect{\tau}_i)_{i \in \{1,2\}}$
denote respectively the normal vector of the exit face considered
and two normed vectors, orthogonal to each other and in the plane of the exit face).
This condition is explicit using the pressure field and its gradient
at the previous time step.
Moreover, the pressure being defined to a constant, it is readjusted to a point of
output to store the value \var{P0} (this avoids any drift towards very large values
relatively to the maximum pressure difference on the domain)\footnote{When there is not output,
the spectrum of eigen values of the matrix is shifted by a constant value
in order to make the system invertible: see \fort{cs\_matrix\_wrapper\_scalar}.}.
        \item [-] when the mass flow is directed inwards, an
undesirable situation {\it a priori}\footnote{A message indicates
to the user how many exit faces see a mass flow entering into the domain (i.e. a backflow).
},

A Homogeneous Dirichlet condition is imposed on the on velocity (not on the mass flux),
its value downstream of the domain being unknown. The pressure is
treated as in the previous case where the mass flux is
directed outwards of the domain. For variables other than
velocity and pressure, two cases can occur:
        \begin{itemize}
        \item[-] one can impose a condition of Dirichlet to represent the value of the scalar introduced into the field by the faces of boundary concerned.
        \item[-] one can impose, as when the mass flux is outgoing, a homogeneous Neumann condition (this is not a desirable situation, since the information carried on the boundary faces then comes from {\it the downstream} of the local flow). This is the case by default if no value is given for the Dirichlet.
        \end{itemize}
\end{itemize}
\item [$\bullet$] {\bf CS\_SMOOTHWALL}: refer to \fort{clptur} (or to \fort{clptrg} for rough walls) for a description of the treatment of the boundary conditions of the wall (assumed to be impermeable to
fluid). Briefly, we can say here
that a wall law approach is used to impose the constraint
tangential on speed. The wall may slide\footnote{We must then provide
the components of the wall velocity.}. The {\it VarScalars} are assigned
a homogeneous Neumann condition (zero flow) by default.
To impose a wall value for these variables (for example, in the case of a
wall at imposed temperature) a law of similarity is used
to determine the boundary flux taking into account the boundary layer.
In the case of couplings with \syrthes, \CS
receives a wall temperature and sends a heat flux (decomposed as a pseudo interior temperature
and exchange coefficient to allow relaxation).
The standard pressure condition is a homogeneous Neumann condition.
\item [$\bullet$] {\bf CS\_SYMMETRY}: corresponds to homogeneous Neumann conditions for scalar quantities and to classical symmetry conditions for vectors (velocity) and tensors (Reynolds voltages): see \fort{clsyvt}.
\end{itemize}

\etape{Representation of specific boundary conditions in \fort{cs\_user\_boundary\_conditions}\vspace{0,3cm}}
%%%%%%%%%%%%%%%%%%%%%%%%%%%%%%%%%%%%%%%%%%%%%%%%%%%%%%%%%%%%%%%%%%%%%%%%%%%%%%%
We have seen that the assignment to a boundary face of a standard type
(entrance, exit, wall, symmetry) makes it possible to apply coherent
boundary conditions to all variables in a simple manner for the usual
types of physical boundaries.

It is also possible to define specific boundary conditions in
\fort{cs\_user\_boundary\_conditions}, for each boundary face and each variable
\footnote{Such conditions are direclty defined through the
\var{icodcl} and \var{rcodcl} arrays for each face and variable: examples are provided in
\fort{cs\_user\_boundary\_conditions}.}
(these, like the standard conditions, ultimately amount to mixed-type conditions).

The two approaches are compatible and often combined.
Indeed, the standard (default) boundary conditions
can be overridden by the user where necessary, and the defaults used elsewhere.
In any case, it should be ensured that a boundary condition has been defined for each
boundary face and variable.

Compatibility conditions also exist between the different
variables~(see \fort{vericl}):
\begin{itemize}
\item [-] between, wall, symmetry or free exit, it is important that all the
velocity components have the same type of condition~;
\item [-] when the velocity receives an exit condition, it is important
that the pressure is assigned a Dirichlet-like condition. For more
details, refer to the paragraph relating to the exit condition for
pressure, page \pageref{Base_Condli_Sortie_Pression};
\item [-] when one of the velocity or turbulence variables
is assigned a wall condition, so must all such variables;
\item [-] when one of the $velocity or R_{ij}$ components is assigned a symmetry
condition, it should be the case for all components;
\item [-] when a {\it VarScalar} receives a wall condition, the
velocity must have the same type of condition.
\end{itemize}

\newpage

\etape{Internal representation of boundary conditions}
%%%%%%%%%%%%%%%%%%%%%%%%%%%%%%%%%%%%%%%%%%%%%%%%%%%%%%%%%%%%%%%%%%%%%%%%%%%%%%%

{\bf Objective}

The conditions defined by the user are
converted in the form of pairs of coefficients $A_b$ and $B_b$
for each variable~$f$ and boundary face. These coefficients
are used for the calculation of
discrete terms intervening in the solved equation and
make it possible in particular to determine a boundary face value
$f_{b,int}$. It is important to insist on the fact
that this value is, in general, a simple numerical value
which does not necessarily reflect a physical reality (in particular
 to the walls, for the quantities affected by the turbulent boundary layer).
We detail the calculation of $A_b$, $B_b$ and $f_{b,int}$ below.

{\bf Notations}
\begin{itemize}

\item[-] We consider the equation (\ref{Base_Condli_eq_conv_diff_condli})
on the scalar $f$, in which
$\rho$ represents the density, $\vect{u}$ the velocity, $\alpha$ the
conductivity and $S$ additional source terms. $C$ is defined below.
\begin{equation}\label{Base_Condli_eq_conv_diff_condli}
\displaystyle\rho\frac{\partial f}{\partial t} + div(\rho \vect{u} f)=div\left(\displaystyle\frac{\alpha}{C}\, \grad f\right)+S
\end{equation}

\item[-] The coefficient $\alpha$ represents the sum of the
molecular and turbulent conductivities (depending on the models used), or
$\alpha=\alpha_m+\alpha_t$, with, for a turbulent viscosity based modeling,
$\displaystyle\alpha_t=C\,\frac{\mu_t}{\sigma_t}$, where $\sigma_t$ is the
turbulent Prandtl number\footnote{The turbulent Prandtl number is dimensionless and,
in some usual cases, taken as equal to $0.7$.}.

\item[-] The coefficient $C_p$ represents the specific heat, with unit $m^{2}\,s^{-2}\,K^{-1}=J\,kg^{-1}\,K^{ -1}$.

\item[-] We note $\lambda$ the thermal conductivity, with unit
$kg\,m\,s^{-3}\,K^{-1}=W\,m^{-1}\,K^{-1}$.

\item[-] It should be specified that $C=1$ for all variables except for the
temperature, for which $C=C_p$\footnote{More precisely, we have
$C=C_p$ for all {\it scalar variables} $f$ that we want to consider as the
temperature for the boundary conditions. These {\it scalar variables} can
be checked using the \var{is\_temperature} field keyword. By default this
value is set to 0 for all variables except for the temperature.}.
In the code, the value which the user must define is
$\displaystyle\frac{\alpha_m}{C}$ (if the property is constant, values are
initialized to \var{viscl0} for the velocity and \var{visls0} for {\it scalar variables};
If the property is variable, equivalent field values must be defined in the GUI or
in \fort{cs\_user\_physical\_properties.c}).

\item[-] For the variance of the fluctuations of a {\it scalar variable}, the
conductivity $\alpha$ and the coefficient $C$ are inherited from the
associated variable.

\end{itemize}


{\bf Simple Dirichlet condition}: for a simple Dirichlet condition, we
naturally obtain (special case of(\ref{Base_Condli_eq_fbord_condli})):
\begin{equation}
\underbrace{\ \ f_{b,int}\ \ }_{\text{boundary value used by the calculation}}
= \underbrace{\ \ f_{\text{\it real}}\ \ }_{\text{real value imposed on the boundary}}
\end{equation}
{\bf Other cases}: when the boundary condition is based on a flux,
this is of a diffusive flow \footnote{Indeed, the total outgoing flow of the domain is
given by the
sum of the convective flux (if the variable is actually convected)
and diffusive flux. However, for solid walls and symmetries, the mass flow is zero and
the condition reduced to a constraint on the diffusive flux. Moreover, for the
outlets (exiting mass flow), the boundary condition relates only to the
diffusive flux (often a homogeneous Neumann condition), convective flux
dependent on the upstream conditions (therefore it does not need
boundary conditions). Finally, at the inlets, a simple Dirichlet condition is most often
used, and the diffusive flux is deduced from it.}.
We then have:
\begin{equation}
\underbrace{\ \ \phi_{int}\ \ }_{\text{diffusive flux towards the internal domain}}
= \underbrace{\ \ \phi_{\text{\it real}}\ \ }_{\text{real diffusive flux imposed on the boundary}}
\end{equation}

The imposed real diffusive flux can be given
\begin{itemize}
\item [-] directly (Neumann condition), or
$\phi_{\text{\it r\'eel}}=\phi_{\text{\it imp,ext}}$ or
\item [-] implicitly deduced from two imposed values: an exterior value
$f_{imp,ext}$ and an exchange coefficient $h_{imp,ext}$
(generalized Dirichlet condition).
\end{itemize}

\vspace{1cm}
Depending on the type of condition (Dirichlet or Neumann) and assuming
the conservation of the flux normal to the boundary,
we can write (see figure \ref{Base_Condli_fig_flux_condli}):
\begin{equation}\label{Base_Condli_eq_flux_condli}
\begin{array}{l}
    \underbrace{h_{int}(f_{b,int}-f_{I'})}_{\phi_{int}}
  = \underbrace{h_{b}(f_{b,ext}-f_{I'})}_{\phi_{b}}
  = \left\{\begin{array}{ll}
    \underbrace{h_{imp,ext}(f_{imp,ext}-f_{b,ext})}_{\phi_{\text{\it imposed real}}} &\text{(condition of Dirichlet)}\\
    \underbrace{\phi_{\text{\it imp,ext}}}_{\phi_{\text{\it imposed real}}}
            &\text{(Neumann condition)}
           \end{array}\right.
\end{array}
\end{equation}

The ratio between the coefficient $h_{b}$ and the coefficient $h_{int}$ accounts for
the importance of crossing the near-boundary zone and is of particular importance in
the case of walls along which a boundary layer develops (whose properties are then taken
into account by $h_{b}$: see \fort{clptur}).
In the simpler framework considered here, we will limit ourselves to
case $h_{b}=h_{int}$ and $f_{b,ext}=f_{b,int}=f_{b}$.
The relation (\ref{Base_Condli_eq_flux_condli}) is then written:

\begin{equation}
\underbrace{h_{int}(f_{b}-f_{I'})}_{\phi_{int}}
  = \left\{\begin{array}{ll}
    \underbrace{h_{imp,ext}(f_{imp,ext}-f_{b})}_{\phi_{\text{\it imposed real}}} &\text{(condition of Dirichlet)}\\
    \underbrace{\phi_{\text{\it imp,ext}}}_{\phi_{\text{\it imposed real}}}
            &\text{(condition  Neumann)}
           \end{array}\right.
\end{equation}

By rearranging, we get the boundary value:
\begin{equation}\label{Base_Condli_eq_fbord_condli}
f_{b}
  = \left\{\begin{array}{cccccl}
    \displaystyle\frac{h_{imp,ext}}{h_{int}+h_{imp,ext}}&f_{imp,ext}&+&
    \displaystyle\frac{h_{int}}{h_{int}+h_{imp,ext}}    &f_{I'}
                         &\text{(Dirichlet condition)}\\
    \displaystyle\frac{1}{h_{int}}&\phi_{\text{\it imp,ext}}&+&
    \ &f_{I'}
            &\text{(Neumann condition)}
           \end{array}\right.
\end{equation}


{\bf Conclusion}: we will therefore note the boundary conditions
in the general form:
\begin{equation}
f_{b}=A_b + B_b\,f_{I'}
\end{equation}
with $A_b$ et $B_b$ defined according to the type of conditions:
\begin{equation}
\begin{array}{c}
\text{Dirichlet}\left\{\begin{array}{ll}
    A_b = &\displaystyle\frac{h_{imp,ext}}{h_{int}+h_{imp,ext}}f_{imp,ext}\\
    B_b = &\displaystyle\frac{h_{int}}{h_{int}+h_{imp,ext}}
                  \end{array}\right.
\text{\ \  Neumann}\left\{\begin{array}{ll}
    A_b = &\displaystyle\frac{1}{h_{int}}\phi_{\text{\it imp,ext}}\\
    B_b = &1
                  \end{array}\right.
\end{array}
\end{equation}

\newpage

{\bf Remarks }
\begin{itemize}
\item [-] The value $f_{I'}$ is calculated using the cell gradient of $f$,
that is: $f_{I'}=f_{I}+\vect{II'}\grad{f}_I$.
\item [-] The value of $h_{int}$ must yet be specified. This is a {\it numerical}, value,
having {\it a priori} no relation to a physical exchange coefficient,
which depends on the mode of calculation of the diffusive flux in the boundary cell.
Thus~$\displaystyle h_{int}=\displaystyle\frac{\alpha}{\overline{I'F}}$
(the unit is naturally deduced).
\item [-] We also repeat, because it can be a cause of error, that in the code, we have:
        \begin{itemize}
        \item [-] for the temperature $\alpha_m=\lambda$ et $C=C_p$
        \item [-] for the enthalpy      $\alpha_m=\displaystyle\frac{\lambda}{C_p}$ et $C=1$
        \end{itemize}
\end{itemize}

{\bf Examples of special cases}
\begin{itemize}
\item [-] In the case of a Dirichlet condition,
the user is therefore led to
provide two values: $f_{imp,ext}$ and $h_{imp,ext}$.
To obtain a simple Dirichlet condition (without exchange coefficient)
just impose $h_{imp,ext}=+\infty$. This is the most commonl case
(in practice, $h_{imp,ext}=cs\_math\_infinite\_r$ ).
\item [-] In the case of a Neumann condition, the user provides a single value
$\phi_{\text{\it imp,ext}}$ (zero for homogeneous Neumann conditions).

\end{itemize}

\etape{Output condition for pressure\vspace{0,3cm}}\label{Base_Condli_Sortie_Pression}
%%%%%%%%%%%%%%%%%%%%%%%%%%%%%%%%%%%%%%%%%%%%%%%%%%%%%%%%%%%%%%%%%%%%
We specify here condition applied to the
pressure for standard outlets.
It is necessary to impose a Dirichlet type condition (accompanied by a
homogeneous Neumann condition on the components of the velocity). We compute it
based on the values of the variable at the previous time step.

\begin{itemize}
\item [-] Reasoning on a simple configuration (a channel, with a flat output,
perpendicular to the flow), it can be assumed that the shape of the pressure
profiles taken on the surfaces parallel to the outlet are unchanged around
this outlet (hypothesis of an established flow, far from any
perturbation). In this situation, we can write
$\displaystyle\frac{\partial^2P}{\partial\vect{n}\partial\vect{\tau}_i}=0$
($\vect{n}$ is the vector normal to the outlet, $\vect{\tau}_i$ represents
a basis of the exit plane).

\item [-] If, moreover,
it can be assumed that the pressure gradient taken in the direction
perpendicular to the outlet faces is uniform in its neighborhood,
the profile to impose at the output (values $p_b$) can be deduced from the profile
taken on an upstream plane ($p_{upstream}$ values) by simply adding the constant
$R=d\,\grad{(p)}.\vect{n}$ (where $d$ is the distance between the upstream plane
and the outlet), or $p_b=p_{upstream}+R$ (the fact that $R$ is identical for
all outlet faces is important so we can eliminate it in the equation
(\ref{Base_Condli_eq_psortie_condli}) below).

\item [-] With the additional assumption that the points $I'$ relative to
the outlet faces are on a plane parallel to the output, we can use the
values at these points ($p_{I'}$) for upstream values, so
$p_{upstream}=p_{I'}=p_{I}+\vect{II'}.\grad{p}$.

\item [-] Furthermore, the
pressure being defined relative to a constant (incompressible flow)
one can fix its value arbitrarily at a point $A$
\footnote{by default, the center of mass of the outlet faces}) to $p_0$ (value fixed by
user, equal to \var{P0} and null by default),
and therefore shift the imposed profile at the output by adding:\\
$R_0=p_0-(p_{upstream,A}+R)=p_0-(p_{I',A}+R)$.

\item [-] So we finally obtain:
\begin{equation}\label{Base_Condli_eq_psortie_condli}
\begin{array}{lll}
p_b&=&p_{I'}+R+R_0\\
   &=&p_{I'}+R+p_0-(p_{I',A}+R)\\
   &=&p_{I'}+\underbrace{p_0-p_{I',A}}_{\text{constant value $R_1$}}\\
   &=&p_{I'}+R_1
\end{array}
\end{equation}
\end{itemize}
It is therefore noted that the pressure condition at the outlet is Dirichelet condition
whose values are equal to the pressure values (taken at previous time step) on
the plane upstream of the points $I'$ and readjusted to obtain \var{P0} in
an arbitrary exit point.

%%%%%%%%%%%%%%%%%%%%%%%%%%%%%%%%%%
%%%%%%%%%%%%%%%%%%%%%%%%%%%%%%%%%%
\section*{Remaining issues}
%%%%%%%%%%%%%%%%%%%%%%%%%%%%%%%%%%
%%%%%%%%%%%%%%%%%%%%%%%%%%%%%%%%%%
\etape{Representation of conditions by face value}
Although the method used allows simplicity and
homogeneity of treatment of all boundary conditions,
it is quite restrictive in the sense that a single value is not always sufficient to
represent the conditions to be applied when calculating different terms.

Thus, in $k-\varepsilon$ it was necessary, when calculating the boundary conditions of the wall,
to use two ($A_b$, $B_b$) pairs in order to take into account the conditions to apply*
for the calculation of the shear stress and those to be used when calculating the production
term (and a third set of coefficients would be necessary for
allow the treatment of the gradients intervening in the terms of gradient
transposed, in \fort{visecv}).

Perhaps it could be useful to set up a method
allowing to direclty use (at least in some strategic points of the code)
forces, stresses or fluxes, without requiring a general a face value computation.

\etape{Pressure outlet condition}
The outlet pressure condition translates to $p_f=p_{I'}+R1$ and the profile obtained
corresponds to the upstream profile taken at points $I'$ and readjusted to obtain $p_0$ at
an arbitrary point $A$. This type of condition is applied without precautions, but is not
always justified (a Dirichlet condition based on the calculated value directly at the boundary
faces might be more suitable).
The hypotheses are particularly faulty in the following cases:
\begin{itemize}
\item [-] the outlet is close to an area where the flow is not established
in space (or varies in time);
\item [-] the outlet is not perpendicular to the flow~;
\item [-] the pressure gradient in the direction normal to the outlet is not
the same for all output faces (in the case of multiple outputs, for example, the gradient
is probably not the same across all outputs);
\item [-] points $I'$ are not on a surface parallel to the output
(case of irregular meshes for example).
\end{itemize}

Moreover, in the absence of an outlet condition, it could perhaps
prove useful to set a reference value on a given cell
or to bring the average pressure back to a reference value (with the
shift of the spectrum, one ensures the invertibility of the matrix at each step
time, but we should also check whether the pressure is not likely to drift during
of the calculation).

\etape{Terms unaccounted for}
The current boundary conditions seem to cause issues when treating the transposed
velocity gradient term in the Navier-Stokes equations (handled as explicit term in the
time scheme). This term can be deactivated setting the \var{ivisse} keyword to $0$
in the velocity-pressure model.

%-------------------------------------------------------------------------------

% This file is part of Code_Saturne, a general-purpose CFD tool.
%
% Copyright (C) 1998-2020 EDF S.A.
%
% This program is free software; you can redistribute it and/or modify it under
% the terms of the GNU General Public License as published by the Free Software
% Foundation; either version 2 of the License, or (at your option) any later
% version.
%
% This program is distributed in the hope that it will be useful, but WITHOUT
% ANY WARRANTY; without even the implied warranty of MERCHANTABILITY or FITNESS
% FOR A PARTICULAR PURPOSE.  See the GNU General Public License for more
% details.
%
% You should have received a copy of the GNU General Public License along with
% this program; if not, write to the Free Software Foundation, Inc., 51 Franklin
% Street, Fifth Floor, Boston, MA 02110-1301, USA.

%-------------------------------------------------------------------------------

\programme{covofi}\label{ap:covofi}

\hypertarget{covofi}{}

\vspace{1cm}
%-------------------------------------------------------------------------------
\section*{Fonction}
%-------------------------------------------------------------------------------
Dans ce sous-programme, on r\'{e}sout : \\
{\tiny$\bigstar$} soit l'\'{e}quation de convection-diffusion d'un scalaire en
pr\'{e}sence de termes sources :
%\begin{equation}\label{Base_Covofi_EQ_cvd)
\begin{equation}
\frac {\partial  (\rho a)}{\partial t} +
\underbrace{\,\dive\,((\rho \underline{u})\,a)}_{\text{convection}}
- \underbrace{\,\dive\,(K \grad a)}_{\text{diffusion}} = T_s^{\,imp} a
+T_s^{\,exp} +\Gamma\,a_i
\end{equation}
Ici $a$ repr\'{e}sente la valeur instantan\'{e}e du scalaire en approche laminaire ou,
en approche RANS, sa moyenne de Reynolds $\widetilde{a}$. Les deux approches
\'{e}tant exclusives et les \'{e}quations obtenues similaires, on utilisera le plus
souvent aussi la notation $a$ pour $\widetilde{a}$.\\
{\tiny$\bigstar$} soit, dans le cas d'une mod\'{e}lisation RANS, la variance de la
fluctuation d'un scalaire en pr\'{e}sence de termes sources\footnote{Davroux A. et
Archambeau F. : Calcul de la variance des fluctuations
d'un scalaire dans le solveur commun. Application \`{a} l'exp\'{e}rience du CEGB dite
``Jet in Pool'', HE-41/99/043.} :
\begin{equation}
\begin{array}{lcl}
&\displaystyle
 \frac {\partial  (\rho \widetilde{{a"}^2})}{\partial t} +
\underbrace{\dive\,((\rho\,\underline{u})\ \widetilde{{a"}^2})}_{\text{convection}}
- \underbrace{\dive\,(K\ \grad \widetilde{{a"}^2})}_{\text{diffusion}} = T_s^{\,imp} \widetilde{{a"}^2}
+T_s^{\,exp} +\ \Gamma\,\widetilde{{a"}^2}_i \\
&\displaystyle \underbrace {+ 2\,\frac{\mu_t}{\sigma_t}(\grad \widetilde{a})^2 -
\frac{\rho\,\varepsilon}{R_f k}\ \widetilde{{a"}^2}}_{\text{termes de production et
de dissipation dus \`{a} la turbulence moyenne}}
\end{array}
\end{equation}
$\widetilde{{a"}^2}$ repr\'esente ici la moyenne du carr\'e des fluctuations\footnote{$a$ et
$\widetilde{{a"}^2}$, sous forme discr\`ete en espace, correspondent donc en
fait \`a des vecteurs dimensionn\'es \`a \var{NCELET} de composantes $a_I$ et $\widetilde{{a"}^2}_{I}$
respectivement, I d\'ecrivant l'ensemble des cellules.} de $a$. $K$, $\Gamma$,
$T_s^{imp}$ et  $T_s^{exp}$ repr\'{e}sentent respectivement le coefficient de
diffusion, la valeur du terme source de masse, les termes sources implicite et
explicite du scalaire $a$ ou $\widetilde{{a"}^2}$. $\mu_t$ et $\sigma_t$
sont respectivement la viscosit\'{e} turbulente et le nombre de Schmidt ou de
Prandtl turbulent, $\varepsilon$ est la dissipation de l'\'{e}nergie turbulente $k$
et $R_f$ d\'{e}finit le rapport constant entre les \'{e}chelles dissipatives de $k$ et
de $\widetilde{{a"}^2}$ ($R_f$ est constant selon le mod\`{e}le assez simple adopt\'{e} ici).\\
On \'{e}crit les deux \'{e}quations pr\'{e}c\'{e}dentes sous la forme commune suivante~:
\begin{equation}
\frac {\partial  (\rho f)}{\partial t} + \dive\,((\rho\,\underline{u}) f)
- \dive\,(K \grad f) = T_s^{\,imp} f + T_s^{\,exp} + \Gamma\,f_i + T_s^{\,pd}
\end{equation}
avec, pour $f=a$ ou $f=\widetilde{{a"}^2}$ :\\
\begin{equation}
\begin{array}{lll}
&\displaystyle
T_s^{\,pd}=
\begin{cases}
0 & \text{pour $f=a$}, \\
2\ \displaystyle \frac{\mu_t}{\sigma_t}(\grad \widetilde{a})^2 -
\displaystyle \frac{\rho\,\varepsilon}{R_f k}\ \widetilde{{a"}^2} & \text{pour $f=\widetilde{{a"}^2}$ }
\end{cases}
\end{array}
\end{equation}

Le terme $\displaystyle \frac {\partial  (\rho f)}{\partial t}$ est d\'{e}compos\'{e} de la sorte :
\begin{equation}
\frac {\partial  (\rho f)}{\partial t}=\rho \frac {\partial f}{\partial t} + f
\frac {\partial \rho}{\partial t}
\end{equation}
En utilisant l'\'{e}quation de conservation de la masse (cf. \fort{predvv}),
l'\'{e}quation pr\'{e}c\'{e}dente s'\'{e}crit finalement :\\
\begin{equation}\label{Base_Covofi_Eq_cv_scal}
\rho\ \displaystyle \frac {\partial f}{\partial t} +
\dive\,((\rho\,\underline{u})\,f) - \dive\,(K\ \grad f)
= T_s^{\,imp} f + T_s^{\,exp} + \Gamma (f_i - f) + T_s^{\,pd} + f\,\dive\,(\rho\,\underline{u})
\end{equation}

See the \doxygenfile{covofi_8f90.html}{programmers reference of the dedicated subroutine} for further details.

%%%%%%%%%%%%%%%%%%%%%%%%%%%%%%%%%
%%%%%%%%%%%%%%%%%%%%%%%%%%%%%%%%%%
\section*{Discr\'etisation}
%%%%%%%%%%%%%%%%%%%%%%%%%%%%%%%%%%
%%%%%%%%%%%%%%%%%%%%%%%%%%%%%%%%%%
Pour int\'{e}grer l'\'{e}quation (\ref{Base_Covofi_Eq_cv_scal}), une discr\'{e}tisation temporelle de
type $\theta$-sch\'{e}ma est appliqu\'{e}e \`{a} la variable r\'{e}solue\footnote{Si
$\theta=1/2$, ou qu'une extrapolation est utilis\'{e}e, le pas de temps $\Delta t$
est suppos\'{e} uniforme en temps et en espace.}~:
\begin{equation}
f^{n+\theta} = \theta \,\, f^{n+1} + (1-\theta)\,\, f^{n}
\end{equation}

L'\'{e}quation (\ref{Base_Covofi_Eq_cv_scal}) est discr\'etis\'ee au temps $n+\theta$ en
supposant les termes sources explicites pris au temps $n+\theta_{S}$, et ceux
implicites en $n+\theta$.
Par souci de clart\'{e}, on suppose, en l'absence d'indication, que les propri\'{e}tes
physiques $\Phi$ ($K,\,\rho$,...) et le flux de masse $(\rho\,\underline{u})$
sont pris respectivement aux instants $n+\theta_\Phi$ et $n+\theta_F$, o\`{u}
$\theta_\Phi$ et $\theta_F$ d\'{e}pendent des sch\'{e}mas en temps sp\'{e}cifiquement
utilis\'{e}s pour ces grandeurs\footnote{cf. \fort{introd}}.

\begin{equation}
\begin{array}{lcl}
&\displaystyle
 \rho \frac {f^{n+1}-f^{n}}{\Delta t} +
\underbrace{\dive\,((\rho\,\underline{u})\,f^{n+\theta})}_{\text{convection}}
- \underbrace{\dive\,(\,K\,\grad f^{n+\theta})}_{\text{diffusion}} =
T_s^{\,imp}\,f^{n+\theta} + T_s^{\,exp,\,n+\theta_{S}}\\
& + (\Gamma\,f_i)^{n+\theta_{S}}-\Gamma^{n}\,f^{n+\theta} +\
T_s^{\,pd,\,n+\theta_S} + f^{n+\theta}\,\dive\,(\rho\underline{u})
\end{array}
\end{equation}
o\`{u} :
\begin{equation}
 T_s^{\,pd,\,n+\theta_S} =
\begin{cases}
0 & \text{pour $f=a$}, \\
2 \displaystyle \left[\frac{\mu_t}{\sigma_t}(\grad \widetilde{a})^2\right]^{n+\theta_S}-\frac{\rho\,
\varepsilon^{n}}{R_f\,k^{n}}(\widetilde{{a"}^2})^{n+\theta}& \text{pour $f=\widetilde{{a"}^2}$ }
\end{cases}
\end{equation}
Le terme de production affect\'{e} d'un indice $n+\theta_{S}$ est un terme source
explicite et il est donc trait\'{e} comme tel :
\begin{equation}
\begin{array}{rll}
\displaystyle
\left[\frac{\mu_t}{\sigma_t}(\grad
\widetilde{a})^2\right]^{n+\theta_{S}}&=&\displaystyle
(1+\theta_{S})\,\,\frac{\mu_t^{n}}{\sigma_t}(\grad
\widetilde{a}^{n})^2-\theta_{S}\,\,\frac{\mu_t^{n-1}}{\sigma_t}(\grad
\widetilde{a}^{n-1})^2\\
\end{array}
\end{equation}
\\

L'\'{e}quation (\ref{Base_Covofi_Eq_cv_scal}) s'\'ecrit :
\begin{equation}\label{Base_Covofi_Eq_scal_tempo}
\begin{array}{c}
\displaystyle
 \rho\,\frac {f^{n+1}-f^{n}}{\Delta t} +
\theta \,\,\dive\,((\rho\,\underline{u})\,f^{n+1})- \theta \,\,\dive\,(\,K\ \grad f^{n+1})
\\
-\left[ \theta\,\, T_s^{\,imp}\,- \theta\,\, \Gamma^{n} + \theta\,\, T_s^{\,pd,\,imp}+\theta\,\,
\dive\,(\rho\ \underline{u})\right]\,f^{n+1}
\\
= (1-\theta)\,\,T_s^{\,imp}\,f^{n} + T_s^{\,exp,\,n+\theta_S} +
(\Gamma\,f_i)^{n+\theta_S}-(1-\theta)\,\,\Gamma^{n}\,
f^{n}+ T_s^{\,pd,\,exp}-\theta\,T_s^{\,pd,\,imp}\,f^{n}
\\
+ (1-\theta) \,\, f^{n}\,\dive\,(\rho\ \underline{u})- (1-\theta) \,\, \dive\,((\rho\,\underline{u})\,f^{n})
+ (1-\theta)\,\, \dive\,(\,K\ \grad f^{n})
\end{array}
\end{equation}
avec :
\begin{equation}
T_s^{\,pd,\,imp} =
\begin{cases}
0 & \text{pour $f=a$}, \\
- \displaystyle \frac{\rho\,\varepsilon^n}{R_f \,k^n} &  \text{pour $f=\widetilde{{a"}^2}$}
\end{cases}
\end{equation}
\begin{equation}
T_s^{\,pd,\,exp}=
\begin{cases}
0 & \text{pour $f=a$}, \\
2\ \displaystyle\left[\frac{\mu_t}{\sigma_t}(\grad
\widetilde{a})^2\right]^{n+\theta_S} -
\frac{\rho\,\varepsilon^n}{R_f\,k^n}(\widetilde{{a"}^2})^n & \text{pour
$f=\widetilde{{a"}^2}$}
\end{cases}
\end{equation}
On rappelle que, pour un scalaire $f$, le sous-programme \fort{codits}
r\'{e}sout une \'{e}quation du type suivant
\label{Base_Covofi_Eq_Codits}
\begin{equation}
\begin{array}{c}
\displaystyle f_s^{\,imp} (f^{n+1} - f^{n}) +
\theta \,\, \dive((\rho\,\underline{u})\,f^{n+1})- \theta \,\, \dive\,(\,K\,\grad f^{n+1})
\\
= f_s^{\,exp} -\underbrace{(1-\theta) \,\, \dive((\rho\,\underline{u})\,f^{n}) + (1-\theta)
\,\, \dive\,(\,K\,\grad f^{n})}_{\text{convection diffusion explicite}}
\end{array}
\end{equation}
$f_s^{exp}$ repr\'{e}sente les termes sources discr\'etis\'es de mani\`ere explicite
en temps (hormis contributions de la convection diffusion explicite provenant du
$\theta$-sch\'ema) et $f_s^{imp}\,f^{n+1}$ repr\'esente les termes lin\'eaires
en $f^{n+1}$ dans l'\'equation discr\'etis\'ee en temps.\\
On r\'{e}\'{e}crit l'\'{e}quation (\ref{Base_Covofi_Eq_scal_tempo}) sous la forme (\ref{Base_Covofi_Eq_scal_final})
qui est ensuite r\'{e}solue par \fort{codits}.
\begin{equation}
\label{Base_Covofi_Eq_scal_final}
\begin{array}{c}
\displaystyle
\underbrace{\left(\frac {\rho}{\Delta t}- \theta\,\, T_s^{\,imp}+ \theta\,\,
\Gamma^{n} -\theta\,\, T_s^{\,pd,\,imp} - \theta\,\,
\dive\,(\rho\,\underline{u})\right)}_{\text {$f_s^{\,imp}$}}\ \delta f^{n+1}
\\
+\theta\,\, \dive(\,(\rho \underline{u})\,f^{n+1}\,)
-\theta\,\, \dive\,(K\,\grad \,f^{n+1}) = \\
\underbrace{T_s^{\,imp}\,f^n +  T_s^{\,exp,\,n+\theta_S}
+\,(\Gamma f_i)^{n+\theta_S}\, - \Gamma^{n}\,f^n +\ T_s^{\,pd,\,exp} +
 f^{n}\,\dive(\rho\,\underline{u})}_{\text{$f_s^{exp}$}}\\
-(1-\theta)\,\dive(\,(\rho \underline{u})\,f^{n}\,)
+(1-\theta)\,\dive\,(K\,\grad f^{n})
\end{array}
\end{equation}
\\
%Pour la discr\'{e}tisation spatiale de ce syst\`{e}me, on pourra se reporter au
%sous-programme \fort{navstv}

%%%%%%%%%%%%%%%%%%%%%%%%%%%%%%%%%%
%%%%%%%%%%%%%%%%%%%%%%%%%%%%%%%%%%
\section*{Mise en \oe uvre}
%%%%%%%%%%%%%%%%%%%%%%%%%%%%%%%%%%
%%%%%%%%%%%%%%%%%%%%%%%%%%%%%%%%%%
On distingue deux cas suivant le type de sch\'{e}ma en temps choisi pour les termes sources :
\\
$\bullet$ Si les termes sources ne sont pas extrapol\'{e}s, toutes les contributions
du second membre vont directement dans le vecteur
\var{SMBRS}.\\
$\bullet$ Sinon, un vecteur suppl\'{e}mentaire est n\'{e}cessaire afin de stocker les
contributions du pas de temps pr\'{e}cedent (\var{PROPCE}). Dans ce cas :
\begin{itemize}
\item [-] le vecteur \var{PROPCE} sert \`{a} stocker les contributions explicites du
second membre au temps $n-1$ (pour l'extrapolation en $n+\theta_S$).
\item [-] le vecteur \var{SMBRS} est compl\'{e}t\'{e} au fur et \`{a} mesure.
\\
\end{itemize}
L'algorithme de ce sous-programme est le suivant :
\begin{itemize}
\item mise \`{a} z\'{e}ro des vecteurs repr\'{e}sentant le second membre (\var{SMBRS}) et
de la diagonale de la matrice (\var{ROVSDT}).
\item calcul des termes sources du scalaire d\'{e}finis par l'utilisateur en
appelant le sous-programme \fort{ustssc}.
\\\\
$\star$ Si les termes sources sont extrapol\'{e}s, \var{SMBRS} re�oit $-\theta_S$
fois la contribution au temps $n-1$ des termes sources qui sont extrapol\'{e}s
(stock\'{e}s dans \var{PROPCE}). La contribution des termes sources utilisateurs (au
pas temps courant) est r\'{e}partie entre \var{PROPCE} (pour la partie $T_s^{exp}$
qui est \`{a} stocker en vue de l'extrapolation) et \var{SMBRS} (pour la partie
explicite provenant de l'utilisation du $\theta$ sch\'{e}ma pour $T_s^{imp}$). La
contribution implicite est alors mise dans \var{ROVSDT} (apr\`{e}s multiplication
par $\theta$) quel que soit son signe, afin de ne pas utiliser des
discr\'{e}tisations temporelles diff\'{e}rentes entre deux pas de temps successifs, dans
le cas par exemple o\`{u} $T_s^{imp}$ change de signe\footnote{cf. \fort{predvv}}.
\\\\
$\star$ Sinon la contibution de $T_s^{exp}$ est directement mise dans
\var{SMBRS}. Celle de $T_s^{imp}$ est ajout\'{e}e \`{a} \var{ROVSDT} si elle est
positive (de mani\`{e}re \`{a} conserver la dominance de la diagonale), ou explicit\'{e}e et
mise dans le second membre sinon.
\\
\item prise en compte des physiques particuli\`{e}res : arc \'{e}lectrique, rayonnement,
combustion gaz et charbon pulv\'{e}ris\'{e}. Seuls les vecteurs \var{ROVSDT} et
\var{SMBRS} sont compl\'{e}t\'{e}s (sch\'{e}ma d'ordre 1 sans extrapolation).
\item ajout des termes sources de masse en $\Gamma\,(f_i-f)$ par appel au sous-programme \fort{catsma}.
\\\\
$\star$ Si les termes sources sont extrapol\'{e}s, le terme explicite en
$\Gamma\,f_i$ est stock\'{e} dans \var{PROPCE}. Le $\theta$-sch\'{e}ma est appliqu\'{e} au
terme implicite, puis les contributions implicite et explicite r\'{e}parties entre
\var{ROVSDT} et \var{SMBRS}.
\\\\
$\star$ Sinon, la partie implicte en $-\Gamma\,f$ va dans \var{ROVSDT}, et tout le reste dans \var{SMBRS}.
\\
\item calcul du terme d'accumulation de masse en $\dive(\rho \underline{u})$ par
appel \`a \fort{divmas} et ajout de sa contribution dans \var{SMBRS}, et dans
\var{ROVSDT} apr\`{e}s multiplication par $\theta$\footnote{cette op\'{e}ration est
faite quel que soit le sch\'{e}ma en temps de fa�on \`{a} rester coh\'{e}rent avec ce qui
est fait dans \fort{bilsc2}}.

\item ajout du terme instationnaire \`{a} \var{ROVSDT}.

\item calcul des termes de production ($2
\displaystyle\frac{\mu_t}{\sigma_t}(\grad \widetilde{a})^2$) et de dissipation
($\displaystyle - \frac{\rho
\varepsilon}{R_f k}\widetilde{{a"}^2}$) si on \'{e}tudie la variance des
fluctuations d'un scalaire avec un mod\`{e}le de turbulence de type
RANS. Ce calcul s'effectue en calculant pr\'{e}alablement
le gradient du scalaire $f$ par appel au sous-programme \fort{grdcel}.
\\\\
$\star$ Si les termes sources sont extrapol\'{e}s, la production est mise dans
\var{PROPCE} puis l'\'{e}nergie cin\'{e}tique $k$ et la dissipation turbulentes
$\varepsilon$ sont calcul\'{e}es (\var{XK} et \var{XE}) en
fonction du mod\`{e}le de turbulence utilis\'{e}. \var{SMBRS} re�oit
$\displaystyle - \frac{\rho \varepsilon}{R_f k}\widetilde{{a"}^2}$ au temps $n$
et \var{ROVSDT} le coefficient d'implicitation
$\displaystyle \frac{\rho \varepsilon}{R_f k}$ apr\`{e}s multiplication par
\var{THETAP} = $\theta$.
\\\\
$\star$ Sinon, la production va dans \var{SMBRS}, et la dissipation est r\'{e}partie
de la m\^eme mani\`{e}re que pr\'{e}c\'{e}demment avec \var{THETAP} = 1.
\\
\item une fois la contribution de tous les termes sources calcul\'{e}e, le second
membre est assembl\'{e}, et le vecteur \var{PROPCE} ajout\'{e} apr\`{e}s multiplication par
$1+\theta_S$ \`{a} \var{SMBRS}, dans le cas o\`{u} les termes sources sont extrapol\'{e}s.

\item calcul du coefficient de diffusion $K$ au centre des cellules, et des
valeurs aux faces par appel au sous-programme \fort{viscfa}.

\item r\'{e}solution de l'\'{e}quation compl\`{e}te (avec les termes de convection
diffusion) par un appel au sous-programme \fort{codits} avec
$f_s^{exp}=\var{SMBRS}$ et $f_s^{imp}=\var{ROVSDT}$.

\item ajustement (clipping) du scalaire ou de la fluctuation du scalaire en
appelant le sous-programme \fort{clpsca}.

\item impression du bilan explicite d'expression
$||\mathcal{E}_{n}(f^n)\,- \displaystyle \frac {\rho^n}{\Delta t} (\,f^{\,n+1} -
f^n\,)|| $ , o\`u $|| . ||$ d\'esigne la norme euclidienne.
\\\\
\end{itemize}

On r\'{e}sume dans les tableaux \ref{Base_Covofi_tab_ext} et \ref{Base_Covofi_tab_exp} les diff\'{e}rentes
contributions (hors convection-diffusion) affect\'{e}es \`{a} chacun des vecteurs
\var{PROPCE}, \var{SMBRS} et \var{ROVSDT} suivant le sch\'{e}ma en temps choisi pour
les termes sources. En l'absence d'indication, les propri\'{e}t\'{e}s physiques
$\rho,\mu,\,...$ sont suppos\'{e}es prises en  au temps $n+\theta_\Phi$, et le flux
de masse $(\,\rho \underline{u})$ pris au temps $n+\theta_F$, les valeurs de
$\theta_F$ et de $\theta_\Phi$ d\'{e}pendant du type de sch\'{e}ma s\'{e}lectionn\'{e}
sp\'{e}cifiquement pour ces grandeurs\footnote{cf. \fort{introd}}.
\\

\minititre{Avec extrapolation des termes sources :}
\begin{equation}\label{Base_Covofi_tab_ext}
\begin{array}{|l|c|}
\hline
\var{ROVSDT}^{n} &
\displaystyle
\frac{\rho}{\Delta t}-\theta\,T_s^{\,imp}- \theta\,\dive(\,\rho \underline{u}) +
\theta\,\Gamma^{n}+\theta\,\frac{\rho\,\varepsilon^{n}}{R_f\,k^{n}} \\
\hline
\var{PROPCE}^{n} &
\displaystyle
T_s^{\,exp,\,n} + \Gamma^{n}\,f_i^{n} + 2\, \frac{\mu_t^{n}}{\sigma_t}(\grad f^{n})^2\\
\hline
\var{SMBRS}^{n} &
\displaystyle
(1+\theta_S)\,\var{PROPCE}^{n}-\theta_S\,\var{PROPCE}^{n-1}+ T_s^{\,imp}\,f^{n}
+\dive(\,\rho \underline{u})\,f^{n}-\Gamma^{n}\,f^{n} -
\frac{\rho\,\varepsilon^{n}}{R_f\,k^{n}}\,f^{n}\\
\hline
\end{array}
\end{equation}

\minititre{Sans extrapolation des termes sources :}
\begin{equation}\label{Base_Covofi_tab_exp}
\begin{array} {|l|c|}
\hline
\var{ROVSDT}^{n} &
\displaystyle
\frac{\rho}{\Delta t} + Max(-T_s^{\,imp},0) - \theta\,\dive(\,\rho
\underline{u}) + \Gamma^{n} + \frac{\rho\,\varepsilon^{n}}{R_f\,k^{n}} \\
\hline
\var{SMBRS}^{n} &
\displaystyle
T_s^{\,exp} + T_s^{\,imp}\,f^{n}+\dive(\,\rho
\underline{u})\,f^{n}+\Gamma^{n}\,(\,f_i^{n}-f^{n}) -
\frac{\rho\,\varepsilon^{n}}{R_f\,k^{n}}\,f^{n} + 2\,
\frac{\mu_t}{\sigma_t}(\grad f^{n})^2 \\
\hline
\end{array}
\end{equation}
%\underline{Remarque :}
%\\
%Le $\theta$ en facteur du terme de compressibilt\'{e} provient de la fa�on dont est compl\'{e}t\'{e} le second membre lors de l'appel au sous-programme \fort{bilcs2}.
%%%%%%%%%%%%%%%%%%%%%%%%%%%%%%%%%%%%%%%%%%%%%%%
%%%%%%%%%%%%%%%%%%%%%%%%%%%%%%%%%%%%%%%%%%%%%%%
\section*{Points \`{a} traiter}\label{Base_Covofi_section4}
%%%%%%%%%%%%%%%%%%%%%%%%%%%%%%%%%%%%%%%%%%%%%%%
%%%%%%%%%%%%%%%%%%%%%%%%%%%%%%%%%%%%%%%%%%%%%%%
\etape{Int\'{e}gration du terme de convection-diffusion}
Dans ce sous-programme, les points litigieux sont dus \`{a} l'int\'{e}gration du
terme de convection-diffusion. On renvoie donc le lecteur au sous-programme
\fort{bilsc2} qui les explicite.
%\pagebreak
\clearpage
%%%%%%%%%%%%%%%%%%%%%%%%%%%%%%%%%%%%%%%%%%%%%%%%%%%%%%%%%%%%%%%%%%%%%%%%%%%%%%
%%%%%%%%%%%%%%%%%%%%%%%%%%%%%%%%%%%%%%%%%%%%%%%%%%%%%%%%%%%%%%%%%%%%%%%%%%%%%%
\section*{Annexe 1 : Inversibilit\'e de la matrice $\tens{EM}_{\,n}$ }
%%%%%%%%%%%%%%%%%%%%%%%%%%%%%%%%%%%%%%%%%%%%%%%%%%%%%%%%%%%%%%%%%%%%%%%%%%%%%%
%%%%%%%%%%%%%%%%%%%%%%%%%%%%%%%%%%%%%%%%%%%%%%%%%%%%%%%%%%%%%%%%%%%%%%%%%%%%%%%
Dans cette section, on va \'etudier plus particuli\`{e}rement l'inversibilit\'e de
la matrice $\tens{EM}_{\,n}$, matrice du syst\`eme lin\'eaire
\`a r\'esoudre associ\'ee \`a $\mathcal{EM}_{n}$ pour le cas d'un sch\'{e}ma en temps
de type Euler implicite d'ordre un ($\theta=1$). Pour toutes les notations, on
se reportera \`{a} la documentation sur le sous-programme \fort{covofi}.
On va montrer que la d\'emarche adopt\'ee permet de
s'assurer que la matrice des syst\`emes de convection-diffusion dans les cas
courants est toujours inversible.

%%%%%%%%%%%%%%%%
\subsection*{\bf Introduction }
%%%%%%%%%%%%%%%%

Pour montrer l'inversibilit\'e, on va utiliser le fait que la dominance
stricte de la diagonale l'implique\footnote{Ce faisant, on choisit cependant une condition forte
et la d\'emonstration n'est probablement pas optimale.}. On cherche donc \`a
d\'eterminer sous quelles conditions les matrices de convection diffusion sont
\`a diagonale strictement dominante.

On va montrer qu'en incluant
dans la matrice le terme en $\dive(\rho \,\vect{u})$
issu de $\displaystyle \frac {{\partial}\,\rho}{{\partial}\,t}$, on peut
\'etablir directement et exactement\footnote{Hormis
dans le cas de conditions aux limites mixtes, qu'il conviendrait d'examiner plus
en d\'etail.} la propri\'et\'e. Par contre, si ce terme n'est pas pris en compte dans la matrice,
il est n\'ecessaire de faire intervenir le pendant discret de la relation~:
\begin{equation}\label{Base_Covofi_Eq_Div_Int}
\displaystyle \int_{\Omega_i} \dive (\rho\,\vect{u})\, d\Omega = 0
\end{equation}
Cette relation n'est cependant v\'erifi\'ee au niveau discret qu'\`a la pr\'ecision du
solveur de pression pr\`es (et, en tous les cas, ne peut \^etre approch\'ee au
mieux
qu'\`a la pr\'ecision machine pr\`es). Il para\^\i t donc pr\'ef\'erable de s'en
affranchir. \\


Avant d'entrer dans les d\'etails de l'analyse, on rappelle quelques
propri\'et\'es et d\'efinitions.

Soit $\tens{C}$ une matrice carr\'ee d'ordre N,
d'\'el\'ement g\'en\'erique $C_{ij}$. On a par d\'efinition~:

$\underline{\text{D\'efinition~:}}$
La matrice $\tens{C}$ est \`{a} diagonale {\bf strictement dominante} {\it ssi}
\begin{equation}\label{Base_Covofi_Eq_Propriete_1}
\forall i \in [1,N],\ \ \ |C_{ii}| > \sum\limits_{j=1,\,j\neq i}^{j=N}|C_{ij}|
\end{equation}

On convient de dire que  $\tens{C}$ est \`{a} diagonale {\bf simplement dominante} {\it
ssi} l'in\'egalit\'e n'est pas stricte, soit~:
\begin{equation}
\forall i \in [1,N],\ \ \ |C_{ii}| \geqslant \sum\limits_{j=1,\,j\neq i}^{j=N}|C_{ij}|
\end{equation}

\underline{Remarque :} Si, sur chaque ligne, la somme des \'el\'ements d'une
matrice est nulle, que les \'el\'ements extradiagonaux sont
n\'egatifs et que les \'el\'ements diagonaux sont positifs, alors la matrice est
\`a diagonale simplement dominante. Si la somme est strictement positive, la
diagonale est strictement dominante.

On a l'implication suivante~:

$\underline{\text{Propri\'et\'e~:}}$
Si la matrice $\tens{C}$ est \`{a} diagonale strictement dominante, elle est
inversible. \\

Cette propri\'et\'e\footnote{Lascaux, P. et Th\'{e}odor,
R. : Analyse Num\'{e}rique Matricielle Appliqu\'{e}e \`{a} l'art de l'Ing\'{e}nieur, Tome 2,
Ed. Masson.} se d\'emontre simplement si l'on admet le th\'eor\`eme de
Gerschg\"orin ci-dessous~:

$\underline{\text{Th\'eor\`eme~:}}$
Soit $\tens{B}$ une matrice carr\'ee d'ordre N dans $\mathbb{C}\,\times\,\mathbb{C}$,
d'\'el\'ement g\'en\'erique $B_{\,ij}$, les valeurs propres $\lambda_l$ de $B$ sont, dans
le plan complexe, telles que
$||\lambda_l - B_{ii}||_{\,\mathbb{C}} \leqslant \sum\limits_{j=1,\,j\neq
i}^{j=N}{||B_{ij}||_{\,\mathbb{C}}}$

Si $B$ est \`a \'el\'ements r\'eels, on \'ecrira $||\lambda_l - B_{ii}||_{\,\mathbb{C}}
\leqslant \sum\limits_{j=1,\,j\neq i}^{j=N}|B_{ij}|$

$\underline{\text{D\'emontration de la propri\'et\'e pr\'ec\'edente~:}}$

Soit $C$ \`a diagonale strictement dominante \`a \'el\'ements r\'eels.
On montre qu'il est possible
d'inverser le syst\`eme $CX=S$ d'inconnue $X$, quel que soit le second membre
$S$. Pour cela, on d\'ecompose $C$ en partie
diagonale ($D$) et extradiagonale ($-E$) soit~:
$$C=D-E$$
$C$ \'etant \`a diagonale strictement dominante, tous
ses \'el\'ements diagonaux sont non nuls. $D$ est donc inversible (et
les \'elements de l'inverse sont r\'eels). On
consid\`ere alors la suite\footnote{On reconna\^\i t la m\'ethode de Jacobi}~:

$$(X^n)_{n\in\mathbb{N}}, \text{\hspace*{1cm}avec\hspace*{1cm}} X^0=D^{-1}S
\text{\hspace*{1cm}et\hspace*{1cm}} DX^n=S+EX^{n-1}$$
On peut \'ecrire~:
$$X^n = \sum\limits_{k=0}^{k=n} \left(D^{-1}E\right)^k D^{-1}S$$
Cette somme converge si
le rayon spectral de $B=D^{-1}E$ est strictement inf\'erieur \`a 1. Or, la
matrice $C$ est \`a diagonale strictement dominante. On a donc pour tout $i\in\mathbb{N}$
(\`a partir de la relation (\ref{Base_Covofi_Eq_Propriete_1}) et en
divisant par $|C_{ii}|$)~:
$$\forall i \in [1,N],\ \ \ \frac{|C_{ii}|}{|C_{ii}|} > \sum\limits_{j=1,\,j\neq
i}^{j=N}\frac{|C_{ij}|}{|C_{ii}|}$$ ce qui s'\'ecrit encore~:
$$\forall i \in [1,N],\ \ \ \frac{|D_{ii}|}{|D_{ii}|} > \sum\limits_{j=1,\,j\neq
i}^{j=N}\frac{|E_{ij}|}{|D_{ii}|}=\sum\limits_{j=1,\,j\neq
i}^{j=N}|\left[D^{-1}E\right]_{ij}|$$ ou bien~:
$$\forall i \in [1,N],\ \ \  1 > \sum\limits_{j=1,\,j\neq
i}^{j=N}|B_{ij}|$$  d'o\`u, avec le th\'eor\`eme de Gerschg\"orin, une
relation sur les valeurs propres $\lambda_l$ de $B$~:
$$\forall i \in [1,N],\ \ \  ||\lambda_l - B_{ii}||_{\,\mathbb{C}} \leqslant \sum\limits_{j=1,\,j\neq i}^{j=N}|B_{ij}| < 1 $$
et comme $B_{ii}=0$~:
$$ ||\lambda_l ||_{\,\mathbb{C}}  < 1 $$
en particulier, la valeur propre dont la norme est la plus grande v\'erifie
\'egalement cette
\'equation. Ceci implique que le rayon spectral de $B$ est strictement
inf\'erieur \`a 1. La suite $(X^n)_{n\in\mathbb{N}}$ converge donc (et la m\'ethode
de Jacobi converge). Il existe donc une solution \`a l'\'equation  $CX=S$. Cette
solution est unique\footnote{On peut le voir ``par l'absurde''.
En effet, supposons qu'il existe deux solutions
distinctes $X_1$ et $X_2$ \`a l'\'equation  $CX=S$. Alors $Y=X_2-X_1$ v\'erifie
$CY=0$, soit $DY=-EY$, donc $D^{-1}EY=-Y$. Ceci signifie que $Y$
(qui n'est pas nul, par
hypoth\`ese) est vecteur propre de $D^{-1}E$ avec $\lambda=-1$ pour valeur
propre associ\'ee. Or, le rayon spectral de $D^{-1}E$ est strictement
inf\'erieur \`a 1 et $\lambda=-1$ ne peut donc pas \^etre une valeur propre de
$D^{-1}E$.  En cons\'equence, il ne peut exister qu'une seule solution
\`a l'\'equation  $CX=S$.} et la matrice $C$ est donc inversible.


%%%%%%%%%%%%%%%%
\subsection*{Avec prise en compte des termes issus de
$\displaystyle \frac{{\partial}\,\rho}{{\partial}\,t}$ dans  $\tens{EM}_{\,n}$}
%%%%%%%%%%%%%%%%
\label{Base_Covofi_Avecdrhodt}
%---------------
\subsubsection*{Introduction}
%---------------
Pour montrer que la matrice $\tens{EM}_{\,n}$ est inversible, on va
montrer qu'elle est \`a diagonale strictement dominante. Pour cela, on
va consid\'erer successivement les contributions~:\\
\hspace*{1cm}- des termes diff\'erentiels
d'ordre 0 lin\'eaires en $\delta f^{\,n+1,k+1}$,\\
\hspace*{1cm}- des termes issus de la prise en compte de
$\displaystyle \frac {{\partial}\,\rho}{{\partial}\,t}$,\\
\hspace*{1cm}- des termes diff\'erentiels d'ordre 1 (convection),\\
\hspace*{1cm}- des termes diff\'erentiels d'ordre 2 (diffusion).

Pour chacune de ces contributions, on va examiner la
dominance de la diagonale de l'op\'erateur lin\'eaire associ\'e.
Si, pour chaque contribution, la dominance de la diagonale est acquise, on pourra
alors conclure \`a la dominance de la diagonale pour la matrice (somme)
compl\`ete\footnote{Ce raisonnement n'est pas optimal (la somme de valeurs
absolues \'etant sup\'erieure \`a la valeur absolue de la somme), mais permet
d'obtenir des conclusions dans le cas pr\'esent (condition
suffisante).}
$\tens{EM}_{\,n}$ et donc \`a son inversibilit\'e.


%---------------
\subsubsection*{Contributions des termes diff\'erentiels d'ordre 0 lin\'eaires en
$\delta f^{\,n+1,k+1}$}
%---------------
\label{Base_Covofi_ContributionTermesdOrdre0}
L'unique contribution est sur la diagonale~: il faut donc v\'erifier qu'elle
est strictement positive.

Pour chaque ligne $I$,  ${f_s^{\,imp}}_I $
 (cf. (\ref{Base_Covofi_Eq_scal_final})) contient au minimum la quantit\'e strictement
positive\footnote{Ceci permettra de conclure \`a la stricte dominance de la
diagonale de la matrice somme compl\`ete $\tens{EM}_{\,n}$.}
 $\displaystyle \frac {\rho_I^n\ |\Omega_i|}{\Delta t}$.
Les autres expressions,
($-\,|\Omega_i|\,(T_s^{\,imp})_I\ $, $\ +|\Omega_i|\, \Gamma_I\ $, $\
-\,|\Omega_i|\,{(T_s^{\,pd,imp})}_{\,I}$),
lorsqu'elles existent, contribuent toutes positivement\footnote{Le terme de
dissipation $\rho\frac{1}{R_f}\,\frac{\varepsilon}{k}$, sp\'ecifique \`a l'\'{e}tude de la
variance des fluctuations, est positif par d\'{e}finition et ne remet donc pas en cause la
conclusion.}.

L'op\'erateur lin\'eaire associ\'e \`a ces contributions
v\'erifie donc bien la {\bf dominance stricte} de la diagonale (propri\'et\'e
1). Ce n'est cependant pas vrai si on extrapole les termes source, \`{a} cause de
$T_s^{\,imp}$. Il en r\'{e}sulte une contrainte sur la valeur du pas de temps.

%
%---------------
\subsubsection*{Contributions  des termes
diff\'erentiels d'ordre 1 et des termes issus de la prise en compte de
$\displaystyle \frac {{\partial}\,\rho}{{\partial}\,t}$}
%---------------
\label{Base_Covofi_Contributionsdrhodtconvection}
Les termes consid\'er\'es sont au nombre de deux dans
(\ref{Base_Covofi_Eq_scal_tempo})~:\\
\hspace*{1cm}- la contribution issue de la prise en compte de $\displaystyle
\frac
{{\partial}\,\rho}{{\partial}\,t}$ se retrouve dans ${f_s^{\,imp}}_I $
(\'equation \ref{Base_Covofi_Eq_scal_final}),\\
\hspace*{1cm}- la contribution du terme de convection
lin\'earis\'e.


Apr\`es int\'{e}gration spatiale, la somme de ces deux termes discrets s'\'ecrit~:\\
%$C^{int}_{IJ}+C^{bord}_{b_{IK}}$, avec~:\\
\begin{equation}
\begin{array}{lll}\label{Base_Covofi_Eq_Avec_Faces_Int}
&\displaystyle \frac{1}{2}\sum\limits_{j\in Vois(i)}\left[(\ -\,m_{\,ij}^n + |\
m_{\,ij}^n|\ )\,\delta f_I^{\,n+1,k+1}+ (\ m_{\,ij}^n - |\ m_{\,ij}^n|)\,\delta f_J^{\,n+1,k+1}\right]\\
\end{array}
\end{equation}
\begin{equation}\label{Base_Covofi_Eq_Avec_Faces_Bord}
\begin{array}{lll}
&+\displaystyle\frac{1}{2}\sum\limits_{k\in {\gamma_b(i)}}\left[(\ -\,
m_{\,{b}_{ik}}^n + |\ m_{\,{b}_{ik}}^n|\ )\,\delta f_I^{\,n+1,k+1} + (\
m_{\,{b}_{ik}}^n - |m_{\,{b}_{ik}}^n|)\,\delta f_{\,{b}_{ik}}^{\,n+1,k+1}\right]\\
\end{array}
\end{equation}

Pour chaque ligne $I$, on va chercher les propri\'et\'es de dominance de la
diagonale en traitant s\'epar\'ement les faces internes (\'equation
(\ref{Base_Covofi_Eq_Avec_Faces_Int})) et les faces de bord
(\'equation~(\ref{Base_Covofi_Eq_Avec_Faces_Bord})).

\hspace*{0.5cm}$\bullet$ la contribution des {\bf faces internes} $ij$ (facteur de $\delta
f_I^{\,n+1,k+1}$) \`{a} la diagonale est positive ; la contribution aux
extradiagonales est n\'{e}gative (facteur de $\delta f_J^{\,n+1,k+1}$)
et la somme de ces contributions est exactement nulle (\'equation~
(\ref{Base_Covofi_Eq_Avec_Faces_Int})). Si l'on note $C_{IJ}$ les coefficients de la matrice
issus de la contribution de ces termes, on a donc $|C_{II}| \geqslant
\sum\limits_{J=1,\,J\neq I}^{J=N}|C_{IJ}|$ qui traduit la {\bf dominance ``simple''}
(l'in\'egalit\'e n'est pas ``stricte'') de la diagonale et r\`egle la question
des contributions des faces internes.

\hspace*{0.5cm}$\bullet$ la contribution des {\bf faces de bord} doit \^etre
r\'e\'ecrite en utilisant l'expression des conditions aux limites sur $f$
pour pr\'eciser la valeur de $\delta f_{\,b_{ik}}$ (on omet
l'exposant $(\,n+1,k+1)$ pour all\'eger les notations)~: \\
$\hspace*{1.5cm}$ - pour une condition de Dirichlet : $\delta f_{\,b_{ik}}\,=\,0$,\\
$\hspace*{1.5cm}$ - pour une condition de Neumann : $\delta f_{\,b_{ik}}\,=\,\delta f_I$, \\
$\hspace*{1.5cm}$ - pour une condition mixte ($f_{\,b_{ik}}\,=\,\alpha\,+\,\beta f_i$) : $\delta
f_{\,b_{ik}}\,=\,\beta\ \delta f_I$.\\

\hspace*{1cm}Pour la contribution des faces de bord, il faut alors consid\'erer deux cas de
figure possibles.
\begin{itemize}

\item {\bf Le flux de masse au bord est positif ou nul}
($\ m_{\,{b}_{ik}}^n = (\rho\
\underline{u})^{n}_{\,b_{ik}}\,.\,\underline{S}_{\,b_{ik}} \geqslant 0$). Cette
situation correspond par exemple aux sorties standards (fluide sortant du
domaine), aux sym\'etries ou aux parois \'etanches (flux de masse nul). Les contributions aux
faces de bord sont alors toutes nulles, quelles que soient les conditions aux limites
portant sur la variable $f$. En cons\'equence, la diagonale issue de ces
contribution est {\bf simplement dominante}.
\hspace*{0.5cm}
\item {\bf Le flux de masse au bord est strictement n\'{e}gatif}. Cette situation
correspond \`a une entr\'ee de fluide dans
le domaine. Les contributions consid\'er\'ees s'\'ecrivent alors~:
\begin{equation}
\displaystyle\sum\limits_{k\in {\gamma_b(i)}}\left[(\ -\,
m_{\,{b}_{ik}}^n\ )\,\delta f_I^{\,n+1,k+1} + (\
m_{\,{b}_{ik}}^n\ )\,\delta f_{\,{b}_{ik}}^{\,n+1,k+1}\right]
\end{equation}
Il convient alors de distinguer plusieurs situations, selon le type de condition
\`a la limite portant sur $f$~: \\
\hspace*{1.cm} {\tiny$\bigstar$} si la condition \`{a} la limite de $f$ est de type
{\bf Dirichlet}, seule subsiste une contribution positive ou nulle \`{a} la diagonale, qui assure
donc la {\bf dominance simple}~:
\begin{equation}
\displaystyle\sum\limits_{k\in {\gamma_b(i)}}(\ -\,
m_{\,{b}_{ik}}^n\ )\,\delta f_I^{\,n+1,k+1}
\end{equation}
\hspace*{1.cm} {\tiny$\bigstar$} si la condition \`{a} la limite de $f$ est de type
 {\bf Neumann}, la somme des contributions dues aux faces de bord
est alors nulle, ce qui assure
donc la {\bf dominance simple}.\\
\hspace*{1.cm} {\tiny$\bigstar$} si la condition \`{a} la limite de $f$ est de type
{\bf mixte}, la contribution des faces de bord est sur la diagonale et vaut~:
\begin{equation}
\displaystyle \frac{1}{2}\sum\limits_{k \in \gamma_b(i)}(1-\beta)
(\ -\,m_{\,{b}_{ik}}^n\ )\,\delta f_I^{\,n+1,k+1}
\end{equation}
On ne peut pas se prononcer quand \`a la dominance de la diagonale, \`a
cause de la pr\'esence de $(1-\beta)$ (la valeur de $\beta$ est fix\'ee par
l'utilisateur) et la d\'emarche adopt\'ee ici
{\bf ne permet donc pas de conclure}. Il faut n\'eanmoins noter que cette
situation est rare dans les calculs standards. Elle demande un
compl\'ement d'analyse et sera pour le moment exclue des
consid\'{e}rations expos\'ees dans le pr\'esent document.\\
\end{itemize}

{\bf On peut conclure}, quand il n'y a pas de condition \`a la limite de type mixte,
que la matrice associ\'ee aux contributions des termes
diff\'erentiels d'ordre 1 (convectifs) et \`a la prise en compte des termes
issus de $\displaystyle \frac{{\partial}\,\rho}{{\partial}\,t}$ et est \`a
diagonale {\bf simplement dominante}.




%---------------
\subsubsection*{Contributions des termes diff\'erentiels d'ordre 2}
%---------------

On va consid\'erer enfin les contributions des termes diff\'erentiels
d'ordre 2 (issus du terme \\
$-\ \dive\,(K^n\ \grad \delta f^{\,n+1,k+1})$).
Pour ces termes, la contribution  \`{a} la
diagonale est positive\footnote{\label{Base_Covofi_transmittivite}Ceci n'est en fait pas
toujours
vrai. En effet, pour chaque face $ij$, la transmittivit\'e
$\frac{K^n}{\overline{I'J'}}S_{ij}$
fait intervenir la mesure alg\'ebrique du segment $I'J'$, o\`u $I'$ et $J'$
sont les projet\'es orthogonaux sur la normale \`a la face du centre
des cellules voisines. Cette
grandeur est une valeur alg\'ebrique et peut th\'eoriquement devenir
n\'egative sur certains maillages pathologiques, contenant par exemple des
mailles non convexes. On pourra se reporter au dernier point \`a traiter du sous-programme
\fort{matrix}.},
n\'{e}gative aux extradiagonales$^{\text{\scriptsize \thefootnote}}$, compte tenu de~:
\begin{equation}
\begin{array}{ll}
&\int_{\Omega_i}-\ \dive\,(K^n\ \grad \delta f^{\,n+1,k+1})\,d\Omega\\
&= -\sum\limits_{j \in Vois(i)} K_{\,ij}^n
\displaystyle \frac{\delta f_{J}^{\,n+1,k+1} -\,\delta f_{I}^{\,n+1,k+1}}{\overline{I'J'}}\,.\,S_{ij}
-\sum\limits_{k \in \gamma_b(i)} K_{\,b_{ik}}^n
\displaystyle\frac{\delta f_{\,b_{ik}}^{\,n+1,k+1} -\,\delta f_{I}^{\,n+1,k+1}}{\overline{I'F}}\,.\,
S_{b_{ik}}
\end{array}
\end{equation}



Consid\'erons deux cas~:\\
\hspace*{1cm}- la cellule courante $I$ n'a {\bf que des faces internes} au domaine de
calcul (pas de faces de bord). La somme des contributions est nulle. On a donc
{\bf dominance simple} de la diagonale. \\
\hspace*{1cm}- la cellule courante $I$ a  {\bf des faces de bord}.  La somme des
contributions diagonales et extradiagonales est positive quand on a une
condition \`a la limite de type {\bf Dirichlet} ou de type {\bf Neumann} sur $f$. La
diagonale est alors {\bf strictement dominante}.
Lorsqu'il y a des conditions \`{a} la limite de type mixte, il n'est plus possible
de conclure (situation \'ecart\'ee pr\'ec\'edemment).\\

{\bf On peut conclure}, quand il n'y a pas de condition \`a la limite de type mixte,
que la matrice associ\'ee aux contributions des termes diff\'erentiels d'ordre 2
 est au moins \`a diagonale {\bf simplement dominante}.



%---------------
\subsubsection*{Conclusion}
%---------------
En travaillant sur
des maillages non pathologiques (\`a transmittivit\'e positive, voir la note de
bas de page num\'ero \ref{Base_Covofi_transmittivite}) et en n'imposant pas de condition \`a la limite
de type mixte sur les variables, on peut donc conclure que
$\tens{EM}_{\,n}$ est la somme de matrices \`a diagonale simplement
dominante et d'une matrice \`a diagonale strictement dominante (paragraphe
\ref{Base_Covofi_ContributionTermesdOrdre0}). Elle est donc  \`a {\bf diagonale strictement
dominante}, et donc {\bf inversible} (de plus, la
m\'ethode it\'erative de Jacobi converge).


\subsection*{Sans prise en compte des termes issus de
$\displaystyle \frac {{\partial}\,\rho}{{\partial}\,t}$ dans
$\tens{EM}_{\,n}$}


%---------------
\subsubsection*{Introduction}
%---------------

Pour identifier les cas dans lesquels la matrice $\tens{EM}_{\,n}$ est
inversible,  on va rechercher les
conditions qui assurent la dominance de la diagonale. Par rapport \`a l'analyse
pr\'esent\'ee au paragraphe \ref{Base_Covofi_Avecdrhodt}, seules diff\`erent les
consid\'erations relatives aux contributions des termes diff\'erentiels d'ordre
1, puisqu'elles sont trait\'ees au paragraphe
\ref{Base_Covofi_Contributionsdrhodtconvection} avec les termes issus de la prise en compte
de $\displaystyle\frac {{\partial}\,\rho}{{\partial}\,t}$.

%---------------
\subsubsection*{Contributions des termes diff\'erentiels d'ordre 1}
%---------------

La contribution du terme de convection est la seule \`a prendre en compte. Elle
s'\'ecrit, d'apr\`{e}s les \'{e}quations (\ref{Base_Covofi_Eq_scal_final}) et la discr\'{e}tisation
explicit\'{e}e pour le sous-programme \fort{covofi} :
\begin{equation}\label{Base_Covofi_Eq_Sans_Faces_Int}
\displaystyle\frac{1}{2}\sum\limits_{j\in Vois(i)}\left[(\ +\,m_{\,ij}^n + |\
m_{\,ij}^n|\ )\,\delta f_I^{\,n+1,k+1}+ (\ m_{\,ij}^n - |\ m_{\,ij}^n|)\,\delta f_J^{\,n+1,k+1}\right]
\end{equation}
\begin{equation}\label{Base_Covofi_Eq_Sans_Faces_Bord}
\displaystyle\frac{1}{2}\sum\limits_{k\in {\gamma_b(i)}}\left[(\ +\,
m_{\,{b}_{ik}}^n + |\ m_{\,{b}_{ik}}^n|\ )\,\delta f_I^{\,n+1,k+1} + (\
m_{\,{b}_{ik}}^n - |m_{\,{b}_{ik}}^n|)\,\delta f_{\,{b}_{ik}}^{\,n+1,k+1}\right]
\end{equation}


On constate que pour chaque ligne $I$,  la contribution des faces
internes (facteur de $\delta f_I^{\,n+1,k+1}$) \`a la diagonale est positive et
qu'elle est n\'egative aux extradiagonales (facteur de $\delta
f_J^{\,n+1,k+1}$). {\bf Cependant}, contrairement au cas pr\'esent\'e au
paragraphe~\ref{Base_Covofi_Contributionsdrhodtconvection}, la
somme de ces contributions n'est pas nulle dans le cas g\'en\'eral. Pour obtenir
un r\'esultat quant \`a la dominance de la diagonale, il faut faire intervenir
la version discr\`ete de la propri\'et\'e (\ref{Base_Covofi_Eq_Div_Int})
rappel\'ee ci-dessous~: $$\displaystyle \int_{\Omega_i} \dive (\rho\,\vect{u})\,
d\Omega = 0$$
Soit, sous forme discr\`ete~:
\begin{equation}\label{Base_Covofi_Eq_Continuite_discrete}
\sum\limits_{j\in Vois(i)}\,m_{\,ij}^n
+ \sum\limits_{k\in {\gamma_b(i)}}\,m_{\,{b}_{ik}}^n\ = 0
\end{equation}

Il n'est donc pas possible d'analyser
s\'epar\'ement les contributions des faces internes et celles des faces
de bord (contrairement \`a la situation rencontr\'ee au
paragraphe~\ref{Base_Covofi_Contributionsdrhodtconvection}). On se place ci-apr\`es dans le
cas g\'en\'eral d'une cellule qui a des faces internes {\em et} des faces de
bord (si elle n'a que des faces internes, la d\'emonstration est la m\^eme, mais
plus simple. On peut l'\'ecrire en consid\'erant formellement que la cellule
``a z\'ero faces de bord'', c'est \`a dire que $\gamma_b(i)$ est l'ensemble vide). \\

Il faut alors consid\'erer deux cas de figure, selon la valeur du flux de masse
aux faces de bord (\'eventuelles) de la cellule~:
\begin{itemize}

\item {\bf Le flux de masse au bord est positif ou nul} ($\ m_{\,{b}_{ik}}^n = (\rho\
\underline{u})^{n}_{\,b_{ik}}\,.\,\underline{S}_{\,b_{ik}} \geqslant 0$). Cette
situation correspond \`a des cellules qui ont des faces de bord de sortie
standard (fluide sortant du
domaine), de sym\'etrie ou de paroi \'etanche (flux de masse nul). Les
contributions s'\'ecrivent alors~:
\begin{equation}
\displaystyle\frac{1}{2}\sum\limits_{j\in Vois(i)}\left[(\ +\,m_{\,ij}^n + |\
m_{\,ij}^n|\ )\,\delta f_I^{\,n+1,k+1}+ (\ m_{\,ij}^n - |\ m_{\,ij}^n|)\,\delta f_J^{\,n+1,k+1}\right]
+\sum\limits_{k\in {\gamma_b(i)}}\ m_{\,{b}_{ik}}^n \,\delta f_I^{\,n+1,k+1}
\end{equation}
Dans ce cas, la somme des contributions \`a la diagonale est positive, les
contributions aux extradiagonales sont n\'egatives et, avec la relation
(\ref{Base_Covofi_Eq_Continuite_discrete}), on v\'erifie que la somme des contributions
diagonales et extradiagonales est nulle.  On a donc {\bf dominance simple} de la
diagonale.
\item {\bf Le flux de masse au bord est strictement n\'{e}gatif}. Cette situation
correspond \`a des cellules qui ont des faces de bord d'entr\'ee standard
(entr\'ee de fluide dans le domaine).
Les contributions consid\'er\'ees s'\'ecrivent alors~:
\begin{equation}
\displaystyle\frac{1}{2}\sum\limits_{j\in Vois(i)}\left[(\ +\,m_{\,ij}^n + |\
m_{\,ij}^n|\ )\,\delta f_I^{\,n+1,k+1}+ (\ m_{\,ij}^n - |\ m_{\,ij}^n|)\,\delta f_J^{\,n+1,k+1}\right]
+\sum\limits_{k\in {\gamma_b(i)}}\ m_{\,{b}_{ik}}^n \,\delta f_{\,{b}_{ik}}^{\,n+1,k+1}
\end{equation}
Il convient alors de distinguer plusieurs situations, selon le type de condition
\`a la limite portant sur $f$ (on omet
l'exposant $(\,n+1,k+1)$ pour all\'eger les notations)~:\\
$\hspace*{1.cm}$- pour une condition de Dirichlet : $\delta f_{\,b_{ik}}\,=\,0$, \\
$\hspace*{1.cm}$- pour une condition de Neumann : $\delta f_{\,b_{ik}}\,=\,\delta f_I$, \\
$\hspace*{1.cm}$- pour une condition mixte
($f_{\,b_{ik}}\,=\,\alpha\,+\,\beta f_I$) : $\delta f_{\,b_{ik}}\,=\,\beta\
\delta f_I$.\\
Selon le cas on se trouve dans une des situations suivantes~:\\
\hspace*{1.cm} {\tiny$\bigstar$} si la condition \`{a} la limite de $f$ est de type
{\bf Dirichlet}, la contribution des faces de bord est nulle dans la matrice. La
contribution des faces internes \`a la diagonale est positive, la contribution
aux extradiagonales n\'egative et la somme de ces contributions vaut
$\sum\limits_{j\in Vois(i)}\,m_{\,ij}^n$, soit avec la relation
(\ref{Base_Covofi_Eq_Continuite_discrete})~:
$$\sum\limits_{j\in Vois(i)}\,m_{\,ij}^n=-\sum\limits_{k\in {\gamma_b(i)}}\
m_{\,{b}_{ik}}^n  $$ Elle est strictement positive et la diagonale est donc
{\bf strictement dominante}.\\
\hspace*{1.cm} {\tiny$\bigstar$} si la condition \`{a} la limite de $f$ est de type
{\bf Neumann}, la contribution des faces de bord se r\'{e}duit au terme~:
$
\sum\limits_{k\in {\gamma_b(i)}}\ m_{\,{b}_{ik}}^n \,\delta f_I^{\,n+1,k+1}
$.
La somme des contributions \`a la diagonale est alors~$SC_{i}$:
$$SC_{i}=\frac{1}{2}\sum\limits_{j\in Vois(i)}(\ +\,m_{\,ij}^n + |\ m_{\,ij}^n|\ )
+\sum\limits_{k\in {\gamma_b(i)}}\ m_{\,{b}_{ik}}^n $$
En utilisant deux fois la relation (\ref{Base_Covofi_Eq_Continuite_discrete}), on obtient donc pour la
diagonale~:
$$SC_{i}=\frac{1}{2}\left[\sum\limits_{j\in Vois(i)}|\ m_{\,ij}^n|
+\sum\limits_{k\in {\gamma_b(i)}}\
m_{\,{b}_{ik}}^n\right]=\frac{1}{2}\left[\sum\limits_{j\in Vois(i)}(\ |\ m_{\,ij}^n|-\ m_{\,ij}^n\ )\right]$$
Cette grandeur est positive et \'egale \`a l'oppos\'e de la somme des termes
extradiagonaux qui sont tous n\'egatifs. La diagonale est donc
{\bf simplement dominante}.\\
\hspace*{1.cm} {\tiny$\bigstar$} si la condition \`{a} la limite de $f$ est de type
{\bf mixte}, la somme des contributions dues aux faces de bord est~:\\
\begin{equation}
\sum\limits_{k \in \gamma_b(i)}\beta \ m_{\,{b}_{ik}}^n \ \delta f_I^{\,n+1,k+1}
\end{equation}
On ne peut donc {\bf pas conclure} quant au signe de cette contribution, le facteur
$\beta$ \'etant choisi librement par l'utilisateur. Cette situation a \'et\'e
\'ecart\'ee dans le paragraphe \ref{Base_Covofi_Avecdrhodt}.\\

\end{itemize}

{\bf On peut donc conclure}, quand il n'y a pas de condition \`a la limite de type mixte,
que la matrice associ\'ee aux contributions
 des termes
diff\'erentiels d'ordre 1 (convectifs) est \`a diagonale {\bf simplement
dominante}, \`a condition que la relation (\ref{Base_Covofi_Eq_Continuite_discrete}) soit
v\'erifi\'ee exactement.


%---------------
\subsubsection*{Conclusion}
%---------------
En travaillant sur
des maillages non pathologiques (\`a transmittivit\'e positive, voir la note de
bas de page num\'ero \ref{Base_Covofi_transmittivite}) et en n'imposant pas de condition \`a la limite
de type mixte sur les variables, on peut donc conclure que
$\tens{EM}_{\,n}$
est \`a {\bf diagonale strictement dominante}, donc {\bf inversible} (et la
m\'ethode it\'erative de Jacobi converge) \`a condition que la relation
(\ref{Base_Covofi_Eq_Continuite_discrete}) soit
v\'erifi\'ee exactement. Ce n'est g\'en\'eralement pas le cas (la pr\'ecision du
solveur de pression et la pr\'ecision machine sont finies). M\^eme si
la contribution diagonale en
$\displaystyle \frac {\rho_I^n\ |\Omega_i|}{\Delta t}$ peut suffire \`a
assurer la dominance, on a cependant souhait\'e, dans \CS, s'affranchir du
probl\`eme potentiel en prenant en compte les termes issus de $\displaystyle \frac
{{\partial}\,\rho}{{\partial}\,t}$ dans la matrice.

\newpage
%%%%%%%%%%%%%%%%%%%%%%%%%%%%%%%%%%%%%%%%%%%%%%%%%%%%%%%%%%%%%%%%%%%%%%%%%%%%%%%%%%%%
%%%%%%%%%%%%%%%%%%%%%%%%%%%%%%%%%%%%%%%%%%%%%%%%%%%%%%%%%%%%%%%%%%%%%%%%%%%%%%%%%%%%
\section*{Annexe 2 : Remarques \`a propos du respect du principe du maximum discret}
%%%%%%%%%%%%%%%%%%%%%%%%%%%%%%%%%%%%%%%%%%%%%%%%%%%%%%%%%%%%%%%%%%%%%%%%%%%%%%%%%%%%
%%%%%%%%%%%%%%%%%%%%%%%%%%%%%%%%%%%%%%%%%%%%%%%%%%%%%%%%%%%%%%%%%%%%%%%%%%%%%%%%%%%%
%---------------
\subsection*{Introduction}
%---------------
%
Les consid\'erations expos\'ees ici sont relatives au fait que, en continu,
une variable qui n'est {\em que} convect\'ee par un champ de d\'ebit \`a
divergence nulle doit rester dans les bornes minimales et maximales d\'efinies
par les conditions initiales et par les conditions aux limites en espace. Ainsi,
les valeurs d'un scalaire passif initialement nul
dont les conditions aux limites sont des conditions de Neumann homog\`ene
et des conditions de Dirichlet de valeur 1 devront n\'ecessairement rester
comprises dans l'intervalle $[0\,;\,1]$. C'est ce que l'on entend ici par
``principe du maximum''.

Soient $\vect{u}$ un champ de vitesse fig\'e et connu et $t$ un r\'eel
positif. On consid\`ere le
probl\`eme mod\`ele $\mathcal{P}$ de convection des variables scalaires $\rho$ et $\rho\,f$, d\'efini par :
\begin{equation}
\left\{\begin{array}{ll}
&\displaystyle \frac {\partial \rho}{\partial t} + \,\dive\,(\rho\,\underline{u}) = 0\\
&\displaystyle \frac {\partial  (\rho f)}{\partial t} + \,\dive\,((\rho\,\underline{u})\,f) = 0
\end{array}\right.
\end{equation}
avec une condition initiale $f^0$ donn\'ee ainsi que des conditions aux
limites associ\'ees sur $f$ de type Dirichlet ou Neumann.\\
Dans \CS , la deuxi\`eme \'equation de $\mathcal{P}$ est r\'e\'ecrite en
continu, en utilisant la premi\`ere, sous la forme :\\
\begin{equation}
\displaystyle \rho\,\frac {\partial f}{\partial t} -
f\,\dive\,(\rho\,\underline{u}) + \,\dive\,((\rho\,\underline{u})\,f) = 0
\end{equation}
et discr\'etis\'ee temporellement comme suit :
\begin{equation}\label{Base_Covofi_Eq_Avec_Temp}
\displaystyle \rho^n \,\frac {\,f^{n+1} -\,f^n}{\Delta t} -
f^{n+1}\,\dive\,(\rho\,\underline{u})^n + \,\dive\,((\rho\,\underline{u})^n\,f^{n+1}) = 0
\end{equation}
Dans un premier temps, on va \'etudier la discr\'etisation spatiale associ\'ee
\`a (\ref{Base_Covofi_Eq_Avec_Temp}), qui correspond donc \`a la prise en compte de la
contribution de $\displaystyle\frac{\partial \rho}{\partial t}$ dans
l'\'equation en continu (et se traduit par la pr\'esence de
$-\dive((\rho\,\underline{u})^{n})$ dans l'expression de ${f_s^{\,imp}}_I $),
puis dans un deuxi\`eme temps, la discr\'etisation spatiale de l'expression ;\\
\begin{equation}\label{Base_Covofi_Eq_Sans_Temp}
\displaystyle \rho^n \,\frac {\,f^{n+1} -\,f^n}{\Delta t} +
\,\dive\,((\rho\,\underline{u})^n\,f^{n+1}) = 0
\end{equation}
qui correspond \`a un probl\`eme de convection pure classique.

On \'etudiera ensuite un exemple simplifi\'e (monodimensionnel \`a masse
volumique constante).


Les consid\'erations pr\'esentes m\'eriteraient d'\^etre approfondies.


%==============
\subsection*{Cas g\'en\'eral}
%==============


%-------------
\subsubsection* {Discr\'etisation spatiale de $ \displaystyle \rho^n \,\frac {\,f^{n+1} -\,f^n}{\Delta t} -
f^{n+1}\,\dive\,(\rho\,\underline{u})^n +
\,\dive\,((\rho\,\underline{u})^n\,f^{n+1}) = 0$}
%-------------

En int\'egrant sur une cellule $\Omega_i$ \`a l'aide de la formulation volumes
finis habituelle, on obtient :
\begin{equation}
\begin{array}{lll}
&\displaystyle\int_{\Omega_i}[\displaystyle \rho^n \,\frac {\,f^{n+1} -\,f^n}{\Delta t} -
f^{n+1}\,\dive\,(\rho\,\underline{u})^n +
\,\dive\,((\rho\,\underline{u})^n\,f^{n+1})]\ d\Omega \\
&= \left[\rho_I^n\ \displaystyle\ \frac{|\Omega_i|}{\Delta t}  -
(\sum\limits_{j\in Vois(i)}(\rho\
\underline{u})^{n}_{ij}\,.\,\underline{S}_{ij} + \sum\limits_{k\in
{\gamma_b(i)}}(\rho\
\underline{u})^{n}_{\,b_{ik}}\,.\,\underline{S}_{\,b_{ik}}\,)\right]\, f_I^{\,n+1} \\
&+\sum\limits_{j\in Vois(i)} (\rho\
\underline{u})^{n}_{ij}\,.\,\underline{S}_{ij}\,
f^{\,n+1}_{f\,ij}+\sum\limits_{k\in {\gamma_b(i)}} (\rho\
\underline{u})^{n}_{\,b_{ik}}\,.\,\underline{S}_{\,b_{ik}}\,f^{\,n+1}_{f\,{b_{\,ik}}}
\\
&-\rho_I^n\ \displaystyle\ \frac{|\Omega_i|}{\Delta t}\,f_I^n \\
\end{array}
\end{equation}
o\`{u} $f^{\,n+1}_{f\,ij}$ et $f^{\,n+1}_{f\,{b_{\,ik}}}$ sont les valeurs de $f$
aux faces internes et de bord issues du choix du sch\'{e}ma convectif.

En reprenant les notations pr\'ec\'edentes, en imposant un sch\'ema d\'ecentr\'e
amont au premier membre ({\it i.e.} en exprimant $\delta f_{\,ij}^{\,n+1,k+1}$ et
$\delta f_{\,{b}_{ik}}^{\,n+1,k+1}$) et en raisonnant en incr\'ements
(cf. sous-programme \fort{navstv}), on aboutit \`a :
\begin{equation}
\begin{array}{lll}
&\rho_I^n\,\displaystyle\ \frac{|\Omega_i|}{\Delta
t}\,\delta f_I^{\,n+1,k}\\
& +\displaystyle\frac{1}{2}\sum\limits_{j\in Vois(i)}\left[(\
-\,m_{\,ij}^n + |\ m_{\,ij}^n|\ )\,\delta f_I^{\,n+1,k+1}+ (\ m_{\,ij}^n - |\ m_{\,ij}^n|)\,\delta f_J^{\,n+1,k+1}\right]\\
&+\displaystyle\frac{1}{2}\sum\limits_{k\in {\gamma_b(i)}}\left[(\ -\,
m_{\,{b}_{ik}}^n + |\ m_{\,{b}_{ik}}^n|\ )\,\delta f_I^{\,n+1,k+1} + (\
m_{\,{b}_{ik}}^n - |m_{\,{b}_{ik}}^n|)\,\delta
f_{\,{b}_{ik}}^{\,n+1,k+1}\right]\\
& =\ -\ \displaystyle \rho^n \,\frac {|\Omega_i| }{\Delta t}\,(\,f_I^{\,n+1,k}
-\,f_I^n\,)\\
& - \left[\sum\limits_{j\in Vois(i)} (\rho\ \underline{u})^{n}_{ij}\,.\,\underline{S}_{ij}\, f^{\,n+1,k}_{\,ij}+\sum\limits_{k\in {\gamma_b(i)}} (\rho\
\underline{u})^{n}_{\,b_{ik}}\,.\,\underline{S}_{\,b_{ik}}\,f^{\,n+1,k}_{\,{b_{ik}}}\right]\\
&-\left(\sum\limits_{j\in Vois(i)} (\rho\ \underline{u})^{n}_{ij}\,.\,\underline{S}_{ij}+\sum\limits_{k\in {\gamma_b(i)}} (\rho\
\underline{u})^{n}_{\,b_{ik}}\,.\,\underline{S}_{\,b_{ik}}\right)\,f_I^{\,n+1,k}
\end{array}
\end{equation}
avec :
\begin{equation}
\left\{\begin{array}{ll}
f_I^{\,n+1,0} = f_I^{\,n}\\
\delta f_I^{\,n+1,k+1} = f_I^{\,n+1,k+1} - f_I^{\,n+1,k},{\text {$k\in \mathbb{N}$}}
\end{array}\right.
\end{equation}
et $(f^{\,n+1,k})_{k\in \mathbb{N}}$ suite convergeant vers  $f^{\,n+1}$, $n$
entier donn\'e, solution de (\ref{Base_Covofi_Eq_Avec_Temp}) .\\

 \newpage
%-------------
\subsubsection* {Discr\'etisation spatiale de
 $ \displaystyle \rho^n \,\frac {\,f^{n+1} -\,f^n}{\Delta t} +\,\dive\,((\rho\,\underline{u})^n\,f^{n+1}) = 0$}
%-------------

En proc\'edant de fa\c con analogue et en adoptant les m\^emes hypoth\`eses, on
obtient :
\begin{equation}
\begin{array}{lll}
&\rho^n\,\displaystyle\ \frac{|\Omega_i|}{\Delta
t}\,\delta f_I^{\,n+1,k+1}\\
& +\displaystyle\frac{1}{2}\sum\limits_{j\in Vois(i)}\left[(\
+\,m_{\,ij}^n + |\ m_{\,ij}^n|\ )\,\delta f_I^{\,n+1,k+1}+ (\ m_{\,ij}^n - |\
m_{\,ij}^n|)\,\delta f_J^{\,n+1,k+1}\right]\\
&+\displaystyle\frac{1}{2}\sum\limits_{k\in {\gamma_b(i)}}\left[(\ +\,
m_{\,{b}_{ik}}^n + |\ m_{\,{b}_{ik}}^n|\ )\,\delta f_I^{\,n+1,k+1} + (\
m_{\,{b}_{ik}}^n - |m_{\,{b}_{ik}}^n|)\,\delta
f_{\,{b}_{ik}}^{\,n+1,k+1}\right]\\
& =\ -\ \displaystyle \rho^n \,\frac {|\Omega_i| }{\Delta t}\,(\,f_I^{\,n+1,k}
-\,f_I^n\,)\\
& - \left[\sum\limits_{j\in Vois(i)} (\rho\
\underline{u})^{n}_{ij}\,.\,\underline{S}_{ij}\,
f^{\,n+1,k}_{f\,ij}+\sum\limits_{k\in {\gamma_b(i)}} (\rho\
\underline{u})^{n}_{\,b_{ik}}\,.\,\underline{S}_{\,b_{ik}}\,f^{\,n+1,k}_{f\,{b_{ik}}}\right]
\end{array}
\end{equation}
(o\`{u} $f^{\,n+1}_{f\,ij}$ et $f^{\,n+1}_{f\,{b_{\,ik}}}$ sont les valeurs de $f$
aux faces internes et de bord issues du choix du sch\'{e}ma convectif)

avec :
\begin{equation}
\left\{\begin{array}{ll}
f_I^{\,n+1,0} = f_I^{\,n}\\
\delta f_I^{\,n+1,k+1} = f_I^{\,n+1,k+1} - f_I^{\,n+1,k},{\text {$k\in \mathbb{N}$}}
\end{array}\right.
\end{equation}
et $(f^{\,n+1,k})_{k\in \mathbb{N}}$ suite convergeant vers  $f^{\,n+1}$, $n$
entier donn\'e, solution de (\ref{Base_Covofi_Eq_Sans_Temp}) .\\

%=============
\subsection*{Exemple pour le principe du maximum}
%=============

On va maintenant se placer en monodimensionnel, sur un maillage r\'egulier
form\'e de trois cellules de pas $h$ constant (figure \ref{Base_Covofi_domaine1d_fig}) et \'etudier le comportement
du premier membre pour les deux types d'expressions, entre le pas de temps
 $n\,\Delta t$ et le pas de temps $(n+1)\,\Delta t$, avec, comme condition
initiale $f_1^0 = f_2^0 =f_3^0 = 0$ et comme conditions aux limites, une de type
Dirichlet et l'autre de type Neumann homog\`ene :
\begin{equation}
\left\{\begin{array}{ll}
&f_{\,b_1} = 1 \\
&\displaystyle \left.\frac{\partial f}{\partial x} \right|_{\,b_2} = 0
\end{array}\right.
\end{equation}

On supposera de plus que~:\\
\hspace*{1cm}{\tiny$\bigstar$} le sch\'ema convectif utilis\'e est le sch\'ema upwind\\
\hspace*{1cm}{\tiny$\bigstar$} la masse volumique est constante\\
\hspace*{1cm}{\tiny$\bigstar$} $(\rho\ u)^n_{\,b_1}\,>\,0,\ (\rho\ u)^n_{\,12}\,>\,0\ ,\
(\rho\ u)^n_{\,23}\,>\, 0\ ,\ (\rho\ u)^n_{\,b_2}\,>\,0$\ \ \
et\ \ \ \ \ $S_{\,b_1}\,<\,0$.

\begin{figure}[htp]
\centerline{\includegraphics[width=3.5cm,angle=-90]{domaine1d}}
\caption{\label{Base_Covofi_domaine1d_fig}D\'efinition du domaine de calcul unidimensionnel
consid\'er\'e.}
\end{figure}


On s'int\'eresse \`a l'influence sur le respect du principe du maximum discret
de la pr\'ecision avec laquelle est v\'erifi\'ee sous forme discr\`ete la
relation~:
$$\forall i \in [1,N],\ \ \ \displaystyle\int_{\Omega_i}\dive\,(\rho\,\underline{u})\ d\Omega = 0$$
soit, ici~:
\begin{equation}
\label{Base_Covofi_ContinuiteDiscreteExemple}
-\,(\rho\ u)^n_{\,b_1}\,.\,S_{\,b_1}\,=\,(\rho\ u)^n_{\,12}\,.\,S_{\,12}\,
=\,(\rho\ u)^n_{\,23}\,.\,S_{\,23}\,=\,(\rho\ u)^n_{\,b_2}\,.\,S_{\,b_2}
\end{equation}

%-------------
\subsubsection*{Prise en compte de la contribution de
$\displaystyle\frac{\partial \rho}{\partial t}$ dans la matrice}
%-------------
\label{Base_Covofi_PpmaxAvecDrhoDt}
Le syst\`eme \`a r\'esoudre est alors, en omettant pour simplifier l'exposant $(\,n+1,k+1)$ :
\begin{equation}
\begin{array}{lllllllll}
&\displaystyle \rho_1^{\,n}\ \displaystyle\ \frac{|\Omega_1|}{\Delta
t}\,\delta f_1&&\ \ \ \ \ \ \ \    &-&(\rho\ u)^n_{\,b_1}\,.\,S_{\,b_1}\,\delta
f_1 &= &-(\rho\ u)^n_{\,b_1}\,.\,S_{\,b_1}\,f_{\,b_1}\\
&\displaystyle\rho_2^{\,n}\ \displaystyle\ \frac{|\Omega_2|}{\Delta
t}\,\delta f_2 &+&(\rho\ u)^n_{\,12}\,.\,S_{\,12}\,\delta f_2  &-&(\rho\
u)^n_{\,12}\,.\,S_{\,12}\,\delta f_1 &= &0\\
&\displaystyle\ \rho_3^{\,n} \displaystyle\ \frac{|\Omega_3|}{\Delta
t}\,\delta f_3 &+&(\rho\ u)^n_{\,23}\,.\,S_{\,23}\,\delta f_3 &-&(\rho\
u)^n_{\,23}\,.\,S_{\,23}\,\delta f_2 &= &0\\
\end{array}
\end{equation}
ce qui donne :
\begin{equation}
\left\{\begin{array}{lll}
&\delta f_1 =& \,f_{\,b_1}\displaystyle\ \frac{-(\rho\ u)^n_{\,b_1}\,.\,S_{\,b_1}}{\rho_1^{\,n}\ \displaystyle\ \frac{|\Omega_1|}{\Delta
t} - (\rho\ u)^n_{\,b_1}\,.\,S_{\,b_1}}\\
&\delta f_2 =&+ \,\delta f_1 \displaystyle\ \frac{(\rho\
u)^n_{\,12}\,.\,S_{\,12}}{\rho_2^{\,n}\ \displaystyle\ \frac{|\Omega_2|}{\Delta t} + (\rho\ u)^n_{\,12}\,.\,S_{\,12}}\\
&\delta f_3 =&+ \,\delta f_2 \displaystyle\ \frac{(\rho\
u)^n_{\,23}\,.\,S_{\,23}}{\rho_3^{\,n}\ \displaystyle\ \frac{|\Omega_3|}{\Delta
t} + (\rho\ u)^n_{\,23}\,.\,S_{\,23}}\\
\end{array}\right.
\end{equation}
d'o\`u :
\begin{equation}
\left\{\begin{array}{lll}
& \delta f_1 & <\ 1 \\
& \delta f_2 & <\ 1 \\
& \delta f_3 & <\ 1 \\
\end{array}\right.
\end{equation}
On obtient donc bien une solution qui v\'erifie le principe du maximum discret,
m\^eme pour des grands pas de temps $\Delta t$, et ce, quelle que soit la
pr\'ecision avec laquelle est v\'erifi\'ee, \`a l'\'etape de correction, la
forme discr\`ete (\ref{Base_Covofi_ContinuiteDiscreteExemple}) de la conservation de la
masse $\displaystyle\int_{\Omega_i}\dive\,(\rho\,\underline{u})\ d\Omega = 0$
dont on ne s'est pas servi ici.


%-------------
\subsubsection*{Sans la contribution de
$\displaystyle\frac{\partial \rho}{\partial t}$ dans la matrice}
%-------------


On obtient de m\^eme :
\begin{equation}
\begin{array}{lllllllll}
&\displaystyle \rho_1^{\,n}\ \displaystyle\ \frac{|\Omega_1|}{\Delta t}\,\delta f_1&+& (\rho\ u)^n_{\,12}\,.\,S_{\,12}\,\delta f_1& &\ \ \ \ \  &= &-(\rho\ u)^n_{\,b_1}\,.\,S_{\,b_1}\,f_{\,b_1}\\
&\displaystyle\rho_2^{\,n}\ \displaystyle\ \frac{|\Omega_2|}{\Delta
t}\,\delta f_2 &+&(\rho\ u)^n_{\,23}\,.\,S_{\,23}\,\delta f_2  &-&(\rho\
u)^n_{\,12}\,.\,S_{\,12}\,\delta f_1 &= &0\\
&\displaystyle\rho_3^{\,n}\ \displaystyle\ \frac{|\Omega_3|}{\Delta t}\,\delta
f_3 &-&(\rho\ u)^n_{\,23}\,.\,S_{\,23}\,\delta f_2     &+&(\rho\
u)^n_{\,b_2}\,.\,S_{\,b_2}\,\delta f_3 &= &0\\
\end{array}
\end{equation}

soit :

\begin{equation}
\left\{\begin{array}{lll}
&\delta f_1 =& \,f_{\,b_1}\displaystyle\ \frac{- ( \rho\ u)^n_{\,b_1}\,.\,S_{\,b_1}}{\rho_1^{\,n}\ \displaystyle\ \frac{|\Omega_1|}{\Delta
t} + (\rho\ u)^n_{\,12}\,.\,S_{\,12}}\\
&\delta f_2 =&\ \,\delta f_1 \displaystyle\ \frac{(\rho\
u)^n_{\,12}\,.\,S_{\,12}}{\rho_2^{\,n}\ \displaystyle\ \frac{|\Omega_2|}{\Delta t} + (\rho\ u)^n_{\,23}\,.\,S_{\,23}}\\
&\delta f_3 =&\ \,\delta f_2 \displaystyle\ \frac{(\rho\
u)^n_{\,23}\,.\,S_{\,23}}{\rho_3^{\,n}\ \displaystyle\ \frac{|\Omega_3|}{\Delta
t} + (\rho\ u)^n_{\,b_2}\,.\,S_{\,b_2}}\\
\end{array}\right.
\end{equation}
Ici, on constate que le respect du principe du maximum discret :\\
\begin{equation}
\left\{\begin{array}{lll}
&\delta f_1 &\leqslant \ 1 \\
&\delta f_2 &\leqslant \ 1 \\
&\delta f_3 &\leqslant \ 1 \\
\end{array}\right.
\end{equation}
est \'equivalent \`a la condition :
\begin{equation}
\left\{\begin{array}{lll}
- &( \rho\ u)^n_{\,b_1}\,.\,S_{\,b_1}&\leqslant \displaystyle\ \rho_1^{\,n}
\,\frac{|\Omega_1|}{\Delta t} + (\rho\ u)^n_{\,12}\,.\,S_{\,12}\\
&(\rho\ u)^n_{\,12}\,.\,S_{\,12}&\leqslant \displaystyle\ \rho_2^{\,n}
\frac{|\Omega_2|}{\Delta t} + (\rho\ u)^n_{\,23}\,.\,S_{\,23}\\
& (\rho\ u)^n_{\,23}\,.\,S_{\,23}&\leqslant \ \displaystyle\ \rho_3^{\,n}
\frac{|\Omega_3|}{\Delta t} + (\rho\ u)^n_{\,b_2}\,.\,S_{\,b_2}\\
\end{array}\right.
\end{equation}
Contrairement \`a la situation du paragraphe
\ref{Base_Covofi_PpmaxAvecDrhoDt}, on ne peut obtenir ici un r\'esultat qu'en
faisant intervenir l'\'egalit\'e (\ref{Base_Covofi_ContinuiteDiscreteExemple}), forme
discr\`ete de la conservation de la masse. On obtient bien alors, \`a partir du
syst\`eme pr\'ec\'edent~:
\begin{equation}
\left\{\begin{array}{lll}
&\delta f_1 &< \ 1 \\
&\delta f_2 &< \ 1 \\
&\delta f_3 &< \ 1 \\
\end{array}\right.
\end{equation}

Si l'on s'int\'eresse \`a la cellule $\Omega_1$ et que l'on
suppose $(\rho\ u)^n_{\,12}\,.\,S_{\,12}=-\,(\rho\
u)^n_{\,b_1}\,.\,S_{\,b_1}-\varepsilon (\rho\ u)^n_{\,12}\,.\,S_{\,12}$ (o\`u
$\varepsilon$ est la pr\'ecision locale relative pour l'\'equation de
conservation de la masse discr\`ete), on constate que l'on obtient $\delta\,f_1
> f_{b_1} = 1 $ (valeur non admissible) d\`es lors que~:
 $$\frac{1}{\varepsilon} < \frac{(\rho\
u)^n_{\,12}\,.\,S_{\,12}\Delta t}{\rho_1|\Omega_1|}$$
c'est-\`a-dire d\`es que
le nombre de CFL local $\displaystyle\frac{(\rho\
u)^n_{\,12}\,.\,S_{\,12}\Delta t}{\rho_1|\Omega_1|}$ exc\`ede l'inverse de la
pr\'ecision relative locale $\varepsilon$.

%=============
\subsection*{Conclusion}
%=============

Prendre en compte la contribution de
$\displaystyle\frac{\partial \rho}{\partial t}$ dans la matrice permet un meilleur respect du principe du maximum discret, lorsque la
pr\'ecision de $\displaystyle\int_{\Omega_i}\dive\,(\rho\,\underline{u})\
d\Omega = 0$ n'est pas exactement v\'erifi\'ee.

%-------------------------------------------------------------------------------

% This file is part of Code_Saturne, a general-purpose CFD tool.
%
% Copyright (C) 1998-2020 EDF S.A.
%
% This program is free software; you can redistribute it and/or modify it under
% the terms of the GNU General Public License as published by the Free Software
% Foundation; either version 2 of the License, or (at your option) any later
% version.
%
% This program is distributed in the hope that it will be useful, but WITHOUT
% ANY WARRANTY; without even the implied warranty of MERCHANTABILITY or FITNESS
% FOR A PARTICULAR PURPOSE.  See the GNU General Public License for more
% details.
%
% You should have received a copy of the GNU General Public License along with
% this program; if not, write to the Free Software Foundation, Inc., 51 Franklin
% Street, Fifth Floor, Boston, MA 02110-1301, USA.

%-------------------------------------------------------------------------------

\programme{gradmc}

\vspace{1cm}
%%%%%%%%%%%%%%%%%%%%%%%%%%%%%%%%%%
%%%%%%%%%%%%%%%%%%%%%%%%%%%%%%%%%%
\section*{Fonction}
%%%%%%%%%%%%%%%%%%%%%%%%%%%%%%%%%%
%%%%%%%%%%%%%%%%%%%%%%%%%%%%%%%%%%
Le but de ce sous-programme est de calculer, au centre des cellules, le gradient
d'une fonction scalaire, \'egalement connue au centre des cellules.
Pour obtenir la valeur de toutes les composantes du gradient, une m\'ethode de
minimisation par moindres carr\'es est mise en
\oe uvre~: elle utilise l'estimation d'une composante du gradient aux faces,
obtenue \`a partir des
valeurs de la fonction au centre des cellules voisines. Cette m\'ethode est
activ\'ee lorsque l'indicateur IMRGRA vaut~1 et on l'utilise alors pour le calcul
des gradients de toutes les grandeurs. Elle est beaucoup plus rapide que la m\'ethode
utilis\'ee par d\'efaut (\var{IMRGRA}=0), mais pr\'esente l'inconv\'enient
d'\^etre moins robuste
sur des maillages non orthogonaux, le gradient produit \'etant moins r\'egulier.
% (bien que de m\^eme ordre en espace) J'en sais rien dans l'absolu !.

%%%%%%%%%%%%%%%%%%%%%%%%%%%%%%%%%%
%%%%%%%%%%%%%%%%%%%%%%%%%%%%%%%%%%
\section*{Discr\'etisation}
%%%%%%%%%%%%%%%%%%%%%%%%%%%%%%%%%%
%%%%%%%%%%%%%%%%%%%%%%%%%%%%%%%%%%

\begin{figure}[h]
\parbox{8cm}{%
\centerline{\includegraphics[height=4cm]{facette}}}
\parbox{8cm}{%
\centerline{\includegraphics[height=4cm]{facebord}}}
\caption{\label{Base_Gradmc_fig_geom_gradmc}D\'efinition des diff\'erentes entit\'es
g\'eom\'etriques pour les faces internes (gauche) et de bord (droite).}
\end{figure}

On se reportera aux notations de la figure \ref{Base_Gradmc_fig_geom_gradmc}.
On cherche \`a calculer $\vect{G}_{c,i}$, gradient au centre de la cellule $i$ de la
fonction scalaire $P$. Soit  $\vect{G}_{f,ij}\,.\,\vect{d}_{ij}$ une estimation
\`a la face $ij$ (dont les voisins sont les cellules $i$ et $j$)
du gradient projet\'e dans la direction du vecteur $\vect{d}_{ij}$ (\`a pr\'eciser).
De m\^eme, on note
$\vect{G}_{fb,ik}\,.\,\vect{d}_{b,ik}$ une estimation  \`a la face de bord $ik$
($k^{\text{i\`eme}}$ face de bord appuy\'ee sur la cellule $i$) du gradient projet\'e dans
la direction du vecteur $\vect{d}_{b,ik}$ (\`a pr\'eciser).
L'id\'eal serait de pouvoir trouver un vecteur $\vect{G}_{c,i}$ tel que, pour toute
face interne $ij$ ($j\in Vois(i)$) et toute face de bord $ik$
($k\in\gamma_b(i))$, on ait~:
\begin{equation}
\left\{\begin{array}{c}
\vect{G}_{c,i}\,.\,\vect{d}_{ij}=\vect{G}_{f,ij}\,.\,\vect{d}_{ij}\\
\vect{G}_{c,i}\,.\,\vect{d}_{b,ik}=\vect{G}_{f,b,ik}\,.\,\vect{d}_{b,ik}
\end{array}\right.
\end{equation}




Comme il est g\'en\'eralement impossible d'obtenir l'\'egalit\'e, on cherche
\`a minimiser la fonctionnelle~$\mathcal{F}_i$ suivante~:
\begin{equation}\label{Base_Gradmc_eq_fonctionnelle_gradmc}
\mathcal{F}_i(\vect{G}_{c,i},\vect{G}_{c,i})=
\frac{1}{2}\sum\limits_{j\in Vois(i)}\left[
\vect{G}_{c,i}\,.\,\vect{d}_{ij}-\vect{G}_{f,ij}\,.\,\vect{d}_{ij}
\right]^2+
\frac{1}{2}\sum\limits_{k\in \gamma_b(i)}\left[
\vect{G}_{c,i}\,.\,\vect{d}_{b,ik}-\vect{G}_{f,b,ik}\,.\,\vect{d}_{b,ik}
\right]^2
\end{equation}

Pour ce faire, on annule la d\'eriv\'ee de
$\mathcal{F}_i(\vect{G}_{c,i},\vect{G}_{c,i})$
par rapport \`a
chacune des trois composantes ($G_{c,i,x}, G_{c,i,y}, G_{c,i,z}$) du vecteur
inconnu $\vect{G}_{c,i}$ et l'on r\'esout le syst\`eme qui en r\'esulte.

Pour pouvoir inverser le syst\`eme localement et donc \`a faible co\^ut, on
cherche \`a \'eviter les d\'ependances de $\vect{G}_{f,ij}\,.\,\vect{d}_{ij}$ et
de $\vect{G}_{f,b,ik}\,.\,\vect{d}_{b,ik}$ au gradient $\vect{G}_{c,j}$
(gradient pris dans les cellules voisines). Un choix particulier du vecteur
$\vect{d}$ permet d'atteindre ce but~:
\begin{equation}
\vect{d}_{ij} = \frac{\vect{IJ}}{||\vect{IJ}||} \text{\ \ et\ \ } \vect{d}_{b,ik} = \frac{(\vect{I'F})_l}{||\vect{I'F}||}=\vect{n}_{b,ik}
\end{equation}


Ainsi, pour les faces internes, le vecteur $\vect{d}$ est le vecteur norm\'e joignant
le centre des cellules voisines. La quantit\'e
$\vect{G}_{f,ij}\,.\,\vect{d}_{ij}$ est reli\'ee directement aux
valeurs de la variable $P$ prises au centre des cellules, sans faire intervenir
de gradient~:
\begin{equation}
\vect{G}_{f,ij}\,.\,\vect{d}_{ij}=\frac{P_j-P_i}{||\vect{IJ}||}
\end{equation}

Pour les faces de bord, il est possible d'opter pour un choix plus naturel sans pour
autant faire intervenir le gradient des cellules voisines~: on utilise pour
$\vect{d}$ le vecteur
norm\'e orthogonal \`a la face, dirig\'e vers l'ext\'erieur (le gradient le
mieux connu, en particulier au bord, \'etant le gradient normal aux faces).
On a alors~:
\begin{equation}
\vect{G}_{f,b,ik}\,.\,\vect{d}_{b,ik}=\frac{P_{b,ik}-P_{i'}}{||\vect{I'F}||}
\end{equation}

On utilise alors les relations (\ref{Base_Gradmc_eq_val_bord_gradmc}) au bord ($A_{ik}$ et $B_{ik}$
permettent de
repr\'esenter les conditions aux limites impos\'ees, $P_{b,ik}$ en est issue et
repr\'esente la valeur \`a la face de bord)~:
\begin{equation}
\left\{\begin{array}{ll}\label{Base_Gradmc_eq_val_bord_gradmc}
P_{i'}&=P_{i}+\vect{II'}.\vect{G}_{c,i}\\
P_{b,ik}&=A_{ik}+B_{ik}\,P_{i'}=A_{ik}+B_{ik}\,(P_{i}+\vect{II'}.\vect{G}_{c,i})
\end{array}\right.
\end{equation}

On obtient finalement~:
\begin{equation}\label{Base_Gradmc_eq_grad_bord_gradmc}
\vect{G}_{f,b,ik}\,.\,\vect{d}_{b,ik}
=\frac{1}{||\vect{I'F}||}\left[A_{ik}+(B_{ik}-1)\,(P_{i}+\vect{II'}.\vect{G}_{c,i})\right]
\end{equation}

L'\'equation (\ref{Base_Gradmc_eq_grad_bord_gradmc}), qui fait intervenir $\vect{G}_{c,i}$,
doit \^etre utilis\'ee pour modifier
l'expression (\ref{Base_Gradmc_eq_fonctionnelle_gradmc}) de la fonctionnelle avant de prendre sa
diff\'erentielle. Ainsi~:
\begin{equation}\label{Base_Gradmc_eq_fonctionnelle_mod_gradmc}
\begin{array}{ll}
\mathcal{F}_i(\vect{G}_{c,i},\vect{G}_{c,i})=&
\displaystyle\frac{1}{2}\sum\limits_{j\in Vois(i)}\left[
\vect{G}_{c,i}\,.\,\vect{d}_{ij}-\vect{G}_{f,ij}\,.\,\vect{d}_{ij}
\right]^2+\\
&\displaystyle\frac{1}{2}\sum\limits_{k\in \gamma_b(i)}\left[
\vect{G}_{c,i}\,.\,(\vect{d}_{b,ik}-\frac{B_{ik}-1}{||\vect{I'F}||}\,\vect{II'})
-\frac{1}{||\vect{I'F}||}\left(A_{ik}+(B_{ik}-1)\,P_{i}\right)
\right]^2
\end{array}
\end{equation}





On annule alors la d\'eriv\'ee de
$\mathcal{F}_i(\vect{G}_{c,i},\vect{G}_{c,i})$
par rapport \`a
chacune des trois composantes ($G_{c,i,x}, G_{c,i,y}, G_{c,i,z}$) du vecteur
inconnu $\vect{G}_{c,i}$. On
obtient, pour chaque cellule $i$, le syst\`eme $3\times3$ local
(\ref{Base_Gradmc_eq_systeme_matriciel_gradmc})~:
\begin{equation}\label{Base_Gradmc_eq_systeme_matriciel_gradmc}
\underbrace{
\left[\begin{array}{ccc}
\displaystyle
C_{i,x\,x}
& C_{i,x\,y}
& C_{i,x\,z}\\
\displaystyle
C_{i,y\,x}
& C_{i,y\,y}
& C_{i,y\,z}\\
\displaystyle
C_{i,z\,x}
& C_{i,z\,y}
& C_{i,z\,z}
\end{array}\right]
}_{\tens{C}_i}
\underbrace{
\left[\begin{array}{c}
G_{c,i,x} \\ G_{c,i,y} \\ G_{c,i,z}
\end{array}\right]
}_{\vect{G}_{c,i}}
=
\underbrace{
\left[\begin{array}{c}
\displaystyle
T_{i,x}\\
\displaystyle
T_{i,y}\\
\displaystyle
T_{i,z}
\end{array}\right]
}_{\vect{T}_{i}}
\end{equation}

avec


\begin{equation}
\left\{\begin{array}{ll}
C_{i,l\,m} &=\displaystyle
 \sum\limits_{j\in Vois(i)}(\vect{d}_{ij})_{l}(\vect{d}_{ij})_{m}
+\sum\limits_{k\in\gamma_b(i)}\left(\vect{d}_{b,ik}-\frac{B_{ik}-1}{||\vect{I'F}||}\,\vect{II'}\right)_{l}
                              \left(\vect{d}_{b,ik}-\frac{B_{ik}-1}{||\vect{I'F}||}\,\vect{II'}\right)_{m} \\
T_{i,l} &=\displaystyle
 \sum\limits_{j\in Vois(i)}(\vect{G}_{f,ij}\,.\,\vect{d}_{ij})(\vect{d}_{ij})_l
+\sum\limits_{k\in \gamma_b(i)}\frac{1}{||\vect{I'F}||}\left(A_{ik}+(B_{ik}-1)\,P_{i}\right)
                              \left(\vect{d}_{b,ik}-\frac{B_{ik}-1}{||\vect{I'F}||}\,\vect{II'}\right)_l
\end{array}\right.
\end{equation}

On obtient finalement~:
\begin{equation}
\begin{array}{ll}
C_{i,l\,m} &= \displaystyle
\sum\limits_{j\in Vois(i)}\frac{1}{||\vect{IJ}||^2}(\vect{IJ})_l(\vect{IJ})_m
+\sum\limits_{k\in \gamma_b(i)}\left(\vect{n}_{b,ik}+\frac{1-B_{ik}}{||\vect{I'F}||}\,\vect{II'}\right)_{l}
                               \left(\vect{n}_{b,ik}+\frac{1-B_{ik}}{||\vect{I'F}||}\,\vect{II'}\right)_{m}\\
T_{i,l} &=\displaystyle
\sum\limits_{j\in Vois(i)}\left(P_j-P_i\right)\frac{(\vect{IJ})_l}{||\vect{IJ}||^2}
+\sum\limits_{k\in\gamma_b(i)}\frac{1}{||\vect{I'F}||}\left(A_{ik}+(B_{ik}-1)\,P_i\right)
                             \left(\vect{n}_{b,ik}-\frac{B_{ik}-1}{||\vect{I'F}||}\,\vect{II'}\right)_{l}
\end{array}
\end{equation}

%%%%%%%%%%%%%%%%%%%%%%%%%%%%%%%%%%
%%%%%%%%%%%%%%%%%%%%%%%%%%%%%%%%%%
\section*{Mise en \oe uvre}
%%%%%%%%%%%%%%%%%%%%%%%%%%%%%%%%%%
%%%%%%%%%%%%%%%%%%%%%%%%%%%%%%%%%%

La variable dont il faut calculer le gradient est contenue dans le tableau
\var{PVAR}. Les conditions aux limites associ\'ees sont disponibles au travers
des tableaux \var{COEFAP} et \var{COEFBP} qui repr\'esentent respectivement les
grandeurs $A$ et $B$ utilis\'ees ci-dessus. Les trois composantes du gradient
seront contenues, en sortie du sous-programme, dans les tableaux \var{DPDX},
\var{DPDY} et \var{DPDZ}.


\etape{Calcul de la matrice}
Les \var{NCEL} matrices $\tens{C}_{i}$ (matrices $3\times 3$) sont
stock\'ees dans le tableau \var{COCG},
(de dimension $\text{\var{NCELET}}\times 3\times 3$). Ce dernier est initialis\'e \`a z\'ero,
puis son remplissage est r\'ealis\'e dans des boucles sur les faces internes et
les faces de bord. Les matrices $\tens{C}_{i}$ \'etant sym\'etriques, ces boucles ne
servent qu'\`a remplir la partie triangulaire sup\'erieure, le reste \'etant
compl\'et\'e \`a la fin par sym\'etrie.

Pour \'eviter de r\'ealiser plusieurs fois les m\^emes calculs
g\'eom\'etriques, on conserve, en sortie de sous-programme,
dans le  tableau \var{COCG}, l'inverse des \var{NCEL} matrices $\tens{C}_{i}$.
De plus, pour les \var{NCELBR} cellules qui ont au moins une face
de bord, on conserve dans tableau \var{COCGB}, de dimension
$\text{\var{NCELBR}}\times 3\times 3$, la contribution aux matrices $\tens{C}_{i}$
des termes purement g\'eom\'etriques. On pr\'ecise ces points ci-dessous.
Notons donc d\`es \`a pr\'esent qu'il ne faut pas utiliser les tableaux
\var{COCG} et \var{COCGB} par ailleurs comme tableaux de travail.


\hspace*{1cm}{\bf Cellule ne poss\'edant pas de face de bord}\\
Lorsque, pour une cellule, aucune des faces n'est une face de bord du domaine,
l'expression de la matrice $\tens{C}_{i}$ ne fait intervenir que des grandeurs
g\'eom\'etriques et elle reste inchang\'ee tant que le maillage n'est
pas d\'eform\'e. Son inverse n'est donc calcul\'e qu'une seule
fois, au premier appel de \var{GRADMC} avec \var{ICCOCG=1} (l'indicateur
\var{INICOC}, local \`a \var{GRADMC}, est positionn\'e \`a 0
d\`es lors que ces calculs g\'eom\'etriques ont \'et\'e r\'ealis\'es une fois).
Le tableau \var{COCG} est ensuite
r\'eutilis\'e lors des appels ult\'erieurs au sous-programme
\var{GRADMC}.

\hspace*{1cm}{\bf Cellule poss\'edant au moins une face de bord}\\
Lorsque l'ensemble des faces d'une cellule contient au moins une face de bord
du domaine, un terme contributeur aux matrices  $\tens{C}_{i}$ est
sp\'ecifique \`a la variable dont on cherche
\`a calculer le gradient, au travers du coefficient $B_{ik}$
issu des conditions aux limites. Il s'agit de~:

\begin{equation}\label{Base_Gradmc_eq_termes_de_bord_de_C_gradmc}
\sum\limits_{k\in \gamma_b(i)}\left(\vect{n}_{b,ik}+\frac{1-B_{ik}}{||\vect{I'F}||}\,\vect{II'}\right)_{l}
                               \left(\vect{n}_{b,ik}+\frac{1-B_{ik}}{||\vect{I'F}||}\,\vect{II'}\right)_{m}
\end{equation}



Au premier appel r\'ealis\'e avec \var{ICCOCG=1}, on calcule la contribution des faces
internes et on les stocke dans le tableau \var{COCGB}, qui
sera disponible lors des appels ult\'erieurs. En effet, la
contribution des faces internes est de nature purement g\'eom\'etrique et reste
donc inchang\'ee tant que le maillage ne subit pas de d\'eformation. Elle s'\'ecrit~:
\begin{equation}\notag
\displaystyle\sum\limits_{j\in Vois(i)}\frac{1}{||\vect{IJ}||^2}(\vect{IJ})_l(\vect{IJ})_m
\end{equation}
\`A tous les appels r�alis�s avec \var{ICCOCG=1}, les termes
qui d\'ependent des faces de bord (\ref{Base_Gradmc_eq_termes_de_bord_de_C_gradmc}) sont ensuite
calcul\'es et on additionne cette contribution et \var{COCGB} qui contient celle
des faces internes : on obtient ainsi les matrices $\tens{C}_i$ dans le tableau
\var{COCG}. Leur inverse se calcule ind\'ependamment pour chaque
cellule et on le conserve dans \var{COCG} qui sera disponible lors des appels
ult\'erieurs.

Lorsque \var{GRADMC} a \'et\'e appel\'e une fois avec \var{ICCOCG=1},
des calculs peuvent \^etre \'evit\'es en
positionnant l'indicateur \var{ICCOCG} \`a 0 (si
\var{ICCOCG} est positionn\'e \`a 1, tous les calculs relatifs aux cellules
ayant au moins une face de bord sont refaits).

\begin{itemize}
\item [-] Si \var{GRADMC} est utilis\'e pour
calculer le gradient de la m\^eme variable (ou, plus g\'en\'eralement, d'une
variable dont les conditions aux limites conduisent aux m\^emes valeurs du
coefficient $B_{ik}$), les matrices $\tens{C}_{i}$ sont inchang\'ees et leur
inverse est disponible dans \var{COCG} (on positionne \var{ICCOCG} \`a 0 pour
\'eviter de refaire les calculs).
\item [-] Dans le cas contraire, les termes relatifs aux faces de
bord (\ref{Base_Gradmc_eq_termes_de_bord_de_C_gradmc}) sont recalcul\'es et on additionne
cette contribution et \var{COCGB} qui fournit celle
des faces internes : on obtient ainsi les matrices $\tens{C}_{i}$ dans
\var{COCG}. Il reste alors \`a inverser ces matrices.
\end{itemize}

\hspace*{1cm}{\bf Remarque~:}\\
Pour sauvegarder les contributions g\'eom\'etriques dans \var{COCGB}, on a
recours a une boucle portant sur les \var{NCELBR} cellules
dont au moins une face est une face de bord du domaine. Le num\'ero de ces
cellules est donn\'e par \var{IEL = ICELBR(II)} (\var{II} variant de 1 \`a
\var{NCELBR}). Les op\'erations r\'ealis\'ees dans cette
boucle sont du type \var{COCGB(II,1,1) = COCG(IEL,1,1)}. La structure
(injective) de \var{ICELBR} permet de forcer la vectorisation de la boucle.


\etape{Inversion de la matrice}
On calcule les coefficients de la comatrice, puis l'inverse.
Pour des questions de vectorisation, la boucle sur les \var{NCEL} \'el\'ements
est remplac\'ee par une
s\'erie de boucles en vectorisation forc\'ee sur des blocs de \var{NBLOC=1024}
\'el\'ements. Le reliquat ($\var{NCEL}-E(\var{NCEL}/1024)\times 1024$) est
trait\'e apr\`es les boucles.
La matrice inverse est ensuite stock\'ee dans \var{COCG}
(toujours en utilisant sa propri\'et\'e de sym\'etrie).

\etape{Calcul du second membre et r\'esolution}
Le second membre est stock\'e dans \var{BX}, \var{BY} et \var{BZ}. Le gradient
obtenu par r\'esolution des syst\`emes locaux est stock\'e dans \var{DPDX},
\var{DPDY} et \var{DPDZ}.


\etape{Remarque : gradient sans reconstruction} (non consistant sur maillage non
orthogonal)\\
Dans le cas o\`u l'utilisateur souhaite ne pas reconstruire le gradient ({\it
i.e.} ne pas inclure les termes de non orthogonalit\'e au calcul du gradient),
une m\'ethode sp\'ecifique est mise en \oe uvre, qui n'a pas de rapport avec la
m\'ethode de moindres carr\'es pr\'esent\'ee ci-dessus.

Le volume de la cellule $i$ est not\'e $\Omega_i$.
$P_{ij}$ (resp. $P_{b,ik}$) repr\'esente la valeur
estim\'ee de la variable $P$ \`a la
face interne $ij$ (resp. \`a la face de bord $ik$) de vecteur normal associ\'e
$\vect{S}_{ij}$ (resp. $\vect{S}_{b,ik}$). Le gradient est
simplement calcul\'e en utilisant la formule suivante~:
\begin{equation}
\begin{array}{ll}
\vect{G}_{c,i}=
&=\displaystyle\frac{1}{\Omega_i}\left[
\sum\limits_{j\in Vois(i)}P_{ij}\vect{S}_{ij} +
\sum\limits_{k\in\gamma_b(i)}P_{b,ik}\vect{S}_{b,ik}\right]
\end{array}
\end{equation}

Les valeurs aux faces sont obtenues simplement comme suit (avec $\displaystyle\alpha_{ij}=\frac{\overline{FJ'}}{\overline{I'J'}}$)~:
\begin{equation}
\left\{\begin{array}{ll}
P_{ij}
&= \alpha_{ij}P_i+(1-\alpha_{ij})P_j\\
P_{b,ik}
&=A_{ik} +B_{ik}\,P_i
\end{array}\right.
\end{equation}




%%%%%%%%%%%%%%%%%%%%%%%%%%%%%%%%%%
%%%%%%%%%%%%%%%%%%%%%%%%%%%%%%%%%%
\section*{Points \`a traiter}
%%%%%%%%%%%%%%%%%%%%%%%%%%%%%%%%%%
%%%%%%%%%%%%%%%%%%%%%%%%%%%%%%%%%%
\etape{Vectorisation forc\'ee}
Il est peut-\^etre possible de s'affranchir du d\'ecoupage en boucles de 1024 si
les compilateurs sont capables
d'effectuer la vectorisation sans cette aide. On note cependant que ce
d\'ecoupage en boucles de 1024 n'a pas de co\^ut CPU suppl\'ementaire, et que
le co\^ut m\'emoire associ\'e est n\'egligeable.
Le seul inconv\'enient r\'eside dans la relative complexit\'e  de l'\'ecriture.

\etape{Choix du vecteur $d$}
Le choix $\vect{d}_{ij}= \frac{\vect{IJ}}{||\vect{IJ}||}$ permet de calculer
simplement une composante du gradient \`a la face en ne faisant intervenir que
les valeurs de la variable au centre des cellules voisines. Le choix
$\vect{d}_{ij}= \frac{\vect{I'J'}}{||\vect{I'J'}||}$ serait \'egalement
possible, et peut-\^etre meilleur, mais conduirait naturellement \`a faire
intervenir, pour le calcul de la composante du gradient normale aux faces, les
valeurs de la variable aux points $I'$ et $J'$, et donc les valeurs du gradient
dans les cellules voisines. Il en r\'esulterait donc un syst\`eme coupl\'e,
auquel un algorithme it\'eratif (voir \var{GRADRC}) pourrait \^etre appliqu\'e.
L'aspect temps calcul, atout majeur de la m\'ethode actuelle,
s'en ressentirait sans doute.

\etape{Am\'elioration de la m\'ethode}
Cette m\'ethode rencontre des difficult\'es dans le cas de maillages assez ``non
orthogonaux'' (cas de la voiture maill\'e en t\'etra\`edres par exemple). Une
voie d'am\'elioration possible est d'utiliser un support \'etendu (le support est
l'ensemble des cellules utilis\'ees pour calculer le gradient en une cellule
donn\'ee). Un exemple est fourni sur la figure \ref{Base_Gradmc_fig_support_gradmc} ci-dessous~: si la cellule $I$
est la cellule courante, on choisit pour
support les cellules de centre $J$ telles que la droite $(IJ)$ soit la plus
orthogonale possible \`a une face de la cellule $I$.


\begin{figure}[htp]
\centerline{\includegraphics[height=8cm]{support}}
\caption{\label{Base_Gradmc_fig_support_gradmc}Diff\'erents supports pour le calcul du gradient.}
\end{figure}




%-------------------------------------------------------------------------------

% This file is part of code_saturne, a general-purpose CFD tool.
%
% Copyright (C) 1998-2022 EDF S.A.
%
% This program is free software; you can redistribute it and/or modify it under
% the terms of the GNU General Public License as published by the Free Software
% Foundation; either version 2 of the License, or (at your option) any later
% version.
%
% This program is distributed in the hope that it will be useful, but WITHOUT
% ANY WARRANTY; without even the implied warranty of MERCHANTABILITY or FITNESS
% FOR A PARTICULAR PURPOSE.  See the GNU General Public License for more
% details.
%
% You should have received a copy of the GNU General Public License along with
% this program; if not, write to the Free Software Foundation, Inc., 51 Franklin
% Street, Fifth Floor, Boston, MA 02110-1301, USA.

%-------------------------------------------------------------------------------

\programme{gradrc}\label{ap:gradrc}

\vspace{1cm}
%-------------------------------------------------------------------------------
\section*{Fonction}
%-------------------------------------------------------------------------------

Le but de ce sous-programme est de calculer, au centre des cellules, le gradient
d'une fonction scalaire, \'egalement connue au centre des cellules.
Pour obtenir la valeur du gradient, une m\'ethode it\'erative de
reconstruction pour les maillages non orthogonaux est mise en
\oe uvre~: elle fait appel \`a un d\'eveloppement limit\'e d'ordre 1 en espace
sur la variable, obtenu \`a partir de la
valeur de la fonction et de son gradient au centre de la cellule. Cette
m\'ethode,
choisie comme option par d\'efaut, correspond \`a \var{imrgra}\,=\,0 et est utilis\'ee pour le calcul
des gradients de toutes les grandeurs.

%%%%%%%%%%%%%%%%%%%%%%%%%%%%%%%%%%
%%%%%%%%%%%%%%%%%%%%%%%%%%%%%%%%%%
\section*{Discr\'etisation}
%%%%%%%%%%%%%%%%%%%%%%%%%%%%%%%%%%
%%%%%%%%%%%%%%%%%%%%%%%%%%%%%%%%%%

La m\'ethode est d\'ecrite \'a la section~\ref{sec:spadis:iteratif_gradient}.

\minititre{Remarque}
Pour les conditions aux limites en pression, un traitement particulier est mis
en  \oe uvre, surtout utile dans les cas o\`u un gradient de pression (hydrostatique
ou autre) n\'ecessite une attention sp\'ecifique aux bords, o\`u une condition
\`a la limite de type Neumann homog\`ene est g\'en\'eralement inadapt\'ee. Soit
$P_{F_{\,b_{\,ik}}}$ la  valeur de la pression \`a la face associ\'ee, que
l'on veut calculer.

On note que ~:
\begin{equation}\notag
\vect{I'F}_{\,b_{\,ik}} \,.\,(\grad P)_I = \vect{I'F}_{\,b_{\,ik}}
\,.\,\vect{G}_{\,c,i} = \overline{I'F}_{\,b_{\,ik}} \,.\left. \displaystyle\frac{\delta P}{\delta
n}\right|_{F_{\,b_{\,ik}}}
\end{equation}
avec les conventions pr\'ec\'edentes.\\
\paragraph{\bf Sur maillage orthogonal }
On se place dans le cas d'un maillage orthogonal , {\it i.e.} pour
toute cellule $\Omega_I$, $I$ et son projet\'e $I'$ sont identiques.
Soit $M_{\,b_{\,ik}}$ le milieu du segment $IF_{\,b_{\,ik}}$.\\
On peut \'ecrire les \'egalit\'es suivantes~:
\begin{equation}\notag
\begin{array}{ll}
P_{F_{\,b_{\,ik}}} & = P_{M_{\,b_{\,ik}}} + \overline{M_{\,b_{\,ik}}F_{\,b_{\,ik}}}\,.\left. \displaystyle\frac{\delta P}{\delta
n}\right|_{M_{\,b_{\,ik}}} +
\overline{M_{\,b_{\,ik}}F_{\,b_{\,ik}}}^{\,2}\,.\left.\displaystyle\frac{1}{2}\frac{{\delta}^{2} P}{\delta
n^2}\right|_{M_{\,b_{\,ik}}} + \mathcal{O}(h^3)\\
P_I & = P_{M_{\,b_{\,ik}}} + \overline{M_{\,b_{\,ik}}I}\,.
\left. \displaystyle\frac{\delta P}{\delta n}\right|_{M_{\,b_{\,ik}}} +
\overline{M_{\,b_{\,ik}}I}^{\,2}\,.\left.\displaystyle\frac{1}{2}
\frac{{\delta}^{2} P}{\delta n^2}\right|_{M_{\,b_{\,ik}}} + \mathcal{O}(h^3)
\end{array}
\end{equation}
avec $\overline{M_{\,b_{\,ik}}I} = - \overline{M_{\,b_{\,ik}}F_{\,b_{\,ik}}}$.\\
On obtient donc~:
\begin{equation}\label{Base_Gradrc_eq_orthogonal}
P_{F_{\,b_{\,ik}}} - P_I = \overline{IF}_{\,b_{\,ik}}\,.\left. \displaystyle\frac{\delta P}{\delta
n}\right|_{M_{\,b_{\,ik}}} + \mathcal{O}(h^3)
\end{equation}
Gr\^ace \`a la formule des accroissements finis :
\begin{equation}\label{Base_Gradrc_eq_derivee_normale}
\left. \displaystyle\frac{\delta P}{\delta n}\right|_{M_{\,b_{\,ik}}} =
\displaystyle\frac{1}{2}\left[\left. \displaystyle\frac{\delta P}{\delta
n}\right|_{I} +  \left. \displaystyle\frac{\delta P}{\delta
n}\right|_{F_{\,b_{\,ik}}}\right] + \mathcal{O}(h^2)
\end{equation}\\
On s'int\'eresse aux cas suivants :\\\\
\hspace*{0.5cm}{ $\bullet${\underline { condition \`a la limite de type Dirichlet}}}\\
$P_{F_{\,b_{\,ik}}} = P_{IMPOSE}$, aucun traitement particulier\\\\
\hspace*{0.5cm}{ $\bullet ${\underline { condition \`a la limite de type Neumann
homog\`ene}}}\\
On veut imposer :
\begin{equation}
\left. \displaystyle\frac{\delta P}{\delta n}\right|_{F_{\,b_{\,ik}}} = 0 + \mathcal{O}(h)
\end{equation}
On a~:
\begin{equation}\notag
\left. \displaystyle\frac{\delta P}{\delta n}\right|_{I} =
\displaystyle\left. \displaystyle\frac{\delta P}{\delta
n}\right|_{F_{\,b_{\,ik}}} + \mathcal{O}(h)
\end{equation}
et comme :
\begin{equation}
P_{F_{\,b_{\,ik}}} = P_I + \overline{IF}_{\,b_{\,ik}}\,.\left. \displaystyle\frac{\delta P}{\delta
n}\right|_I + \mathcal{O}(h^2)
\end{equation}
on en d\'eduit :
\begin{equation}
P_{F_{\,b_{\,ik}}} = P_I + \overline{IF}_{\,b_{\,ik}}\,.\left. \displaystyle\frac{\delta P}{\delta
n}\right|_{F_{\,b_{\,ik}}} + \mathcal{O}(h^2)
\end{equation}
soit~:
\begin{equation}
P_{F_{\,b_{\,ik}}} = P_I +  \mathcal{O}(h^2)
\end{equation}
On obtient donc une approximation d'ordre 1.\\
\paragraph{\bf Sur maillage non orthogonal}
Dans ce cas, on peut seulement \'ecrire~:\\
\begin{equation}
P_{F_{\,b_{\,ik}}}  = P_{I'} +
\displaystyle\frac{1}{2}\,\vect{I'F}_{\,b_{\,ik}}\,.\,[\,(\grad P)_{I'} + (\grad
P)_{F_{\,b_{\,ik}}}\,] + \mathcal{O}(h^3)
\end{equation}
\hspace*{0.5cm}{ $\bullet ${\underline { condition \`a la limite de type Dirichlet}}\\
$P_{F_{\,b_{\,ik}}} = P_{IMPOSE}$, toujours aucun traitement particulier\\\\
\hspace*{0.5cm}{ $\bullet ${\underline { condition \`a la limite de type Neumann
homog\`ene}}}\\
On veut :
\begin{equation}
\left. \displaystyle\frac{\delta P}{\delta n}\right|_{F_{\,b_{\,ik}}} = 0 + \mathcal{O}(h)
\end{equation}
ce qui entra\^\i ne :
\begin{equation}\label{Base_Gradrc_eq_ortho}
\vect{I'F}_{\,b_{\,ik}} \,.\,(\grad P)_{F_{\,b_{\,ik}}} = \mathcal{O}(h^2)
\end{equation}
On peut \'ecrire :
\begin{equation}\notag
(\grad P)_{I'} = (\grad P)_{F_{\,b_{\,ik}}} +  \mathcal{O}(h)
\end{equation}
d'o\`u~:
\begin{equation}
P_{F_{\,b_{\,ik}}} = P_{I'}  + \mathcal{O}(h^2)
\end{equation}
On obtient donc une approximation d'ordre 1.\\

\hspace*{0.5cm}{\bf $\bullet $ Conclusion }\\
On peut r\'ecapituler toutes ces situations {\it via} la formule :
\begin{equation}\notag
P_{F_{\,b_{\,ik}}}\,=\,P_{I'}
\end{equation}


%-------------------------------------------------------------------------------

% This file is part of code_saturne, a general-purpose CFD tool.
%
% Copyright (C) 1998-2022 EDF S.A.
%
% This program is free software; you can redistribute it and/or modify it under
% the terms of the GNU General Public License as published by the Free Software
% Foundation; either version 2 of the License, or (at your option) any later
% version.
%
% This program is distributed in the hope that it will be useful, but WITHOUT
% ANY WARRANTY; without even the implied warranty of MERCHANTABILITY or FITNESS
% FOR A PARTICULAR PURPOSE.  See the GNU General Public License for more
% details.
%
% You should have received a copy of the GNU General Public License along with
% this program; if not, write to the Free Software Foundation, Inc., 51 Franklin
% Street, Fifth Floor, Boston, MA 02110-1301, USA.

%-------------------------------------------------------------------------------

\programme{inimas}

\vspace{1cm}
%%%%%%%%%%%%%%%%%%%%%%%%%%%%%%%%%%
%%%%%%%%%%%%%%%%%%%%%%%%%%%%%%%%%%
\section*{Fonction}
%%%%%%%%%%%%%%%%%%%%%%%%%%%%%%%%%%
%%%%%%%%%%%%%%%%%%%%%%%%%%%%%%%%%%
Le but de ce sous-programme est principalement de calculer le flux de masse aux
faces. Il prend une variable vectorielle associ\'ee au centre des cellules
(g\'en\'eralement la vitesse), la projette aux faces en la multipliant par la
masse volumique, et la multiplie scalairement par le vecteur surface.
Plus g\'en\'eralement, \fort{inimas} est aussi appel\'e comme premi\`ere \'etape
du calcul d'une divergence (terme en $\dive(\rho\tens{R})$ en
$R_{ij}-\varepsilon$, filtre Rhie \& Chow, ...).

%%%%%%%%%%%%%%%%%%%%%%%%%%%%%%%%%%
%%%%%%%%%%%%%%%%%%%%%%%%%%%%%%%%%%
\section*{Discr\'etisation}
%%%%%%%%%%%%%%%%%%%%%%%%%%%%%%%%%%
%%%%%%%%%%%%%%%%%%%%%%%%%%%%%%%%%%

La figure \ref{Base_Inimas_fig_geom} rappelle les diverses d\'efinitions g\'eom\'etriques
pour les faces internes et les faces de bord. On notera
$\displaystyle \alpha=\frac{\overline{FJ^\prime}}{\overline{I^\prime J^\prime}}$ (d\'efini aux faces
internes uniquement).

\begin{figure}[h]
\parbox{8cm}{%
\centerline{\includegraphics[height=4cm]{facette}}}
\parbox{8cm}{%
\centerline{\includegraphics[height=4cm]{facebord}}}
\caption{\label{Base_Inimas_fig_geom}D\'efinition des diff\'erentes entit\'es
g\'eom\'etriques pour les faces internes (gauche) et de bord (droite).}
\end{figure}


\subsection*{Faces internes}
On ne conna\^\i t pas la masse volumique \`a la face, cette derni\`ere doit donc
aussi \^etre interpol\'ee. On utilise la discr\'etisation suivante :

\begin{equation}
(\rho \vect{u})_F = \alpha(\rho_I \vect{u}_I)
+(1-\alpha)(\rho_J \vect{u}_J)
+\ggrad\!(\rho\vect{u})_O.\vect{OF}
\end{equation}
La partie en $\alpha(\rho_I \vect{u}_I)
+(1-\alpha)(\rho_J \vect{u}_J)$ correspondant en fait \`a
$(\rho\vect{u})_O$. Le gradient en $O$ est calcul\'e par interpolation :
$\displaystyle\ggrad\!(\rho\vect{u})_O=
\frac{1}{2}\left[\ggrad\!(\rho\vect{u})_I+\ggrad\!(\rho\vect{u})_J\right]$. La
valeur $\displaystyle\frac{1}{2}$ s'est impos\'ee de mani\`ere heuristique au
fil des tests
comme apportant plus de stabilit\'e \`a l'algorithme global qu'une interpolation
faisant intervenir $\alpha$. L'erreur commise sur $\rho\vect{u}$ est en
$O(h^2)$.


\subsection*{Faces de bord}
Le traitement des faces de bord est n\'ecessaire pour y calculer le flux de
masse, bien s\^ur, mais aussi pour obtenir des conditions aux limites pour le
calcul du $\ggrad\!(\rho \vect{u})$ utilis\'e pour les faces internes.

Pour les faces de bord, on conna\^\i t la valeur de $\rho_F$, qui est stock\'ee
dans la variable \var{ROMB}. De plus, les conditions aux limites pour $\vect{u}$
sont donn\'ees par des coefficients $A$ et $B$ tels que :
\begin{equation}
u_{k,F} = A_k + B_ku_{k,I^\prime} =
A_k + B_k\left(u_{k,I} + \grad\!(u_k)_I.\vect{II^\prime}\right)
\end{equation}
($k\in\{1,2,3\}$ est la composante de la vitesse, l'erreur est en $O(B_kh)$)

On a donc \`a l'ordre 1 :
\begin{equation}
(\rho u_k)_F = \rho_F\left[A_k + B_k\left(u_{k,I} +
\grad\!(u_k)_I.\vect{II^\prime}\right)\right]
\end{equation}

Mais pour utiliser cette formule, il faudrait calculer $\ggrad\!(\vect{u})$ (trois
appels \`a \fort{GRDCEL}), alors qu'on a d\'ej\`a calcul\'e
$\ggrad\!(\rho\vect{u})$ pour les faces internes. Le surco\^ut en temps serait alors
important. On r\'e\'ecrit donc :
\begin{eqnarray}
(\rho u_k)_F & = & \rho_F A_k + \rho_F B_ku_{k,I^\prime}\\
& = & \rho_F A_k + B_k\frac{\rho_F}{\rho_{I^\prime}}(\rho u_k)_{I^\prime}
\label{Base_Inimas_eq_rhoufacea}\\
& = & \rho_F A_k + B_k\frac{\rho_F}{\rho_{I^\prime}}(\rho u_k)_I
+B_k\frac{\rho_F}{\rho_{I^\prime}}\grad\!(\rho u_k)_I.\vect{II^\prime}
\label{Base_Inimas_eq_rhoufaceb}
\end{eqnarray}

Pour calculer les gradients de $\rho\vect{u}$, il faudrait donc en th\'eorie
utiliser les coefficients de conditions aux limites \'equivalents :\\
$\tilde{A}_k = \rho_F A_k$\\
$\displaystyle \tilde{B}_k = B_k\frac{\rho_F}{\rho_{I^\prime}}$

Ceci para\^\i t d\'elicat, \`a cause du terme en
$\displaystyle \frac{\rho_F}{\rho_{I^\prime}}$, et en particulier \`a l'erreur
que l'on peut commettre sur $\rho_{I^\prime}$ si la reconstruction des gradients
est imparfaite (sur des maillages fortement non orthogonaux par exemple).
On r\'e\'ecrit donc l'\'equation
(\ref{Base_Inimas_eq_rhoufaceb}) sous la forme suivante :
\begin{equation}
(\rho u_k)_F=\rho_F A_k + B_k\frac{\rho_I\rho_F}{\rho_{I^\prime}}u_{k,I}
+B_k\frac{\rho_F}{\rho_{I^\prime}}\grad\!(\rho u_k)_I.\vect{II^\prime}
\end{equation}


Pour le calcul du flux de masse au bord, on va faire deux approximations. Pour
le deuxi\`eme terme, on va supposer $\rho_{I^\prime}\approx\rho_I$ (ce qui
conduit \`a une erreur en $O(B_kh)$ sur $\rho\vect{u}$ si
$\rho_{I^\prime}\ne \rho_I$). Pour le
troisi\`eme terme, on va supposer $\rho_{I^\prime}\approx\rho_F$. Cette
derni\`ere approximation est plus forte, mais elle n'intervient que dans la
reconstruction des non-orthogonalit\'es ; l'erreur finale reste donc faible
(erreur en $O(B_kh^2)$ sur $\rho\vect{u}$ si
$\rho_{I^\prime}\ne \rho_F$).
Et au final, le flux de masse au bord est calcul\'e par :
\begin{equation}
\dot{m}_F = \sum\limits_{k=1}^{3}\left[\rho_F A_k + B_k\rho_Fu_{k,I}
+B_k\grad\!(\rho u_k)_I.\vect{II^\prime}\right]S_k
\end{equation}

Pour le calcul des gradients, on repart de l'\'equation (\ref{Base_Inimas_eq_rhoufacea}), sur
laquelle on fait l'hypoth\`ese que $\rho_{I^\prime}\approx\rho_F$. Encore une
fois, cette hypoth\`ese peut \^etre assez forte, mais les gradients obtenus ne
sont utilis\'es que pour des reconstructions de non-orthogonalit\'es ; l'erreur
finale reste donc l\`a encore assez faible.
Au final, les gradients sont calcul\'es \`a partir de la formule suivante :
\begin{equation}
(\rho u_k)_F = \rho_F A_k + B_k(\rho u_k)_{I^\prime}
\end{equation}
ce qui revient \`a utiliser les conditions aux limites suivantes pour
$\rho \vect{u}$:\\
$\tilde{A}_k = \rho_F A_k$\\
$\tilde{B}_k = B_k$

\minititre{Remarque}

Dans la plupart des cas, les approximations effectu\'ees n'engendrent aucune
erreur. En effet :\\
- dans le cas d'une entr\'ee on a g\'en\'eralement $B_k=0$, avec un flux de
masse impos� par la condition � la limite.\\
- dans le cas d'une sortie, on a g\'en\'eralement flux nul sur les scalaires
donc sur $\rho$, soit \mbox{$\rho_F=\rho_{I^\prime}=\rho_I$}.\\
- dans le cas d'une paroi, on a g\'en\'eralement $B_k=0$ et le flux de masse
est impos� nul.\\
- dans le cas d'une sym\'etrie, on a g\'en\'eralement
$\rho_F=\rho_{I^\prime}=\rho_I$ et le flux de masse est impos� nul.\\
Pour sentir un effet de ces approximations, il faudrait par exemple une paroi
glissante ($B_k\ne0$) avec un gradient de temp\'erature ($\rho_F\ne\rho_I$).

%%%%%%%%%%%%%%%%%%%%%%%%%%%%%%%%%%
%%%%%%%%%%%%%%%%%%%%%%%%%%%%%%%%%%
\section*{Mise en \oe uvre}
%%%%%%%%%%%%%%%%%%%%%%%%%%%%%%%%%%
%%%%%%%%%%%%%%%%%%%%%%%%%%%%%%%%%%

La vitesse est pass\'ee par les arguments \var{UX}, \var{UY} et \var{UZ}. Les
conditions aux limites de la vitesse sont \var{COEFAX}, \var{COEFBX}, ... Le
flux de masse r\'esultat est stock\'e dans les variables \var{FLUMAS} (faces
internes) et \var{FLUMAB} (faces de bord). \var{QDMX}, \var{QDMY} et \var{QDMZ}
sont des variables de travail qui serviront \`a stocker $\rho\vect{u}$, et
\var{COEFQA} servira \`a stocker les $\tilde{A}$.

\etape{Initialisation \'eventuelle du flux de masse}
Si \var{INIT} vaut 1, le flux de masse est remis \`a z\'ero. Sinon, le
sous-programme rajoute aux variables \var{FLUMAS} et \var{FLUMAB} existantes le
flux de masse calcul\'e.


\etape{Remplissage des tableaux de travail}
$\rho\vect{u}$ est stock\'e dans \var{QDM}, et $\tilde{A}$ dans \var{COEFQA}.


\etape{Cas sans reconstruction}
On calcule alors directement\\
$\displaystyle \var{FLUMAS}=\sum\limits_{k=1}^{3}\left[
\alpha(\rho_I u_{k,I})+(1-\alpha)(\rho_J u_{k,J})\right]S_k$\\
et\\
$\displaystyle \var{FLUMAB}=\sum\limits_{k=1}^{3}\left[
\rho_F A_k + B_k\rho_Fu_{k,I}\right]S_k$


\etape{Cas avec reconstruction}
On r\'ep\`ete trois fois de suite les op\'erations suivantes, pour $k=1$, 2 et 3
:\\
- Appel de \fort{GRDCEL} pour le calcul de $\grad\!(\rho u_k)$.\\
- Mise \`a jour du flux de masse\\
$\displaystyle \var{FLUMAS}=\var{FLUMAS} + \left[
\alpha(\rho_I u_{k,I})+(1-\alpha)(\rho_J u_{k,J})
+\frac{1}{2}\left[\grad\!(\rho u_k)_I+\grad\!(\rho u_k)_J\right]
.\vect{OF}\right]S_k$\\
et\\
$\displaystyle \var{FLUMAB}=\var{FLUMAB}+\left[
\rho_F A_k + B_k\rho_Fu_{k,I}
+B_k\grad\!(\rho u_k)_I.\vect{II^\prime}\right]S_k$


\etape{Annulation du flux de masse au bord}
Quand le sous-programme a \'et\'e appel\'e avec la valeur \var{IFLMB0=1}
(c'est-\`a-dire quand il est r\'eellement appel\'e pour calculer un flux de
masse, et pas pour calculer le terme en $\dive(\rho\tens{R})$ par exemple), le flux
de masse au bord \var{FLUMAB} est forc\'e \`a 0, pour les faces de paroi et de
sym\'etrie (identifi\'ees par \var{ISYMPA=0}).

%-------------------------------------------------------------------------------

% This file is part of Code_Saturne, a general-purpose CFD tool.
%
% Copyright (C) 1998-2018 EDF S.A.
%
% This program is free software; you can redistribute it and/or modify it under
% the terms of the GNU General Public License as published by the Free Software
% Foundation; either version 2 of the License, or (at your option) any later
% version.
%
% This program is distributed in the hope that it will be useful, but WITHOUT
% ANY WARRANTY; without even the implied warranty of MERCHANTABILITY or FITNESS
% FOR A PARTICULAR PURPOSE.  See the GNU General Public License for more
% details.
%
% You should have received a copy of the GNU General Public License along with
% this program; if not, write to the Free Software Foundation, Inc., 51 Franklin
% Street, Fifth Floor, Boston, MA 02110-1301, USA.

%-------------------------------------------------------------------------------

\programme{itrmas/itrgrp}

\hypertarget{itrmas}{}

\vspace{1cm}
%%%%%%%%%%%%%%%%%%%%%%%%%%%%%%%%%%
%%%%%%%%%%%%%%%%%%%%%%%%%%%%%%%%%%
\section*{Fonction}
%%%%%%%%%%%%%%%%%%%%%%%%%%%%%%%%%%
%%%%%%%%%%%%%%%%%%%%%%%%%%%%%%%%%%
Le but du sous-programme \fort{itrmas} est de calculer un gradient de pression
``facette''. Il est utilis\'e dans la phase de correction de pression
(deuxi\`eme phase de \fort{navstv}) o\`u le flux de masse est mis \`a jour \`a l'aide de termes en $-\Delta t_{\,ij}(\grad_f P)_{\,ij}.\vect{S}_{\,ij}$ et en $-\Delta t_{\,b_{ik}}(\grad_f P)_{\,b_{ik}}\,.\,\vect{S}_{\,b_{ik}}$.

Le sous-programme \fort{itrgrp} calcule un gradient ``facette'' de pression et
en prend la divergence, c'est-\`a-dire calcule le terme :
\begin{displaymath}
-\sum\limits_{j\in Vois(i)}\Delta t_{\,ij}(\grad_f P)_{\,ij}.\vect{S}_{\,ij}
-\sum\limits_{k\in\gamma_b(i)}\Delta t_{\,b_{ik}}(\grad_f P)_{\,b_{ik}}\,.\,\vect{S}_{\,b_{ik}}
\end{displaymath}
En pratique \fort{itrgrp} correspond \`a la combinaison de \fort{itrmas} et de
\fort{divmas}, mais permet par son traitement en un seul bloc d'\'eviter la
d\'efinition de tableaux de travail de taille \var{NFAC} et \var{NFABOR}.

See the \doxygenanchor{cs__convection__diffusion_8c.html\#itrmas}{programmers reference of the dedicated subroutine}
for further details.

%%%%%%%%%%%%%%%%%%%%%%%%%%%%%%%%%%
%%%%%%%%%%%%%%%%%%%%%%%%%%%%%%%%%%
\section*{Discr\'etisation}
%%%%%%%%%%%%%%%%%%%%%%%%%%%%%%%%%%
%%%%%%%%%%%%%%%%%%%%%%%%%%%%%%%%%%

La figure \ref{Base_Itrmas_fig_geom} rappelle les diverses d\'efinitions g\'eom\'etriques
pour les faces internes et les faces de bord.

\begin{figure}[h]
\parbox{8cm}{%
\centerline{\includegraphics[height=4cm]{facette}}}
\parbox{8cm}{%
\centerline{\includegraphics[height=4cm]{facebord}}}
\caption{\label{Base_Itrmas_fig_geom}D\'efinition des diff\'erentes entit\'es
g\'eom\'etriques pour les faces internes (gauche) et de bord (droite).}
\end{figure}


\subsection*{Calcul sans reconstruction des non orthogonalit\'es}
Pour les faces internes, on \'ecrit simplement :
\begin{equation}
\label{Base_Itrmas_eq_intssrec}
-\Delta t_{\,ij}(\grad_f P)_{\,ij}\,.\,\vect{S}_{\,ij}=
\frac{\Delta t_{\,ij}S_{\,ij}}{\overline{I^\prime J^\prime}}(P_I-P_J)
\end{equation}

Pour les faces de bord, on \'ecrit :
\begin{equation}
\label{Base_Itrmas_eq_brdssrec}
-\Delta t_{b_{ik}}(\grad_f P)_{\,b_{ik}}\,.\,\vect{S}_{\,b_{ik}}=
\frac{\Delta t_{\,b_{ik}}S_{\,b_{ik}}}{\overline{I^\prime F}}
\left((1-B_{b,ik})P_I-\var{INC}\times A_{b,ik}\right)
\end{equation}

Les pas de temps aux faces $\Delta t_{\,ij}$ et $\Delta t_{\,b_{ik}}$ sont calcul\'es
par interpolation par les sous-programmes \fort{viscfa} (cas isotrope,
\var{IPUCOU=0}) ou \fort{visort} (cas anisotrope, \var{IPUCOU=1}).


\subsection*{Calcul avec reconstruction des non orthogonalit\'es}
Plusieurs discr\'etisations peuvent \^etre propos\'ees pour le traitement des
non orthogonalit\'es. Celle retenue dans le code est issue des premiers tests
r\'ealis\'es sur le prototype, et fait intervenir non seulement le pas de temps
interpol\'e \`a la face, mais aussi les pas de temps dans chaque
cellule. Il
serait sans doute bon de revenir sur cette \'ecriture et \'evaluer d'autres
solutions. La forme utilis\'ee pour les faces internes est :
\begin{multline}
\label{Base_Itrmas_eq_intavcrec}
-\Delta t_{\,ij}(\grad_f P)_{\,ij}\,.\,\vect{S}_{\,ij}=
\frac{\Delta t_{\,ij}S_{\,ij}}{\overline{I^\prime J^\prime}}(P_I-P_J)\\
+(\vect{II}^\prime-\vect{JJ}^\prime).\frac{1}{2}\left[
\Delta t_I(\grad P)_I+\Delta t_J(\grad P)_J\right]
\frac{S_{\,ij}}{\overline{I^\prime J^\prime}}
\end{multline}

Pour les faces de bord, on \'ecrit :
\begin{equation}
\label{Base_Itrmas_eq_brdavcrec}
-\Delta t_{\,b_{ik}}(\grad_f P)_{\,b_{ik}}\,.\,\vect{S}_{\,b_{ik}}=
\frac{\Delta t_{\,b_{ik}} S_{\,b_{ik}}}{\overline{I^\prime F}}
\left[(1-B_{b,ik})(P_I+\vect{II}^\prime.(\grad P)_I)-\var{INC}\times A_{b,ik}\right]
\end{equation}

%%%%%%%%%%%%%%%%%%%%%%%%%%%%%%%%%%
%%%%%%%%%%%%%%%%%%%%%%%%%%%%%%%%%%
\section*{Mise en \oe uvre}
%%%%%%%%%%%%%%%%%%%%%%%%%%%%%%%%%%
%%%%%%%%%%%%%%%%%%%%%%%%%%%%%%%%%%
Les principaux arguments pass\'es \`a \fort{itrmas} et \fort{itrgrp} sont la
variable trait\'ee \var{PVAR} (la pression), ses conditions aux limites, le pas
de temps projet\'e aux faces\footnote{%
Plus pr\'ecis\'ement, le pas de temps projet\'e aux faces, multipli\'e par la
surface et divis\'e par $\overline{I^\prime J^\prime}$ ou $\overline{I^\prime F}$, cf. \fort{viscfa}}
(\var{VISCF} et \var{VISCB}), le pas de temps au
centre des cellules, \'eventuellement anisotrope (\var{VISELX}, \var{VISELY},
\var{VISELZ}). \fort{itrmas} retourne les tableaux \var{FLUMAS} et \var{FLUMAB}
(flux de masse aux faces) mis \`a jour. \fort{itrgrp} retourne directement la
divergence du flux de masse mis \`a jour, dans le tableau \var{DIVERG}.

\etape{Initialisation}
Si \var{INIT} vaut 1, les variables \var{FLUMAS} et \var{FLUMAB} ou \var{DIVERG}
sont mises \`a z\'ero.

\etape{Cas sans reconstruction}
La prise en compte ou non des non orthogonalit\'es est d\'etermin\'ee par
l'indicateur \var{NSWRGR} de la variable trait\'ee (nombre de sweeps de
reconstruction des non orthogonalit\'es dans le calcul des gradients), pass\'e
par la variable \var{NSWRGP}. Une valeur inf\'erieure ou \'egale \`a 1 enclenche
le traitement sans reconstruction.\\
Des boucles sur les faces internes et les faces de bord utilisent directement
les formules (\ref{Base_Itrmas_eq_intssrec}) et (\ref{Base_Itrmas_eq_brdssrec}) pour remplir les
tableaux \var{FLUMAS} et \var{FLUMAB} (\fort{itrmas}) ou des variables de
travail \var{FLUMAS} et \var{FLUMAB} qui servent \`a mettre \`a jour le tableau
\var{DIVERG} (\fort{itrgrp}).

\`A noter que les tableaux \var{VISCF} et \var{VISCB} contiennent respectivement
$\displaystyle\frac{\Delta t_{\,ij}S_{\,ij}}{\overline{I^\prime J^\prime}}$ et
$\displaystyle\frac{\Delta t_{\,b_{ik}}S_{\,b_{ik}}}{\overline{I^\prime F}}$.

\etape{Cas avec reconstruction}
Apr\`es un appel \`a \fort{GRDCEL} pour calculer le gradient cellule de
pression, on remplit les tableaux \var{FLUMAS} et \var{FLUMAB} ou \var{DIVERG}
l\`a encore par une application directe des formules (\ref{Base_Itrmas_eq_intavcrec}) et
(\ref{Base_Itrmas_eq_brdavcrec}).

%%%%%%%%%%%%%%%%%%%%%%%%%%%%%%%%%%
%%%%%%%%%%%%%%%%%%%%%%%%%%%%%%%%%%
\section*{Points \`a traiter}
%%%%%%%%%%%%%%%%%%%%%%%%%%%%%%%%%%
%%%%%%%%%%%%%%%%%%%%%%%%%%%%%%%%%%
Il est un peu redondant de passer en argument \`a la fois le pas de temps
projet\'e aux faces et le pas de temps au centre des cellules. Il faudrait
s'assurer de la r\'eelle n\'ecessit\'e de cela, ou alors \'etudier des
formulations plus simples de la partie reconstruction.


%-------------------------------------------------------------------------------

% This file is part of Code_Saturne, a general-purpose CFD tool.
%
% Copyright (C) 1998-2020 EDF S.A.
%
% This program is free software; you can redistribute it and/or modify it under
% the terms of the GNU General Public License as published by the Free Software
% Foundation; either version 2 of the License, or (at your option) any later
% version.
%
% This program is distributed in the hope that it will be useful, but WITHOUT
% ANY WARRANTY; without even the implied warranty of MERCHANTABILITY or FITNESS
% FOR A PARTICULAR PURPOSE.  See the GNU General Public License for more
% details.
%
% You should have received a copy of the GNU General Public License along with
% this program; if not, write to the Free Software Foundation, Inc., 51 Franklin
% Street, Fifth Floor, Boston, MA 02110-1301, USA.

%-------------------------------------------------------------------------------

\programme{matrix}
\label{ap:matrix}

\hypertarget{matrix}{}

\vspace{1cm}
%%%%%%%%%%%%%%%%%%%%%%%%%%%%%%%%%%
%%%%%%%%%%%%%%%%%%%%%%%%%%%%%%%%%%
\section*{Fonction}
%%%%%%%%%%%%%%%%%%%%%%%%%%%%%%%%%%
%%%%%%%%%%%%%%%%%%%%%%%%%%%%%%%%%%

Le but de ce sous-programme, appel\'e par \fort{codits} et \fort{covofi}, est de construire la
matrice de convection/diffusion, incluant les contributions ad\'equates des termes sources,
intervenant dans le membre de gauche d'\'equations discr\'etis\'ees telles que
celle de la
quantit\'e de mouvement, d'une \'equation de convection diffusion d'un scalaire
ou de mod\`ele de turbulence.\\
Le type d'\'equation consid\'er\'ee est, pour la variable scalaire $a$ :
\begin{equation}\notag
\displaystyle \frac{\partial a}{\partial t} + \dive (\,(\rho \vect{u})\, a) -
\displaystyle \frac{\partial }{\partial x}\left(\beta\,\frac{\partial a}{\partial x}\right) = 0
\end{equation}
La matrice ne s'applique qu'aux termes non reconstruits, les autres \'etant pris en compte au second membre et
trait\'es dans le sous-programme \fort{bilsc2}. La partie
convective, lorsqu'elle existe, est issue du sch\'ema upwind pur, quelque soit
le type de sch\'ema convectif choisi par l'utilisateur. En effet, c'est, \`a
l'heure actuelle, la seule fa\c con d'obtenir un op\'erateur lin\'eaire �
diagonale dominante.\\\\
La matrice est donc associ\'ee \`a $\mathcal{EM_{\it{scal}}}$, op\'erateur
agissant sur un scalaire $a$ (inspir\'e de celui vectoriel $\mathcal{EM}$
d\'efini dans \fort{navstv}) d'expression actuelle, pour tout cellule $\Omega_i$ de
centre $I$  :
\begin{equation}\notag
\begin{array}{lll}
\mathcal{EM_{\it{scal}}}(a,I) &=  f_s^{imp}\ a_I\ \\
&+\sum\limits_{j\in Vois(i)}{F^{\,amont}_{\,ij}((\rho\vect{u})^n,a)}
+\sum\limits_{k\in {\gamma_b(i)}} {F^{\,amont}_{\,{b}_{ik}}((\rho
\vect{u})^n,a)}\\
&-\sum\limits_{j\in Vois(i)}{D^{NRec}_{\,ij}(\beta,a)}
-\sum\limits_{k\in {\gamma_b(i)}} {D^{NRec}_{\,{b}_{ik}}(\beta,a)}\\
\end{array}
\end{equation}
avec~:\\
$\bullet$ $f_s^{imp}$ le coefficient issu du terme instationnaire
$\displaystyle\frac{\rho \ |\Omega_i|}{\Delta t}$, s'il y a lieu, et de
l'implicitation de certains termes sources (contribution d\'ecoulant de la prise
en compte de la
variation $\displaystyle\frac{\partial \rho }{\partial t}$ de
la masse volumique $\rho$ au cours du temps, diagonale du tenseur de pertes de
charges par exemple...): cette initialisation est en fait effectu\'ee en amont
de ce sous-programme, \\
$\bullet$ $F^{\,amont}_{\,ij}$ le flux num\'erique convectif scalaire
d\'ecentr\'e amont calcul\'e \`a la face interne $ij$ de la cellule $\Omega_i$,\\
$\bullet$ $F^{\,amont}_{\,b_{ik}}$ le flux num\'erique convectif scalaire
d\'ecentr\'e amont associ\'e calcul\'e \`a la face de bord $ik$ de la cellule $\Omega_i$
jouxtant le bord du domaine $\Omega$,\\
$\bullet$ $D^{\,NRec}_{\,ij}$ le flux num\'erique diffusif scalaire non
reconstruit associ\'e calcul\'e \`a la face interne $ij$ de la cellule $\Omega_i$,\\
$\bullet$ $D^{\,NRec}_{\,{b}_{ik}}$ le flux num\'erique diffusif scalaire non
reconstruit associ\'e calcul\'e \`a la face de bord $ik$ de la cellule $\Omega_i$ jouxtant le bord du domaine $\Omega$,\\
$\bullet$ $Vois(i)$ repr\'esente toujours l'ensemble des cellules
voisines de ${\Omega_i}$ et $\gamma_b(i)$ l'ensemble des faces de
bord de ${\Omega_i}$.\\
Du fait de la r\'esolution en incr\'ements, $a$ est un incr\'ement et ses
conditions aux limites associ\'ees sont donc de type Dirichlet homog\`ene ou de
type
Neumann homog\`ene.

See the \doxygenfile{cs__matrix_8c.html}{programmers reference of the dedicated subroutine} for further details.

%%%%%%%%%%%%%%%%%%%%%%%%%%%%%%%%%%
%%%%%%%%%%%%%%%%%%%%%%%%%%%%%%%%%%
\section*{Discr\'etisation}
%%%%%%%%%%%%%%%%%%%%%%%%%%%%%%%%%%
%%%%%%%%%%%%%%%%%%%%%%%%%%%%%%%%%%

\begin{figure}[h]
\parbox{8cm}{%
\centerline{\includegraphics[height=4cm]{facette}}}
\parbox{8cm}{%
\centerline{\includegraphics[height=4cm]{facebord}}}
\caption{\label{Base_Matrix_fig_geom_gradmc}D\'efinition des diff\'erentes entit\'es
g\'eom\'etriques pour les faces internes (gauche) et de bord (droite).}
\end{figure}

L'op\'erateur $\mathcal{EM_{\it{scal}}}$ s'\'ecrit, pour tout $I$ centre de cellule :
\begin{equation}
\begin{array}{lll}
\mathcal{EM_{\it{scal}}}(a,I) &=  f_s^{imp}\ a_I \\
&+\sum\limits_{j\in Vois(i)}\left[{(\rho \vect{u})_{\,ij}^n} \text{.}\, \vect{S}_{\,ij}\right]\ a_{\,f,ij}
+\sum\limits_{k\in {\gamma_b(i)}} \left[{(\rho \vect{u})_{\,{b}_{ik}}^n}
\text{.}\, \vect{S}_{\,{b}_{ik}}\right]\ {a_f}_{\,{b}_{ik}} \\
&-\sum\limits_{j\in Vois(i)} \beta_{\,ij}\displaystyle
\frac{a_{\,J}- a_{\,I}}{\overline{I'J'}} S_{\,ij}
-\sum\limits_{k\in {\gamma_b(i)}} \beta_{\,b_{ik}}\displaystyle
\frac{a_{\,b_{ik}}-a_{\,I}}{\overline{I'F}} S_{\,b_{ik}} \\
\end{array}
\end{equation}
o\`u
$a_{\,f,ij} = a_{\,I} \text{ ou }  a_{\,J}$
selon le signe de $(\rho \vect{u})_{\,ij}^n.\vect{S}_{\,ij}$ (sch�ma
convectif upwind syst�matique),
et avec $\overline{I'J'}$, mesure alg\'ebrique, orient\'ee comme la
normale sortante \`a la face, {\it i.e.} allant de $I$ vers $J$ pour la cellule
$\Omega_i$ de centre $I$. On la
notera ${\overline{I'J'}^{\tiny {\,I}}}$ lorsqu'on aura besoin d'expliciter
clairement l'orientation.\\
${a_f}_{\,{b}_{ik}} = a_I \text{ ou  }
a_{\ {b}_{ik}}$ selon le signe de
${(\rho \vect{u})_{\,{b}_{ik}}^n}\text{.}\, \vect{S}_{\,{b}_{ik}}$ (sch�ma
upwind syst�matique)
et $a_{\ {b}_{ik}}$ valeur au bord est donn\'ee directement par les conditions
aux limites (valeur non reconstruite). $\overline{I'F}$, mesure alg\'ebrique, orient\'ee relativement \`a la
normale sortante \`a la face, {\it i.e.} allant de $I$ vers l'ext\'erieur du domaine.\\
En g\'en\'eral, sauf cas pathologiques, les mesures alg\'ebriques
$\overline{I'J'}$ et $\overline{I'F}$
sont positives et correspondent aux distances $I'J'$ et $I'F$. On se reportera
au paragraphe Points \`a traiter pour plus de d\'etails.\\
Soit ${\tens{EM}}_{\,scal}$ la matrice associ\'ee ; sa taille est {\it a priori} de
$\var{NCEL} * \var{NCEL}$, mais compte-tenu de la nature de la structure de
donn\'ees, seuls deux tableaux \var{DA(NCEL)} contenant les valeurs
diagonales et \var{XA(NFAC,*)} les contributions des termes extra-diagonaux sont n\'ecessaires, avec \var{NCEL} nombre de
cellules du maillage consid\'er\'e et \var{NFAC} nombre de faces internes associ\'e.\\
Du fait des simplifications effectu\'ees sur la matrice (non reconstruction des
termes), les composantes extradiagonales de la ligne $I$ ne sont diff\'erentes de z\'ero que pour
les indices $J$ des cellules voisines de $I$. On peut donc stocker toutes les
contributions non nulles de la matrice dans deux tableaux \var{DA(NCEL)} et \var{XA(NFAC,2)} :\\
$\bullet$ \var{DA(I)} est le coefficient de la colonne $I$ dans la ligne $I$,\\
$\bullet$ si \var{IFAC} est une face qui s\'epare les cellules $\Omega_i$
et $\Omega_j$, orient\'ee de $I$ vers $J$, alors :\\
$\var{XA(IFAC,1)}$ est le coefficient de la colonne $J$ dans la ligne $I$ et
$\var{XA(IFAC,2)}$ est le coefficient de la colonne $I$ dans la ligne $J$.
Lorsque la matrice est sym\'etrique, {\it
i.e.} lorsqu'il n'y a pas de convection (\var{ICONVP} = 0) et que seule la
diffusion est \`a prendre en compte, alors $\var{XA(IFAC,1)} = \var{XA(IFAC,2)}
$ et on r\'eduit \var{XA} \`a $\var{XA(NFAC,1)}$.\\\\
Soit $m_{\,ij}^n$ (\ $m_{\,{b}_{ik}}^n$\ ) la valeur de $(\rho
\vect{u})_{\,ij}^n.\vect{S}_{\,ij}$ (respectivement $(\rho
\vect{u})_{\,{b}_{ik}}^n\text{.}\,\vect{S}_{\,{b}_{ik}}$).\\
Alors~:\\
\hspace*{1cm}{\tiny$\blacksquare$}\ \underline{contribution volumique} : $ f_s^{\,imp}\ a_I $\\\\
\hspace*{1cm}{\tiny$\blacksquare$}\ \underline{contribution d'une face purement interne $ij$} \\
L'expression \\
\begin{equation}\notag
+ \sum\limits_{j\in Vois(i)}{F^{\,amont}_{\,ij}((\rho \vect{u})^n, a)}
- \sum\limits_{j\in Vois(i)}{D^{NRec}_{\,ij}(\,\beta, a)}
\end{equation}
 s'\'ecrit :
\begin{equation}\label{Base_Matrix_eq_face_int}
\begin{array}{ll}
&\ \sum\limits_{j\in Vois(i)}\displaystyle\left({ \left[{(\rho \vect{u})_{\,ij}^n} \text{.}\,
\vect{S}_{\,ij}\right]\ \ a_{\,f,ij} - \beta_{\,ij}\frac{a_J - a_I}{\overline{I'J'}} S_{\,ij}}\right)\\
&=\sum\limits_{j\in Vois(i)}\left[\displaystyle\frac{1}{2}(\  m_{\,ij}^n + |\
m_{\,ij}^n|\ )\,a_I + \displaystyle\frac{1}{2}(\ m_{\,ij}^n - |\
m_{\,ij}^n|)\,a_J\right] - \sum\limits_{j\in Vois(i)}\displaystyle \beta_{\,ij}\frac{a_J - a_I}{\overline{I'J'}} S_{\,ij}
\end{array}
\end{equation}\\
Ici, $\overline{I'J'} = {\overline{I'J'}^{\tiny {\,I}}}$.\\\\
\hspace*{1cm}{\tiny$\blacksquare$}\ \underline{contribution d'une face de bord $ik$} \\
De m\^eme :
\begin{equation}\label{Base_Matrix_eq_face_bord}
\begin{array}{ll}
&\ \sum\limits_{k\in {\gamma_b(i)}} {F^{\,amont}_{\,{b}_{ik}}((\rho \vect{u})^n,a)}
- \sum\limits_{k\in {\gamma_b(i)}} {D^{NRec}_{\,{b}_{ik}}(\beta, a)}\\
&=\sum\limits_{k\in {\gamma_b(i)}}\displaystyle\left(\left[{(\rho
\vect{u})_{\,{b}_{ik}}^n} \text{.}\, \vect{S}_{\,{b}_{ik}}\right]\
{a_f}_{\,{b}_{ik}} - \beta_{\,b_{ik}}
\frac{a_{\,b_{ik}}- a_I}{\overline{I'F}} S_{\,b_{ik}}\right)\\
&=\sum\limits_{k\in {\gamma_b(i)}}\left[\displaystyle\frac{1}{2}(\ m_{\,{b}_{ik}}^n + |\ m_{\,{b}_{ik}}^n|\ )\,a_I +
\displaystyle\frac{1}{2}(\ m_{\,{b}_{ik}}^n -
|m_{\,{b}_{ik}}^n|)\,a_{\,{b}_{ik}}\right] - \sum\limits_{k\in {\gamma_b(i)}}\displaystyle\beta_{\,b_{ik}}
\frac{a_{\,b_{ik}}- a_I}{\overline{I'F}} S_{\,b_{ik}}
\end{array}
\end{equation}
avec~:
\begin{equation}\notag
a_{\,{b}_{ik}} = \var{INC}\,A_{\,b,ik} + B_{\,b,ik}\,a_I =  B_{\,b,ik}\,a_I
\end{equation}
$a$ n'\'etant associ\'ee qu'\`a des conditions aux limites de type Dirichlet
homog\`ene ou de type Neumann homog\`ene.

%%%%%%%%%%%%%%%%%%%%%%%%%%%%%%%%%%
%%%%%%%%%%%%%%%%%%%%%%%%%%%%%%%%%%
\section*{Mise en \oe uvre}
%%%%%%%%%%%%%%%%%%%%%%%%%%%%%%%%%%
%%%%%%%%%%%%%%%%%%%%%%%%%%%%%%%%%%
\subsection*{\bf Initialisations}
L'indicateur de sym\'etrie \var{ISYM} de la matrice consid\'er\'ee est affect\'e comme suit :\\
\hspace*{1cm}$\bullet$ $\var{ISYM}$ = 1 , si la matrice est sym\'etrique ;
on travaille en diffusion pure , \var{ICONVP} = 0 et \var{IDIFFP} = 1,\\
\hspace*{1cm}$\bullet$ $\var{ISYM}$ = 2 , si la matrice est non sym\'etrique ;
on travaille soit en convection pure (~\var{ICONVP}~=~1, \var{IDIFFP}~=~0~), soit en
convection/diffusion (~\var{ICONVP}~=~1,~\var{IDIFFP}~=~1~).\\
Les termes diagonaux de la matrice sont stock\'es dans le tableau
$\var{DA(NCEL)}$. Ceux extra-diagonaux le sont dans $\var{XA(NFAC,1)}$ si la
matrice est sym\'etrique, dans $\var{XA(NFAC,2)}$ sinon.


Le tableau $\var{DA}$ est initialis\'e \`a z\'ero pour un calcul avec
$ \var{ISTATP} = 0 $ (en fait, ceci ne concerne que les calculs relatifs \`a la
pression). Sinon, on lui affecte la valeur \var{ROVSDT} comprenant la partie instationnaire, la contribution du terme continu en $-\ a\ \dive(\rho \vect{u})^n$
et la partie diagonale des termes sources implicit\'es. Le tableau
$\var{XA(NFAC,*)}$ est initialis\'e \`a z\'ero.\\
\subsection*{\bf Calcul des termes extradiagonaux stock\'es dans \var{XA} }
Ils ne se calculent que pour des faces purement internes \var{IFAC}, les faces
de bord ne contribuant qu'\`a la diagonale.
\subsubsection*{matrice non sym\'etrique ( pr\'esence de convection ) }
%\hspace*{1cm}{\tiny$\blacksquare$}\ \underline{pour une face purement interne
%$ij\ ( \var{IFAC} )$} \\
Pour chaque face interne \var{IFAC}, les contributions extradiagonales relatives
au terme $a_I$ et \`a son voisin  associ\'e $a_J$ sont calcul\'ees dans
$\var{XA(IFAC,1)}$ et $\var{XA(IFAC,2)}$ respectivement (pour une face orient�e
de I vers J).\\
On a les relations suivantes :\\
\begin{equation}\label{Base_Matrix_eq_extra}
\begin{array}{ll}
\var{XA(IFAC,1)}& = \var{ICONVP}\,*\,\var{FLUI} - \var{IDIFFP}\,*\,\var{VISCF(IFAC)}\\
\var{XA(IFAC,2)}& = \var{ICONVP}\,*\,\var{FLUJ} - \var{IDIFFP}\,*\,\var{VISCF(IFAC)}\\
\end{array}
\end{equation}
avec $\var{FLUMAS(IFAC)}$  correspondant \`a $\ m_{\,ij}^n$, $\var{FLUI}$ \`a $ \displaystyle\frac{1}{2}\,(\ m_{\,ij}^n - |\
m_{\,ij}^n|\ )$, $\var{VISCF(IFAC)} $ \`a $ \ \displaystyle \beta_{\,ij}\frac {
S_{\,ij}}{\overline{I'J'}} $.\\\\
$\var{XA(IFAC,1)}$ repr\'esente le facteur de $a_J$ dans la
derni\`ere expression de (\ref{Base_Matrix_eq_face_int}).\\

$\var{FLUJ}$ correspond \`a $\ -\displaystyle\frac{1}{2}\,(\ m_{\,ij}^n + |\
m_{\,ij}^n|\ )$. En effet, $\var{XA(IFAC,2)}$ est le facteur de $a_I$ dans l'expression \'equivalente de
la derni\`ere ligne de (\ref{Base_Matrix_eq_face_int}), mais \'ecrite en J.\\
Ce qui donne :\\
\begin{equation}\label{Base_Matrix_eq_extra_J}
\sum\limits_{i\in
Vois(j)}\left[\displaystyle\frac{1}{2}(\ m_{\,ji}^n + |\ m_{\,ji}^n|\ )\,a_J +
\displaystyle\frac{1}{2}(\ m_{\,ji}^n - |\ m_{\,ji}^n|)\,a_I\right]
 - \sum\limits_{i\in Vois(j)}\displaystyle \beta_{\,ji}\frac{a_I - a_J}{\overline{J'I'}} S_{\,ji}
\end{equation}\\
Le terme recherch\'e est donc :
$\ \displaystyle\frac{1}{2}(\ m_{\,ji}^n - |\ m_{\,ji}^n|\ )-\displaystyle
\beta_{\,ji}\frac {S_{\,ji}}{\overline{J'I'}}$ .\\
Or :\\ $ m_{\,ji}^n $ = $\ - m_{\,ij}^n $  ($\vect{S}_{\,ji} = -
\vect{S}_{\,ij}$ et $(\rho \vect{u})_{\,ji}^n\  =\ (\rho \vect{u})_{\,ij}^n\
$), avec $\overline{J'I'}$ mesure alg\'ebrique, orient\'ee relativement \`a la
normale sortante \`a la face, {\it i.e.} allant de $J$ vers $I$. On la note
${\overline{J'I'}^{\tiny {\,J}}}$. \\
On a la relation :\\
\begin{equation}\label{Base_Matrix_Eq_mesure_alg}
\overline{J'I'}^{\tiny {\,J}}=\ \overline{I'J'}^{\tiny {\,I}} = (\ \overline{I'J'})
\end{equation}
d'o\`u la deuxi\`eme \'egalit\'e dans (\ref{Base_Matrix_eq_extra}).
\subsubsection*{matrice sym\'etrique ( diffusion pure ) }
Pour chaque face interne \var{IFAC}, la contribution extradiagonale relative au
terme $a_I$ est calcul\'ee dans
$\var{XA(IFAC,1)}$ par la  relation suivante :\\
\begin{equation}
\var{XA(IFAC,1)} =  - \var{IDIFFP}\,*\,\var{VISCF(IFAC)}\\
\end{equation}
avec $\var{VISCF(IFAC)} $ \`a $ \ \displaystyle \beta_{\,ij}\frac {
S_{\,ij}}{\overline{I'J'}} $.
\subsection*{\bf Calcul des termes diagonaux stock\'es dans \var{DA} }
\subsubsection*{matrice non sym\'etrique ( pr\'esence de convection ) }
Pour chaque face interne $ij\ ( \var{IFAC} )$ s\'eparant les cellules $\Omega_i$
de centre $I$ et $\Omega_j$ de centre $J$, $\var{DA(II)}$ est la quantit\'e en facteur de $a_I$ dans la
derni\`ere expression de (\ref{Base_Matrix_eq_face_int}), soit :
\begin{equation}\label{Base_Matrix_eq_diag_II}
\displaystyle\frac{1}{2}(\ m_{\,ij}^n + |\ m_{\,ij}^n|\ )+\displaystyle
\beta_{\,ij}\frac {S_{\,ij}}{\overline{I'J'}}
\end{equation}
De m\^eme, pour \var{DA(JJ)}, on a :
\begin{equation}\label{Base_Matrix_eq_diag_JJ}
\displaystyle\frac{1}{2}(\ - m_{\,ij}^n + |\ m_{\,ij}^n|\ )+\displaystyle
\beta_{\,ji}\frac {S_{\,ij}}{\overline{I'J'}}
\end{equation}
d'apr\`es l'expression de (\ref{Base_Matrix_eq_extra_J}) et l'\'egalit\'e (\ref{Base_Matrix_Eq_mesure_alg}).\\
L'implantation dans \CS associ\'ee est la suivante~:\\
pour toute face \var{IFAC} d'\'el\'ements voisins $\var{II} =
\var{IFACEL(1,IFAC)}$ et $\var{JJ} = \var{IFACEL(2,IFAC)}$,\\
on ajoute \`a
$\var{DA(II)}$ la contribution crois\'ee $-\var{XA(IFAC,2)}$ ({\it cf.}
(\ref{Base_Matrix_eq_diag_II}))  et
\`a
$\var{DA(JJ)}$ la contribution $-\var{XA(IFAC,1)}$ ({\it cf.}
(\ref{Base_Matrix_eq_diag_JJ})).
\subsection*{\bf Prise en compte des conditions aux limites}
Elles interviennent juste dans le tableau \var{DA}, compte-tenu de leur
\'ecriture et d\'efinition. Elles se calculent {\it via} la derni\`ere
expression de (\ref{Base_Matrix_eq_face_bord}). Pour chaque face \var{IFAC}, de l'\'el\'ement
de centre $I$, jouxtant le bord, on s'int\'eresse \`a :
\begin{equation}
\begin{array}{ll}
\sum\limits_{k\in {\gamma_b(i)}}\left[\displaystyle\frac{1}{2}(\ m_{\,{b}_{ik}}^n + |\ m_{\,{b}_{ik}}^n|\ )\,a_I +
\displaystyle\frac{1}{2}(\ m_{\,{b}_{ik}}^n -
|m_{\,{b}_{ik}}^n|)\,a_{\,{b}_{ik}}\right] - \sum\limits_{k\in {\gamma_b(i)}}\displaystyle\beta_{\,b_{ik}}
\frac{a_{\,b_{ik}}- a_I}{\overline{I'F}} S_{\,b_{ik}}
\end{array}
\end{equation}
avec~:
\begin{equation}\notag
a_{\,{b}_{ik}} =  B_{\,b,ik}\,a_I\\
\end{equation}
soit :
\begin{equation}
\begin{array}{ll}
\left(\sum\limits_{k\in {\gamma_b(i)}}\left[\displaystyle\frac{1}{2}(\ m_{\,{b}_{ik}}^n + |\ m_{\,{b}_{ik}}^n|\ )\,+
\displaystyle\frac{1}{2}(\ m_{\,{b}_{ik}}^n -
|m_{\,{b}_{ik}}^n|)B_{\,b,ik}\,\right] + \sum\limits_{k\in {\gamma_b(i)}}\displaystyle\beta_{\,b_{ik}}
\frac{1 -\ B_{\,b,ik}}{\overline{I'F}} S_{\,b_{ik}}\right) a_I
\end{array}
\end{equation}
ce qui, pour le terme sur lequel porte la somme, se traduit par :\\
$\var{ICONVP} * (- \var{FLUJ} + \var{FLUI} * \var{COEFBP(IFAC)} + \var{IDIFFP} *
\var{VISCB(IFAC)} * (\ 1 -\ \var{COEFBP(IFAC)})$ \\ avec,
$\ m_{\,{b}_{ik}}^n\ $ repr\'esent\'e par $\ \var{FLUMAB(IFAC)}\ $,
$\ \displaystyle\frac{1}{2}\ (\
m_{\,{b}_{ik}}^n + |\ m_{\,{b}_{ik}}^n|\ )\ $ par $\ \var{-\ FLUJ}\ $,\\
$\ \displaystyle\frac{1}{2}\ (\ m_{\,{b}_{ik}}^n -
|m_{\,{b}_{ik}}^n|)B_{\,b,ik}\ $ par $\ \var{FLUI}\ $,
$B_{\,b,ik}$ par $\var{COEFBP(IFAC)}$, $\beta_{\,b_{ik}}\displaystyle\frac
{S_{\,b_{ik}}}{\overline{I'F}} $ par $\var{VISCB(IFAC)}$.\\
\subsection*{\bf D\'ecalage du spectre}
Lorsqu'il n'existe aucune condition \`a la limite de type Dirichlet et que
$\var{ISTATP} = 0 $ (c'est-\`a-dire pour la pression uniquement), on
d\'eplace le spectre de la matrice ${\tens{EM}}_{\,scal}$ de $\var{EPSI}$  ({\it i.e.} on multiplie chaque terme diagonal par $(1 + \var{EPSI})$ ) afin
de la rendre inversible. \var{EPSI} est fix\'e en dur dans \fort{matrix} \`a
 ${10}^{-7}$.
%%%%%%%%%%%%%%%%%%%%%%%%%%%%%%%%%%
%%%%%%%%%%%%%%%%%%%%%%%%%%%%%%%%%%
\section*{Points \`a traiter}
%%%%%%%%%%%%%%%%%%%%%%%%%%%%%%%%%%
%%%%%%%%%%%%%%%%%%%%%%%%%%%%%%%%%%
\etape{Initialisation}
Le tableau \var{XA} est initialis\'e \`a z\'ero lorsqu'on veut annuler la
contribution du terme en
$\displaystyle\frac{\rho \ |\Omega_i|}{\Delta t}$, {\it i.e.} $\var{ISTATP} = 0 $ . Ce qui ne permet donc pas la prise en
compte effective des parties diagonales des termes sources \`a impliciter,
d\'ecid\'ee par l'utilisateur. Actuellement, ceci ne sert que pour la variable
pression et reste donc {\it a priori} correct, mais cette d\'emarche est \`a
corriger dans l'absolu.\\\\
\etape{Nettoyage}
La contribution $\var{ICONVP}\ \var{FLUI}$, dans le calcul du terme
\var{XA(IFAC,1)} lorsque la matrice est sym\'etrique est inutile, car
$\var{ICONVP}\ = 0$. \\\\
\etape{Prise en compte du type de sch\'ema de convection dans
${\tens{EM}}_{\,scal}$}
Actuellement, les contributions des  flux convectifs non reconstruits sont
trait\'ees par sch\'ema d\'ecentr\'e amont, quelque soit le sch\'ema choisi par
l'utilisateur. Ceci peut \^etre handicapant. Par exemple, m\^eme sur
maillage orthogonal, on est oblig\'e de faire plusieurs sweeps pour obtenir une
vitesse pr\'edite correcte. Un sch\'ema centr\'e sans test de pente peut
\^etre implant\'e facilement, mais cette \'ecriture pourrait, dans l'\'etat
actuel des connaissances, entra\^\i ner des instabilit\'es
num\'eriques. Il serait souhaitable d'avoir d'autres sch\'emas tout aussi
robustes, mais plus adapt\'es \`a certaines configurations.\\\\
\etape{Maillage localement pathologique}
Il peut arriver, notamment au bord, que les mesures alg\'ebriques,
$\overline{I'J'}$ ou $\overline{I'F}$ soient n\'egatives (en cas de maillages
non convexes par exemple). Ceci peut engendrer des probl\`emes plus ou moins
graves : perte de l'existence et l'unicit\'e de la solution (l'op\'erateur associ\'e n'ayant plus les bonnes propri\'et\'es de r\'egularit\'e
ou de coercivit\'e), d\'egradation de la matrice (perte de la positivit\'e) et donc r\'esolution par solveur lin\'eaire
associ\'e non appropri\'e (gradient conjugu\'e par exemple).\\
Une impression permettant de signaler et de localiser le probl\`eme serait souhaitable.



\include{navsto}
%-----------------------------------------------------------------------
%
%     This file is part of the Code_Saturne Kernel, element of the
%     Code_Saturne CFD tool.
%
%     Copyright (C) 1998-2008 EDF S.A., France
%
%     contact: saturne-support@edf.fr
%
%     The Code_Saturne Kernel is free software; you can redistribute it
%     and/or modify it under the terms of the GNU General Public License
%     as published by the Free Software Foundation; either version 2 of
%     the License, or (at your option) any later version.
%
%     The Code_Saturne Kernel is distributed in the hope that it will be
%     useful, but WITHOUT ANY WARRANTY; without even the implied warranty
%     of MERCHANTABILITY or FITNESS FOR A PARTICULAR PURPOSE.  See the
%     GNU General Public License for more details.
%
%     You should have received a copy of the GNU General Public License
%     along with the Code_Saturne Kernel; if not, write to the
%     Free Software Foundation, Inc.,
%     51 Franklin St, Fifth Floor,
%     Boston, MA  02110-1301  USA
%
%-----------------------------------------------------------------------
%

\programme{preduv}
%
\vspace{1cm}
%%%%%%%%%%%%%%%%%%%%%%%%%%%%%%%%%%
%%%%%%%%%%%%%%%%%%%%%%%%%%%%%%%%%%
\section{Fonction}
%%%%%%%%%%%%%%%%%%%%%%%%%%%%%%%%%%
%%%%%%%%%%%%%%%%%%%%%%%%%%%%%%%%%%
Dans ce sous-programme, on effectue l'\'etape de pr\'ediction de la vitesse
$\vect{u}$. Ceci consiste � r\'esoudre l'�quation de quantit\'e de
mouvement (Q.D.M.) en traitant la pression $p$ de mani�re explicite. La solution en vitesse-pression est obtenue apr�s une �tape de correction sur la pression
effectu�e dans le sous-programme \fort{resolp}, en utilisant la loi de conservation de la masse :
\begin{equation}
\frac{\partial \rho } {\partial t}+ \dive(\rho \underline{u}) = \Gamma,
\end{equation}
o\`u $\Gamma$ est le terme source de masse\footnote{ en $kg.m^{-3}.s^{-1}$ }.\\
L'�quation de conservation de la quantit� de mouvement moyenne obtenue par application
du th�or�me fondamental de la dynamique est :
\begin{equation}
\frac {\partial (\rho \underline {u})} {\partial t }+
\dive(\rho \underline{u} \otimes \underline{u}) =
\dive(\underline{\underline{\sigma}}) + \underline{S} - \dive{(\rho\,\tens{R})}\end{equation}
o� :
\begin{equation}
\underline{\underline{\sigma}} = - p \underline{\underline{Id}} + \underline{\underline{\tau }}
\end{equation}
avec pour les �coulements newtoniens, la relation lin�aire suivante :
\begin{equation}
\begin{array}{lcl}
&\displaystyle \underline{\underline{\tau}} = 2\ \mu\ \underline{\underline{D}}
+\,
 \lambda\ tr(\underline{\underline{D }})\ \underline{\underline{Id}} &\\
&\displaystyle \underline{\underline{D}}=\frac{1}{2}\ (\ggrad \underline
{u} +\ ^t\ggrad \underline {u})
\end{array}
\end{equation}


$\tens{\sigma}$ repr\'esente le tenseur de contraintes, $\tens{\tau}$ le tenseur
des contraintes visqueuses, $\mu$ la viscosit\'e dynamique (mol\'eculaire et
\'eventuellement turbulente), $\tens{D}$
 le tenseur taux de d\'eformation\footnote{\`A ne pas confondre, malgr\'e la m\^eme notation $D$, avec les flux
diffusifs d\'ecrits dans le sous-programme \fort{navsto}},
$\tens{R}$ le tenseur de Reynolds qui appara\^\i t lors de l'application de
l'op\'erateur moyenne \`a l'\'equation instantan\'ee, $\underline{S}$ les termes
sources.\\
$\lambda$ est le second coefficient de viscosit�. Il est reli� � la viscosit� de
volume $\kappa$ par la relation
\begin{equation}
\lambda=\kappa-\frac{2}{3}\mu
\end{equation}
Quand l'hypoth�se de Stokes est v�rifi�e, la viscosit� de volume $\kappa$ est
nulle, soit $3\lambda+2\mu=0$. Cette hypoth�se est implicite dans le code et
dans le reste du doument, sauf en compressible.\\


L'\'equation de conservation de la quantit� de mouvement s'\'ecrit finalement
 :
\begin{equation}
\begin{array}{lcl}
&\displaystyle \rho\,
\frac{\partial \underline {u} } {\partial t} = -\
\underbrace {\dive(\rho \underline{u} \otimes \underline{u})}_{\text{
convection}} +\ \underbrace {\dive (\mu\ \ggrad \underline {u})}_{\text{
diffusion}} &\\
&\displaystyle \underbrace { +\ \dive (\mu \,^t\ggrad \underline {u}) }_{\text{
terme en gradient transpos\'e}}
\underbrace { - \ \frac {2} {3}\ \grad (\mu\ \dive \underline {u})}_{\text{
viscosit� secondaire}}\ \ - \dive{(\rho \tens{R})}
 -\ \grad(p) + (\rho -\rho_0)\,\underline {g} +
\underline{u}\,\dive (\rho\,\underline {u})&\\
&\displaystyle +\underbrace {\Gamma
(\underline{u}_{\,i}-\underline{u})}_{\text{terme source de Q.D.M. d� � la source
de masse}}- \underbrace {\rho\
\tens{K}_{\,pdc} \underline {u}}_{\text{perte
de charge}} + \underbrace { \underline{T}_{\,s}^{\,exp}+
T_{s}^{\,imp}\ \underline{u}}_{\text{autres termes sources de
Q.D.M.}}
\label{Base_Preduv_eqqdm}

\end{array}
\end{equation}
avec $p$ d�finissant l'�cart � la pression hydrostatique de r\'ef\'erence (la
pression hydrostatique r\'eelle \'etant calcul\'ee avec la masse volumique $\rho$
et non $\rho_{\,0}$) :
\begin{equation}
p=p^*-\rho_{\,0}\ \underline{g}\,.\,\underline{X}
\end{equation}
(\underline{X} \'etant le vecteur de composantes $x$, $y$ et $z$).\\
$\mu_t$, $\tens{K}_{\,pdc}$, $\underline{u}_{\,i}$ repr\'esentent respectivement
la viscosit� dynamique turbulente, le tenseur des pertes de charge et la valeur de la variable associ�e � la source de
masse.\\
La divergence du tenseur des contraintes de Reynolds s'\'ecrit :
\begin{equation}
-\dive{(\rho\,\tens{R})}=
\begin{cases}
\vect{0} & \text{en laminaire}, \\
 -\displaystyle\frac {2} {3}\, \grad (\mu_t\ \dive \underline {u})+\dive (\mu_t\ (\ggrad \underline {u}+ \,^t\ggrad \underline {u}))-\frac {2}{3}\,\grad (\rho\, k) & \text{pour les mod�les}\\
 & \text{� viscosit� turbulente}, \\
 -\dive(\rho\,\tens{R})& \text{pour les mod�les}\\
 & \text{au second ordre},\\
-\displaystyle\frac {2} {3}\, \grad (\mu_t\ \dive \underline {u})+\dive (\mu_t\ (\ggrad \underline {u}+ \,^t\ggrad \underline {u})) & \text{en  LES}\\
\end{cases}
\end{equation}
Le terme source de masse fait intervenir la vitesse locale $\underline {u}$ et
aussi une vitesse $\underline {u}_{\,i}$ associ\'ee \`a la masse inject\'ee (ou retir\'ee).
Lorsque $\Gamma<0$, on \^ote de la masse au syst\`eme et on a donc
$\underline{u}_{\,i} = \underline{u}$. Le terme est nul (\emph{i.e.} $\Gamma
(\underline{u}_{\,i}-\underline{u})= \underline{0} $). Quand $\Gamma>0$, on a un
terme non nul si $\underline{u}_{\,i} \ne \underline{u}$.
Dans ce sous-programme, tous les termes intervenant dans
l'�quation de conservation de la
quantit� de mouvement, except� les termes de convection et diffusion, sont
calcul�s et transmis au sous-programme \fort{codits} qui r�sout l'�quation compl�te
(convection-diffusion avec termes sources).

%%%%%%%%%%%%%%%%%%%%%%%%%%%%%%%%%%
%%%%%%%%%%%%%%%%%%%%%%%%%%%%%%%%%%
\section{Discr\'etisation}
%%%%%%%%%%%%%%%%%%%%%%%%%%%%%%%%%%
%%%%%%%%%%%%%%%%%%%%%%%%%%%%%%%%%%

Le terme convectif en $\dive(\underline{u} \otimes \rho\,\underline{u})$
introduit une non lin\'earit\'e et un couplage des composantes de la vitesse
$\vect{u}$ dans l'�quation (\ref{Base_Preduv_eqqdm}). Une lin\'earisation et un d\'ecouplage
des trois composantes de la
vitesse sont r\'ealis\'es lors de la discr\'etisation de cette \'etape de
pr\'ediction.\\
En effet, soit :
\begin{equation}
\vect{\widetilde{u}}= \vect{u}^n + \delta \vect{u}
\end{equation}
La contribution exacte du terme convectif \`a prendre en compte dans cette
\'etape de pr\'ediction serait :\\
\begin{equation}\label{Base_Preduv_Conv_exact}
\begin{array}{ll}
\dive(\vect{\widetilde{u}} \otimes \rho\,\vect{\widetilde{u}}) =
\dive(\vect{u}^{n} \otimes \rho\,\vect{u}^{n}) + \dive(\delta \vect{u} \otimes
\rho\,\vect{u}^{n}) +  \underbrace { \dive(\vect{u}^{n} \otimes
\rho\,\delta \vect{u})}_{\text {terme couplant lin\'eaire}} +  \underbrace { \dive(\delta \vect{u} \otimes
\rho\,\delta \vect{u})}_{\text {terme couplant et non lin\'eaire}}\\
\end{array}
\end{equation}
Les deux derniers termes de l'expression (\ref{Base_Preduv_Conv_exact}) sont {\em a priori} n�glig�s
de mani�re � obtenir un syst\`eme en vitesse qui soit d\'ecoupl\'e et donc,
�viter l'inversion d'une matrice pouvant \^etre de tr\`es grande taille. Ces
deux termes peuvent n�anmoins �tre pris en compte de mani�re plus ou moins
approch�e par extrapolation explicite du flux de masse en $n+\theta_F$ (pour le
terme couplant lin�aire provenant de la convection de $\vect{u}^{n}$ par $\delta
\vect{u}$) et utilisation d'un point-fixe par sous it�ration sur le sous
programme \fort{navsto} (pour le terme non-lin�aire, en sp�cifiant $\var{NTERUP}>1$).

L'�quation (\ref{Base_Preduv_eqqdm}) est discr�tis�e au temps $n+\theta$ � l'aide d'un
$\theta$-sch�ma, et le tenseur des pertes de charges d�compos� en une partie
diagonale $\tens{K}_{d}$ et une extradiagonale $\tens{K}_{e}$ (soit
 $\tens{K}_{pdc}=\tens{K}_{d}+\tens{K}_{e}$).\\
$\bullet$ La pression est suppos�e connue en $n-1+\theta$ (d�calage temporel
pression-vitesse) et le gradient naturellement calcul� � cet instant.\\
$\bullet$ Les termes sources de viscosit� secondaire, de gradient transpos\'e,
ceux provenant du mod�le de turbulence\footnote{except� $\dive (\mu_t\ (\ggrad
\underline {u}))$}, $\rho\,\tens{K}_{\,e}\ \underline{u}$, $(\rho -\rho_0)
\underline {g}$ ainsi que $\underline{T}_{s}^{\,exp}$ et
$\Gamma\,\underline{u}_{\,i}$ sont pris de mani�re explicite au temps $n$, ou
extrapol�s suivant le sch�ma en temps choisi pour les propri�t�s physique et les
termes sources.\\
$\bullet$ Les termes sources $\underline{u}\,\,\dive (\rho\,\underline {u})$,
$\Gamma\,\,\underline{u}$, $T_{s}^{\,imp}\,\,\underline{u}$ et
$-\rho\,\tens{K}_{\,d}\,\,\underline{u}$ sont implicit�s est calcul�s �
l'instant $n+\theta$.\\
$\bullet$ Le terme de diffusion $\dive (\mu_{\,tot}\,\ggrad \underline{u})$ est
implicit� : la vitesse est prise � l'instant $n+\theta$ et la viscosit�
explicit�e ou extrapol�e.\\
$\bullet$ Enfin, le terme de convection en $\dive(\,\underline{u} \otimes
(\rho\underline{u})\,)$ est implicit� : la composante r�solue de la vitesse est
prise en $n+\theta$, et le flux de masse, explicit�, ou extrapol� en
$n+\theta_F$.

Par souci de clart�, on suppose, en l'absence d'indication, que les propri�tes
physiques $\Phi$ ($\rho,\,\mu_{tot},\,...$) et le flux de masse
$(\rho\underline{u})$ sont pris respectivement aux instants $n+\theta_\Phi$ et
$n+\theta_F$, o� $\theta_\Phi$ et $\theta_F$ d�pendent des sch�mas en temps
sp�cifiquement utilis�s pour ces grandeurs\footnote{cf. \fort{introd}}.

La discr�tisation temporelle de l'�quation (\ref{Base_Preduv_eqqdm}) s'�crit alors comme suit :

\begin{equation}\label{Base_Preduv_eq_di1}
 \begin{array}{c}
\displaystyle \rho\,\ \frac{ \underline {\widetilde{u}}^{n+1} -\underline {u}^{n} }
{\Delta t} + \dive(\,\underline{\widetilde{u}}^{n+\theta} \otimes (\rho\underline{u})\,) -\dive
(\mu_{\,tot}\,\ggrad \underline{\widetilde{u}}^{n+\theta}) =
\\
\displaystyle
 - \grad p^{n-1+\theta} + \dive (\rho\,\underline {u})\,\underline{\widetilde{u}}^{n+\theta} +(\Gamma\,\underline{u}_{\,i})^{n+\theta_S}-\Gamma^n\,\,\underline{\widetilde{u}}^{n+\theta}
\\
\begin{array}{c}
\displaystyle
- \rho\,\tens{K}_{\,d}^{n}\,\,\underline{\widetilde{u}}^{n+\theta} - (\rho\,\tens{K}_{\,e}\ \underline{u})^{n+\theta_S} + (\underline{T}_{s}^{\,exp})^{\,n+\theta_S} + T_{s}^{\,imp}\,\,\underline{\widetilde{u}}^{n+\theta}
\\
\displaystyle
+[\dive (\mu_{\,tot}\,^t\ggrad \underline {u})]^{n+\theta_S}-\frac {2} {3}[\,\grad (\mu_{\,tot}\,\dive \underline {u})]^{n+\theta_S} + (\rho -\rho_0) \underline {g}
 - (\underline{turb})^{n+\theta_S}
\end{array}
\end{array}
\end{equation}
o\`u, par souci de simplification, on a pos\'e :
\begin{equation}
\mu_{\,tot}=
\begin{cases}
\mu+\mu_t & \text{pour les mod�les � viscosit� turbulente ou en LES}, \\
\mu & \text{pour les mod�les au second ordre ou en laminaire}
\end{cases} \
\end{equation}
\\
et :
\begin{equation}
\underline{turb}^{n}=
\begin{cases}
\displaystyle\frac {2}{3}\grad (\rho^{n}\,k^{n}) & \text{pour les mod�les � viscosit� turbulente}, \\
\dive(\rho^{n}\,\tens{R}^n) & \text{pour les mod�les au second ordre},\\
0 & \text{en laminaire ou en LES}\\
\end{cases}
\end{equation}
Par analogie avec l'�criture du $\theta$-sch�ma pour une variable scalaire, $\,
\underline {\widetilde{u}}^{n+\theta}$ est interpol�e � partir de la vitesse
pr�dite $\underline {\widetilde{u}}^{n+1}$ de la mani\`ere suivante\footnote{si
$\theta=1/2$, ou qu'une extrapolation est utilis�e, l'ordre 2 n'est obtenu que si
le pas de temps $\Delta t$ est uniforme en temps et en espace.}~:
\begin{equation}
\underline {\widetilde{u}}^{n+\theta}=\theta\, \underline
{\widetilde{u}}^{n+1}+(1-\theta)\, \underline {u}^{n}\\
\end{equation}
Avec :
\begin{equation}
\left\{
\begin{array}{ll}
\theta = 1   & \text{Pour un sch\'ema de type Euler implicite d'ordre 1.}\\
\theta = 1/2 & \text{Pour un sch\'ema de type Cranck-Nicolson d'ordre 2.}\\
\end{array}
\right.
\end{equation}

L'�quation (\ref{Base_Preduv_eq_di1}) est alors r��crite sous la forme :

\begin{equation}\label{Base_Preduv_eq_di2}
\begin{array}{c}
\displaystyle \underbrace{\left(\frac{\rho}{\Delta t} -\theta \,\dive (\rho\,\underline {u}) +\theta \,\, \Gamma^n +
\theta \,\, \rho\,\tens{K}_{\,d}^n-\theta \,T_s^{\,imp} \right)}_{\displaystyle f_s^{imp}}\, (\underline {\,\widetilde{u}}^{n+1} -\underline {u}^{n})
\\
 +\, \theta\, \dive(\underline {\widetilde{u}}^{n+1} \otimes (\rho\underline{u}))-\, \theta\,\dive (\mu_{\,tot}\,\ggrad \underline {\widetilde{u}}^{n+1}) =
\\
-\,(1-\theta)\, \dive(\underline {u}^{n} \otimes (\rho\underline{u})) +\,(1-\theta)\,\dive (\mu_{\,tot}\,\ggrad \underline {u}^{n})
\\
f_s^{exp}\left\{
\begin{array}{c}
\displaystyle
- \grad p^{n-1+\theta} + \dive (\rho\,\underline {u})\,\underline{u}^{n} +\,(\,\Gamma^{n}\,\underline{u}_{\,i}\,)^{n+\theta_S}- \Gamma^n\,\,\underline{u}^{n}
\\
\displaystyle
-(\,\rho\,\tens{K}_{\,e}\ \underline{u}\,)^{n+\theta_S} -\rho\,\tens{K}_{\,d}^n\ \underline{u}^{n}+ (\underline{T}_{s}^{\,exp})^{\,n+\theta_S} + T_s^{\,imp}\,\,\underline {u}^{n}
\\
\displaystyle
+[\dive (\mu_{\,tot}\,^t\ggrad \underline {u}\,)]^{n+\theta_S}-\frac {2} {3}[\,\grad (\mu_{\,tot}\,\dive \underline {u}\,)]^{n+\theta_S} + (\rho -\rho_0) \underline {g}-(\underline{turb})^{n+\theta_S}
\end{array}
\right.
\end{array}
\end{equation}

d'o� l'�quation r�solue par le sous-programme \fort{codits} :
\begin{equation}\begin{array}{c}
\displaystyle
f_s^{\,imp}(\underline {\widetilde{u}}^{n+1}-\underline {u}^{n}) + \theta\, \dive(\underline{\widetilde{u}}^{n+1} \otimes (\rho
\underline{u})) - \theta\,\dive (\,\mu_{\,tot}\,\ggrad \underline{\widetilde{u}}^{n+1}) =
\\\\
\displaystyle
-(1-\theta)\,\dive(\underline{u}^{n} \otimes (\rho \underline{u}))+(1-\theta)\,\dive (\,\mu_{\,tot}\,\ggrad \underline{u}^{n})
+ \underline{f}_{\,s}^{\,exp}
\end{array}
\end{equation}
La m\'ethode de discr\'etisation spatiale est d\'evelopp\'ee dans le sous-programme \fort{codits}.\\



\minititre{Remarques :}
{\tiny$\blacksquare$} Dans le cas standard sans extrapolation, le terme
$-\,T_s^{\,imp}$ n'est ajout� � $f_s^{\,imp}$ que s'il est positif afin de ne
pas affaiblir la dominance de la diagonale de la matrice � inverser.\\
{\tiny$\blacksquare$} Si une extrapolation est utilis�e, par contre,
$\,T_s^{\,imp}$ est ajout� � $f_s^{\,imp}$ quel que soit son signe. En effet, l'id�e intuitive qui
consiste � prendre~:
\begin{equation}
\begin{cases}
\displaystyle
(\underline{T}_{s}^{\,exp} + T_{s}^{\,imp}\,\underline {u})^{\,n+\theta_S} &
\text{si } T_{s}^{\,imp} > 0\\
\displaystyle
(\underline{T}_{s}^{\,exp})^{\,n+\theta_S} + T_{s}^{\,imp}\,\underline{u}^{n+\theta} &\text{sinon}\\
\end{cases}
\end{equation}
aboutit � une incoh�rence dans le traitement si $T_s^{imp}$ change de signe
entre deux pas de temps.\\
{\tiny$\blacksquare$} la partie diagonale $\tens{K}_{\,d}$ du terme
de perte de charge est utilis�e dans $f_s^{\,imp}$. En fait, pour \^etre rigoureux,
il faudrait ne retenir que les contributions positives (point signal\'e dans le
sous-programme utilisateur associ\'e \fort{uskpdc}). Cette prise en compte sera \`a am\'eliorer.\\
{\tiny$\blacksquare$} Le terme $\theta\,\Gamma^{n}-\theta\,\dive
(\rho\,\underline {u})$ ne pose pas de probl�me pour la
dominance de la diagonale de la matrice car il est exactement compens� par le
terme de convection (cf. \fort{covofi}).

%%%%%%%%%%%%%%%%%%%%%%%%%%%%%%%%%%
%%%%%%%%%%%%%%%%%%%%%%%%%%%%%%%%%%
\section{Mise en \oe uvre}
%%%%%%%%%%%%%%%%%%%%%%%%%%%%%%%%%%
%%%%%%%%%%%%%%%%%%%%%%%%%%%%%%%%%%

L'�quation de conservation de la quantit� de mouvement est donc r�solue de fa�on
d�coupl�e. Ainsi, l'int�gration des diff�rents termes a �t� effectu�e afin de
traiter s�par�ment l'�quation obtenue pour chaque composante de la vitesse.\\
Dans le sous-programme \fort{preduv}, on calcule pour chaque
composante le second membre $f_s^{exp}$ du syst�me  (\ref{Base_Preduv_eq_di2}), les termes implicites du syst�me (� l'exception des termes de convection-diffusion), et le terme de viscosit� totale aux faces internes\footnote{valeur n�cessaire pour l'int�gration du terme de diffusion dans \fort{codits}, $\displaystyle(\mu_{\,tot})_{ij}\frac{\var{SURFN}}{\var{DIST}}$} et de
bord. Ces termes sont alors transmis au sous-programme \fort{codits} qui construit et
r�sout le syst�me complet obtenu pour chaque composante de la vitesse avec les termes de convection-diffusion.
\\\\
Le r\'esidu de normalisation pour la r\'esolution du syst\`eme en pression (\fort{resolp}) est calcul\'e dans
\fort{preduv}. Il est d\'efini par la norme de la
grandeur  $$\dive(\rho\,\widetilde{\vect{u}}^{n+1}+\Delta t \grad{P^{n-1+\theta}})-\Gamma$$
int\'egr\'ee sur chaque cellule \var{IEL} du maillage ($\Omega_{iel}$) soit, symboliquement, par la racine carr\'ee de la
somme sur les cellules du maillage de la quantit\'e
\begin{center}
\var{XNORMP(IEL)}=
$\int\limits_{\Omega_{iel}}[{\dive(\rho\,\widetilde{\vect{u}}^{n+1}+\Delta\,t\,\grad{P^{n-1+\theta}})
-\Gamma\,]\,d\Omega}$.
\end{center}

Il repr\'esente le second membre du syst\`eme qui porterait sur la pression si
le gradient de pression n'\'etait pas pris en compte lors de l'\'etape de pr\'ediction des vitesses. On note que si l'on utilisait directement le second membre de
l'\'equation portant sur l'incr\'ement de pression, on obtiendrait, pour un calcul
stationnaire men\'e \`a convergence, un r\'esidu de normalisation tendant vers z\'ero, ce qui serait p\'enalisant et
peu utile.

Au d\'ebut de \fort{preduv}, on ne dispose pas encore de $\widetilde{\vect{u}}^{n+1}$
et il n'est donc pas possible de calculer le r\'esidu de normalisation en
totalit\'e. Cependant, le calcul du r\'esidu complet \`a la fin de \fort{preduv}
n'est pas souhaitable non plus, car on devrait alors  monopoliser un tableau de
travail pour conserver le gradient de pression tout au long de \fort{preduv}.
Le calcul du r\'esidu de normalisation est donc r\'ealis\'e en deux fois.

La quantit\'e $\int\limits_{\Omega_{iel}}{\dive(\Delta\,t\,\grad{P^{n-1+\theta}}) -\Gamma d\Omega}$ est calcul\'ee au d\'ebut de \fort{preduv} et on y ajoute le compl\'ement $\int\limits_{\Omega_{iel}}{\dive(\rho\,\widetilde{\vect{u}}^{n+1})d\Omega}$ \`a la fin de \fort{preduv}.

On calcule donc tout d'abord le gradient de pression aux cellules \`a l'instant
$n-1+\theta$ par un appel \`a \fort{grdcel}. On utilise alors \fort{inimas} pour
\'evaluer $\Delta\,t\,S\,\grad{P^{n-1+\theta}}\cdot\vect{n}$ aux faces (de surface $S$ et de normale $\vect{n}$). Pour cela, en entr\'ee de \fort{inimas} , le tableau de travail \var{TRAV} contient $\frac{\Delta\,t}{\rho}\,\grad{P^{n-1+\theta}}$~; en sortie, les tableaux \var{VISCF} et \var{VISCB} contiennent la valeur de $\Delta\,t\,S\,\grad{P^{n-1+\theta}}\cdot\vect{n}$ aux faces internes et de bord respectivement.

On utilise ensuite \fort{divmas} qui place alors dans \var{XNORMP} la valeur de $\int\limits_{\Omega_{iel}}{\dive(\Delta\,t\,\grad{P^{n-1+\theta}}) d\Omega}$ aux cellules
\`a partir des tableaux \var{VISCF} et \var{VISCB}.

On ajoute enfin \`a \var{XNORMP} la contribution $\int\limits_{\Omega_{iel}}{ -\Gamma^{n} d\Omega}$ du terme source de masse.

On applique pour $\rho\,\widetilde{\vect{u}} + \Delta t \grad{P}$ les conditions
aux limites de la vitesse. Les conditions aux limites utilis�es pour le gradient
de pression (ou
plut\^ot pour $\frac{\Delta\,t}{\rho}\,\grad{P^{n-1+\theta}}$) pour le calcul de
$\int\limits_{\Omega_{iel}}{\dive(\Delta\,t\,\grad{P^{n-1+\theta}}) d\Omega}$
sont donc les conditions aux limites de la vitesse homog\'en\'eis\'ees~: ainsi,
on suppose
que dans la direction normale aux entr\'ees et aux parois, le gradient de
pression (ou plut\^ot $\frac{\Delta\,t}{\rho}\,\grad{P^{n-1+\theta}}$) est nul
et que dans la direction normale aux sym\'etries et aux sorties, il reste
inchang\'e.

De plus, pour gagner du temps calcul lors du passage par \fort{inimas},
on se contente, sur les maillages non orthogonaux,
d'une \'evaluation des valeurs aux faces \`a l'ordre 1 en espace (pas de
reconstruction~: \var{NSWRP=1}). En effet, on cherche \`a \'evaluer un simple
r\'esidu de normalisation global~: la pr\'ecision locale n'a donc pas
d'int\'er\^et.

Le calcul du r\'esidu sera compl\'et\'e \`a la fin de \fort{preduv}.


\etape{Calcul en partie du r\'esidu de normalisation pour l'\'etape de pression}

Dans cette premi�re �tape on calcule dans le tableau \var{XNORMP(NCELET)} la
grandeur $$\dive(\Delta t \,\grad{P^{n-1+\theta}})-\Gamma$$ int\'egr\'ee sur chaque cellule \var{IEL} du maillage ($\Omega_{iel}$)
soit, symboliquement,
$$\var{XNORMP(IEL)}=\int\limits_{\Omega_{iel}}{\dive(\Delta\,t\,\grad{P^{n-1+\theta}})-\Gamma
d\Omega}$$
On r\'ealise cette op\'eration en utilisant successivement \fort{inimas}
(calcul aux faces dans \var{VISCF} et \var{VISCB} de $\Delta t\,\grad{P^{n-1+\theta}}$ \`a
partir du tableau de travail \var{TRAV}=$\frac{\Delta t}{\rho}
\grad{P^{n-1+\theta}}$, assorti des conditions  aux limites de vitesse
homog\`enes et sans reconstruction) et \fort{divmas} (calcul dans \var{XNORMP}
de l'int\'egrale sur les cellules).  Par une simple boucle, on
ajoute ensuite la contribution du terme source de masse $\Gamma$.
Ce calcul est compl�t� \`a la fin de \fort{preduv}.\\

\etape{Calcul en partie du terme $\underline{f}_s^{\,exp}$}

Pour repr�senter le second membre correspondant � chaque composante de la
vitesse, on utilise les tableaux \var{TRAV}(\var{IEL},\var{DIR}),
\var{TRAVA}(\var{IEL},\var{DIR}) et \var{PROPCE},
o� \var{IEL} est le num�ro de la cellule et \var{DIR} la direction (x, y,
z). Quatre cas sont � consid�rer suivant que les termes sources sont extrapol�s
en $n+\theta_S$, ou que l'on it�re par un point fixe sur le syst�me en
vitesse-pression (\var{NTERUP}$>1$).
\\
$\bullet$ Si on extrapole les termes sources et que l'on it�re sur \fort{navsto}\\
\begin{itemize}
\item [-]\var{TRAV} re�oit les termes sources qui sont recalcul�s au cours de
toutes les it�rations sur \fort{navsto} et qui ne sont pas extrapol�s
($\grad{P^{n-1+\theta}}$ et $(\rho -\rho_0) \underline {g}$\footnote{en r�alit�
$(\rho -\rho_0) \underline {g}$ ne change pas, mais il est rapide � calculer ce
qui �vite d'avoir un traitement suppl�mentaire pour ce terme.}).\\
\item [-]\var{TRAVA} re�oit les termes sources qui ne changent pas au cours des
it�rations sur \fort{navsto} et qui ne sont pas extrapol�s
($T_s^{imp}\,u^n\,,-\rho\,\tens{K}_{d}\,u^n\,,\,-\Gamma^n\,u^n,...$).\\
\item [-]\var{PROPCE} re�oit les termes sources devant \^etre extrapol�s.\\
\\
\end{itemize}
$\bullet$ Sans it�ration sur \fort{navsto}, \var{TRAVA} est inutile et son contenu est directement stock� dans \var{TRAV}.\\
\\
$\bullet$ Sans extrapolation des termes sources, \var{PROPCE} est inutile et son contenu est directement stock� dans \var{TRAVA} (ou dans \var{TRAV} si \var{TRAVA} est inutile).\\
Ainsi, sans extrapolation des termes sources, et sans it�ration sur \fort{navsto}, tout les termes sources vont directement dans \var{TRAV}.\\
\\
\begin{itemize}
\item On dispose d\'ej\`a du gradient de pression sur les cellules \`a l'instant $n-1+\theta$. Le terme de gravit\'e est alors ajout� au vecteur \var{TRAV} qui contient d�j� le gradient de pression. Ainsi, on a par exemple pour la direction x :
\begin{equation}
\var{TRAV}(\var{IEL},1) = |\Omega_{IEL}| (-\displaystyle (\frac {\partial p}
{\partial x})_{\var{IEL}}+(\rho(\var{IEL})-\rho_0)g_x)
\end{equation}

\item Si une extrapolation des termes sources est utilis�e, le vecteur
\var{TRAV} (ou \var{TRAVA}) re�oit � la premi�re it�ration sur \fort{navsto},
$-\theta_S$ fois la contribution au temps $n-1$ des termes sources devant \^etre
extrapol�s\footnote{car
$(\underline{T}_s^{exp})^{n+\theta_S}=(1+\theta_S)\,(\underline{T}_s^{exp})^n
-\,\theta_S\, (\underline{T}_s^{exp})^{n-1}$} (stock�e dans
\var{PROPCE}). \var{PROPCE} est ensuite r�initialis� � z�ro de fa�on � pouvoir
recevoir plus tard la contribution au pas de temps courant des termes sources qui
sont extrapol�s.

\item Le terme correspondant au mod�le de turbulence n'est calcul� que lors de la premiere it�ration sur \fort{navsto} puis ajout� � \var{TRAVA}, \var{TRAV} ou \var{PROPCE} suivant que les termes sources sont extrapol�s, ou que l'on it�re sur \fort{navsto}.\\

{\tiny$\blacksquare$} Mod�les � viscosit� turbulente :\\
Si $\var{IGRHOK}=1$, alors on calcule $-\displaystyle \frac{2}{3}\ \rho\ \grad k$ (et non, comme on devrait,
$-\displaystyle \frac{2}{3}\grad (\rho k)$) par  simplification
(cf. paragraphe~\ref{Base_Preduv_section4}). Le gradient de $k$ est calcul� sur la cellule
par le
sous-programme \fort{grdcel}. \\
Si $\var{IGRHOK}=0$, ce terme est suppos\'e \^etre implicitement pris en compte
dans la pression.\\
{\tiny$\blacksquare$} Mod�les  au second ordre :\\
Le calcul du terme $-\dive(\rho \tens{R})$ s'effectue en deux temps. Tout
d'abord, on appelle le sous-programme \fort{divrij} qui projette le vecteur
$\tens{R}.\underline{e}_{\var{DIR}}$ aux faces, pour la direction \var{DIR}. Puis, on
appelle le sous-programme \fort{divmas} qui en calcule la divergence.\\
\linebreak
\item Les termes de viscosit� secondaire $- \displaystyle \frac {2} {3}\grad (\mu_{\,tot} \dive \underline\,{u})$ et de gradient
transpos� $ \dive (\mu_{\,tot} \,^t\ggrad \underline {u})$ sont calcul�s (s'ils sont pris en compte \emph{i.e.} \var{IVISSE}\,(\var{IPHAS})\ = 1, o� \var{IPHAS} est le
num�ro de la phase trait�e) par le sous-programme \fort{vissec}. Il ne sont calcul�s qu'� la premi�re it�ration sur \fort{navsto}. Au cours de cette �tape, le tableau \var{TRAV} est utilis� comme tableau de travail lors de l'appel au sous-programme \fort{vissec}. Il retrouve sa valeur � la fin de cet appel, son contenu �tant temporairement stock� dans les vecteurs \var{W7} � \var{W9}.
\\
\item Les termes correspondant aux pertes de charges ($\rho \tens{K}_{\,pdc}
{u}$), s'ils existent (\ $\var{NCEPDP}> 0$\ ), sont calcul�s par le
sous-programme \fort{tsepdc} � la premi�re it�ration sur \fort{navsto}. Ils sont
d�compos�s en deux parties :\\
{\tiny$\blacksquare$} Une premi�re, correspondant � la contribution des termes
diagonaux ($-\rho\,\tens{K}_{\,d}\underline{u}$) qui n'est pas extrapol�e.\\
{\tiny$\blacksquare$} Une seconde, correspondant aux  termes extradiagonaux
($-\rho\,\tens{K}_{\,e}\underline{u}$) qui peut l'\^etre ou non.\\
Au cours de cette �tape, le tableau \var{TRAV} est utilis� comme tableau de
travail lors de l'appel au sous-programme \fort{tsepdc}. Il retrouve sa valeur �
la fin de cet appel, son contenu �tant temporairement stock� dans les vecteurs
\var{W7} � \var{W9}.

\end{itemize}

\etape{Calcul du terme de viscosit� aux faces
$\displaystyle(\mu_{\,tot})_{ij}\frac{\var{SURFN}}{\var{DIST}}$}
Le calcul du terme de viscosit� totale aux faces est effectu� par le
sous-programme \fort{viscfa} et stock� dans les tableaux \var{VISCF} et
\var{VISCB} pour les faces internes et faces de bord respectivement.\\
Lors de l'int�gration des termes de convection-diffusion dans le sous-programme
\fort{codits}, on distingue les termes non reconstruits, int\'egr\'es dans la
matrice $\tens{EM}$ , de l'ensemble des termes (non reconstruits +
gradients de reconstruction) associ\'es \`a l'op\'erateur $\mathcal{E}$ (non
lin\'eaire)\footnote{ par coh\'erence avec les op\'erateurs $\mathcal{EM}$ et $\mathcal{E}$ d\'efinis dans \fort{navsto} }.
De la m�me mani�re, on distingue la viscosit� totale aux faces utilis�e dans
$\mathcal{E}$, tableaux \var{VISCF} et \var{VISCB}, de la viscosit� totale aux
faces utilis�e dans $\tens{EM}$, tableaux \var{VISCFI} et
\var{VISCBI}.\\
Pour les mod�les � viscosit� turbulente et en LES, ces deux tableaux sont identiques et contiennent $\mu_t+\mu$.
Pour les mod�les au second ordre, ils contiennent normalement $\mu$, mais pour des simples raisons de stabilit�
num�rique, on peut choisir de mettre $\mu_t+\mu$ dans la matrice (\textit{i.e.} dans $\tens{EM}$) en
conservant $\mu$ au second menbre (\textit{i.e.} dans $\mathcal{E}$). De par la r�solution par incr�ments, cette
 manipulation ne change pas le r�sultat. Cette option est activ�e par l'indicateur $\var{IRIJNU}\ =\ 1$\\
Si la vitesse n'est pas diffus�e (\ \var{IDIFF}(\var{IUIPH})\ $<$\ 1), alors les termes \var{VISCF} et \var{VISCB} sont mis � z�ro.
\linebreak

\etape{Calcul du second membre complet, de $f_s^{\,imp}$ et r�solution de l'�quation}
Les �quations d'�volution des composantes de la quantit� de mouvement sont
r�solues de fa�on d�coupl�e. On utilise, par cons�quent, un seul tableau
\var{ROVSDT} pour repr�senter la partie diagonale de la matrice obtenue pour chaque composante de la vitesse.\\
Pour chaque composante de la vitesse :\\

\begin{itemize}
\item Lors de la premi�re it�ration sur le sous-programme \fort{navsto}, les parties implicites et explicites des termes sources utilisateurs sont calcul�es par appel au sous-programme \fort{ustsns}.\\
{\tiny$\blacksquare$} La partie implicite ($T_s^{imp}$) est conserv�e dans le vecteur \var{XIMPA} pour les it�rations suivantes en cas d'utilisation du point-fixe sur le syst�me en vitesse-pression, et la contribution issue des m\^emes termes implicites ($T_s^{imp}\,\underline{u}^n$) ajout�e � \var{TRAVA} ou � \var{TRAV}.\\
{\tiny$\blacksquare$} La partie explicite ($T_s^{exp}$) est ajout�e � \var{TRAVA}, \var{TRAV} ou \var{PROPCE} suivant que les termes sources sont extrapol�s, ou que l'on it�re sur \fort{navsto}.\\

\item Le terme d'accumulation de masse ($\dive(\rho \underline {u})$) est
calcul� en appelant le sous-programme \fort{divmas} avec en argument le flux de
masse. Lors de la premi�re it�ration faite sur le sous-programme \fort{navsto},
le terme correspondant � la contribution explicite de l'accumulation de masse
($\underline{u}^{n}\ \dive(\rho \underline{u})$) est ajout� \`a \var{TRAVA} ou �
\var{TRAV}. Le vecteur \var{ROVSDT} est initialis� par $\theta\,\dive(\rho
\underline{u})$ (par coh�rence avec ce qui est fait dans le sous-programme
\fort{bilsc2}) puis la contribution du terme instationnaire ($\displaystyle
\frac{\rho}{\Delta t}$) ajout�e � ce dernier.\\

\item Le vecteur \var{ROVSDT} est ensuite compl�t� avec la contribution des termes sources implicites utilisateur (stock�e dans \var{XIMPA}) et avec celle des perte de charge ($\rho\,\tens{K}_{\,d}$) si $\var{NCEPDP}>0$.\\
{\tiny$\blacksquare$} Dans le cas ou les termes sources ne sont pas extrapol�s, la partie implicite des termes sources utilisateur n'est ajout�e � \var{ROVSDT} que si elle est n�gative de fa\c con � ne pas affaiblir la diagonale du syt�me.\\
{\tiny$\blacksquare$} Dans le cas ou ils sont extrapol�s par contre, elle est prise en compte quel que soit son signe.\\

\item Les termes sources implicite et explicite de masse, s'ils existent
(~$\var{NCESMP}>0$~), sont calcul�s � la premi�re it�ration sur \fort{navsto}
par le sous-programme \fort{catsma}. $\Gamma\,\underline{u}_i$ est ajout� �
\var{TRAV}, \var{TRAVA} ou \var{PROPCE} pour �tre �ventuellement
extrapol�. $\Gamma\,\underline{u}^n$ est rajout� � \var{TRAV} ou \var{TRAVA} et
$-\Gamma$ � \var{ROVSDT}.
\\
\item Le second membre est enfin assembl� en tenant compte de toutes les
contributions stock�es dans les tableaux \var{PROPCE}, \var{TRAVA} et
\var{TRAV}.\\
{\tiny$\blacksquare$} Si les termes sources sont extrapol�s alors :
$$\var{SMBR}=(1-\theta_S)\,\var{PROPCE}+\var{TRAVA}+\var{TRAV}$$
{\tiny$\blacksquare$} Sinon on a directement :
$$\var{SMBR}=\var{TRAVA}+\var{TRAV}$$

\item Prise en compte des physiques particuli�res (lagrangien, arc �lectrique,
...) ajout�s directement � \var{SMBR}.
\\
\item La resolution du syst�me lin�aire est faite par le sous-programme
\fort{codits} avec pour argument \var{ROVSDT} et \var{SMBR}.\\

\item Si on utilise le couplage instationnaire renforc� vitesse-pression ($\ \var{IPUCOU} = 1\ $) (uniquement disponible avec l'ordre 1, sans extrapolation des termes sources et sans it�ration sur \fort{navsto}) on r�sout, en utilisant pour \fort{codits} :
\begin{equation}\label{Base_Preduv_Eq_Tdir}
\tens{EM}_{\,\var{DIR}}\,.\, (\tens{RHO}^{\,n})^{-1}\,.\,\underline{T}_{\,\var{DIR}} =
\tens{\Omega}\,.\,\vect{1}
\end{equation}
avec $\tens{RHO}^n$ le tenseur diagonal d'\'el\'ement $\rho^{n}_{IEL}$,
$\tens{\Omega}$ le tenseur diagonal d'\'el\'ement $|\Omega_{IEL}|$, $\vect{1}$ le
vecteur de composantes toutes \'egales \`a 1.\\
 L'inversion du syst�me par \fort{codits} fournit
$(\tens{RHO}^{\,n})^{-1}\,.\,\underline{T}_{\,\var{DIR}}$, qui est ensuite multipli� par $\tens{RHO}^{\,n}$
pour obtenir $\underline{T}_{\,\var{DIR}}$.
Ceci est r\'ealis\'e pour chaque composante \var{DIR} de la vitesse. $\underline{T}_{\,\var{DIR}}$
est alors une approximation de type matrice diagonale de
$\tens{RHO}^{\,n}\,.\,\tens{EM}_{\,\var{DIR}}^{-1}$, avec
$\tens{EM}_{\,\var{DIR}}$ repr�sentant toujours la partie implicite de l'�quation de quantit� de mouvement
(\emph{i.e.} \var{ROVSDT} + contribution des
termes de convection-diffusion pris en compte dans le sous-programme
\fort{matrix}). $\underline{T}_{\,\var{DIR}}$
intervient dans l'�tape corrective (cf. sous-programme \fort{resolp}).\\
\end{itemize}
Fin de la boucle sur les composantes de la vitesse.\\

\etape{Fin du calcul du r\'esidu de normalisation pour l'\'etape de pression}

Comme indiqu\'e pr\'ec\'edemment, on peut maintenant compl\'eter le calcul du
r\'esidu de normalisation pour l'\'etape de pression de \fort{resolp}.

Le tableau \var{XNORMP} contient d\'ej\`a
$\int\limits_{\Omega_{iel}}{\dive(\Delta\,t\,\grad{P^{n-1+\theta}}) -\Gamma d\Omega}$ . On
lui ajoute donc
$\int\limits_{\Omega_{iel}}{\dive(\rho\,\widetilde{\vect{u^{n+1}}})d\Omega}$.

Pour
cela, on proc\`ede comme pr\'ec\'edemment pour le calcul de
$\int\limits_{\Omega_{iel}}{\dive(\Delta\,t\,\grad{P^{n-1+\theta}}) d\Omega}$. Un appel \`a
\fort{inimas} permet d'obtenir
$\rho\,S\,\widetilde{\vect{u}}^{n+1}\cdot\widetilde{\vect{n}}$ aux faces \`a partir de
$\widetilde{\vect{u}}^{n+1}$ connu aux cellules (tableau \var{RTP}). Les conditions
aux limites pour \fort{inimas} sont naturellement celles de la vitesse. Comme
pr\'ec\'edemment, on se contente pour gagner du temps calcul lors du passage par
\fort{inimas}, d'une \'evaluation des valeurs aux faces \`a l'ordre 1 en espace
sur les maillages non orthogonaux (pas de
reconstruction~: \var{NSWRP=1}). On utilise ensuite \fort{divmas} pour calculer
aux cellules la divergence
$\int\limits_{\Omega_{iel}}{\dive(\rho\,\widetilde{\vect{u}}^{n+1})d\Omega}$ et
l'ajouter directement \`a \var{XNORMP}.

Pour finir, le r\'esidu de normalisation est d\'etermin\'e et stock\'e dans \var{RNORMP(IIPHAS)} par un appel \`a \fort{prodsc}  (qui r\'ealise le calcul de la somme sur les cellules du
carr\'e des valeurs de \var{XNORMP} et en prend la racine carr\'ee).\\


\newpage

On r�sume dans les tableaux (\ref{Base_Preduv_tab_ext0}), (\ref{Base_Preduv_tab_ext1}), (\ref{Base_Preduv_tab_exp0})
et (\ref{Base_Preduv_tab_exp1}) les diff�rentes contributions (hors convection-diffusion)
affect�es � chacun des vecteurs \var{TRAV}, \var{TRAVA}, \var{PROPCE} et
\var{ROVSDT} � l'it�ration $n$. On diff�rencie pour chacun des sch�mas en temps
appliqu�s aux les termes sources, deux cas suivant qu'un point fixe sur le
syst�me en vitesse-pression est utilis� ou non (it�ration sur \fort{navsto} pour
\var{NTERUP}$>1$). En l'absence d'indication les propri�t�s physiques $\Phi$
($\rho$,$\mu$,etc...) sont suppos�es prises au temps $n+\theta_\Phi$, et le flux
de masse $(\,\rho \underline{u})$ pris au temps $n+\theta_F$, o� $\theta_\Phi$
et $\theta_F$ d�pendent des sch�mas en temps sp�cifiquement utilis�s pour ces
grandeurs (cf. \fort{introd}).
\\\\
Les termes figurant dans ces tableaux sont �crits tels qu'ils ont �t� implant�s
dans le code, d'o� l'origine de certaines diff�rences par rapport � l'�criture
adopt�e dans l'�quation (\ref{Base_Preduv_eq_di2}).
\\\\
Par souci de simplification, on pose~:
\begin{equation}\notag
\mu_{\,tot}=
\begin{cases}
\mu+\mu_t & \text{pour les mod�les � viscosit� turbulente ou en LES}, \\
\mu & \text{pour les mod�les au second ordre ou en laminaire}\\
\end{cases} \
\end{equation}

\minititre{Avec extrapolation des termes sources}
\begin{equation}\notag
\underline{turb}^{n}=
\begin{cases}
\displaystyle\frac {2}{3}\,\rho^{n}\,\grad (\,k^{n}) &
\text{pour les mod�les � viscosit� turbulente}, \\
\dive(\rho^{n}\,\tens{R}^n) & \text{pour les mod�les au second ordre},\\
0 & \text{en laminaire ou en LE.}\\
\end{cases}
\end{equation}
$\bullet$ \var{NTERUP} $=$ 1 : $\var{SMBR}^n=(1-\theta_S)\,\var{PROPCE}^n+\var{TRAV}^n$
\begin{equation}\label{Base_Preduv_tab_ext0}
\begin{array}{|l|c|}
\hline
\var{ROVSDT}^{n}
& \displaystyle
\frac{\rho}{\Delta t} -\theta \,\dive (\rho\,\underline {u}) +\theta \,\, \Gamma^n + \theta \,\, \rho\,\tens{K}_{\,d}^n-\theta \,T_s^{\,imp} \\
\hline
\var{PROPCE}^{n}
& \displaystyle
\underline{T}_{s}^{\,exp,\,n}-\,\rho^{n}\,\tens{K}_{\,e}^{n}\ \underline{u}^{n} + \,\Gamma^{n}\,\underline{u}_{\,i}^{n}\\
& \displaystyle
-\underline{turb}^{n}+ \dive (\mu_{\,tot}^{n}\,^t\ggrad \underline {u}^{n}\,)+ \frac {2} {3}\,\grad (\mu_{\,tot}^{n}\,\dive \frac{(\rho \underline {u})}{\rho^{n}}\,)\\
\hline
\var{TRAV}^{n} & \displaystyle
- \grad p^{n-1+\theta} + (\rho -\rho_0) \underline {g} \\
& \displaystyle
-\theta_S\,\var{PROPCE}^{n-1} -\rho\,\tens{K}_{\,d}^n\ \underline{u}^{n} \\
& \displaystyle
+ T_s^{\,imp}\,\,\underline {u}^{n} + \dive (\rho\,\underline {u})\,\underline{u}^{n} - \Gamma^n\,\,\underline{u}^{n}\\
\hline
\end{array}
\end{equation}
\\\\
$\bullet$ \var{NTERUP} $>$ 1 (sous-it�ration $k$) : $\var{SMBR}^n=(1-\theta_S)\,\var{PROPCE}^n+\var{TRAVA}^n+\var{TRAV}^n$
\begin{equation}\label{Base_Preduv_tab_ext1}
\begin{array}{|l|c|}
\hline
\var{ROVSDT}^{n}
& \displaystyle
\frac{\rho}{\Delta t} -\theta \,\dive (\rho\,\underline {u}) +\theta \,\, \Gamma^n + \theta \,\, \rho\,\tens{K}_{\,d}^n-\theta \,T_s^{\,imp} \\
\hline
\var{PROPCE}^{n}
& \displaystyle
\underline{T}_{s}^{\,exp,\,n}-\,\rho^{n}\,\tens{K}_{\,e}^{n}\ \underline{u}^{n} + \,\Gamma^{n}\,\underline{u}_{\,i}^{n}\\
& \displaystyle
-\underline{turb}^{n}+ \dive (\mu_{\,tot}^{n}\,^t\ggrad \underline {u}^{n}\,)+ \frac {2} {3}\,\grad (\mu_{\,tot}^{n}\,\dive \frac{(\rho \underline {u})}{\rho^{n}}\,)\\
\hline
\var{TRAVA}^{n} &
\displaystyle
-\theta_S\,\var{PROPCE}^{n-1} -\rho\,\tens{K}_{\,d}^n\ \underline{u}^{n} + T_s^{\,imp}\,\,\underline {u}^{n} + \dive (\rho\,\underline {u})\,\underline{u}^{n} - \Gamma^n\,\,\underline{u}^{n}\\
\hline
\var{TRAV}^{n}
& \displaystyle
- \grad (p^{n+\theta})^{(k-1)} + (\rho -\rho_0) \underline {g} \\
\hline
\end{array}
\end{equation}

\minititre{Sans extrapolation des termes sources}

\begin{equation}\notag
\underline{turb}^{n}=
\begin{cases}
\displaystyle\frac {2}{3}\,\rho\,\grad (\,k^{n}) &
\text{pour les mod�les � viscosit� turbulente}, \\
\dive(\rho\,\tens{R}^n) & \text{pour les mod�les au second ordre},\\
0 & \text{en laminaire ou en LES.}\\
\end{cases}
\end{equation}
$\bullet$ \var{NTERUP} $=$ 1 : $\var{SMBR}^n=\var{TRAV}^n$
\\\\
\begin{equation}\label{Base_Preduv_tab_exp0}
\begin{array}{|l|c|}
\hline
\var{ROVSDT}^{n} &
\displaystyle \frac{\rho}{\Delta t} -\theta \,\dive (\rho\,\underline {u}) +\, \Gamma^n + \, \rho\,\tens{K}_{\,d}^n+Max(-\,T_s^{\,imp},0)\\
\hline
\var{TRAV}^{n}
& \displaystyle
- \grad p^{n-1+\theta} + (\rho -\rho_0) \underline {g} \\
& \displaystyle
+ \underline{T}_{s}^{\,exp}-\,\rho\,\tens{K}_{\,e}^{n}\ \underline{u}^{n} + \,\Gamma^{n}\,\underline{u}_{\,i}^{n}\\
& \displaystyle
-\underline{turb}^{n}+ \dive (\mu_{\,tot}\,^t\ggrad \underline {u}^{n}\,)+ \frac {2} {3}\,\grad (\mu_{\,tot}\,\dive \frac{(\rho \underline {u})}{\rho}\,)\\
& \displaystyle
-\rho\,\tens{K}_{\,d}^n\ \underline{u}^{n}+ T_s^{\,imp}\,\,\underline {u}^{n} + \dive (\rho\,\underline {u})\,\underline{u}^{n} - \,\Gamma^{n}\,\underline{u}^{n}\\
\hline
\end{array}
\end{equation}
\\\\
$\bullet$ \var{NTERUP} $>$ 1 (sous-it�ration $k$) : $\var{SMBR}^n=\var{TRAVA}^n+\var{TRAV}^n$
\begin{equation}\label{Base_Preduv_tab_exp1}
\begin{array}{|l|c|}
\hline
\var{ROVSDT}^{n} &
\displaystyle \frac{\rho}{\Delta t} -\theta \,\dive (\rho\,\underline {u}) +\, \Gamma^n + \, \rho\,\tens{K}_{\,d}^n+Max(-\,T_s^{\,imp},0)\\
\hline
\var{TRAVA}^{n} &
\displaystyle
\underline{T}_{s}^{\,exp}-\,\rho\,\tens{K}_{\,e}^{n}\ \underline{u}^{n} + \,\Gamma^{n}\,\underline{u}_{\,i}^{n}\\
& \displaystyle
-\underline{turb}^{n}+ \dive (\mu_{\,tot}\,^t\ggrad \underline {u}^{n}\,)+ \frac {2} {3}\,\grad (\mu_{\,tot}\,\dive \frac{(\rho \underline {u})}{\rho}\,)\\
& \displaystyle
-\rho\,\tens{K}_{\,d}^n\ \underline{u}^{n}+ T_s^{\,imp}\,\,\underline {u}^{n} + \dive (\rho\,\underline {u})\,\underline{u}^{n} - \Gamma^n\,\,\underline{u}^{n}\\
\hline
\var{TRAV}^{n} &
\displaystyle
- \grad (p^{n+\theta})^{(k-1)} + (\rho -\rho_0) \underline {g} \\
\hline
\end{array}
\end{equation}

%%%%%%%%%%%%%%%%%%%%%%%%%%%%%%%%%%
%%%%%%%%%%%%%%%%%%%%%%%%%%%%%%%%%%
\section{Points \`a traiter}\label{Base_Preduv_section4}
%%%%%%%%%%%%%%%%%%%%%%%%%%%%%%%%%%
%%%%%%%%%%%%%%%%%%%%%%%%%%%%%%%%%%
\etape{Prise en compte du terme $\grad(\rho k)$ pour les mod�les � viscosit� turbulente}
Pour les mod�les � viscosit� turbulente, on calcule $\rho\ \grad k$ au lieu de $\grad(\rho k)$. Cette
approximation, historique, provient du fait que les conditions aux limites de
$\rho k$ ne sont pas directement accessibles, contrairement \`a celles de
$k$.\\

\etape{Prise en compte de la diagonale de $\tens{K}_{\,pdc}$}
Actuellement, dans le sous-programme utilisateur \fort{uskpdc}, une mise en
garde explicite est \'ecrite, mais en commentaire.
% \footnote{Son contenu est : Veillez
%\`a ce que les coefficients diagonaux  (du tenseur de pertes de charge consid\'er\'e) soient positifs. Vous risquez un PLANTAGE
%si ce n'est pas le cas. AUCUN contr\^ole ult\'erieur ne sera effectu\'e.}.
 La partie diagonale $\tens{K}_{\,d}$ du tenseur de pertes de charge
$\tens{K}_{\,pdc}$ peut donc intervenir syst\'ematiquement dans le calcul du
coefficient $f_s^{\,imp}$, que sa contribution $K_{\,d}$ soit positive ou non, si
l'utilisateur n'y prend garde. Un test de positivit\'e  sur les \'el\'ements de
$\tens{K}_{\,d}$ assurant une prise en compte correcte (contribution renfor\c
cant r\'eellement la diagonale de la matrice globale) devrait \^etre implant\'e.

\etape{\'Ecriture de $\tens{EM}$}
Dans la r\'esolution proc\'edant par incr\'ements, il n'est pas indispensable
\`a convergence
que la viscosit\'e utilis\'ee pour l'\'ecriture de l'op\'erateur $\mathcal{E}$  soit la m\^eme que celle prise en
compte dans $\tens{EM}$, matrice du syst\`eme en incr\'ements. Ainsi, en $R_{ij}-\varepsilon$, la
viscosit\'e totale utilis\'ee dans $\tens{EM}$ contient la viscosit\'e mol\'eculaire mais aussi la
viscosit\'e turbulente si l'on choisit l'option \var{IRIJNU = 1 }, alors que
dans $\mathcal{E}$ intervient seule la viscosit\'e mol\'eculaire. Cet ajout de
la viscosit\'e turbulente qui n'a pas de raison d'appara\^\i tre en
$R_{ij}-\varepsilon$, a \'et\'e h\'erit\'e des pratiques mises en \oe uvre dans
ESTET et N3S-EF pour renforcer la stabilit\'e (lissage \'eventuel de l'incr\'ement). Mais, ce
n'est peut \^etre pas le seul effet produit. En outre, cette pratique n'a pas aujourd'hui montr\'e son absolue
n\'ecessit\'e dans \CS. Par cons\'equent, une \'etude approfondie serait int\'eressante.


\etape{R\'esidu de normalisation de l'\'etape de pression}
On pourra v\'erifier le calcul du r\'esidu de normalisation et en
particulier l'utilisation des conditions aux limites de vitesse.

\etape{Calcul des pertes de charges}
Avec extrapolation des termes sources on a :
\begin{equation}\notag
(\tens{K}_{\,e}\underline{u})^{n+\theta_S} + \tens{K}_{\,d}^{n}\ \underline{u}^{n+\theta}
\end{equation}
Il serait aussi envisageable d'utiliser~:
\begin{equation}\notag
(\tens{K}_{\,e}\underline{u})^{n+\theta_S} + \tens{K}_{\,d}^{n+\theta_S}\ \underline{u}^{n+\theta}
\end{equation}

%-------------------------------------------------------------------------------

% This file is part of Code_Saturne, a general-purpose CFD tool.
%
% Copyright (C) 1998-2011 EDF S.A.
%
% This program is free software; you can redistribute it and/or modify it under
% the terms of the GNU General Public License as published by the Free Software
% Foundation; either version 2 of the License, or (at your option) any later
% version.
%
% This program is distributed in the hope that it will be useful, but WITHOUT
% ANY WARRANTY; without even the implied warranty of MERCHANTABILITY or FITNESS
% FOR A PARTICULAR PURPOSE.  See the GNU General Public License for more
% details.
%
% You should have received a copy of the GNU General Public License along with
% this program; if not, write to the Free Software Foundation, Inc., 51 Franklin
% Street, Fifth Floor, Boston, MA 02110-1301, USA.

%-------------------------------------------------------------------------------

\programme{recvmc}

\vspace{1cm}
%%%%%%%%%%%%%%%%%%%%%%%%%%%%%%%%%%
%%%%%%%%%%%%%%%%%%%%%%%%%%%%%%%%%%
\section{Fonction}
%%%%%%%%%%%%%%%%%%%%%%%%%%%%%%%%%%
%%%%%%%%%%%%%%%%%%%%%%%%%%%%%%%%%%
Le but de ce sous-programme est de calculer la vitesse au centre des cellules
\`a partir du flux de masse aux faces, par moindres carr\'es. Utilis\'ee apr\`es
l'\'etape de correction de pression ({\it cf.}~\fort{navsto}) cette m\'ethode est une
alternative \`a la technique de reconstruction \`a partir du gradient de
l'incr\'ement de pression (technique standard).
Elle est activ\'ee quand l'indicateur \var{IREVMC} vaut~1 ou 2.

On rappelle que, � la fin de l'�tape de correction de pression, le flux de masse aux faces vaut :
\begin{equation}
(\rho \vect{u})^{n+1}_{\,ij}\text{.}\vect{S}_{\,ij} =
(\rho
\vect{\widetilde{u}})^{n+1}_{\,ij}.\,\vect{S}_{\,ij}
-\vect{D}_{\,ij}(\Delta t^n,\delta P^{n+\theta})
+\text{RC}_{\,ij}
\end{equation}
o� $\vect{\widetilde{u}}$ est la vitesse issue de l'�tape de pr�diction, $D_{\,ij}$ un op�rateur de gradient aux faces
et $\text{RC}_{\,ij}$ le terme d'Arakawa (cf. \fort{navsto} pour une d�finition pr�cise des notations).
Une premi�re m�thode, activ�e par \var{IREVMC} = 2, consiste � partir directement de
$(\rho \vect{u})^{n+1}_{\,ij}\text{.}\vect{S}_{\,ij}$ pour calculer $\vect{u}^{n+1}$ par moindres carr�s. Son utilisation a
montr\'e qu'elle semblait plus diffusive que la m\'ethode standard (par exemple, dans le cas de la cavit\'e entra\^\i n\'ee)
et pouvait conduire � des r�sultats erron�s sur des maillages ne comportant pas uniquement des t�tra�dres
(ou des prismes � base triangulaire en ``2D'') et des pav�s (hexa�dres orthogonaux).\\
On note que, dans la m�thode ci-dessus, on est parti d'une vitesse $\vect{\widetilde{u}}$ au centre des cellules, qu'on a projet�e aux faces pour obtenir le flux de masse, et qu'on ram�ne au centre des cellules par moindres carr�s. Fort de cette constatation, une m�thode alternative est disponible, activ�e par \var{IREVMC} = 1. Elle consiste � n'appliquer la m�thode des moindres carr�s qu'� la partie $-\vect{D}_{\,ij}(\Delta t^n,\delta P^{n+\theta}) +\text{RC}_{\,ij}$ du flux de masse et � rajouter directement
$\vect{\widetilde{u}}$ (connu au centre des cellules) au r�sultat obtenu\footnote{cette derni�re �tape est faite dans
\fort{navsto}.}. Cette m�thode donne des r�sultats sensiblement meilleurs.

%%%%%%%%%%%%%%%%%%%%%%%%%%%%%%%%%%
%%%%%%%%%%%%%%%%%%%%%%%%%%%%%%%%%%
\section{Discr\'etisation}
%%%%%%%%%%%%%%%%%%%%%%%%%%%%%%%%%%
%%%%%%%%%%%%%%%%%%%%%%%%%%%%%%%%%%
Soit une cellule $\Omega_i$, $\phi_{ij}$  le flux de masse (total ou uniquement la partie en
gradient de pression) \`a travers la face la
s\'eparant d'une cellule voisine $\Omega_j$ et $\phi_{\,b_ik}$ le flux de masse (total ou uniquement la partie en
gradient de pression)\`a travers la face de bord $\,b_{ik}$.
L'id\'eal serait de pouvoir trouver un vecteur $\vect{v}_i$ telle que, pour toute cellule voisine $\Omega_j$ on ait :
\begin{equation}
\rho_i\vect{v}_i.\vect{S}_{ij} = \phi_{ij}
\end{equation}
et l'\'equivalent aux faces de bords, {\it i.e.} :
\begin{equation}
\rho_i \vect{v}_i.\vect{S}_{\,b_{ik}} = \phi_{\,b_{ik}}
\end{equation}
Comme c'est g\'en\'eralement impossible d'obtenir les deux \'egalit\'es pr\'ec\'edentes\footnote{%
sauf en incompressible pour des triangles en 2D et des
t\'etra\`edres en 3D}, on va simplement chercher \`a minimiser la fonction $F_i$ :
\begin{equation}
F_i=\sum\limits_{j\in Vois(i)}\left[
\rho_i\vect{v}_i.\vect{S}_{ij}-\phi_{ij}\right]^2 + \sum\limits_{k\in {\gamma_b(i)}}\left[\rho_i\vect{v}_i.\vect{S}_{\,b_{ik}}-\phi_{\,b_{ik}}\right]^2
\end{equation}

Pour ce faire, on d\'erive $F_i$ par rapport aux trois composantes du vecteur $\vect{v}_i$,
et on r\'esout le syst\`eme $3\times3$ local qui r\'esulte :\\
\begin{equation}
\begin{array}{lll}
&\displaystyle \tens{\mathcal{S}}^{\,i} \,
\left[\begin{array}{c}
v_{i,x} \\ v_{i,y} \\ v_{i,z}
\end{array}\right]
&=\left[\begin{array}{c}
\displaystyle
\frac{1}{\rho_i}(\sum\limits_{j\in Vois(i)}\phi_{ij}S_{ij,x} +\sum\limits_{k\in {\gamma_b(i)}}\phi_{\,b_{ik}}S_{{\,b_{ik}},x})\\
\displaystyle
\frac{1}{\rho_i}(\sum\limits_{j\in Vois(i)}\phi_{ij}S_{ij,y} +\sum\limits_{k\in {\gamma_b(i)}}\phi_{\,b_{ik}}S_{{\,b_{ik}},y})\\
\displaystyle
\frac{1}{\rho_i}(\sum\limits_{j\in Vois(i)}\phi_{ij}S_{ij,z} +\sum\limits_{k\in {\gamma_b(i)}}\phi_{\,b_{ik}}S_{{\,b_{ik}},z})
\end{array}\right]
\end{array}
\end{equation}

avec $\tens{\mathcal{S}}^{\,i}$ matrice carr\'ee $3\times3$ d'\'el\'ement $S^{\,i}_{\,ml}$ courant d\'efini par :\\
\begin{equation}
S^{\,i}_{\,ml} = \sum\limits_{j\in Vois(i)}S_{ij,\,l}\,S_{ij,\,m} + \sum\limits_{k\in {\gamma_b(i)}}S_{{\,b_{ik}},\,l}\,S_{{\,b_{ik}},\,m}
\end{equation}

%\begin{equation}
%\left[\begin{array}{ccc}
%\displaystyle
%\sum\limits_jS_{ij,x}S_{ij,x} & \sum\limits_jS_{ij,x}S_{ij,y}
%& \sum\limits_jS_{ij,x}S_{ij,z}\\
%\displaystyle
%\sum\limits_jS_{ij,x}S_{ij,y} & \sum\limits_jS_{ij,y}S_{ij,y}
%& \sum\limits_jS_{ij,y}S_{ij,z}\\
%\displaystyle
%\sum\limits_jS_{ij,x}S_{ij,z} & \sum\limits_jS_{ij,y}S_{ij,z}
%& \sum\limits_jS_{ij,z}S_{ij,z}
%\end{array}\right]
%\left[\begin{array}{c}
%u_{i,x} \\ u_{i,y} \\ u_{i,z}
%\end{array}\right]
%=\left[\begin{array}{c}
%\displaystyle
%\frac{1}{\rho_i}\sum\limits_j\phi_{ij}S_{ij,x}\\
%\displaystyle
%\frac{1}{\rho_i}\sum\limits_j\phi_{ij}S_{ij,y}\\
%\displaystyle
%\frac{1}{\rho_i}\sum\limits_j\phi_{ij}S_{ij,z}
%\end{array}\right]
%\end{equation}

%%%%%%%%%%%%%%%%%%%%%%%%%%%%%%%%%%
%%%%%%%%%%%%%%%%%%%%%%%%%%%%%%%%%%
\section{Mise en \oe uvre}
%%%%%%%%%%%%%%%%%%%%%%%%%%%%%%%%%%
%%%%%%%%%%%%%%%%%%%%%%%%%%%%%%%%%%
Le flux de masse est pass\'e par les arguments \var{FLUMAS} et \var{FLUMAB}.

\etape{Calcul de la matrice}
Les \var{NCEL} matrices $3\times 3$ sont stock\'ees dans le tableau de travail
\var{COCG},
de dimension $NCELET\times 3\times 3$. Ce dernier est d'abord mis \`a z\'ero, puis
son remplissage se fait dans des boucles sur les faces internes et les faces de
bord. La matrice \'etant sym\'etrique, ces boucles ne
servent qu'\`a remplir la partie triangulaire sup\'erieure, le reste \'etant
rempli par sym\'etrie \`a la fin.

\etape{Inversion de la matrice}
On calcule les coefficients de la comatrice, puis l'inverse.
Pour des questions de vectorisation, la boucle sur les \var{NCEL} \'el\'ements
est remplac\'ee par une
s\'erie de boucles en vectorisation forc\'ee sur des blocs de \var{NBLOC=1024}
\'el\'ements. Le reliquat ($\var{NCEL}-E(\var{NCEL}/1024)\times 1024$) est
trait\'e apr\`es les boucles.
\`A la fin, la matrice inverse est stock\'ee dans \var{COCG}
(toujours en utilisant sa propri\'et\'e de sym\'etrie).

\etape{Calcul du second membre et r\'esolution}
Le second membre est stock\'e dans \var{BX}, \var{BY} et \var{BZ}. La vitesse
finale est stock\'ee dans \var{UX}, \var{UY} et \var{UZ}.


%%%%%%%%%%%%%%%%%%%%%%%%%%%%%%%%%%
%%%%%%%%%%%%%%%%%%%%%%%%%%%%%%%%%%
\section{Points \`a traiter}
%%%%%%%%%%%%%%%%%%%%%%%%%%%%%%%%%%
%%%%%%%%%%%%%%%%%%%%%%%%%%%%%%%%%%
\etape{Vectorisation forc\'ee}
Le d\'ecoupage en boucles de 1024 est-il vraiment n\'ecessaire ? Les machines
vectorielles et les compilateurs sont-ils aujourd'hui capables
d'effectuer la vectorisation sans cette aide ? On note cependant que ce
d\'ecoupage en boucles de 1024 n'a pas de co\^ut CPU suppl\'ementaire, et un
co\^ut m\'emoire n\'egligeable. Le seul inconv\'enient r\'eside dans la
complexit\'e de l'\'ecriture.

\etape{Suppression de la m�thode \var{IREVMC} = 2}
Sur un maillage ``1D'' d'hexa�dres tous orthogonaux sauf une face, on peut montrer que la m�thode fait appara�tre
une composante de vitesse aberrante non nulle et directement d�termin�e par l'angle de non orthogonalit� de la
face (non consistance). On pourrait donc songer � supprimer purement cette m�thode, dans la mesure o� elle n'est
{\em a priori} consistante que sur une classe r�duite de maillages.


\include{resolp}
%-------------------------------------------------------------------------------

% This file is part of Code_Saturne, a general-purpose CFD tool.
%
% Copyright (C) 1998-2020 EDF S.A.
%
% This program is free software; you can redistribute it and/or modify it under
% the terms of the GNU General Public License as published by the Free Software
% Foundation; either version 2 of the License, or (at your option) any later
% version.
%
% This program is distributed in the hope that it will be useful, but WITHOUT
% ANY WARRANTY; without even the implied warranty of MERCHANTABILITY or FITNESS
% FOR A PARTICULAR PURPOSE.  See the GNU General Public License for more
% details.
%
% You should have received a copy of the GNU General Public License along with
% this program; if not, write to the Free Software Foundation, Inc., 51 Franklin
% Street, Fifth Floor, Boston, MA 02110-1301, USA.

%-------------------------------------------------------------------------------

\programme{turbke}\label{ap:turbke}

\hypertarget{turbke}{}

\vspace{1cm}
%-------------------------------------------------------------------------------
\section*{Fonction}
%-------------------------------------------------------------------------------
Le but de ce sous-programme est de r\'esoudre le syst\`eme des \'equations de
$k$ et $\varepsilon$ de mani\`ere semi-coupl\'ee.\\
Le syst\`eme d'\'equations r\'esolu est le suivant :

\begin{equation}
\left\{\begin{array}{l}
\displaystyle
\rho\frac{\partial k}{\partial t} +
\dive\left[\rho \vect{u}\,k-(\mu+\frac{\mu_t}{\sigma_k})\grad{k}\right] =
\mathcal{P}+\mathcal{G}-\rho\varepsilon+k\dive(\rho\vect{u})
+\Gamma(k_i-k)\\
\multicolumn{1}{c}{+\alpha_k k +\beta_k}\\
\displaystyle
\rho\frac{\partial \varepsilon}{\partial t} +
\dive\left[\rho \vect{u}\,\varepsilon-
(\mu+\frac{\mu_t}{\sigma_\varepsilon})\grad{\varepsilon}\right] =
C_{\varepsilon_1}\frac{\varepsilon}{k}\left[\mathcal{P}
+(1-C_{\varepsilon_3})\mathcal{G}\right]
-\rho C_{\varepsilon_2}\frac{\varepsilon^2}{k}
+\varepsilon\dive(\rho\vect{u})\\
\multicolumn{1}{c}{+\Gamma(\varepsilon_i-\varepsilon)
+\alpha_\varepsilon \varepsilon +\beta_\varepsilon}
\end{array}\right.
\end{equation}

$\mathcal{P}$ est le terme de production par cisaillement moyen :
\begin{displaymath}
\begin{array}{rcl}
\mathcal{P} & = & \displaystyle -\rho R_{ij}\frac{\partial u_i}{\partial x_j}
= -\left[-\mu_t \left(\frac{\partial u_i}{\partial x_j} +
\frac{\partial u_j}{\partial x_i}\right)
+\frac{2}{3}\mu_t\frac{\partial u_k}{\partial x_k}\delta_{ij}
+\frac{2}{3}\rho k\delta_{ij}\right]
\frac{\partial u_i}{\partial x_j}\\
& = & \displaystyle \mu_t \left(\frac{\partial u_i}{\partial x_j} +
\frac{\partial u_j}{\partial x_i}\right)\frac{\partial u_i}{\partial x_j}
-\frac{2}{3}\mu_t(\dive\vect{u})^2-\frac{2}{3}\rho k \dive(\vect{u})\\
& = & \displaystyle \mu_t \left[
2\left(\frac{\partial u}{\partial x}\right)^2+
2\left(\frac{\partial v}{\partial y}\right)^2+
2\left(\frac{\partial w}{\partial z}\right)^2+
\left(\frac{\partial u}{\partial y}+\frac{\partial v}{\partial x}\right)^2+
\left(\frac{\partial u}{\partial z}+\frac{\partial w}{\partial x}\right)^2+
\left(\frac{\partial v}{\partial z}+\frac{\partial w}{\partial y}\right)^2
\right]\\
\multicolumn{3}{r}%
{\displaystyle -\frac{2}{3}\mu_t(\dive\vect{u})^2-\frac{2}{3}\rho k \dive(\vect{u})}
\end{array}
\end{displaymath}

$\mathcal{G}$ est le terme de production par gravit\'e :
$\displaystyle
\mathcal{G}=-\frac{1}{\rho}\frac{\mu_t}{\sigma_t}
\frac{\partial\rho}{\partial x_i}g_i$

La viscosit\'e turbulente est
$\displaystyle \mu_t=\rho C_\mu\frac{k^2}{\varepsilon}$.

Les constantes sont :\\
$C_\mu=0,09$ ;
$C_{\varepsilon_2}=1,92$ ; $\sigma_k=1$ ; $\sigma_\varepsilon=1,3$\\
$C_{\varepsilon_3}=0$ si $\mathcal{G}\geqslant0$ (stratification instable) et
$C_{\varepsilon_3}=1$ si $\mathcal{G}\leqslant0$ (stratification stable).

$\Gamma$ est un \'eventuel terme source de masse (tel que l'\'equation de
conservation de masse devienne
$\displaystyle \frac{\partial \rho}{\partial t}+\dive(\rho\vect{u})=\Gamma$).
$\varphi_i$ ($\varphi=k$ ou $\varepsilon$) est la valeur de $\varphi$
associ\'ee \`a la masse inject\'ee ou retir\'ee. Dans le cas o\`u on retire de
la masse ($\Gamma<0$), on a forc\'ement $\varphi_i=\varphi$. De m\^eme, quand on
injecte de la masse, on sp\'ecifie souvent aussi $\varphi_i=\varphi$. Dans ces
deux cas, le terme dispara\^\i t de l'\'equation. Dans la suite du document, on
qualifiera d'{\em injection forc\'ee} les cas o\`u on a $\Gamma>0$ et
$\varphi_i\ne\varphi$.

$\alpha_k$, $\beta_k$, $\alpha_\varepsilon$, $\beta_\varepsilon$ sont des termes
sources utilisateur \'eventuels, conduisant \`a une implicitation partielle, impos\'es le cas
\'ech\'eant par le sous-programme \fort{ustske}.

See the \doxygenfile{turbke_8f90.html}{programmers reference of the dedicated subroutine} for further details.

%%%%%%%%%%%%%%%%%%%%%%%%%%%%%%%%%%
%%%%%%%%%%%%%%%%%%%%%%%%%%%%%%%%%%
\section*{Discr\'etisation}
%%%%%%%%%%%%%%%%%%%%%%%%%%%%%%%%%%
%%%%%%%%%%%%%%%%%%%%%%%%%%%%%%%%%%
La r\'esolution se fait en trois \'etapes, afin de coupler partiellement les
deux variables $k$ et $\varepsilon$. Pour simplifier, r\'e\'ecrivons le
syst\`eme de la fa\c con suivante :

\begin{equation}
\left\{\begin{array}{l}
\displaystyle
\rho\frac{\partial k}{\partial t} =
D(k) + S_k(k,\varepsilon)+k\dive(\rho\vect{u})+\Gamma(k_i-k)+\alpha_k k +\beta_k\\
\displaystyle
\rho\frac{\partial \varepsilon}{\partial t}  =
D(\varepsilon) + S_\varepsilon(k,\varepsilon)
+\varepsilon\dive(\rho\vect{u})
+\Gamma(\varepsilon_i-\varepsilon)+\alpha_\varepsilon \varepsilon +\beta_\varepsilon
\end{array}\right.
\end{equation}

$D$ est l'op\'erateur de convection/diffusion.
$S_k$ (resp. $S_\varepsilon$) est le terme source de $k$ (resp. $\varepsilon$).

\minititre{Premi\`ere phase : bilan explicite}

On r\'esout le bilan explicite :
\begin{equation}
\left\{\begin{array}{l}
\displaystyle
\rho^{(n)}\frac{k_e-k^{(n)}}{\Delta t} =
D(k^{(n)}) + S_k(k^{(n)},\varepsilon^{(n)})
+k^{(n)}\dive(\rho\vect{u})+\Gamma(k_i-k^{(n)})+\alpha_k k^{(n)} +\beta_k\\
\displaystyle
\rho^{(n)}\frac{\varepsilon_e-\varepsilon^{(n)}}{\Delta t}  =
D(\varepsilon^{(n)}) + S_\varepsilon(k^{(n)},\varepsilon^{(n)})
+\varepsilon^{(n)}\dive(\rho\vect{u})
+\Gamma(\varepsilon_i-\varepsilon^{(n)})
+\alpha_\varepsilon \varepsilon^{(n)} +\beta_\varepsilon
\end{array}\right.
\end{equation}

(le terme en $\Gamma$ n'est pris en compte que dans le cas de l'injection forc\'ee)

\minititre{Deuxi\`eme phase : couplage des termes sources}

On implicite les termes sources de mani\`ere coupl\'ee :
\begin{equation}
\left\{\begin{array}{l}
\displaystyle
\rho^{(n)}\frac{k_{ts}-k^{(n)}}{\Delta t} =
D(k^{(n)}) + S_k(k_{ts},\varepsilon_{ts})
+k^{(n)}\dive(\rho\vect{u})+\Gamma(k_i-k^{(n)})+\alpha_k k^{(n)} +\beta_k\\
\displaystyle
\rho^{(n)}\frac{\varepsilon_{ts}-\varepsilon^{(n)}}{\Delta t}  =
D(\varepsilon^{(n)}) + S_\varepsilon(k_{ts},\varepsilon_{ts})
+\varepsilon^{(n)}\dive(\rho\vect{u})
+\Gamma(\varepsilon_i-\varepsilon^{(n)})
+\alpha_\varepsilon \varepsilon^{(n)} +\beta_\varepsilon
\end{array}\right.
\end{equation}
soit
\begin{equation}
\left\{\begin{array}{l}
\displaystyle
\rho^{(n)}\frac{k_{ts}-k^{(n)}}{\Delta t} =
\rho^{(n)}\frac{k_e-k^{(n)}}{\Delta t}
+S_k(k_{ts},\varepsilon_{ts})-S_k(k^{(n)},\varepsilon^{(n)})\\
\displaystyle
\rho^{(n)}\frac{\varepsilon_{ts}-\varepsilon^{(n)}}{\Delta t}  =
\rho^{(n)}\frac{\varepsilon_e-\varepsilon^{(n)}}{\Delta t}
+S_\varepsilon(k_{ts},\varepsilon_{ts})-S_\varepsilon(k^{(n)},\varepsilon^{(n)})
\end{array}\right.
\end{equation}

Le terme en $\dive(\rho\vect{u})$ n'est pas implicit\'e car il est li\'e au
terme $D$ pour assurer que la matrice d'implicitation sera \`a diagonale
dominante. Le terme en $\Gamma$ et les termes sources utilisateur ne sont
pas implicit\'es non plus, mais ils le seront dans la troisi\`eme phase.

Et on \'ecrit (pour $\varphi=k$ ou $\varepsilon$)
\begin{equation}
S_\varphi(k_{ts},\varepsilon_{ts})-S_\varphi(k^{(n)},\varepsilon^{(n)})
=(k_{ts}-k^{(n)})
\left.\frac{\partial S_\varphi}{\partial k}\right|_{k^{(n)},\varepsilon^{(n)}}
+(\varepsilon_{ts}-\varepsilon^{(n)})
\left.\frac{\partial S_\varphi}{\partial \varepsilon}\right|_{k^{(n)},\varepsilon^{(n)}}
\end{equation}

On r\'esout donc finalement le syst\`eme $2\times 2$ :
\begin{equation}
\left(\begin{array}{cc}
\displaystyle \frac{\rho^{(n)}}{\Delta t}
-\left.\frac{\partial S_k}{\partial k}\right|_{k^{(n)},\varepsilon^{(n)}}
&\displaystyle
-\left.\frac{\partial S_k}{\partial \varepsilon}\right|_{k^{(n)},\varepsilon^{(n)}}\\
\displaystyle
-\left.\frac{\partial S_\varepsilon}{\partial k}\right|_{k^{(n)},\varepsilon^{(n)}}
&\displaystyle
\displaystyle \frac{\rho^{(n)}}{\Delta t}
-\left.\frac{\partial S_\varepsilon}{\partial \varepsilon}\right|_{k^{(n)},\varepsilon^{(n)}}
\end{array}\right)
\left(\begin{array}{c}
(k_{ts}-k^{(n)})\\(\varepsilon_{ts}-\varepsilon^{(n)})
\end{array}\right)
=\left(\begin{array}{c}
\displaystyle\rho^{(n)}\frac{k_e-k^{(n)}}{\Delta t}\\
\displaystyle\rho^{(n)}\frac{\varepsilon_e-\varepsilon^{(n)}}{\Delta t}
\end{array}\right)
\end{equation}

\vspace*{0.2cm}

\minititre{Troisi\`eme phase : implicitation de la convection/diffusion}

On r\'esout le syst\`eme :
\begin{equation}
\left\{\begin{array}{l}
\displaystyle
\rho^{(n)}\frac{k^{(n+1)}-k^{(n)}}{\Delta t} =
D(k^{(n+1)}) + S_k(k_{ts},\varepsilon_{ts})
+k^{(n+1)}\dive(\rho\vect{u})+\Gamma(k_i-k^{(n+1)})\\
\multicolumn{1}{r}{+\alpha_k k^{(n+1)} +\beta_k}\\
\displaystyle
\rho^{(n)}\frac{\varepsilon^{(n+1)}-\varepsilon^{(n)}}{\Delta t}  =
D(\varepsilon^{(n+1)}) + S_\varepsilon(k_{ts},\varepsilon_{ts})
+\varepsilon^{(n+1)}\dive(\rho\vect{u})
+\Gamma(\varepsilon_i-\varepsilon^{(n+1)})\\
\multicolumn{1}{r}{+\alpha_\varepsilon \varepsilon^{(n+1)} +\beta_\varepsilon}
\end{array}\right.
\end{equation}
soit
\begin{equation}
\left\{\begin{array}{l}
\displaystyle
\rho^{(n)}\frac{k^{(n+1)}-k^{(n)}}{\Delta t} =
D(k^{(n+1)})-D(k^{(n)})+\rho^{(n)}\frac{k_{ts}-k^{(n)}}{\Delta t}
+(k^{(n+1)}-k^{(n)})\dive(\rho\vect{u})\\
\multicolumn{1}{r}{-\Gamma(k^{(n+1)}-k^{(n)})+\alpha_k(k^{(n+1)}-k^{(n)})}\\
\displaystyle
\rho^{(n)}\frac{\varepsilon^{(n+1)}-\varepsilon^{(n)}}{\Delta t}  =
D(\varepsilon^{(n+1)})-D(\varepsilon^{(n)})
+\rho^{(n)}\frac{\varepsilon_{ts}-\varepsilon^{(n)}}{\Delta t}
+(\varepsilon^{(n+1)}-\varepsilon^{(n)})\dive(\rho\vect{u})\\
\multicolumn{1}{r}{-\Gamma(\varepsilon^{(n+1)}-\varepsilon^{(n)})
+\alpha_\varepsilon(\varepsilon^{(n+1)}-\varepsilon^{(n)})}
\end{array}\right.
\end{equation}

Le terme en $\Gamma$ n'est l\`a encore pris en compte que dans le cas de
l'injection forc\'ee. Le terme en $\alpha$ n'est pris en compte que si $\alpha$ est
n\'egatif, pour \'eviter d'affaiblir la diagonale de la matrice qu'on va
inverser.


\minititre{Pr\'ecisions sur le couplage}
Lors de la phase de couplage, afin de privil\'egier la stabilit\'e et la
r\'ealisabilit\'e du r\'esultat, tous les termes ne sont pas pris en
compte. Plus pr\'ecis\'ement, on peut \'ecrire :

\begin{equation}
\left\{\begin{array}{l}
\displaystyle
S_k =
\rho C_\mu\frac{k^2}{\varepsilon}\left(\tilde{\mathcal{P}}+\tilde{\mathcal{G}}\right)
-\frac{2}{3}\rho k \dive(\vect{u})
-\rho\varepsilon\\
\displaystyle
S_\varepsilon =
\rho C_{\varepsilon_1} C_\mu k\left(\tilde{\mathcal{P}}
+(1-C_{\varepsilon_3})\tilde{\mathcal{G}}\right)
-\frac{2}{3}C_{\varepsilon_1}\rho \varepsilon \dive(\vect{u})
-\rho C_{\varepsilon_2}\frac{\varepsilon^2}{k}
\end{array}\right.
\end{equation}

en notant
$\displaystyle\tilde{\mathcal{P}}
= \left(\frac{\partial u_i}{\partial x_j} +
\frac{\partial u_j}{\partial x_i}\right)\frac{\partial u_i}{\partial x_j}
-\frac{2}{3}(\dive\vect{u})^2$\\
et
$\displaystyle\tilde{\mathcal{G}}
= -\frac{1}{\rho\sigma_t}
\frac{\partial\rho}{\partial x_i}g_i$

On a donc en th\'eorie
\begin{equation}
\left\{\begin{array}{l}
\displaystyle \frac{\partial S_k}{\partial k}=
2\rho C_\mu\frac{k}{\varepsilon}\left(\tilde{\mathcal{P}}+\tilde{\mathcal{G}}\right)
-\frac{2}{3}\rho \dive(\vect{u})\\
\displaystyle \frac{\partial S_k}{\partial \varepsilon}= -\rho\\
\displaystyle \frac{\partial S_\varepsilon}{\partial k}=
\rho C_{\varepsilon_1} C_\mu \left(\tilde{\mathcal{P}}
+(1-C_{\varepsilon_3})\tilde{\mathcal{G}}\right)
+\rho C_{\varepsilon_2}\frac{\varepsilon^2}{k^2}\\
\displaystyle \frac{\partial S_\varepsilon}{\partial \varepsilon}=
-\frac{2}{3}C_{\varepsilon_1}\rho \dive(\vect{u})
-2\rho C_{\varepsilon_2}\frac{\varepsilon}{k}
\end{array}\right.
\end{equation}

En pratique, on va chercher \`a assurer $k_{ts}>0$ et $\varepsilon_{ts}>0$. En se
basant sur un calcul simplifi\'e, ainsi que sur le retour d'exp\'erience
d'ESTET, on montre qu'il est pr\'ef\'erable de ne pas prendre en compte
certains termes. Au final, on r\'ealise le couplage suivant :

\begin{equation}
\left(\begin{array}{cc}
A_{11}&A_{12}\\
A_{21}&A_{22}
\end{array}\right)
\left(\begin{array}{c}
(k_{ts}-k^{(n)})\\(\varepsilon_{ts}-\varepsilon^{(n)})
\end{array}\right)
=\left(\begin{array}{c}
\displaystyle\frac{k_e-k^{(n)}}{\Delta t}\\
\displaystyle\frac{\varepsilon_e-\varepsilon^{(n)}}{\Delta t}
\end{array}\right)
\end{equation}
avec
\begin{equation}
\left\{\begin{array}{l}
\displaystyle A_{11}=\frac{1}{\Delta t}
-2 C_\mu\frac{k^{(n)}}{\varepsilon^{(n)}}
\Min\left[\left(\tilde{\mathcal{P}}+\tilde{\mathcal{G}}\right),0\right]
+\frac{2}{3}\Max\left[\dive(\vect{u}),0\right]\\
\displaystyle A_{12}= 1\\
\displaystyle A_{21}=
- C_{\varepsilon_1} C_\mu \left(\tilde{\mathcal{P}}
+(1-C_{\varepsilon_3})\tilde{\mathcal{G}}\right)
- C_{\varepsilon_2}\left(\frac{\varepsilon^{(n)}}{k^{(n)}}\right)^2\\
\displaystyle A_{22}=\frac{1}{\Delta t}
+\frac{2}{3}C_{\varepsilon_1}\Max\left[\dive(\vect{u}),0\right]
+2 C_{\varepsilon_2}\frac{\varepsilon^{(n)}}{k^{(n)}}
\end{array}\right.
\end{equation}

(par d\'efinition de $C_{\varepsilon_3}$,
$\tilde{\mathcal{P}}+(1-C_{\varepsilon_3})\tilde{\mathcal{G}}$
est toujours positif)

%%%%%%%%%%%%%%%%%%%%%%%%%%%%%%%%%%
%%%%%%%%%%%%%%%%%%%%%%%%%%%%%%%%%%
\section*{Mise en \oe uvre}
%%%%%%%%%%%%%%%%%%%%%%%%%%%%%%%%%%
%%%%%%%%%%%%%%%%%%%%%%%%%%%%%%%%%%

\etape{Calcul du terme de production}
On appelle trois fois \fort{grdcel} pour calculer les gradients de $u$, $v$ et
$w$. Au final, on a \\
$\displaystyle \var{TINSTK}=
2\left(\frac{\partial u}{\partial x}\right)^2+
2\left(\frac{\partial v}{\partial y}\right)^2+
2\left(\frac{\partial w}{\partial z}\right)^2+
\left(\frac{\partial u}{\partial y}+\frac{\partial v}{\partial x}\right)^2+
\left(\frac{\partial u}{\partial z}+\frac{\partial w}{\partial x}\right)^2+
\left(\frac{\partial v}{\partial z}+\frac{\partial w}{\partial y}\right)^2$\\
et\\
$\displaystyle \var{DIVU}=
\frac{\partial u}{\partial x}+\frac{\partial v}{\partial y}
+\frac{\partial w}{\partial z}$

(le terme $div(\vect{u})$ n'est pas calcul\'e par \fort{divmas}, pour
correspondre exactement \`a la trace du tenseur des d\'eformations qui est
calcul\'e pour la production)


\etape{Lecture des termes sources utilisateur}
Appel de \fort{cs\_user\_turbulence\_source\_terms} pour charger les termes sources utilisateurs. Ils sont
stock\'es dans les tableaux suivants :\\
$\var{W7}=\Omega\beta_k$\\
$\var{W8}=\Omega\beta_\varepsilon$\\
$\var{DAM}=\Omega\alpha_k$\\
$\var{W9}=\Omega\alpha_\varepsilon$

Puis on ajoute le terme en $(div\vect{u})^2$ \`a \var{TINSTK}. On a donc \\
$\var{TINSTK}=\tilde{\mathcal{P}}$

\etape{Calcul du terme de gravit\'e}
Calcul uniquement si $\var{IGRAKE}=1$.\\
On appelle \fort{grdcel} pour \var{ROM}, avec comme conditions aux limites
$\var{COEFA}=\var{ROMB}$ et \mbox{$\var{COEFB}=\var{VISCB}=0$}.\\
$\var{PRDTUR}=\sigma_t$ est mis \`a 1 si on n'a pas de scalaire temp\'erature.

$\tilde{\mathcal{G}}$ est calcul\'e et les termes sources sont mis \`a jour :\\
$\var{TINSTK}=\tilde{\mathcal{P}}+\tilde{\mathcal{G}}$\\
$\var{TINSTE}=\tilde{\mathcal{P}}+\Max\left[\tilde{\mathcal{G}},0\right]
=\tilde{\mathcal{P}}+(1-C_{\varepsilon_3})\tilde{\mathcal{G}}$

Si $\var{IGRAKE}=0$, on a simplement\\
$\var{TINSTK}=\var{TINSTE}=\tilde{\mathcal{P}}$

\etape{Calcul du terme d'accumulation de masse}
On stocke
$\displaystyle \var{W1}=\Omega\dive(\rho\vect{u})$
calcul\'e par \fort{divmas} (pour correspondre aux termes de convection de la
matrice).

\etape{Calcul des termes sources explicites}
On affecte les termes sources explicites de $k$ et $\varepsilon$ pour la
premi\`ere \'etape.\\
$\displaystyle\var{SMBRK}=\Omega\left(\mu_t(\tilde{\mathcal{P}}+\tilde{\mathcal{G}})
-\frac{2}{3}\rho^{(n)} k^{(n)}\dive{\vect{u}}
-\rho^{(n)} \varepsilon^{(n)}\right)+\Omega k^{(n)}\dive(\rho\vect{u})$\\
$\displaystyle\var{SMBRE}=\Omega\frac{\varepsilon^{(n)}}{k^{(n)}}
\left(C_{\varepsilon_1}\left(
\mu_t(\tilde{\mathcal{P}}+(1-C_{\varepsilon_3})\tilde{\mathcal{G}})
-\frac{2}{3}\rho^{(n)} k^{(n)}\dive{\vect{u}}\right)
-C_{\varepsilon_2}\rho^{(n)}\varepsilon^{(n)}\right)
+\Omega\varepsilon^{(n)}\dive(\rho\vect{u})$

soit $\var{SMBRK}=\Omega S_k^{(n)}+\Omega k^{(n)}\dive(\rho\vect{u})$
et $\var{SMBRE}=\Omega S_\varepsilon^{(n)}+\Omega\varepsilon^{(n)}\dive(\rho\vect{u})$.


\etape{Calcul des termes sources utilisateur}
On ajoute les termes sources utilisateur explicites \`a \var{SMBRK} et
\var{SMBRE}, soit :\\
$\var{SMBRK}=\Omega S_k^{(n)}+\Omega k^{(n)}\dive(\rho\vect{u})+\Omega\alpha_k k^{(n)} +\Omega\beta_k$\\
$\var{SMBRE}=\Omega S_\varepsilon^{(n)}+\Omega\varepsilon^{(n)}\dive(\rho\vect{u})
+\Omega\alpha_\varepsilon \varepsilon^{(n)} +\Omega\beta_\varepsilon$

Les tableaux \var{W7} et \var{W8} sont lib\'er\'es, \var{DAM} et \var{W9} sont
conserv\'es pour \^etre utilis\'es dans la troisi\`eme phase de r\'esolution.

\etape{Calcul des termes de convection/diffusion explicites}
\fort{bilsc2} est appel\'e deux fois, pour $k$ et pour $\varepsilon$, afin
d'ajouter \`a \var{SMBRK} et \var{SMBRE} les termes de convection/diffusion
explicites $D(k^{(n)})$ et $D(\varepsilon^{(n)})$. Ces termes sont d'abord
stock\'es dans \var{W7} et \var{W8}, pour \^etre conserv\'es et r\'eutilis\'es
dans la troisi\`eme phase de r\'esolution.


\etape{Termes source de masse}
Dans le cas d'une injection forc\'ee de mati\`ere, on passe deux fois dans
\fort{catsma} pour ajouter les termes en
$\Omega \Gamma (k_i-k^{(n)})$ et
$\Omega \Gamma (\varepsilon_i-\varepsilon^{(n)})$ \`a \var{SMBRK} et
\var{SMBRE}. La partie implicite ($\Omega\Gamma$) est stock\'ee dans les
variables \var{W2} et \var{W3}, qui seront utilis\'ees lors de la troisi\`eme
phase (les deux variables sont bien n\'ecessaires, au cas o\`u on aurait une
injection forc\'ee sur $k$ et pas sur $\varepsilon$, par exemple).

\etape{Fin de la premi\`ere phase}
Ceci termine la premi\`ere phase. On a \\
$\displaystyle \var{SMBRK}=\Omega \rho^{(n)}\frac{k_e-k^{(n)}}{\Delta t}$\\
$\displaystyle \var{SMBRE}=\Omega \rho^{(n)}\frac{\varepsilon_e-\varepsilon^{(n)}}{\Delta t}$

\etape{Phase de couplage}
(uniquement si $\var{IKECOU}=1$)

On renormalise \var{SMBRK} et \var{SMBRE} qui deviennent les seconds membres du
syst\`eme de couplage.\\
$\displaystyle \var{SMBRK}=\frac{1}{\Omega\rho^{(n)}}\var{SMBRK}
=\frac{k_e-k^{(n)}}{\Delta t}$\\
$\displaystyle \var{SMBRE}=\frac{1}{\Omega\rho^{(n)}}\var{SMBRE}
=\frac{\varepsilon_e-\varepsilon^{(n)}}{\Delta t}$\\
et $\displaystyle \var{DIVP23}=\frac{2}{3}\Max\left[\dive(\vect{u}),0\right]$.

On remplit la matrice de couplage\\
$\displaystyle \var{A11}=\frac{1}{\Delta t}
-2 C_\mu\frac{k^{(n)}}{\varepsilon^{(n)}}
\Min\left[\left(\tilde{\mathcal{P}}+\tilde{\mathcal{G}}\right),0\right]
+\frac{2}{3}\Max\left[\dive(\vect{u}),0\right]$\\
$\displaystyle \var{A12}= 1$\\
$\displaystyle \var{A21}=
- C_{\varepsilon_1} C_\mu \left(\tilde{\mathcal{P}}
+(1-C_{\varepsilon_3})\tilde{\mathcal{G}}\right)
- C_{\varepsilon_2}\left(\frac{\varepsilon^{(n)}}{k^{(n)}}\right)^2$\\
$\displaystyle \var{A22}=\frac{1}{\Delta t}
+\frac{2}{3}C_{\varepsilon_1}\Max\left[\dive(\vect{u}),0\right]
+2 C_{\varepsilon_2}\frac{\varepsilon^{(n)}}{k^{(n)}}$

On inverse le syst\`eme $2\times 2$, et on r\'ecup\`ere :\\
$\displaystyle \var{DELTK}=k_{ts}-k^{(n)}$\\
$\displaystyle \var{DELTE}=\varepsilon_{ts}-\varepsilon^{(n)}$

\etape{Fin de la deuxi\`eme phase}
On met \`a jour les variables \var{SMBRK} et \var{SMBRE}.\\
$\displaystyle \var{SMBRK}=\Omega \rho^{(n)}\frac{k_{ts}-k^{(n)}}{\Delta t}$\\
$\displaystyle \var{SMBRE}=
\Omega \rho^{(n)}\frac{\varepsilon_{ts}-\varepsilon^{(n)}}{\Delta t}$

Dans le cas o\`u on ne couple pas ($\var{IKECOU}=0$), ces deux variables gardent
la m\^eme valeur qu'\`a la fin de la premi\`ere \'etape.

\etape{Calcul des termes implicites}
On retire \`a \var{SMBRK} et \var{SMBRE} la partie en convection diffusion au
temps $n$, qui \'etait stock\'ee dans \var{W7} et \var{W8}.\\
$\displaystyle \var{SMBRK}=\Omega \rho^{(n)}\frac{k_{ts}-k^{(n)}}{\Delta t}
-\Omega D(k^{(n)})$\\
$\displaystyle \var{SMBRE}=
\Omega \rho^{(n)}\frac{\varepsilon_{ts}-\varepsilon^{(n)}}{\Delta t}
-\Omega D(\varepsilon^{(n)})$


On calcule les termes implicites, hors convection/diffusion, qui correspondent
\`a la diagonale de la matrice.\\
$\displaystyle \var{TINSTK}=\frac{\Omega \rho^{(n)}}{\Delta t}
-\Omega\dive(\rho\vect{u})+\Omega\Gamma+\Omega\Max[-\alpha_k,0]$\\
$\displaystyle \var{TINSTE}=\frac{\Omega \rho^{(n)}}{\Delta t}
-\Omega\dive(\rho\vect{u})+\Omega\Gamma+\Omega\Max[-\alpha_\varepsilon,0]$\\
($\Gamma$ n'est pris en compte qu'en injection forc\'ee, c'est-\`a-dire qu'il
est forc\'ement positif et ne risque pas d'affaiblir la diagonale de la matrice).

\etape{R\'esolution finale}
On passe alors deux fois dans \fort{codits}, pour $k$ et $\varepsilon$,
pour r\'esoudre les \'equations du type :

$\var{TINST}\times(\varphi^{(n+1)}-\varphi^{(n)}) = D(\varphi^{(n+1)})+\var{SMBR}$.

\etape{clipping final}
On passe enfin dans la routine \fort{clipke} pour faire un clipping \'eventuel
de $k^{(n+1)}$ et $\varepsilon^{(n+1)}$.




%-------------------------------------------------------------------------------

% This file is part of code_saturne, a general-purpose CFD tool.
%
% Copyright (C) 1998-2022 EDF S.A.
%
% This program is free software; you can redistribute it and/or modify it under
% the terms of the GNU General Public License as published by the Free Software
% Foundation; either version 2 of the License, or (at your option) any later
% version.
%
% This program is distributed in the hope that it will be useful, but WITHOUT
% ANY WARRANTY; without even the implied warranty of MERCHANTABILITY or FITNESS
% FOR A PARTICULAR PURPOSE.  See the GNU General Public License for more
% details.
%
% You should have received a copy of the GNU General Public License along with
% this program; if not, write to the Free Software Foundation, Inc., 51 Franklin
% Street, Fifth Floor, Boston, MA 02110-1301, USA.

%-------------------------------------------------------------------------------

\programme{turrij}
\label{ap:turrij}

\hypertarget{turrij}{}

\vspace{1cm}
%-------------------------------------------------------------------------------
\section*{Fonction}
%-------------------------------------------------------------------------------
Le but de ce sous-programme est de r\'esoudre le syst\`eme des \'equations des
tensions de Reynolds et de la dissipation $\varepsilon$ de mani\`ere totalement d\'ecoupl\'ee dans le cadre de l'utilisation du mod\`ele $R_{ij}-\varepsilon$  LRR\footnote{la description du SSG est pr\'evue pour une version ult\'erieure de la documentation} (option $\var{ITURB}=30$ dans \fort{usini1}).\\
Le tenseur sym\'etrique des tensions de Reynolds est not\'e $\tens{R}$. Les composantes de ce tenseur repr\'esentent le moment d'ordre deux de la vitesse : $R_{ij} = \overline{u_iu_j}$.

Pour chaque composante $R_{ij}$, on r\'esout :

\begin{equation}
\begin{array}{ll}
\displaystyle
\rho\frac{\partial R_{ij}}{\partial t} +
\dive(\rho \vect{u}\,R_{ij} - \mu\,\grad{R_{ij}}) = &
\mathcal{P}_{ij} + \mathcal{G}_{ij}+\Phi_{ij} + \it{d}_{ij} - \varepsilon_{ij} +
R_{ij}\,\dive{(\rho \vect{u})} \\
& \displaystyle + \Gamma(R^{\,in}_{ij}-R_{ij}) + \alpha_{R_{ij}} R_{ij} + \beta_{R_{ij}}
\end{array}
\end{equation}

$\tens{\mathcal{P}}$ est le tenseur de production par cisaillement moyen :

\begin{equation}
\displaystyle \mathcal{P}_{ij} = \displaystyle -\rho \left[ R_{ik} \frac{\partial u_j}{\partial x_k} + R_{jk} \frac{\partial u_i}{\partial x_k} \right]
\end{equation}


$\tens{\mathcal{G}}$ est le tenseur de production par gravit\'e :

\begin{equation}
\displaystyle
\mathcal{G}_{ij}= \left[ G_{ij} - C_3 (G_{ij}-\frac{1}{3} \delta_{ij} G_{kk}) \right]
\end{equation}

avec

\begin{equation}
\left\{
\begin{array} {c}
\displaystyle G_{ij} = - \frac{3}{2} \frac{C_{\mu}}{\sigma_{t}} \frac{k}{\varepsilon} (r_i g_j + r_j g_i) \\
\displaystyle k = \frac{1}{2} R_{ll} \\
\displaystyle r_i = R_{ik} \frac{\partial \rho}{\partial x_k}
\end{array}\right.
\end{equation}

Dans ce qui pr\'ec\`ede, $k$ repr\'esente l'\'energie turbulente\footnote{Les
sommations se font sur l'indice $l$ et on applique plus
g\'en\'eralement la convention de sommation d'Einstein.}, $g_i$ la composante de
la gravit\'e dans la direction $i$, $\sigma_{t}$ le nombre de Prandlt turbulent  et $C_{\mu}$, $C_3$ des constantes d\'efinies dans Tab.~\ref{Base_Turrij_table_Cstes}.


$\tens{\Phi}$ est le terme de corr\'elations pression-d\'eformation. Il est mod\'elis\'e avec le terme de dissipation $\tens{\varepsilon}$ de la mani\`ere suivante :

\begin{equation}
\displaystyle
\Phi_{ij} - [\varepsilon_{ij}- \frac{2}{3} \rho \ \delta_{ij} \varepsilon] = \phi_{ij,1} + \phi_{ij,2} + \phi_{ij,w}
\end{equation}

Il en r\'esulte :

\begin{equation}
\displaystyle
\Phi_{ij} - \varepsilon_{ij} = \phi_{ij,1} + \phi_{ij,2} + \phi_{ij,w}  -\frac{2}{3} \rho \ \delta_{ij} \varepsilon
\end{equation}

Le terme $\phi_{ij,1}$ est un terme "lent" de retour \`a l'isotropie. Il est donn\'e par :

\begin{equation}
\displaystyle
\phi_{ij,1} = -\rho\,C_1 \frac{\varepsilon}{k} (R_{ij} - \frac{2}{3} k \delta_{ij})
\end{equation}

Le terme $\phi_{ij,2}$ est un terme "rapide" d'isotropisation de la production. Il est donn\'e par :
\begin{equation}
\displaystyle
\phi_{ij,2} = -C_2 (\mathcal{P}_{ij} - \frac{2}{3} \mathcal{P} \delta_{ij})
\end{equation}

avec,

$$\displaystyle \mathcal{P} = \frac{1}{2} \mathcal{P}_{kk}$$

Le terme $\phi_{ij,w}$ est appel\'e "terme d'echo de paroi". Il n'est pas
utilis\'e par d\'efaut dans \CS, mais peut \^etre activ\'e avec $\var{IRIJEC} = 1$. Si $y$ repr\'esente la distance \`a la paroi :

\begin{equation}
\begin{array} {ll}
\displaystyle
\phi_{ij,w}  = &
\displaystyle \rho\,C'_1 \frac{k}{\varepsilon} \left[ R_{km} n_k n_m \delta_{ij} -
\frac{3}{2} R_{ki} n_k n_j -
\frac{3}{2} R_{kj} n_k n_i \right] f(\frac{l}{y})  \\
&
+\displaystyle \rho\,C'_2 \left[ \phi_{km,2} n_k n_m \delta_{ij} -
\frac{3}{2} \phi_{ki,2} n_k n_j -
\frac{3}{2} \phi_{kj,2} n_k n_i \right] f(\frac{l}{y})
\end{array}
\end{equation}

$f$ est une fonction d'amortissement construite pour valoir 1 en paroi et tendre
vers 0 en s'\'eloignant des parois.\\
La longueur $l$ repr\'esente
$\displaystyle\frac{k^{\,\frac{3}{2}}}{\varepsilon}$, caract\'eristique de la turbulence. On prend :

\begin{equation}
f(\frac{l}{y}) = min(1, \ C^{\,0,75}_{\mu} \
\frac{k^{\,\frac{3}{2}}}{\varepsilon\ \kappa y})
\end{equation}


$\it{d}_{ij}$ est un terme de diffusion turbulente\footnote{Dans la litt\'erature, ce terme contient en g\'en\'eral la dissipation par viscosit\'e mol\'eculaire.} qui vaut :

\begin{equation}
\it{d}_{ij} = C_{S} \frac{\partial}{\partial x_k} (\rho \frac{k}{\varepsilon} R_{km} \frac{\partial R_{ij}}{\partial x_m})
\end{equation}

On notera par la suite $\displaystyle \tens{A} = C_S\,\rho\,\frac{k}{\varepsilon}\,\tens{R}$. Ainsi, $\displaystyle d_{ij} = \dive(\,\tens{A}\,\grad(R_{ij}))$ est une diffusion avec un coefficient tensoriel.

Le terme de dissipation turbulente $\tens{\varepsilon}$ est trait\'e dans ce qui pr\'ec\`ede avec le terme $\tens{\Phi}$.

$\Gamma$ est le terme source de masse\footnote{Dans ce cas, l'\'equation de continuit\'e s'\'ecrit : $\displaystyle \frac{\partial \rho}{\partial t} + \dive{(\rho \vect{u})} = \Gamma$.}, $R^{\,in}_{ij}$ est la valeur de $R_{ij}$ associ\'ee \`a la masse inject\'ee ou retir\'ee.

($\alpha_{R_{ij}}\,R_{ij} + \beta_{R_{ij}}$) repr\'esente le terme source
utilisateur (sous-programme \fort{cs\_user\_turbulence\_source\_terms}) \'eventuel avec une d\'ecomposition
permettant d'impliciter la partie $\alpha_{R_{ij}}\,R_{ij}$ si $\alpha_{R_{ij}} \geqslant 0$.

De m\^eme, on r\'esout une \'equation de convection/diffusion/termes sources pour la dissipation $\varepsilon$. Cette \'equation est tr\`es semblable \`a celle du mod\`ele $k-\varepsilon$ (voir \fort{turbke}), seuls les termes de viscosit\'e turbulente et de gravit\'e changent. On r\'esout :

\begin{equation}
\begin{array} {ll}
\displaystyle \rho\frac{\partial \varepsilon}{\partial t} +
\dive\left[\rho \vect{u}\,\varepsilon-
(\mu \grad{\varepsilon})\right] = &
\displaystyle \it{d}_{\,\varepsilon}
+ C_{\varepsilon_1}\frac{\varepsilon}{k}\left[\mathcal{P}
+\mathcal{G}_{\varepsilon}\right]
-\rho C_{\varepsilon_2}\frac{\varepsilon^2}{k}
+\varepsilon\dive(\rho\vect{u})\\
&
\displaystyle
+\Gamma(\varepsilon^{\,in}-\varepsilon)
+\alpha_\varepsilon \varepsilon +\beta_\varepsilon
\end{array}
\end{equation}


$\it{d}_{\,\varepsilon}$ est le terme de diffusion turbulente :
\begin{equation}
\displaystyle
\it{d}_{\,\varepsilon} = C_{\varepsilon} \displaystyle \frac{\partial}{\partial x_k} \left( \rho \frac{k}{\varepsilon} R_{km} \frac{\partial \varepsilon}{\partial x_m} \right)
\end{equation}
On notera par la suite $\tens{A'} = \displaystyle \rho \,C_{\varepsilon} \frac{k}{\varepsilon} \tens{R}$.
Le terme de diffusion turbulente est donc mod\'elis\'e par : $$\it{d}_{\,\varepsilon} =
\dive(\tens{A'}\,\grad(\varepsilon))$$
La viscosit\'e turbulente usuelle ($\nu_t$) en mod\`ele $k-\varepsilon$ est remplac\'ee par le tenseur visqueux~$\tens{A'}$.

$\mathcal{P}$ est le terme de production par cisaillement moyen :
$\mathcal{P} =\displaystyle \frac{1}{2} \mathcal{P}_{kk}$. Ce terme est
mod\'elis\'e avec la notion de viscosit\'e turbulente dans le cadre du mod\`ele
$k-\varepsilon$. Dans le cas pr\'esent, il est calcul\'e \`a l'aide des tensions
de Reynolds (\`a partir de $\mathcal{P}_{ij}$).

$\mathcal{G}_{\varepsilon}$ est le terme de production des effets de gravit\'e pour la variable $\varepsilon$.
\begin{equation}
\mathcal{G}_{\varepsilon} = max(0,\frac{1}{2}G_{kk})
\end{equation}
\begin{table}
{\scriptsize
\begin{center}
\begin{tabular}{|l|l|l|l|l|l|l|l|l|l|}
\hline
$C_\mu$  & $C_{\varepsilon}$  & $C_{\varepsilon_1}$ &
$C_{\varepsilon_2}$  & $C_1$ & $C_2$ & $C_3$ & $C_S$
& $C'_1$ & $C'_2$ \\
\hline
$0,09$ & $ 0,18$ & $1,44$ & $1,92$ & $1,8$ & $0,6$ & $0,55$ & $0,22$ & $0,5$ &
$0,3$ \\
\hline
\end{tabular}
\end{center}
}
\caption{D\'efinition des constantes utilis\'ees.}\label{Base_Turrij_table_Cstes}
\end{table}

See the \doxygenfile{turrij_8f90.html}{programmers reference of the dedicated subroutine} for further details.

%%%%%%%%%%%%%%%%%%%%%%%%%%%%%%%%%%
%%%%%%%%%%%%%%%%%%%%%%%%%%%%%%%%%%
\section*{Discr\'etisation}
%%%%%%%%%%%%%%%%%%%%%%%%%%%%%%%%%%
%%%%%%%%%%%%%%%%%%%%%%%%%%%%%%%%%%
La r\'esolution se fait en d\'ecouplant totalement les tensions de Reynolds
entre elles et la dissipation $\varepsilon$. On r\'esout ainsi une \'equation de
convection/diffusion/termes sources pour chaque variable. Les variables sont
r\'esolues dans l'ordre suivant : $R_{11}$, $R_{22}$, $R_{33}$, $R_{12}$,
$R_{13}$, $R_{23}$ et $ \varepsilon$. L'ordre de la r\'esolution n'est pas
important puisque l'on a opt\'e pour une r\'esolution totalement d\'ecoupl\'ee
en n'implicitant que les termes pouvant \^etre lin\'earis\'es par rapport \`a la
variable courante\footnote{En effet, aucune variable n'est actualis\'ee pour la r\'esolution de la suivante.}.

Les \'equations sont r\'esolues \`a l'instant $n+1$.
\subsection*{\bf Variables tensions de Reynolds}
Pour chaque composante $R_{ij}$, on \'ecrit :
\begin{equation}\label{Base_Turrij_Eq_Temp_Rij}
\begin{array}{ll}
\displaystyle
\rho^n\ \frac {R_{ij}^{\,n+1}-R_{ij}^{\,n}}{\Delta t^n}
+\ \dive\left[ (\rho \underline{u})^{n} R_{ij}^{\,n+1}
- \mu^n\ \grad{R}_{ij}^{\,n+1} \right]
=  &
\displaystyle
\mathcal{P}^{\,n}_{ij}
+ \mathcal{G}^n_{ij} \\
&
\displaystyle
+ \phi^{\,n,n+1}_{ij,1} + \phi^{\,n}_{ij,2} + \phi^{\,n}_{ij,w} \\
&
\displaystyle
+ \text{\it{d}}^{\,n,n+1}_{ij}
- \displaystyle \frac{2}{3} \rho^n \varepsilon^n \delta_{ij}
+ R^{\,n+1}_{ij} \dive{(\rho \underline{u})^n} \\
&
\displaystyle
+ \Gamma(R^{\,in}_{ij} - R^{\,n+1}_{ij}) \\
&
\displaystyle
+ \alpha^n_{R_{ij}} R^{\,n+1}_{ij} + \beta^n_{R_{ij}}
\end{array}
\end{equation}
$\mu^n$ est la viscosit\'e mol\'eculaire\footnote{La viscosit\'e peut
d\'ependre \'eventuellement de la temp\'erature ou d'autres variables.}.\\
L'indice $(\,n,n+1)$ est relatif \`a une semi implicitation des termes (voir ci-dessous). Quand seul l'indice $(n)$ est utilis\'e, il suffit de reprendre l'expression des termes et de consid\'erer que toutes les variables sont explicites.

Dans le terme $\phi^{n,n+1}_{ij,1}$ donn\'e ci-dessous, la tension de Reynolds
 $R_{ij}$ est implicite (les tensions diagonales apparaissent aussi dans l'\'energie
turbulente $k$). Ainsi :
\begin{equation}
\displaystyle
\phi^{\,n,n+1}_{ij,1} = -\rho^n \,C_1\,\frac{\varepsilon^n}{k^n}\left[
(1-\frac{\delta_{ij}}{3}) R^{\,n+1}_{ij}- \delta_{ij} \frac{2}{3} (k^n-\frac{1}{2} R^{\,n}_{ii}) \right]
\end{equation}

Le terme de diffusion turbulente $\tens{\it{d}}$ s'\'ecrit : $\it{d}_{ij} = \dive{\left[ \tens{A}\,\grad{R}_{ij} \right]}$.
Le tenseur $\tens{A}$ est toujours explicite.
En int\'egrant sur un volume de contr\^ole (cellule) $\Omega_l$, le terme $\tens{\it{d}}$ de diffusion turbulente de $R_{ij}$ s'\'ecrit :

\begin{equation}
\displaystyle\int_{\Omega_l} \it{d}^{\,n,n+1}_{ij}\ d\Omega =
\sum\limits_{m\in
Vois(l)} \left[
\tens{A}^n\,\grad{R}^{\,n+1}_{ij} \right]_{\,lm}\,.\,\vect{n}_{\,lm}S_{\,lm}
\end{equation}

$\vect{n}_{\,lm}$ est la normale unitaire \`a la face\footnote{La notion de
face purement interne ou de bord n'est pas explicit\'ee ici, pour all\'eger l'expos\'e. Pour \^etre rigoureux et homog\`ene avec les notations
adopt\'ees, il faudrait distinguer $ m\in {Vois(l)} $ et $ m\in {\gamma_b(l)}$.}
$ \partial \Omega_{\,lm} = \Gamma_{\,lm}$ de la fronti\`ere
 $\partial \Omega_{\,l} = \underset{\text{\it m}}{\cup}\ \partial
\Omega_{\,lm}$ de $\Omega_l$, face d\'esign\'ee par abus par $lm$ et $S_{\,lm}$ sa surface associ\'ee.

On d\'ecompose $\tens{A}^n$ en partie diagonale $\tens{D}^n$ et
extra-diagonale $\tens{E}^n$ :\\
$$\tens{A}^n =\tens{D}^n + \tens{E}^n$$
Ainsi,
\begin{equation}
\begin{array}{l}
\displaystyle \int_{\Omega_l} \it{d}_{ij}\ d\Omega =
\sum\limits_{m\in
Vois(l)} \underbrace{ \left[
\tens{D}^n\,\grad{R}_{ij}\right]_{\,lm}\,.\,\vect{n}_{\,lm}S_{\,lm} }
_{\text {partie diagonale}}\\
+ \displaystyle\sum\limits_{m\in
Vois(l)} \underbrace{ \left[
\tens{E}^n\,\grad{R}_{ij} \right]_{\,lm}\,.\,\vect{n}_{\,lm}S_{\,lm}\ }
_{\text {partie extra-diagonale}}
\end{array}
\end{equation}

La partie extra-diagonale sera prise totalement explicite et interviendra donc
dans l'expression regroupant les termes purement explicites $f_s^{\,exp}$ du
second membre de \fort{codits}.\\
Pour la partie diagonale, on introduit la composante normale du gradient de la
variable principale $R_{ij}$. Cette  contribution normale sera trait\'ee en
implicite pour la variable et interviendra \`a la fois dans l'expression de la matrice simplifi\'ee du syst\`eme r\'esolu par \fort{codits} et dans
le second membre trait\'e par \fort{bilsc2}. La
contribution tangentielle sera, elle, purement explicite et donc prise en compte
dans $f_s^{\,exp}$ intervenant dans le second membre de \fort{codits}.\\
On a :
\begin{equation}
\displaystyle
\grad{R}_{ij}  = \grad{R}_{ij} - (\grad{R}_{ij}\,.\,\vect{n}_{\,lm})\,\vect{n}_{\,lm} + (\grad{R}_{ij}\,.\,\vect{n}_{\,lm})\,\vect{n}_{\,lm}
\end{equation}

Comme $$\left[ \tens{D}^n\,\left[ (\grad{R}_{ij}\,.\,\vect{n}_{\,lm})\,\vect{n}_{\,lm}
\right] \right]\,.\,\vect{n}_{\,lm}  = \gamma^n_{\,lm} (\grad{R}_{ij}\,.\,\vect{n}_{\,lm})$$
 avec :
$$\gamma^n_{\,lm} = (D^n_{11})\,n^2_{\,1,\,lm} + (D^n_{22})\,n^2_{\,2,\,lm} +
(D^n_{33})\,n^2_{\,3,\,lm}$$
 on peut traiter ce terme $\gamma^n_{\,lm}$ comme une diffusion avec un
coefficient de diffusion ind\'ependant de la direction.\\

Finalement, on \'ecrit :
\begin{equation}
\begin{array} {lll}
&\displaystyle\int_{\Omega_l} \it{d}_{ij}^{\,n,n+1}\ d\Omega =\\
&\displaystyle
+ \sum\limits_{m\in
Vois(l)} \left[\ \tens{E}^n\,\grad{R}^{\,n}_{ij} \right]_{\,lm}\,.\,\vect{n}_{\,lm}S_{\,lm}\\
&+ \sum\limits_{m\in Vois(l)} \left[\
\tens{D}^n\,\grad{R}^{\,n}_{ij} \right]_{\,lm}\,.\,\vect{n}_{\,lm}S_{\,lm}\\
& - \sum\limits_{m\in Vois(l)} \gamma^n_{\,lm} \left(
\grad{R}^{\,n}_{ij}\,.\,\vect{n}_{\,lm} \right) S_{\,lm} +  \sum\limits_{m\in
Vois(l)} \gamma^n_{\,lm} \left( \grad{R}^{\,n+1}_{ij}\,.\,\vect{n}_{\,lm} \right)  S_{\,lm}
\end{array}
\end{equation}
Les trois premiers termes sont totalement explicites et correspondent \`a la
discr\'etisation de l'op\'erateur continu :
$$\dive(\,\tens{E}^n\,\grad{R}^{\,n}_{ij}) + \dive(\,\tens{D}^n\,[\,\grad{R}^{\,n}_{ij} - ( \grad{R}^{\,n}_{ij}\,.\,\vect{n}
)\,\vect{n}\,]\,)$$ en omettant la notion de face.\\
Le dernier terme est implicite relativement \`a la variable $R_{ij}$ et correspond \`a l'op\'erateur continu :
 $$\dive(\,\tens{D}^n\,(\grad{R}^{\,n+1}_{ij}\,.\,\vect{n} )\,\vect{n})$$
\subsection*{\bf Variable $\varepsilon$ }
On r\'esout l'\'equation de $\varepsilon$ de fa\c con analogue \`a celle de
$R_{ij}$.
\begin{equation}
\begin{array}{ll}
\displaystyle
\rho^n\ \frac {\varepsilon^{n+1}-\varepsilon^{n}}{\Delta t^n} +
\dive((\rho\,\underline{u})^{n} \varepsilon^{n+1})
- \dive(\mu^n\ \grad \varepsilon^{n+1})
=  &
\displaystyle
d_{\,\varepsilon}^{\,n,n+1} \\
&
\displaystyle
+ C_{\varepsilon_1} \frac{k^n}{\varepsilon^n} \left[ \mathcal{P}^n + \mathcal{G}^n_{\varepsilon} \right]
- \rho^n C_{\varepsilon_2} \frac{(\varepsilon^n)^2}{k^n} \\
&
\displaystyle
+ \varepsilon^{n+1} \dive{(\rho \underline{u})^n} \\
&
\displaystyle
+ \Gamma(\varepsilon^{\,in} - \varepsilon^{n+1})
+ \alpha^n_{\varepsilon} \varepsilon^{n+1} + \beta^n_{\varepsilon}
\end{array}
\end{equation}

Le terme de diffusion turbulente $\it{d}^{\,n,n+1}_{\,\varepsilon}$ est trait\'e comme celui des
variables $R_{ij}$ et s'\'ecrit : $$\it{d}_{\,\varepsilon}^{\,n,n+1} = \dive{\left[
\tens{A'}^{\,n}\,\grad {\varepsilon^{\,n+1}} \right]}$$
Le tenseur $\tens{A'}$ est toujours explicite.
On le d\'ecompose en une partie diagonale $\tens{D'}^{\,n}$ et une partie
extra-diagonale $\tens{E'}^{\,n}$ :\\
$$\tens{A'}^{\,n} =\tens{D'}^{\,n} + \tens{E'}^{\,n}$$
Ainsi :
\begin{equation}
\begin{array} {lcl}
&\displaystyle \int_{\Omega_l} \it{d}_{\,\varepsilon}^{\,n,n+1}\ d\Omega =
\sum\limits_{m\in Vois(l)} \left[
\tens{E'}^{\,n}\,\grad{\varepsilon}^n
\right]_{\,lm}\,.\,\vect{n}_{\,lm}S_{\,lm}\\
& + \sum\limits_{m\in Vois(l)} \left[
\tens{D'}^{\,n}\,\grad{\varepsilon}^n
\right]_{\,lm}\,.\,\vect{n}_{\,lm}S_{\,lm}\
- \sum\limits_{m\in Vois(l)}
 \eta^n_{\,lm} \left(\grad{\varepsilon}^{n}\,.\,\vect{n}_{\,lm} \right) S_{\,lm}\\
&+  \sum\limits_{m\in Vois(l)} \eta^n_{\,lm} \left( \grad{\varepsilon}^{n+1}\,.\,\vect{n}_{\,lm} \right) S_{\,lm}
\end{array}
\end{equation}
avec :
$$\eta^n_{\,lm} = (D'^{\,n}_{11})\,n^2_{\,1,\,lm} + (D'^{\,n}_{22})\,n^2_{\,2,\,lm} +
(D'^{\,n}_{33})\,n^2_{\,3,\,lm}.$$
On peut traiter ce terme $\eta^n_{\,lm}$ comme une diffusion avec un coefficient de diffusion ind\'ependant de la direction.\\
Les trois premiers termes sont totalement explicites et correspondent \`a l'op\'erateur :
$$\dive (\,\tens{E'}^{\,n}\,\varepsilon^{\,n}) +
\dive (\,\tens{D'}^{\,n}\,[\grad{\varepsilon^{\,n}} - (\grad{\varepsilon^{\,n}}.\,\vect{n})\,\vect{n}]\,)$$ en omettant la notion de face.\\
Le dernier terme est implicite relativement \`a la variable $\varepsilon$ et correspond \`a l'op\'erateur :
 $$\dive (\,\tens{D'}^{\,n}\,(\grad{\varepsilon^{\,n+1}}.\,\vect{n} )\,\vect{n})$$

%%%%%%%%%%%%%%%%%%%%%%%%%%%%%%%%%%
%%%%%%%%%%%%%%%%%%%%%%%%%%%%%%%%%%
\section*{Mise en \oe uvre}
%%%%%%%%%%%%%%%%%%%%%%%%%%%%%%%%%%
%%%%%%%%%%%%%%%%%%%%%%%%%%%%%%%%%%
La num\'ero de la phase trait\'ee fait partie des arguments de \fort{turrij}. On
omettra volontairement de le pr\'eciser dans ce qui suit, on indiquera par $(\ )$ la
notion de tableau s'y rattachant.

\etape{Calcul des termes de production $\tens{\mathcal{P}}$}
\begin{itemize}
\item [$\star$] Initialisation \`a z\'ero du tableau \var{PRODUC} dimensionn\'e \`a $\var{NCEL}\times 6$.
\item [$\star$] On appelle trois fois \fort{grdcel} pour calculer les gradients des composantes de la vitesse $u$, $v$ et
$w$ prises au temps $n$.

Au final, on a :\\
$\displaystyle
\begin{array} {ll}
\var{PRODUC(1,IEL)} = & \displaystyle - 2 \left[ R_{11}^{\,n} \frac{\partial u^{\,n}} {\partial x} +R_{12}^{\,n} \frac{\partial u^{\,n}} {\partial y}+R_{13}^{\,n} \frac{\partial u^{\,n}} {\partial z} \right] \text{        (production de $R_{11}^{\,n}$)}\\
\var{PRODUC(2,IEL)} = & \displaystyle - 2 \left[ R_{12}^{\,n} \frac{\partial v^{\,n}} {\partial x} +R_{22}^{\,n} \frac{\partial v^{\,n}} {\partial y}+R_{23}^{\,n} \frac{\partial v^{\,n}} {\partial z} \right] \text{        (production de $R_{22}^{\,n}$)}\\
\var{PRODUC(3,IEL)} = & \displaystyle - 2 \left[ R_{13}^{\,n} \frac{\partial w^{\,n}} {\partial x} +R_{23}^{\,n} \frac{\partial w^{\,n}} {\partial y}+R_{33}^{\,n} \frac{\partial w^{\,n}} {\partial z} \right] \text{        (production de $R_{33}^{\,n}$)}\\
\var{PRODUC(4,IEL)} = & \displaystyle - \left[ R_{12}^{\,n} \frac{\partial u^{\,n}} {\partial x} +R_{22}^{\,n} \frac{\partial u^{\,n}} {\partial y}+R_{23}^{\,n} \frac{\partial u^{\,n}} {\partial z} \right] \\
& \displaystyle - \left[ R_{11}^{\,n} \frac{\partial v^{\,n}} {\partial x} +R_{12}^{\,n} \frac{\partial v^{\,n}} {\partial y}+R_{13}^{\,n} \frac{\partial v^{\,n}} {\partial z} \right] \text{        (production de $R_{12}^{\,n}$)} \\
\var{PRODUC(5,IEL)} = & \displaystyle - \left[ R_{13}^{\,n} \frac{\partial u^{\,n}} {\partial x} +R_{23}^{\,n} \frac{\partial u^{\,n}} {\partial y}+R_{33}^{\,n} \frac{\partial u^{\,n}} {\partial z} \right] \\
& \displaystyle - \left[ R_{11}^{\,n} \frac{\partial w^{\,n}} {\partial x} +R_{12}^{\,n} \frac{\partial w^{\,n}} {\partial y}+R_{13}^{\,n} \frac{\partial w^{\,n}} {\partial z} \right] \text{        (production de $R_{13}^{\,n}$)} \\
\var{PRODUC(6,IEL)} = & \displaystyle - \left[ R_{13}^{\,n} \frac{\partial v^{\,n}} {\partial x} +R_{23}^{\,n} \frac{\partial v^{\,n}} {\partial y}+R_{33}^{\,n} \frac{\partial v^{\,n}} {\partial z} \right] \\
& \displaystyle - \left[ R_{12}^{\,n} \frac{\partial w^{\,n}} {\partial x} +R_{22}^{\,n} \frac{\partial w^{\,n}} {\partial y}+R_{23}^{\,n} \frac{\partial w^{\,n}} {\partial z} \right]  \text{        (production de $R_{23}^{\,n}$)}
\end{array}
$
\end{itemize}

\etape{Calcul du gradient de la masse volumique $\rho^n$ prise au d\'ebut du pas
de temps courant\footnote{{\it i.e.} calcul\'ee \`a partir des
variables du pas de temps pr\'ec\'edent $n$ si n\'ecessaire.} $(n+1)$}
Ce calcul n'a lieu que si les termes de gravit\'e doivent \^etre pris en compte
($\var{IGRARI()} =1$).
\begin{itemize}
\item [$\star$] Appel de \fort{grdcel}  pour calculer le gradient de $\rho^n$
dans les trois directions de l'espace. Les conditions aux limites sur $\rho^n$
sont des conditions de Dirichlet puisque la valeur de $\rho^n$ aux faces de bord
$ik$ (variable \var{IFAC}) est connue et vaut $\rho_{\,b_{\,ik}}$. Pour \'ecrire les conditions aux limites
sous la forme habituelle, $$\rho_{\,b_{\,ik}} = \var{COEFA} + \var{COEFB}
\,\rho^n_{\,I'}$$ on pose alors $\var{COEFA}=
\var{PROPCE(IFAC,IPPROB(IROM))}$ et $\var{COEFB} = \var{VISCB} = 0$.\\
$\var{PROPCE(1,IPPROB(IROM))}$ (resp.$\var{VISCB}$) est utilis\'e en lieu
et place de l'habituel \var{COEFA} ($\var{COEFB}$), lors de l'appel \`a \fort{grdcel}.\\
On a donc :\\
$\displaystyle \var{GRAROX}= \frac{\partial \rho^n}{\partial x}\ $,$\displaystyle \ \var{GRAROY}= \frac{\partial
\rho^n}{\partial y}$ et $
\displaystyle \ \var{GRAROZ}= \frac{\partial \rho^n}{\partial z}\ $.

\end{itemize}

Le gradient de $\rho^n$ servira \`a calculer les termes de production par effets de gravit\'e si ces derniers sont pris en compte.

\etape{Boucle \var{ISOU} de $1$ \`a $6$ sur les tensions de Reynolds}
Pour $\var{ISOU} = 1,2,3,4,5,6$, on r\'esout respectivement et dans
l'ordre  les
\'equations de $R_{11}$, $R_{22}$, $R_{33}$, $R_{12}$, $R_{13}$ et $R_{23}$ par
l'appel au sous-programme \fort{resrij}.\\
La r\'esolution se fait par incr\'ement $\delta {R}_{ij}^{\,n+1,k+1}$ , en utilisant la m\^eme m\'ethode que
celle d\'ecrite dans le sous-programme \fort{codits}. On adopte ici les m\^emes notations.
\var{SMBR} est le second membre du syst\`eme \`a inverser, syst\`eme portant sur
les incr\'ements de la variable. \var{ROVSDT} repr\'esente la diagonale de la
matrice, hors convection/diffusion.\\
On va r\'esoudre l'\'equation (\ref{Base_Turrij_Eq_Temp_Rij}) sous forme incr\'ementale en
utilisant \fort{codits}, soit :
\begin{equation}\label{Base_Turrij_Eq_Temp_deltaRij}
\begin{array}{ll}
&\displaystyle \underbrace{\left(\frac {\rho^n_L}{\Delta t^n}
+ \rho^n_L \,C_1\,\frac{\varepsilon^n_L}{k^n_L}(1-\frac{\delta_{ij}}{3})
 - m^n_{\,lm} + \Gamma_L\,+ max(-\alpha^n_{R_{ij}},0)\right)\,|\Omega_l|}
_{\text {$\var{ROVSDT}$ contribuant
\`a la diagonale de la matrice simplifi\'ee de \fort{matrix}}}\,(\delta{R}_{ij}^{\,n+1,p+1})_{\,L}\\\\
&  \underbrace{+\sum\limits_{m\in Vois(l)}\displaystyle \left[
 m^n_{\,lm} \delta R_{ij,\,f_{\,lm}}^{\,n+1,p+1}
- (\mu^n_{\,lm} + \gamma^n_{\,lm})\
\frac{({\delta R}_{ij}^{\,n+1,p+1})_{M}-({\delta R}_{ij}^{\,n+1,p+1})_{L})}{\overline{L'M'}}\,
S_{\,lm} \right]}_{\text { convection upwind pur et diffusion non reconstruite
relatives \`a la matrice simplifi\'ee de \fort{matrix}\footnotemark}} \\
% voir le texte de la footmark plus bas
&= - \displaystyle\frac {\rho^n_L}{\Delta t^n}\,\left(\,(R^{\,n+1,p}_{ij})_L - (R^{\,n}_{ij})_L\,\right)\\
&-\,\underbrace{\displaystyle\int_{\Omega_l} \left(
\dive\,[\,(\rho\,\vect{u})^n\,R^{\,n+1,p}_{ij} - (\mu^n\,+ \gamma^n\,)\,
\grad{R^{\,n+1,p}_{ij}}\,]\right)\,d\Omega}_{\text {convection et diffusion
trait\'ees par \fort{bilsc2}}}\\
&+\displaystyle \int_{\Omega_l} \left[\,\mathcal{P}^{\,n+1,p}_{ij} + \mathcal{G}^{\,n+1,p}_{ij}
- \displaystyle\rho^n \,C_1\,\frac{\varepsilon^n}{k^n}\left[R^{\,n+1,p}_{ij}-
\frac{2}{3}\,k^n\,\delta_{ij}\right] + \phi^{\,n+1,p}_{ij,2} +
\phi^{\,n+1,p}_{ij,w}\,\right]\, d\Omega \\
& + \displaystyle\int_{\Omega_l} \left[- \frac{2}{3} \rho^n \varepsilon^n \delta_{ij}
 + \Gamma\,(\,R^{\,in}_{ij} - R^{\,n+1,p}_{ij}\,) +
\alpha^n_{R_{ij}}\,R^{\,n+1,p}_{ij}+ \beta^n_{R_{ij}}\right]\, d\Omega\\
&+ \sum\limits_{m\in
Vois(l)}\displaystyle \left[\ \tens{E}^n\,\grad{R}^{\,n+1,p}_{ij} \right]_{\,lm}\,.\,\vect{n}_{\,lm}S_{\,lm}\\
&+ \sum\limits_{m\in Vois(l)}\displaystyle \left[\
\tens{D}^n\,\grad{R}^{\,n+1,p}_{ij} \right]_{\,lm}\,.\,\vect{n}_{\,lm}S_{\,lm}\\
&- \sum\limits_{m\in Vois(l)} \gamma^n_{\,lm} \left( \grad{R}^{\,n+1,p}_{ij}\,.\,\vect{n}_{\,lm} \right)  S_{\,lm}\\
&+ \sum\limits_{m\in Vois(l)}  m^n_{\,lm}\,(R^{\,n+1,p}_{ij})_L\\
\end{array}
\end{equation}
% si on ne fait pas comme ca, il n'apparait pas
\footnotetext[\thefootnote]{Si $\var{IRIJNU} = 1$, on remplace  $\mu^n_{\,lm}$ par $(\mu +
\mu_t)^n_{\,lm}$ dans l'expression de la diffusion non reconstruite
associ\'ee \`a la matrice simplifi\'ee de \fort{matrix} ($\mu_t$ d\'esigne la
viscosit\'e turbulente calcul\'ee comme en $k-\varepsilon$).}

o\`u on rappelle :\\
pour $n$ donn\'e entier positif, on d\'efinit la suite
 $({R}_{ij}^{\,n+1,p})_{p \in \mathbb{N}}$
 par :
\begin{equation}\notag
\left\{\begin{array}{l}
{R}_{ij}^{\,n+1,0} = {R}_{ij}^{\,n}\\
{R}_{ij}^{\,n+1,p+1} = {R}_{ij}^{\,n+1,p} + \delta{R}_{ij}^{\,n+1,p+1} \\
\end{array}\right.
\end{equation}
$(\delta{R}_{ij}^{\,n+1,p+1})_{\,L}$ d\'esigne la valeur sur l'\'el\'ement
$\Omega_l$ du $\text{$(\,p+1\,)$-i\`eme}$ incr\'ement de ${R}_{ij}^{\,n+1}$,
$ m^n_{\,lm}$ le flux de masse \`a l'instant $n$ \`a travers la face $lm$,
$\delta R_{ij,\,f_{\,lm}}^{\,n+1,p+1}$ vaut $({\delta
R}_{ij}^{\,n+1,p+1})_{L}$  si $ m^n_{\,lm} \geqslant 0$, $({\delta
R}_{ij}^{\,n+1,p+1})_{M}$ sinon,
$\mathcal{P}^{\,n+1,p}_{ij}$, $\phi^{\,n+1,p}_{ij,2}$, $\phi^{\,n+1,p}_{ij,w}$ les valeurs
des quantit\'es associ\'ees correspondant \`a l'incr\'ement
$(\delta{R}_{ij}^{\,n+1,p})$.\\



Tous ces termes sont calcul\'es comme suit :
\begin{itemize}
\item Terme de gauche de l'\'equation (\ref{Base_Turrij_Eq_Temp_deltaRij})\\
Dans \fort{resrij} est calcul\'ee la variable \var{ROVSDT}. Les autres
termes sont compl\'et\'es par \fort{codits}, lors de la construction de la matrice simplifi\'ee , {\it via} un
appel au sous-programme \fort{matrix}. La quantit\'e
 $(\mu^n_{\,lm} + \gamma^n_{\,lm})$ \`a la face $lm$ est calcul\'ee lors de l'appel \`a
\fort{visort}.\\
\item Second membre de l'\'equation (\ref{Base_Turrij_Eq_Temp_deltaRij})\\
Le premier terme non d\'etaill\'e est calcul\'e par le sous-programme
\fort{bilsc2}, qui applique le sch\'ema convectif choisi par l'utilisateur, qui
reconstruit ou non selon le souhait de l'utilisateur les gradients intervenants
dans la convection-diffusion.\\
Les termes sans accolade sont, eux, compl\`etement explicites et ajout\'es au fur et
\`a mesure dans \var{SMBR} pour former
l'expression $f^{\,exp}_s$ de \fort{codits}.
\end{itemize}
On d\'ecrit ci-dessous les \'etapes de \fort{resrij} :
\begin{itemize}

\item DELTIJ = 1, pour $\var{ISOU} \leqslant 3$ et DELTIJ = 0  Si $\var{ISOU} >
3$. Cette valeur repr\'esente le symbole de Kroeneker $\delta_{ij}$.

\item Initialisation \`a z\'ero de \var{SMBR} (tableau contenant le second
membre) et \var{ROVSDT} (tableau contenant la diagonale de la matrice sauf celle
relative \`a la contribution de la
diagonale des op\'erateurs de convection et de diffusion lin\'earis\'es
\footnote{qui correspondent aux sch\'emas convectif upwind pur et diffusif sans
reconstruction.}), tous deux de dimension $\var{NCEL}$.

\item Lecture et prise en compte des termes sources utilisateur pour la variable $R_{ij}$

Appel \`a \fort{cs\_user\_turbulence\_source\_terms} pour charger les termes sources utilisateurs. Ils sont
stock\'es comme suit. Pour la cellule $\Omega_l$ de centre $L$, repr\'esent\'ee par $\var{IEL}$, on a :\\
\begin{equation}\notag
\left\{\begin{array}{lll}
&\var{ROVSDT(IEL)}&= |\Omega_l| \ \alpha_{R_{ij}}\\
&\var{SMBR(IEL)}&=|\Omega_l| \ \beta_{R_{ij}}\\
\end{array}\right.
\end{equation}
On affecte alors les valeurs ad\'equates au second membre \var{SMBR} et \`a la
diagonale \var{ROVSDT} comme suit :
\begin{equation}\notag
\left\{\begin{array}{lll}
&\var{SMBR(IEL)} &= \var{SMBR(IEL)} +\ |\Omega_l| \ \alpha_{R_{ij}} \ (R^n_{ij})_L \\
&\var{ROVSDT(IEL)}&= \text{max }(-\ |\Omega_l| \ \alpha_{R_{ij}},0)\\
\end{array}\right.
\end{equation}
La valeur de $ \var{ROVSDT}$ est ainsi calcul\'ee pour des raisons de stabilit\'e
num\'erique. En effet, on ne rajoute sur la diagonale que les valeurs positives,
ce qui correspond physiquement \`a impliciter les termes de rappel uniquement.
\item{Calcul du terme source de masse  si $\Gamma_L > 0$}

Appel de \fort{catsma} et incr\'ementation si n\'ecessaire de \var{SMBR} et
\var{ROVSDT} {\it via} :\\
\begin{equation}\notag
\left\{\begin{array}{lll}
\displaystyle \var{SMBR(IEL)} = \var{SMBR(IEL)} + |\Omega_l| \ \Gamma_L \
\left[(R^{\,in}_{ij})_L - (R^{\,n}_{ij})_L \right] \\
\displaystyle \var{ROVSDT(IEL)}=\var{ROVSDT(IEL)} + |\Omega_l| \ \Gamma_L
\end{array}\right.
\end{equation}
\item Calcul du terme d'accumulation de masse et du terme instationnaire

On stocke $\displaystyle \var{W1}= \int_{\Omega_l}\dive\,(\rho\,\vect{u})\,d\Omega$
calcul\'e par \fort{divmas} \`a l'aide des flux de masse aux faces internes
$ m^n_{\,lm}=\sum\limits_{m\in Vois(l)}{(\rho \vect{u})_{\,lm}^n} \text{.}\,
\vect{S}_{\,lm} $ (tableau \var{FLUMAS}) et des flux de masse aux bords  $ m^n_{\,b_{lk}} = \sum\limits_{k\in{\gamma_b(l)}}{(\rho \vect{u})_{\,{b}_{lk}}^n} \text{.}\,
\vect{S}_{\,{b}_{lk}} $ (tableau \var{FLUMAB}).
On incr\'emente ensuite \var{SMBR} et \var{ROVSDT}.
\begin{equation}\notag
\left\{\begin{array}{lll}
&\var{SMBR(IEL)} &= \var{SMBR(IEL)} + \var{ICONV}\  (R^n_{ij})_L\,(\displaystyle
\int_{\Omega_l}\dive\,(\rho\,\vect{u})\ d\Omega) \\
&\var{ROVSDT(IEL)}& = \var{ROVSDT(IEL)} +  \var{ISTAT}\,\displaystyle
\frac{\rho^n_L \ |\Omega_l|}{\Delta t^n} -  \var{ICONV}\ (\displaystyle
\int_{\Omega_l}\dive\,(\rho\,\vect{u})\ d\Omega) \\
\end{array}\right.
\end{equation}
\item Calcul des termes sources de production, des termes $\displaystyle
\phi_{\,ij,1}+\phi_{\,ij,2}$ et de la dissipation~$\displaystyle-\frac{2}{3} \varepsilon\,\delta_{\,ij}$ :

On effectue une boucle d'indice \var{IEL} sur les cellules $\Omega_l$ de centre $L$ :
\begin{itemize}
\item [$\Rightarrow$] $\displaystyle \var{TRPROD}= \frac{1}{2} (\mathcal{P}^n_{ii})_L = \frac{1}{2} \left[ \var{PRODUC(1,IEL)} +  \var{PRODUC(2,IEL)} +  \var{PRODUC(3,IEL)} \right] $
\item [$\Rightarrow$] $\displaystyle \var{TRRIJ }= \frac{1}{2} (R^n_{ii})_L $
\item [$\Rightarrow$] $\displaystyle \var{SMBR(IEL)} =\ \var{SMBR(IEL)}\ +$\\
$\ \displaystyle\rho^n_L |\Omega_l| \left[ \displaystyle
\frac{2}{3}\,\delta_{\,ij} \left( \ \displaystyle \frac{ C_2}{2}\,(\mathcal{P}^n_{ii})_L\ +
(C_1-1)\ \varepsilon^n_L\, \right)\right.$\\
$ + \left.\ (1-C_2) \ \var{PRODUC(ISOU,IEL)} -
\displaystyle C_1\ \frac{2\,\varepsilon^n_L}{(R^n_{ii})_L}\ (R^n_{ij})_L \right]$
\item [$\Rightarrow$] $\displaystyle \var{ROVSDT(IEL)} = \var{ROVSDT(IEL)} +
\rho^n_L \ |\Omega_l| \ (- \displaystyle \frac{1}{3} \ \,\delta_{\,ij} + 1) \ C_1
\ \frac{2\ \varepsilon^n_L}{(R^n_{ii})_L}$
\end{itemize}
\item Appel de \fort{rijech} pour le calcul des termes d'\'echo de paroi
 $\phi^n_{ij,w}$ si $\var{IRIJEC()}=1$ et ajout dans \var{SMBR}.\\
$\var{SMBR} = \var{SMBR} + \phi^n_{ij,w}$\\
Suivant son mode de calcul (\var{ICDPAR}), la distance \`a la paroi est directement accessible
par \var{RA(IDIPAR+IEL-1)} (\var{|ICDPAR|} = 1) ou bien
est calcul\'ee \`a partir de $\var{IA(IIFAPA+IEL - 1)}$,
qui donne pour l'\'el\'ement $\var{IEL}$ le num\'ero de la face de bord
paroi la plus  proche (\var{|ICDPAR|} = 2). Ces tableaux ont \'et\'e renseign\'e une fois pour toutes au
d\'ebut de calcul.

\item  Appel de \fort{rijthe} pour le calcul des termes de gravit\'e $\mathcal{G}^n_{ij}$ et ajout dans \var{SMBR}.

Ce calcul n'a lieu que si $\var{IGRARI()} = 1$.
$ \var{SMBR} = \var{SMBR} + \mathcal{G}^n_{ij}$
\item Calcul de la partie explicite du terme de diffusion
 $\dive{\,\left[\tens{A}\,\grad{R}^{\,n}_{ij}\right]}$, plus pr\'ecis\'ement
des contributions du terme extradiagonal pris aux faces purement internes
(remplissage du tableau \var{VISCF}), puis aux faces de bord (remplissage du
tableau \var{VISCB}).
\begin{itemize}
\item [$\star$] Appel de \fort{grdcel} pour le calcul du gradient de
$R^{\,n}_{ij}$ dans chaque direction. Ces gradients sont respectivement
stock\'es dans les tableaux de travail \var{W1}, \var{W2} et \var{W3}.

\item [$\star$] boucle d'indice \var{IEL} sur les cellules $\Omega_l$ de centre
$L$ pour le
calcul de $\tens{E}^n\,\grad{R}^{\,n}_{ij}$ aux cellules dans un premier temps :\\
\begin{itemize}
\item [$\Rightarrow$] $\displaystyle \var{TRRIJ}= \frac{1}{2} (R^{\,n}_{ii})_L $
\item [$\Rightarrow$] $\displaystyle \var{CSTRIJ} = \rho^n_L\ C_S \ \displaystyle\frac{(R^n_{ii})_L}{2\,\varepsilon^n_L}$
\item [$\Rightarrow$] $\displaystyle \var{W4(IEL)} = \rho^n_L\ C_S\
\displaystyle\frac{(R^n_{ii})_L}{2\,\varepsilon^n_L} \left[\,(R^{\,n}_{12})_L \ \var{W2(IEL)} +
(R^{\,n}_{13})_L \ \var{W3(IEL)}\,\right]$
\item [$\Rightarrow$] $\displaystyle \var{W5(IEL)} = \rho^n_L\ C_S\
\displaystyle\frac{(R^n_{ii})_L}{2\,\varepsilon^n_L} \left[\,(R^{\,n}_{12})_L \ \var{W1(IEL)} +
(R^{\,n}_{23})_L \ \var{W3(IEL)}\,\right]$
\item [$\Rightarrow$] $\displaystyle \var{W6(IEL)} = \rho^n_L\ C_S\
\displaystyle\frac{(R^n_{ii})_L}{2\,\varepsilon^n_L} \left[\,(R^{\,n}_{13})_L \ \var{W1(IEL)} + (R^{\,n}_{23})_L \ \var{W2(IEL)}\,\right]$
\end{itemize}



\item [$\star$] Appel de \fort{vectds}\footnote{Le r\'esultat est stock\'e dans
\var{VISCF} et \var{VISCB}. Dans \fort{vectds}, les valeurs aux faces internes
sont interpol\'ees lin\'eairement sans reconstruction et \var{VISCB} est mis \`a
z\'ero.} pour assembler $\displaystyle\left[ \tens{E}^n\,\grad{R}^{\,n}_{ij}
\right]\,.\,\vect{n}_{\,lm}S_{\,lm}$ aux faces $lm$.
\item [$\star$] Appel de \fort{divmas} pour calculer la divergence du flux d\'efini par \var{VISCF} et \var{VISCB}.
Le r\'esultat est stock\'e dans \var{W4}.\\
Ajout au second membre \var{SMBR}.\\
\var{SMBR} = \var{SMBR} + \var{W4}
\end{itemize}

A l'issue de cette \'etape, seule la partie extradiagonale de la diffusion prise
enti\`erement explicite~:
 $$\sum\limits_{m\in
Vois(l)}\left[\ \tens{E}^n\,\grad{R}^{\,n}_{ij} \right]_{\,lm}\,.\,\vect{n}_{\,lm}S_{\,lm}$$ a \'et\'e calcul\'ee.\\

\item Calcul de la partie diagonale du terme de diffusion\footnote{Seule la
composante normale  du  gradient de $R_{ij}$ aux faces sera implicite.} :\\
Comme on l'a d\'eja vu, une partie de cette contribution sera trait\'ee en
implicite (celle relative \`a la composante normale du gradient) et donc
ajout\'ee au second membre par \fort{bilsc2} ; l'autre
partie sera explicite et prise en compte dans $f_s^{\,exp}$.
\begin{itemize}
\item [$\star$] On effectue une boucle d'indice \var{IEL} sur les cellules
$\Omega_l$ de centre $L$ :
\begin{itemize}
\item [$\Rightarrow$] $\displaystyle \var{TRRIJ }= \frac{1}{2} (R^{\,n}_{ii})_L $
\item [$\Rightarrow$] $\displaystyle \var{CSTRIJ} = \rho^n_L \ C_S \ \frac{(R^{\,n}_{ii})_L}{2\,\varepsilon^n_L}$
\item [$\Rightarrow$] $\displaystyle \var{W4(IEL)} = \rho^n_L \ C_S \
\frac{(R^{\,n}_{ii})_L}{2\,\varepsilon^n_L} \ (R^{\,n}_{11})_L$
\item [$\Rightarrow$] $\displaystyle \var{W5(IEL)} = \rho^n_L \ C_S \ \frac{(R^{\,n}_{ii})_L}{2\,\varepsilon^n_L}\ (R^n_{22})_L$
\item [$\Rightarrow$] $\displaystyle \var{W6(IEL)} = \rho^n_L \ C_S \ \frac{(R^{\,n}_{ii})_L}{2\,\varepsilon^n_L} \ (R^n_{33})_L$
\end{itemize}

%\item Traitement du parall\'elisme et de la p\'eriodicit\'e.

\item [$\star$] On effectue une boucle d'indice \var{IFAC} sur les faces
purement internes $lm$ pour remplir le tableau \var{VISCF} :
\begin{itemize}
\item [$\Rightarrow$] Identification des cellules $\Omega_l$ et $\Omega_m$ de
centre respectif $L$ (variable \var{II}) et $M$ (variable \var{JJ}), se trouvant de chaque c\^ot\'e de la face
$lm$\footnote{La normale \`a la face est orient\'ee de L vers M.}.
\item [$\Rightarrow$] Calcul du carr\'e de la surface de la face. La valeur est
stock\'ee dans le tableau \var{SURFN2}.
\item [$\Rightarrow$] Interpolation du gradient de $R^{\,n}_{ij}$ \`a la face
$lm$ (gradient facette $\left[\grad{R}^{\,n}_{ij}\right]_{\,lm}$) :
\begin{equation}\notag
\left\{\begin{array}{ll}
\var{GRDPX} &= \displaystyle \frac{1}{2} \left(\var{W1(II)} + \var{W1(JJ)}
\right) \\
&\\
\var{GRDPY} &= \displaystyle \frac{1}{2} \left(\var{W2(II)} + \var{W2(JJ)} \right) \\
&\\
\var{GRDPZ} &= \displaystyle \frac{1}{2} \left(\var{W3(II)} + \var{W3(JJ)} \right)
\end{array}\right.
\end{equation}
\item [$\Rightarrow$] Calcul du gradient de $R^{\,n}_{ij}$ normal \`a la face
$lm$, $\left[\grad{R}^{\,n}_{ij}\right]_{\,lm}.\vect{n}_{\,lm}\,S_{\,lm}$.\\

$\displaystyle \var{GRDSN} =  \var{GRDPX} \ \var{SURFAC(1,IFAC)} + \var{GRDPY} \ \var{SURFAC(2,IFAC)} +  \var{GRDPZ} \ \var{SURFAC(3,IFAC)}$
$\var{SURFAC}$ est le vecteur surface de la face \var{IFAC}.


\item [$\Rightarrow$] calcul de
 $\left[\grad{R^{\,n}_{ij}} - (\grad
R^{\,n}_{ij}\,.\,\vect{n}_{\,lm})\vect{n}_{\,lm}\right]$, les vecteurs \'etant
calcul\'es \`a la face $lm$ :
\begin{equation}\notag
\left\{\begin{array}{lll}
&\displaystyle \var{GRDPX} &= \var{GRDPX} - \displaystyle\frac{\var{GRDSN}}{\var{SURFN2}} \ \var{SURFAC(1,IFAC)}\\
&&\\
&\displaystyle \var{GRDPY} &= \var{GRDPY} - \displaystyle\frac{\var{GRDSN}}{\var{SURFN2}} \ \var{SURFAC(2,IFAC)} \\
&&\\
&\displaystyle \var{GRDPZ} &= \var{GRDPZ} - \displaystyle\frac{\var{GRDSN}}{\var{SURFN2}} \ \var{SURFAC(3,IFAC)}
\end{array}\right.
\end{equation}
\item [$\Rightarrow$] finalisation du calcul de l'expression totalement
explicite
 $$\left[ \tens{D}^n\,\left( \grad{R^{\,n}_{ij}} - (\grad R^{\,n}_{ij}\,.\,\vect{n}_{\,lm})\,\vect{n}_{\,lm}\right) \right]\,.\,\vect{n}_{\,lm}$$
\begin{equation}\notag
\begin{array} {ll}
\displaystyle \var{VISCF} = &
 \displaystyle\frac{1}{2} (\ \var{W4(II)} +\ \var{W4(JJ)}) \ \var{GRDPX} \
\var{SURFAC(1,IFAC)})\ + \\
&\\
&  \displaystyle\frac{1}{2} (\ \var{W5(II)} +\ \var{W5(JJ)}) \ \var{GRDPY} \
\var{SURFAC(2,IFAC)})\ + \\
&\\
&  \displaystyle\frac{1}{2} (\ \var{W6(II)} +\ \var{W6(JJ)}) \ \var{GRDPZ} \ \var{SURFAC(3,IFAC)})
\end{array}
\end{equation}
\end{itemize}

\item [$\star$] Mise \`a z\'ero du tableau \var{VISCB}.

\item [$\star$] Appel de \fort{divmas} pour calculer la divergence de~:
 $$\tens{D}^{\,n}\,\left( \grad{R^{\,n}_{ij}} - (\grad R^{\,n}_{ij}\,.\,\vect{n}_{\,lm})\vect{n}_{\,lm}\right)$$ d\'efini aux faces dans \var{VISCF} et \var{VISCB}.

Le r\'esultat est stock\'e dans le tableau \var{W1}.\\
Ajout au second membre \var{SMBR}.\\
$\var{SMBR} = \var{SMBR} + \var{W1}$
\end{itemize}
\item Calcul de la viscosit\'e orthotrope $\gamma^n_{\,lm}$ \`a la face $lm$ de la variable principale
$R^{\,n}_{ij}$\\
Ce calcul permet au sous-programme \fort{codits} de compl\'eter le second membre
\var{SMBR} par :
\begin{equation}
\begin{array} {ll}
& \sum\limits_{m\in Vois(l)}
\mu^n_{\,lm}\,\left(\grad{R}^{\,n}_{ij}\,.\,\vect{n}_{\,lm}\right)S_{\,lm}
 + \sum\limits_{m\in Vois(l)} \left(\grad{R}^{\,n}_{ij}
\,.\,\vect{n}_{\,lm}\right)\left[\tens{D}^{\,n}\,\vect{n}_{\,lm}\right]_{\,lm}\,.\,\vect{n}_{\,lm}\
S_{\,lm}\\
& = \sum\limits_{m\in Vois(l)}(\,\mu^n_{\,lm}\, + \,\gamma^n_{\,lm}\,)\,\left(\grad{R}^{\,n}_{ij}\,.\,\vect{n}_{\,lm}\right)S_{\,lm}
\end{array}
\end{equation}
sans pr\'eciser la nature de la face $lm$, {\it via} l'appel \`a \fort{bilsc2}  et de disposer de la quantit\'e
$(\mu^n_{\,lm}\, + \gamma^n_{\,lm})$ pour construire sa
matrice simplifi\'ee.\\
\begin{itemize}
\item [$\star$] On effectue une boucle d'indice \var{IEL} sur les cellules
$\Omega_l$ :
\begin{itemize}
\item [$\Rightarrow$] $\displaystyle \var{TRRIJ }= \frac{1}{2} (R^{\,n}_{ii})_L $
\item [$\Rightarrow$] $\displaystyle \var{RCSTE} = \rho^n_L \ C_S \ \frac{ (R^{\,n}_{ii})_L}{2\,\varepsilon^n_L} $
\item [$\Rightarrow$] $\displaystyle \var{W1(IEL)} = \mu^n +\rho^n_L \ C_S \ \frac{
(R^{\,n}_{ii})_L}{2\,\varepsilon^n_L}\ (R^n_{11})_L$
\item [$\Rightarrow$] $\displaystyle \var{W2(IEL)} = \mu^n + \rho^n_L \ C_S \ \frac{ (R^{\,n}_{ii})_L}{2\,\varepsilon^n_L}\ (R^n_{22})_L$
\item [$\Rightarrow$] $\displaystyle \var{W3(IEL)} = \mu^n + \rho^n_L \ C_S \ \frac{ (R^{\,n}_{ii})_L}{2\,\varepsilon^n_L}\ (R^n_{33})_L$
\end{itemize}

\item [$\star$] Appel de \fort{visort} pour calculer la viscosit\'e orthotrope
\footnote{Comme dans le sous-programme \fort{viscfa}, on multiplie la viscosit\'e par
$\displaystyle \frac{S_{\,lm}}{\overline{L'M'}}$, o\`u $S_{\,lm}$ et
$\overline{L'M'}$ repr\'esentent respectivement la surface de la face $lm$ et la
mesure alg\'ebrique du segment reliant les projections des centres des cellules
voisines sur la normale \`a la face. On garde dans ce sous-programme  la possibilit\'e d'interpoler la viscosit\'e aux cellules lin\'eairement ou d'utiliser une moyenne harmonique. La viscosit\'e au bord est celle de la cellule de bord correspondante.}
$ \gamma^n_{\,lm} = (\tens{D}^{\,n}\,\vect{n}_{\,lm}).\vect{n}_{\,lm}$ aux faces $lm$

Le r\'esultat est stock\'e dans les tableaux \var{VISCF} et \var{VISCB}.
\end{itemize}

\item appel de \fort{codits} pour la r\'esolution de l'\'equation de
convection/diffusion/termes sources de la variable $R_{ij}$. Le terme source est
\var{SMBR}, la viscosit\'e \var{VISCF} aux faces purement internes (
resp. \var{VISCB} aux faces de bord) et \var{FLUMAS} le flux de masse interne
 ( resp. \var{FLUMAB} flux de masse au bord). Le r\'esultat est la variable $R_{ij}$ au temps
$n+1$, donc $R^{\,n+1}_{ij}$. Elle est stock\'ee dans le tableau \var{RTP} des
variables mises \`a jour.

\end{itemize}

\etape{Appel de \fort{reseps} pour la r\'esolution de la variable $\varepsilon$}

On d\'ecrit ci-dessous le sous-programme \fort{reseps}, les commentaires du sous-programme \fort{resrij} ne seront pas r\'ep\'et\'es puisque les deux sous-programmes ne diff\`erent que par leurs termes sources.

\begin{itemize}
\item Initialisation \`a z\'ero de \var{SMBR} et \var{ROVSDT}.

\item{Lecture et prise en compte des termes sources utilisateur pour la variable $\varepsilon$ :}

Appel de \fort{cs\_user\_turbulence\_source\_terms} pour charger les termes sources utilisateurs. Ils sont
stock\'es dans les tableaux suivants :\\
pour la cellule $\Omega_l$ repr\'esent\'ee par $\var{IEL}$ de centre $L$, on a :
\begin{equation}\notag
\left\{\begin{array}{lll}
&\var{ROVSDT(IEL)}&= |\Omega_l| \ \alpha_{\varepsilon}\\
&\var{SMBR(IEL)}&=|\Omega_l| \ \beta_{\varepsilon}\\
\end{array}\right.
\end{equation}
On affecte alors les valeurs ad\'equates au second membre \var{SMBR} et \`a la
diagonale \var{ROVSDT} comme suit :
\begin{equation}\notag
\left\{\begin{array}{lll}
&\var{SMBR(IEL)} &= \var{SMBR(IEL)} +\ |\Omega_l| \ \alpha_{\,\varepsilon} \
\varepsilon^n_L \\
&\var{ROVSDT(IEL)}&= \text{max }(-\ |\Omega_l| \ \alpha_{\,\varepsilon},0)\\
\end{array}\right.
\end{equation}

\item{Calcul du terme source de masse si $\Gamma_L > 0$ :
\begin{equation}\notag
\left\{\begin{array}{lll}
&\displaystyle \var{SMBR(IEL)} = \var{SMBR(IEL)} + |\Omega_l| \ \Gamma_L \
(\varepsilon^{\,in}_L -\varepsilon^n_L) \\
&\displaystyle \var{ROVSDT(IEL)}= \var{ROVSDT(IEL)} + |\Omega_l| \ \Gamma_L
\end{array}\right.
\end{equation}
\item Calcul du terme d'accumulation de masse et du terme instationnaire \\
On stocke $\displaystyle \var{W1}= \int_{\Omega_l}\dive\,(\rho\,\vect{u})\,d\Omega$
calcul\'e par \fort{divmas} \`a l'aide des flux de masse internes et aux bords.\\
On incr\'emente ensuite \var{SMBR} et \var{ROVSDT}.
\begin{equation}\notag
\left\{\begin{array}{lll}
&\var{SMBR(IEL)} &= \var{SMBR(IEL)} + \var{ICONV}\ \varepsilon^n_L\,(\displaystyle
\int_{\Omega_l}\dive\,(\rho\,\vect{u})\ d\Omega) \\
&\var{ROVSDT(IEL)}& = \var{ROVSDT(IEL)} +  \var{ISTAT}\,\displaystyle
\frac{\rho^n_L \ |\Omega_l|}{\Delta t^n} -  \var{ICONV}\ (\displaystyle
\int_{\Omega_l}\dive\,(\rho\,\vect{u})\ d\Omega) \\
\end{array}\right.
\end{equation}

\item Traitement du terme de production
 $\displaystyle \rho\,C_{\varepsilon_1}\,\frac{\varepsilon}{k}\,\mathcal{P}$
 et du terme de dissipation $-\,\displaystyle \rho\,C_{\varepsilon_2}\,\frac{\varepsilon}{k}\,\varepsilon$ \\
pour cela, on effectue une boucle d'indice \var{IEL} sur les cellules $\Omega_l$
de centre $L$ :
\begin{itemize}
\item [$\Rightarrow$] $\displaystyle \var{TRPROD}= \frac{1}{2} (\mathcal{P}^n_{ii})_L = \frac{1}{2} \left[ \var{PRODUC(1,IEL)} +  \var{PRODUC(2,IEL)} +  \var{PRODUC(3,IEL)} \right] $
\item [$\Rightarrow$] $\displaystyle \var{TRRIJ }= \frac{1}{2} (R^n_{ii})_L $
\item [$\Rightarrow$] $\displaystyle \var{SMBR(IEL)} = \var{SMBR(IEL)} + \rho^n_L
|\Omega_l| \left[ -C_{\varepsilon_2} \ \frac{2\,(\varepsilon^n_L)^2}{(R^n_{ii})_L} + C_{\varepsilon_1} \ \frac{\varepsilon^n_L}{(R^n_{ii})_L}\ (\mathcal{P}^n_{ii})_L \right] $
\item [$\Rightarrow$] $\displaystyle \var{ROVSDT(IEL)} = \var{ROVSDT(IEL)} + C_{\varepsilon_2} \ \rho^n_L \ |\Omega_l| \ \frac{2\,\varepsilon^n_L}{(R^n_{ii})_L}$
\end{itemize}

\item Appel de \fort{rijthe} pour le calcul des termes de gravit\'e $\mathcal{G}^n_{\varepsilon}$ et ajout dans \var{SMBR}.

$ \var{SMBR} = \var{SMBR} + \mathcal{G}^n_{\varepsilon}$\\
Ce calcul n'a lieu que si $\var{IGRARI()} = 1$.

\item Calcul de la diffusion de $\varepsilon$ \\
 Le terme $\dive \left[\mu\, \grad(\varepsilon) + \tens{A'}\,\grad(\varepsilon)
\right]$ est calcul\'e exactement de la m\^eme mani\`ere que pour les tensions
de Reynolds $R_{ij}$ en rempla\c cant $\tens{A}$ par $\tens{A'}$.

\item Appel de \fort{codits} pour la r\'esolution de l'\'equation de
convection/diffusion/termes sources de la variable principale $\varepsilon$. Le
r\'esultat $\varepsilon^{\,n+1}$ est stock\'e dans le tableau \var{RTP} des
variables mises \`a jour.
}
\end{itemize}

\etape{clippings finaux}
On passe enfin dans le sous-programme  \fort{clprij} pour faire un clipping \'eventuel
des variables $R^{\,n+1}_{ij}$ et $\varepsilon^{\,n+1}$. Le sous-programme
\fort{clprij} est appel\'e\footnote{L'option
$\var{ICLIP} = 1$ consiste en un clipping minimal des variables $R_{ii}$ et
$\varepsilon$ en prenant la valeur absolue de ces variables puisqu'elles ne
peuvent \^etre que positives.} avec $\var{ICLIP} = 2$ . Cette option
\footnote{Quand la valeur des grandeurs $R_{ii}$ ou $\varepsilon$ est
n\'egative, on la remplace par le minimum entre sa valeur absolue et (1,1)
fois la valeur obtenue au pas de temps pr\'ec\'edent.} contient l'option $\var{ICLIP} = 1$  et permet de v\'erifier l'in\'egalit\'e de Cauchy-Schwarz sur les grandeurs extra-diagonales du tenseur $\tens{R}$ (pour $i \neq j$, $|R_{ij}|^2 \le R_{ii} R_{jj}$).


%%%%%%%%%%%%%%%%%%%%%%%%%%%%%%%%%%
%%%%%%%%%%%%%%%%%%%%%%%%%%%%%%%%%%
\section*{Points \`a traiter}
%%%%%%%%%%%%%%%%%%%%%%%%%%%%%%%%%%
%%%%%%%%%%%%%%%%%%%%%%%%%%%%%%%%%%
Sauf mention explicite, $\phi$ repr\'esentera une tension de Reynolds ou la dissipation turbulente ($\phi = R_{ij} \ \text{ou} \ \varepsilon$).

\begin{itemize}
\item {La vitesse utilis\'ee pour le calcul de la production est explicite. Est-ce qu'une implicitation peut am\'eliorer la pr\'ecision temporelle de $\phi$ \footnote{Cette remarque peut \^etre g\'en\'eralis\'ee. En effet, peut-on envisager d'actualiser les variables d\'ej\`a r\'esolues (sans r\'eactualiser les variables turbulentes apr\`es leur r\'esolution)? Ceci obligerait \`a modifier les sous-programmes tels que \fort{condli} qui sont appel\'es au d\'ebut de la boucle en temps.} ?}
\item {Dans quelle mesure le terme d'\'echo de paroi est-il valide ? En effet, ce terme est remis en question par certains auteurs.}
\item {On peut envisager la r\'esolution d'un syst\`eme hyperbolique pour les
tensions de Reynolds afin d'introduire un couplage avec le champ de vitesse.}
\item {Le flux au bord \var{VISCB} est annul\'e dans le sous-programme
\fort{vectds}. Peut-on envisager de mettre au bord la valeur de la variable
concern\'ee \`a la cellule de bord correspondant? De m\^eme, il faudrait se
pencher sur les hypoth\`eses sous-jacentes \`a l'annulation des contributions
aux bords de \var{VISCB} lors du calcul de : $$\left[ \tens{D}^n\,\left( \grad{R^{\,n}_{ij}} - (\grad R^{\,n}_{ij}\,.\,\vect{n}_{\,lm})\,\vect{n}_{\,lm}\right) \right]\,.\,\vect{n}_{\,lm}.$$}
\item {Un probl\`eme de pond\'eration appara\^\i t plus g\'en\'eralement. Si on prend la partie explicite de $\tens{D}\,\grad(\phi)$, la pond\'eration aux faces internes utilise le coefficient $\displaystyle\frac{1}{2}$ avec pond\'eration s\'epar\'ee de $\tens{D}$ et $\grad(\phi)$, alors que pour $\tens{E}\,\grad(\phi)$, on calcule d'abord ce terme aux cellules pour ensuite l'interpoler lin\'eairement aux faces \footnote{Cette interpolation se fait dans \fort{vectds} avec des coefficients de pond\'eration aux faces.}. Ceci donne donc deux types d'interpolations pour des termes de m\^eme nature.}
\item {On laisse la possibilit\'e dans \fort{visort} d'utiliser une moyenne
harmonique aux faces. Est-ce que ceci est valable puisque les interpolations
utilis\'ees lors du calcul de la partie explicite de $\tens{A}\,\grad{\phi}$
sont des moyennes arithm\'etiques ?}
\item {Les techniques adopt\'ees lors du clipping sont \`a revoir.}
\item {On utilise dans le cadre du mod\`ele $\displaystyle R_{ij}-\varepsilon $ une semi-implicitation de termes comme $\displaystyle \phi_{ij,1}$ ou $\displaystyle -\rho\,C_{\varepsilon_2}\,\frac{\varepsilon}{k}\,\varepsilon$. On peut envisager le m\^eme type d'implicitation dans \fort{turbke} m\^eme en pr\'esence du couplage $\displaystyle k-\varepsilon$.}
\item L'adoption d'une r\'esolution d\'ecoupl\'ee fait perdre l'invariance par rotation.
\item La formulation et l'implantation des conditions aux limites de paroi
devront \^etre v\'erifi\'ees. En effet, il semblerait que, dans certains cas, des ph\'enom\`enes
``oscillatoires'' apparaissent, sans qu'il soit ais\'e d'en d\'eterminer la cause.
\item L'implicitation partielle (du fait de la r\'esolution d\'ecoupl\'ee) des
conditions aux limites conduit souvent \`a des calculs instables. Il
conviendrait d'en conna\^\i tre la raison. L'implicitation partielle avait
\'et\'e mise en \oe uvre afin de tenter d'utiliser un pas de temps plus grand
dans le cas de jets axisym\'etriques en particulier.

\end{itemize}

%-------------------------------------------------------------------------------

% This file is part of Code_Saturne, a general-purpose CFD tool.
%
% Copyright (C) 1998-2020 EDF S.A.
%
% This program is free software; you can redistribute it and/or modify it under
% the terms of the GNU General Public License as published by the Free Software
% Foundation; either version 2 of the License, or (at your option) any later
% version.
%
% This program is distributed in the hope that it will be useful, but WITHOUT
% ANY WARRANTY; without even the implied warranty of MERCHANTABILITY or FITNESS
% FOR A PARTICULAR PURPOSE.  See the GNU General Public License for more
% details.
%
% You should have received a copy of the GNU General Public License along with
% this program; if not, write to the Free Software Foundation, Inc., 51 Franklin
% Street, Fifth Floor, Boston, MA 02110-1301, USA.

%-------------------------------------------------------------------------------

\programme{viscfa}
%

\hypertarget{viscfa}{}

\vspace{1cm}
%%%%%%%%%%%%%%%%%%%%%%%%%%%%%%%%%%
%%%%%%%%%%%%%%%%%%%%%%%%%%%%%%%%%%
\section*{Fonction}
%%%%%%%%%%%%%%%%%%%%%%%%%%%%%%%%%%
%%%%%%%%%%%%%%%%%%%%%%%%%%%%%%%%%%
Dans ce sous-programme est calcul\'e le coefficient de diffusion isotrope aux
faces. Ce coefficient fait intervenir la valeur de la viscosit� aux faces
multipli�e par le rapport surface de la face sur la distance alg\'ebrique $\overline{I'J'}$ ou $\overline{I'F}$({\it cf.} figure \ref{Base_Viscfa_fig_geom}),
rapport r�sultant de l'int�gration du terme de diffusion.
 Par analogie du terme calcul�, ce sous-programme est aussi appel� par le
sous-programme \fort{resopv} pour calculer le coefficient ``diffusif'' de la pression faisant intervenir le pas de temps.\\
La valeur de la viscosit� aux faces est d�termin�e soit par une moyenne
arithm�tique, soit par une moyenne harmonique de la viscosit� au centre des
cellules, suivant le choix de l'utilisateur. Par d�faut, cette valeur est calcul�e par une moyenne arithm�tique.

See the \doxygenanchor{cs__face__viscosity_8c.html\#viscfa}{programmers reference of the dedicated subroutine}
for further details.

%%%%%%%%%%%%%%%%%%%%%%%%%%%%%%%%%%
%%%%%%%%%%%%%%%%%%%%%%%%%%%%%%%%%%
\section*{Discr\'etisation} \label{Base_Viscfa_paragraphe2}
%%%%%%%%%%%%%%%%%%%%%%%%%%%%%%%%%%
%%%%%%%%%%%%%%%%%%%%%%%%%%%%%%%%%%
On rappelle dans la figure \ref{Base_Viscfa_fig_geom}, la d�finition des diff�rents
points g�om�triques utilis�s par la suite.

\begin{figure}[h]
\parbox{8cm}{%
\centerline{\includegraphics[height=4.5cm]{facette}}}
\parbox{8cm}{%
\centerline{\includegraphics[height=4.5cm]{facebord}}}
\caption{\label{Base_Viscfa_fig_geom}D\'efinition des diff\'erentes entit\'es
g\'eom\'etriques pour les faces internes (gauche) et de bord (droite).}
\end{figure}

L'int�gration du terme de diffusion sur une cellule $\Omega_i$ est la suivante :
\begin{equation}
\int_{\Omega_i}\dive (\mu\,\grad(f))\,d\Omega= \sum\limits_{j \in
Vois(i)}\mu_{\,ij} \frac{f_{J'}-f_{I'}}{\overline{I'J'}}\,.\, S_{\,ij} + \sum\limits_{k \in
\gamma_b(i)}\mu_{\,b_{ik}} \frac{f_{\,b_{ik}}- f_{I'}}{\overline{I'F}}\,.\,S_{\,b_{ik}}
\end{equation}
Dans ce sous-programme, on calcule les termes de diffusion
$\displaystyle \mu_{\,ij}\frac{S_{\,ij}}{\overline{I'J'}}$ et $\displaystyle
\mu_{\,b_{ik}}\,.\,\frac{S_{\,b_{ik}}}{\overline{I'F}}$.\\

La valeur de la viscosit� sur la face interne $ij$, $\mu_{\,ij}$, est calcul�e :\\
\hspace*{1.cm} {\tiny$\bigstar$}  soit par moyenne arithm�tique :
\begin{equation}
\mu_{\,ij}=\alpha_{\,ij}\mu_{\,i}+(1-\alpha_{\,ij})\mu_{\,j}
\end{equation}
avec $\alpha_{\,ij} = 0.5$ car ce choix semble stabiliser, bien que cette
interpolation soit d'ordre 1 en espace en convergence.\\
\hspace*{1.cm} {\tiny$\bigstar$} soit par moyenne harmonique :
\begin{equation}\notag
\mu_{\,ij}=\frac{\mu_{\,i}\ \mu_{\,j}}{\alpha_{\,ij}\mu_{\,i}+(1-\alpha_{\,ij})\mu_{\,j}}
\end{equation}
avec $\alpha_{\,ij}=\displaystyle \frac{\overline{FJ'}}{\overline{I'J'}}$.\\

La valeur de la viscosit� sur la face de bord $ik$, $\mu_{\,b_{ik}}$, est d�finie
ainsi :\\
\begin{equation}\notag
\mu_{\,b_{ik}}=\mu_I.
\end{equation}
\minititre{Remarque}
Lors de l'appel de \fort{viscfa} par le sous-programme \fort{resopv}, le terme
\`a consid\'erer est :
\begin{equation}\notag
\dive (\,\Delta t^n \ \grad(\delta p)\,)
\end{equation}
soit :
\begin{equation}\notag
\mu = \mu^n = \Delta t
\end{equation}

%%%%%%%%%%%%%%%%%%%%%%%%%%%%%%%%%%
%%%%%%%%%%%%%%%%%%%%%%%%%%%%%%%%%%
\section*{Mise en \oe uvre}
%%%%%%%%%%%%%%%%%%%%%%%%%%%%%%%%%%
%%%%%%%%%%%%%%%%%%%%%%%%%%%%%%%%%%
La valeur de la viscosit� au centre des cellules est entr�e en argument {\it
via} la variable \var{VISTOT}. On calcule sa valeur moyenne aux faces et on la
multiplie par le rapport surface \var{SURFN} sur la distance alg\'ebrique
\var{DIST} pour une face interne (\var{SURFBN} et \var{DISTBR} respectivement
pour une face de bord). La valeur du terme de diffusion r�sultant est mise dans le vecteur \var{VISCF} pour une face interne et \var{VISCB} pour une face de bord.\\
La variable \var{IMVISF} d�termine quel type de moyenne est utilis� pour
calculer la viscosit� aux faces.\\
Si \var{IMVISF}$=0$, la moyenne est arithm�tique, sinon la moyenne est harmonique.\\

%%%%%%%%%%%%%%%%%%%%%%%%%%%%%%%%%%
%%%%%%%%%%%%%%%%%%%%%%%%%%%%%%%%%%
\section*{Points \`a traiter}
%%%%%%%%%%%%%%%%%%%%%%%%%%%%%%%%%%
%%%%%%%%%%%%%%%%%%%%%%%%%%%%%%%%%%
L'obtention des interpolations utilis�es dans le code \CS  du paragraphe \ref{Base_Viscfa_paragraphe2} est r�sum�e dans le rapport de Davroux et
al\footnote{Davroux A., Archambeau F. et H�rard J.M., Tests num�riques sur
quelques m�thodes de r�solution d'une �quation de diffusion en volumes finis,
HI-83/00/027/A.}.
Les auteurs de ce rapport ont montr� que, pour un maillage monodimensionnel irr�gulier et avec une
viscosit� non constante, la convergence mesur�e est d'ordre 2 en espace avec
l'interpolation harmonique et d'ordre 1 en espace avec l'interpolation
lin�aire (pour des solution r�guli�res).\\
Par cons�quent, il serait pr�f�rable d'utiliser l'interpolation harmonique pour
calculer la valeur de la viscosit� aux faces. Des tests de stabilit� seront n�cessaires au pr�alable.
\\
De m�me, on envisage d'extrapoler la viscosit� sur les faces de bord plut�t que
de prendre la valeur de la viscosit� au centre de la cellule jouxtant cette face.\\
Dans le cas de la moyenne arithm\'etique, l'utilisation de la valeur $0.5$ pour les coefficients $\alpha_{\,ij}$ serait \`a revoir.

%-------------------------------------------------------------------------------

% This file is part of Code_Saturne, a general-purpose CFD tool.
%
% Copyright (C) 1998-2020 EDF S.A.
%
% This program is free software; you can redistribute it and/or modify it under
% the terms of the GNU General Public License as published by the Free Software
% Foundation; either version 2 of the License, or (at your option) any later
% version.
%
% This program is distributed in the hope that it will be useful, but WITHOUT
% ANY WARRANTY; without even the implied warranty of MERCHANTABILITY or FITNESS
% FOR A PARTICULAR PURPOSE.  See the GNU General Public License for more
% details.
%
% You should have received a copy of the GNU General Public License along with
% this program; if not, write to the Free Software Foundation, Inc., 51 Franklin
% Street, Fifth Floor, Boston, MA 02110-1301, USA.

%-------------------------------------------------------------------------------

\programme{visort}
%

\hypertarget{visort}{}

\vspace{1cm}
%%%%%%%%%%%%%%%%%%%%%%%%%%%%%%%%%%
%%%%%%%%%%%%%%%%%%%%%%%%%%%%%%%%%%
\section*{Fonction}
%%%%%%%%%%%%%%%%%%%%%%%%%%%%%%%%%%
%%%%%%%%%%%%%%%%%%%%%%%%%%%%%%%%%%
Dans ce sous-programme est calcul\'e le coefficient de diffusion ``orthotrope'' aux faces. Ce type de coefficient se rencontre pour la diffusion de $R_{\,ij}$ et
$\varepsilon$ en $R_{\,ij}-\varepsilon$ ( {\it cf.} \fort{turrij}), ainsi que pour la
correction de pression dans le cadre de l'algorithme avec couplage
vitesse-pression renforc� (\fort{resopv}).\\
Ce coefficient fait intervenir la valeur de la viscosit� aux faces multipli�e par
le rapport surface de la face sur la distance alg\'ebrique $\overline{I'J'}$,
rapport r�sultant de l'int�gration du terme de diffusion.
La valeur de la viscosit� aux faces est bas�e soit sur une moyenne
arithm�tique, soit sur une moyenne harmonique de la viscosit� au centre des
cellules.

See the \doxygenfile{visort_8f90.html}{programmers reference of the dedicated subroutine} for further details.

%%%%%%%%%%%%%%%%%%%%%%%%%%%%%%%%%%
%%%%%%%%%%%%%%%%%%%%%%%%%%%%%%%%%%
\section*{Discr\'etisation} \label{Base_Visort_paragraphe2}
%%%%%%%%%%%%%%%%%%%%%%%%%%%%%%%%%%
%%%%%%%%%%%%%%%%%%%%%%%%%%%%%%%%%%
La figure \ref{Base_Visort_fig_geom} rappelle les diverses d\'efinitions g\'eom\'etriques
pour les faces internes et les faces de bord.

\begin{figure}[h]
\parbox{8cm}{%
\centerline{\includegraphics[height=4.5cm]{facette}}}
\parbox{8cm}{%
\centerline{\includegraphics[height=4.5cm]{facebord}}}
\caption{\label{Base_Visort_fig_geom}D\'efinition des diff\'erentes entit\'es
g\'eom\'etriques pour les faces internes (gauche) et de bord (droite).}
\end{figure}
L'int�gration du terme de diffusion ``orthotrope'' sur une cellule est la
suivante :
\begin{equation}
\int_{\Omega_i}\dive \,(\tens{\mu}\ \grad f)\,d\Omega =\sum\limits_{j \in
Vois(i)}( \tens{\mu}\ \grad f)_{\,ij}\,.\,\underline{S}_{\,ij} + \sum\limits_{k \in
\gamma_b(i)}( \tens{\mu}\ \grad f)_{\,b_{ik}}\,.\,\underline{S}_{\,b_{ik}}
\end{equation}
avec :
\begin{equation}
\tens{\mu}=\begin{bmatrix}\mu_x & 0 & 0 \\ 0 & \mu_y & 0 \\ 0 & 0 & \mu_z \end{bmatrix}
\end{equation}
et :
\begin{equation}
\begin{array}{ll}
&\underline{S}_{\,ij} = S_{\,ij} \underline{n}_{\,ij} \\
&\underline{S}_{\,b_{ik}} = S_{\,b_{ik}} \underline{n}_{\,b_{ik}}
\end{array}
\end{equation}
Le terme $(\tens{\mu}\ \grad(f))_{\,ij}\underline{n}_{\,ij}$ est calcul� �
l'aide de la d�composition suivante :
\begin{equation}
(\tens{\mu}\ \grad f)_{\,ij} = (\grad f \,.\,\underline{n}_{\,ij})\ \tens{\mu}\
\underline{n}_{\,ij}+
(\grad f .\underline{\tau}_{ij})\ \tens{\mu}\ \underline{\tau}_{\,ij}
\end{equation}
o\`u $\underline{\tau}_{ij}$ repr�sente un vecteur tangent (unitaire) �
la face. Une d�composition similaire est utilis�e aux faces de bord.\\
Dans la matrice, seul le terme
$(\grad f \,.\,\underline{n}_{\,ij})\ \tens{\mu}\ \underline{n}_{\,ij}$ est
int\'egrable facilement en implicite. Par cons\'equent, la partie projet\'ee sur $\underline{\tau}_{\,ij}$
est :
\begin{itemize}
\item n\'eglig\'ee dans le cas du calcul des �chelles de temps relatives au
couplage vitesse-pression renforc�,
\item trait\'ee en explicite dans les termes de diffusion de
$R_{\,ij}-\varepsilon$ (\emph{cf.} \fort{turrij}).\\
\end{itemize}
L'int�gration implicite du terme de diffusion s'\'ecrit :
\begin{equation}
\int_{\Omega_i}\dive\,(\tens{\mu}\ \grad f )\,d\Omega = \sum\limits_{j \in
Vois(i)}(\tens{\mu}\ \underline{n}_{\,ij})\,.\,\underline{S}_{\,ij}\,
\frac{f_{J'}-f_{I'}}{\overline{I'J'}} + \sum\limits_{k \in
\gamma_b(i)}(\tens{\mu}\ \underline{n}_{\,b_{ik}})\,.\,\underline{S}_{\,b_{ik}}
\,\frac{f_{\,b_{ik}}-f_{I'}}{\overline{I'F}}
\end{equation}
Dans ce sous-programme, on calcule le terme
$\displaystyle \frac{(\tens{\mu}\
\underline{n}_{\,ij})\,.\,\underline{S}_{\,ij}}{\overline{I'J'}}$ \`a l'aide la
formule :
\begin{equation}\notag
(\tens{\mu}\ \underline{n}_{\,ij})\,.\,\underline{n}_{\,ij} =
\mu_{\,ij}^{\,moy}=\mu_{\,ij}^{\,x} ( n_{\,ij}^{\,x})^2 + \mu_{\,ij}^{\,y} (n_{\,ij}^{\,y})^2 + \mu_{\,ij}^{\,z}(n_{\,ij}^{\,z})^2
\end{equation}
soit encore :
\begin{equation}\notag
\mu_{\,ij}^{\,moy}=\frac{\mu_{\,ij}^{\,x}
(S_{\,ij}^{\,x})^2 + \mu_{\,ij}^{\,y} (S_{\,ij}^{\,y})^2 +
\mu_{\,ij}^{\,z} (S_{\,ij}^{\,z})^2}{S_{\,ij}^2}
\end{equation}
Au bord, on calcule de m�me :
\begin{equation}\notag
\displaystyle \frac{(\tens{\mu}\
\underline{n}_{\,b_{ik}})\,.\,\underline{S}_{\,b_{ik}}}{\overline{I'F}}
\end{equation}

 avec :
\begin{equation}\notag
(\tens{\mu}\ \underline{n}_{\,b_{ik}})\,.\,\underline{n}_{\,b_{ik}} =
\mu_{\,b_{ik}}^{\,moy} = \displaystyle \frac{\mu_{I}^{\,x}
(S_{\,b_{ik}}^{\,x})^2 + \mu_{I}^{\,y} (S_{\,b_{ik}}^{\,y})^2 +
\mu_{I}^{\,z} (S_{\,b_{ik}}^{\,z})^2}{S_{\,b_{ik}}^2}
\end{equation}

La valeur de la viscosit� dans une direction $l$ sur la face, $\mu_{\,ij}^{\,l}$,
est calcul�e :
\begin{itemize}
\item soit par interpolation lin\'eaire :
\begin{equation}
\mu_{\,ij}^{\,l}=\alpha_{\,ij}\mu_{i}^{\,l}+(1-\alpha_{\,ij})\mu_{j}^{\,l}
\end{equation}
avec $\alpha_{\,ij}= 0.5$ car ce choix semble stabiliser bien que cette
interpolation soit d'ordre 1 en espace en convergence,
\item soit par interpolation harmonique :
\begin{equation}\notag
\mu_{\,ij}^{\,l}=\displaystyle
\frac{\mu_{i}^{\,l}\ \mu_{j}^{\,l}}{\alpha_{\,ij}\mu_{i}^{\,l}+(1-\alpha_{\,ij}) \mu_{j}^{\,l}}
\end{equation}
o� :
\begin{equation}\notag
\displaystyle \alpha_{\,ij}=\frac{\overline{FJ'}}{\overline{I'J'}}
\end{equation}
\end{itemize}

%%%%%%%%%%%%%%%%%%%%%%%%%%%%%%%%%%
%%%%%%%%%%%%%%%%%%%%%%%%%%%%%%%%%%
\section*{Mise en \oe uvre}
%%%%%%%%%%%%%%%%%%%%%%%%%%%%%%%%%%
%%%%%%%%%%%%%%%%%%%%%%%%%%%%%%%%%%
La viscosit� orthotrope au centre des cellules est entr�e en argument {\it via}
les variables $\var{W}_1$, $\var{W}_2$ et $\var{W}_3$. On calcule la valeur
moyenne de chaque viscosit� aux faces de fa�on arithm�tique ou
harmonique. Ensuite, on calcule la viscosit� �quivalente correspondant �
$\displaystyle (\tens{\mu}\ \underline{n}_{\,ij})\,.\,\frac{\underline{S}_{\,ij}}{\overline{I'J'}}$ pour les
faces internes et � $\displaystyle (\tens{\mu}\ \underline{n}_{\,b_{ik}})\,.\,
\frac{\underline{S}_{\,b_{ik}}}{\overline{I'F}}$ pour les faces de bord.\\

Cette \'ecriture fait intervenir les vecteurs surface stock\'es dans le tableau
\var{SURFAC}, la norme de la surface \var{SURFN}
 et la distance alg\'ebrique \var{DIST} pour une face interne (\var{SURFBO},
\var{SURFBN} et \var{DISTBR} respectivement pour une face de bord). La valeur du
terme de diffusion r�sultant est mise dans le vecteur \var{VISCF} (\var{VISCB} aux faces de bord).\\
La variable \var{IMVISF} d�termine quel type de moyenne est utilis� pour
calculer la viscosit� dans une direction \`a la face. Si \var{IMVISF}$=0$, alors
la moyenne est arithm�tique, sinon la moyenne est harmonique).
%%%%%%%%%%%%%%%%%%%%%%%%%%%%%%%%%%
%%%%%%%%%%%%%%%%%%%%%%%%%%%%%%%%%%
\section*{Points \`a traiter}
%%%%%%%%%%%%%%%%%%%%%%%%%%%%%%%%%%
%%%%%%%%%%%%%%%%%%%%%%%%%%%%%%%%%%
L'obtention des interpolations utilis�es dans le code \CS \ du paragraphe
\ref{Base_Visort_paragraphe2} est r�sum�e dans le rapport de Davroux et al\footnote{Davroux A., Archambeau F. et H�rard J.M., Tests num�riques sur
quelques m�thodes de r�solution d'une �quation de diffusion en volumes finis,
HI-83/00/027/A.}.
Les auteurs de ce rapport ont montr� que, pour un maillage monodimensionnel irr�gulier et avec une
viscosit� non constante, la convergence mesur�e est d'ordre 2 en espace avec
l'interpolation harmonique et d'ordre 1 en espace avec l'interpolation
lin�aire (pour des solutions r�guli�res). Par cons�quent, il serait pr�f�rable d'utiliser l'interpolation
harmonique pour calculer la valeur de la viscosit� aux faces. Des tests de stabilit� seront n�cessaires au pr�alable.\\
De m�me, on envisage d'extrapoler la viscosit� sur les faces de bord plut�t que
de prendre la valeur de la viscosit� de la cellule jouxtant cette face.\\
Dans le cas de la moyenne arithm\'etique, l'utilisation de la valeur $0.5$ pour les coefficients $\alpha_{\,ij}$ serait \`a revoir.

\include{vissec}
%-------------------------------------------------------------------------------

% This file is part of Code_Saturne, a general-purpose CFD tool.
%
% Copyright (C) 1998-2020 EDF S.A.
%
% This program is free software; you can redistribute it and/or modify it under
% the terms of the GNU General Public License as published by the Free Software
% Foundation; either version 2 of the License, or (at your option) any later
% version.
%
% This program is distributed in the hope that it will be useful, but WITHOUT
% ANY WARRANTY; without even the implied warranty of MERCHANTABILITY or FITNESS
% FOR A PARTICULAR PURPOSE.  See the GNU General Public License for more
% details.
%
% You should have received a copy of the GNU General Public License along with
% this program; if not, write to the Free Software Foundation, Inc., 51 Franklin
% Street, Fifth Floor, Boston, MA 02110-1301, USA.

%-------------------------------------------------------------------------------

\programme{vortex}

\hypertarget{vortex}{}

\vspace{1cm}
%%%%%%%%%%%%%%%%%%%%%%%%%%%%%%%%%%
%%%%%%%%%%%%%%%%%%%%%%%%%%%%%%%%%%
\section*{Fonction}
%%%%%%%%%%%%%%%%%%%%%%%%%%%%%%%%%%
%%%%%%%%%%%%%%%%%%%%%%%%%%%%%%%%%%
Ce sous-programme est d�di� � la g�n�ration des conditions d'entr�e
turbulente utilis�es en LES.


La m�thode des vortex est bas�e sur une approche de tourbillons
ponctuels. L'id�e de la m�thode consiste � injecter des tourbillons 2D dans le
plan d'entr�e du calcul, puis � calculer le champ de vitesse induit par ces
tourbillons au centre des faces d'entr�e.

See the \doxygenfile{vortex_8f90.html}{programmers reference of the dedicated subroutine} for further details.

%%%%%%%%%%%%%%%%%%%%%%%%%%%%%%%%%
%%%%%%%%%%%%%%%%%%%%%%%%%%%%%%%%%%
\section*{Discr\'etisation}
%%%%%%%%%%%%%%%%%%%%%%%%%%%%%%%%%%
%%%%%%%%%%%%%%%%%%%%%%%%%%%%%%%%%%

Pour utiliser la m�thode, on se place tout d'abord dans un rep�re local d�fini
de mani�re � ce que le plan $(0yz)$, o� sont inject�s les vortex, soit confondu
avec le plan d'entr�e du calcul (voir figure \ref{Base_Vortex_entree}).

\begin{figure}[h]
\centerline{\includegraphics[height=6cm]{entree}}
\caption{\label{Base_Vortex_entree} D�finiton des diff�rentes grandeurs dans le rep�re local
correspondant � l'entr�e d'une conduite de section carr�e.}
\end{figure}

$u$, $v$ et $w$  sont les composantes de la vitesse fluctuante (principale et
transverse) dans ce plan, et
$\displaystyle \omega(y,z) = \frac{\partial w}{\partial y}-\frac{\partial v}{\partial z}$
la vorticit� dans la direction
normale au plan d'entr�e. $\overline{U}(y,z)$ repr�sente ici la vitesse
principale moyenne impos�e par l'utilisateur dans le plan d'entr�e.

Chaque vortex $p$ va �tre caract�ris� par sa fonction de forme $\xi$ (identique
pour tous les vortex), sa
circulation $\Gamma_p$, son rayon $\sigma_p$ et les coordonn�es $(y_p,z_p)$ du
point $P$ o� est situ� le vortex dans le plan $(0yz)$.

Pour cela, on suppose que la vorticit� g�n�r�e par un vortex $p$ au point $M$ de
coordonn�e $(y,z)$ s'�crit :
\begin{equation}\notag
\omega_p(y,z)= \Gamma_p \, \xi_{\sigma_p}(r)
\end{equation}
o� $r = \sqrt{(y-y_p)^2+(z-z_p)^2}$ est la distance s�parant le point $M$ du point $P$.

Dans la m�thode implant�e, la fonction de forme est de type gaussienne modifi�e :
\begin{equation}\notag
\displaystyle
\xi_\sigma (r) = \frac{1}{2\pi \sigma^2}
\left(2 e^{-\frac{r^2}{2\sigma^2}}-1\right) e^{-\frac{r^2}{2\sigma^2}}
\end{equation}

Le champ de vitesse induit par cette distribution de vorticit� s'obtient par
inversion des deux �quations de poisson suivantes qui sont d�duites de la
condition d'incompressibilit� dans la plan\footnote{\textit{i.e}
$\displaystyle \frac{\partial v}{\partial y}+\frac{\partial w}{\partial w} = 0$} :
\begin{equation}\notag
\begin{array}{lcr}
\displaystyle
\frac{\partial \omega}{\partial y} = \Delta w
&
\text{    et    }
&
\displaystyle
\frac{\partial \omega}{\partial y} = -\Delta v
\\
\end{array}
\end{equation}

Dans le cas g�n�ral, ce syst�me peut �tre int�gr� � l'aide de la formule de Biot et Savart.

Dans le cas d'une distribution de vorticit� de type gaussienne modifi�e, les
composantes de vitesse v�rifient :
\begin{equation}\notag
\left\{
\begin{array}{c}
\displaystyle
v_p(y,x) = - \frac{1}{2\pi} \frac{(z-z_p)}{r^2}\left(1 -
e^{-\frac{r^2}{2\sigma^2}}\right)\,e^{-\frac{r^2}{2\sigma^2}}
\\
\displaystyle
w_p(y,z) = \frac{1}{2\pi} \frac{(y-y_p)}{r^2}\left(1 -e^{-\frac{r^2}{2\sigma^2}}
\right)\,e^{-\frac{r^2}{2\sigma^2}}
\end{array}
\right.
\end{equation}

Ces relations s'�tendent de fa�on imm�diate au cas de $N$ vortex distincts.
Le champ de vitesse induit par la distribution de vorticit�
\begin{equation}
\omega(y,z) = \sum_{p=1}^N \Gamma_p \, \xi_{\sigma_p}(r)
\end{equation}
vaut au point $M$ :
\begin{equation}\notag
\begin{array}{lcr}
v(x,y) = \sum_{p=1}^N \Gamma_p\, v_p(y,z)
&
\text{    et    }
&
w(y,z) = \sum_{p=1}^N \Gamma_p\, w_p(y,z)
\\
\label{Base_Vortex_compvit}
\end{array}
\end{equation}
%================================
\subsection*{Param�tres physiques}
%================================

%-------------------------------
\subsubsection*{Marche en temps}
%-------------------------------
La position initiale de chaque vortex est tir�e de mani�re al�atoire. On calcul
les d�placements successifs de chacun des vortex dans le plan d'entr�e par
int�gration explicite du champ de vitesse lagrangien :
\begin{equation}\notag
\begin{array}{lcr}
\displaystyle
\frac{dy_p}{dt} = V(y,z)
&
\text{    et    }
&
\displaystyle
\frac{dz_p}{dt} = W(y,z)
\\
\end{array}
\end{equation}
Les vortex sont alors assimil�s � des particules ponctuelles qui sont convect�es
par le champ $(V,W)$. Ce champ peut �tre impos� par des tirages al�atoires ou
bien d�duit de la vitesse induite par les vortex dans le plan d'entr�e. Dans ce
cas $V(x,y) = \overline{V}(y,z) + v (y,z)$ et $W(y,z)= \overline{W}(y,z) +
w(y,z)$ o� $\overline{V}$ et $\overline{W}$ sont les composantes de la vitesse
transverse moyenne qu'impose l'utilisateur � l'aide des fichiers de donn�es.

%---------------------------------------------------
\subsubsection*{Intensit� et dur�e de vie des vortex}
%---------------------------------------------------
Il serait possible, � partir de l'�quation de transport de la vorticit�,
d'obtenir un mod�le d'�volution pour l'intensit� du vecteur tourbillon
$\omega_p$ associ� � chacun des vortex. En pratique, on pr�f�re utiliser un
mod�le simplifi� dans lequel la circulation des tourbillons ne d�pend que de la
postion de ces derniers � l'instant consid�r�. La circulation initiale de chaque
vortex est alors obtenue � partir de la relation :
\begin{equation}\notag
|\Gamma_p| = 4 \sqrt{\frac{\pi\,S\,k}{3N\,[2ln(3)-3ln(2)]}}
\end{equation}
o� $S$ est la surface du plan d'entr�e, $N$ le nombre de vortex, et $k$
l'�nergie cin�tique turbulente au point o� se trouve le vortex � l'instant
consid�r�. Le signe de $\Gamma_p$ correspond au sens de rotation du vortex et
est tir� al�atoirement.

Ce param�tre est celui qui contr�le l'intensit� des fluctuations. Sa d�pendance
en $k$ exprime que, plus l'�coulement est turbulent, plus les vortex sont
intenses. La valeur de $k$ est sp�cifi�e par
l'utilisateur. Elle peut �tre constante ou impos�e � partir de profils d'�nergie
cin�tique turbulente en entr�e.

Pour �viter que des structures trop allong�es ne se d�veloppent au niveau de
l'entr�e, l'utilisateur doit �galement sp�cifier un temps limites $\tau_p$ au
bout duquel le vortex $p$ va �tre d�truit. Ce temps $\tau_p$ peut �tre pris
constant ou estim� au moyen de la relation :
\begin{equation}\notag
\tau_p = \frac{5 C_{\mu}k^{\frac{3}{2}}}{\varepsilon\,\overline{U}}
\end{equation}

$\overline{U}$ et $\varepsilon$ repr�sentent respectivement la vitesse moyenne
principale et la dissipation turbulente au point o� est initialement g�n�r� le
vortex ($C_{\mu}=0,09$).
\\
Lorsque le temps �coul� depuis la cr�ation du vortex $p$ est sup�rieur �
$\tau_p$, le vortex est d�truit et un nouveau vortex g�n�r� (sa position et le
signe de $\Gamma_p$ sont tir�s de fa�on al�atoire).

%--------------------------------
\subsubsection*{Taille des vortex}
%--------------------------------
La taille des vortex peut �tre prise constante, ou calcul�e � partir des
relations :
\begin{equation}\notag
\begin{array}{ccc}
\displaystyle
\sigma = \frac{C_{\mu}^{\frac{3}{4}}k^{\frac{3}{2}}}{\varepsilon}
& \text{    ou    } &
\sigma = max[L_t,L_k]
\\
\end{array}
\end{equation}
avec:
\begin{equation}\notag
\begin{array}{ccc}
\displaystyle
L_t = \sqrt{\left( \frac{5 \nu k}{\varepsilon} \right)}
& \text{    et    } &
\displaystyle
L_k = 200\, \left(\frac{\nu^3}{\varepsilon}\right)^{\frac{1}{4}}
\end{array}
\end{equation}
o� $\nu$, $k$ et $\varepsilon$ sont la viscosit� dynamique, l'�nergie cin�tique
turbulente et la dissipation turbulente au point o� se trouve le vortex.

Dans tous les cas, la taille du vortex doit �tre sup�rieure � la taille des
mailles en entr�e afin que le vortex soit effectivement simul�.

%==================================
\subsection*{Conditions aux limites}
%==================================
Le champ de vitesse g�n�r� � l'aide de la relation \ref{Base_Vortex_compvit} ne tient pas
compte {\em a priori} des conditions aux limites appliqu�es sur les bords du plan
d'entr�e. Pour obtenir des valeurs de la vitesse qui soient coh�rentes sur les
fronti�res du domaine d'entr�e, des ``vortex images'', pseudo-vortex situ�s en
dehors du domaine d'entr�e, sont g�n�r�s � des positions particuli�res et leur
champ de vitesse associ� est superpos� au champ pr�c�demment calcul�.\\
Seuls les cas d'une conduite rectangulaire et d'une conduite circulaire
permettent la g�n�ration de ces pseudo-vortex.
On distingue pour cela trois types de conditions aux limites.

\begin{figure}[h]
\centerline{\includegraphics[height=6cm]{condlimite}}
\caption{\label{Base_Vortex_condli} Principe de g�n�ration des ``vortex images'' suivant le
type de conditions aux limites dans une conduite carr�e.}
\end{figure}

%----------------------------------
\subsubsection*{Condition de paroi}
%----------------------------------
On cr�e, pour chaque vortex $P$ contenu dans le plan d'entr�e, un vortex image
$P'$ identique � $P$ (\textit{i.e} de m�me caract�ristiques) et sym�trique de $P$
par rapport au
point $J$ ($J$ �tant la projection orthogonalement � la paroi du point $M$
correspondant au centre de la face o� l'on cherche � calculer la vitesse). La
figure \ref{Base_Vortex_condli} illustre la technique dans le cas d'une conduite
carr�e. Dans ce cas les coordonn�es du vortex situ� en $P'$ v�rifient
$(y_{p'}+y_{p})/2 = y_{J}$ et $(z_{p'}+ z_{p})/2 = z_{J}$. Le champ de vitesse
per�u depuis le point $M$ au niveau du point $J$ est nul, ce qui est bien
l'effet recherch�.

%------------------------------------
\subsubsection*{Condition de sym�trie}
%-------------------------------------
La technique est identique � celle utilis�e pour les conditions de paroi, mais
seule la composante pour la vitesse normale au bord est modifi�e dans ce cas.

%---------------------------------------
\subsubsection*{Condition de p�riodicit�}
%---------------------------------------
On cr�e pour chaque vortex, un vortex images $P'$ identique � $P$ mais translat�
d'une quantit� $L$ correspondant � la longueur qui s�pare les deux plans de la
section d'entr�e o� sont appliqu�es les conditions de p�riodicit�. Dans le cas
o� il y a deux directions de p�riodicit�, on cr�e deux vortex image.

%=============================================
\subsection*{Composante de vitesse principale}
%=============================================
La m�thode des vortex ne g�n�rant pas de fluctuation $u$ de la vitesse dans la
direction principale, la derni�re composante est calcul�e � partir d'une
�quation de Langevin. Les coefficients de cette �quation sont d�termin�s par
identification des expressions obtenues pour les contraintes de Reynolds en
$R_{ij}-\varepsilon$. Dans le cas d'un �coulement en canal plan, cette technique
conduit � l'�quation :
\begin{equation}\notag
\displaystyle
\frac{du}{dt} = - \frac{C_1}{2T} u + \left(\frac{2}{3}C_2-1\right)\frac{\partial
U}{\partial y} v + \sqrt{C_0\varepsilon} dW_i
\end{equation}

avec $\displaystyle T=\frac{k}{\varepsilon}$, $C_1 = 1,8$, $C_2 = 0,6$,
$C_0=\frac{14}{15}$, et $dW_i$ une variable al�toire Gaussienne de variance
$\sqrt{dt}$.

En pratique, l'�quation de Langevin n'am�liore pas vraiment les r�sultats. Elle
n'est utilis�e que dans le cas de conduites circulaires.

%%%%%%%%%%%%%%%%%%%%%%%%%%%%%%%%%%
%%%%%%%%%%%%%%%%%%%%%%%%%%%%%%%%%%
\section*{Mise en \oe uvre}
%%%%%%%%%%%%%%%%%%%%%%%%%%%%%%%%%%
%%%%%%%%%%%%%%%%%%%%%%%%%%%%%%%%%%


\begin{itemize}
\item[$\star$] Apr�s une �tape de pr�paration de la m�moire (\fort{memvor}), on
rep�re dans \fort{usvort} les faces d'entr�e pour lesquelles la m�thode va �tre
utilis�e.
\item[$\star$] V�rification des dimensions rentr�es (\fort{vervor}).
\\
\item[$\star$] Le sous-programme \fort{vorpre} se charge ensuite de pr�parer le
calcul (transmission de la g�om�trie des entr�es � tous les processeurs en cas
de parall�lisme, et construction d'un tableau de connectivit�). Le
sous-programme proc�de ainsi :
\\
\begin{itemize}

\item[$\bullet$] On compte, pour chaque entr�e \var{IENT}, le nombre de faces o�
est appliqu�e la m�thode. Celui-ci est stock� dans le tableau
\var{ICVOR(IENT)}. Un passage dans la sous-routine \fort{memvor} (avec
\var{IAPPEL = 2}) permet d'allouer la m�moire n�cessaire � cette phase de
pr�paration.

\item[$\bullet$] Pour chaque processeur, on stocke les coordonn�es des faces
d'entr�e rep�r�es pr�c�demment dans les tableaux de travail
\var{RA(IW1X),RA(IW1Y),RA(IW1Z),...}

\item[$\bullet$]  On regarde ensuite pour chaque processeur (boucle
\var{IPROC=1, NRANGP-1}), si le processeur \var{IPROC} a des donn�es � envoyer
aux autres processeurs (afin que tous disposent des coordonn�es).
\begin{itemize}
\item Si c'est le cas : \var{ICVOR(IENT)>0}, et on place les donn�es � envoyer
dans les tableaux de travail \var{RA(IW2X),RA(IW2Y),RA(IW2Z),...}. La valeur
\var{NCOMV = ICVOR(IENT)} correspond alors � la longueur des tableaux � envoyer.
\item Sinon, on ne fait rien et \var{NCOM=0}.
\end{itemize}
\item[$\bullet$] Le processeur num�ro \var{IPROC} distribue � tous les autres
processeurs la valeur \var{NCOM}. Si \var{NCOM > 0}, il envoie �galement les
donn�es contenues dans les tableaux de travails \var{RA(IW2X),...}. Ces donn�es
sont ensuite stock�es par tous les processeurs dans les tableaux
\var{RA(IXYZV+III),...} afin de lib�rer les tableaux de travail pour la
communication suivante, et l'indice \var{III = III + NCOM} est incr�ment� de
mani�re � ranger les valeurs de fa�on chronologique.
\\\\
$\rightarrow$ Au final de la boucle sur \var{IPROC}, chaque processeur dispose
des coordonn�es des faces d'entr�e pour lesquelles la m�thode va �tre utilis�e,
et il est donc simple de construire la connectivit�.
\\
\item[$\bullet$] Construction de la connectivit�. Au final, la vitesse au centre
de la \var{II} �me face d'entr�e utilisant la m�thode est contenue � la
\var{IA(IIFAGL+II)} �me ligne du tableau \var{RA(IUVORT)}.

\item[$\bullet$] La routine se termine par un appel au sous-programme
\fort{memvor} ( avec \var{IAPPEL = 3}) afin de r�server la m�moire utile � la
m�thode des vortex.
\end{itemize}
\end{itemize}
\bigskip

Cette phase d'initialisation est r�alis�e une seule fois au d�but du
calcul. C'est apr�s cette phase seulement que commence la m�thode des vortex
proprement dite.
\\
\begin{itemize}
\item[$\star$] Initialisation des variables avant intervention utilisateur (\fort{inivor}).
\item[$\star$] Appel au sous-programme utilisateur \fort{usvort} (\var{IAPPEL = 2}).
\item[$\star$] V�rification des param�tres rentr�s (\fort{vervor}).
\item[$\star$] Calcul de la vitesse par la m�thode des vortex (\fort{vortex})
\begin{itemize}
\item[$\bullet$] Initialisation du calcul g�n�ration du champ initial par appel
au sous-programme \fort{vorini} :

\begin{itemize}
\item Construction du rep�re local (et calcul de l'�quation du plan d'entr�e
suivant les cas), localisation du centre de l'entr�e, et transformation des
coordonn�es de l'entr�e dans le rep�re local. Les tableaux \var{YZCEL(II,1)} et
\var{YZCEL(II,2)} contiennent les coordonn�es des faces du plan d'entr�e une
fois ramen�es dans le rep�re $(0yz)$ (\var{II} est compris entre 1 et
\var{NCEVOR} o� \var{NCEVOR}=\var{ICVOR} repr�sente le nombre de faces pour
lesquelles la m�thode va �tre utilis�e a cette entr�e).
\item Lecture du fichier de donn�es, et initialisation des tableaux \var{XDAT},
\var{YDAT}, \var{UDAT}, \var{VDAT}, \var{WDAT}, \var{DUYDAT}, \var{KDAT},
\var{EPSDAT}, ...
\item Si on ne fait pas de suite (\var{ISUIVO=0}) ou que l'on r�initialise le
calcul (\var{INITVO=1}), tirage al�atoire de la position des vortex et de leur
sens de rotation, ainsi que calcul de leur dur�e de vie. Les positions sont
stock�es dans les tableaux \var{YZVOR(IVOR,1)} et \var{YZVOR(IVOR,2)}
(\var{IVOR} d�signant le num�ro du vortex).
\item Stockage de la vitesse principale moyenne au centre de la cellule dans le
tableau \var{XU}, et recherche pour chaque vortex, de la face d'entr�e qui lui
est la plus proche.
\end{itemize}

\item[$\bullet$] D�placement des vortex par appel au sous-programme \fort{vordep} :
\begin{itemize}
\item Convection des vortex.
\item Traitement des conditions aux limites. Les vortex qui sortent du domaine
de calcul sont replac�s � leur position d'origine.
\item R�g�n�ration des vortex ``morts''. Si le temps de vie cumul�
\var{TEMPS(II)} du vortex \var{II} est sup�rieur � sont temps de vie limite
\var{TPSLIM(II)}, alors le vortex est d�truit, et un nouveau vortex est g�n�r�.
\item Recherche pour chaque vortex de la face d'entr�e qui lui est la plus
proche apr�s d�placement (mise � jour du tableau \var{IVORCE}).
\end{itemize}

\item[$\bullet$] Calcul du champ de vitesse induit par appel au sous-programme \fort{vorvit} :
\begin{itemize}
\item Calcul de l'intensit� du vortex.
\item Calcul de la taille du vortex.
\item Calcul du champ de vitesse induit par l'ensemble des vortex au centre des
faces d'entr�e.
\item Traitement suivant les cas, des conditions de p�riodicit� de sym�trie et
des conditions de paroi par g�n�ration de vortex images.
\item Ajout de la vitesse moyenne dans les directions transverse aux tableaux
\var{XV} et \var{XW}.
\end{itemize}

\item[$\bullet$] G�n�ration des fluctuations de vitesse dans la direction
principale par appel au sous-programme \fort{vorlgv}.
\end{itemize}

\item[$\star$] appel au sous-programme \fort{vor2cl} :
\item[$\bullet$] Communication en cas de parall�lisme de la vitesse calcul�e en
entr�e par le processeur 0 aux autres processeurs.
\item[$\bullet$] Application des conditions aux limites apr�s utilisation d'un
changement de rep�re �ventuel.
\end{itemize}


%%%%%%%%%%%%%%%%%%%%%%%%%%%%%%%%%%
%%%%%%%%%%%%%%%%%%%%%%%%%%%%%%%%%%
\section*{Points \`a traiter}
%%%%%%%%%%%%%%%%%%%%%%%%%%%%%%%%%%
%%%%%%%%%%%%%%%%%%%%%%%%%%%%%%%%%%

Il serait possible de gagner de la m�moire en liberant l'espace alou� aux
tableaux \var{IW1X},...,\var{IW2V} apr�s le passage dans \fort{vorpre}.

\part{Module compressible}
%-------------------------------------------------------------------------------

% This file is part of Code_Saturne, a general-purpose CFD tool.
%
% Copyright (C) 1998-2018 EDF S.A.
%
% This program is free software; you can redistribute it and/or modify it under
% the terms of the GNU General Public License as published by the Free Software
% Foundation; either version 2 of the License, or (at your option) any later
% version.
%
% This program is distributed in the hope that it will be useful, but WITHOUT
% ANY WARRANTY; without even the implied warranty of MERCHANTABILITY or FITNESS
% FOR A PARTICULAR PURPOSE.  See the GNU General Public License for more
% details.
%
% You should have received a copy of the GNU General Public License along with
% this program; if not, write to the Free Software Foundation, Inc., 51 Franklin
% Street, Fifth Floor, Boston, MA 02110-1301, USA.

%-------------------------------------------------------------------------------

\programme{cfbl**}\label{ap:cfbase}

%%%%%%%%%%%%%%%%%%%%%%%%%%%%%%%%%%
%%%%%%%%%%%%%%%%%%%%%%%%%%%%%%%%%%
\section*{Fonction}
%%%%%%%%%%%%%%%%%%%%%%%%%%%%%%%%%%
%%%%%%%%%%%%%%%%%%%%%%%%%%%%%%%%%%


On s'int�resse � la r�solution des �quations de Navier-Stokes en compressible,
en particulier pour des configurations sans choc. Le sch�ma global correspond � une
extension des algorithmes volumes finis mis en \oe uvre pour simuler les
�quations de Navier-Stokes en incompressible.

Dans les grandes lignes, le sch\'ema est constitu\'e d'une \'etape
``acoustique'' fournissant la masse volumique (ainsi qu'une pr\'ediction de
pression et un d\'ebit acoustique), suivie de la r\'esolution de l'\'equation de
la quantit\'e de mouvement~; on r\'esout ensuite l'\'equation de l'\'energie
et, pour terminer, la pression est mise \`a jour.
Moyennant une contrainte sur la valeur du pas de temps, le sch\'ema permet
d'assurer la positivit\'e de la masse volumique.

La thermodynamique prise en compte \`a ce jour est celle des gaz parfaits, mais
l'organisation du code \`a \'et\'e  pr\'evue pour permettre \`a l'utilisateur de
fournir ses propres lois.

Pour compl�ter la pr�sentation, on pourra se reporter � la r�f�rence
suivante~: \\
\textbf{[Mathon]} P. Mathon, F. Archambeau, J.-M. H�rard : "Implantation d'un
algorithme compressible dans \CS", HI-83/03/016/A

Le cas de validation "tube \`a choc" de la version 1.2 de \CS permettra
\'egalement d'apporter quelques compl\'ements (tube \`a choc de Sod,
discontinuit\'e de contact instationnaire, double d\'etente sym\'etrique,
double choc sym\'etrique).

\newpage
%=================================
\subsection*{Notations}
%=================================

\begin{table}[h!]
\begin{tabular}{ccp{10,5cm}}

{\bf Symbole} & {\bf Unit\'e} & {\bf Signification}\\

\phantom{$C_v$, ${C_v}_i$} & \phantom{$\lbrack f\rbrack.\,kg/(m^3.\,s)$} & \\

$C_p$, ${C_p}_i$ & $J/(kg.\,K)$        & capacit� calorifique \`a pression constante
        $C_p = \left.\frac{\partial h}{\partial T}\right)_P$\\
$C_v$, ${C_v}_i$ & $J/(kg.\,K)$        & capacit� calorifique \`a volume constant
        $C_v = \left.\frac{\partial \varepsilon}{\partial T}\right)_\rho$\\
$\mathcal{D}_{f/b}$ & $m^2/s$         & diffusivit\'e mol\'eculaire du composant $f$
                                        dans le bain\\
$E$                 & $J/m^3$        & \'energie totale volumique $E = \rho e$\\
$F$                 &                  & centre de gravit\'e d'une face\\
$H$                 & $J/kg$         & enthalpie totale massique
                                        $H = \frac{E+P}{\rho}$\\
$I$                 &                  & point de co-location de la cellule $i$\\
$I'$                 &                  & pour une face $ij$ partag\'ee entre les
                                        cellules $i$ et $j$, $I'$
                                        est le projet\'e de $I$ sur la
                                        normale \`a la $ij$ passant
                                        par $F$, centre de $ij$\\
$K$                 & $kg/(m.\,s)$         & diffusivit\'e thermique\\
$M$, $M_i$         & $kg/mol$         & masse molaire ($M_i$ pour le constituant $i$)\\
$P$                 & $Pa$                 & pression\\
$\vect{Q}$         & $kg/(m^2.\,s)$& vecteur quantit\'e de mouvement
                                        $\vect{Q} = \rho\vect{u}$\\
$\vect{Q}_{ac}$ & $kg/(m^2.\,s)$& vecteur quantit\'e de mouvement issu
                                        de l'\'etape acoustique\\
$Q$                 & $kg/(m^2.\,s)$& norme de $\vect{Q}$\\
$R$                 & $J/(mol.\,K)$ & constante universelle des gaz parfaits\\
$S$                 & $J/(K.\,m^3)$        & entropie volumique\\
$\mathcal{S}$         & $\lbrack f\rbrack.\,kg/(m^3.\,s)$
                                & Terme de production/dissipation volumique
                                        pour le scalaire $f$\\
$T$                 & $K$                 & temp\'erature ($>0$)\\
$Y_i$                 &                 & fraction massique du compos\'e $i$
                                        ($0 \leqslant Y_i \leqslant 1$)\\
\end{tabular}
\end{table}

\clearpage

\begin{table}[htp]
\begin{tabular}{ccp{10,5cm}}

{\bf Symbole} & {\bf Unit\'e} & {\bf Signification}\\

\phantom{$C_v$, ${C_v}_i$} & \phantom{$\lbrack f\rbrack.\,kg/(m^3.\,s)$} & \\

$c^2$                 & $(m/s)^2$         & carr\'e de la vitesse du son
                $c^2 = \left.\frac{\partial P}{\partial \rho}\right)_s$\\
$e$                 & $J/kg$         & \'energie totale massique
                                        $e = \varepsilon + \frac{1}{2}u^2$\\
$\vect{f}_v$         & $N/kg$         & $\rho\vect{f}_v$ repr\'esente le terme
                                        source volumique pour la quantit\'e
                                        de mouvement~: gravit\'e, pertes
                                        de charges, tenseurs des contraintes
                                        turbulentes, forces de Laplace...\\
$\vect{g}$         & $m/s^2$         & acc\'el\'eration de la pesanteur\\
$h$                 & $J/kg$         & enthalpie massique
                                        $h=\varepsilon + \frac{P}{\rho}$\\
$i$                 &                  & indice faisant r\'ef\'erence \`a la
                                        cellule $i$~; $f_i$ est la valeur
                                        de la variable $f$ associ\'ee
                                        au point de co-location $I$\\
$I'$                 &                  & indice faisant r\'ef\'erence \`a la
                                        cellule $i$~; $f_I'$ est la valeur
                                        de la variable $f$ associ\'ee
                                        au point $I'$\\
$\vect{j}\wedge\vect{B}$
                & $N/m^3$         & forces de Laplace\\
$r$, $r_i$         & $J/(kg.\,K)$         & constante massique des gaz parfaits
                                        $r = \frac{R}{M}$
                                        (pour le constituant $i$, on a $r_i=\frac{R}{M_i}$)\\
$s$                 & $J/(K.\,kg)$         & entropie massique\\
$t$                 & $s$                 & temps\\
$\vect{u}$         & $m/s$         & vecteur vitesse\\
$u$                 & $m/s$         & norme de $\vect{u}$\\

\end{tabular}
\end{table}


\newpage

\begin{table}[htp]
\begin{tabular}{ccp{10,5cm}}

{\bf Symbole} & {\bf Unit\'e} & {\bf Signification}\\

\phantom{$C_v$, ${C_v}_i$} & \phantom{$\lbrack f\rbrack.\,kg/(m^3.\,s)$} & \\

$\beta$         & $kg/(m^3.\,K)$ &
        $\beta = \left.\frac{\partial P}{\partial s}\right)_\rho$\\
$\gamma$         & $kg/(m^3.\,K)$ & constante caract\'eristique
                                        d'un gaz parfait
                                        $\gamma = \frac{C_p}{C_v}$\\
$\varepsilon$         & $J/kg$         & \'energie interne massique\\
$\kappa$         & $kg/(m.\,s)$         & viscosit\'e dynamique en volume\\
$\lambda$         & $W/(m.\,K)$         & conductivit\'e thermique\\
$\mu$                 & $kg/(m.\,s)$         & viscosit\'e dynamique ordinaire\\
$\rho$                 & $kg/m^3$         & densit\'e\\
$\vect{\varphi}_f$
                & $\lbrack f\rbrack.\,kg/(m^2.\,s)$
                                & vecteur flux diffusif du compos\'e $f$\\
$\varphi_f$         & $\lbrack f\rbrack.\,kg/(m^2.\,s)$
                                & norme de $\vect{\varphi}_f$\\

\phantom{ouden}        &                 & \\

$\tens{\Sigma}^v$ &$kg/(m^2.\,s^2)$& tenseur des contraintes visqueuses\\
$\vect{\Phi}_s$ & $W/m^2$        & vecteur flux conductif de chaleur\\
$\Phi_s$         & $W/m^2$         & norme de $\vect{\Phi}_s$\\
$\Phi_v$         & $W/kg$         & $\rho\Phi_v$ repr\'esente le terme
                                        source volumique d'\'energie,
                                        comprenant par exemple
                                        l'effet Joule $\vect{j}\cdot\vect{E}$,
                                        le rayonnement...\\
\end{tabular}
\end{table}
\clearpage

%=================================
\subsection*{Syst\`eme d'\'equations laminaires de r\'ef\'erence}
%=================================

L'algorithme d\'evelopp\'e propose de r\'esoudre
l'\'equation de continuit\'e, les \'equations de Navier-Stokes
ainsi que l'\'equation d'\'energie totale de mani\`ere conservative,
pour des \'ecoulements compressibles.

\begin{equation}\label{Cfbl_Cfbase_eq_ref_laminaire_cfbase}
\left\{\begin{array}{l}

\displaystyle\frac{\partial\rho}{\partial t} + \divs(\vect{Q}) = 0 \\
\\
\displaystyle\frac{\partial\vect{Q}}{\partial t}
+ \divv(\vect{u} \otimes \vect{Q}) + \gradv{P}
= \rho \vect{f}_v + \divv(\tens{\Sigma}^v) \\
\\
\displaystyle\frac{\partial E}{\partial t} + \divs( \vect{u} (E+P) )
= \rho\vect{f}_v\cdot\vect{u} + \divs(\tens{\Sigma}^v \vect{u})
- \divs{\,\vect{\Phi}_s} + \rho\Phi_v

\end{array}\right.
\end{equation}

Nous avons pr�sent� ici le syst�me d'�quations laminaires, mais il faut pr�ciser
que la turbulence ne pose pas de probl�me particulier dans la mesure o� les
�quations suppl\'ementaires sont d�coupl�es du syst\`eme~(\ref{Cfbl_Cfbase_eq_ref_laminaire_cfbase}).

%=================================
\subsection*{Expression des termes intervenant dans les \'equations}
%=================================

\begin{itemize}

\item{\'Energie totale volumique :
        \begin{equation}
        E = \rho e = \rho\varepsilon + \frac{1}{2} \rho u^2
        \end{equation}
        avec l'\'energie interne $\varepsilon(P,\rho)$ donn\'ee par l'\'equation d'\'etat}
\\
\item{Forces volumiques : $\rho\vect{f}_v$ (dans la plupart des cas
                                            $\rho\vect{f}_v= \rho\vect{g}$)}
\\
\item{Tenseur des contraintes visqueuses pour un fluide Newtonien :
        \begin{equation}
        \tens{\Sigma}^v = \mu (\gradt{\vect{u}} +\ ^t\gradt{\vect{u}})
        + (\kappa - \frac{2}{3}\mu) \divs\,{\vect{u}}\ \tens{Id}
        \end{equation}
        avec $\mu(T,\ldots)$ et $\kappa(T,\ldots)$ mais souvent $\kappa =0$}
\\
\item{Flux de conduction de la chaleur : loi de Fourier
        \begin{equation}
        \vect{\Phi}_s = -\lambda \gradv{T}
        \end{equation}
        avec $\lambda(T,\ldots)$}
\\
\item{Source de chaleur volumique : $\rho\Phi_v$}

\end{itemize}


%=================================
\subsection*{\'Equations d'\'etat et expressions de l'\'energie interne}
\label{Cfbl_Cfbase_equations_etat_cfbase}
%=================================

%---------------------------------
\subsubsection*{Gaz parfait}
%---------------------------------

\'Equation d'\'etat : $P = \rho r T$\\


\'Energie interne massique :
$\varepsilon = \displaystyle\frac{P}{(\gamma -1) \rho}$

Soit~:
\begin{equation}\label{Cfbl_Cfbase_eq_pression_gp_cfbase}
P = (\gamma -1) \rho (e - \frac{1}{2} u^2)
\end{equation}


%---------------------------------
\subsubsection*{M\'elange de gaz parfaits}
%---------------------------------

On consid\`ere un m\'elange de $N$ constituants de fractions massiques
$(Y_i)_{i=1 \ldots N}$\\

\'Equation d'\'etat : $P = \rho\ r_{m\acute elange}\ T$\\

\'Energie interne massique :
$\varepsilon = \displaystyle\frac{P}{(\gamma_{m\acute elange} -1)\rho}$

Soit~:
\begin{equation}\label{Cfbl_Cfbase_eq_pression_melange_gp_cfbase}
P = (\gamma_{m\acute elange} -1) \rho (e - \frac{1}{2} u^2)
\end{equation}

avec $\gamma_{m\acute elange}
= \displaystyle\frac{\sum\limits_{i=1}^{N} {Y_i C_{pi}}}
{\sum\limits_{i=1}^{N} {Y_i C_{vi}}}$\ \
et\ \ $r_{m\acute elange} = \displaystyle\sum\limits_{i=1}^{N} {Y_i r_i}$


%---------------------------------
\subsubsection*{Equation d'\'etat de Van der Waals}
%---------------------------------

Cette \'equation est une correction de l'\'equation d'\'etat
des gaz parfaits pour tenir compte des forces intermol\'eculaires
et du volume des mol\'ecules constitutives du gaz.
On introduit deux coefficients correctifs~:
$a$ [$Pa.\,m^6 / kg^2$] est li\'e aux forces intermol\'eculaires
et $b$ [$m^3/kg$] est le covolume (volume occup\'e par les mol\'ecules).\\

\'Equation d'\'etat : $(P+a\rho^2)(1-b\rho) = \rho r T$\\

\'Energie interne massique :
$\varepsilon = \displaystyle\frac{(P+a\rho^2)(1-b\rho)}
{(\hat{\gamma} -1)\rho} - a \rho$

Soit~:
\begin{equation}\label{Cfbl_Cfbase_eq_pression_vdw_cfbase}
P = (\hat{\gamma} -1) \displaystyle\frac{\rho}{(1-b\rho)}
(e - \frac{1}{2} u^2 + a\rho) - a \rho^2
\end{equation}

avec $\hat{\gamma} = 1 + \displaystyle\frac{r}{C_v}
= \displaystyle\frac{C_p}{C_v}
\displaystyle\left(\frac{P-a\rho^2 (1-2b\rho)}{P+a\rho^2}\right)
+ \displaystyle\frac{2a\rho^2 (1-b\rho)}{P+a\rho^2}$



%=================================
\subsection*{Calcul des grandeurs thermodymamiques}
%=================================

%---------------------------------
\subsubsection*{Pour un gaz parfait \`a $\gamma$ constant}
%---------------------------------

%`````````````````````````````````
\paragraph{Equation d'\'etat~:}
%,,,,,,,,,,,,,,,,,,,,,,,,,,,,,,,,,

$P = \rho r T$

On suppose connues la chaleur massique \`a pression constante $C_p$
et la masse molaire $M$ du gaz, ainsi que les variables d'\'etat.

%`````````````````````````````````
\paragraph{Chaleur massique \`a volume constant~:}
%,,,,,,,,,,,,,,,,,,,,,,,,,,,,,,,,,

$C_v = C_p - \displaystyle\frac{R}{M} = C_p - r$


%`````````````````````````````````
\paragraph{Constante caract\'eristique du gaz~:}
%,,,,,,,,,,,,,,,,,,,,,,,,,,,,,,,,,

$\gamma = \displaystyle\frac{C_p}{C_v} = \displaystyle\frac{C_p}{C_p - r}$


%`````````````````````````````````
\paragraph{Vitesse du son~:}
%,,,,,,,,,,,,,,,,,,,,,,,,,,,,,,,,,

$c^2 = \gamma \displaystyle\frac{P}{\rho}$


%`````````````````````````````````
\paragraph{Entropie~:}
%,,,,,,,,,,,,,,,,,,,,,,,,,,,,,,,,,

$s = \displaystyle\frac{P}{\rho^{\gamma}}$
\quad et
$\beta = \left.\displaystyle\frac{\partial P}{\partial s}\right)_{\rho}
= \rho^{\gamma}$

\noindent\textit{Remarque~:} L'entropie choisie ici n'est pas l'entropie
physique, mais une entropie math\'ematique qui v\'erifie \quad
$c^2 \left.\displaystyle\frac{\partial s}{\partial P}\right)_{\rho}
+ \left.\displaystyle\frac{\partial s}{\partial \rho}\right)_{P} = 0$


%`````````````````````````````````
\paragraph{Pression~:}
%,,,,,,,,,,,,,,,,,,,,,,,,,,,,,,,,,

$P = (\gamma-1) \rho \varepsilon$


%`````````````````````````````````
\paragraph{Energie interne~:}
%,,,,,,,,,,,,,,,,,,,,,,,,,,,,,,,,,

$\varepsilon = C_v T
= \displaystyle\frac{1}{\gamma-1} \displaystyle\frac{P}{\rho}$\qquad\text{\ \ avec\ \ }
$\varepsilon_{sup} = 0$

%`````````````````````````````````
\paragraph{Enthalpie~:}
%,,,,,,,,,,,,,,,,,,,,,,,,,,,,,,,,,

$h = C_p T
= \displaystyle\frac{\gamma}{\gamma-1} \displaystyle\frac{P}{\rho}$


%---------------------------------
\subsubsection*{Pour un m\'elange de gaz parfaits}
%---------------------------------

Une intervention de l'utilisateur dans le sous-programme utilisateur
\fort{uscfth} est n�cessaire pour pouvoir utiliser ces lois.

%`````````````````````````````````
\paragraph{Equation d'\'etat~:}
%,,,,,,,,,,,,,,,,,,,,,,,,,,,,,,,,,

$P = \rho\ r_{m\acute el}\ T$
\quad avec $r_{m\acute el} = \displaystyle\sum\limits_{i=1}^{N} {Y_i r_i}
= \displaystyle\sum\limits_{i=1}^{N} Y_i \displaystyle\frac{R}{M_i}$


On suppose connues la chaleur massique \`a pression constante
des diff\'erents constituants ${C_p}_i$,
la masse molaire $M_i$ des constituants du gaz,
ainsi que les variables d'\'etat (dont les fractions massiques $Y_i$).

%`````````````````````````````````
\paragraph{Masse molaire du m\'elange~:}
%,,,,,,,,,,,,,,,,,,,,,,,,,,,,,,,,,

$M_{m\acute el} = \left(\displaystyle\sum\limits_{i=1}^{N}
\displaystyle\frac{Y_i}{M_i} \right)^{-1}$

%`````````````````````````````````
\paragraph{Chaleur massique \`a pression constante du m\'elange~:}
%,,,,,,,,,,,,,,,,,,,,,,,,,,,,,,,,,
$\\$
${C_p}_{m\acute el} = \displaystyle\sum\limits_{i=1}^{N} Y_i {C_p}_i$


%`````````````````````````````````
\paragraph{Chaleur massique \`a volume constant du m\'elange~:}
%,,,,,,,,,,,,,,,,,,,,,,,,,,,,,,,,,
$\\$
${C_v}_{m\acute el} = \displaystyle\sum\limits_{i=1}^{N} Y_i {C_v}_i
= {C_p}_{m\acute el} - \displaystyle\frac{R}{M_{m\acute el}}
= {C_p}_{m\acute el} - r_{m\acute el}$


%`````````````````````````````````
\paragraph{Constante caract\'eristique du gaz~:}
%,,,,,,,,,,,,,,,,,,,,,,,,,,,,,,,,,

$\gamma_{m\acute el} = \displaystyle\frac{{C_p}_{m\acute el}}
{{C_v}_{m\acute el}}
= \displaystyle\frac{{C_p}_{m\acute el}}{{C_p}_{m\acute el} - r_{m\acute el}}$


%`````````````````````````````````
\paragraph{Vitesse du son~:}
%,,,,,,,,,,,,,,,,,,,,,,,,,,,,,,,,,

$c^2 = \gamma_{m\acute el} \displaystyle\frac{P}{\rho}$


%`````````````````````````````````
\paragraph{Entropie~:}
%,,,,,,,,,,,,,,,,,,,,,,,,,,,,,,,,,

$s = \displaystyle\frac{P}{\rho^{\gamma_{m\acute el}}}$
\quad et
$\beta = \left.\displaystyle\frac{\partial P}{\partial s}\right)_{\rho}
= \rho^{\gamma_{m\acute el}}$


%`````````````````````````````````
\paragraph{Pression~:}
%,,,,,,,,,,,,,,,,,,,,,,,,,,,,,,,,,

$P = (\gamma_{m\acute el}-1) \rho \varepsilon$


%`````````````````````````````````
\paragraph{Energie interne~:}
%,,,,,,,,,,,,,,,,,,,,,,,,,,,,,,,,,

$\varepsilon = {C_v}_{m\acute el}\ T$\qquad\text{\ \ avec\ \ }
$\varepsilon_{sup} = 0$


%`````````````````````````````````
\paragraph{Enthalpie~:}
%,,,,,,,,,,,,,,,,,,,,,,,,,,,,,,,,,

$h = {C_p}_{m\acute el}\  T
= \displaystyle\frac{\gamma_{m\acute el}}{\gamma_{m\acute el}-1}
\displaystyle\frac{P}{\rho}$

%---------------------------------
\subsubsection*{Pour un gaz de Van der Waals}
%---------------------------------

Ces lois n'ont pas �t� programm�es, mais l'utilisateur peut intervenir
dans le sous-programme utilisateur \fort{uscfth} s'il souhaite le faire.

%`````````````````````````````````
\paragraph{Equation d'\'etat~:}
%,,,,,,,,,,,,,,,,,,,,,,,,,,,,,,,,,

$(P+a\rho^2)(1-b\rho) = \rho r T$

avec $a$ [$Pa.\,m^6 / kg^2$] li\'e aux forces intermol\'eculaires
et $b$ [$m^3/kg$] le covolume (volume occup\'e par les mol\'ecules).

On suppose connus les coefficients $a$ et $b$,
la chaleur massique \`a pression constante $C_p$,
la masse molaire $M$ du gaz et
les variables d'\'etat.


%`````````````````````````````````
\paragraph{Chaleur massique \`a volume constant~:}
%,,,,,,,,,,,,,,,,,,,,,,,,,,,,,,,,,

$C_v = C_p - r
\displaystyle\frac{P+a\rho^2}{P-a\rho^2 (1-2b\rho)}$


%`````````````````````````````````
\paragraph{Constante ``\'equivalente'' du gaz~:}
%,,,,,,,,,,,,,,,,,,,,,,,,,,,,,,,,,

$\hat{\gamma} = 1 + \displaystyle\frac{r}{C_v}
= \displaystyle\frac{C_p}{C_v}
\displaystyle\left(\frac{P-a\rho^2 (1-2b\rho)}{P+a\rho^2}\right)
+ \displaystyle\frac{2a\rho^2 (1-b\rho)}{P+a\rho^2}$

%`````````````````````````````````
\paragraph{Vitesse du son~:}
%,,,,,,,,,,,,,,,,,,,,,,,,,,,,,,,,,

$c^2 = \hat{\gamma} \displaystyle\frac{P+a\rho^2}{\rho(1-b\rho)} - 2a\rho$


%`````````````````````````````````
\paragraph{Entropie~:}
%,,,,,,,,,,,,,,,,,,,,,,,,,,,,,,,,,

$s = (P+a\rho^2)
\left(\displaystyle\frac{1-b\rho}{\rho}\right)^{\hat{\gamma}}$
\quad et
$\beta = \left.\displaystyle\frac{\partial P}{\partial s}\right)_{\rho}
= \left(\displaystyle\frac{\rho}{1-b\rho}\right)^{\hat{\gamma}}$


%`````````````````````````````````
\paragraph{Pression~:}
%,,,,,,,,,,,,,,,,,,,,,,,,,,,,,,,,,

$P = (\hat{\gamma} -1) \displaystyle\frac{\rho}{(1-b\rho)}
(\varepsilon + a\rho) - a \rho^2$

%`````````````````````````````````
\paragraph{Energie interne~:}
%,,,,,,,,,,,,,,,,,,,,,,,,,,,,,,,,,

$\varepsilon = C_v T - a \rho$\qquad\text{\ \ avec\ \ }
$\varepsilon_{sup} = - a \rho$


%`````````````````````````````````
\paragraph{Enthalpie~:}
%,,,,,,,,,,,,,,,,,,,,,,,,,,,,,,,,,

$h = \displaystyle\frac{\hat{\gamma}-b\rho}{\hat{\gamma}-1}
 \displaystyle\frac{P+a\rho^2}{\rho} - 2a\rho$


%=================================
\subsection*{Algorithme de base}
%=================================

On suppose connues toutes les variables au temps $t^n$ et on cherche
\`a les d\'eterminer \`a l'instant $t^{n+1}$.
On r\'esout en deux blocs principaux~: d'une part le syst\`eme masse-quantit\'e
de mouvement, de l'autre l'\'equation portant sur l'\'energie et les scalaires
transport\'es.
Dans le premier bloc, on distingue le traitement du syst\`eme (coupl\'e)
acoustique et le traitement de l'\'equation de la quantit\'e de mouvement
compl\`ete.

Au d\'ebut du pas de temps, on commence par mettre \`a jour
les propri\'et\'es physiques variables (par exemple $\mu(T)$, $\kappa(T)$,
$C_p(Y_1,\ldots ,Y_N)$ ou $\lambda(T)$), puis on
r\'esout les \'etapes suivantes~:

\begin{enumerate}

  \item {\bf Acoustique~: sous-programme \fort{cfmsvl}} \\
     R�solution d'une �quation de convection-diffusion portant sur $\rho^{n+1}$.\\
     On obtient � la fin de l'�tape $\rho^{n+1}$, $Q^{n+1}_{ac}$ et
�ventuellement une
pr\'ediction de la pression $P^{pred}(\rho^{n+1},e^{n})$.\\

  \item {\bf Quantit\'e de mouvement~: sous-programme \fort{cfqdmv}}\\
     R�solution d'une �quation de convection-diffusion portant sur $u^{n+1}$ qui
     fait intervenir  $Q^{n+1}_{ac}$ et $P^{pred}$.\\
     On obtient � la fin de l'�tape $u^{n+1}$.\\

  \item {\bf \'Energie totale~: sous-programme \fort{cfener}}\\
     R�solution d'une �quation de convection-diffusion portant sur $e^{n+1}$ qui
     fait intervenir  $Q^{n+1}_{ac}$, $P^{pred}$ et  $u^{n+1}$.\\
     On obtient \`a la fin de l'�tape $e^{n+1}$ et une valeur actualis�e de la
pression $P(\rho^{n+1},e^{n+1})$.\\

  \item {\bf Scalaires passifs}\\
     R�solution d'une �quation de convection-diffusion standard par
     scalaire, avec  $Q^{n+1}_{ac}$ pour flux convectif.
\end{enumerate}

%%%%%%%%%%%%%%%%%%%%%%%%%%%%%%%%%%
%%%%%%%%%%%%%%%%%%%%%%%%%%%%%%%%%%
\section*{Discr\'etisation}
%%%%%%%%%%%%%%%%%%%%%%%%%%%%%%%%%%
%%%%%%%%%%%%%%%%%%%%%%%%%%%%%%%%%%

On se reportera aux sections relatives aux sous-programmes
\fort{cfmsvl} (masse volumique), \fort{cfqdmv}
(quantit\'e de mouvement) et \fort{cfener} (\'energie).
La documentation du sous-programme
\fort{cfxtcl} fournit des \'el\'ements relatifs aux
conditions
aux limites.

%%%%%%%%%%%%%%%%%%%%%%%%%%%%%%%%%%
%%%%%%%%%%%%%%%%%%%%%%%%%%%%%%%%%%
\section*{Mise en \oe uvre}
%%%%%%%%%%%%%%%%%%%%%%%%%%%%%%%%%%
%%%%%%%%%%%%%%%%%%%%%%%%%%%%%%%%%%

Le module compressible est une ``physique particuli\`ere'' activ\'ee lorsque le
mot-cl\'e \var{IPPMOD(ICOMPF)} est positif ou nul.

Dans ce qui suit, on pr\'ecise les inconnues et les propri\'et\'es
principales utilis\'ees dans le module.
On fournit \'egalement un arbre d'appel simplifi\'e des sous-programmes du
module~: initialisation avec \fort{initi1} puis (\fort{iniva0} et) \fort{inivar} et
enfin, boucle en temps avec \fort{tridim}.


\subsection*{Inconnues et propri\'et\'es}

Les \var{NSCAPP} inconnues scalaires associ\'ees \`a la physique
particuli\`ere sont d\'efinies dans \fort{cfvarp} dans l'ordre
suivant~:
\begin{itemize}
\item la masse volumique \var{RTP(*,ISCA(IRHO))},
\item l'�nergie totale   \var{RTP(*,ISCA(IENERG))},
\item la temp�rature     \var{RTP(*,ISCA(ITEMPK))}
\end{itemize}

On souligne que la temp\'erature est d\'efinie en tant que variable ``\var{RTP}'' et
non pas en tant que propri\'et\'e physique ``\var{PROPCE}''. Ce choix a \'et\'e
motiv\'e par la volont\'e de simplifier la gestion des conditions aux limites,
au prix cependant d'un encombrement m\'emoire l\'eg\`erement sup\'erieur (une
grandeur \var{RTP} consomme plus qu'une grandeur \var{PROPCE}).

La pression et la vitesse sont classiquement associ\'ees aux tableaux suivants~:
\begin{itemize}
\item pression~: \var{RTP(*,IPR)}
\item vitesse~: \var{RTP(*,IU)}, \var{RTP(*,IV)}, \var{RTP(*,IW)}.
\end{itemize}


\bigskip
Outre les propri\'et\'es associ\'ees en standard aux variables
identifi\'ees ci-dessus, le
tableau \var{PROPCE} contient \'egalement~:
 \begin{itemize}
\item la chaleur massique � volume constant $C_v$, stock\'ee dans
\var{PROPCE(*,IPPROC(ICV))},
      si  l'utilisateur a indiqu� dans \fort{uscfth} qu'elle \'etait variable.
\item la viscosit� en volume \var{PROPCE(*,IPPROC(IVISCV))}
      si  l'utilisateur a indiqu� dans \fort{uscfx2} qu'elle \'etait variable.
\end{itemize}


\bigskip
Pour la gestion des conditions aux limites et en particulier pour le calcul du
flux convectif par le sch\'ema de Rusanov
aux entr\'ees et sorties (hormis en sortie supersonique), on
dispose des tableaux suivants dans  \var{PROPFB}~:
\begin{itemize}
\item flux convectif de quantit\'e de mouvement au bord pour les trois
composantes dans les tableaux
\var{PROPFB(*,IPPROB(IFBRHU))} (composante $x$),
\var{PROPFB(*,IPPROB(IFBRHV))} (composante $y$) et
\var{PROPFB(*,IPPROB(IFBRHW))} (composante $z$)
\item flux convectif d'\'energie au bord
\var{PROPFB(*,IPPROB(IFBENE))}
\end{itemize}
et on dispose \'egalement dans \var{IA}~:
\begin{itemize}
\item d'un tableau d'entiers dont la premi\`ere ``case'' est \var{IA(IIFBRU)}, dimensionn\'e au nombre de faces de bord
et permettant de rep\'erer les faces de bord pour lesquelles on calcule
le flux convectif par le sch\'ema de Rusanov,
\item d'un tableau  d'entiers dont la premi\`ere ``case'' est \var{IA(IIFBET)}, dimensionn\'e au nombre de faces de bord
et permettant de rep\'erer les faces de paroi \`a temp\'erature ou \`a flux
thermique impos\'e.
\end{itemize}



\newpage

\subsection*{Arbre d'appel simplifi\'e}
\nopagebreak
\begin{table}[h!]
\begin{center}
\begin{tabular}{lllllp{8cm}}
\fort{usini1}         &                 &                &                &
        & Initialisation des mots-cl\'es utilisateur g\'en\'eraux et positionnement des variables\\
                &\fort{usppmo}         &                &                &
        & D\'efinition du module ``physique particuli\`ere'' employ\'e\\
                &\fort{varpos}         &                &                &
        & Positionnement des variables \\
                &                 & \fort{pplecd} &                &
        & Branchement des physiques particuli\`eres pour la lecture du fichier de donn\'ees \'eventuel \\
                &                 & \fort{ppvarp} &                &
        & Branchement des physiques particuli\`eres pour le positionnement des inconnues \\
                &                 &                 & \fort{cfvarp} &
        & Positionnement des inconnues sp\'ecifiques au module compressible \\
                &                 &                 &               & \fort{uscfth}
        & Appel� avec ICCFTH=-1, pour indiquer que $C_p$ et $C_v$ sont constants ou variables\\
                &                 &                 &               & \fort{uscfx2}
        & Conductivit� thermique mol\'eculaire constante ou variable et viscosit� en volume
           constante ou variable (ainsi que leur valeur, si elles sont constantes)\\
                &                 & \fort{ppprop} &                &
        & Branchement des physiques particuli\`eres pour le positionnement des propri\'et\'es\\
                &                 &                 & \fort{cfprop} &
        & Positionnement des propri\'et\'es sp\'ecifiques au module compressible \\
%
\fort{ppini1}         &                &                &                &
        & Branchement des physiques particuli\`eres pour l'initialisation des
mots-cl\'es sp\'ecifiques \\
                &\fort{cfini1}         &                &                &
        & Initialisation des mots-cl\'es sp\'ecifiques au module compressible\\
                &\fort{uscfi1}         &                &                &
        & Initialisation des mots-cl\'es utilisateur sp\'ecifiques au module compressible\\
\end{tabular}
\caption{Sous-programme \fort{initi1}~: initialisation des mots-cl\'es et
positionnement des variables}
\end{center}
\end{table}

\newpage

\begin{table}[h!]
\begin{center}
\begin{tabular}{llllp{10cm}}
\fort{ppiniv}         &                &                &
        & Branchement des physiques particuli\`eres pour l'initialisation des variables \\
                & \fort{cfiniv} &                &
        & Initialisation des variables sp\'ecifiques au module compressible \\
                 &                 & \fort{memcfv} &
        & R\'eservation de tableaux de travail locaux   \\
                 &                 & \fort{uscfth} &
        & Initialisation des variables par d�faut (en calcul suite~: seulement
$C_v$~; si le calcul n'est pas une suite~: $C_v$, la masse volumique et l'\'energie) \\
                 &                 & \fort{uscfxi} &
        & Initialisation des variables par l'utilisateur (seulement si le calcul
n'est pas une suite)  \\
\end{tabular}
\caption{Sous-programme \fort{inivar}~: initialisation des variables}
\end{center}
\end{table}


\begin{table}[h!]
\begin{center}
\begin{tabular}{llllp{10cm}}
\fort{phyvar}         &                &                &
        & Calcul des propri\'et\'es physiques variables \\
                & \fort{ppphyv} &                &
        & Branchement des physiques particuli\`eres pour le calcul des
                propri\'et\'es physiques variables \\
                &               & \fort{cfphyv} &
        & Calcul des propri\'et\'es physiques variables pour le module
                compressible \\
                 &                 &               & \fort{usphyv}
        & Calcul par l'utilisateur des propri\'et\'es physiques variables pour
                le module
                compressible ($C_v$ est calcul\'e dans \fort{uscfth} qui est
                appel\'e par \fort{usphyv}) \\
\end{tabular}
\caption{Sous-programme \fort{tridim}~: partie 1 (propri\'et\'es physiques)}
\end{center}
\end{table}

\newpage

\begin{table}[h!]
\begin{center}
\begin{tabular}{llllp{10cm}}
\fort{dttvar}         &                &                &
        & Calcul du pas de temps variable  \\
                & \fort{cfdttv} &                &
        & Calcul de la contrainte li�e au CFL en compressible \\
                &                    &\fort{memcft}         &
        & Gestion de la m\'emoire pour le calcul de la contrainte en CFL \\
                &                    &\fort{cfmsfl}         &
        & Calcul du flux associ\'e \`a la contrainte en CFL \\

\fort{precli}         &                  &                &
        & Initialisation des tableaux avant calcul des conditions aux
                limites (\var{IITYPF}, \var{ICODCL}, \var{RCODCL})\\
                & \fort{ppprcl} &                &
        & Initialisations sp\'ecifiques aux diff\'erentes physiques
                particuli\`eres avant calcul des conditions aux limites
                (pour le module compressible~: \var{IZFPPP}, \var{IA(IIFBRU)},
                \var{IA(IIFBET)}, \var{RCODCL}, flux convectifs pour la
                quantit\'e de mouvement et l'\'energie)\\

\fort{ppclim}         &                  &                &
        & Branchement des physiques particuli\`eres pour les conditions aux limites (en lieu et place de \fort{usclim})\\
                & \fort{uscfcl} &                &
        & Intervention de l'utilisateur pour les conditions aux limites (en lieu
                et place de \fort{usclim}, m\^eme pour les variables qui ne sont
                pas sp\'ecifiques au module compressible) \\

\fort{condli}         &                  &                &
        & Traitement des conditions aux limites\\
                & \fort{pptycl} &                &
        & Branchement des physiques particuli\`eres pour le traitement des conditions aux limites \\
                &                 &\fort{cfxtcl}         &
        & Traitement des conditions aux limites pour le compressible \\
                &                 &                &\fort{uscfth}
        & Calculs de thermodynamique pour le calcul des conditions aux limites \\
                &                 &                &\fort{cfrusb}
        & Flux de Rusanov (entr\'ees ou sorties sauf sortie supersonique) \\
\end{tabular}
\caption{Sous-programme \fort{tridim}~: partie 2 (pas de temps variable et conditions
                                                  aux limites)}
\end{center}
\end{table}

\newpage

\begin{table}[h!]
\begin{center}
\begin{tabular}{llllp{10cm}}
\fort{memcfm}        &                  &                &
        & Gestion de la m\'emoire pour la r\'esolution de l'\'etape ``acoustique'' \\
\fort{cfmsvl}         &                  &                &
        & R\'esolution de l'�tape ``acoustique'' \\
                 & \fort{cfmsfl}  &                &
        & Calcul du "flux de masse" aux faces
                (not\'e $\rho\,\vect{w}\cdot\vect{n}\,S$ dans la documentation
                du sous-programme \fort{cfmsvl}) \\
                 &                  & \fort{cfdivs}&
        & Calcul du terme en divergence du tenseur des contraintes visqueuses
                  (trois appels), �ventuellement \\
                 &                  &                &
        & Apr\`es \fort{cfmsfl}, on impose le flux de masse aux faces de bord
                \`a partir des conditions aux limites \\
                 & \fort{cfmsvs}  &                &
        & Calcul de la "viscosit�" aux faces
                (not\'ee $\Delta\,t\,c^2\frac{S}{d}$ dans la documentation
                du sous-programme \fort{cfmsvl}) \\
                 &                  &                &
        & Apr\`es \fort{cfmsvs}, on annule la viscosit\'e aux faces de bord
                pour que le flux de masse soit bien celui souhait\'e \\
                 & \fort{codits}  &                &
        & R\'esolution du syst\`eme portant sur la masse volumique \\
                 & \fort{clpsca}  &                &
        & Impression des bornes et clipping \'eventuel (pas de clipping en standard)  \\
                 & \fort{uscfth}  &                &
        & Gestion  \'eventuelle des bornes par l'utilisateur  \\
                 & \fort{cfbsc3}  &                &
        & Calcul du flux de masse acoustique aux faces
                (not\'e $\vect{Q}_{ac}\cdot\vect{n}$ dans la documentation
                du sous-programme \fort{cfmsvl}) \\
                 & \fort{uscfth}  &                &
        & Actualisation de la pression, �ventuellement  \\
\fort{cfqdmv}         &                  &                &
        & R\'esolution de la quantit� de mouvement\\
                & \fort{cfcdts}         &                &
        & R�solution du syst\`eme\\
                &                  & \fort{cfbsc2}&
        & Calcul des termes de convection et de diffusion au second membre\\
\end{tabular}
\caption{Sous-programme \fort{tridim}~: partie 3 (Navier-Stokes)}
\end{center}
\end{table}

\newpage

\begin{table}[h!]
\begin{center}
\begin{tabular}{llllp{10cm}}
\fort{scalai}          &                  &                &
        & R\'esolution des \'equations sur les scalaires  \\
                & \fort{cfener}         &                &
        & R\'esolution de l'�quation sur l'�nergie totale\\
                &                  & \fort{memcfe}&
        & Gestion de la m\'emoire locale\\
                &                  & \fort{cfdivs}&
        & Calcul du terme en divergence du produit
           ``tenseur des contraintes par vitesse''\\
                &                  & \fort{uscfth}&
        & Calcul de l'\'ecart  ``\'energie interne - $C_v\,T$''
                ($\varepsilon_{sup}$)\\
                &                  & \fort{cfcdts}&
        & R�solution du syst\`eme\\
                &                  &                  &\fort{cfbsc2}
        & Calcul des termes de convection et de diffusion au second membre\\
                 &                  & \fort{clpsca}&
        & Impression des bornes et clipping \'eventuel (pas de clipping en standard)  \\
                 &                  & \fort{uscfth}&
        & Gestion \'eventuelle des bornes par l'utilisateur  \\
                 &                  & \fort{uscfth}&
        & Mise \`a jour de la pression  \\
\end{tabular}
\caption{Sous-programme \fort{tridim}~: partie 4 (scalaires)}
\end{center}
\end{table}

%\newpage

Le sous-programme \fort{cfbsc3} est similaire \`a \fort{bilsc2}, mais il produit
des flux aux faces et n'est \'ecrit que pour un sch\'ema upwind, \`a l'ordre 1
en temps (ce qui est coh\'erent avec les choix faits dans l'algorithme compressible).

Le sous-programme \fort{cfbsc2} est similaire \`a \fort{bilsc2}, mais
n'est \'ecrit que pour un sch\'ema d'ordre 1 en
temps.
%et fait encore appara\^itre la variable IITURB au lieu de IITYTU (il
%faudrait corriger ce dernier point).
Le sous-programme \fort{cfbsc2} permet d'effectuer un traitement
sp\'ecifique aux faces de bord pour lesquelles on a appliqu\'e
un sch\'ema de Rusanov pour calculer le flux convectif total.
Ce sous-programme est appel\'e pour la r\'esolution de l'\'equation de
la quantit\'e de mouvement et de l'\'equation de l'\'energie.
On pourra se reporter \`a la documentation du sous-programme \fort{cfxtcl}.

Le sous-programme \fort{cfcdts} est similaire \`a \fort{codits} mais fait appel
\`a \fort{cfbsc2} et non pas \`a \fort{bilsc2}.
Il diff\`ere de \fort{codits} par quelques autres d\'etails qui ne sont pas
g\^enants dans l'imm\'ediat~:
initialisation de PVARA et de SMBINI,
%nombre d'it\'erations pour le second membre (NSWRSM-1 au lieu de NSWRSM),
%mode de d\'etermination du solveur (IRESLP),
%test de convergence sur RNORM (compar\'e \`a 0.D0 au lieu de EPZERO),
ordre en temps (ordre 2 non pris en compte).
%Mis \`a part pour l'ordre en temps (l'algorithme
%compressible est \`a l'ordre 1), il serait bon de modifier \fort{cfcdts} pour
%qu'il soit conforme \`a  \fort{codits}.

\newpage
%%%%%%%%%%%%%%%%%%%%%%%%%%%%%%%%%%
%%%%%%%%%%%%%%%%%%%%%%%%%%%%%%%%%%
\section*{Points \`a traiter}
%%%%%%%%%%%%%%%%%%%%%%%%%%%%%%%%%%
%%%%%%%%%%%%%%%%%%%%%%%%%%%%%%%%%%

Des actions compl\'ementaires sont identifi\'ees ci-apr\`es, dans l'ordre
d'urgence d\'ecroissante (on se reportera
\'egalement \`a la section "Points \`a traiter" de la documentation
des autres sous-programmes du module compressible).

\begin{itemize}
\item Assurer la coh\'erence des sous-programmes suivants (ou, \'eventuellement,
les fusionner pour \'eviter qu'ils ne divergent)~:
        \begin{itemize}
        \item \fort{cfcdts} et \fort{codits},
%(actuellement pour PVARA et
%                SMBINI, mais \`a plus long terme pour \'eviter que les
%                deux sous-programmes ne divergent),
% propose en patch 1.2.1
%        \item \fort{cfcdts} et \fort{codits}
%                (au moins pour PVARA, SMBINI, NSWRSM,
%                IRESLP, RNORM),
        \item \fort{cfbsc2} et \fort{bilsc2},
        \item \fort{cfbsc3} et \fort{bilsc2}.
        \end{itemize}
% propose en patch 1.2.1
%        \item Remplacer la valeur 100 par 90 pour ICCFTH dans \fort{uscfth}
%                (plus grande coh\'erence avec les choix faits dans le reste de ce
%                sous-programme).
% propose en patch 1.2.1
%        \item \'Eliminer \fort{memcff} qui ne sert plus.
\item Permettre les suites de calcul incompressible/compressible et
        compressible/incompressible.
\item Apporter un compl\'ement de validation (exemple~: IPHYDR).
\item Assurer la compatibilit\'e avec certaines physiques particuli\`eres, selon
        les besoins. Par exemple~: arc \'electrique, rayonnement, combustion.
\item Identifier les causes des difficult\'es rencontr\'ees sur certains cas
acad\'emiques, en particulier~:
        \begin{itemize}
        \item canal subsonique (comment s'affranchir des effets ind\'esirables
        associ\'es aux conditions d'entr\'ee et de sortie, comment r\'ealiser un
        calcul p\'eriodique, en particulier pour la temp\'erature dont le
        gradient dans la direction de l'\'ecoulement n'est pas nul, si
        les parois sont adiabatiques),
        \item cavit\'e ferm\'ee sans vitesse ni effets de gravit\'e,
        avec temp\'erature ou flux thermique impos\'e en paroi (il pourrait
        \^etre utile d'extrapoler le gradient de pression au bord~:
        la pression d\'epend de la temp\'erature et une simple condition de
        Neumann homog\`ene est susceptible de cr\'eer un terme source de
        quantit\'e de mouvement parasite),
        \item maillage non conforme (non conformit\'e dans la direction
        transverse d'un canal),
      \item ``tube � choc'' avec terme source d'�nergie.
        \end{itemize}
\item Compl\'eter certains points de documentation, en particulier les
        conditions aux limites thermiques pour le couplage avec \syrthes.
\item Am\'eliorer la rapidit\'e \`a faible nombre de Mach (est-il
possible de lever la limite
actuelle sur la valeur du pas de temps~?).
\item Enrichir, au besoin~:
        \begin{itemize}
        \item les thermodynamiques prises en compte (multiconstituant,
        gamma variable, Van der Waals...),
        \item la gamme des conditions aux limites d'entr\'ee
        disponibles (condition \`a d\'ebit massique et d\'ebit enthalpique
        impos\'es par exemple).
        \end{itemize}
\item Tester des variantes de l'algorithme~:
        \begin{itemize}
        \item prise en compte des termes sources de l'\'equation de la
        quantit\'e de mouvement autres que la gravit\'e dans l'\'equation de la
        masse r\'esolue lors de l'\'etape ``acoustique'' (les tests r\'ealis\'es
        avec cette variante de l'algorithme devront \^etre repris dans la
        mesure o\`u, dans \fort{cfmsfl}, IIROM et IIROMB n'\'etaient pas
        initialis\'es),
        \item implicitation du terme de convection dans
        l'\'equation de la masse (\'eliminer cette possibilit\'e si
        elle n'apporte rien),
        \item \'etape de pr\'ediction de la pression,
        \item non reconstruction de la masse volumique pour le terme convectif
        (actuellement, les termes convectifs sont trait\'es avec
        d\'ecentrement amont, d'ordre 1 en espace~;
        pour l'\'equation de la quantit\'e de mouvement et l'\'equation de
        l'\'energie, on utilise les valeurs prises au centre des cellules
        sans reconstruction~: c'est l'approche standard de \CS, traduite
        dans \fort{cfbsc2}~; par contre, dans \fort{cfmsvl}, on reconstruit
        les valeurs de la masse volumique utilis\'ees pour le terme
        convectif~; il n'y a pas de raison d'adopter des strat\'egies
        diff\'erentes, d'autant plus que la reconstruction de la masse
        volumique ne permet pas de monter en ordre et augmente le risque
        de d\'epassement des bornes physiques),
        \item mont\'ee en ordre en espace (en v\'erifier l'utilit\'e et
        la robustesse, en particulier relativement au principe du
        maximum pour la masse volumique),
        \item mont\'ee en ordre en temps (en v\'erifier l'utilit\'e et
        la robustesse).
        \end{itemize}
\item Optimiser l'encombrement m\'emoire.
\end{itemize}


%-------------------------------------------------------------------------------

% This file is part of Code_Saturne, a general-purpose CFD tool.
%
% Copyright (C) 1998-2019 EDF S.A.
%
% This program is free software; you can redistribute it and/or modify it under
% the terms of the GNU General Public License as published by the Free Software
% Foundation; either version 2 of the License, or (at your option) any later
% version.
%
% This program is distributed in the hope that it will be useful, but WITHOUT
% ANY WARRANTY; without even the implied warranty of MERCHANTABILITY or FITNESS
% FOR A PARTICULAR PURPOSE.  See the GNU General Public License for more
% details.
%
% You should have received a copy of the GNU General Public License along with
% this program; if not, write to the Free Software Foundation, Inc., 51 Franklin
% Street, Fifth Floor, Boston, MA 02110-1301, USA.

%-------------------------------------------------------------------------------

\programme{cfener}

\hypertarget{cfener}{}

\vspace{1cm}
%%%%%%%%%%%%%%%%%%%%%%%%%%%%%%%%%%
%%%%%%%%%%%%%%%%%%%%%%%%%%%%%%%%%%
\section*{Fonction}
%%%%%%%%%%%%%%%%%%%%%%%%%%%%%%%%%%
%%%%%%%%%%%%%%%%%%%%%%%%%%%%%%%%%%

Pour les notations et l'algorithme dans son ensemble,
on se reportera \`a \fort{cfbase}.

Apr\`es masse (acoustique) et quantit\'e de mouvement,
on consid\`ere un dernier pas fractionnaire (de $t^{**}$ \`a $t^{***}$)
au cours duquel seule varie l'\'energie totale $E = \rho e$.

\begin{equation}\label{Cfbl_Cfener_eq_energie_cfener}
\left\{\begin{array}{l}
\rho^{***}=\rho^{**}=\rho^{n+1}\\
\\
\vect{Q}^{***}=\vect{Q}^{**}=\vect{Q}^{n+1}\\
\\
\displaystyle\frac{\partial \rho e}{\partial t}
+ \divs\left( \vect{Q}_{ac} \left(e+\displaystyle\frac{P}{\rho}\right) \right)
= \rho\vect{f}_v\cdot\vect{u}
+ \divs(\tens{\Sigma}^v \vect{u})
- \divs{\,\vect{\Phi}_s} + \rho\Phi_v
\end{array}\right.
\end{equation}

Pour conserver la positivit\'e de l'\'energie, il est indispensable ici,
comme pour les scalaires, d'utiliser le flux de masse convectif acoustique
$\vect{Q}_{ac}^{n+1}$ compatible avec l'\'equation de la masse.
De plus, pour obtenir des propri\'et\'es de positivit\'e sur les scalaires,
un sch\'ema upwind pour le terme convectif doit \^etre utilis\'e
(mais les termes sources introduisent des contraintes suppl\'ementaires
qui peuvent \^etre pr\'epond\'erantes et g\^enantes).

\vspace{0.5cm}

\`A la fin de cette \'etape, on actualise �ventuellement
(mais par d�faut non)
une deuxi\`eme et derni\`ere fois la pression
en utilisant la loi d'\'etat pour obtenir la pression finale~:
\begin{equation}
\displaystyle P^{n+1}=P(\rho^{n+1},\varepsilon^{n+1})
\end{equation}

See the \doxygenfile{cfener_8f90.html}{programmers reference of the dedicated subroutine} for further details.

%%%%%%%%%%%%%%%%%%%%%%%%%%%%%%%%%
%%%%%%%%%%%%%%%%%%%%%%%%%%%%%%%%%%
\section*{Discr\'etisation}
%%%%%%%%%%%%%%%%%%%%%%%%%%%%%%%%%%
%%%%%%%%%%%%%%%%%%%%%%%%%%%%%%%%%%

%---------------------------------
\subsection*{Discr\'etisation en temps}
%---------------------------------

La mod\'elisation des flux de chaleur choisie jusqu'\`a pr\'esent est de la
forme $-\divs(\,\vect{\Phi}_s) = \divs(\lambda \gradv{T})$.

Pour faire appara\^itre un terme diffusif stabilisant dans la
matrice de r\'esolution, on cherche \`a exprimer le flux diffusif de chaleur
($-\divs(\,\vect{\Phi}_s)$)
en fonction de la variable r\'esolue (l'\'energie totale).

Avec $\varepsilon_{sup}(P,\rho)$
d\'ependant de la loi d'\'etat, on exprime l'\'energie totale de la fa\c con suivante~:
\begin{equation}
e = \varepsilon + \frac{1}{2} u^2
= (C_v T + \varepsilon_{sup}) + \frac{1}{2} u^2
\end{equation}

En supposant $C_v$ constant\footnote{Pour $C_v$ non constant, les
d\'eveloppements restent \`a faire~: on pourra se
reporter \`a  P. Mathon, F. Archambeau, J.-M. H�rard : "Implantation d'un
algorithme compressible dans \CS", HI-83/03/016/A}, on a alors~:
\begin{equation}\label{Cfbl_Cfener_eq_flux_thermique_cfener}
-\divs(\,\vect{\Phi}_s)
= \divs(K \gradv(e - \frac{1}{2} u^2 - \varepsilon_{sup}))\qquad
\text{avec } K=\lambda / C_v
\end{equation}

Lorsqu'un mod\`ele de turbulence est activ\'e, on conserve la m\^eme forme
de mod\'elisation pour les flux thermiques et $K$ int\`egre alors
la diffusivit\'e turbulente. On pourra se reporter \`a
la documentation de \fort{cfxtcl} \`a ce sujet.

Avec la formulation~(\ref{Cfbl_Cfener_eq_flux_thermique_cfener}),
on peut donc impliciter le terme en $\gradv{e}$.

\bigskip
De plus, puisque la vitesse a
d\'ej\`a \'et\'e r\'esolue, on implicite \'egalement le terme en
$\gradv{\frac{1}{2} u^2}$. L'exposant $n+\frac{1}{2}$ de $\varepsilon_{sup}$
indique que l'implicitation de ce terme est partielle (elle d\'epend de la forme
de la loi d'\'etat).

Par ailleurs, on implicite le terme de convection, le terme de puissance
des forces volumiques, �ventuellement le terme de puissance des forces de
pression (suivant la valeur de \var{IGRDPP}, on utilise la pr\'ediction de
pression obtenue apr\`es r\'esolution de l'\'equation portant sur la masse
volumique ou bien la pression du pas de temps pr�c�dent)
et le terme de puissance des forces visqueuses. On implicite le terme de puissance
volumique en utilisant $\rho^{n+1}$.

\bigskip
On obtient alors l'\'equation discr\`ete portant sur $e$~:
\begin{equation}\label{Cfbl_Cfener_eq_energie_totale_cfener}
\begin{array}{l}
\displaystyle\frac{(\rho e)^{n+1} - (\rho e)^n}{\Delta t^n}
+ \divs(\vect{Q}_{ac}^{n+1} e^{n+1}) - \divs(K^n \gradv{e^{n+1}})
= \rho^{n+1} \vect{f}_v \cdot \vect{u}^{n+1}
- \divs(\vect{Q}_{ac}^{n+1} \displaystyle\frac{\widetilde{P}}{\rho^{n+1}} )\\
\text{\ \ \ \ }+ \divs((\tens{\Sigma}^v)^{n+1} \vect{u}^{n+1})
- \divs(K^n \gradv(\frac{1}{2} (u^2)^{n+1}
+ \varepsilon_{sup}^{n+\frac{1}{2}}))
+ \rho^{n+1}\Phi_v\\
\end{array}
\end{equation}
avec $\widetilde{P}=P^{Pred}\text{ ou }P^n$ suivant la valeur de \var{IGRDPP}
($P^n$ par d�faut).

En pratique, dans \CS, on r\'esout cette \'equation en faisant appara\^itre \`a
gauche l'\'ecart $e^{n+1} - e^n$. Pour cela, on \'ecrit la d\'eriv\'ee
en temps discr\`ete sous la forme suivante~:

\begin{equation}
\begin{array}{ll}
\displaystyle
\frac{(\rho e)^{n+1} - (\rho e)^n}{\Delta t^n}
& =
\displaystyle
\frac{\rho^{n+1}\, e^{n+1} - \rho^n\, e^n}{\Delta t^n}\\
& =
\displaystyle
\frac{\rho^{n}\, e^{n+1} - \rho^n\, e^n}{\Delta t^n}+
\frac{\rho^{n+1}\, e^{n+1} - \rho^n\, e^{n+1}}{\Delta t^n}\\
& =
\displaystyle
\frac{\rho^{n}}{\Delta t^n}\left(e^{n+1} - e^n\right)+
e^{n+1}\frac{\rho^{n+1} - \rho^n}{\Delta t^n}
\end{array}
\end{equation}

et l'on utilise l'\'equation de la masse discr\`ete pour \'ecrire~:
\begin{equation}
\displaystyle
\frac{(\rho e)^{n+1} - (\rho e)^n}{\Delta t^n}
=
\frac{\rho^{n}}{\Delta t^n}\left(e^{n+1} - e^n\right)-
e^{n+1}\dive\,\vect{Q}_{ac}^{n+1}
\end{equation}



%---------------------------------
\subsection*{Discr\'etisation en espace}
%---------------------------------


%.................................
\subsubsection*{Introduction}
%.................................

On int\`egre l'\'equation (\ref{Cfbl_Cfener_eq_energie_totale_cfener})
sur la cellule $i$ de volume $\Omega_i$ et l'on proc\`ede comme
pour l'\'equation de la masse et de la quantit\'e de mouvement.

On obtient alors l'\'equation discr\`ete
suivante~:
\begin{equation}\label{Cfbl_Cfener_eq_energie_totale_discrete_cfener}
\begin{array}{l}
\displaystyle\frac{\Omega_i}{\Delta t^n}
(\rho_i^{n+1} e_i^{n+1}-\rho_i^n e_i^n)
+ \displaystyle\sum\limits_{j\in V(i)}
\left(e^{n+1} \vect{Q}_{ac}^{n+1}\right)_{ij} \cdot \vect{S}_{ij}
- \displaystyle\sum\limits_{j\in V(i)}
\left(K^n\gradv(e^{n+1})\right)_{ij}\cdot\vect{S}_{ij}\\
\\
\text{\ \ \ \ } = \Omega_i\rho_i^{n+1} {\vect{f}_v}_i \cdot \vect{u}_i^{n+1}
- \displaystyle\sum\limits_{j\in V(i)}
\left(\displaystyle\frac{P^{Pred}}{\rho^{n+1}}\
\vect{Q}_{ac}^{n+1}\right)_{ij} \cdot \vect{S}_{ij}
+ \displaystyle\sum\limits_{j\in V(i)}
\left((\tens{\Sigma}^v)^{n+1} \vect{u}^{n+1} \right)_{ij}\cdot \vect{S}_{ij}\\
\\
\text{\ \ \ \ } - \displaystyle\sum\limits_{j\in V(i)}
\left(K^n \gradv\left(\frac{1}{2}(u^2)^{n+1}
+ \varepsilon_{sup}^{n+\frac{1}{2}}\right)\right)_{ij}\cdot\vect{S}_{ij}
+ \Omega_i\rho_i^{n+1}{\Phi_v}_i\\
\end{array}
\end{equation}


%.................................
\subsubsection*{Discr\'etisation de la partie ``convective''}
%.................................

La valeur \`a la face s'\'ecrit~:
\begin{equation}
\left(e^{n+1} \vect{Q}_{ac}^{n+1}\right)_{ij} \cdot \vect{S}_{ij}
= e_{ij}^{n+1}(\vect{Q}_{ac}^{n+1})_{ij} \cdot \vect{S}_{ij}
\end{equation}
avec un d\'ecentrement sur la valeur de $e^{n+1}$ aux faces~:
\begin{equation}
\begin{array}{lllll}
e_{ij}^{n+1}
& = & e_i^{n+1}
& \text{si\ } & (\vect{Q}_{ac}^{n+1})_{ij} \cdot \vect{S}_{ij} \geqslant 0 \\
& = & e_j^{n+1}
& \text{si\ } & (\vect{Q}_{ac}^{n+1})_{ij} \cdot \vect{S}_{ij} < 0 \\
\end{array}
\end{equation}
que l'on peut noter~:
\begin{equation}
 e_{ij}^{n+1}
 = \beta_{ij}e_i^{n+1} + (1-\beta_{ij})e_j^{n+1}
\end{equation}
avec
\begin{equation}
\left\{\begin{array}{lll}
\beta_{ij} = 1 & \text{si\ }
& (\vect{Q}_{ac}^{n+1})_{ij} \cdot \vect{S}_{ij} \geqslant 0 \\
\beta_{ij} = 0 & \text{si\ }
& (\vect{Q}_{ac}^{n+1})_{ij} \cdot \vect{S}_{ij} < 0 \\
\end{array}\right.
\end{equation}


%.................................
\subsubsection*{Discr\'etisation de la partie ``diffusive''}
%.................................

La valeur \`a la face s'\'ecrit~:
\begin{equation}
\begin{array}{c}
\left(K^n\gradv(e^{n+1})\right)_{ij}\cdot\vect{S}_{ij}
= K_{ij}^n
\displaystyle \left( \frac{\partial e}{\partial n} \right)^{n+1}_{ij}S_{ij}\\
\text{et}\\
\left(K^n \gradv\left(\frac{1}{2}(u^2)^{n+1}
+ \varepsilon_{sup}^{n+\frac{1}{2}}\right)\right)_{ij}\cdot\vect{S}_{ij}
= K_{ij}^n
\displaystyle \left( \frac{\partial \left(\frac{1}{2} u^2
+ \varepsilon_{sup}\right)}{\partial n} \right)^{n+\frac{1}{2}}_{ij}S_{ij}
\end{array}
\end{equation}
avec une interpolation lin\'eaire pour
$K^n$ aux faces (et en pratique, $\alpha_{ij}=\frac{1}{2}$)~:
\begin{equation}
K_{ij}^n
= \alpha_{ij}K_{i}^n+(1-\alpha_{ij})K_{j}^n
\end{equation}
et un sch\'ema centr\'e avec reconstruction pour le gradient normal aux faces~:
\begin{equation}
\displaystyle \left( \frac{\partial e}{\partial n} \right)^{n+1}_{ij}
= \displaystyle\frac{e_{J'}^{n+1} - e_{I'}^{n+1}}{\overline{I'J'}}
\quad \text{et} \quad
\displaystyle \left( \frac{\partial \left(\frac{1}{2} u^2
+ \varepsilon_{sup}\right)}{\partial n} \right)^{n+\frac{1}{2}}_{ij}
= \displaystyle\frac{(\frac{1}{2} u^2
+ \varepsilon_{sup})_{J'}^{n+\frac{1}{2}} - (\frac{1}{2} u^2
+ \varepsilon_{sup})_{I'}^{n+\frac{1}{2}}}{\overline{I'J'}}
\end{equation}



%.................................
\subsubsection*{Discr\'etisation de la puissance des forces de pression}
%.................................

Ce terme
est issu du terme convectif, on le discr\'etise donc de la m\^eme fa\c con.

\begin{equation}
\left(\displaystyle\frac{\widetilde{P}}{\rho^{n+1}}\
\vect{Q}_{ac}^{n+1}\right)_{ij} \cdot \vect{S}_{ij}
= \left(\displaystyle\frac{\widetilde{P}}{\rho^{n+1}}\right)_{ij}
(\vect{Q}_{ac}^{n+1})_{ij} \cdot \vect{S}_{ij}
\end{equation}

avec un d\'ecentrement sur la valeur de
$\displaystyle\frac{P}{\rho}$ aux faces~:
\begin{equation}
\begin{array}{lll}
\left(\displaystyle\frac{\widetilde{P}}{\rho^{n+1}}\right)_{ij}
 = \beta_{ij}\displaystyle\frac{\widetilde{P}_i}{\rho^{n+1}_i}
+ (1-\beta_{ij})\displaystyle\frac{\widetilde{P}_j}{\rho^{n+1}_j}
& \text{avec}
& \left\{\begin{array}{lll}
\beta_{ij} = 1 & \text{si\ }
& (\vect{Q}_{ac}^{n+1})_{ij} \cdot \vect{S}_{ij} \geqslant 0 \\
\beta_{ij} = 0 & \text{si\ }
& (\vect{Q}_{ac}^{n+1})_{ij} \cdot \vect{S}_{ij} < 0 \\
\end{array}\right.
\end{array}
\end{equation}



%.................................
\subsubsection*{Discr\'etisation de la puissance des forces visqueuses}
%.................................

On calcule les termes dans les cellules puis on utilise une
interpolation lin\'eaire (on utilise
$\alpha_{ij}=\frac{1}{2}$ dans la relation ci-dessous)~:
\begin{equation}
\left((\tens{\Sigma}^v)^{n+1} \vect{u}^{n+1} \right)_{ij}\cdot \vect{S}_{ij}
= \left\{\alpha_{ij} \left((\tens{\Sigma}^v)^{n+1} \vect{u}^{n+1}\right)_i
+ (1-\alpha_{ij}) \left((\tens{\Sigma}^v)^{n+1} \vect{u}^{n+1}\right)_j
\right\} \cdot \vect{S}_{ij}
\end{equation}


%.................................
\subsubsection*{Remarques}
%.................................


Les termes ``convectifs'' associ\'es \`a
$\displaystyle\dive\left(\left(e^{n+1}+\frac{\widetilde{P}}{\rho^{n+1}}\right)\,
\vect{Q}_{ac}^{n+1}\right)$ sont calcul\'es avec un d\'ecentrement amont
(consistant, d'ordre 1 en espace). Les valeurs utilis\'ees sont bien prises au
centre de la cellule amont ($e_i$, $P_i$, $\rho_i$) et non pas au projet\'e $I'$
du centre de la cellule sur la normale \`a la face passant par son centre de
gravit\'e (sur un cas test en triangles, l'utilisation de $P_I'$ et de $\rho_I'$
pour le terme de transport de pression a conduit \`a un r\'esultat
insatisfaisant, mais des corrections ont \'et\'e apport\'ees aux sources depuis
et il serait utile de v\'erifier que cette conclusion n'est pas remise en question).

Les termes diffusifs associ\'es \`a
$\displaystyle\dive\left(K\,\grad\left(e+\frac{1}{2} u^2 +
\varepsilon_{sup}\right)\right)$ sont calcul\'es en utilisant des valeurs aux
faces reconstruites pour s'assurer de la consistance du sch\'ema.

%%%%%%%%%%%%%%%%%%%%%%%%%%%%%%%%%%
%%%%%%%%%%%%%%%%%%%%%%%%%%%%%%%%%%
\section*{Mise en \oe uvre}
%%%%%%%%%%%%%%%%%%%%%%%%%%%%%%%%%%
%%%%%%%%%%%%%%%%%%%%%%%%%%%%%%%%%%


Apr\`es une \'etape de gestion de la m\'emoire (\fort{memcfe}), on calcule les
diff\'erents termes sources (au centre des cellules)~:
\begin{itemize}
\item source volumique de chaleur (\fort{ustssc}),
\item source associ\'ee aux sources de masse (\fort{catsma}),
\item source associ\'ee \`a l'accumulation de masse $\dive\,\vect{Q}_{ac}$ (directement dans \fort{cfener}),
\item dissipation visqueuse (\fort{cfdivs}),
\item transport de pression (directement dans \fort{cfener}),
\item puissance de la pesanteur (directement dans \fort{cfener}),
\item termes diffusifs en $\displaystyle\dive\left(K\,\grad\left(\frac{1}{2} u^2 +
\varepsilon_{sup}\right)\right)$ (calcul de $\varepsilon_{sup}$ par
\fort{uscfth}, puis calcul du terme diffusif directement dans \fort{cfener}).
\end{itemize}

\bigskip
Le syst\`eme (\ref{Cfbl_Cfener_eq_energie_totale_discrete_cfener}) est r\'esolu par une m\'ethode
d'incr\'ement et r\'esidu  en utilisant une m\'ethode de Jacobi (\fort{cfcdts}).

L'impression des bornes et
la limitation \'eventuelle de l'\'energie sont ensuite effectu\'ees par
\fort{clpsca} suivi de \fort{uscfth} (intervention utilisateur optionnelle).

On actualise enfin la pression et on calcule la
temp\'erature (\fort{uscfth}).

Pour
terminer, en parall\`ele ou en p\'eriodique, on \'echange les variables
pression, \'energie et temp\'erature.



%%%%%%%%%%%%%%%%%%%%%%%%%%%%%%%%%%
%%%%%%%%%%%%%%%%%%%%%%%%%%%%%%%%%%
\section*{Points \`a traiter}
%%%%%%%%%%%%%%%%%%%%%%%%%%%%%%%%%%
%%%%%%%%%%%%%%%%%%%%%%%%%%%%%%%%%%

% propose en patch 1.2.1

%Corriger \fort{cfener} dans lequel \var{W1} produit par \fort{uscfth} est
%\'ecras\'e par \fort{grdcel}, causant probablement des d\'eg\^ats
%dans les cas o\`u le gradient de l'\'energie cin\'etique dans la direction $x$
%est sensiblement non nul sur des faces de bord dont la normale a une
%composante en $x$ et lorsque la conductivit\'e n'est pas n\'egligeable.


\etape{Choix de $\widetilde{P}$}
En standard, on utilise $\widetilde{P}=P^n$, mais ce n'est pas le seul choix
possible. On pourrait �tudier le comportement de l'algorithme avec $P^{Pred}$ et
$P^{n+1}$ (avec $P^{n+1}$, en particulier,
$\displaystyle\frac{\widetilde{P}}{\rho^{n+1}}$
est �valu� avec la masse volumique et l'�nergie prises au m�me instant).

\etape{Terme source dans l'�quation de l'�nergie}
La pr�sence d'un terme source externe dans l'�quation de l'�nergie g�n�re des
oscillations de vitesse qu'il est important d'analyser et de comprendre.

%-------------------------------------------------------------------------------

% This file is part of Code_Saturne, a general-purpose CFD tool.
%
% Copyright (C) 1998-2018 EDF S.A.
%
% This program is free software; you can redistribute it and/or modify it under
% the terms of the GNU General Public License as published by the Free Software
% Foundation; either version 2 of the License, or (at your option) any later
% version.
%
% This program is distributed in the hope that it will be useful, but WITHOUT
% ANY WARRANTY; without even the implied warranty of MERCHANTABILITY or FITNESS
% FOR A PARTICULAR PURPOSE.  See the GNU General Public License for more
% details.
%
% You should have received a copy of the GNU General Public License along with
% this program; if not, write to the Free Software Foundation, Inc., 51 Franklin
% Street, Fifth Floor, Boston, MA 02110-1301, USA.

%-------------------------------------------------------------------------------

\programme{cfmsvl}
%
\vspace{1cm}
%%%%%%%%%%%%%%%%%%%%%%%%%%%%%%%%%%
%%%%%%%%%%%%%%%%%%%%%%%%%%%%%%%%%%
\section*{Fonction}
%%%%%%%%%%%%%%%%%%%%%%%%%%%%%%%%%%
%%%%%%%%%%%%%%%%%%%%%%%%%%%%%%%%%%

Pour les notations et l'algorithme dans son ensemble,
on se reportera \`a \fort{cfbase}.

On consid\`ere un premier pas fractionnaire au cours duquel l'\'energie totale
est fixe. Seules varient la masse volumique et le flux de masse acoustique
normal aux faces (d\'efini et calcul\'e aux faces).

On a donc le syst\`eme suivant, entre $t^n$ et $t^*$~:
\begin{equation}\label{Cfbl_Cfmsvl_eq_acoustique_cfmsvl}
\left\{\begin{array}{l}

\displaystyle\frac{\partial\rho}{\partial t}+\divs{\vect{Q}_{ac}} = 0 \\
\\
\displaystyle\frac{\partial\vect{Q}_{ac}}{\partial t}+\gradv{P} =
\rho \vect{f}\\
\\
\vect{Q}^*=\vect{Q}^n\\
\\
e^*=e^n\\

\end{array}\right.
\end{equation}

Une partie des termes sources de l'\'equation de la
quantit\'e de mouvement peut \^etre prise en compte dans cette \'etape
(les termes les plus importants, en pr\^etant attention aux sous-\'equilibres).

Il faut noter que si $\vect{f}$ est effectivement nul, on aura bien un
syst\`eme ``acoustique'', mais que si l'on place des termes suppl\'ementaires
dans $\vect{f}$, la d\'enomination est abusive (on la conservera cependant).

On obtient $\rho^* = \rho^{n+1}$ en r\'esolvant (\ref{Cfbl_Cfmsvl_eq_acoustique_cfmsvl}),
et l'on actualise alors le flux de masse acoustique $\vect{Q}_{ac}^{n+1}$,
qui servira pour la convection (en particulier pour la convection de
l'enthalpie totale et de tous les scalaires transport\'es).

Suivant la valeur de \var{IGRDPP}, on actualise �ventuellement la pression, en
utilisant la loi d'\'etat :
$$
\displaystyle P^{Pred}=P(\rho^{n+1},\varepsilon^{n})
$$

%%%%%%%%%%%%%%%%%%%%%%%%%%%%%%%%%
%%%%%%%%%%%%%%%%%%%%%%%%%%%%%%%%%%
\section*{Discr\'etisation}
%%%%%%%%%%%%%%%%%%%%%%%%%%%%%%%%%%
%%%%%%%%%%%%%%%%%%%%%%%%%%%%%%%%%%
%---------------------------------
\subsection*{Discr\'etisation en temps}
%---------------------------------

Le syst\`eme (\ref{Cfbl_Cfmsvl_eq_acoustique_cfmsvl}) discr\'etis\'e en temps donne :
\begin{equation}\label{Cfbl_Cfmsvl_eq_acoustique_discrete_cfmsvl}
\left\{\begin{array}{l}

\displaystyle\frac{\rho^{n+1}-\rho^n}{\Delta t^n}
+ \divs{\vect{Q}_{ac}^{n+1}} = 0 \\
\\
\displaystyle\frac{\vect{Q}_{ac}^{n+1}-\vect{Q}^n}{\Delta t^n}+\gradv{P^*} =
\rho^n \vect{f}^n\\
\\
Q^*=Q^n\\
\\
e^*=e^n\\

\end{array}\right.
\end{equation}

\begin{equation}\label{Cfbl_Cfmsvl_eq_forces_supplementaires_cfmsvl}
\begin{array}{llll}
\text{avec\ }&\vect{f}^n &=& \vect{0} \\
\text{ou\ }&\vect{f}^n &=& \vect{g} \\
\text{ou m\^eme\ }&\vect{f}^n &=& \vect{f}_v
 + \displaystyle\frac{1}{\rho^n}
\left( - \divs(\vect{u} \otimes \vect{Q}) + \divv(\tens{\Sigma}^v)
 + \vect{j}\wedge\vect{B} \right)^n
\end{array}
\end{equation}

Dans la pratique nous avons d�cid� de prendre $\vect{f}^n=\vect{g}$~:
\begin{itemize}
  \item le terme $\vect{j}\wedge\vect{B}$ n'a pas �t� test�,
  \item le terme $\divv(\tens{\Sigma}^v)$ \'etait n�gligeable sur les tests
        r�alis�s,
  \item le terme $\divs(\vect{u} \otimes \vect{Q})$ a paru d�stabiliser les
        calculs (mais au moins une partie des tests a \'et\'e r\'ealis\'ee
        avec une erreur de programmation et il faudrait donc les reprendre).
\end{itemize}
\bigskip

Le terme $\vect{Q}^n$ dans la 2\textsuperscript{\`eme} \'equation
de (\ref{Cfbl_Cfmsvl_eq_acoustique_discrete_cfmsvl}) est le vecteur ``quantit\'e de mouvement''
qui provient de l'\'etape de r\'esolution de la quantit\'e de mouvement du pas
de temps pr\'ec\'edent, $\vect{Q}^n = \rho^n \vect{u}^n$.
On pourrait th�oriquement utiliser un vecteur quantit\'e de mouvement issu
de l'\'etape acoustique du pas de temps pr\'ec\'edent, mais il ne constitue
qu'un ``pr\'edicteur'' plus ou moins satisfaisant (il n'a pas ``vu'' les termes
sources qui ne sont pas dans  $\vect{f}^n$) et cette solution
n'a pas �t� test�e.

\bigskip
On \'ecrit alors la pression sous la forme~:
\begin{equation}
\gradv{P}=c^2\,\gradv{\rho}+\beta\,\gradv{s}
\end{equation}

avec $c^2 = \left.\displaystyle\frac{\partial P}{\partial \rho}\right|_s$
et $\beta = \left.\displaystyle\frac{\partial P}{\partial s}\right|_\rho$
tabul\'es ou analytiques \`a partir de la loi d'\'etat.

On discr\'etise l'expression pr\'ec\'edente en~:
\begin{equation}
\gradv{P^*}=(c^2)^n\gradv(\rho^{n+1})+\beta^n\gradv(s^n)
\end{equation}

On obtient alors une \'equation
portant sur $\rho^{n+1}$ en substituant l'expression de $\vect{Q}_{ac}^{n+1}$
issue de la 2\textsuperscript{\`eme} \'equation
de~(\ref{Cfbl_Cfmsvl_eq_acoustique_discrete_cfmsvl})
dans la 1\textsuperscript{\`ere} \'equation
de~(\ref{Cfbl_Cfmsvl_eq_acoustique_discrete_cfmsvl})~:
\begin{equation}\label{Cfbl_Cfmsvl_eq_densite_cfmsvl}
\displaystyle\frac{\rho^{n+1}-\rho^n}{\Delta t^n}
+\divs(\vect{w}^n \rho^n)
-\divs\left(\Delta t^n (c^2)^n \gradv(\rho^{n+1})\right) = 0
\end{equation}

o\`u~:
\begin{equation}
\begin{array}{lll}
\vect{w}^n&=&  \vect{u}^n + \Delta t^n
\displaystyle\left(\vect{f}^n-\frac{\beta^n}{\rho^n}\gradv(s^n)\right)
\end{array}
\end{equation}

Formulation alternative (programm\'ee mais non test\'ee)
avec le terme de convection implicite~:
\begin{equation}\label{Cfbl_Cfmsvl_eq_densite_bis_cfmsvl}
\displaystyle\frac{\rho^{n+1}-\rho^n}{\Delta t^n}
+\divs(\vect{w}^n \rho^{n+1})
-\divs\left(\Delta t^n (c^2)^n \gradv(\rho^{n+1})\right) = 0
\end{equation}


%---------------------------------
\subsection*{Discr\'etisation en espace}
%---------------------------------


%.................................
\subsubsection*{Introduction}
%.................................

On int\`egre l'\'equation pr\'ec\'edente ( (\ref{Cfbl_Cfmsvl_eq_densite_cfmsvl})
ou (\ref{Cfbl_Cfmsvl_eq_densite_bis_cfmsvl}) ) sur la cellule $i$ de volume $\Omega_i$.
On transforme les int\'egrales de volume en int\'egrales surfaciques
et l'on discr\'etise ces int\'egrales. Pour simplifier l'expos�, on se
place sur une cellule $i$ dont aucune face n'est sur le bord du domaine.

On obtient alors l'\'equation discr\`ete
suivante\footnote{L'exposant $^{n+\frac{1}{2}}$ signifie que le terme
peut \^etre implicite ou explicite. En pratique on a choisi
$\rho^{n+\frac{1}{2}} = \rho^{n}$.}~:
\begin{equation}\label{Cfbl_Cfmsvl_eq_densite_discrete_cfmsvl}
\Omega_i \displaystyle\frac{\rho_i^{n+1}-\rho_i^n}{\Delta t^n}
+\sum\limits_{j\in Vois(i)}(\rho^{n+\frac{1}{2}} \vect{w}^n)_{ij} \cdot \vect{S}_{ij}
-\sum\limits_{j\in Vois(i)} \left(\Delta t^n (c^2)^n
\gradv(\rho^{n+1})\right)_{ij} \cdot \vect{S}_{ij}
= 0
\end{equation}

%.................................
\subsubsection*{Discr\'etisation de la partie ``convective''}
%.................................

La valeur \`a la face s'\'ecrit~:
\begin{equation}
(\rho^{n+\frac{1}{2}} \vect{w}^n)_{ij} \cdot \vect{S}_{ij}
= \rho^{n+\frac{1}{2}}_{ij} \vect{w}^n_{ij} \cdot \vect{S}_{ij}
\end{equation}
avec, pour $\vect{w}^n_{ij}$,
une simple interpolation lin\'eaire~:
\begin{equation}
\vect{w}^n_{ij}
= \alpha_{ij} \vect{w}^n_i + (1-\alpha_{ij}) \vect{w}^n_j
\end{equation}
et un d\'ecentrement sur la valeur de $\rho^{n+\frac{1}{2}}$ aux faces~:
\begin{equation}
\begin{array}{lllll}
\displaystyle\rho_{ij}^{n+\frac{1}{2}} &=& \rho_{I'}^{n+\frac{1}{2}}
                &\text{si\ }& \vect{w}^n_{ij} \cdot \vect{S}_{ij} \geqslant 0 \\
                         &=& \rho_{J'}^{n+\frac{1}{2}}
                &\text{si\ }& \vect{w}^n_{ij} \cdot \vect{S}_{ij} < 0 \\
\end{array}
\end{equation}
que l'on peut noter~:
\begin{equation}
\displaystyle\rho_{ij}^{n+\frac{1}{2}}
 = \beta_{ij}\rho_{I'}^{n+\frac{1}{2}} + (1-\beta_{ij})\rho_{J'}^{n+\frac{1}{2}}
\end{equation}
avec
\begin{equation}
\left\{\begin{array}{lll}
\beta_{ij} = 1 & \text{si\ } & \vect{w}^n_{ij} \cdot \vect{S}_{ij} \geqslant 0 \\
\beta_{ij} = 0 & \text{si\ } & \vect{w}^n_{ij} \cdot \vect{S}_{ij} < 0 \\
\end{array}\right.
\end{equation}

%.................................
\subsubsection*{Discr\'etisation de la partie ``diffusive''}
%.................................

La valeur \`a la face s'\'ecrit~:
\begin{equation}
\left(\Delta t^n (c^2)^n \gradv(\rho^{n+1})\right)_{ij}\cdot \vect{S}_{ij}
= \Delta t^n (c^2)^n_{ij}
\displaystyle \left( \frac{\partial \rho}{\partial n} \right)^{n+1}_{ij}S_{ij}
\end{equation}
avec, pour assurer la continuit\'e du flux normal \`a l'interface,
une interpolation harmonique de $(c^2)^n$~:
\begin{equation}\label{Cfbl_Cfmsvl_eq_harmonique_cfmsvl}
\displaystyle(c^2)_{ij}^n
= \frac{(c^2)_{i}^n (c^2)_{j}^n}
{\alpha_{ij}(c^2)_{i}^n+(1-\alpha_{ij})(c^2)_{j}^n}
\end{equation}
et un sch\'ema centr\'e pour le gradient normal aux faces~:
\begin{equation}
\displaystyle \left( \frac{\partial \rho}{\partial n} \right)^{n+1}_{ij}
= \displaystyle\frac{\rho_{J'}^{n+1}-\rho_{I'}^{n+1}}{\overline{I'J'}}
\end{equation}

%.................................
\subsubsection*{Syst\`eme final}
%.................................

On obtient maintenant le syst\`eme final, portant sur
$(\rho_i^{n+1})_{i=1 \ldots N}$~:
\begin{equation}\label{Cfbl_Cfmsvl_eq_densite_finale_cfmsvl}
\displaystyle\frac{\Omega_i}{\Delta t^n} (\rho_i^{n+1}-\rho_i^n)
+\sum\limits_{j\in Vois(i)}\rho_{ij}^{n+\frac{1}{2}}
\vect{w}_{ij}^n \cdot \vect{S}_{ij}
-\sum\limits_{j\in Vois(i)} \Delta t^n (c^2)_{ij}^n
\displaystyle\frac{\rho_{J'}^{n+1}-\rho_{I'}^{n+1}}{\overline{I'J'}}\ S_{ij}
= 0
\end{equation}



%.................................
\subsubsection*{Remarque~: interpolation aux faces pour le terme de diffusion}
%.................................

Le choix de la forme de la moyenne pour le cofacteur du flux
normal n'est pas sans cons\'equence sur la vitesse de convergence, surtout
lorsque l'on est en pr\'esence de fortes inhomog\'en\'eit\'es.

On utilise une interpolation harmonique pour $c^2$
afin de conserver la continuit\'e du flux diffusif normal
$\Delta t (c^2) \displaystyle\frac{\partial \rho}{\partial n}$
\`a l'interface $ij$. En effet, on suppose que le flux est d\'erivable \`a
l'interface. Il doit donc y \^etre continu.\\
%
\'Ecrivons la continuit\'e du flux normal \`a l'interface,
avec la discr\'etisation
suivante\footnote{On ne reconstruit pas les valeurs de $\Delta\,t\,c^2$
aux points $I'$ et
$J'$.}~:
\begin{equation}
\left(\Delta t (c^2)\displaystyle\frac{\partial \rho}{\partial n}\right)_{ij}
= \Delta t (c^2)_i  \displaystyle\frac{\rho_{ij} - \rho_{I'}   }{\overline{I'F}}
=  \Delta t (c^2)_j  \displaystyle\frac{\rho_{J'}    - \rho_{ij}}{\overline{FJ'}}
\end{equation}
En \'egalant les flux \`a gauche et \`a droite de l'interface, on obtient
\begin{equation}
\rho_{ij} = \displaystyle\frac{\overline{I'F}\,(c^2)_j\rho_{J'} + \overline{FJ'}\,(c^2)_i\rho_{I'}}
{\overline{I'F}\,(c^2)_j + \overline{FJ'}\,(c^2)_i}
\end{equation}
On introduit cette formulation dans la d\'efinition du flux (par exemple, du
flux \`a gauche)~:
\begin{equation}
\left(\Delta t (c^2)\displaystyle\frac{\partial \rho}{\partial n}\right)_{ij}
= \Delta t (c^2)_i  \displaystyle\frac{\rho_{ij} - \rho_{I'}   }{\overline{I'F}}
\end{equation}
et on utilise la d\'efinition de $(c^2)_{ij}$ en fonction de ce m\^eme flux
\begin{equation}
\left(\Delta t (c^2)\displaystyle\frac{\partial \rho}{\partial n}\right)_{ij}
 \stackrel{\text{d\'ef}}{=}
 \Delta t (c^2)_{ij} \displaystyle\frac{\rho_{J'}    - \rho_{I'}   }{\overline{I'J'}}
\end{equation}
pour obtenir la valeur de $(c^2)_{ij}$ correspondant \`a l'\'equation (\ref{Cfbl_Cfmsvl_eq_harmonique_cfmsvl})~:
\begin{equation}
(c^2)_{ij} = \displaystyle\frac{\overline{I'J'}\,(c^2)_i(c^2)_j}{\overline{FJ'}\,(c^2)_i + \overline{I'F}\,(c^2)_j}
\end{equation}

%%%%%%%%%%%%%%%%%%%%%%%%%%%%%%%%%%
%%%%%%%%%%%%%%%%%%%%%%%%%%%%%%%%%%
\section*{Mise en \oe uvre}
%%%%%%%%%%%%%%%%%%%%%%%%%%%%%%%%%%
%%%%%%%%%%%%%%%%%%%%%%%%%%%%%%%%%%
Le syst\`eme (\ref{Cfbl_Cfmsvl_eq_densite_finale_cfmsvl}) est r\'esolu par une m\'ethode
d'incr\'ement et r\'esidu en utilisant
une m\'ethode de Jacobi pour inverser le syst\`eme si le terme convectif
est implicite et en utilisant une m\'ethode de gradient conjugu\'e
si le terme convectif est explicite (qui est le cas par d�faut).

Attention, les valeurs du flux de masse $\rho\,\vect{w}\cdot\vect{S}$ et
de la viscosit\'e $\Delta\,t\,c^2\frac{S}{d}$ aux faces de
bord, qui sont calcul\'ees dans \fort{cfmsfl} et \fort{cfmsvs} respectivement,
sont modifi\'ees imm\'ediatement apr\`es l'appel \`a ces sous-programmes.
En effet, il est indispensable que la contribution de bord de
$\left(\rho\,\vect{w}-\Delta\,t\,(c^2)\,\gradv\,\rho\right)\cdot\vect{S}$
repr\'esente exactement $\vect{Q}_{ac}\cdot\vect{S}$.
Pour cela,
\begin{itemize}
\item imm\'ediatement apr\`es l'appel \`a
\fort{cfmsfl}, on remplace la contribution de bord de
$\rho\,\vect{w}\cdot\vect{S}$
par le flux de masse exact, $\vect{Q}_{ac}\cdot\vect{S}$,
d\'etermin\'e \`a partir des conditions aux limites,
\item puis, imm\'ediatement apr\`es l'appel \`a
\fort{cfmsvs}, on annule la viscosit\'e au bord $\Delta\,t\,(c^2)$ pour
\'eliminer la contribution de $-\Delta\,t\,(c^2)\,(\gradv\,\rho)\cdot\vect{S}$
(l'annulation de la viscosit\'e n'est pas probl\'ematique pour la matrice,
puisqu'elle porte sur des incr\'ements).
\end{itemize}

\bigskip

Une fois qu'on a obtenu $\rho^{n+1}$,
on peut actualiser le flux de masse acoustique
aux faces $(\vect{Q}_{ac}^{n+1})_{ij} \cdot \vect{S}_{ij}$,
qui servira pour la convection des autres variables~:
\begin{equation}\label{Cfbl_Cfmsvl_eq_flux_masse_acoustique_cfmsvl}
\displaystyle(\vect{Q}_{ac}^{n+1})_{ij}\cdot\vect{S}_{ij}=
-\left(\Delta t^n (c^2)^n \gradv(\rho^{n+1})\right)_{ij}\cdot\vect{S}_{ij}
+\left(\rho^{n+\frac{1}{2}} \vect{w}^n\right)_{ij}\cdot\vect{S}_{ij}\\
\end{equation}
Ce calcul de flux est r\'ealis\'e par \fort{cfbsc3}.
Si l'on a choisi l'algorithme standard, \'equation~(\ref{Cfbl_Cfmsvl_eq_densite_cfmsvl}),
on compl\`ete le flux dans \fort{cfmsvl} imm\'ediatement apr\`es l'appel
\`a \fort{cfbsc3}.
En effet, dans ce cas,
la convection est explicite ($\rho^{n+\frac{1}{2}}=\rho^{n}$,
obtenu en imposant \var{ICONV(ISCA(IRHO))=0})
et le sous-programme \fort{cfbsc3},
qui calcule le flux de masse aux faces,
ne prend pas en compte la contribution du terme
$\rho^{n+\frac{1}{2}}\,\vect{w}^n\cdot\vect{S}$. On ajoute donc cette
contribution dans \fort{cfmsvl}, apr\`es l'appel \`a \fort{cfbsc3}.
Au bord, en particulier, c'est bien le flux de masse calcul\'e \`a partir
des conditions aux limites que l'on obtient.

On actualise la pression \`a la fin de l'\'etape, en utilisant la loi d'\'etat~:
\begin{equation}
\displaystyle P_i^{pred}=P(\rho_i^{n+1},\varepsilon_i^{n})
\end{equation}


%%%%%%%%%%%%%%%%%%%%%%%%%%%%%%%%%%
%%%%%%%%%%%%%%%%%%%%%%%%%%%%%%%%%%
\section*{Points \`a traiter}
%%%%%%%%%%%%%%%%%%%%%%%%%%%%%%%%%%
%%%%%%%%%%%%%%%%%%%%%%%%%%%%%%%%%%
Le calcul du flux de masse au  bord n'est pas enti\`erement satisfaisant
si la convection est trait\'ee de mani\`ere implicite
(algorithme non standard, non test\'e,
associ\'e \`a l'\'equation~(\ref{Cfbl_Cfmsvl_eq_densite_bis_cfmsvl}),
correspondant au choix $\rho^{n+\frac{1}{2}}=\rho^{n+1}$ et
obtenu en imposant \var{ICONV(ISCA(IRHO))=1}).
En effet, apr\`es \fort{cfmsfl}, il faut d\'eterminer la vitesse de
convection $\vect{w}^n$ pour qu'apparaisse
$\rho^{n+1} \vect{w}^n\cdot\vect{n}$
au cours de la r\'esolution par \fort{codits}. De ce fait, on doit d\'eduire
une valeur de $\vect{w}^n$ \`a partir de la valeur
du flux de masse. Au bord, en particulier, il faut
donc diviser le flux de masse
issu des conditions aux limites par la valeur de bord de $\rho^{n+1}$.
Or, lorsque des conditions de Neumann sont appliqu\'ees \`a la
masse volumique,
la valeur de $\rho^{n+1}$ au bord n'est pas connue avant la r\'esolution du
syst\`eme. On utilise donc, au lieu de la valeur de bord inconnue de
$\rho^{n+1}$ la valeur de bord prise au pas de temps
pr\'ec\'edent $\rho^{n}$. Cette approximation est susceptible
d'affecter la valeur du flux de masse au bord.

%-------------------------------------------------------------------------------

% This file is part of Code_Saturne, a general-purpose CFD tool.
%
% Copyright (C) 1998-2018 EDF S.A.
%
% This program is free software; you can redistribute it and/or modify it under
% the terms of the GNU General Public License as published by the Free Software
% Foundation; either version 2 of the License, or (at your option) any later
% version.
%
% This program is distributed in the hope that it will be useful, but WITHOUT
% ANY WARRANTY; without even the implied warranty of MERCHANTABILITY or FITNESS
% FOR A PARTICULAR PURPOSE.  See the GNU General Public License for more
% details.
%
% You should have received a copy of the GNU General Public License along with
% this program; if not, write to the Free Software Foundation, Inc., 51 Franklin
% Street, Fifth Floor, Boston, MA 02110-1301, USA.

%-------------------------------------------------------------------------------

\programme{cfqdmv}
%
\vspace{1cm}
%%%%%%%%%%%%%%%%%%%%%%%%%%%%%%%%%%
%%%%%%%%%%%%%%%%%%%%%%%%%%%%%%%%%%
\section*{Fonction}
%%%%%%%%%%%%%%%%%%%%%%%%%%%%%%%%%%
%%%%%%%%%%%%%%%%%%%%%%%%%%%%%%%%%%

Pour les notations et l'algorithme dans son ensemble,
on se reportera \`a \fort{cfbase}.

Dans le premier pas fractionnaire (\fort{cfmsvl}), on a r\'esolu une
\'equation sur la masse volumique, obtenu une pr\'ediction de la pression
et un flux convectif "acoustique".
On consid\`ere ici un second pas fractionnaire au cours duquel seul varie
le vecteur flux de masse $\vect{Q}=\rho\vect{u}$
(seule varie la vitesse au centre des cellules).
On r\'esout l'\'equation de Navier-Stokes ind\'ependamment
pour chaque direction d'espace, et l'on utilise le flux de masse acoustique
calcul\'e pr\'ec\'edemment comme flux convecteur (on pourrait aussi utiliser
le vecteur quantit\'e de mouvement du pas de temps pr\'ec\'edent).
De plus, on r\'esout en variable $\vect{u}$ et non $\vect{Q}$.

Le syst\`eme \`a r\'esoudre entre $t^*$ et $t^{**}$ est (on exclut
la turbulence, dont le traitement n'a rien de particulier dans le
module compressible)~:

\begin{equation}\label{Cfbl_Cfqdmv_eq_qdm_cfqdmv}
\left\{\begin{array}{l}

\rho^{**}=\rho^{*}=\rho^{n+1}\\
\\
\displaystyle\frac{\partial \rho\vect{u}}{\partial t}+
\divv(\vect{u} \otimes \vect{Q}_{ac}) + \gradv{P}
= \rho \vect{f}_v + \divv(\tens{\Sigma}^v)\\
\\
e^{**}=e^{*}=e^n\\

\end{array}\right.
\end{equation}

La r\'esolution de cette \'etape est similaire \`a l'\'etape
de pr\'ediction des vitesses du sch\'ema de base de \CS.

%%%%%%%%%%%%%%%%%%%%%%%%%%%%%%%%%
%%%%%%%%%%%%%%%%%%%%%%%%%%%%%%%%%%
\section*{Discr\'etisation}
%%%%%%%%%%%%%%%%%%%%%%%%%%%%%%%%%%
%%%%%%%%%%%%%%%%%%%%%%%%%%%%%%%%%%
%---------------------------------
\subsection*{Discr\'etisation en temps}
%---------------------------------

On implicite le terme de convection, �ventuellement
le gradient de pression (suivant la valeur de \var{IGRDPP}, en utilisant la
pression pr\'edite lors de l'\'etape acoustique) et le terme en gradient
du tenseur des contraintes visqueuses.
On explicite les autres termes du tenseur des contraintes visqueuses.
On implicite les forces
volumiques en utilisant $\rho^{n+1}$.

On obtient alors l'\'equation discr\`ete suivante~:
\begin{equation}\label{Cfbl_Cfqdmv_eq_vitesse_cfqdmv}
\begin{array}{l}
\displaystyle\frac{(\rho\vect{u})^{n+1}-(\rho\vect{u})^n}{\Delta t^n}
+ \divv(\vect{u}^{n+1} \otimes \vect{Q}_{ac}^{n+1})
- \divv(\mu^n \gradt{\vect{u}^{n+1}})\\
\\
\text{\ \ \ \ }= \rho^{n+1} \vect{f}_v - \gradv{\widetilde{P}}
+ \divv\left(\mu^n\ ^t\gradt{\vect{u}^n}
+ (\kappa^n-\frac{2}{3}\mu^n)\divs{\vect{u}^n}\ \tens{Id}\right)\\
\end{array}
\end{equation}
avec $\widetilde{P}=P^n\text{ ou }P^{Pred}$ suivant la valeur de \var{IGRDPP}
($P^n$ par d�faut).

En pratique, dans \CS, on r\'esout cette \'equation en faisant appara\^itre \`a
gauche l'\'ecart $\vect{u}^{n+1} - \vect{u}^n$. Pour cela, on \'ecrit la
d\'eriv\'ee en temps discr\`ete sous la forme suivante~:

\begin{equation}
\begin{array}{ll}
\displaystyle
\frac{(\rho \vect{u})^{n+1} - (\rho \vect{u})^n}{\Delta t^n}
& =
\displaystyle
\frac{\rho^{n+1}\, \vect{u}^{n+1} - \rho^n\, \vect{u}^n}{\Delta t^n}\\
& =
\displaystyle
\frac{\rho^{n}\, \vect{u}^{n+1} - \rho^n\, \vect{u}^n}{\Delta t^n}+
\frac{\rho^{n+1}\, \vect{u}^{n+1} - \rho^n\, \vect{u}^{n+1}}{\Delta t^n}\\
& =
\displaystyle
\frac{\rho^{n}}{\Delta t^n}\left(\vect{u}^{n+1} - \vect{u}^n\right)+
\vect{u}^{n+1}\frac{\rho^{n+1} - \rho^n}{\Delta t^n}
\end{array}
\end{equation}

et l'on utilise alors l'\'equation de la masse discr\`ete pour \'ecrire~:
\begin{equation}
\displaystyle
\frac{(\rho \vect{u})^{n+1} - (\rho \vect{u})^n}{\Delta t^n}
=
\frac{\rho^{n}}{\Delta t^n}\left(\vect{u}^{n+1} - \vect{u}^n\right)-
\vect{u}^{n+1}\dive\,\vect{Q}_{ac}^{n+1}
\end{equation}



%---------------------------------
\subsection*{Discr\'etisation en espace}
%---------------------------------

%.................................
\subsubsection*{Introduction}
%.................................

On int\`egre l'\'equation (\ref{Cfbl_Cfqdmv_eq_vitesse_cfqdmv})
sur la cellule $i$ de volume $\Omega_i$ et
on obtient l'\'equation discr\'etis\'ee en espace~:

\begin{equation}\label{Cfbl_Cfqdmv_eq_vitesse_discrete_cfqdmv}
\begin{array}{l}
\displaystyle\frac{\Omega_i}{\Delta t^n}
(\rho_i^{n+1}\vect{u}_i^{n+1}-\rho_i^n\vect{u}_i^n)
+ \displaystyle\sum\limits_{j\in V(i)}
(\vect{u}^{n+1} \otimes \vect{Q}_{ac}^{n+1})_{ij} \cdot \vect{S}_{ij}
- \displaystyle\sum\limits_{j\in V(i)}
\left(\mu^n\gradt{\vect{u}^{n+1}}\right)_{ij} \cdot \vect{S}_{ij}\\
\\
= \Omega_i\rho_i^{n+1} {\vect{f}_v}_i
- \Omega_i(\gradv{\widetilde{P}})_i
+ \displaystyle\sum\limits_{j\in V(i)}
\left(\mu^n\ ^t\gradt{\vect{u}^n} + (\kappa^n-\frac{2}{3}\mu^n)
\divs{\vect{u}^n}\ \tens{Id}\right)_{ij}\vect{S}_{ij}\\
\end{array}
\end{equation}

%.................................
\subsubsection*{Discr\'etisation de la partie ``convective''}
%.................................

La valeur \`a la face s'\'ecrit~:
\begin{equation}
(\vect{u}^{n+1} \otimes \vect{Q}_{ac}^{n+1})_{ij}\cdot \vect{S}_{ij}
= \vect{u}_{ij}^{n+1}(\vect{Q}_{ac}^{n+1})_{ij} \cdot \vect{S}_{ij}
\end{equation}
avec un d\'ecentrement sur la valeur de $\vect{u}^{n+1}$ aux faces~:
\begin{equation}
\begin{array}{lllll}
\vect{u}_{ij}^{n+1}
& = & \vect{u}_i^{n+1}
& \text{si\ } & (\vect{Q}_{ac}^{n+1})_{ij} \cdot \vect{S}_{ij} \geqslant 0 \\
& = & \vect{u}_j^{n+1}
& \text{si\ } & (\vect{Q}_{ac}^{n+1})_{ij} \cdot \vect{S}_{ij} < 0 \\
\end{array}
\end{equation}
que l'on peut noter~:
\begin{equation}
\vect{u}_{ij}^{n+1}
 = \beta_{ij}\vect{u}_i^{n+1} + (1-\beta_{ij})\vect{u}_j^{n+1}
\end{equation}
avec
\begin{equation}
\left\{\begin{array}{lll}
\beta_{ij} = 1 & \text{si\ }
& (\vect{Q}_{ac}^{n+1})_{ij} \cdot \vect{S}_{ij} \geqslant 0 \\
\beta_{ij} = 0 & \text{si\ }
& (\vect{Q}_{ac}^{n+1})_{ij} \cdot \vect{S}_{ij} < 0 \\
\end{array}\right.
\end{equation}

%.................................
\subsubsection*{Discr\'etisation de la partie ``diffusive''}
%.................................

La valeur \`a la face s'\'ecrit~:
\begin{equation}
\left(\mu^n\gradt{\vect{u}^{n+1}}\right)_{ij}\vect{S}_{ij}
= \mu_{ij}^n
\displaystyle \left( \frac{\partial \vect{u}}{\partial n} \right)^{n+1}_{ij}
S_{ij}
\end{equation}
avec une interpolation lin\'eaire pour $\mu^n$ aux faces (en pratique avec
$\alpha_{ij}=\frac{1}{2}$)~:
\begin{equation}
\mu_{ij}^n
= \alpha_{ij}\mu_{i}^n+(1-\alpha_{ij})\mu_{j}^n
\end{equation}
et un sch\'ema centr\'e pour le gradient normal aux faces~:
\begin{equation}
\displaystyle \left( \frac{\partial \vect{u}}{\partial n} \right)^{n+1}_{ij}
= \displaystyle\frac{\vect{u}_{J'}^{n+1}-\vect{u}_{I'}^{n+1}}{\overline{I'J'}}
\end{equation}

%.................................
\subsubsection*{Discr\'etisation du gradient de pression}
%.................................

On utilise \fort{grdcel} standard. Suivant la valeur de \var{IMRGRA},
cela correspond � une reconstruction it�rative ou par moindres carr�s.

%.................................
\subsubsection*{Discr\'etisation du ``reste'' du tenseur des contraintes visqueuses}
%.................................

On calcule des gradients aux cellules et on utilise une
interpolation lin\'eaire aux
faces (avec, en pratique, $\alpha_{ij}=\frac{1}{2}$)~:
\begin{equation}
\begin{array}{r}
\left(\mu^n\ ^t\gradt{\vect{u}^n} + (\kappa^n-\frac{2}{3}\mu^n)
\divs{\vect{u}^n}\ \tens{Id}\right)_{ij}\cdot\vect{S}_{ij}
= \left\{\alpha_{ij} \left(\mu^n\ ^t\gradt{\vect{u}^n}
+ (\kappa^n-\frac{2}{3}\mu^n)\divs{\vect{u}^n}\ \tens{Id}\right)_i\right.\\
\\
\left.+ (1-\alpha_{ij}) \left(\mu^n\ ^t\gradt{\vect{u}^n}
+ (\kappa^n-\frac{2}{3}\mu^n)\divs{\vect{u}^n}\ \tens{Id}\right)_j
\right\} \cdot\vect{S}_{ij}\\
\end{array}
\end{equation}

%%%%%%%%%%%%%%%%%%%%%%%%%%%%%%%%%%
%%%%%%%%%%%%%%%%%%%%%%%%%%%%%%%%%%
\section*{Mise en \oe uvre}
%%%%%%%%%%%%%%%%%%%%%%%%%%%%%%%%%%
%%%%%%%%%%%%%%%%%%%%%%%%%%%%%%%%%%
On r\'esout les trois directions d'espace du syst\`eme
(\ref{Cfbl_Cfqdmv_eq_vitesse_discrete_cfqdmv}) successivement et ind\'ependamment~:
\begin{equation}\label{Cfbl_Cfqdmv_eq_vitesse_finale_cfqdmv}
\left\{\begin{array}{l}
\displaystyle\frac{\Omega_i}{\Delta t^n}
(\rho_i^{n+1}{u_i}_{(\alpha)}^{n+1}-\rho_i^n{u_i}_{(\alpha)}^n)
+ \displaystyle\sum\limits_{j\in V(i)}
{u_{ij}}_{(\alpha)}^{n+1}(\vect{Q}_{ac}^{n+1})_{ij}\cdot\vect{S}_{ij}
- \displaystyle\sum\limits_{j\in V(i)}
\mu_{ij}^n\frac{{u_j}_{(\alpha)}^{n+1}-{u_i}_{(\alpha)}^{n+1}}{\overline{I'J'}}S_{ij}\\
\qquad\qquad\qquad\qquad= \Omega_i\rho_i^{n+1} {{f_v}_i}_{(\alpha)}
- \Omega_i{(\gradv{\widetilde{P}})_{i}}_{(\alpha)}\\
\qquad\qquad\qquad\qquad + \displaystyle\sum\limits_{j\in V(i)}
\left((\mu^n\ ^t\gradt{\vect{u}^n})_{ij}\cdot\vect{S}_{ij}\right)_{(\alpha)}
 + \displaystyle\sum\limits_{j\in V(i)} \left((\kappa^n-\frac{2}{3}\mu^n)
\divs{\vect{u}^n}\right)_{ij}{S_{ij}}_{(\alpha)}\\
i = 1\ldots N \qquad \text{et} \qquad (\alpha) = x,\ y,\ z\\
\end{array}\right.
\end{equation}

Chaque syst\`eme associ\'e \`a une direction est r\'esolu par une m\'ethode
d'incr\'ement et r\'esidu en utilisant une m\'ethode de Jacobi.


%%%%%%%%%%%%%%%%%%%%%%%%%%%%%%%%%%
%%%%%%%%%%%%%%%%%%%%%%%%%%%%%%%%%%
%\section*{Points \`a traiter}
%%%%%%%%%%%%%%%%%%%%%%%%%%%%%%%%%%
%%%%%%%%%%%%%%%%%%%%%%%%%%%%%%%%%%

% propose en patch 1.2.1

%Compl\'eter le commentaire en ent\^ete de \fort{visecv} pour prendre en compte
%la viscosit\'e en volume.

%-------------------------------------------------------------------------------

% This file is part of code_saturne, a general-purpose CFD tool.
%
% Copyright (C) 1998-2025 EDF S.A.
%
% This program is free software; you can redistribute it and/or modify it under
% the terms of the GNU General Public License as published by the Free Software
% Foundation; either version 2 of the License, or (at your option) any later
% version.
%
% This program is distributed in the hope that it will be useful, but WITHOUT
% ANY WARRANTY; without even the implied warranty of MERCHANTABILITY or FITNESS
% FOR A PARTICULAR PURPOSE.  See the GNU General Public License for more
% details.
%
% You should have received a copy of the GNU General Public License along with
% this program; if not, write to the Free Software Foundation, Inc., 51 Franklin
% Street, Fifth Floor, Boston, MA 02110-1301, USA.

%-------------------------------------------------------------------------------

\programme{cfxtcl}

\hypertarget{cfxtcl}{}

\vspace{1cm}
%%%%%%%%%%%%%%%%%%%%%%%%%%%%%%%%%%
%%%%%%%%%%%%%%%%%%%%%%%%%%%%%%%%%%
\section*{Fonction}
%%%%%%%%%%%%%%%%%%%%%%%%%%%%%%%%%%
%%%%%%%%%%%%%%%%%%%%%%%%%%%%%%%%%%

Pour le traitement des conditions aux limites, on considère
le système (\ref{Cfbl_Cfxtcl_eq_ref_laminaire_cfxtcl})

\begin{equation}\label{Cfbl_Cfxtcl_eq_ref_laminaire_cfxtcl}
\left\{\begin{array}{l}

\displaystyle\frac{\partial\rho}{\partial t} + \divs(\vect{Q}) = 0 \\
\\
\displaystyle\frac{\partial\vect{Q}}{\partial t}
+ \divv(\vect{u} \otimes \vect{Q}) + \gradv{P}
= \rho \vect{f}_v + \divv(\tens{\Sigma}^v) \\
\\
\displaystyle\frac{\partial E}{\partial t} + \divs( \vect{u} (E+P) )
= \rho\vect{f}_v\cdot\vect{u} + \divs(\tens{\Sigma}^v \vect{u})
- \divs{\,\vect{\Phi}_s} + \rho\Phi_v

\end{array}\right.
\end{equation}

en tant que système hyperbolique portant sur la variable vectorielle
$\vect{W}=\ ^t(\rho,\vect{Q},E)$.

Le système s'écrit alors~:
\begin{equation}\label{Cfbl_Cfxtcl_eq_hyperbolique_cfxtcl}
\displaystyle\frac{\partial \vect{W}}{\partial t}
+ \displaystyle\sum\limits_{i=1}^3
\frac{\partial}{\partial x_i}\vect{F}_i(\vect{W})
= \displaystyle\sum\limits_{i=1}^3
\frac{\partial}{\partial x_i}\vect{F}_i^D(\vect{W},\nabla \vect{W})
+ \vect{\mathcal{S}}
\end{equation}
où les $\vect{F}_i(\vect{W})$ sont les vecteurs flux convectifs
et les $\vect{F}_i^D(\vect{W})$ sont les vecteurs flux diffusifs
dans les trois directions d'espace,
et $\vect{\mathcal{S}}$ est un terme source.

La démarche classique de \CS est adoptée~: on impose les conditions
aux limites en déterminant, pour chaque variable, des valeurs numériques
de bord. Ces valeurs sont calculées de telle façon que, lorsqu'on
les utilise dans les formules standard donnant les flux discrets, on obtienne
les contributions souhaitées au bord.

Pour rendre compte des flux convectifs (aux entrées et aux sorties en particulier),
on fait abstraction des flux diffusifs et des termes
sources pour résoudre un problème de Riemann qui
fournit un vecteur d'état au bord. Celui-ci permet de calculer un flux,
soit directement (par les formules discrètes standard),
soit en appliquant un schéma de Rusanov (schéma de flux décentré).

En paroi, on résout  également, dans certains cas, un problème de
Riemann pour déterminer une pression au bord.

See the \doxygenfile{cfxtcl_8f90.html}{programmers reference of the dedicated subroutine} for further details.

%%%%%%%%%%%%%%%%%%%%%%%%%%%%%%%%%%
%%%%%%%%%%%%%%%%%%%%%%%%%%%%%%%%%%
\section*{Discrétisation}
%%%%%%%%%%%%%%%%%%%%%%%%%%%%%%%%%%
%%%%%%%%%%%%%%%%%%%%%%%%%%%%%%%%%%

%=================================
\subsection*{Introduction}
%=================================

%---------------------------------
\subsubsection*{Objectif}
%---------------------------------

On résume ici les différentes conditions aux limites utilisées pour l'algorithme
compressible afin de fournir une vue d'ensemble. Pour atteindre cet objectif,
il est nécessaire de faire référence à  des éléments relatifs
à la discrétisation et au mode d'implantation des conditions aux limites.

Lors de l'implantation, on a cherché à préserver la cohérence avec l'approche
utilisée dans le cadre standard de l'algorithme incompressible de \CS.
Il est donc conseillé d'avoir pris connaissance du mode de traitement des conditions
aux limites incompressibles avant d'aborder les détails de l'algorithme compressible.

Comme pour l'algorithme incompressible, les conditions aux limites sont imposées
par le biais d'une valeur de bord associée à chaque variable. De plus,
pour certaines
frontières (parois à température imposée ou à flux thermique imposé),
on dispose de deux valeurs de bord pour la même variable, l'une d'elles étant dédiée au calcul du
flux diffusif.
Enfin, sur certains types d'entrée et de sortie, on définit également
une valeur du flux convectif au bord.

Comme pour l'algorithme incompressible, l'utilisateur peut définir,
pour chaque face
de bord, des conditions aux limites pour chaque variable, mais on conseille cependant
d'utiliser uniquement les types prédéfinis
décrits ci-après (entrée,
sortie, paroi, symétrie) qui ont l'avantage d'assurer la cohérence entre
les différentes variables et les différentes étapes de calcul.


%---------------------------------
\subsubsection*{Parois}
%---------------------------------

{\bf Pression} : on doit disposer d'une condition pour le calcul du gradient
qui intervient dans l'étape de quantité de mouvement.
On dispose de deux types de condition, au choix de l'utilisateur~:
\begin{itemize}
\item par défaut, la pression imposée au bord est proportionnelle
à la valeur interne (la pression au bord est obtenue comme solution
d'un problème de Riemann sur les équations d'Euler
avec un état miroir~; on distingue les cas de choc et de
détente et, dans le cas d'une détente trop forte, une condition de
Dirichlet homogène est utilisée pour éviter de voir apparaître une
pression négative),
\item si l'utilisateur le souhaite (\var{ICFGRP=1}), le gradient de
pression est imposé à partir du profil de pression hydrostatique.
\end{itemize}
\bigskip

{\bf Vitesse et turbulence}~: traitement standard (voir la documentation des
sous-programmes \fort{cs\_boundary\_conditions}
et \fort{cs\_boundary\_conditions\_set\_coeffs\_turb}).

{\bf Scalaires passifs}~: traitement standard
(flux nul par défaut imposé dans \fort{typecl}).

{\bf Masse volumique}~: traitement standard des scalaires
(flux nul par défaut imposé dans \fort{typecl}).

{\bf Énergie et température\footnote{Le gradient de température est
{\it a priori} inutile, mais peut être requis par l'utilisateur.}}~:
traitement standard des scalaires
(flux nul par défaut imposé dans \fort{typecl}), hormis pour le
calcul du flux diffusif dans le cas de parois à température imposée
ou à flux thermique imposé.

{\bf Flux diffusif pour l'énergie en paroi}~:
l'utilisateur peut choisir (dans \fort{uscfcl}) entre une température de paroi imposée
et un flux thermique diffusif (ou "conductif") imposé.
S'il ne précise rien, on considère que la paroi est adiabatique
(flux thermique diffusif imposé et de valeur nulle).
Dans tous les cas, il faut donc disposer d'un moyen d'imposer le flux diffusif
souhaité. Pour cela, on détermine une valeur de bord pour l'énergie
qui, introduite dans la formule donnant le flux discret, permettra
d'obtenir la contribution attendue
(voir le paragraphe~\ref{Cfbl_Cfxtcl_section_cl_flux_diffusif_energie_cfener}).
Conformément à l'approche classique de \CS, cette valeur est
stockée sous la forme d'un couple de coefficients
(de type \var{COEFAF}, \var{COEFBF}).
Il est important de souligner que cette valeur de bord
ne doit être utilisée que pour le calcul
du flux diffusif~: dans les autres situations pour lesquelles
une valeur de bord de l'énergie ou de la température est requise
(calcul de gradient par exemple), on utilise une condition de flux nul
(traitement standard des scalaires). Pour cela, on dispose d'une
seconde valeur de bord qui est stockée au moyen d'un
couple de coefficients (\var{COEFA}, \var{COEFB}) distinct du précédent.

{\bf Flux convectifs}~: le flux de masse dans la direction normale à la paroi est
pris nul. De ce fait, les flux convectifs seront nuls quelle que soit les valeurs
de bord imposées pour les différentes variables transportées.

%---------------------------------
\subsubsection*{Symétrie}
%---------------------------------

Les conditions appliquées sont les conditions classiques de l'algorithme
incompressible (vitesse normale nulle, flux nul pour les autres variables).

Elles sont imposées dans le sous-programme \fort{typecl} essentiellement. Pour
la pression, la condition de flux nul est imposée dans \fort{cfxtcl}
(au début des développements, on appliquait le même traitement qu'en paroi,
mais une condition de flux nul a été préférée afin de s'affranchir des
problèmes potentiels dans les configurations 2D).


%---------------------------------
\subsubsection*{Entrées et sorties}
%---------------------------------

On obtient, par résolution d'un problème de Riemann au bord, complété par
des relations de thermodynamique (\fort{uscfth}), des valeurs de bord pour toutes
les variables (on suppose qu'en entrée, toutes les composantes de la vitesse
sont fournies~; elles sont supposées nulles par défaut, hormis pour les
entrées à $(\rho,\vect{u})$ imposés, \var{IERUCF}, pour lesquelles il faut
fournir la vitesse explicitement).

Ces valeurs de bord sont utilisées de deux façons~:
\begin{itemize}
\item elles sont utilisées pour calculer les flux convectifs, en faisant
appel au schéma de Rusanov (sauf en sortie supersonique)~; ces flux sont
directement intégrés au second membre des équations à résoudre.
\item elles servent de valeur de Dirichlet dans toutes les autres configurations
pour lesquelles une valeur de bord est requise (calcul de flux diffusif,
calcul de gradient...)
\end{itemize}

Deux cas particuliers~:
\begin{itemize}
\item aux entrées ou sorties pour lesquelles toutes les variables sont imposées
(\var{IESICF}), on utilise une condition de Neumann homogène pour la pression
(hormis pour le calcul du gradient intervenant dans l'équation de la quantité
de mouvement, qui est pris en compte par le flux convectif déterminé par le
schéma de Rusanov). Ce choix est arbitraire (on n'a pas testé le comportement
de l'algorithme si l'on conserve une condition de Dirichlet sur la pression), mais
a été fait en supposant qu'une condition de Neumann homogène serait {\it a
priori} moins déstabilisante, dans la mesure où, pour ce type de frontière,
l'utilisateur peut imposer une valeur de pression très différente de
celle régnant à l'intérieur du domaine (la valeur imposée est utilisée
pour le flux convectif).
\item pour les grandeurs turbulentes et les scalaires utilisateur, si
le flux de masse est entrant et que l'on a fourni
une valeur de Dirichlet (\var{RCODCL(*,*,1)} dans \fort{uscfcl}),
on l'utilise, pour le calcul du flux convectif et du flux diffusif~;
sinon, on utilise une condition de Neumann homogène (le concept
de sortie de type 9 ou 10 est couvert par cette approche).
\end{itemize}



%=================================
\subsection*{Problème de Riemann au bord}
\label{Cfbl_Cfxtcl_section_pb_riemann_cfener}
%=================================

%---------------------------------
\subsubsection*{Introduction}
%---------------------------------

On cherche à obtenir un état au bord,
pour les entrées, les sorties et les parois.

Pour cela, on fait abstraction des flux diffusifs et des sources.
Le système résultant est alors appelé système
d'équations d'Euler. On se place de plus dans un
repère orienté suivant la normale au bord considéré
$(\vect{\tau}_1,\vect{\tau}_2,\vect{n})$
et l'on ne considère que les variations suivant cette normale.
Le système devient donc~:
\begin{equation}\label{Cfbl_Cfxtcl_eq_euler_cfxtcl}
\begin{array}{lllll}
\displaystyle\frac{\partial \vect{W}}{\partial t}
+ \frac{\partial}{\partial n}\vect{F}_n(\vect{W})
= 0
&\text{avec}
& \vect{F}_n(\vect{W})
 = \displaystyle\sum\limits_{i=1}^3 n_i \vect{F}_i(\vect{W})
& \text{et}
& \displaystyle\frac{\partial}{\partial n}
= \displaystyle\sum\limits_{i=1}^3 n_i \frac{\partial}{\partial x_i}
\end{array}
\end{equation}

Pour déterminer les valeurs des variables au bord, on recherche
l'évolution du problème instationnaire suivant,
appelé problème de Riemann~:

\unitlength=1cm
\begin{picture}(20,2.6)
\put(1.5,0){\framebox(12,2.5){}}
\put(7.5,0){\line(0,1){2.5}}
\put(7.5,2.2){bord}
\put(7.5,1.7){\vector(1,0){0.7}}
\put(8.25,1.65){$\vect{n}$}
\multiput(7.5,0)(0,0.5){4}{\line(2,3){.4}}
\put(2,1.2){$\begin{array}{c}
\text{intérieur}\\
\vect{W}_{\,i}\\
\text{état constant dans la cellule $i$}
\end{array}$}
\put(9.5,1.2){$\begin{array}{c}
\text{extérieur}\\
\vect{W}_{\,\infty}\\
\text{état constant}
\end{array}$}
\end{picture}

avec $\vect{W}_{\,\infty}$ dépendant du type de bord et différent
de $\vect{W}_{\,i}$ {\it a priori}.

\vspace{0.3cm}

Pour résoudre ce problème de Riemann, on utilisera les variables
non-conservatives $\widetilde{\vect{W}}=\ ^t(\rho, \vect{u}, P)$
et l'on retrouvera l'énergie grâce à l'équation d'état.

Pour alléger l'écriture, dans le présent
paragraphe~\ref{Cfbl_Cfxtcl_section_pb_riemann_cfener},
on notera aussi $\vect{W}$ le vecteur
$^t(\rho, \vect{u}, P)$ et $\vect{u} = \vect{u}_\tau + u\,\vect{n}$
(en posant $u=\vect{u} \cdot \vect{n}$
et $\vect{u}_\tau = \vect{u} - (\vect{u} \cdot \vect{n})\vect{n}$).

La solution est une suite d'états constants, dont les valeurs
dépendent de $\vect{W}_{\,i}$ et $\vect{W}_{\,\infty}$,
séparés par des ondes se déplaçant à des vitesses données
par les valeurs propres du système $(\lambda_i)_{i=1\ldots 5}$.
On représente les caractéristiques du système sur le schéma suivant~:

\unitlength=1cm
\begin{picture}(20,3.5)
\put(3,0){\vector(1,0){8}}
\put(7,0){\vector(0,1){3}}
\put(11,0.1){$x$}
\put(6.8,2.9){$t$}
\put(7,3.1){bord}
\put(7,2.6){\vector(1,0){0.5}}
\put(7.55,2.65){$\vect{n}$}
\multiput(7,0)(0,0.5){5}{\line(2,3){.4}}
\put(7,0){\line(-1,2){1.4}}
\put(4,2.9){$\lambda_1=u-c$}
\put(7,0){\qbezier[15](0,0)(0.7,1.4)(1.4,2.8)}
\put(8.3,2.9){$\lambda_{2,3,4}=u$}
\put(7,0){\line(2,1){3}}
\put(9.5,1.6){$\lambda_5=u+c$}
\put(5,1.5){$\vect{W}_{\,i}$}
\put(6.3,2){$\vect{W}_{\,1}$}
\put(8.4,1.6){$\vect{W}_{\,2}$}
\put(9.5,0.5){$\vect{W}_{\,\infty}$}
\end{picture}

Comme valeurs des variables au bord, on prendra les valeurs correspondant à
l'état constant qui contient le bord ($\vect{W}_1$ dans l'exemple
précédent).

Il faut remarquer que la solution du problème de Riemann dépend de la
thermodynamique et devra donc être calculée et codée
par l'utilisateur si la thermodynamique n'a pas été prévue
(en version 1.2, la seule
thermodynamique prévue est celle des gaz parfaits).



%---------------------------------
\subsubsection*{En paroi, pour la condition de pression (sans effet de gravité)}
%---------------------------------

Pour les faces de paroi, on définit à l'extérieur du domaine
un état miroir $\vect{W}_{\,\infty}$ par~:
\begin{equation}\label{Cfbl_Cfxtcl_eq_paroi_cfxtcl}
\begin{array}{lllll}
\vect{W}_{\,i} &=&
\left(\begin{array}{l}
\rho_i\\ {\vect{u}_\tau}_i\\ u_i\\ P_i
\end{array}\right)
\qquad
\vect{W}_{\,\infty} &=&
\left(\begin{array}{lll}
\rho_\infty &=& \rho_i\\
{\vect{u}_\tau}_\infty &=& {\vect{u}_\tau}_i\\
u_\infty &=& -u_i\\
P_\infty &=& P_i
\end{array}\right)
\end{array}
\end{equation}

\vspace{0.5cm}

Les solutions dépendent de l'orientation de la vitesse dans la cellule
de bord~:

\vspace{0.5cm}

\begin{enumerate}

\item Si $u_i \leqslant 0$,
la solution est une double détente symétrique.

\unitlength=1cm
\begin{picture}(20,4)
\put(0,0.5){\vector(1,0){8}}
\put(4,0.5){\vector(0,1){3}}
\put(8,0.6){$x$}
\put(3.8,3.4){$t$}
\put(4,3.7){paroi}
\put(4,3.2){\vector(1,0){0.5}}
\put(4.55,3.15){$\vect{n}$}
\multiput(4,0.5)(0,0.5){5}{\line(2,3){.4}}
\put(4,0.5){\line(-1,1){2.5}}
\put(4,0.5){\line(-5,4){2.8}}
\put(4,0.5){\line(-4,5){2}}
\put(0,3.1){$\lambda_1=u-c\ (<0)$}
\put(4.2,2.7){$\lambda_{2,3,4}=0$}
\put(4,0.5){\line(1,1){2.5}}
\put(4,0.5){\line(5,4){2.8}}
\put(4,0.5){\line(4,5){2}}
\put(6.5,3.1){$\lambda_5=u+c\ (>0)$}
\put(1.5,1.5){$\vect{W}_{\,i}$}
\put(3.1,2){$\vect{W}_{\,1}$}
\put(4.5,2){$\vect{W}_{\,2}$}
\put(6,1.5){$\vect{W}_{\,\infty}$}
\put(8.5,2){$\vect{W}_{\,paroi} = \vect{W}_{\,1} = \vect{W}_{\,2}$}
\put(12,2)
{$\left\{\begin{array}{l}
\rho_p = \rho_1 = \rho_2\\
u_p = u_1 = u_2\\
P_p = P_1 = P_2
\end{array}\right.$}
\end{picture}

La conservation de la vitesse tangentielle à travers la 1-onde donne
${\vect{u}_\tau}_p = {\vect{u}_\tau}_i$.
Par des considérations de symétrie on trouve $u_p = 0$.
Puis on obtient $\rho_p$ et $P_p$ en écrivant la conservation
des invariants de Riemann à travers la 1-détente~:
\begin{equation}\label{Cfbl_Cfxtcl_eq_invariants_detente_cfxtcl}
\begin{array}{lll}
\left\{\begin{array}{l}
u_1 + \displaystyle\int_0^{\rho_1} \frac{c}{\rho} d\rho
= u_i + \displaystyle\int_0^{\rho_i} \frac{c}{\rho} d\rho\\
\\
s_1 = s_i
\end{array}\right.
&
\Rightarrow
\left\{\begin{array}{ll}
\displaystyle\int_{\rho_i}^{\rho_1} \frac{c}{\rho} d\rho = u_i
& \Rightarrow \rho_p=\rho_1\\
\\
s(P_1,\rho_1) = s(P_i,\rho_i)
& \Rightarrow P_p=P_1
\end{array}\right.
\end{array}
\end{equation}

\vspace{0.5cm}

\item  Si $u_i > 0$,
la solution est un double choc symétrique.

\unitlength=1cm
\begin{picture}(20,4)
\put(0,0.5){\vector(1,0){8}}
\put(4,0.5){\vector(0,1){3}}
\put(8,0.6){$x$}
\put(3.8,3.4){$t$}
\put(4,3.7){paroi}
\put(4,3.2){\vector(1,0){0.5}}
\put(4.55,3.15){$\vect{n}$}
\multiput(4,0.5)(0,0.5){5}{\line(2,3){.4}}
\put(4,0.5){\line(-1,1){2.5}}
\put(0,3.1){$\lambda_1=u-c\ (<0)$}
\put(4.4,2.7){$\lambda_{2,3,4}=0$}
\put(4,0.5){\line(1,1){2.5}}
\put(6.5,3.1){$\lambda_5=u+c\ (>0)$}
\put(1.5,1.5){$\vect{W}_{\,i}$}
\put(3,2){$\vect{W}_{\,1}$}
\put(4.7,2){$\vect{W}_{\,2}$}
\put(6,1.5){$\vect{W}_{\,\infty}$}
\put(8.5,2){$\vect{W}_{\,paroi} = \vect{W}_{\,1} = \vect{W}_{\,2}$}
\put(12,2)
{$\left\{\begin{array}{l}
\rho_p = \rho_1 = \rho_2\\
u_p = u_1 = u_2\\
P_p = P_1 = P_2
\end{array}\right.$}
\end{picture}

De même que précédemment,
on trouve ${\vect{u}_\tau}_p = {\vect{u}_\tau}_i$ et $u_p = 0$,
puis $\rho_p$ et $P_p$ en écrivant les relations de saut
à travers le 1-choc~:
\begin{equation}\label{Cfbl_Cfxtcl_eq_saut_choc_cfxtcl}
\begin{array}{lll}
\left\{\begin{array}{l}
\rho_1 \rho_i (u_1 - u_i)^2
= (P_1 - P_i)(\rho_1 - \rho_i)\\
\\
2\rho_1 \rho_i (\varepsilon_1 - \varepsilon_i)
= (P_1 + P_i)(\rho_1 - \rho_i)
\end{array}\right.
& \text{avec } \varepsilon = \varepsilon(P,\rho)
&
\Rightarrow
\left\{\begin{array}{l}
\rho_p=\rho_1\\
\\
P_p=P_1
\end{array}\right.
\end{array}
\end{equation}

\bigskip
Pour les gaz parfaits,
avec $M_i = \displaystyle\frac{\vect{u}_i \cdot \vect{n}}{c_i}$
(Nombre de Mach de paroi), on a~:

\begin{itemize}

\item Cas détente ($M_i \leqslant 0$)~:\\
$$
\begin{array}{l}
\left\{\begin{array}{lll}
P_p=0 & \text{si} & 1 + \displaystyle\frac{\gamma-1}{2}M_i<0\\
P_p = P_i \left(1 + \displaystyle\frac{\gamma-1}{2}M_i\right)
^{\frac{2\gamma}{\gamma-1}} & \text{sinon}\\
\end{array}\right.\\
\\
\rho_p=\rho_i \left(\displaystyle\frac{P_p}{P_i}\right)^{\frac{1}{\gamma}}\\
\end{array}
$$

\item Cas choc ($M_i > 0$)~:\\
$$
\begin{array}{l}
P_p = P_i \left(1 + \displaystyle\frac{\gamma(\gamma+1)}{4}M_i^2
+\gamma M_i \displaystyle\sqrt{1+\displaystyle\frac{(\gamma+1)^2}{16}M_i^2}\right)
\\
\\
\rho_p=\rho_i \left(\displaystyle\frac{P_p-P_i}
{P_p-P_i-\rho_i (\vect{u}_i \cdot \vect{n})^2}\right)\\
\end{array}
$$

\end{itemize}

\end{enumerate}

En pratique, le flux convectif normal à la paroi est nul et seule
la condition de pression déterminée ci-dessus est effectivement
utilisée (pour le calcul du gradient sans effet de gravité).

%---------------------------------
\subsubsection*{En sortie}
%---------------------------------

Il existe deux cas de traitement des conditions en sortie,
selon le nombre de Mach normal à la face de bord
($c_i$ est la vitesse du son dans la cellule de bord)~:
$$M_i = \displaystyle\frac{u_i}{c_i}
= \displaystyle\frac{\vect{u}_i \cdot \vect{n}}{c_i}$$

\paragraph{Sortie supersonique (condition ISSPCF de
\fort{uscfcl})~:}
$M_i \geqslant 1 \Rightarrow u_i - c_i \geqslant 0$
\nopagebreak
\linebreak
\unitlength=1cm
\begin{picture}(20,4.5)
\put(0,0.5){\vector(1,0){8}}
\put(4,0.5){\vector(0,1){3}}
\put(8,0.6){$x$}
\put(3.8,3.4){$t$}
\put(4,3.7){bord}
\put(4,3.2){\vector(1,0){0.5}}
\put(4.55,3.15){$\vect{n}$}
\multiput(4,0.5)(0,0.5){5}{\line(2,3){.4}}
\put(4,0.5){\line(1,2){1.4}}
\put(5.3,3.4){$\lambda_1=u-c\ (\geqslant 0)$}
\put(4,0.5){\qbezier[20](0,0)(1.2,1.2)(2.4,2.4)}
\put(6.5,2.9){$\lambda_{2,3,4}=u$}
\put(4,0.5){\line(2,1){3}}
\put(7.1,2){$\lambda_5=u+c$}
\put(2,2){$\vect{W}_{\,i}$}
\put(5.2,2.5){$\vect{W}_{\,1}$}
\put(6,2){$\vect{W}_{\,2}$}
\put(6.5,1){$\vect{W}_{\,\infty}$}
\put(10,2){$\vect{W}_{\,bord} = \vect{W}_{\,i}$}
\end{picture}

Toutes les caractéristiques sont sortantes,
on connaît donc toutes les conditions au bord~:

\begin{equation}
\left\{\begin{array}{l}
\rho_b = \rho_i\\
{\vect{u}_\tau}_b = {\vect{u}_\tau}_i\\
u_b = u_i\\
P_b = P_i
\end{array}\right.
\end{equation}

\paragraph{Sortie subsonique (condition ISOPCF de
\fort{uscfcl})~:}
$0 \leqslant M_i < 1 \Rightarrow (u_i \geqslant 0 \text{ et } u_i - c_i < 0)$

\unitlength=1cm
\begin{picture}(20,4.5)
\put(0,0.5){\vector(1,0){8}}
\put(4,0.5){\vector(0,1){3}}
\put(8,0.6){$x$}
\put(3.8,3.4){$t$}
\put(4,3.7){bord}
\put(4,3.2){\vector(1,0){0.5}}
\put(4.55,3.15){$\vect{n}$}
\multiput(4,0.5)(0,0.5){5}{\line(2,3){.4}}
\put(4,0.5){\line(-1,2){1.4}}
\put(1,3.4){$\lambda_1=u-c\ (<0)$}
\put(4,0.5){\qbezier[15](0,0)(0.7,1.4)(1.4,2.8)}
\put(5.3,3.4){$\lambda_{2,3,4}=u\ (\geqslant 0)$}
\put(4,0.5){\line(2,1){3}}
\put(6.5,2.1){$\lambda_5=u+c$}
\put(2,2){$\vect{W}_{\,i}$}
\put(3.3,2.5){$\vect{W}_{\,1}$}
\put(5.4,2.1){$\vect{W}_{\,2}$}
\put(6.5,1){$\vect{W}_{\,\infty}$}
\put(10,2){$\vect{W}_{\,bord} = \vect{W}_{\,1}$}
\put(12.5,2)
{$\left\{\begin{array}{l}
\rho_b = \rho_1\\
u_b = u_1\\
P_b = P_1
\end{array}\right.$}
\end{picture}

On a une caractéristique entrante,
on doit donc imposer une seule condition  au bord
(en général la pression de sortie $P_{ext}$).

On connaît alors $P_b = P_{ext}$ et ${\vect{u}_\tau}_b = {\vect{u}_\tau}_i$
(par conservation de la vitesse tangentielle à travers la 1-onde).
Pour trouver les inconnues manquantes ($\rho_b$ et $u_b$)
on doit résoudre le passage de la 1-onde~:

\begin{enumerate}

\item Si $P_{ext} \leqslant P_i$,
on a une 1-détente.

On écrit la conservation
des invariants de Riemann à travers la 1-détente~:
\begin{equation}
\begin{array}{lll}
\left\{\begin{array}{l}
s_1 = s_i\\
\\
u_1 + \displaystyle\int_0^{\rho_1} \frac{c}{\rho} d\rho
= u_i + \displaystyle\int_0^{\rho_i} \frac{c}{\rho} d\rho
\end{array}\right.
&
\Rightarrow
\left\{\begin{array}{ll}
s(P_{ext},\rho_1) = s(P_i,\rho_i)
& \Rightarrow \rho_b=\rho_1\\
\\
u_1 = u_i - \displaystyle\int_{\rho_i}^{\rho_1} \frac{c}{\rho} d\rho
& \Rightarrow u_b = u_1
\end{array}\right.
\end{array}
\end{equation}

\item Si $P_{ext} > P_i$,
on a un 1-choc.

On écrit les relations de saut à travers le 1-choc~:
\begin{equation}
\begin{array}{lll}
\left\{\begin{array}{l}
\rho_1 \rho_i (u_1 - u_i)^2
= (P_{ext} - P_i)(\rho_1 - \rho_i)\\
\\
2\rho_1 \rho_i (\varepsilon(P_{ext},\rho_1) - \varepsilon(P_i,\rho_i))
= (P_{ext} + P_i)(\rho_1 - \rho_i)
\end{array}\right.
&
\Rightarrow
\left\{\begin{array}{l}
\rho_b=\rho_1\\
\\
u_b = u_1
\end{array}\right.
\end{array}
\end{equation}

\bigskip
Pour les gaz parfaits, on a~:
\begin{itemize}

\item Cas détente ($P_{ext} \leqslant P_i$)~:\\
$$
\begin{array}{l}
P_b=P_{ext}\\
\\
\rho_b=\rho_i \left(\displaystyle\frac{P_{ext}}{P_i}\right)^{\frac{1}{\gamma}}
\end{array}
$$

\item Cas choc ($P_{ext} > P_i$)~:\\
$$
\begin{array}{l}
P_b=P_{ext}\\
\\
\rho_b=\rho_i \left(\displaystyle\frac{P_{ext}-P_i}{P_{ext}-P_i-\rho_i
(\vect{u}_i \cdot \vect{n} - \vect{u}_b \cdot \vect{n})^2}\right)
= \rho_i \left(\displaystyle\frac{(\gamma+1)P_{ext}+(\gamma-1)P_i}
{(\gamma-1)P_{ext}+(\gamma+1)P_i}\right)\\
\end{array}
$$

\end{itemize}

\end{enumerate}

La valeur de la masse volumique au bord intervient en particulier
dans le flux de masse.


%---------------------------------
\subsubsection*{En entrée}
%---------------------------------

L'utilisateur impose les valeurs qu'il souhaite pour les variables
en entrée~:
$$
\begin{array}{lllll}
\vect{W}_{\,ext} &=&
\left(\begin{array}{l}
\rho_{ext}\\ {\vect{u}_\tau}_{ext}\\ u_{ext}\\ P_{ext}
\end{array}\right)
\end{array}
$$

De même que précédemment, il existe deux cas de traitement
des conditions en entrée,
pilotés par le nombre de Mach entrant, normalement à la face de bord
(avec $c_{ext}$ la vitesse du son en entrée)~:
$$M_{ext} = \displaystyle\frac{u_{ext}}{c_{ext}}
= \displaystyle\frac{\vect{u}_{ext} \cdot \vect{n}}{c_{ext}}$$

\paragraph{Entrée supersonique (condition IESICF de
\fort{uscfcl})~:}
$M_{ext} \leqslant -1 \Rightarrow u_{ext} + c_{ext} \leqslant 0$

\unitlength=1cm
\begin{picture}(20,4.5)
\put(0,0.5){\vector(1,0){8}}
\put(4,0.5){\vector(0,1){3}}
\put(8,0.6){$x$}
\put(3.8,3.4){$t$}
\put(4,3.7){bord}
\put(4,3.2){\vector(1,0){0.5}}
\put(4.55,3.15){$\vect{n}$}
\multiput(4,0.5)(0,0.5){5}{\line(2,3){.4}}
\put(4,0.5){\line(-2,1){3}}
\put(0,2.1){$\lambda_1=u-c$}
\put(4,0.5){\qbezier[20](0,0)(-1.2,1.2)(-2.4,2.4)}
\put(0,2.9){$\lambda_{2,3,4}=u$}
\put(4,0.5){\line(-1,2){1.4}}
\put(1,3.4){$\lambda_5=u+c\ (\leqslant 0)$}
\put(1.5,1){$\vect{W}_{\,i}$}
\put(2,1.7){$\vect{W}_{\,1}$}
\put(2.2,2.5){$\vect{W}_{\,2}$}
\put(6,2){$\vect{W}_{\,\infty}$}
\put(10,2){$\vect{W}_{\,bord} = \vect{W}_{\,\infty} = \vect{W}_{\,ext}$}
\end{picture}

Toutes les caractéristiques sont entrantes,
toutes les conditions au bord sont donc imposées par l'utilisateur.

\begin{equation}
\left\{\begin{array}{l}
\rho_b = \rho_{ext}\\
{\vect{u}_\tau}_b = {\vect{u}_\tau}_{ext}\\
u_b = u_{ext}\\
P_b = P_{ext}
\end{array}\right.
\end{equation}


\paragraph{Entrée subsonique (condition IERUCF de
\fort{uscfcl})~: }
$$-1 < M_{ext} \leqslant 0
\Rightarrow (u_{ext} \leqslant 0 \text{ et } u_{ext} + c_{ext} > 0)$$


\unitlength=1cm
\begin{picture}(20,4.5)
\put(0,0.5){\vector(1,0){8}}
\put(4,0.5){\vector(0,1){3}}
\put(8,0.6){$x$}
\put(3.8,3.4){$t$}
\put(4,3.7){bord}
\put(4,3.2){\vector(1,0){0.5}}
\put(4.55,3.15){$\vect{n}$}
\multiput(4,0.5)(0,0.5){5}{\line(2,3){.4}}
\put(4,0.5){\line(-2,1){3}}
\put(0,2.1){$\lambda_1=u-c$}
\put(4,0.5){\qbezier[15](0,0)(-0.7,1.4)(-1.4,2.8)}
\put(1.1,3.4){$\lambda_{2,3,4}=u\ (\leqslant 0)$}
\put(4,0.5){\line(1,2){1.4}}
\put(5.3,3.4){$\lambda_5=u+c\ (>0)$}
\put(1.5,1){$\vect{W}_{\,i}$}
\put(2,2.1){$\vect{W}_{\,1}$}
\put(3.3,2.5){$\vect{W}_{\,2}$}
\put(6,2){$\vect{W}_{\,\infty}$}
\put(10,2){$\vect{W}_{\,bord} = \vect{W}_{\,2}$}
\put(12.5,2)
{$\left\{\begin{array}{l}
\rho_b = \rho_2\\
u_b = u_2\\
P_b = P_2
\end{array}\right.$}
\end{picture}


On a une caractéristique sortante.
L'utilisateur doit donc laisser un degré de liberté.

En général, on impose le flux de masse en entrée, donc $\rho_{ext}$
et $u_{ext}$, et l'on calcule la pression au bord en résolvant
le passage des 1$\sim$4-ondes.
On connaît aussi ${\vect{u}_\tau}_b = {\vect{u}_\tau}_{ext}$,
par conservation de la vitesse tangentielle à travers la 5-onde.

\begin{enumerate}

\item Si $u_{ext} \geqslant u_i$,
on a une 1-détente.

On écrit la conservation
des invariants de Riemann à travers la 1-détente
et la conservation de la vitesse et de la pression à travers le contact~:
\begin{equation}
\begin{array}{lll}
\begin{array}{l}
\left\{\begin{array}{l}
u_1 + \displaystyle\int_0^{\rho_1} \frac{c}{\rho} d\rho
= u_i + \displaystyle\int_0^{\rho_i} \frac{c}{\rho} d\rho\\
\\
s_1 = s_i
\end{array}\right.\\
\\
\left\{\begin{array}{l}
u_1 = u_2 = u_{ext}\\
\\
P_1 = P_2
\end{array}\right.
\end{array}
&
\Rightarrow
\left\{\begin{array}{ll}
\displaystyle\int_{\rho_i}^{\rho_1} \frac{c}{\rho} d\rho
= u_i - u_{ext}
& \Rightarrow \rho_1\\
\\
s(P_2,\rho_1) = s(P_i,\rho_i)
& \Rightarrow P_b = P_2
\end{array}\right.
\end{array}
\end{equation}

\item Si $u_{ext} < u_i$,
on a un 1-choc.

On écrit les relations de saut à travers le 1-choc
et la conservation de la vitesse et de la pression à travers le contact~:
\begin{equation}
\begin{array}{lll}
\begin{array}{l}
\left\{\begin{array}{l}
\rho_1 \rho_i (u_1 - u_i)^2
= (P_1 - P_i)(\rho_1 - \rho_i)\\
\\
2\rho_1 \rho_i (\varepsilon_1 - \varepsilon_i)
= (P_1 + P_i)(\rho_1 - \rho_i)\\
\\
\varepsilon = \varepsilon(P,\rho)
\end{array}\right.\\
\\
\left\{\begin{array}{l}
u_1 = u_2 = u_{ext}\\
\\
P_1 = P_2
\end{array}\right.
\end{array}
&
\Rightarrow
\left\{\begin{array}{l}
\rho_1\\
\\
P_b = P_2
\end{array}\right.
\end{array}
\end{equation}

\bigskip
Pour les gaz parfaits, on a~:

\begin{itemize}

\item Cas détente ($\delta M \leqslant 0$)~:\\
$$
\begin{array}{l}
\left\{\begin{array}{lll}
P_b=0 & \text{si} & 1 + \displaystyle\frac{\gamma-1}{2}\delta M<0\\
P_b = P_i \left(1 + \displaystyle\frac{\gamma-1}{2}\delta M\right)
^{\frac{2\gamma}{\gamma-1}} & \text{sinon}\\
\end{array}\right.\\
\\
\rho_b= \rho_{ext}\\
\end{array}
$$

\item Cas choc ($\delta M > 0$)~:\\
$$
\begin{array}{l}
P_b = P_i \left(1 + \displaystyle\frac{\gamma(\gamma+1)}{4}\delta M^2
+\gamma \delta M \displaystyle\sqrt{1+\displaystyle\frac{(\gamma+1)^2}{16}\delta M^2}\right)
\\
\\
\rho_b=\rho_{ext}\\
\end{array}
$$

\end{itemize}

\end{enumerate}


%=================================
\subsection*{Condition de pression en paroi avec effets de gravité}
%=================================

Le problème de Riemann considéré précédemment
ne prend pas en compte les effets de la gravité.
Or, dans certains cas, si l'on ne prend pas en compte le gradient de
pression ``hydrostatique'',  on peut obtenir une solution erronée
(en particulier, par exemple,
on peut créer une source de quantité de mouvement
non physique dans un milieu initialement au repos).

Écrivons l'équilibre local dans la maille de bord~:
\begin{equation}\label{Cfbl_Cfxtcl_eq_equilibre_local_cfxtcl}
\gradv{P} = \rho \vect{g}
\end{equation}

Pour simplifier la résolution, on peut utiliser la formulation
de (\ref{Cfbl_Cfxtcl_eq_equilibre_local_cfxtcl}) en incompressible
(c'est cette approche qui a été adoptée dans \CS)~:
\begin{equation}\label{Cfbl_Cfxtcl_eq_equilibre_incompressible_cfxtcl}
\begin{array}{lll}
\left(\gradv{P}\right)_i = \rho_i \vect{g}
& \text{ce qui donne}
& P_{paroi} = P_i + \rho_i \vect{g} \cdot (\vect{x}_{paroi} - \vect{x}_i)
\end{array}
\end{equation}


Une autre approche (dépendante de l'équation d'état)
consiste à résoudre l'équilibre local avec la formulation
compressible (\ref{Cfbl_Cfxtcl_eq_equilibre_local_cfxtcl}), en supposant de plus que
la maille est isentropique~:
\begin{equation}
\left\{\begin{array}{lll}
\gradv{P} = \rho \vect{g}\\
\\
P = P(\rho,s_i)
\end{array}\right.
\end{equation}
Ce qui donne, pour un gaz parfait~:
\begin{equation}
\label{Cfbl_Cfxtcl_eq_equilibre_compressible_cfxtcl}
 P_{paroi} = P_i \left( 1+ \displaystyle\frac{\gamma -1}{\gamma}
\displaystyle\frac{\rho_i}{P_i} \vect{g} \cdot (\vect{x}_{paroi} - \vect{x}_i)
\right)^{\frac{\gamma}{\gamma -1}}
\end{equation}

\paragraph{Remarque~:}
la formule issue de l'incompressible (\ref{Cfbl_Cfxtcl_eq_equilibre_incompressible_cfxtcl})
est une linéarisation de la formule (\ref{Cfbl_Cfxtcl_eq_equilibre_compressible_cfxtcl}).
Dans les cas courants elle s'éloigne très peu de la formule exacte.
Dans des conditions extrêmes,
si l'on considère par exemple
de l'air à $1000K$ et $10bar$, avec une accélération
de la pesanteur $g=1000m/s^2$ et une différence de hauteur entre
le centre de la cellule et le centre de la face de bord de $10m$,
l'expression (\ref{Cfbl_Cfxtcl_eq_equilibre_compressible_cfxtcl}) donne $P_{paroi} = 1034640,4Pa$
et l'expression (\ref{Cfbl_Cfxtcl_eq_equilibre_incompressible_cfxtcl}) donne $P_{paroi} = 1034644,7Pa$,
soit une différence relative de moins de $0,001\%$.
On voit aussi que la différence entre la pression calculée au centre
de la cellule et celle calculée au bord est de l'ordre de~$3\%$.

%=================================
\subsection*{Schéma de Rusanov pour le calcul de flux convectifs au bord}
%=================================


%---------------------------------
\subsubsection*{Introduction}
%---------------------------------

Le schéma de Rusanov est utilisé pour certains types de conditions aux
limites afin de passer du vecteur d'état calculé au bord comme indiqué
précédemment (solution du problème de Riemann) à un flux convectif de
bord (pour la masse, la quantité de
mouvement et l'énergie). L'utilisation de ce schéma (décentré amont)
permet de gagner en  stabilité.

Le schéma de Rusanov est appliqué aux frontières auxquelles on considère
qu'il est le plus probable de rencontrer des conditions en accord imparfait
avec l'état régnant dans le domaine, conditions qui sont donc susceptibles de
déstabiliser le calcul~: il s'agit des entrées et des sorties (frontières
de type IESICF, ISOPCF, IERUCF, IEQHCF). En sortie
supersonique (ISSPCF) cependant, le schéma de Rusanov est inutile et
n'est donc pas appliqué~:
en effet, pour ce type de frontière, l'état imposé au bord est exactement
l'état amont et le décentrement du schéma de Rusanov n'apporterait donc
rien.

%---------------------------------
\subsubsection*{Principe}
%---------------------------------

Pour le calcul du flux décentré de Rusanov, on considère
le système hyperbolique
constitué des seuls termes convectifs issus
des équations de masse, quantité de mouvement et énergie. Ce
système est écrit, par changement de variable, en non conservatif
(on utilise la relation
$\displaystyle P=\frac{\rho\varepsilon}{\gamma-1}$ et
on note $u_\xi$ les composantes de $\vect{u}$)~:

\begin{equation}
\left\{\begin{array}{lllll}
\displaystyle\frac{\partial\rho}{\partial t}
&+&\rho\divv{\,\vect{u}} + \vect{u}\,\grad{\,\rho}&=& 0 \\
\displaystyle\frac{\partial u_\xi}{\partial t}
&+& \vect{u}\,\grad{u_\xi}+\displaystyle\frac{1}{\rho}\,\frac{\partial
P}{\partial \xi} &=& 0 \\
\displaystyle\frac{\partial P}{\partial t}
&+&\gamma\,P\,\dive{\vect{u}}+\vect{u}\,\grad{P}&=& 0
\end{array}\right.
\end{equation}

En notant le vecteur d'état $\vect{W}= (\rho,\vect{u},P)^t$,
ce système est noté~:
\begin{equation}
\displaystyle\frac{\partial \vect{W}}{\partial t} +\dive{\,\vect{F}(\vect{W})} = 0
\end{equation}

Avec $\delta\,\vect{W}$ l'incrément temporel du vecteur d'état, $\vect{n}$ la
normale à une face, $ij$ la face interne partagée par les cellules $i$ et
$j$ et $ik$ la face de bord $k$ associée à la cellule $i$,
la discrétisation spatiale conduit à~:
\begin{equation}
\displaystyle\frac{|\Omega_i|}{\Delta t}\delta\,\vect{W}_i
+\sum\limits_{j\in\,Vois(i)}\int_{S_{ij}} \vect{F}(\vect{W})\,\vect{n}\,dS
+\sum\limits_{k\in {\gamma_b(i)}}\int_{S_{\,{b}_{ik}}} \vect{F}(\vect{W})\,\vect{n}\,dS
=0
\end{equation}

Sur une face de bord donnée,
on applique le schéma de Rusanov pour calculer le flux
comme suit~:
\begin{equation}
\frac{1}{|S_{\,{b}_{ik}}|}\int_{S_{\,{b}_{ik}}} \vect{F}(\vect{W})\,\vect{n}\,dS
=\frac{1}{2}\left(\vect{F}(\vect{W}_i)+\vect{F}(\vect{W}_{\,{b}_{ik}})\right)\cdot\vect{n}_{\,{b}_{ik}}
-\frac{1}{2}\rho_{rus\,{b}_{ik}}\left(\vect{W}_{\,{b}_{ik}}-\vect{W}_i\right)=\vect{F}_{rus\,{b}_{ik}}(\vect{W})
\end{equation}

Dans cette relation, $\vect{W}_{\,{b}_{ik}}$ est  le vecteur d'état
$\vect{W}_{\infty}$, connu au bord (tel
qu'il résulte de la résolution du problème de Riemann au bord
présentée plus haut pour chaque type de frontière considéré).

%---------------------------------
\subsubsection*{Paramètre de décentrement $\rho_{rus\,{b}_{ik}}$}
%---------------------------------

Pour chaque face de bord, le scalaire $\rho_{rus\,{b}_{ik}}$ est la
plus grande valeur du rayon spectral de la matrice jacobienne
$\displaystyle\frac{\partial\,\vect{F}_n(\vect{W})}{\partial \vect{W}}$
obtenu pour les vecteurs d'état $\vect{W}_i$ et $\vect{W}_{\,{b}_{ik}}$.

$\vect{F}_n$ est la composante du
flux $\vect{F}$ dans la direction de la normale à la face de bord,
$\vect{n}_{\,{b}_{ik}}$. Utiliser $\vect{F}_n$
pour la détermination du
paramètre de décentrement $\rho_{rus\,{b}_{ik}}$
relève d'une approche classique qui consiste
à remplacer le système tridimensionnel
initial par le système unidimensionnel projeté dans la direction
normale à la face, en négligeant les variations du vecteur d'état
$\vect{W}$ dans la direction tangeante à la face~:
\begin{equation}
\displaystyle\frac{\partial \vect{W}}{\partial t} +\frac{\partial\,\vect{F}_n(\vect{W})}{\partial
\vect{W}}\,\frac{\partial \vect{W}}{\partial n} = 0
\end{equation}

De manière plus explicite, si l'on se place dans un repère de calcul ayant
$\vect{n}_{\,{b}_{ik}}$ comme  vecteur de base, et si l'on note $u$ la
composante de vitesse associée, le système est le suivant (les équations
portant sur les composantes transverses de la vitesse sont découplées,
associées à la valeur propre $u$, comme le serait un scalaire simplement
convecté et ne sont pas écrites ci-après)~:
\begin{equation}
\left\{\begin{array}{lllll}
\displaystyle\frac{\partial\rho}{\partial t}
&+&\displaystyle\rho\frac{\partial\,u}{\partial\,n} + u\,\frac{\partial\,\rho}{\partial\,n}&=& 0 \\
\displaystyle\frac{\partial u}{\partial t}
&+&\displaystyle u\,\frac{\partial\,u}{\partial\,n}+\frac{1}{\rho}\,\frac{\partial
P}{\partial n} &=& 0 \\
\displaystyle\frac{\partial P}{\partial t}
&+&\displaystyle\gamma\,P\,\frac{\partial\,u}{\partial\,n}+u\,\frac{\partial\,P}{\partial\,n}&=& 0
\end{array}\right.
\end{equation}

La matrice jacobienne associée est donc~:
\begin{equation}
\left(\begin{array}{lll}
\displaystyle u & \rho                 & 0                                \\
\displaystyle 0 & u                    & \displaystyle\frac{1}{\rho}        \\
\displaystyle 0 & \gamma\, P        & 0                                 \\
\end{array}\right)
\end{equation}

Les valeurs propres sont $u$ et $\displaystyle\,u\pm c$ (avec
$c=\sqrt\frac{\gamma\,P}{\rho}$). Le rayon spectral est donc
$|u|+c$ et le paramètre de décentrement s'en déduit~:
\begin{equation}
\rho_{rus\,{b}_{ik}} = max\left(|u_i|+c_i,|u_{{b}_{ik}}|+c_{{b}_{ik}}\right)
\end{equation}


%---------------------------------
\subsubsection*{Expression des flux convectifs}
%---------------------------------

Les flux convectifs calculés par le schéma de Rusanov
pour les variables masse, quantité de mouvement
et énergie représentent donc la discrétisation des termes suivants~:
\begin{equation}
\left\{\begin{array}{l}
\displaystyle\dive(\vect{Q})\\
\displaystyle\divv(\vect{u}\otimes\vect{Q})+\grad\,P\\
\displaystyle\dive\left(\vect{Q}\,(e+\frac{P}{\rho})\right)
\end{array}\right.
\end{equation}

Pour une face de bord $ik$ adjacente à la cellule $i$ et
avec la valeur précédente de $\rho_{rus\,{b}_{ik}}$, on a~:
\begin{equation}
\left\{\begin{array}{lll}
\displaystyle\int_{S_{\,{b}_{ik}}}\vect{Q}\cdot\vect{n}\,dS
&=&
\displaystyle\frac{1}{2}\left(
(\vect{Q}_i+\vect{Q}_{\,{b}_{ik}})\cdot\vect{n}_{\,{b}_{ik}}\right)\,S_{\,{b}_{ik}}\\
&&\displaystyle \qquad\qquad\qquad\qquad\qquad\qquad
-\frac{1}{2}\,\rho_{rus\,{b}_{ik}}
\left(\rho_{\,{b}_{ik}}-\rho_i \right)\,S_{\,{b}_{ik}}\\
%
\displaystyle\int_{S_{\,{b}_{ik}}}(\vect{u}\otimes\vect{Q}+\grad\,P)\cdot\vect{n}\,dS
&=&
\displaystyle\frac{1}{2}\left(
 \vect{u}_i(\vect{Q}_i\cdot\vect{n}_{\,{b}_{ik}})
+P_i\,\vect{n}_{\,{b}_{ik}}
+\vect{u}_{\,{b}_{ik}}(\vect{Q}_{\,{b}_{ik}}\cdot\vect{n}_{\,{b}_{ik}})
+P_{\,{b}_{ik}}\,\vect{n}_{\,{b}_{ik}}\right)\,S_{\,{b}_{ik}}\\
&&\displaystyle \qquad\qquad\qquad\qquad\qquad\qquad
-\frac{1}{2}\,\rho_{rus\,{b}_{ik}}
\left(\vect{Q}_{\,{b}_{ik}}-\vect{Q}_i \right)\,S_{\,{b}_{ik}}\\
%
\displaystyle\int_{S_{\,{b}_{ik}}}(e+\frac{P}{\rho})\,\vect{Q}\cdot\vect{n}\,dS
&=&
\displaystyle\frac{1}{2}\left(
(e_i+\frac{P_i}{\rho_i})\,(\vect{Q}_i\cdot\vect{n}_{\,{b}_{ik}})
+(e_{\,{b}_{ik}}+\frac{P_{\,{b}_{ik}}}{\rho_{\,{b}_{ik}}})(\vect{Q}_{\,{b}_{ik}}\cdot\vect{n}_{\,{b}_{ik}})
\right)\,S_{\,{b}_{ik}}\\
&&\displaystyle \qquad\qquad\qquad\qquad\qquad\qquad
-\frac{1}{2}\,\rho_{rus\,{b}_{ik}}
\left(\rho_{\,{b}_{ik}}\,e_{\,{b}_{ik}}-\rho_i\,e_i \right)\,S_{\,{b}_{ik}}\\
\end{array}\right.
\end{equation}



%=================================
\subsection*{Conditions aux limites pour le flux diffusif d'énergie}
\label{Cfbl_Cfxtcl_section_cl_flux_diffusif_energie_cfener}
%=================================

%---------------------------------
\subsubsection*{Rappel}
%---------------------------------

Pour le flux de diffusion d'énergie, les conditions aux limites sont
imposées de manière similaire à ce qui est décrit dans
la documentation de \fort{cs\_boundary\_conditions\_set\_coeffs\_turb} et de
\fort{cs\_boundary\_conditions}. La figure~(\ref{Cfbl_Cfxtcl_fig_flux_cfxtcl}) rappelle quelques notations
usuelles et l'équation~(\ref{Cfbl_Cfxtcl_eq_flux_cfxtcl}) traduit la conservation du flux
normal au bord pour la variable $f$.

\begin{figure}[htp]
\centerline{\includegraphics[height=7cm]{fluxbord}}
\caption{\label{Cfbl_Cfxtcl_fig_flux_cfxtcl}Cellule de bord.}
\end{figure}

\begin{equation}\label{Cfbl_Cfxtcl_eq_flux_cfxtcl}
\begin{array}{l}
    \underbrace{h_{int}(f_{b,int}-f_{I'})}_{\phi_{int}}
  = \underbrace{h_{b}(f_{b,ext}-f_{I'})}_{\phi_{b}}
  = \left\{\begin{array}{ll}
    \underbrace{h_{imp,ext}(f_{imp,ext}-f_{b,ext})}_{\phi_{\text{\it réel
imposé}}} &\text{(condition de Dirichlet)}\\
    \underbrace{\phi_{\text{\it imp,ext}}}_{\phi_{\text{\it réel imposé}}}
            &\text{(condition de Neumann)}
           \end{array}\right.
\end{array}
\end{equation}


L'équation~(\ref{Cfbl_Cfxtcl_eq_fbint_cfxtcl}) rappelle la formulation des
conditions aux limites pour une variable $f$.
\begin{equation}\label{Cfbl_Cfxtcl_eq_fbint_cfxtcl}
f_{b,int}
  = \left\{\begin{array}{cccccl}
    \displaystyle\frac{h_{imp,ext}}{h_{int}+h_r h_{imp,ext} }&f_{imp,ext}&+&
    \displaystyle\frac{h_{int}+h_{imp,ext}(h_r-1)}{h_{int}+h_r h_{imp,ext} }&f_{I'}
                         &\text{(condition de Dirichlet)}\\
    \displaystyle\frac{1}{h_{int}}&\phi_{\text{\it imp,ext}}&+&
    \ &f_{I'}
            &\text{(condition de Neumann)}
           \end{array}\right.
\end{equation}

Les coefficients d'échange sont définis comme suit\footnote{On rappelle que, comme
dans \fort{cs\_boundary\_conditions}, $\alpha$ désigne $\lambda+C_p\,\frac{\mu_t}{\sigma_t}$
si $f$ est la température,
$\frac{\lambda}{C_p}+\frac{\mu_t}{\sigma_t}$ si $f$ représente l'enthalpie.
Le coefficient $C$ représente $C_p$ pour la température et vaut
$1$ pour l'enthalpie. La grandeur adimensionnelle $f^+$ est obtenue par
application d'un principe de similitude en paroi~: pour la température,
elle dépend du nombre de
Prandlt moléculaire, du nombre de Prandtl turbulent et de la distance adimensionnelle à la paroi $y^+$ dans la cellule de bord.}~:
\begin{equation}
\left\{\begin{array}{lll}
h_{int}&=&\displaystyle\frac{\alpha}{\overline{I'F}}\\
h_r&=&\displaystyle\frac{h_{int}}{h_{b}} \\
h_b&=&\displaystyle\frac{\phi_b}{f_{b,ext}-f_{I'}}=\frac{\rho\,C\,u_k}{f^+_{I'}}
\end{array}\right.
\end{equation}

Dans \CS, on note les conditions aux limites de manière générale sous
la forme suivante~:
\begin{equation}
f_{b,int}=A_b + B_b\,f_{I'}
\end{equation}
avec $A_b$ et $B_b$ définis selon le type des conditions~:
\begin{equation}
\begin{array}{c}
\text{Dirichlet}\left\{\begin{array}{ll}
    A_b = &\displaystyle\frac{h_{imp,ext}}{h_{int}+h_r h_{imp,ext} } f_{imp,ext}\\
    B_b = &\displaystyle\frac{h_{int}+h_{imp,ext}(h_r-1)}{h_{int}+h_r h_{imp,ext} }
                  \end{array}\right.
\text{\ \  Neumann}\left\{\begin{array}{ll}
    A_b = &\displaystyle\frac{1}{h_{int}}\phi_{\text{\it imp,ext}}\\
    B_b = &1
                  \end{array}\right.
\end{array}
\end{equation}

%---------------------------------
\subsubsection*{Flux diffusif d'énergie}
%---------------------------------

Dans le module compressible, on résout une équation sur l'énergie, qui s'écrit, si
l'on excepte tous les termes hormis le flux de diffusion et le terme
instationnaire, pour faciliter la présentation~:

\begin{equation}
\begin{array}{lll}
\displaystyle\frac{\partial \rho e}{\partial t} &=& - \dive{\,\vect{\Phi}_s}\\
&=& \displaystyle\dive{(K\,\grad{T})} \text{\ \ avec \ \ } K=\lambda+C_p\,\frac{\mu_t}{\sigma_t}\\
&=& \displaystyle\dive{\left(K\,\grad{\frac{e-\frac{1}{2}\,u^2-\varepsilon_{sup}}{C_v}}\right)} \\
&=& \displaystyle\dive{\left(\frac{K}{C_v}\,\grad{(e-\frac{1}{2}\,u^2-\varepsilon_{sup})}\right)} \text{\ \
si \ \ } C_v \text{\ est constant}\\
&=& \displaystyle\dive{\left(\frac{K}{C_v}\,\grad\,e\right)}
-\dive{\left(\frac{K}{C_v}\,\grad{(\frac{1}{2}\,u^2+\varepsilon_{sup})}\right)} \\

\end{array}
\end{equation}

La décomposition en $e$ et $\frac{1}{2}\,u^2+\varepsilon_{sup}$ est purement
mathématique (elle résulte du fait que l'on résout en énergie alors que
le flux thermique s'exprime en fonction de la température). Aussi,  pour imposer un
flux de bord ou une température de bord (ce qui revient au même puisque l'on
impose toujours finalement la conservation du flux normal), on {\it choisit}
de reporter la totalité de la condition à la limite sur le terme
$\displaystyle\frac{K}{C_v}\,\grad\,e$
et donc d'annuler le flux associé au terme
$\displaystyle\frac{K}{C_v}\,\grad{(\frac{1}{2}\,u^2+\varepsilon_{sup})}$
(en pratique, pour l'annuler, on se contente de ne pas l'ajouter
au second membre de l'équation). Conformément à l'approche retenue dans \CS et
rappelée précédemment, on déterminera donc une valeur de bord {\it
fictive} de l'énergie qui permette de reconstruire le flux diffusif total
attendu à partir
de la discrétisation du seul terme $\displaystyle\frac{K}{C_v}\,\grad\,e$.

Remarque : dans la version 1.2.0,
on utilise $\displaystyle
\frac{K}{C_v}=\left(\frac{\lambda}{C_v}+\frac{\mu_t}{\sigma_t}\right)$, à
partir de 1.2.1, on utilise la valeur  $\displaystyle
\frac{K}{C_v}=\left(\frac{\lambda}{C_v}+\frac{C_p}{C_v}\frac{\mu_t}{\sigma_t}\right)$.
On notera que le nombre de Prandtl turbulent $\sigma_t$ est associé à la variable
résolue et peut être fixé par l'utilisateur.


%---------------------------------
\subsubsection*{Condition de Neumann}
%---------------------------------

La conservation du flux s'écrit~:

\begin{equation}
    \underbrace{h_{int}(e_{b,int}-e_{I'})}_{\phi_{int}}
    =\underbrace{\phi_{\text{\it imp,ext}}}_{\phi_{\text{\it réel imposé}}}
\end{equation}

On a donc dans ce cas~:
\begin{equation}
\left\{\begin{array}{lll}
  A_b &= &\displaystyle\frac{1}{h_{int}}\phi_{\text{\it imp,ext}}\\
  B_b &= &1
\end{array}\right.
\end{equation}


%---------------------------------
\subsubsection*{Condition de Dirichlet}
%---------------------------------

On suppose que la condition de Dirichlet porte sur la température $T_{b,ext}$.


La conservation du flux s'écrit~:
\begin{equation}\label{Cfbl_Cfxtcl_eq_conservation_flux_cfxtcl}
    \underbrace{h_{int}(e_{b,int}-e_{I'})}_{\phi_{int}\text{\ (forme numérique
du flux)}}
  = \underbrace{h_{b}(T_{b,ext}-T_{I'})}_{\phi_{b}\text{ qui intègre l'effet
de couche limite}}
  =
    \underbrace{h'_{imp,ext}(T_{imp,ext}-T_{b,ext})}_{\phi_{\text{\it réel
imposé}}}
\end{equation}

Avec pour les coefficients d'échange~:
\begin{equation}
\left\{\begin{array}{lll}
h_{int}&=&\displaystyle\frac{K}{C_v\,\overline{I'F}}\\
h_b&=&\displaystyle\frac{\phi_b}{T_{b,ext}-T_{I'}}=\frac{\rho\,C_p\,u_k}{T^+_{I'}}
\end{array}\right.
\end{equation}

On tire $T_{b,ext}$
de la seconde partie de l'égalité~(\ref{Cfbl_Cfxtcl_eq_conservation_flux_cfxtcl})
traduisant la conservation du flux~:
\begin{equation}
\displaystyle T_{b,ext} = \frac{h'_{imp,ext}\,T_{imp,ext}+h_b\,T_{I'}}{h_b+h'_{imp,ext}}
\end{equation}

En utilisant cette valeur et la première partie de l'équation de conservation
du flux~(\ref{Cfbl_Cfxtcl_eq_conservation_flux_cfxtcl}), on obtient~:
\begin{equation}
e_{b,int} = \frac{h_b\,h'_{imp,ext}}{h_{int}\,(h_b+h'_{imp,ext})}\,(T_{imp,ext}-T_{I'})+e_{I'}
\end{equation}

On utilise alors
$\displaystyle T_{I'}=\frac{1}{C_v}\left(e_{I'}-\frac{1}{2}u^2_{i}-\varepsilon_{sup,i}\right)$ pour
écrire (sans reconstruction pour la vitesse et $\varepsilon_{sup}$)~:
\begin{equation}
\displaystyle e_{b,int} =
\frac{ -\frac{h_b\,h'_{imp,ext}}{C_v}+h_{int}\,(h_b+h'_{imp,ext}) }
     { h_{int}\,(h_b+h'_{imp,ext}) } \,e_{I'}
+\frac{h_b\,h'_{imp,ext}}{h_{int}\,(h_b+h'_{imp,ext})}\,
  \left(T_{imp,ext}+\frac{\frac{1}{2}u^2_{i}+\varepsilon_{sup,i}}{C_v}\right)
\end{equation}


Et on a donc, avec $\displaystyle h'_r=\frac{h_{int}}{\frac{h_b}{C_v}}$~:
\begin{equation}
\displaystyle e_{b,int} =
\underbrace{\frac{ h'_{imp,ext} }{ C_v\,h_{int}+h'_r\,h'_{imp,ext} }\,
  \left(C_v\,T_{imp,ext}+\frac{1}{2}u^2_{i}+\varepsilon_{sup,i}\right)}_{A_b}
+\underbrace{\frac{ C_v\,h_{int}+h'_{imp,ext}(h'_r-1) }{ C_v\,h_{int}+h'_r\,h'_{imp,ext} }}_{B_b}\,e_{I'}
\end{equation}

Avec ces notations, $h_b$ est le coefficient habituellement calculé pour la
température.

Le coefficient $h'_{imp,ext}$ est le coefficient d'échange externe qui est
imposé pour la température\footnote{Le coefficient $h'_{imp,ext}$
est utile pour les cas où l'on
souhaite relaxer la condition à la limite~:
pour la température, cela correspond à imposer une valeur sur la face
externe d'une paroi unidimensionnelle idéale, sans inertie,
caractérisée par un simple coefficient d'échange.}.
Pour obtenir l'équivalent dimensionnel de $h'_{imp,ext}$ pour l'énergie,
il faut diviser sa valeur par $C_v$ (ce qui ne fait pas de différence dans
la majorité des cas, car il est habituellement pris infini).

%%%%%%%%%%%%%%%%%%%%%%%%%%%%%%%%%%
%%%%%%%%%%%%%%%%%%%%%%%%%%%%%%%%%%
\section*{Mise en \oe uvre}
%%%%%%%%%%%%%%%%%%%%%%%%%%%%%%%%%%
%%%%%%%%%%%%%%%%%%%%%%%%%%%%%%%%%%

%=================================
\subsection*{Introduction}
%=================================

Les conditions aux limites sont imposées par une suite de sous-programmes,
dans la mesure où l'on a cherché à rester cohérent avec la structure
standard de \CS.

Dans \fort{ppprcl} (appelé par \fort{precli}), on initialise les tableaux
avant le calcul des conditions aux limites~:
\begin{itemize}
\item \var{IZFPPP} (numéro de zone, inutilisé, fixé à zéro),
\item \var{IA(IIFBRU)} (repérage des faces de bord pour
lesquelles on applique un schéma de Rusanov~: initialisé à zéro,
on imposera la valeur 1 dans \fort{cfrusb} pour les faces auxquelles on applique le schéma
de Rusanov)
\item \var{IA(IIFBET)} (repérage des faces de paroi à température ou
à flux thermique imposé~: initialisé à 0, on imposera la valeur 1
dans \fort{cfxtcl} lorsque la température ou le flux est imposé),
\item \var{RCODCL(*,*,1)} (initialisé à \var{-RINFIN} en prévision
du traitement des sorties réentrantes pour lesquelles l'utilisateur
aurait fourni une valeur à imposer en Dirichlet),
\item flux convectifs de bord pour la quantité de mouvement et l'énergie
(initialisés à zéro).
\end{itemize}


\bigskip
Les types de frontière (\var{ITYPFB}) et les valeurs nécessaires
(\var{ICODCL}, \var{RCODCL}) sont imposés par l'utilisateur dans \fort{uscfcl}.

On convertit ensuite ces données dans \fort{cs\_boundary\_conditions} pour qu'elles
soient directement utilisables lors du calcul des matrices et des seconds membres.

Pour cela, \fort{cfxtcl} permet de réaliser le calcul des valeurs de bord et,
pour certaines frontières, des flux convectifs. On fait appel,
en particulier,
à \fort{uscfth} (utilisation de la thermodynamique) et à \fort{cfrusb}
(flux convectifs par le schéma de Rusanov). Lors de ces calculs, on utilise
\var{COEFA} et \var{COEFB} comme tableaux de travail (transmission de valeurs
à \fort{uscfth} en particulier) afin de renseigner \var{ICODCL} et
\var{RCODCL}.
Après \fort{cfxtcl},
le sous-programme \fort{typecl} complète quelques valeurs par défaut
de \var{ICODCL} et de \var{RCODCL}, en particulier pour les scalaires passifs.

Après \fort{cfxtcl} et \fort{typecl}, les tableaux \var{ICODCL} et \var{RCODCL}
sont complets. Ils sont utilisés dans la suite de \fort{cs\_boundary\_conditions} et en particulier
dans \fort{cs\_boundary\_conditions\_set\_coeffs\_turb} pour construire les tableaux \var{COEFA} et \var{COEFB}
(pour l'énergie, on dispose de deux couples (\var{COEFA}, \var{COEFB}) afin de
traiter les parois).

On présente ci-après les points dont l'implantation diffère
de l'approche standard. Il s'agit de
l'utilisation d'un schéma de Rusanov pour le calcul des flux convectifs
en entrée et sortie (hormis sortie supersonique)
et du mode de calcul des flux diffusifs d'énergie en paroi.
On insiste en particulier sur l'impact des conditions aux limites
sur la construction des seconds membres de l'équation de la quantité
de mouvement et de l'équation de l'énergie (\fort{cfqdmv} et \fort{cfener}).

%=================================
\subsection*{Flux de Rusanov pour le calcul des flux convectifs en entrée et sortie}
%=================================

Le schéma de Rusanov est utilisé pour calculer des flux convectifs de bord
(masse, quantité de mouvement et énergie) aux entrées et des sorties
de type IESICF, ISOPCF, IERUCF, IEQHCF.

La gestion des conditions aux limites est différente de celle adoptée
classiquement dans \CS, bien que l'on se soit efforcé de s'y conformer le
mieux possible.

En volumes finis, il faut disposer de conditions aux
limites pour trois utilisations principales au moins~:
         \begin{itemize}
        \item imposer les flux de convection,
        \item imposer les flux de diffusion,
        \item calculer les gradients pour les reconstructions.
        \end{itemize}
Dans l'approche standard de \CS, les conditions aux limites sont définies par
variable et non pas par terme discret\footnote{Par exemple, pour un scalaire
convecté et diffusé, on définit une valeur de bord unique {\it pour le scalaire}
et non pas une valeur de bord pour le {\it flux convectif} et une valeur de bord
pour le {\it flux diffusif}.}. On dispose donc, {\it pour chaque variable},
d'une valeur de bord dont devront être déduits les flux de
convection, les flux de diffusion et les gradients\footnote{Néanmoins, pour
certaines variables comme la vitesse par exemple, \CS dispose de deux valeurs
de bord (et non pas d'une seule) afin de pouvoir imposer de manière
indépendante le gradient normal et le flux de diffusion.}.
Ici, avec l'utilisation d'un schéma de
Rusanov, dans lequel le flux convectif est traité dans son ensemble,
il est impératif
de disposer d'un moyen d'imposer directement sa valeur au bord\footnote{Il
serait possible de calculer une valeur de bord fictive des variables d'état qui
permette de retrouver le flux convectif calculé par le schéma de Rusanov,
mais cette valeur ne permettrait pas d'obtenir
un flux de diffusion et un gradient satisfaisants.}.

Le flux convectif calculé par le schéma de Rusanov
sera ajouté directement au second membre
des équations de masse, de quantité de mouvement et d'énergie. Comme ce
flux contient, outre la contribution des termes convectifs ``usuels''
($\dive(\vect{Q})$, $\dive(\vect{u}\otimes\vect{Q})$ et
$\dive(\vect{Q}\,e)$), celle des termes en $\grad\,P$ (quantité de
mouvement) et $\dive(\vect{Q}\,\frac{P}{\rho})$
(énergie), il faut veiller à ne pas
ajouter une seconde fois les termes de bord issus de  $\grad\,P$ et de
$\dive(\vect{Q}\,\frac{P}{\rho})$
au second membre des équations de quantité de
mouvement et d'énergie.


Pour la masse, le flux convectif calculé par le schéma de Rusanov
définit simplement le flux de masse au bord
(\var{PROPFB(IFAC,IPPROB(IFLUMA(ISCA(IENERG))))}).

Pour la quantité de mouvement, le flux convectif calculé par le schéma de
Rusanov  est stocké dans les tableaux
\var{PROPFB(IFAC,IPPROB(IFBRHU))}, \var{PROPFB(IFAC,IPPROB(IFBRHV))} et
\var{PROPFB(IFAC,IPPROB(IFBRHW))}. Il est ensuite ajouté au second membre de
l'équation directement dans \fort{cfqdmv} (boucle sur les faces de bord).
Comme ce flux contient la contribution du terme convectif usuel
$\divv(\vect{u}\otimes\vect{Q})$, il ne faut pas l'ajouter dans
le sous-programme \fort{cfbsc2}.
De plus, le flux convectif calculé par le schéma de Rusanov
contient la contribution du
gradient de pression. Or, le gradient de pression est calculé dans
\fort{cfqdmv} au moyen de \fort{grdcel} et ajouté au second membre
sous forme de contribution volumique (par cellule)~: il faut donc retirer
la contribution des faces de bord auxquelles est appliqué le schéma de
Rusanov, pour ne pas la compter deux fois (cette opération est réalisée
dans \fort{cfqdmv}).

Pour l'énergie, le flux convectif calculé par le schéma de
Rusanov est stocké dans le tableau
\var{PROPFB(IFAC,IPPROB(IFBENE))}. Pour les faces auxquelles n'est pas
appliqué le schéma de Rusanov, on ajoute la contribution
du terme de transport de pression $\dive(\vect{Q}\,\frac{P}{\rho})$
au second membre de l'équation dans \fort{cfener}
et on complète le second membre dans \fort{cfbsc2} avec la contribution du
terme convectif usuel $\dive(\vect{Q}\,e)$. Pour les faces auxquelles est
appliqué le schéma de Rusanov, on ajoute directement le flux de Rusanov au second
membre de l'équation dans \fort{cfener}, en lieu et place de la contribution
du terme de transport de pression et l'on prend garde de ne pas
comptabiliser une seconde fois le flux convectif usuel
$\divv(\vect{Q}\,e)$ dans le sous-programme \fort{cfbsc2}.

C'est l'indicateur \var{IA(IIFBRU)}
(renseigné dans \fort{cfrusb}) qui permet, dans \fort{cfbsc2},
\fort{cfqdmv} et \fort{cfener},
de repérer les faces de bord pour lesquelles on a calculé
un flux convectif avec le schéma de Rusanov.


%=================================
\subsection*{Flux diffusif d'énergie}
%=================================

%---------------------------------
\subsubsection*{Introduction}
%---------------------------------

Une condition doit être fournie sur toutes les frontières pour le calcul du
flux diffusif d'énergie.

Il n'y a pas lieu de
s'étendre particulièrement sur le traitement de certaines frontières.
Ainsi, aux entrées et sorties, on dispose
d'une valeur de bord (issue de la résolution du problème
de Riemann)
que l'on utilise dans la formule discrète classique donnant le
flux\footnote{Les valeurs de $u^2$ et de $\varepsilon_{sup}$ ne sont pas
reconstruites pour le calcul du gradient au bord dans
$\displaystyle\dive{\left(\frac{K}{C_v}\,\grad{(\frac{1}{2}\,u^2+\varepsilon_{sup})}\right)}$}.
La situation est simple aux symétries également, où un flux nul est imposé.

Par contre, en paroi, les conditions de température ou de flux thermique
imposé doivent être traitées avec plus d'attention, en particulier
lorsqu'une couche limite turbulente est présente.

%---------------------------------
\subsubsection*{Coexistence de deux conditions de bord}
%---------------------------------

Comme indiqué dans la partie "discrétisation",
les conditions de température ou de flux conductif
imposé en paroi se traduisent,
pour le flux d'énergie, au travers du terme
$\displaystyle\dive{\left(\frac{K}{C_v}\,\grad\,e\right)}$,
en imposant une condition de flux nul sur le terme
$\displaystyle-\dive{\left(\frac{K}{C_v}\,\grad{(\frac{1}{2}\,u^2+\varepsilon_{sup})}\right)}$.
Les faces IFAC
concernées sont repérées dans \fort{cfxtcl} par l'indicateur
\var{IA(IIFBET+IFAC-1) = 1} (qui vaut 0 sinon, initialisé
dans \fort{ppprcl}).

Sur ces faces,
on calcule une valeur de bord de l'énergie, qui, introduite dans la
formule générale de flux utilisée au bord dans \CS, permettra de retouver le
flux souhaité. La valeur de bord est une simple valeur numérique sans
signification physique et ne doit être utilisée que pour calculer le flux
diffusif.

En plus de cette valeur de bord destinée à retrouver le
flux diffusif, il est nécessaire de disposer
d'une seconde valeur de bord de l'énergie afin de pouvoir en calculer le
gradient.

Ainsi, comme pour la vitesse en $k-\varepsilon$, il est nécessaire de
disposer pour l'énergie de deux couples de coefficients
(\var{COEFA},\var{COEFB}), correspondant à deux valeurs de bord distinctes,
dont l'une est utilisée pour le calcul du flux diffusif spécifiquement.

%---------------------------------
\subsubsection*{Calcul des \var{COEFA} et \var{COEFB} pour les faces de paroi
à température imposée}
%---------------------------------

Les  faces de paroi  \var{IFAC} à température imposée sont identifées par
l'utilisateur dans \fort{uscfcl} au moyen de  l'indicateur
\var{ICODCL(IFAC,ISCA(ITEMPK))=5} (noter que
ce tableau est associé à la température).

Dans \fort{cfxtcl}, on impose alors \var{ICODCL(IFAC,ISCA(IENERG))=5} et
on calcule la quantité
$C_v\,T_{imp,ext}+\frac{1}{2}u^2_{I}+\varepsilon_{sup,I}$, que l'on
stocke dans \var{RCODCL(IFAC,ISCA(IENERG),1)} (on ne reconstruit pas les
valeurs de $u^2$ et $\varepsilon_{sup}$ au bord, cf. \S\ref{Cfbl_Cfxtcl_prg_a_traiter}).

à partir de ces valeurs de \var{ICODCL} et \var{RCODCL},
on renseigne ensuite dans \fort{cs\_boundary\_conditions\_set\_coeffs\_turb}
les tableaux de conditions aux limites  permettant le calcul du flux~:
\var{COEFA(*,ICLRTP(ISCA(IENERG),ICOEFF))} et
\var{COEFB(*,ICLRTP(ISCA(IENERG),ICOEFF))} (noter
l'indicateur \var{ICOEFF} qui renvoie aux coefficients dédiés au flux
diffusif).


%---------------------------------
\subsubsection*{Calcul des \var{COEFA} et \var{COEFB} pour les faces de paroi
à flux thermique imposé}
%---------------------------------

Les  faces de paroi  \var{IFAC} à flux thermique
imposé sont identifées par
l'utilisateur dans \fort{uscfcl} au moyen de  l'indicateur
\var{ICODCL(IFAC,ISCA(ITEMPK))=3} (noter que le tableau est
associé à la température).

Dans \fort{cfxtcl}, on impose alors \var{ICODCL(IFAC,ISCA(IENERG))=3} et
on transfère la valeur du flux de  \var{RCODCL(IFAC,ISCA(ITEMPK),3)}
à \var{RCODCL(IFAC,ISCA(IENERG),3)}.

à partir de ces valeurs de \var{ICODCL} et \var{RCODCL},
on renseigne ensuite dans \fort{cs\_boundary\_conditions} les tableaux de conditions aux limites
permettant le calcul du flux,
\var{COEFA(*,ICLRTP(ISCA(IENERG),ICOEFF))} et
\var{COEFB(*,ICLRTP(ISCA(IENERG),ICOEFF))} (noter
l'indicateur \var{ICOEFF} qui renvoie aux coefficients dédiés au flux
diffusif).

%---------------------------------
\subsubsection*{Gradient de l'énergie en paroi à température ou à flux thermique imposé}
%---------------------------------

Dans les deux cas (paroi à température ou à flux thermique imposé),
on utilise les tableaux
\var{COEFA(*,ICLRTP(ISCA(II),ICOEF))},
\var{COEFB(*,ICLRTP(ISCA(II),ICOEF))} (noter le \var{ICOEF}) pour disposer d'une
condition de flux nul pour l'énergie (avec \var{II=IENERG}) et
pour la température (avec \var{II=ITEMPK})
si un calcul de gradient est requis.

Un gradient est en particulier utile pour les reconstructions
de l'énergie sur maillage non orthogonal.
Pour la température, il s'agit d'une précaution, au cas
où l'utilisateur aurait besoin d'en calculer le gradient.

%---------------------------------
\subsubsection*{Autres frontières que les parois à température ou à flux thermique imposé}
%---------------------------------

Pour les frontières qui ne sont pas des parois à température ou
à flux thermique imposé, les conditions aux limites de l'énergie et
de la température sont complétées classiquement dans \fort{cs\_boundary\_conditions} selon
les choix faits dans \fort{cfxtcl} pour \var{ICODCL} et \var{RCODCL}.

En particulier,
dans le cas de conditions de Dirichlet sur l'énergie (entrées, sorties), les
deux jeux de conditions aux limites sont identiques (tableaux
\var{COEFA}, \var{COEFB} avec \var{ICOEFF} et \var{ICOEF}).

Si un flux est imposé pour l'énergie totale (condition assez rare,
l'utilisateur ne raisonnant pas,
d'ordinaire, en énergie totale), on le stocke au moyen de
\var{COEFA(*,ICLRTP(ISCA(IENERG),ICOEFF))} et
\var{COEFB(*,ICLRTP(ISCA(IENERG),ICOEFF))} (tableaux associés au flux
diffusif). Pour le gradient, une condition de flux nul est stockée
dans
\var{COEFA(*,ICLRTP(ISCA(IENERG),ICOEF))} et
\var{COEFB(*,ICLRTP(ISCA(IENERG),ICOEF))}. On peut remarquer que les deux
jeux de conditions aux limites sont identiques pour les faces de symétrie.

%---------------------------------
\subsubsection*{Impact dans \fort{cfener}}
%---------------------------------

Lors de la construction des seconds membres, dans \fort{cfener}, on utilise les
conditions aux limites stockées dans les tableaux associés au flux
diffusif
\var{COEFA(*,ICLRTP(ISCA(IENERG),ICOEFF))} et
\var{COEFB(*,ICLRTP(ISCA(IENERG),ICOEFF))} pour le terme de flux diffusif
$\displaystyle\dive{\left(\frac{K}{C_v}\,\grad\,e\right)}$
en prenant soin d'annuler la contribution de bord du terme
$\displaystyle-\dive{\left(\frac{K}{C_v}\,\grad{(\frac{1}{2}\,u^2+\varepsilon_{sup})}\right)}$
sur les faces pour lesquelles cette condition
prend les deux termes en compte, c'est-à-dire sur les faces pour lesquelles
\var{IA(IIFBET+IFAC-1) = 1}.

Pour tous les autres termes qui requièrent une valeur de bord, on utilise les
conditions aux limites que l'on a stockées au moyen des deux tableaux
\var{COEFA(*,ICLRTP(ISCA(IENERG),ICOEF))} et
\var{COEFB(*,ICLRTP(ISCA(IENERG),ICOEF))}. Ces conditions sont
donc en particulier utilisées pour le calcul du gradient de l'énergie,
lors des reconstructions sur maillage non orthogonal.


\newpage
%%%%%%%%%%%%%%%%%%%%%%%%%%%%%%%%%%
%%%%%%%%%%%%%%%%%%%%%%%%%%%%%%%%%%
\section*{Points à traiter}
%%%%%%%%%%%%%%%%%%%%%%%%%%%%%%%%%%
%%%%%%%%%%%%%%%%%%%%%%%%%%%%%%%%%%
\label{Cfbl_Cfxtcl_prg_a_traiter}%
% propose en patch 1.2.1
%Corriger \fort{ppprcl} pour que l'indicateur
%\var{IA(IIFBET+IFAC-1)} soit
%initialisé à 0, positionné à 1 aux faces de paroi à température
%ou flux thermique imposé. Dans \fort{cfener}, lorsque l'indicateur vaudra 1,
%on ne prendra pas en compte le flux correspondant à
%$\displaystyle-\dive{\left(\frac{K}{C_v}\,\grad{(\frac{1}{2}\,u^2+\varepsilon_{sup})}\right)}$.

% propose en patch 1.2.1
%Pour l'énergie, on utilise comme diffusivité turbulente la valeur
%$\displaystyle \frac{K}{C_v}=\frac{\lambda}{C_v}+\frac{\mu_t}{\sigma_t}$.
%Par cohérence avec une équation
%portant sur la température, il serait plus logique d'utiliser
%$\displaystyle \frac{K}{C_v}=\frac{\lambda}{C_v}+\frac{C_p}{C_v}\,\frac{\mu_t}{\sigma_t}$.
%On peut temporairement utiliser le nombre de Prandtl turbulent pour prendre en compte
%le rapport $\displaystyle\frac{C_p}{C_v}$, mais il serait
%souhaitable de corriger en ce sens le calcul de \var{W1} pour \fort{cs\_face\_viscosity} dans
%le sous-programme \fort{cfener} et le calcul similaire de \var{HINT} dans
%\fort{cs\_boundary\_conditions} et \fort{cs\_boundary\_conditions\_set\_coeffs\_turb} (RAS pour les conversions en couplage avec \syrthes).

Apporter un complément de test sur une cavité fermée
sans vitesse et sans gravité, avec flux de bord ou température de bord imposée.
Il semble que le transfert d'énergie {\it via} les termes de pression génère de
fortes vitesses non physiques dans la première maille de paroi et que la
conduction thermique ne parvienne pas à établir le profil de température
recherché. Il est également possible que la condition de bord sur la pression
génère une perturbation (une extrapolation pourrait se révéler
indispensable).

Il pourrait être utile de généraliser à l'incompressible l'approche
utilisée en compressible pour unifier simplement le traitement
des sorties de type 9 et 10.

Il pourrait être utile d'étudier plus en détail l'influence de la non
orthogonalité des mailles en sortie supersonique (pas de reconstruction,
ce qui n'est pas consistant pour les flux de diffusion).

De même, il serait utile d'étudier l'influence de l'absence de
reconstruction pour la vitesse et $\varepsilon_{sup}$ dans la relation
$\displaystyle T_{I'}=\frac{1}{C_v}\left(e_{I'}-\frac{1}{2}u^2_{i}-\varepsilon_{sup,i}\right)$
utilisée pour les parois à température imposée.

Apporter un complément de documentation pour le couplage avec \syrthes (conversion
énergie température). Ce n'est pas une priorité.

Pour les thermodynamiques à $\gamma$ variable, il sera nécessaire de
modifier non
seulement \fort{uscfth} mais également \fort{cfrusb} qui doit disposer de
$\gamma$ en argument.

Pour les thermodynamiques à $C_v$ variable, il sera nécessaire de
prendre en compte un terme en $\grad\,C_v$, issu des flux diffusifs,
au second membre de l'équation de
l'énergie (on pourra cependant remarquer qu'actuellement, en incompressible,
on néglige le terme en $\grad\,C_p$ dans l'équation de l'enthalpie).

\part{Module �lectrique}
%-------------------------------------------------------------------------------

% This file is part of Code_Saturne, a general-purpose CFD tool.
%
% Copyright (C) 1998-2018 EDF S.A.
%
% This program is free software; you can redistribute it and/or modify it under
% the terms of the GNU General Public License as published by the Free Software
% Foundation; either version 2 of the License, or (at your option) any later
% version.
%
% This program is distributed in the hope that it will be useful, but WITHOUT
% ANY WARRANTY; without even the implied warranty of MERCHANTABILITY or FITNESS
% FOR A PARTICULAR PURPOSE.  See the GNU General Public License for more
% details.
%
% You should have received a copy of the GNU General Public License along with
% this program; if not, write to the Free Software Foundation, Inc., 51 Franklin
% Street, Fifth Floor, Boston, MA 02110-1301, USA.

%-------------------------------------------------------------------------------

\programme{cs\_elec\_model}\label{ap:elbase}

\hypertarget{electric}{}

\vspace{0,5cm}
On s'int\'eresse \`a la r\'esolution des \'equations de la
magn\'etohydrodynamique, constitu\'ees de la r\'eunion des \'equations de
l'a\'erothermodynamique et des \'equations de Maxwell.

On se place dans deux cadres d'utilisation bien sp\'ecifiques et distincts,
qui permettront chacun de r\'ealiser des simplifications~: les \'etudes dites
``d'arc
\'electrique'' (dans lesquelles sont prises en compte les forces de Laplace et
l'effet Joule) et les \'etudes dites ``Joule'' (dans lesquelles seul
l'effet Joule est pris en compte).

Les \'etudes d'arc \'electrique sont associ\'ees en grande partie, pour EDF, aux
probl\'ematiques relatives aux transformateurs. Les \'etudes Joule sont plus
s\'ecifiquement li\'ees aux ph\'enom\`enes rencontr\'es dans les fours verriers.

Outre la prise en compte ou non des forces de Laplace, ces deux types d'\'etudes
se diff\'erencient \'egalement par le mode de d\'etermination de l'effet Joule
(utilisation d'un potentiel complexe pour les \'etudes Joule faisant intervenir
un courant alternatif non monophas\'e).

On d\'ecrit tout d'abord les \'equations r\'esolues pour les \'etudes d'arc
\'electrique. Les sp\'ecificit\'es des \'etudes Joule seront abord\'ees ensuite.


Pour l'arc \'electrique,
les r\'ef\'erences [douce] et [delalondre] pourront compl\'eter la
pr\'esentation~:

\noindent{\bf [delalondre] }Delalondre, Clarisse~: ``Mod\'elisation a\'erothermodynamique d'arcs
\'electriques \`a forte intensit\'e avec prise en compte du d\'es\'equilibre
thermodynamique local et du transfert thermique \`a la cathode'', Th\`ese de
l'Universit\'e de Rouen, 1990

\noindent{\bf [douce]} Douce, Alexandre~: ``Mod�lisation 3-D du chauffage d'un
bain m�tallique par plasma d'arc transf\'er\'e. Application \`a un r\'eacteur
axisym\'etrique'', HE-26/99/027A, HE-44/99/043A, Th\`ese de l'Ecole Centrale
Paris et EDF, 1999

See the \doxygenfile{cs__elec__model_8c.html}{programmers reference of the dedicated subroutine} for further details.

%%%%%%%%%%%%%%%%%%%%%%%%%%%%%%%%%%
%%%%%%%%%%%%%%%%%%%%%%%%%%%%%%%%%%
\section*{Fonction}
%%%%%%%%%%%%%%%%%%%%%%%%%%%%%%%%%%
%%%%%%%%%%%%%%%%%%%%%%%%%%%%%%%%%%

\subsection*{Notations}

{\bf Variables utilis\'ees}
\nopagebreak

\begin{tabular}{lp{6cm}l}
$\vect{A}$        &potentiel vecteur r\'eel        &$kg\,m\,s^{-2}\,A^{-1}$ \\
$\vect{B}$        &champ magn\'etique                 &$T$ (ou $kg\,s^{-2}\,A^{-1}$) \\
$\vect{D}$         &d\'eplacement \'electrique        &$A\,s\,m^{-2}$ \\
$\vect{E}$         &champ \'electrique                 &$V\,m^{-1}$ \\
$E$                 &\'energie totale massique         &$J\,kg^{-1}$ (ou $m^{2}\,s^{-2}$) \\
$e$                 &\'energie interne massique         &$J\,kg^{-1}$ (ou $m^{2}\,s^{-2}$) \\
$e_c$                 &\'energie cin\'etique massique &$J\,kg^{-1}$ (ou $m^{2}\,s^{-2}$) \\
$\vect{H}$         &excitation magn\'etique        &$A\,m^{-1}$ \\
$h$                   &enthalpie massique                 &$J\,kg^{-1}$ (ou $m^{2}\,s^{-2}$) \\
$\vect{j}$        &densit\'e de courant                 &$A\,m^{-2}$  \\
$P$                   &pression                         &$kg\,m^{-1}\,s^{-2}$ \\
$P_R$, $P_I$         &potentiel scalaire r\'eel, imaginaire
                                                &$V$ (ou $kg\,m^{2}\,s^{-3}\,A^{-1}$) \\
$\vect{u}$         &vitesse                        &$m\,s^{-1}$  \\
                 &                                & \\
$\varepsilon$ &permittivit\'e \'electrique
                                                &$F\,m^{-1}$ (ou $m^{-3}\,kg^{-1}\,s^{4}\,A^{2}$) \\
$\varepsilon_0$ &permittivit\'e \'electrique du vide
                                                &$8,854\,10^{-12}\,\,F\,m^{-1}$ (ou $m^{-3}\,kg^{-1}\,s^{4}\,A^{2}$) \\
$\mu$           &perm\'eabilit\'e \'electrique
                                                &$H\,m^{-1}$ (ou $m\,kg\,s^{-2}\,A^{-2}$)\\
$\mu_0$         &perm\'eabilit\'e \'electrique du vide
                                                &$4\,\pi\,10^{-7}\,\,H\,m^{-1}$ (ou $m\,kg\,s^{-2}\,A^{-2}$)\\
$\sigma$         &conductivit\'e \'electrique        &$S\,m^{-1}$ (ou $m^{-3}\,kg^{-1}\,s^3\,A^2$)\\
\end{tabular}

\vspace*{0,5cm}
{\bf Notations d'analyse vectorielle}
\nopagebreak

On rappelle \'egalement la d\'efinition des notations employ\'ees\footnote{en
utilisant la convention de sommation d'Einstein.}~:
\begin{equation}\notag
\left\{\begin{array}{lll}
\left[\ggrad{\vect{a}}\right]_{ij} &=& \partial_j a_i\\
\left[\dive(\tens{\sigma})\right]_i &=& \partial_j \sigma_{ij}\\
\left[\vect{a}\otimes\vect{b}\right]_{ij} &= &
a_i\,b_j\\
\end{array}\right.
\end{equation}
et donc :
\begin{equation}\notag
\begin{array}{lll}
\left[\dive(\vect{a}\otimes\vect{b})\right]_i &= &
\partial_j (a_i\,b_j)
\end{array}
\end{equation}


\subsection*{Arcs \'electriques}

\subsubsection*{Introduction}

Pour les \'etudes d'arc \'electrique, on calcule,
\`a un pas de temps donn\'e~:
\begin{itemize}
\item la vitesse $\vect{u}$, la pression $P$, la variable \'energ\'etique
enthalpie $h$ (et les grandeurs turbulentes),
\item un potentiel scalaire r\'eel $P_R$
(dont le gradient permet d'obtenir le champ \'electrique $\vect{E}$ et
la densit\'e de courant $\vect{j}$),
\item un potentiel vecteur r\'eel $\vect{A}$ (dont
le rotationnel permet d'obtenir  le champ magn\'etique $\vect{B}$).
\end{itemize}

\bigskip
Le champ \'electrique, la
densit\'e de courant et le champ magn\'etique sont utilis\'es pour calculer les
termes sources d'effet Joule et les forces de Laplace qui interviennent
respectivement dans l'\'equation de l'enthalpie et dans celle
de la quantit\'e de mouvement.


\subsubsection*{\'Equations continues}

{\bf Syst\`eme d'\'equations}
\nopagebreak

Les \'equations continues qui sont r\'esolues sont les suivantes~:
\begin{equation}
\left\{\begin{array}{l}
{\color{blue}\dive(\rho \vect{u}) = 0}\\
{\color{blue}\displaystyle\frac{\partial}{\partial t}(\rho \vect{u})
+\dive(\rho\, \vect{u} \otimes \vect{u})
=\dive(\tens{\sigma}) + \vect{TS} + {\color{red}\vect{j} \times \vect{B}}}\\
{\color{blue}\displaystyle\frac{\partial}{\partial t}(\rho h)
+\dive(\rho\, \vect{u} h)
=\Phi_v +
\dive{\left(\left(\frac{\lambda}{C_p}+\frac{\mu_t}{\sigma_t}\right)\grad{h}\right)}
+ {\color{red}P_J}}\\
{\color{red}\dive(\sigma\,\grad{P_R})=0}\\
{\color{red}\dive(\ggrad{\vect{A}})=-\mu_0\vect{j}}
\end{array}\right.
\end{equation}

avec les relations suivantes~:
\begin{equation}
\left\{\begin{array}{l}
{\color{red}P_J=\vect{j}\cdot\vect{E}}\\
{\color{red}\vect{E}=-\grad{P_R}}\\
{\color{red}\vect{j}=\sigma\vect{E}}\\
\end{array}\right.
\end{equation}

%On donne ci-apr\`es diff\'erents \'el\'ements permettant de pr\'eciser le mode
%d'obtention de ces \'equations.

\vspace*{0,5cm}
{\bf \'Equation de la masse}
\nopagebreak

C'est l'\'equation r\'esolue en standard par \CS (contrainte
stationnaire).
Elle n'a pas de traitement particulier dans
le cadre du module pr\'esent. Un terme source de  masse peut \^etre pris en
compte au second membre si l'utilisateur le souhaite. Pour simplifier l'expos\'e
le terme source sera suppos\'e nul ici, dans la mesure o\`u il n'est pas
sp\'ecifique au module \'electrique.

\vspace*{0,5cm}
{\bf \'Equation de la quantit\'e de mouvement}
\nopagebreak

Elle pr\'esente, par rapport \`a
l'\'equation standard r\'esolue par \CS, un seul terme additionnel
($\vect{j} \times \vect{B}$) qui rend compte des forces de Laplace.
Pour l'obtenir, on fait l'hypoth\`ese que le milieu est
\'electriquement neutre.

En effet, une charge $q_i$ (Coulomb) anim\'ee d'une
vitesse $\vect{v}_i$ subit,
sous l'effet du champ \'electrique $\vect{E}$ ($V\,m^{-1}$) et du champ magn\'etique
$\vect{B}$ (Tesla),  une force $\vect{f}_i$ ($kg\,m\,s^{-2}$)~:
\begin{equation}
\vect{f}_i=q_i\left(\vect{E} + \vect{v}_i \times \vect{B}\right)
\end{equation}
Avec $n_i$ charges de type $q_i$ par unit\'e de volume et en sommant sur tous
les types de charge $i$ (\'electrons, ions, mol\'ecules ionis\'ees...), on
obtient la force de Laplace totale $\vect{F}_L$ ($kg\,m^{-2}\,s^{-2}$) subie par unit\'e de
volume~:
\begin{equation}
\vect{F}_L=\sum\limits_i\left[{n_i\,q_i\left(\vect{E} + \vect{v}_i \times \vect{B}\right)}\right]
\end{equation}
On introduit alors la densit\'e de courant $\vect{j}$ ($A\,m^{-2}$)~:
\begin{equation}
\vect{j}=\sum\limits_i n_i\,q_i\,\vect{v}_i
\end{equation}
Avec l'hypoth\`ese que le milieu est \'electriquement neutre (\`a un
niveau macroscopique)~:
\begin{equation}
\sum\limits_i n_i\,q_i = 0
\end{equation}
la force totale $\vect{F}_L$ s'\'ecrit alors~:
\begin{equation}
\vect{F}_L=\vect{j} \times \vect{B}
\end{equation}
et on peut donc \'ecrire l'\'equation de la quantit\'e de mouvement~:
\begin{equation}
\displaystyle{\color{blue}\frac{\partial}{\partial t}(\rho \vect{u})
+\dive(\rho\, \vect{u} \otimes \vect{u})
=\dive(\tens{\sigma}) + \vect{TS} + {\color{red}\vect{j} \times \vect{B}}}
\end{equation}

\vspace*{0,5cm}
{\bf \'Equation de l'enthalpie}
\nopagebreak

Elle est obtenue \`a partir de l'�quation de
l'\'energie apr\`es plusieurs approximations utilis\'ees en standard dans \CS et en
prenant en compte le terme d'effet Joule li\'e \`a l'\'energie
\'electromagn\'etique.

\underline{\'Energie \'electromagn\'etique}
\nopagebreak

Avec les m\^emes notations que pr\'ec\'edemment mais sans qu'il soit
besoin de supposer que le milieu est \'electriquement neutre,
la puissance re\c cue par une charge $q_i$ (particule dou\'ee de masse)
de vitesse $\vect{v}_i$ (vitesse du porteur de charge, contenant \'eventuellement
l'effet de la vitesse du fluide) sous l'effet
du champ \'electrique $\vect{E}$ ($V\,m^{-1}$)
et du champ magn\'etique  $\vect{B}$ ($T$) est (sans
sommation sur $i$)~:
\begin{equation}
P_i=\vect{f}_i\cdot\vect{v}_i=
q_i(\vect{E}+\vect{v}_i\times\vect{B})\cdot\vect{v}_i
= q_i\vect{v}_i\cdot\vect{E}
\end{equation}
Avec $n_i$ charges par unit\'e de volume et en sommant sur tous les types
de charges $i$, on obtient la puissance totale par unit\'e de volume~:
\begin{equation}
P_J=
\sum\limits_i n_i\,q_i\,\vect{v}_i\cdot\vect{E}
\end{equation}
On introduit alors la densit\'e de courant $\vect{j}=\sum\limits_i n_i\,q_i\,\vect{v}_i$ (en $A\,m^{-2}$) et on obtient l'expression usuelle de la puissance
\'electromagn\'etique dissip\'ee par effet Joule (en $W\,m^{-3}$)~:
\begin{equation}
P_J=\vect{j}\cdot\vect{E}
\end{equation}

Pour reformuler la puissance dissip\'ee par effet Joule et obtenir
une \'equation d'\'evolution de l'\'energie \'electromagn\'etique, on utilise
alors les \'equations de Maxwell.
Les \'equations s'\'ecrivent (lois d'Amp\`ere
et de Faraday)~:
\begin{equation}
\left\{
\begin{array}{l}
\displaystyle\frac{\partial \vect{D}}{\partial t} - \rot\vect{H} = -\vect{j}\\
\displaystyle\frac{\partial \vect{B}}{\partial t} + \rot\vect{E} = 0
\end{array}
\right.
\end{equation}

On a donc~:
\begin{equation}
P_J=\vect{j}\cdot\vect{E}=\left(-\frac{\partial \vect{D}}{\partial t} +\rot\vect{H}\right)\cdot\vect{E}
\end{equation}
On utilise alors la relation suivante~:
\begin{equation}
\rot\vect{H}\cdot\vect{E}=\vect{H}\cdot\rot\vect{E}-\dive(\vect{E}\times\vect{H})
\end{equation}
En effet, elle permet de faire appara\^\i tre un
terme en divergence, caract\'eristique d'une redistribution spatiale~:
\begin{equation}
\begin{array}{lll}
\vect{j}\cdot\vect{E}&=&
\displaystyle-\frac{\partial \vect{D}}{\partial t}\cdot\vect{E}
+\vect{H}\cdot\rot\vect{E}-\dive(\vect{E}\times\vect{H})\\
\end{array}
\end{equation}
Et en utilisant la loi de Faraday pour faire appara\^\i tre la d\'eriv\'ee en temps du champ magn\'etique~:
\begin{equation}
\begin{array}{lll}
\vect{j}\cdot\vect{E}&=&
\displaystyle-\frac{\partial \vect{D}}{\partial t}\cdot\vect{E}
-\vect{H}\cdot\frac{\partial \vect{B}}{\partial t}-\dive(\vect{E}\times\vect{H})\\
\end{array}
\end{equation}

Dans le cadre de \CS, on fait les hypoth\`eses suivantes~:
\begin{itemize}
\item la perm\'eabilit\'e $\varepsilon$ et la permittivit\'e $\mu$
sont constantes et uniformes (pour les gaz, en pratique, on utilise
les propri\'et\'es du vide $\varepsilon_0$ et $\mu_0$).
\item on utilise $\vect{B} = \mu \vect{H}$ et $\vect{D} = \varepsilon \vect{E}$
\end{itemize}

On a alors~:

\begin{equation}
\begin{array}{lll}
\displaystyle\vect{j}\cdot\vect{E}&=&
\displaystyle-\frac{\varepsilon_0}{2}\frac{\partial E^2}{\partial t}
-\frac{1}{2\,\mu_0}\frac{\partial B^2}{\partial t}
-\frac{1}{\mu_0}\dive(\vect{E}\times\vect{B})
\end{array}
\end{equation}

\underline{\'Energie totale}
\nopagebreak

On \'etablit l'\'equation de l'\'energie totale en prenant en compte la
puissance des forces de Laplace et le terme
d'effet Joule.

Sans prendre en compte l'\'energie \'electromagn\'etique,
le premier principe de la thermodynamique s'\'ecrit d'ordinaire sous la
forme suivante (pour un volume mat\'eriel suivi sur une unit\'e de temps)~:
\begin{equation}\label{Elec_Elbase_premier_ppe_eq}
d\int_V \rho E dV=\delta Q+\delta W
\end{equation}
Dans cette relation, $E$ est l'\'energie totale par unit\'e de masse\footnote{Ne pas
confondre le scalaire $E$, \'energie totale, avec le vecteur $\vect{E}$, champ
\'electrique.}, soit $E=e+e_c$, $e$ \'etant l'\'energie interne massique et
$e_c=\frac{1}{2}\,\vect{u}\cdot\vect{u}$ l'\'energie cin\'etique massique. Le terme
$\delta Q$  repr\'esente  la chaleur re\c cue au travers des fronti\`eres du
domaine consid\'er\'e tandis que le terme  $\delta W$ repr\'esente le travail
des forces ext\'erieures re\c cu par le syst\`eme (y compris les forces
d\'erivant d'une \'energie potentielle).

Pour prendre en compte l'\'energie \'electromagn\'etique, il suffit d'int\'egrer
\`a la relation (\ref{Elec_Elbase_premier_ppe_eq}) la puissance des forces de Laplace $(\vect{j} \times
\vect{B})\cdot\vect{u}$ et le terme d'effet Joule $\vect{j}\cdot\vect{E}$
(transformation volumique d'\'energie \'electromagn\'etique en \'energie
totale\footnote{Le terme en divergence
$-\frac{1}{\mu_0}\dive(\vect{E}\times\vect{B})$
traduit une redistribution spatiale d'\'energie \'electromagn\'etique~:
ce n'est donc pas un terme source pour l'\'energie totale.}).
Dans cette relation, la vitesse $\vect{u}$ est la vitesse du fluide et non pas
celle des porteurs de charge~: elle n'est donc pas n\'ecessairement coli\'eaire
au vecteur \vect{j} (par exemple, si le courant est d\^u \`a des \'electrons,
la vitesse du fluide pourra \^etre consid\'er\'ee comme d\'ecorr\'el\'ee de la
vitesse des porteurs de charges~; par contre, si le courant est d\^u \`a des ions,
la vitesse du fluide pourra \^etre plus directement influenc\'ee par
le d\'eplacement des porteurs de charge).
Ainsi, le premier principe de la thermodynamique s'\'ecrit~:
\begin{equation}
d\int_V \rho E dV=\delta Q+\delta W+\vect{j}\cdot\vect{E}\,V\,dt +(\vect{j} \times
\vect{B})\cdot\vect{u}\,V\,dt
\end{equation}
et l'\'equation locale pour l'\'energie totale est alors~:
\begin{equation}
\displaystyle\frac{\partial}{\partial t}(\rho E)
+\dive(\rho\, \vect{u} E)
=\dive(\tens{\sigma}\,\vect{u}) + \vect{TS}\cdot\vect{u} + {\color{red}(\vect{j} \times
\vect{B})\cdot\vect{u}} + \Phi_v - \dive{\vect{\Phi}_s} + {\color{red}\vect{j}\cdot \vect{E}}
\end{equation}
Le terme $\Phi_v$ repr\'esente les termes sources volumiques d'\'energie
autres que l'effet Joule (par exemple, il inclut le terme source de
rayonnement, pour un milieu optiquement non transparent). Le terme $\vect{\Phi}_s$ est
le flux d'\'energie surfacique\footnote{Dans \CS, il est mod\'elis\'e par
une hypoth\`ese de gradient et inclut �galement la ``diffusion'' turbulente.}.

\underline{Enthalpie}
\nopagebreak

Pour obtenir une \'equation sur l'enthalpie, qui est la variable \'energ\'etique
choisie dans \CS dans le module \'electrique, on
soustrait tout d'abord \`a l'\'equation de l'\'energie totale celle de l'\'energie
cin\'etique pour obtenir une \'equation sur l'\'energie interne.

L'\'equation de l'\'energie
cin\'etique (obtenue \`a partir de l'\'equation de la quantit\'e de mouvement
\'ecrite sous forme non conservative) est~:
\begin{equation}
\displaystyle\frac{\partial}{\partial t}(\rho e_c)
+\dive(\rho\, \vect{u} e_c)
=\dive(\tens{\sigma}\,\vect{u}) - \tens{\sigma}:\left(\ggrad(\vect{u})\right)^t +
\vect{TS}\cdot\vect{u} + (\vect{j} \times \vect{B})\cdot\vect{u}
\end{equation}
de sorte que, pour l'\'energie interne, on a~:
\begin{equation}
\displaystyle\frac{\partial}{\partial t}(\rho e)
+\dive(\rho\, \vect{u} e)
=\tens{\sigma}:\left(\ggrad(\vect{u})\right)^t + \Phi_v - \dive{\Phi_s} + \vect{j}\cdot \vect{E}
\end{equation}
et enfin, pour l'enthalpie $h=e+\frac{P}{\rho}$~:
\begin{equation}
\displaystyle\frac{\partial}{\partial t}(\rho h)
+\dive(\rho\, \vect{u} h)
=\tens{\sigma}:\left(\ggrad(\vect{u})\right)^t + \Phi_v - \dive{\Phi_s} + \vect{j}\cdot \vect{E}+\rho\frac{d}{dt}\left(\frac{P}{\rho}\right)
\end{equation}
En faisant appara\^itre la pression dans le tenseur des contraintes
$\tens{\sigma}=-P\tens{Id}+\tens{\tau}$, on peut \'ecrire~:
\begin{equation}
\displaystyle\frac{\partial}{\partial t}(\rho h)
+\dive(\rho\, \vect{u} h)
=\tens{\tau}:\left(\ggrad(\vect{u})\right)^t + \Phi_v - \dive{\Phi_s}
+ \vect{j}\cdot\vect{E} + \frac{dP}{dt}
\end{equation}

Les approximations habituelles de \CS consistent alors
\`a n\'egliger le terme ``d'\'echauffement'' issu du tenseur des contraintes
$\tens{\tau}:\left(\ggrad(\vect{u})\right)^t$ et le terme en d\'eriv\'ee totale de la
pression $\frac{dP}{dt}$, suppos\'es faibles en comparaison des autres termes
dans les applications trait\'ees (exemple~: terme d'effet Joule important, effets de
compressibilit\'e faibles...).
De plus, le terme de flux est mod\'elis\'e en suivant
une hypoth\`ese de gradient appliqu\'e \`a l'enthalpie (et non pas \`a la
temp\'erature), soit donc~:
\begin{equation}
{\color{blue}\displaystyle\frac{\partial}{\partial t}(\rho h)
+\dive(\rho\, \vect{u} h)
=\Phi_v -
\dive{\left(\left(\frac{\lambda}{C_p}+\frac{\mu_t}{\sigma_t}\right)\grad h\right)} + {\color{red}\vect{j}\cdot\vect{E}}}
\end{equation}


\vspace*{0,5cm}
{\bf \'Equations \'electromagn\'etiques}
\nopagebreak

Elles sont obtenues \`a partir des
\'equations de Maxwell sous les hypoth\`eses d\'etaill\'ees dans [douce],
paragraphe 3.3.

\underline{Densit\'e de courant}
\nopagebreak

La relation liant la densit\'e de courant et le champ \'electrique est issue de
la loi d'Ohm que l'on suppose pouvoir utiliser sous la forme
simplifi\'ee suivante~:
\begin{equation}\label{Elec_Elbase_ohm_eq}
{\color{red}\vect{j}=\sigma\,\vect{E}}
\end{equation}

\underline{Champ \'electrique}
\nopagebreak

Le champ \'electrique s'obtient \`a partir d'un potentiel vecteur.

En effet, la loi de Faraday s'\'ecrit~:
\begin{equation}
\frac{\partial\vect{B}}{\partial t}+\rot\vect{E}=0
\end{equation}
Avec une hypoth\`ese quasi-stationnaire, il reste~:
\begin{equation}
\rot\vect{E}=0
\end{equation}
Il est donc possible de postuler l'existence d'un potentiel scalaire $P_R$
tel que~:
\begin{equation}\label{Elec_Elbase_e_eq}
{\color{red}\vect{E}=-\grad{P_R}}
\end{equation}

\underline{Potentiel scalaire}
\nopagebreak

Le potentiel scalaire est solution d'une \'equation de Poisson.

En effet, la conservation de la charge $q$ s'\'ecrit~:
\begin{equation}
\displaystyle\frac{\partial q}{\partial t}
+\dive(\vect{j}) = 0
\end{equation}
Pour un milieu \'electriquement neutre (\`a l'\'echelle macroscopique), on a
$\displaystyle\frac{\partial q}{\partial t}=0$ soit donc~:
\begin{equation}
\dive(\vect{j}) = 0
\end{equation}
C'est-\`a-dire, avec la loi d'Ohm (\ref{Elec_Elbase_ohm_eq}),
\begin{equation} \label{Elec_Elbase_div_sigma_e_eq}
\dive(\sigma\,\vect{E}) = 0
\end{equation}
Avec (\ref{Elec_Elbase_e_eq}), on obtient donc une \'equation permettant de
calculer le potentiel scalaire~:
\begin{equation}
{\color{red}\dive(\sigma\,\grad{P_R}) = 0}
\end{equation}

\underline{Champ magn\'etique}
\nopagebreak

Le champ magn\'etique s'obtient \`a partir d'un potentiel vecteur.

En effet, la loi d'Amp\`ere s'\'ecrit~:
\begin{equation}
\displaystyle\frac{\partial\vect{D}}{\partial t}-\rot\vect{H}=-\vect{j}
\end{equation}
Sous les hypoth\`eses indiqu\'ees pr\'ec\'edemment, on \'ecrit~:
\begin{equation}
\displaystyle\varepsilon_0\,\mu_0\,\frac{\partial\vect{E}}{\partial t}-\rot\vect{B}=-\mu_0\vect{j}
\end{equation}
Avec une hypoth\`ese quasi-stationnaire, il reste~:
\begin{equation}\label{Elec_Elbase_rot_b_eq}
\rot\vect{B}=\mu_0\vect{j}
\end{equation}
De plus, la conservation du flux magn\'etique s'\'ecrit\footnote{Prendre la
divergence de la loi de Faraday, avec $\dive(\rot\vect{E})=0$ (par analyse
vectorielle) donne $\dive\vect{B} = \text{cst}$.}~:
\begin{equation}
\dive\,\vect{B} = 0
\end{equation}
et on peut donc postuler l'existence d'un potentiel vecteur $\vect{A}$ tel que~:
\begin{equation}\label{Elec_Elbase_b_eq}
{\color{red}\vect{B} = \rot{\vect{A}}}
\end{equation}

\underline{Potentiel vecteur}
\nopagebreak

Le potentiel vecteur est solution d'une \'equation de Poisson.

En prenant le rotationnel de (\ref{Elec_Elbase_b_eq}) et avec (\ref{Elec_Elbase_rot_b_eq}), on obtient~:
\begin{equation}
-\rot(\rot{\vect{A}}) = -\mu_0\vect{j}
\end{equation}
Avec la relation donnant le Laplacien\footnote{En
coordonn\'ees cart\'esiennes, le
Laplacien du vecteur $\vect{a}$ est le vecteur dont les
composantes sont \'egales au Laplacien de chacune des composantes de $\vect{a}$.}
d'un vecteur $\dive(\ggrad\vect{a})=\grad(\dive{\vect{a}})-\rot(\rot\vect{a})$ et sous
la contrainte\footnote{La condition $\dive\vect{A}=0$, dite ``jauge de
Coulomb'', est n\'ecessaire pour assurer
l'unicit\'e du potentiel vecteur.}
que $\dive\vect{A}=0$, on obtient finalement une \'equation
permettant de calculer le potentiel vecteur~:
\begin{equation}
{\color{red}\dive\,(\ggrad{\vect{A}}) = -\mu_0\vect{j}}
\end{equation}



\subsection*{Effet Joule}

\subsubsection*{Introduction}

Pour les \'etudes Joule, on calcule,
\`a un pas de temps donn\'e~:
\begin{itemize}
\item la vitesse $\vect{u}$, la pression $P$, la variable \'energ\'etique
enthalpie $h$ (et les grandeurs turbulentes \'eventuelles),
\item un potentiel scalaire r\'eel $P_R$,
\item et, si le courant n'est ni continu, ni alternatif
monophas\'e, un potentiel scalaire imaginaire $P_I$.
\end{itemize}

\bigskip
Le gradient du potentiel permet d'obtenir  le champ \'electrique $\vect{E}$ et
la densit\'e de courant $\vect{j}$ (partie r\'eelle et, \'eventuellement, partie
imaginaire). Le champ \'electrique et la
densit\'e de courant sont utilis\'es pour calculer le
terme source d'effet Joule qui intervient dans l'\'equation de l'enthalpie.

{\bf La puissance instantan\'ee dissip\'ee par effet Joule} est \'egale
au produit instantan\'e $\vect{j}\cdot\vect{E}$.
Dans le cas g\'en\'eral, $\vect{j}$ et $\vect{E}$ sont des signaux alternatifs
($\vect{j}=\vect{|j|}cos(\omega\,t+\phi_j)$ et
$\vect{E}=\vect{|E|}cos(\omega\,t+\phi_E)$) que l'on peut repr\'esenter
par des complexes
($\vect{j}=\vect{|j|}\,e^{i\,(\omega\,t+\phi_j)}$  et
$\vect{E}=\vect{|E|}\,e^{i\,(\omega\,t+\phi_E)}$).
La puissance instantan\'ee s'\'ecrit alors
$(\vect{|j|}\cdot\vect{|E|})cos(\omega\,t+\phi_j)cos(\omega\,t+\phi_E)$.

\begin{itemize}

\item {\bf En courant continu} ($\omega=\phi_j=\phi_E=0$),
la puissance se calcule donc simplement comme
le produit scalaire $P_J=\vect{|j|}\cdot\vect{|E|}$.
Le calcul de la puissance
dissip\'ee par effet Joule ne pose donc pas de probl\`eme particulier
car les variables densit\'e de courant et champ \'electrique r\'esolues
par \CS sont pr\'ecis\'ement $\vect{|j|}$ et $\vect{|E|}$ (les variables sont r\'eelles).

\item {\bf En courant alternatif}, la periode du courant est beaucoup plus petite que
les \'echelles de temps des ph\'enom\`enes thermohydrauliques pris en compte.
Il n'est donc pas utile de disposer de la puissance instantan\'ee dissip\'ee
par effet Joule~: la moyenne sur une p\'eriode est suffisante et elle
s'\'ecrit\footnote{L'int\'egrale de $cos^2 x$ sur un intervalle de longueur
$2\,\pi$ est $\pi$.}~:
% eh oui, car l'integrale de cos^2+sin^2 (qui vaut 1), c'est 2 \pi !!
$P_J=\frac{1}{2}(\vect{|j|}\cdot\vect{|E|})cos(\phi_j-\phi_E)$. Cette formule
peut \'egalement s'\'ecrire de mani\`ere \'equivalente sous forme complexe~:
$P_J=\frac{1}{2}\vect{j}\cdot\vect{E}^*$, o\`u $\vect{E}^*$ est le complexe
conjugu\'e de $\vect{E}$.

  \begin{itemize}
  \item En courant alternatif monophas\'e ($\phi_j=\phi_E$), en particulier,
la formule donnant la puissance se simplifie sous la forme
$P_J=\frac{1}{2}(\vect{|j|}\cdot\vect{|E|})$, ou encore~:
$P_J=\frac{1}{\sqrt{2}}\vect{|j|}\cdot\frac{1}{\sqrt{2}}\vect{|E|}$.
Il s'agit donc du produit des valeurs efficaces. Or, les variables r\'esolues
par \CS en courant alternatif monophas\'e sont pr\'ecis\'ement les
valeurs efficaces (valeurs que l'on
d\'enomme abusivement "valeurs r\'eelles" dans le code source).

  \item En courant alternatif non monophas\'e (triphas\'e, en particulier),
la formule donnant la puissance est utilis\'ee directement sous la forme
$P_J=\frac{1}{2}\vect{j}\cdot\vect{E}^*$.
On utilise pour la calculer les variables r\'esolues qui sont
la partie r\'eelle et la partie imaginaire de $\vect{j}$ et $\vect{E}$.

  \end{itemize}

\item {\bf En conclusion},

  \begin{itemize}
  \item en continu, les variables r\'esolues
$\vect{j}_{Res}$ et $\vect{E}_{Res}$
sont les variables r\'eelles continues
et la puissance se calcule par la formule suivante~:
$P_J=\vect{j}_{Res}\cdot\vect{E}_{Res}$
  \item en alternatif monophas\'e, les variables r\'esolues
$\vect{j}_{Res}$ et $\vect{E}_{Res}$
sont les valeurs efficaces
et la puissance se calcule par la formule suivante~:
$P_J=\vect{j}_{Res}\cdot\vect{E}_{Res}$
  \item en alternatif non monophas\'e, les variables r\'esolues
$\vect{j}_{Res,R}$, $\vect{j}_{Res,I}$ et $\vect{E}_{Res,R}$, $\vect{E}_{Res,I}$
sont la partie r\'eelle et la partie imaginaire de $\vect{j}$ et $\vect{E}$,
et la puissance se calcule par la formule suivante~:
$P_J=\frac{1}{2}(\vect{j}_{Res,R}\cdot\vect{E}_{Res,R}-\vect{j}_{Res,I}\vect{E}_{Res,I})$
  \end{itemize}

\end{itemize}


{\bf Le potentiel imaginaire n'est donc utilis\'e dans le code que lorsque
le courant est alternatif et non monophas\'e.}
En particulier, le potentiel imaginaire n'est pas utilis\'e lorsque le courant est
continu ou alternatif monophas\'e.
En effet, la partie imaginaire n'est introduite en compl\'ement de la partie
r\'eelle que dans le cas o\`u il est n\'ecessaire de disposer de deux grandeurs
pour d\'efinir le potentiel, c'est-\`a-dire lorsqu'il importe de conna\^itre son
amplitude et sa phase.
En courant continu, on n'a naturellement besoin que d'une seule information. En
alternatif monophas\'e, la valeur de la phase importe peu
(on ne travaille pas sur des grandeurs \'electriques instantan\'ees)~:
il suffit de conna\^itre l'amplitude du potentiel et il est donc inutile
d'introduire une variable imaginaire.


{\bf La variable d\'enomm\'ee ``potentiel r\'eel'', $P_R$, repr\'esente une
valeur efficace
si le courant est monophas\'e et une partie r\'eelle sinon.}
De mani\`ere plus explicite, pour un potentiel physique alternatif sinuso\"idal
$Pp$, de valeur maximale not\'ee $Pp_\text{max}$, de phase not\'ee $\phi$, la
variable $P_R$ repr\'esente $\frac{1}{\sqrt{2}}\,Pp_\text{max}$ en
monophas\'e et $Pp_\text{max}\,cos\phi$ sinon. En courant continu, $P_R$
repr\'esente naturellement le potentiel (r\'eel, continu).
{\bf Il est donc indispensable de pr\^eter une attention particuli\`ere aux
valeurs de potentiel impos\'ees aux limites} (facteur $\frac{1}{\sqrt{2}}$ ou
$cos\phi$).

\subsubsection*{\'Equations continues}

{\bf Syst\`eme d'\'equations}
\nopagebreak

Les \'equations continues qui sont r\'esolues sont les suivantes~:
\begin{equation}
\left\{\begin{array}{l}
{\color{blue}\dive(\rho \vect{u}) = 0}\\
{\color{blue}\displaystyle\frac{\partial}{\partial t}(\rho \vect{u})
+\dive(\rho\, \vect{u} \otimes \vect{u})
=\dive(\tens{\sigma}) + \vect{TS}}  \\
{\color{blue}\displaystyle\frac{\partial}{\partial t}(\rho h)
+\dive(\rho\, \vect{u} h)
=\Phi_v +
\dive{\left(\left(\frac{\lambda}{C_p}+\frac{\mu_t}{\sigma_t}\right)\grad{h}\right)}
+ {\color{red}P_J}}\\
{\color{red}\dive(\sigma\,\grad{P_R})=0}\\
{\color{red}\dive(\sigma\,\grad{P_I})=0}\text{\ \ \ en alternatif non monophas\'e uniquement}\\
\end{array}\right.
\end{equation}

avec, en continu ou alternatif monophas\'e~:
\begin{equation}
\left\{\begin{array}{l}
{\color{red}P_J=\vect{j}\cdot\vect{E}} \\
{\color{red}\vect{E}=-\grad{P_R}}\\
{\color{red}\vect{j}=\sigma\vect{E}}\\
\end{array}\right.
\end{equation}

et, en alternatif non monophas\'e (avec $i^2=-1$)~:
\begin{equation}
\left\{\begin{array}{l}
{\color{red}\displaystyle P_J=\frac{1}{2}\,\vect{j}\cdot\vect{E}^*}\\
{\color{red}\vect{E}=-\grad{(P_R+i\,P_I)}}\\
{\color{red}\vect{j}=\sigma\vect{E}}\\
\end{array}\right.
\end{equation}

\vspace*{0,5cm}
{\bf \'Equation de la masse}
\nopagebreak

C'est l'\'equation r\'esolue en standard par \CS (contrainte
stationnaire d'incompressibilit\'e). Elle n'a pas de traitement particulier dans
le cadre du module pr\'esent. Un terme source de  masse peut \^etre pris en
compte au second membre si l'utilisateur le souhaite. Pour simplifier
l'expos\'e,
le terme source sera suppos\'e nul ici, dans la mesure o\`u il n'est pas
sp\'ecifique au module \'electrique.

\vspace*{0,5cm}
{\bf \'Equation de la quantit\'e de mouvement}
\nopagebreak

C'est l'\'equation r\'esolue en standard par \CS (les forces de Laplace
($\vect{j} \times \vect{B}$) sont suppos\'ees n\'egligeables).

\vspace*{0,5cm}
{\bf \'Equation de l'enthalpie}
\nopagebreak

On l'\'etablit comme dans le cas des arcs \'electriques\footnote{\`A ceci pr\`es
que la puissance des
forces de Laplace n'appara\^it pas du tout, au lieu de dispara\^itre lorsque
l'on soustrait l'\'equation de l'\'energie cin\'etique \`a celle de l'\'energie
totale.} \`a partir de l'�quation de
l'\'energie apr\`es plusieurs approximations utilis\'ees en standard dans \CS et en
prenant en compte le terme d'effet Joule li\'e \`a l'\'energie
\'electromagn\'etique.

Par rapport \`a l'\'equation utilis\'ee pour les \'etudes d'arc \'electrique,
seule l'expression de l'effet Joule diff\`ere lorsque le courant est
alternatif non monophas\'e.

\vspace*{0,5cm}
{\bf \'Equations \'electromagn\'etiques}
\nopagebreak

Elles sont obtenues comme indiqu\'e dans la partie relative aux arcs
\'electriques, mais on ne conserve que les relations associ\'ees \`a la
densit\'e de courant, au champ \'electrique et au potentiel dont il d\'erive.

%%%%%%%%%%%%%%%%%%%%%%%%%%%%%%%%%%
%%%%%%%%%%%%%%%%%%%%%%%%%%%%%%%%%%
\section*{Discr\'etisation}
%%%%%%%%%%%%%%%%%%%%%%%%%%%%%%%%%%
%%%%%%%%%%%%%%%%%%%%%%%%%%%%%%%%%%

La discr\'etisation des \'equations ne pose pas de probl\`eme particulier
(ajout de termes sources explicites pour l'effet Joule et les forces de Laplace,
\'equations de Poisson pour la d\'etermination des potentiels).

Un point sur les conditions aux limites doit cependant \^etre fait ici, en
particulier pour pr\'eciser la m\'ethode de recalage automatique des
potentiels.



\subsection*{Arcs \'electriques}

\subsubsection*{Conditions aux limites}

Seules les conditions aux limites pour les potentiels sont \`a pr\'eciser.

{\bf Les conditions aux limites sur le potentiel scalaire} sont des conditions de
Neumann homog\`enes sur toutes les fronti\`eres hormis \`a la cathode et \`a
l'anode. \`A la cathode, on impose une condition de Dirichlet homog\`ene (potentiel nul par convention). \`A l'anode, on impose une
condition de Dirichlet permettant de fixer la diff\'erence de potentiel
souhait\'ee entre l'anode et la cathode.
L'utilisateur peut fixer le potentiel de l'anode directement ou
demander qu'un recalage automatique du potentiel soit effectu\'e pour atteindre
une intensit\'e de courant pr\'ed\'etermin\'ee.

Lorsque le recalage automatique est demand\'e (\var{IELCOR}=1), l'utilisateur doit fixer la
valeur cible de l'intensit\'e, \var{COUIMP}, ($A$) et une valeur \'elev\'ee
de d\'epart
de la diff\'erence de potentiel entre l'anode et la cathode\footnote{Plus pr\'ecis\'ement, l'utilisateur doit imposer un potentiel nul
en cathode et le potentiel \var{DPOT} \`a l'anode, en utilisant explicitement,
dans le sous-programme utilisateur \fort{cs\_user\_boundary\_conditions}, la variable \var{DPOT} qui
sera automatiquement recal\'ee au cours du calcul.}, \var{DPOT}, ($V$).
Le recalage est effectu\'e en fin de pas temps et permet de disposer, pour le pas
de temps suivant, de valeurs recal\'ees des forces de Laplace et de l'effet
Joule.
\begin{itemize}
\item Pour effectuer le recalage, \CS d\'etermine l'int\'egrale de l'effet Joule
estim\'e sur le domaine (en $W$) et en compare la
valeur au produit de l'intensit\'e \var{COUIMP} par la diff\'erence de
potentiel\footnote{\var{DPOT} est la diff\'erence de
potentiel impos\'ee entre l'anode et la cathode au
pas de temps qui s'ach\`eve. \var{DPOT} a conditionn\'e le champ \'electrique et
la densit\'e de courant utilis\'es pour le calcul de l'effet Joule.} \var{DPOT}.
Un coefficient multiplicatif de recalage \var{COEPOT} en
est d\'eduit (pour \'eviter des variations trop brusques,
on s'assure qu'il reste born\'e). % entre 0,75 et 1,5).
\item On multiplie alors par \var{COEPOT} la
diff\'erence de potentiel entre l'anode et la cathode, \var{DPOT}, et le vecteur $\vect{j}$. L'effet
Joule, produit de $\vect{j}$ par $\vect{E}$, est multipli\'e par
le carr\'e de \var{COEPOT}. Pour assurer la coh\'erence du post-traitement des
variables, le potentiel vecteur et le potentiel scalaire sont \'egalement
multipli\'es par \var{COEPOT}.
\item Le champ \'electrique n'\'etant pas explicitement
stock\'e, on ne le recale pas. Le potentiel vecteur et les forces de Laplace seront d\'eduits de la
densit\'e de courant et int\'egreront donc naturellement le recalage.
\end{itemize}

\bigskip
{\bf Les conditions aux limites sur le potentiel vecteur} sont des conditions de
Neumann homog\`ene sur toutes les fronti\`eres hormis sur une zone de bord
arbitrairement choisie (paroi par exemple) pour laquelle une condition de
Dirichlet est utilis\'ee afin que le syst\`eme soit inversible
(la valeur impos\'ee est la valeur du potentiel vecteur
calcul\'ee au pas de temps pr\'ec\'edent).



\subsection*{Effet Joule}

\subsubsection*{Conditions aux limites}

Seules les conditions aux limites pour les potentiels sont \`a pr\'eciser.

{\bf Les conditions aux limites sur le potentiel scalaire} sont \`a pr\'eciser
au cas par cas selon la configuration des \'electrodes. Ainsi, on dispose
classiquement de conditions de Neumann homog\`enes ou de Dirichlet (potentiel
impos\'e). On peut \'egalement avoir besoin d'imposer des conditions
d'antisym\'etrie (en utilisant des conditions de Dirichlet homog\`enes par exemple).
L'utilisateur peut \'egalement souhaiter qu'un recalage automatique du potentiel
soit effectu\'e pour atteindre une valeur pr\'ed\'etermin\'ee de la puissance
dissip\'ee par effet Joule.

Lorsque le recalage automatique est demand\'e (\var{IELCOR}=1), l'utilisateur doit fixer la
valeur cible de la puissance dissip\'ee dans le domaine, \var{PUISIM}, ($V.A$).
Il doit en outre, sur les fronti\`eres o\`u il
souhaite que le potentiel (r\'eel ou complexe) s'adapte automatiquement, fournir en
condition \`a la limite une valeur initiale du potentiel et la multiplier par
la variable \var{COEJOU} qui sera automatiquement recal\'ee au cours du calcul
(\var{COEJOU} vaut 1 au premier pas de temps).
Le recalage est effectu\'e en fin de pas temps et permet de disposer, pour le pas
de temps suivant, d'une valeur recal\'ee de l'effet
Joule.
\begin{itemize}
\item Pour effectuer le recalage, \CS d\'etermine l'int\'egrale de l'effet Joule
estim\'e sur le domaine (en $W$) et en compare la
valeur \`a la puissance cible. Un coefficient multiplicatif de recalage \var{COEPOT} en
est d\'eduit (pour \'eviter des variations trop brusques,
on s'assure qu'il reste born\'e entre 0,75 et 1,5).
\item On multiplie alors par
\var{COEPOT} le facteur multiplicatif \var{COEJOU} utilis\'e pour les conditions aux
limites. La puissance dissip\'ee par effet
Joule est multipli\'ee par
le carr\'e de \var{COEPOT}. Pour assurer la coh\'erence du post-traitement des
variables, le potentiel est \'egalement
multipli\'e par \var{COEPOT}.
\item Le champ \'electrique n'\'etant pas explicitement
stock\'e, on ne le recale pas.
\end{itemize}

\bigskip
On notera que la variable \var{DPOT} est \'egalement recal\'ee et qu'elle
peut donc \^etre utilis\'ee si besoin pour imposer les conditions aux limites.

%%%%%%%%%%%%%%%%%%%%%%%%%%%%%%%%%%
%%%%%%%%%%%%%%%%%%%%%%%%%%%%%%%%%%
\section*{Mise en \oe uvre}
%%%%%%%%%%%%%%%%%%%%%%%%%%%%%%%%%%
%%%%%%%%%%%%%%%%%%%%%%%%%%%%%%%%%%


\subsection*{Introduction}

Le module \'electrique est une ``physique particuli\`ere'' activ\'ee lorsque les
mots-cl\'es \var{IPPMOD(IELARC)} (arc \'electrique) ou \var{IPPMOD(IELJOU)}
(Joule) sont strictement positifs. Les d\'eveloppements concernant la conduction ionique
(mot-cl\'e \var{IPPMOD(IELION)}) ont \'et\'e pr\'evus dans le code mais restent \`a
r\'ealiser. Pour l'arc \'electrique, dans la version actuelle
de \CS, seule est op\'erationnelle l'option \var{IPPMOD(IELARC)}=2~: la version 2D axisym\'etrique qui permettrait de
        s'affranchir du potentiel vecteur (option \var{IPPMOD(IELARC)}=1) n'est pas
        activable.
Pour l'effet Joule, lorsqu'il n'est pas utile d'introduire un potentiel scalaire
complexe
(en courant continu ou alternatif monophas\'e), on utilise
\var{IPPMOD(IELJOU)}=1. Lorsqu'un potentiel scalaire complexe est indispensable (courant
alternatif triphas\'e, par exemple), on utilise
\var{IPPMOD(IELJOU)}=2.

Dans ce qui suit, on pr\'ecise les inconnues et les propri\'et\'es
principales utilis\'ees dans le module.
On fournit \'egalement un arbre d'appel simplifi\'e des sous-programmes du module
(initialisation avec \fort{initi1} puis \fort{inivar} et boucle en temps avec \fort{tridim}).
Les sous-programmes marqu\'es d'un ast\'erisque sont d\'etaill\'es ensuite.

\newpage

\subsection*{Inconnues et propri\'et\'es}

Les d\'eveloppements ont \'et\'e r\'ealis\'es pour une unique phase (\var{NPHAS}=1).

Les \var{NSCAPP} inconnues scalaires associ\'ees \`a la physique
particuli\`ere sont d\'efinies dans \fort{cs\_elec\_add\_variable\_fields} dans l'ordre
suivant (en particulier afin de limiter le stockage en m\'emoire lors de
la r\'esolution s\'equentielle des
scalaires par \fort{scalai})~:
\begin{itemize}
\item l'enthalpie \var{RTP(*,ISCA(IHM))},
\item un potentiel scalaire r\'eel \var{RTP(*,ISCA(IPOTR))},
\item un potentiel scalaire imaginaire \var{RTP(*,ISCA(IPOTI))} {\it ssi}
          \var{IPPMOD(IELJOU)}=2 (\'etudes Joule en courant alternatif non monophas\'e),
\item les trois composantes d'un potentiel vecteur r\'eel
          \var{RTP(*,ISCA(IPOTVA(i)))} (avec \var{i} variant de 1 \`a 3) {\it ssi}
        \var{IPPMOD(IELARC)}=2 (arc \'electrique),
\item \var{NGAZG}-1 fractions massiques \var{RTP(*,ISCA(IYCOEL(j)))}
        (avec \var{j} variant de 1 \`a \var{NGAZG}-1) pour un fluide \`a \var{NGAZG}
        constituants (avec \var{NGAZG} strictement sup\'erieur \`a 1).
        En arc \'electrique, la composition est fournie dans le fichier de donn\'ees
        \fort{dp\_ELE}. La fraction massique du
        dernier constituant n'est pas stock\'ee en m\'emoire. Elle est
        d\'etermin\'ee chaque fois que n\'ecessaire en calculant le compl\'ement \`a l'unit\'e
        des autres fractions massiques (et, en particulier, lorsque \fort{cs\_elec\_convert\_h\_t} est
        utilis\'e pour le calcul des propri\'et\'es physiques).
\end{itemize}

\bigskip
Outre les propri\'et\'es associ\'ees en standard aux variables scalaires
identifi\'ees ci-dessus, le
tableau \var{PROPCE} contient \'egalement~:
 \begin{itemize}
\item la temp\'erature, \var{PROPCE(*,IPPROC(ITEMP))}. En th\'eorie, on
        pourrait \'eviter de stocker cette variable, mais l'utilisateur est
        presque toujours int\'eress\'e par sa valeur en post-traitement et
        les propri\'et\'es physiques sont souvent donn\'ees par des lois qui en
        d\'ependent explicitement.
        Son unit\'e (Kelvin ou Celsius) d\'epend des tables
        enthalpie-temp\'erature fournies par l'utilisateur.
\item la puissance \'electromagn\'etique dissip\'ee par effet Joule,
        \var{PROPCE(*,IPPROC(IEFJOU))} (terme source positif pour l'enthalpie),
\item les trois composantes des forces de Laplace,
        \var{PROPCE(*,IPPROC(ILAPLA(i)))} (avec \var{i} variant de 1 \`a 3)
        en arc \'electrique (\var{IPPMOD(IELARC)}=2).
\end{itemize}

\bigskip
La conductivit\'e \'electrique est {\it a priori} variable et
conserv\'ee dans le tableau de propri\'et\'es aux cellules
\var{PROPCE(*,IPPROC(IVISLS(IPOTR)))}. Elle intervient dans l'\'equation de
Poisson portant sur le potentiel scalaire. Lorsque le potentiel scalaire a une
partie imaginaire, la conductivit\'e n'est pas dupliqu\'ee~:
les entiers \var{IPPROC(IVISLS(IPOTI))} et \var{IPPROC(IVISLS(IPOTR))} pointent sur la
m\^eme case du tableau \var{PROPCE}. La conductivit\'e associ\'ee au potentiel
vecteur est uniforme et de valeur unit\'e (\var{VISLS0(IPOTVA(i))}=1.D0
avec \var{i} variant de 1 \`a 3).

Le champ \'electrique, la densit\'e de courant et le champ magn\'etique ne sont
stock\'es que de mani\`ere temporaire (voir \fort{cs\_compute\_electric\_field}).


\newpage

\subsection*{Arbre d'appel simplifi\'e}

\begin{table}[htp]
\begin{center}
\begin{tabular}{llllp{10cm}}
\fort{usini1}         &                 &                &
        & Initialisation des mots-cl\'es utilisateur g\'en\'eraux et positionnement des variables\\
                &\fort{usppmo}         &                &
        & D\'efinition du module ``physique particuli\`ere'' employ\'e\\
                &\fort{varpos}         &                &
        & Positionnement des variables \\
                &                 & \fort{pplecd} &
        & Branchement des physiques particuli\`eres pour la lecture de fichier de donn\'ees \\
                &                 &                 & \fort{cs\_electrical\_properties\_read}*
        & Lecture du fichier de donn\'ees pour les arcs \'electriques  \fort{dp\_ELE} \\
                &                 & \fort{ppvarp} &
        & Branchement des physiques particuli\`eres pour le positionnement des inconnues \\
                &                 &                 & \fort{cs\_elec\_add\_variable\_fields}*
        & Positionnement des inconnues (enthalpie, potentiels, fractions massiques) \\
                &                 & \fort{ppprop} &
        & Branchement des physiques particuli\`eres pour le positionnement des propri\'et\'es\\
                &                 &                 & \fort{cs\_elec\_add\_property\_fields}*
        & Positionnement des propri\'et\'es (temp\'erature, effet Joule, forces de Laplace) \\
%
\fort{ppini1}         &                &                &
        & Branchement des physiques particuli\`eres pour l'initialisation des
mots-cl\'es sp\'ecifiques \\
                &\fort{cs\_electrical\_model\_specific\_initialization}         &                &
        & Initialisation des mots-cl\'es pour le module \'electrique\\
                &\fort{cs\_user\_parameters}         &                &
        & Initialisation des mots-cl\'es utilisateur pour le module \'electrique\\
\end{tabular}
\caption{Sous-programme \fort{initi1}~: initialisation des mots-cl\'es et
positionnement des variables}
\end{center}
\end{table}


\begin{table}[htp]
\begin{center}
\begin{tabular}{llllp{10cm}}
\fort{ppiniv}         &                &                &
        & Branchement des physiques particuli\`eres pour l'initialisation des variables \\
                & \fort{cs\_electrical\_model\_initialize}*&                &
        & Initialisation des variables sp\'ecifiques au module \'electrique \\
                 &                 & \fort{cs\_elec\_convert\_h\_t}*&
        & Transformation temp\'erature-enthalpie et enthalpie-temp\'erature par
                interpolation sur la base du fichier de donn\'ees \fort{dp\_ELE}
                (arc \'electrique uniquement) \\
                 &                 & \fort{cs\_user\_initialization} &
        & Initialisation des variables par l'utilisateur  \\
                 &                 &                 & \fort{cs\_elec\_convert\_h\_t}*
        & Transformation temp\'erature-enthalpie et enthalpie-temp\'erature par
                interpolation sur la base du fichier de donn\'ees \fort{dp\_ELE}
                (arc \'electrique uniquement) \\
\end{tabular}
\caption{Sous-programme \fort{inivar}~: initialisation des variables}
\end{center}
\end{table}


\begin{table}[htp]
\begin{center}
\begin{tabular}{llllp{10cm}}
\fort{phyvar}         &                &                &
        & Calcul des propri\'et\'es physiques variables \\
                & \fort{ppphyv} &                &
        & Branchement des physiques particuli\`eres pour le calcul des
                propri\'et\'es physiques variables \\
                & \fort{cs\_elec\_physical\_properties} &                &
        & Calcul des propri\'et\'es physiques variables pour le module
                \'electrique. En arc \'electrique, les propri\'et\'es sont
                calcul\'ees par interpolation \`a partir des tables fournies
                dans le fichier de donn\'ees \fort{dp\_ELE}\\
                 &                 & \fort{cs\_elec\_convert\_h\_t}*&
        & Transformation temp\'erature-enthalpie et enthalpie-temp\'erature par
                interpolation sur la base du fichier de donn\'ees \fort{dp\_ELE}
                (arc \'electrique uniquement) \\
                 &                 & \fort{cs\_user\_physical\_properties} &
        & Calcul par l'utilisateur des propri\'et\'es physiques variables pour le module
                \'electrique. Pour les \'etudes Joule, en particulier, les propri\'et\'es
                doivent \^etre fournies ici sous forme de loi (des exemples sont
                disponibles)\\
                 &                 &                 & \fort{usthht}
        & Transformation temp\'erature-enthalpie et enthalpie-temp\'erature
                fournie par l'utilisateur (plus sp\'ecifiquement pour les
                \'etudes Joule, pour lesquelles on ne dispose pas d'un fichier
                de donn\'ees \`a partir duquel r\'ealiser des interpolations avec \fort{cs\_elec\_convert\_h\_t}) \\
\end{tabular}
\caption{Sous-programme \fort{tridim}~: partie 1 (propri\'et\'es physiques)}
\end{center}
\end{table}

\begin{table}[htp]
\begin{center}
\begin{tabular}{llllp{10cm}}
\fort{ppclim}         &                  &                &
        & Branchement des physiques particuli\`eres pour les conditions aux limites\\
                & \fort{cs\_user\_boundary\_conditions} &                &
        & Intervention de l'utilisateur pour les conditions aux limites (en lieu
                et place de \fort{usclim}, m\^eme pour les variables qui ne sont
                pas sp\'ecifiques au module \'electrique). Si un recalage
                automatique des potentiels est demand\'e (\var{IELCOR=1}), il
                doit \^etre pris en compte ici par le biais des variables
                \var{DPOT} ou \var{COEJOU} (voir la description des
                conditions aux limites).   \\
\fort{navstv}         &                  &                &
        & R\'esolution des \'equations de Navier-Stokes\\
                & \fort{predvv} &                &
        & Pr\'ediction de la vitesse~: prise en compte des forces de Laplace
                calcul\'ees dans \fort{cs\_compute\_electric\_field} au pas de temps pr\'ec\'edent\\
\fort{``turb''} &                  &                &
        & Turbulence : r\'esolution des \'equations pour les mod\`eles
                n\'ecessitant des \'equations de convection-diffusion\\
\fort{scalai}*         &                  &                &
        & R\'esolution des \'equations portant sur les scalaires associ\'es aux
                physiques particuli\`eres et des scalaires ``utilisateur''  \\
                & \fort{covofi}         &                &
        & R\'esolution successive de l'enthalpie, du potentiel scalaire
                r\'eel et, si \var{IPPMOD(IELJOU)=2}, de la partie imaginaire du
                potentiel scalaire (appels successifs \`a \fort{covofi} qui appelle
                \fort{cs\_elec\_source\_terms} pour le calcul du terme d'effet Joule au second
                membre de l'\'equation de l'enthalpie)\\
                & \fort{cs\_compute\_electric\_field}* &                &
        & Calcul du champ \'electrique, de la densit\'e de courant et de l'effet
                Joule (premier de deux appels au cours du pas de temps courant) \\
                & \fort{cs\_user\_electric\_scaling}* &                &
        & Recalage automatique \'eventuel
                de
                la densit\'e de courant, de l'effet Joule, des potentiels et
                des coefficients \var{DPOT} et \var{COEJOU}.
                Ce recalage, s'il a \'et\'e demand\'e
                par l'utilisateur (\var{IELCOR}=1), est effectu\'e \`a partir
                du deuxi\`eme pas de temps. \\
                & \fort{covofi}         &                &
        & R\'esolution successive, si \var{IPPMOD(IELARC)=2}, des trois
                composantes du potentiel vecteur. On proc\`ede par
                appels successifs \`a \fort{covofi} qui appelle
                \fort{cs\_elec\_source\_terms} pour le calcul du second membre de l'\'equation de
                Poisson portant sur chaque composante du potentiel. \\
                & \fort{covofi}         &                &
        & R\'esolution successive des \var{NGAZG-1} fractions massiques
                caract\'erisant la composition du fluide, s'il est
                multiconstituant.
                On proc\`ede par appels successifs \`a \fort{covofi}. \\
                & \fort{cs\_compute\_electric\_field}* &                &
        & En arc \'electrique, calcul du champ magn\'etique et
                des trois composantes des forces de
                Laplace (deuxi\`eme et dernier appel au cours du pas de temps courant)\\
                & \fort{covofi}         &                &
        & R\'esolution des scalaires ``utilisateur''\\
\end{tabular}
\caption{Sous-programme \fort{tridim}~: partie 2 (conditions aux limites,
Navier-Stokes, turbulence et scalaires)}
\end{center}
\end{table}

\newpage

\subsection*{Pr\'ecisions}

\etape{\fort{cs\_electrical\_properties\_read}}

Ce sous-programme r\'ealise la lecture du fichier de donn\'ees sp\'ecifique
aux arcs \'electriques. On donne ci-dessous, \`a titre d'exemple, l'ent\^ete
explicative et deux lignes de donn\'ees d'un fichier type. Ces valeurs sont interpol\'ees chaque
fois que n\'ecessaire par \fort{cs\_elec\_convert\_h\_t} pour d\'eterminer les propri\'et\'es
physiques du fluide \`a une temp\'erature (une enthalpie) donn\'ee.

{\scriptsize
\begin{verbatim}
# Nb d'especes NGAZG et Nb de points NPO (le fichier contient NGAZG blocs de NPO lignes chacun)
# NGAZG NPO
   1   238
#
#  Proprietes
#      T           H          ROEL       CPEL           SIGEL        VISEL        XLABEL        XKABEL
#  Temperature  Enthalpie   Masse vol. Chaleur       Conductivite  Viscosite   Conductivite   Coefficient
#                           volumique  massique       electrique   dynamique     thermique   d'absorption
#       K         J/kg         kg/m3     J/(kg K)       Ohm/m        kg/(m s)     W/(m K)         -
#
   300.00       14000.       1.6225       520.33      0.13214E-03  0.34224E-04  0.26712E-01   0.0000
   400.00       65800.       1.2169       520.33      0.13214E-03  0.34224E-04  0.26712E-01   0.0000
\end{verbatim}
}


\etape{\fort{cs\_elec\_add\_variable\_fields}}

Ce sous-programme permet de positionner les inconnues de calcul list\'ees
pr\'ec\'edemment. On y pr\'ecise \'egalement que la chaleur massique \`a
pression constante est variable, ainsi que la conductivit\'e de tous les
scalaires associ\'es au module \'electrique, hormis la conductivit\'e de
l'\'eventuel potentiel vecteur (celle-ci est uniforme et de valeur unit\'e).


\etape{\fort{cs\_elec\_add\_property\_fields}}

C'est dans ce sous-programme que sont positionn\'ees les propri\'et\'es stock\'ees
dans le tableau \var{PROPCE}, et en particulier la temp\'erature, l'effet Joule
et les forces de Laplace.

\etape{\fort{cs\_elec\_fields\_initialize}}

Ce sous-programme permet de r\'ealiser les initialisations par d\'efaut
sp\'ecifiques au module.

En particulier, en $k-\varepsilon$, les deux variables
turbulentes sont initialis\'ees \`a $10^{-10}$ (choix historique arbitraire,
mais r\'eput\'e, lors de tests non r\'ef\'erenc\'es, permettre le d\'emarrage de
certains calculs qui \'echouaient avec une initialisation classique).

Les potentiels sont initialis\'es \`a z\'ero, de m\^eme que l'effet Joule. En
arc \'electrique, les forces de Laplace sont initialis\'ees \`a z\'ero.

Le fluide est suppos\'e monoconstituant (seule est pr\'esente la premi\`ere
esp\`ece).

En arc \'electrique, l'enthalpie est initialis\'ee \`a la valeur de l'enthalpie du m\'elange
suppos\'e monoconstituant \`a la temp\'erature \var{T0} donn\'ee
dans \fort{usini1}.  En effet Joule, l'enthalpie est initialis\'ee \`a z\'ero
(mais l'utilisateur peut fournir une valeur diff\'erente dans \fort{cs\_user\_physical\_properties}).

\etape{\fort{cs\_elec\_convert\_h\_t}}

Ce sous-programme permet de r\'ealiser (en arc \'electrique) les interpolations
n\'ecessaires \`a la d\'e\-ter\-mi\-na\-tion des propri\'et\'es physiques du fluide, \`a
partir des tables fournies dans le fichier de donn\'ees \fort{dp\_ELE}.

On notera en particulier que ce sous-programme prend en argument le tableau
\var{YESP(NESP)} qui repr\'esente la fraction massique des \var{NGAZG}
constituants du fluide. Dans le code, on ne r\'esout que la fraction massique
des \var{NGAZG}-1 premiers constituants. Avant chaque appel \`a \fort{cs\_elec\_convert\_h\_t},
la fraction massique du dernier constituant doit \^etre calcul\'ee comme le
compl\'ement \`a l'unit\'e des autres fractions massiques.

\etape{\fort{scalai}, \fort{cs\_compute\_electric\_field}, \fort{cs\_user\_electric\_scaling}}

Le sous-programme \fort{scalai} permet de calculer, dans l'ordre souhait\'e,
les \var{NSCAPP} scalaires ``physique particuli\`ere'' associ\'es au module
\'electrique, puis de calculer les grandeurs interm\'ediaires n\'e\-ces\-sai\-res et
enfin de
r\'ealiser les op\'erations qui permettent d'assurer le recalage automatique
des potentiels, lorsqu'il est requis par l'utilisateur ({\it i.e.} si \var{IELCOR=1}).

Les \var{NSCAPP} scalaires ``physique particuli\`ere'' sont calcul\'es successivement par un
appel \`a \fort{covofi} plac\'e dans une boucle portant sur les \var{NSCAPP}
scalaires. L'algorithme tire profit de l'ordre sp\'ecifique dans lequel ils sont d\'efinis et donc
r\'esolus (dans l'ordre~: enthalpie, potentiel scalaire, potentiel vecteur, fractions massiques).

Pour \'eviter des variations trop brutales en d\'ebut de calcul, le terme source
d'effet Joule n'est pris en compte dans l'\'equation de l'enthalpie qu'\`a
partir du troisi\`eme pas de temps.

Apr\`es la r\'esolution de l'enthalpie et du potentiel scalaire
(r\'eel ou complexe), le sous-programme \fort{cs\_compute\_electric\_field} permet de calculer
les trois composantes du champ \'electrique
(que l'on stocke dans des tableaux de travail), puis la densit\'e de courant
et enfin l'effet Joule, que l'on conserve dans le tableau \var{PROPCE(*,IPPROC(IEFJOU))}
pour le pas temps suivant (apr\`es recalage \'eventuel dans \fort{cs\_user\_electric\_scaling} comme
indiqu\'e ci-apr\`es).
Lorsque \var{IPPMOD(IELJOU)=2},  l'apport de la partie imaginaire est pris en
compte pour le calcul de l'effet Joule. Lorsque \var{IPPMOD(IELARC)=2}
(arc \'electrique), le vecteur densit\'e de courant est
conserv\'e dans \var{PROPCE}, en lieu et place des forces de Laplace
\var{PROPCE(*,IPPROC(ILAPLA(i)))}~: il est utilis\'e pour le calcul du potentiel vecteur dans le
second appel \`a \fort{cs\_compute\_electric\_field},
apr\`es recalage \'eventuel par \fort{cs\_user\_electric\_scaling} (en effet, il n'est plus
n\'ecessaire de conserver les forces de Laplace \`a ce stade puisque
la seule \'equation dans laquelle elles interviennent est l'\'equation de la
quantit\'e de mouvement et qu'elle a d\'ej\`a \'et\'e r\'esolue).

\`A la suite de \fort{cs\_compute\_electric\_field},
le sous-programme \fort{cs\_user\_electric\_scaling} effectue le recalage permettant d'adapter automatiquement
les conditions aux limites portant sur les potentiels, si l'utilisateur l'a
demand\'e ({\it i.e.} si \var{IELCOR=1}).
On se reportera au paragraphe relatif aux conditions aux limites. On pr\'ecise
ici que le coefficient de recalage \var{COEPOT} permet d'adapter l'effet Joule
\var{PROPCE(*,IPPROC(IEFJOU))} et la diff\'erence de potentiel \var{DPOT}
(utile pour les conditions aux limites portant sur le potentiel scalaire au pas
de temps suivant\footnote{{\it A priori}, \var{DPOT} n'est pas n\'ecessaire pour les
cas Joule.}).
Pour les cas d'arc \'electrique, \var{COEPOT} permet \'egalement de
recaler le vecteur densit\'e de courant que l'on vient
de stocker temporairement dans \var{PROPCE(*,IPPROC(ILAPLA(i)))} et qui va
servir imm\'ediatement \`a calculer le potentiel vecteur.
Pour les cas Joule,  on recale en outre le coefficient \var{COEJOU} (utile
pour les conditions aux limites portant sur le potentiel scalaire au pas de
temps suivant).

Pour les cas d'arc \'electrique (\var{IPPMOD(IELARC)=2}),
apr\`es \fort{cs\_compute\_electric\_field} et \fort{cs\_user\_electric\_scaling},
la r\'esolution s\'equentielle des inconnues scalaires se poursuit dans
\fort{scalai} avec le calcul des trois composantes du potentiel vecteur. Le second membre de
l'\'equation de Poisson consid\'er\'ee d\'epend de la densit\'e de courant qui,
dans \fort{cs\_compute\_electric\_field}, a \'et\'e temporairement stock\'ee dans le tableau
\var{PROPCE(*,IPPROC(ILAPLA(i)))} et qui,  dans \fort{cs\_user\_electric\_scaling}, vient d'\^etre
recal\'ee si \var{IELCOR=1}.
Les valeurs du potentiel vecteur obtenues int\`egrent donc  naturellement l'\'eventuel
recalage.

Pour les cas d'arc \'electrique (\var{IPPMOD(IELARC)=2}), un second appel \`a
\fort{cs\_compute\_electric\_field} permet alors de calculer le champ magn\'etique
que l'on stocke dans des tableaux de travail et les forces de Laplace que l'on stocke dans
\var{PROPCE(*,IPPROC(ILAPLA(i)))} pour le pas de temps suivant (la densit\'e de
courant, que l'on avait temporairement conserv\'ee dans ce tableau, ne servait
qu'\`a calculer le second membre de l'\'equation de Poisson portant sur le
potentiel vecteur~: il n'est donc plus n\'ecessaire de la conserver).

La r\'esolution s\'equentielle des inconnues scalaires sp\'ecifiques au module se poursuit dans
\fort{scalai}, avec le calcul des \var{NGAZG}-1 fractions massiques permettant
de d\'efinir la composition du fluide.

Pour terminer, \fort{scalai} permet la r\'esolution des scalaires
``utilisateurs'' (appel \`a \fort{covofi} dans une boucle portant sur les
\var{NSCAUS} scalaires utilisateur).

On peut remarquer pour finir que les termes sources des \'equations de la quantit\'e de
mouvement (forces de Laplace) et de l'enthalpie (effet Joule) sont disponibles
\`a la fin du pas de temps $n$ pour une utilisation au pas de temps $n+1$ (de ce
fait, pour permettre les reprises de calcul, ces termes sources sont stock\'es dans le fichier suite auxiliaire, ainsi que \var{DPOT}
et \var{COEJOU}).

\newpage
%%%%%%%%%%%%%%%%%%%%%%%%%%%%%%%%%%
%%%%%%%%%%%%%%%%%%%%%%%%%%%%%%%%%%
\section*{Points \`a traiter}
%%%%%%%%%%%%%%%%%%%%%%%%%%%%%%%%%%
%%%%%%%%%%%%%%%%%%%%%%%%%%%%%%%%%%

\etape{Mobilit\'e ionique} Le module est \`a d\'evelopper.

\etape{Conditions aux limite en Joule} La prise en compte de conditions aux
limites coupl\'ees entre \'electrodes reste  \`a faire.

\etape{Compressible en arc \'electrique} Les  d\'eveloppements du module
compressible de \CS doivent \^etre rendus compatibles avec le module arc \'electrique.

\part{Module combustion}
%-------------------------------------------------------------------------------

% This file is part of Code_Saturne, a general-purpose CFD tool.
%
% Copyright (C) 1998-2011 EDF S.A.
%
% This program is free software; you can redistribute it and/or modify it under
% the terms of the GNU General Public License as published by the Free Software
% Foundation; either version 2 of the License, or (at your option) any later
% version.
%
% This program is distributed in the hope that it will be useful, but WITHOUT
% ANY WARRANTY; without even the implied warranty of MERCHANTABILITY or FITNESS
% FOR A PARTICULAR PURPOSE.  See the GNU General Public License for more
% details.
%
% You should have received a copy of the GNU General Public License along with
% this program; if not, write to the Free Software Foundation, Inc., 51 Franklin
% Street, Fifth Floor, Boston, MA 02110-1301, USA.

%-------------------------------------------------------------------------------

\programme{ Introduction}
{\huge sub-routines: co**, cp**, fu** ...}

%%%%%%%%%%%%%%%%%%%%%%%%%%%%%%%%%%
%%%%%%%%%%%%%%%%%%%%%%%%%%%%%%%%%%
\section{Use \& call}
%%%%%%%%%%%%%%%%%%%%%%%%%%%%%%%%%%
%%%%%%%%%%%%%%%%%%%%%%%%%%%%%%%%%%

From a CFD point of view combustion is a ({\small sometimes very})
complicated way to determine $\rho$.\\ Models needs few extra scalar
fields with regular transport equations, some of them with a
reactive or interfacial source term.\\ Modelling of combustion is able
to deal with gas phase combustion ({\small diffusion, premix, partial
premix}), and with solid or liquid fuels.\\ Combustion of condensed
fuels involves one-way interfacial flux due to phenomena in the
condensed phase ({\small evaporation or pyrolisis}) and reciprocal
ones ({\small heterogeneous combustion}). Many of the species injected
in the gas phase are afterwards involved in gas phase combsution.\\
That is the reason why many modules are similar for gas, coal and fuel
combustion modelling. Obviously, the thermodynamical description of
gas species is similar in every version as close as possible to the
JANAF rules.\\ All models are developped in both adiabatic and
unadiabatic ({\small permeatic: heat loss, eg. by radiation})
version, beyond the standard, the rule to call models is:

IPPMOD(index model)  =  -1     unused

IPPMOD(index model)  =   0     simplest adiabatic version

IPPMOD(index model)  =   1     simplest permeatic version

Eventually

IPPMOD(index model)  =  2.p    P� adiabatic version

IPPMOD(index model)  =  2.p+1  P� permeatic version 


Every permeatic version involves the transport of enthalpy ({\small one more variable}). 

%=================================
\subsection{Gas combustion modelling}
%=================================

Gas combustion is limited by disponibility ({\small in the same fluid
particle}) of both fuel and oxidant and by kinetic effects ({\small a
lot of chemical reactions involved can be described using an Arrhenius
law with high activation energy}). The mixing of mass ({\small atoms})
incoming with fuel and oxydant is described by a mixture fraction
({\small mass fraction of matter incoming with fuel}), this variable
is not affected by combustion. A progress variable is used to describe
the transformation of the mixture from fuel and oxydant to products
({\small carbon dioxyde and so on}).Combustion of gas is,
traditionnaly, splitted in premix and diffusion regimes.\\

In premix combustion process a first stage of mixing have been
realised ({\small without blast ...}) and the mixture is introduced in
the boiler ({\small or combustor can}). In common industrial
conditions the combustion is mainly limited by the mixing of fresh
gases ({\small inert}) and burnt ones resulting in the inflammation of
the first and their conversion to burnt ones; so an assumption of
chemistry much faster than mixing induces an intermittent regime. The
gas flow is constituted of totally fresh and totally burnt gases
({\small the flamme containing the gases during their transformation
is "extremely" thin}). With this previous assumptions,
Spalding \cite{1} established the "Eddy Break Up" model, which allows
a complete description with only one progress variable ({\small
mixture fraction is homogeneous}).\\

In diffusion flames the fuel and the oxydant are introduced by two
({\small at least}) inlets, in common industrial conditions, their
mixing is the main limitation and the mixture fraction is enough to
qualify a fluid particle, but in turbulent flows a {\em P}robability
{\em D}ensity {\em F}unction of the mixture fraction is needed to
qualify the thermodynamical state of the bulk. So both the mean and
the variance of the mixture fraction are needed ({\small two
variables}).\\

Real world's chemistry is not so fast and, unfortunately, the mixing
can not be so homogeneous as wished. Then industrial combustion occurs
in partial premix regime. Partial premix occurs if mixing is not
finished ({\small at molecular level}) when the mixture is introduced,
or if air or fuel, are staggered, or if a diffusion flame is blown
off. For these situations, and specifically for lean premix gas
turbines Libby \& Williams \cite{2} developped a model allowing a
description of both mixing and chemical limitations. A collaboration
between the LCD Poitiers \cite{3} and EDF R\&D allows a simpler
version of their algorithm. Not only the mean and the variance of both
mixture fraction and progress variable are needed but also their
covariance ({\small five variables}).


%=================================
\subsection{Coal combustion modelling}
%=================================

Coal combustion is the main way to produce electricity in the world.
Coal is a natural product with a very complex composition. During the
industrial process of milling the raw coal is broken in tiny particles
of different sizes. After its introduction in the boiler, coal
particles undergoes drying, devolatilisation ({\small heating of coal
turn it in a mixture of char and gases}), heterogenous combustion
({\small of char in carbon monoxide}) leaving ash particles.\\ Each of
these phenomena are taken into account for some class of particles: a
class is caracterised by a coal ({\small it is useful to burn mixture
of coals with differents ranks or mixture of coal with biomasse ...})
and an initial diameter. For each class, \CS computes the number and
the mass of particles by unit mass of mixture.\\ The main assumption
is to solve only one velocity ({\small and pressure}) field: it means
the discrepancy of velocity between coal particles and gases is
assumed negligible.\\ Due to the radiation and heterogeneous
combustion, temperature can be different for gas and different size
particles: so the specific enthalpy of each particle class is
solved.\\ The description of coal pyrolysis proposed by Kobayashi \&
Ubhayakar \cite{4} is used, leading to two source terms for light and
heavy volatile matters ({\small the moderate temperature reaction
produces gases with low molecular mass, the high temperature reaction
produces heavier gases and less char}) represented by two passive
scalars: mixture fractions.  The description of the heterogeneous
reaction ({\small which produce carbon monoxide}) produces a source
term for the carbon: a mixture fraction who can't be greater than the
results of stoechiometric oxidation of char by air ({\small carbon
can't be free in gas phase, it is always linked in an oxide}).\\ The
retained model for the gas phase combustion is the assumption of
diffusion flammelets surrounding each particle, so the previous
gaseous fuels are mixed in a local mean fuel and the mixing with air
is represented by a pdf between air and the mean local fuel
constructed with the variance of a passive scalar linked with air
({\small interfacial mass flux produce a source term for this
scalar}).
 



%=================================
\subsection{Heavy Fuel Oil combustion modelling}
%=================================

Heavy fuel oil combustion have been hugely used to produced electrical
energy. Environmental regulation turning it more difficult and less
acceptable, a focus is needed on pollutant emission mainly sulphur
oxide and particles ({\small condensation of sulphuric acid can
aggregate soot}).\\ The description of fuel evaporation is done with
respect to its heaviness: after a minimum temperature is reached, the
gain of enthalpy is splitted between heating and evaporation. This way
the evaporation takes place on a range of temperature ({\small which
can be large}). The "total" evaporation is common for light ({\small
domestic}) oil but impossible for heavy ones: at high temperature,
during the last evaporation, a crakink reaction appears: so a
particle similar to char leaves. The heterogeneous oxydation of this
char particle is very similar to coal char ones.\\ Fuel injection is
described ({\small 2006 version}) with only one class of particles
({\small i.e. initial diameter}), the number, mass and specific
enthalpy of particles are calculated eveywhere. So three variables are
used to describe the condensed phase. In the same way as for coal,
only one velocity field is computed.\\ The model for gas combustion is
very similar to coal one but a special attention is paid to sulphur
({\small assumed to leave the particle as H2S during evaporation and
to be converted to SO2 during gas combustion}).


%==================================
%==================================
\section{Bibliography}
%==================================
%==================================
\begin{thebibliography}{4}


\bibitem{1}
{\sc Spalding, D.B., {\em et al.}},\\
{\em Mixing and chemical reaction in steady confined turbulent turbulent flames},\\
13th Int.Symp. on Combustion , pp. 649-657, (1971).

\bibitem{2}
{\sc Libby, P.A. and Williams, F.A.},\\
{\em A presumed PDF analysis of lean partially premixed turbulent combustion},\\
Combust. Sci. Technol., 161, pp. 351-390, (2000)

\bibitem{3}
{\sc Ribert, G.; Champion, M. and Plion, P.},\\
{\em Modeling turbulent reactive flows with variable equivalence ratio: application to the calculation of a reactive shear layer},\\
Combust. Sci. Technol., 176, pp. 907-923, (2004)

\bibitem{4}
{\sc Kobayashi, H. {\em et al.}},\\
16th Int.Symp. on Combustion , pp. 425-441, (1976).





\end{thebibliography}
\newpage
%%%%%%%%%%%%%%%%%%%%%%%%%%%%%%%%%%
%%%%%%%%%%%%%%%%%%%%%%%%%%%%%%%%%%
\section{Discr\'etisation}
%%%%%%%%%%%%%%%%%%%%%%%%%%%%%%%%%%
%%%%%%%%%%%%%%%%%%%%%%%%%%%%%%%%%%

On se reportera aux sections relatives aux sous-programmes 
\fort{cfmsvl} (masse volumique), \fort{cfqdmv} 
(quantit\'e de mouvement) et \fort{cfener} (\'energie). 
La documentation du sous-programme 
\fort{cfxtcl} fournit des \'el\'ements relatifs aux 
conditions
aux limites. 

\programme{ Gas combustion}
{\huge subroutines : co** and d3p*, ebu*, lwc*, pdf*}

Flames with gaseous fuels can be categorized in premix, diffusion or
partial-premix.


%%%%%%%%%%%%%%%%%%%%%%%%%%%%%%%%%%
%%%%%%%%%%%%%%%%%%%%%%%%%%%%%%%%%%
\section{Premix : Eddy Break Up}
%%%%%%%%%%%%%%%%%%%%%%%%%%%%%%%%%%
%%%%%%%%%%%%%%%%%%%%%%%%%%%%%%%%%%

The original Spalding model \cite{1} was written for a situation where
the whole boiler is full with the same mixture ({\small only one
inlet}) ; the key-word of which is {\em"If it mixes, it burns"}.\\

If the chemistry is fast, vs. mixing, fluid particles are made of
fresh gases or of burned ones. This situation is described by a
progress variable ({\small valued 0 in fresh gases and 1 in burnt
ones}) with a source term : the reaction one. The mixture of totally
fresh or totally burnt gases, called intermittency, leads to a maximal
variance of the progress variable determined by the mean value of the
progress variable.\\
\begin{equation}
C"^{2} max = (Cmoy -Cmin).(Cmax-Cmoy) = Cmoy . (1-Cmoy)
\end{equation}


The mixing of fresh and burnt gases is the dissipation of this
variance and it induces the conversion of fresh gases in burnt
ones. So the source term for the ({\small mean}) progress variable is
the dissipation of its ({\small algebraic}) variance.\\

In \CS the progress variable chosen is the mass fraction of fresh
gases, so :\\
\begin{equation}
S(Ygf) = - Cebu . \rho . \frac{\epsilon}{k} . Ygf . (1-Ygf)
\end{equation}

Where Cebu is a constant, supposedly "universal", fitted around 1.6
({\small only advanced users can adjust this value}).\\


Some improvements have been proposed, and largely used, for situations
with mixture fraction gradient ({\small staggering of reactant(s)})
but are not theorically funded. The simplest extension is available
({\small options 2 and 3 }) in \CS with one extra equation solved for
f the mean of mixture fraction : the corresponding hypothesis is "no
variance for mixture fraction" ... a little bit surprising in an EBU
context ({\small maximal variance for progress variable}). The choice
of the fresh gas mass fraction appears now quite pertinent : the
computation of species mass fraction can be done, with respect to the
mean mixture fraction, both in fresh ({\small the mass fraction of
which is Ygf}) and burnt gases ({\small the mass fraction of which is
(1-Ygf)}).\\
\begin{equation}
Y_{Fuel} = Y_{gf}.f + (1-Y_{gf}) . Max(0 ; \frac{f-fs}{1-fs})
\end{equation}
\begin{equation}
Y_{Ox} = Y_{gf}.(1-f) + (1-Y_{gf}) . Max(0 ; \frac{fs-f}{fs})
\end{equation}
\begin{equation}
Y_{Prod} = (1-Y_{gf}) . Min( \frac{f}{fs} ; \frac{1-f}{1-fs} )
\end{equation}
Where fs is the stoechiometric mixture fraction.\\

In adiabatic conditions the specific enthalpy of gases ({\small in
every combustion model the considered enthalpy contains formation one
and heat content, but no terms for velocity or pressure}) is directly
related to the mixture fraction ({\small as long as the inlet
temperature for air and fuel is known}). When heat losses, like
radiation, are taken in account, an equation has to be solved for the
mean enthalpy ({\small such an equation is needed so when some entries
have different temperatures -partial preheat- enthalpy is then used as
an extra passive scalar}). In industrial processes, the aim is often
to transfer the heat from burnt gases to wall ; even for heat loss the
wall temperature is near to fresh gases temperature and the main heat
flux takes place between burnt gases and wall. So in \CS the specific
enthalpy of the fresh gases is supposed to be related to mixture
fraction and the specific enthalpy of burnt gases is locally computed
to reach the mean specific enthalpy. This way every heat loss removed
from the mean enthalpy is "paid" by the hottest gases.\\

\begin{equation}
hmean = Ygf.hgf(f) + (1-Ygf).hbg => hbg
= \frac{hmean-Ygf.hgf(f)}{1-Ygf}
\end{equation}

Where f ({\small in hgf(f)}) is the local mean of the mixture fraction
or a constant value ({\small in regular EBU model}).

%%%%%%%%%%%%%%%%%%%%%%%%%%%%%%%%%%
%%%%%%%%%%%%%%%%%%%%%%%%%%%%%%%%%%
\section{Diffusion : PDF with 3 points chemistry}
%%%%%%%%%%%%%%%%%%%%%%%%%%%%%%%%%%
%%%%%%%%%%%%%%%%%%%%%%%%%%%%%%%%%%

In diffusion model, the combustion is supposed limited only by mixing
({\small between air and fuel}), so the reaction is assumed to reach
instantaneously its equilibrium state and the temperature and
concentrations can be computed for every value of the mixture
fraction. In \CS the implemented version uses an extra hypothesis :
the reaction is complete ; so if the mixture is stoechiometric, the
burnt gases contains only final products ({\small none unburnt like
CO...}). As a consequence, every concentration is a piecewise linear
function of the mixture fraction (subroutines : \fort{D3PPHY, D3PINT,
CPCYM,FUCYM}) .\\
\begin{equation}
0<f<fs => Yi(f) = Yair + \frac{f}{fs} . (Ys-Yair)
\end{equation}
\begin{equation}
fs<f<1 => Yi(f) = Ys + \frac{f-fs}{1-fs} . (Yfuel-Ys)
\end{equation}\\
\begin{figure}[h]
%\centerline{\includegraphics[height=3cm]{../Comb/Cogz/Images/Yf.pdf}}
\caption{mass fraction of global species are piecewise linear with mixture fraction}
\end{figure}
Where fs is the stoechiometric mixture fraction, Yair and Yfuel are
concentrations in ({\small supposed unable to react : inert}) initial
reactant, Ys concentrations in products of the complete reaction of a
stoechiometric mixture ({\small in such products, the chemical
reaction is no more possible : inert}).\\

The diffusion model uses two equations for the mixture fraction and
its variance both of them having no reaction term. The mean and the
variance of the mixture fraction are used to presume the Probability
Density Function for the mixture fraction. In \CS the shape proposed
by Borghi \cite{4} with a rectangle and Dirac's peak is used
(subroutines \fort{COPDF, CPPDF, FUPDF}).\\

\begin{figure}[h]
%\centerline{\includegraphics[height=4cm]{../Comb/Cogz/Images/Pf.pdf}}
%\centerline{\includegraphics[height=2.5cm]{../Comb/Cogz/Images/Pf2.pdf}}
\caption{Examples of presumed PDF : cheapest form}
\end{figure}
The determination of mean concentration is done by integration the
local concentration weighted by the probability density function. As a
matter of fact, integrate the product of a piecewise linear function
by a constant ({\small height of the rectangle}) is a very simple
exercise : analytic solution are available ({\small the original
Borghi formulation \cite{3} wich uses $\beta$ function was much more
computationaly expensive}).

In adiabatic condition, the specific enthalpy of the mixture is a
linear function of the mixture fraction ({\small the enthalpy is not
modified by the reaction}). As for premix combustion, an assumption is
done "the hotter the gases, the worse the heat losses", so the
enthalpy of pure oxidiser and fuel are supposed not to be modified in
permeatic conditions, the enthalpy of products hs ({\small at the
stoechimodtric mixture fraction}) is an unknown or auxiliary
variable. The enthalpy of the mixture is supposed linear piecewise
with f ({\small like concentrations but with an unkwnon at fs}) and
the resulting mean enthalpy ({\small weighted by PDF}is linear in
hs. Fitting with the equation for the mean enthalpy ({\small wich
takes in account radiation and other heat fluxes}), hs is determined
and, consequently the temperature at fs and the mean temperature can
be computed.

\begin{figure}[h]
%\centerline{\includegraphics[height=4cm]{../Comb/Cogz/Images/hf.pdf}}
\caption{Enthalpy of products is determined to account for heat losses }
\end{figure}

%%%%%%%%%%%%%%%%%%%%%%%%%%%%%%%%%%
%%%%%%%%%%%%%%%%%%%%%%%%%%%%%%%%%%
\section{Partial premix : Libby Williams models}
%%%%%%%%%%%%%%%%%%%%%%%%%%%%%%%%%%
%%%%%%%%%%%%%%%%%%%%%%%%%%%%%%%%%%


\CS has been the test-bed for some versions of Libby-Williams model \cite{2}, like inplemented then incremented by Ribert \cite{5} and Robin \cite{6}.

The Libby \& Williams model have been developped to address the
description of the combustor can of gas turbine in regime allowing a
reduction of NOx production using ({\small sometimes very}) lean
premix : by this way, the combustion occurs at moderate temperatures
avoiding the hot spots which are favourable to thermal NOx
production. As a consequence of the moderate temperatures, the
chemistry is no more so fast and the stability is questionnable. To
ensure it a diffusion flame called pilot takes place in the center of
the combustor. So, if the main flow is premixed, both pure fuel and
pure oxidiser are introduced in the combsutor leading to continuous
variation of the equivalence ratio ({\small always the mixture
fraction}). This situation is clearly out of the range of both EBU and
PDF models, moreover the limitation by the chemistry is needed
({\small for stability studies}).\\

Originally, Libby \& Williams proposed a presumed PDF made of two
Dirac's peak, Ribert showed that this PDF can be determined with only
the mean and the variance of the mixture fraction and a reactive
variable ({\small by now, the mass fraction of fuel is used}). Then
some undeterminations seem awkward and Robin \& al. propose a four
Dirac's peak PDF whose parameters are determined with the same
variables and the solved ({\small transported}) covariance ({\small of
the reactive variable and the mixture fraction}) as an extra solved
field. With the condition corresponding to every Dirac's peak a global
chemistry description is used ({\small source term for every variables
is a weighting of the reaction fluxes}).\\ With two peaks
distribution, the two-variable PDF is restricted to one line, crossing
the mean state and the slope of which is the ratio of variances
({\small chose of the sign is user free, ... but relevant : expertise
is needed}). The correlation between variables is unity.\\ On this
line the distribution is supposed to obey a modified Curl
model \cite{7}.\\
\begin{figure}[h]
%\centerline{\includegraphics[height=5cm]{../Comb/Cogz/Images/LW2.pdf}}
\caption{PDF recommended by Libby \& Williams : still undetermined}
\end{figure}
With three or four peaks distribution, the whole concentration space
is available and the determination of the covariance allows evolution
of the correlation ({\small with f and Yf, it have been shown that the
correlation is positive in mixing layer and can become negative across
flame : the particle nearer of stoechiometry being able to burn -then
destroy Yf- faster than poor ones}). \\
\begin{figure}[h]
%\centerline{\includegraphics[height=5cm]{../Comb/Cogz/Images/LW4.pdf}}
\caption{PDF form in LWP approach : succesive modified Curl distributions}
\end{figure}
In adiabatic conditions, the temperature can be computed for every
pair (f,Yfuel), allowing the determination of the kinetic constant.\\
As previously, with heat losses, it is assumed that the hottest gases
have lost the more, the enthalpy of the products at stoechiometric
point (fs,0) is an auxiliary variable, the enthaply being considered
as a piecewise bilinear function. Fitting mean transported enthalpy
and integrated ones, allows to determine the enthalpy of
stoechiometric products, then the enthalpy and temperature in the
peaks conditions and, ({\em in fine}) the reactions fluxes.




%==================================
%==================================
\section{Bibliography}
%==================================
%==================================
\begin{thebibliography}{7}

\bibitem{1}
{\sc Spalding, D.B., {\em et al.}},\\
{\em Mixing and chemical reaction in steady confined turbulent turbulent flames},\\
13th Int.Symp. on Combustion , pp. 649-657, (1971).

\bibitem{2}
{\sc Borghi, R. {\em et } Moreau, P.},\\
{\em Turbulent combustion in a premixed flow},\\
Acta Astronautica, 4, pp. 321-341, (1977)

\bibitem{3}
{\sc Borghi, R. {\em et } Dutoya, D.},\\
{\em On the scale of the fluctuation in turbulent combustion},\\
17th Int.Symp. on Combustion, (1978).

\bibitem{4}
{\sc Libby, P.A. {\em et } Willimas, F.A.},\\
{\em A presumed PDF analysis of lean partially premixed turbulent combustion},\\
Comb. Sci. Technol., 161 , pp. 351-390, (2000).

\bibitem{5}
{\sc Ribert, G. {\em et al.}},\\
{\em Modeling turbulent reactive flows with variable equivalence ratio : application to the calculation of a reactive shear layer},\\
Comb. Sci. Technol., 176 , pp. 907-923, (2004).

\bibitem{6}
{\sc Robin, V. {\em et al.}},\\
{\em Relevance of approximated PDF shapes for turbulent combustion modeling with variable equivalence ratio},\\
ICDERS, (2005) to be published.

\bibitem{7}
{\sc Curl, R.L.},\\
{\em Dispersed phase mixing : theory and effects in simple reactors},\\
AIChE J., 9 , pp. 175-181 (1963).




\end{thebibliography}
\newpage
%%%%%%%%%%%%%%%%%%%%%%%%%%%%%%%%%%
%%%%%%%%%%%%%%%%%%%%%%%%%%%%%%%%%%
\section{Discretisation}
%%%%%%%%%%%%%%%%%%%%%%%%%%%%%%%%%%
%%%%%%%%%%%%%%%%%%%%%%%%%%%%%%%%%%

Some applied mathematics are involved in pdf parameters determination
({\small rectangle and modified Curl}) and for integration ({\small
species mass fraction, temperature, density and so on}).\\

%%%%%%%%%%%%%%%%%%%%%%%%%%%%%%%%%%
%%%%%%%%%%%%%%%%%%%%%%%%%%%%%%%%%%
\subsection{Rectangle and Dirac's peaks probability density function}
%%%%%%%%%%%%%%%%%%%%%%%%%%%%%%%%%%
%%%%%%%%%%%%%%%%%%%%%%%%%%%%%%%%%%

This type of pdf is used in diffusion flames both for gas combustion
or dispersed fuel ones ({\small coal and heavy fuel oil}). In gas
mixture, the pdf is build for an equivalence ratio for fuel ({\small
inert scalar variable}) ranging on [0, 1]. For disperse fuel, due to
vaporisation, or pyrolysis, and heterogeneous combustion two or three
gaseous fuels are taken in account, each of them having its own inert
scalar, so the PDF is build for an inert scalar incoming with air and
ranging from a minimum value and one ; the minimum value for air
tracker is due to heterogeneous combustion : some air is needed to
allow heterogeneous combsution and carbon releasing ({\small as carbon
monoxide}) ; after some heteregenous reaction took place, air is
lacking for gas combustion ({\small at the bottom stone mile of the
range allowed for the inert scalar associated with air, oxygen
concentration is zero valued}). The algorithm for pdf parameters
determination, can be wrotten in a general form on every variable's
range.\\

If the allowed range for the variable is [$f_{min} ; f_{max}$],
knowing the mean and variance of the variable allow to determine first
the shape ({\small alone rectangle, rectangle and one Dirac's peak at
one boundary, two Dirac's peak at boundaries and rectangle}) and then
the three pertinent parameters ({\small three conditions given by
momenta $m_{0}=1, m_{1}=mean, m_{2}=mean^{2}+variance$}).\\

- for a lonesome rectangle Dirac's peak intensity is null, the three
  parameters are : the begin and end values of the rectangle and its
  heigth\\

- for a rectangle with a Dirac's peak at one boundary ({\small which
  is determined during the choose of shape}), one of the rectangle
  hedge is fixed at this boundary, so the three parameters are : the
  other rectangle hedge, height of rectangle, intensity of the Dirac's
  peak\\

- for a two Dirac's peak distribution, both rectangle hedges are at
   the boudaries, so the parameters are the rectangle height and the
   Dirac's peak intensity.\\

Quite complicated tests ({\small computationnaly expensive}) can be
done to determine only the relevant parameters, and therefore spare
computations, or general computaions of parameters can be done and
afterwards tested ({\small if rectangle too large then attempt with
one Dirac's peak. If too large again then two peaks would be
convenient}).
 
 

%%%%%%%%%%%%%%%%%%%%%%%%%%%%%%%%%%
%%%%%%%%%%%%%%%%%%%%%%%%%%%%%%%%%%
\subsection{Composition in total turbulent reaction assumption}
%%%%%%%%%%%%%%%%%%%%%%%%%%%%%%%%%%
%%%%%%%%%%%%%%%%%%%%%%%%%%%%%%%%%%

With the above Probability Density Function and assumption of
concentrations piecewise linear vs. the variable, it is quite easy to
integrate and found the mean concentrations. For gas diffusion flame
this is done by \fort{D3PPHY, D3PINT}, it seems more relevant to
explain the algorithm in a more complicated case : for heavy fuel
oil \fort{FYUCYM} or for coal \fort{CPCYM} three reactions are
considered. \\

- Coal is assumed to undergoes two competitve pyrolysis reactions, the
  first releasing organic compound summarized as $CH_{x1}$, the second
  releasing $CH_{x2}$ ({\small with $x1 > x2$}), both of them
  releasing $CO$. Then the heterogeneous combustion of char release
  $CO$. So three reactions are supposed to succed ({\small both in
  time and in priority to access to oxygen}). First of all partial
  dehydrogenation ({\small lowering saturation}) of $CH_{x1}$ to
  produce water vapor and $CH_{x2}$. Then the $CH_{x2}$ ({\small
  produced by pyrolisis or by $CH_{x1}$ partial oxydation}) is
  converted to water vapor and carbon monoxide. Last, carbon monoxide
  $CO$ is converted to carbon dioxide $CO_{2}$.\\

- Heavy fuel oil is supposed to undergoes a progressive evaporation,
  releasing a fuel vapor $CH_{x}$, $CO$, $H_{2}S$ and a char
  particle. Then heterogenous oxidation of char releases $CO$. Three
  reactions are supposed to succeed. First, the conversion of $CH_{x}$
  to water vapor and $CO$. Then the oxidation of $H_{2}S$ to water
  vapor and $SO_{2}$. Last, carbon monoxide is fully oxidised.\\

Both for coal and heavy fuel oil, the assumption of a diffusion
flamelet surrounding particles is done. All of the reducing gases are
supposed mixed ({\small to constitue a local mean fuel}) and the
diffusion flammelet take place between this mixture and pure oxidiser
({\small but for solid particles introduced wet, when a first drying
process releases water vapor which is mixed with air}) and the
composition is described with respect to the inert scalar introduced
with air : \fort{f4}.\\

For the description of compositions ({\small as piecewise linear
functions}) the composition is computed in some remarkable points :
the local mean fuel \fort{CL}, the oxidiser \fort{F4} and the three
mixture ratio corresponding to the stoechiometry of the three
successive reactions.\\ Before reaction between gases, only exist
species coming from entries or interfacial source term :\\
\begin{figure}[h]
%\centerline{\includegraphics[height=6cm]{../Comb/Cogz/Images/YF0f.pdf}}
\caption{Heavy Fuel Oil before any gas combustion} 
\end{figure}
{\centerline {\em Caution : axis for f4 is reverted, so fuel is on the
right side like in a scheme with a mixture fraction direct axis}}}\\

The oxygen and the hydrocarbon vapor have concentrations linear in f4
on [CL, 1], as long as the stoechiometry of the reaction is known, a
simple equation allows to determine f4s1 the stoechiometric point for
the first reaction ({\small where both oxygen and hydrocarbon
vanish}).  The first reaction is the conversion of some hydrocarbon
vapor to carbon monixide and water vapor ({\small not plotted in an
optimistic attempt to lighten the sketch}).\\
\\
\centerline{$CH_{x} + \frac{2+x}{4} O_{2} => CO + \frac{x}{2} H_{2}O $}\\
\\
\begin{figure}[h!]
%\centerline{\includegraphics[height=6cm]{../Comb/Cogz/Images/YF1f.pdf}}
\caption{Heavy Fuel Oil after hydrocarbon conversion}
\end{figure}
\\
Then the rich area can't undergo any reaction ({\small no oxygen
available}) if the PDF(f4) is not zero before F4s1, then some $CH_{x}$
is unburnt.\\

Some $H_{2}S$ can be converted to $SO_{2}$, the carbon monoxide
existing between F4cl and F4s2 is protected from oxidation ({\small
the two first reactions have destroyed free oxygen}). Like previously,
oxygen and hydrogen sulphide have concentrations linear in f4 on
[f4s1, 1] as long as the stoechiometry of the reaction is known, a
simple equation allows to determine f4s2 the stoechiometric point for
the second reaction ({\small where both oxygen and hydrogen sulphide
vanish}).\\
\\
\centerline{$H_{2}S + \frac{3}{2} O_{2} => SO_{2} + H_{2}O $ }\\
\\
\begin{figure}[h!]
%\centerline{\includegraphics[height=6cm]{../Comb/Cogz/Images/YF2f.pdf}}
\caption{Heavy Fuel Oil after H2S oxidation}
\end{figure}
\\
Now oxygen and carbon monoxide have concentrations linear in f4 on
[f4s2, 1], and f4s3 ({\small the point wher the whole oxygne and
carbon monoxide are converted to carbon dioxide}) is easy to
compute.\\
\\
\centerline{$CO + \frac{1}{2} O_{2} => CO_{2} $ }\\
\\
\begin{figure}[h!]
%\centerline{\includegraphics[height=6cm]{../Comb/Cogz/Images/YF3f.pdf}}
\caption{Heavy Fuel Oil after final carbon oxidation}
\end{figure}
\\
To have not any unburnt, the PDF may be zero valued for F4 $<$ F4s3 :
every gas particle may have an F4 greater than F4s3.\\

During pulverised coal combustion, two kinds of volatile matters are
considered and the sketch of concentrations during the three
successive reactions is quite similar.\\

The first reaction is a partial dehydrogenation of the light volatile
$CH_{x1}$ to form the species caracteristic of heavy volatile
$CH_{x2}$ : in f4s1, all of $CH_{x1}$ ({\small issued from the low
temperature pyrolisis reaction}) is converted, and the $CH_{x2}$
({\small issued from the high temperature pyrolisis reaction}) is
incremented.\\
\\
\centerline{$CH_{x1} + \frac{x1-x2}{4} O_{2} => CH_{x2} + \frac{x1-x2}{2} H_{2}O $}\\
\\
The second reaction is the conversion of this unsaturated hydrocarbon
to carbon monoxide and water.\\
\\
\centerline{$CH_{x2} + \frac{2+x2}{4} O_{2} => CO + \frac{x2}{2} H_{2}O $}\\
\\
And the last ({\small not the least from an energetic point of view})
is the same final oxidation of carbon monoxide to carbon dioxide.\\
\\
Comparisons of the PDF rectangle hedges [$f_{deb} , f_{fin}$] and
remarkable composition points [CL, f4s1, f4s2, f4s3, F4] allows a
simple integration : 1) Dirac's peak intensity are used to weight
composition at boundaries, 2) the piece linear part is integrated with
analytical formulae on each band :\\
\begin{enumerate}
\item rich range, here exists species with the higher calorific value : $CH_{x}$ ({\small in fuel case}) or $CH_{x2}$ ({\small in coal case }) :\\  \centerline{[Max($f_{deb}$,CL) ; Min($f_{fin}$,f4s1)]}\\
\item middle-class range $H_{2}S$ or $CH_{x1}$ conversion :\\
 \centerline{[Max($f_{deb}$ , f4s1 ); Min ($f_{fin}$ , f4s2 )]}\\
\item working range, carbon monoxide consumption frees enthalpy :\\ \centerline{ [Max($f_{deb}$ ,f4s2 ); Min ($f_{fin}$ , f4s3 )]}\\
\item poor range, only products and oxidisers :\\ \centerline{ [Max($f_{deb}$ , f4s3 ); Min ($f_{fin}$ , 1)]}\\
\end{enumerate}

For each band (eg. [f4si , f4sj]) concentrations can be written :\\
\centerline{$Ye = Ye(f4si) + \frac{f4-f4si}{f4sj-f4si} . \left( Ye(f4sj)-Ye(f4si) \right) $}\\ 
Integration on the band [b1 , b2] ({\small obviously b1$\geq$f4si \&
b2$\leq$f4sj}) gives the increment :\\
\centerline{$Ye := Ye + h_{rec}.(b2-b1) + \left[ \frac{Ye(f4si).f4sj-Ye(f4sj).f4si}{f4sj-f4si}+\frac{Ye(f4sj)-Ye(f4si)}{f4sj-f4si}.\frac{b1+b2}{2} \right] $}\\
Where $h_{rec}$ is the height of the PDF's rectangle. 

 
 

%%%%%%%%%%%%%%%%%%%%%%%%%%%%%%%%%%
%%%%%%%%%%%%%%%%%%%%%%%%%%%%%%%%%%
\subsection{Composition in partial turbulent reaction assumption for CO}
%%%%%%%%%%%%%%%%%%%%%%%%%%%%%%%%%%
%%%%%%%%%%%%%%%%%%%%%%%%%%%%%%%%%%

If the final oxidation of carbon monoxide, can't be assumed fast in
respect to mixing ; the main limitation in the kinetic one. So the
above process of concentrations determination is stopped after the two
first reactions ; the resulting, high, content in carbon monoxide
({\small and oxygen}) is the molar sum of both carbon monoxide and
carbon dioxyde. With an extra budget equation for carbon dioxide, the
effective concentrations of carbon monoxide ({\small difference
between "turbulent" integration and transported "already" consumed }),
oxygen and carbon dioxide can be computed, then a reactive source term
for oxidation ({\small of carbon monoxide}) and dissociation ({\small
of carbon dioxide}).

\newpage





%-------------------------------------------------------------------------------

% This file is part of Code_Saturne, a general-purpose CFD tool.
%
% Copyright (C) 1998-2011 EDF S.A.
%
% This program is free software; you can redistribute it and/or modify it under
% the terms of the GNU General Public License as published by the Free Software
% Foundation; either version 2 of the License, or (at your option) any later
% version.
%
% This program is distributed in the hope that it will be useful, but WITHOUT
% ANY WARRANTY; without even the implied warranty of MERCHANTABILITY or FITNESS
% FOR A PARTICULAR PURPOSE.  See the GNU General Public License for more
% details.
%
% You should have received a copy of the GNU General Public License along with
% this program; if not, write to the Free Software Foundation, Inc., 51 Franklin
% Street, Fifth Floor, Boston, MA 02110-1301, USA.

%-------------------------------------------------------------------------------

\programme{cpbase}

%%%%%%%%%%%%%%%%%%%%%%%%%%%%%%%%%%
%%%%%%%%%%%%%%%%%%%%%%%%%%%%%%%%%%
\section{Function}
%%%%%%%%%%%%%%%%%%%%%%%%%%%%%%%%%%
%%%%%%%%%%%%%%%%%%%%%%%%%%%%%%%%%%


Pulverised coal combustion is described ({\small excluding grid
burning}) allowing the use of mixture of coals ({\small or of coal and
biomass}) and a description of granulometry ({\small as many classes
of initial diameter as wished}). After a particle enters the furnace,
radiation increases its temperature.
\begin{enumerate}
  \item As particle's temperature increases, evaporation of free water
  ({\small if any}) begins. During evaporation, the vapor pressure
  gradient extract some water from the particle. The interfacial mass
  flux brings some mass, water and enthalpy ({\small computed for
  water vapour}) then latent heath is taken from the particle's
  enthalpy so the heating is slowed ({\small during evaporation, the
  water can't reach the boiling point}).
\item After drying is achevied, the temperature  reachs higher level, allowing pyrolysis to take place. The pyrolysis is described by two competitive reactions : the first one with a moderate activation energy is able to free peripherals atoms group from skeleton leaving to light gases and a big amount of char ; the second one with an higher activation enrgy is able to break links deeper in the skeleton leaving to heavier gases ({\small or tar}) and less char ({\small more porous}. So a complete description needs two set of three parameters ({\small two kinetics ones and a partitioning one}): \\
\\
\centerline{$Coal =(k01, T01)=> Y1. light volatiles + (1-Y1).char$}\\
\centerline{$Coal =(k02, T02)=> Y2. heavy volatiles + (1-Y2).char$}\\
\\ Where Y1, the partitionning ({\small or selectivity}) factor of the "{\em low temperature}" reaction is less than Y2, the "{\em high temperature}" one. A practical rule is to consider that the same hydrogen and oxygen can bring twice more carbon by the second reaction than by the first one.\\


When ultimate analysis are available both for coal and for char, it is
 releavant to check partioning coefficient ({\small $Y_{i}$}) and
 composition of volatiles matters ({\small mainly ratio of Carbon
 monoxide and C$/$H in the hydrocarbon fraction}) : assumptions on
 volatiles composition gives partitionning coefficients ; assumptions
 on $Y_{i}$ determine volatiles equivalent formulae. Pyrolisis
 interfacial mass flux brings energy of volatile gases ({\small
 computed at the particle's temperature}) in wich the formation of
 enthalpy of gaseous species differs from the coal one's, as a result
 the enthalpy for pyrolisis reaction ({\small the most ofen, moderate})
 is taken from particle energy.  \item After pyrolysis, when every
 volatiles are burnt, oxygen is able to reach the surface of the char
 particle. So heterogenous combustion can take place : diffusion of
 oxygen from bulk, heterogeneous reaction ({\small kinetically
 limited}) and back diffusion of carbone monoxide. The heterogeneous
 oxidation interfacial mass flux is the difference of an incoming
 oxygen flux and an outcoming carbon monoxide mass flux, each of them
 at their own temperature. The incoming oxygen has a zero valued
 formation enthalpy ({\small reference state}) and the outcoming
 carbon monoxide has a negative formation enthalpy, as a result the
 enthalpy liberated by the first oxidation of carbon is leaked in the
 particle energy, contributiong to its heating. The heterogenous
 combustion is achieved if all the carbon of the char particle is
 converted, leaving an ash particle. Unburnt carbon can leave the
 boiler as flying particle. The heterogeneous reaction is wrotten
 : \\ \centerline{$Char + \frac{1}{2} 0_{2} =(k0het, T0het)=> CO$}
\end{enumerate}

The $2006$ version is able to deal with many class of particles
\fort{NCLA}, each class beeing described by an initial diameter and a
constituting char. Every char, among \fort{NCHAR} is described by a
complete set of parameters : immediate and ultimate analysis, low
heating value ({\small at user choice on raw, dry or pure}) and
kinetic parameters ({\em for both the two competitive pyrolysis
reaction and for heterogeneous reaction}). This allows to describe the
combustion of a mixture or coals or of coal and every material
describable by such evolution kinetics ({\small woods chips ...}). It
is, obviously, possible to mix fuels with ({\small very}) different
proximate analysis, like dry hard coal and wet biomass.


\newpage
%=================================
\subsection{Notations}
%=================================

\begin{table}[h!]
\begin{tabular}{ccp{10,5cm}}

{\bf Symbol} & {\bf Unit} & {\bf Meaning}\\


$H$ 		& $J/kg$ 	& specific enthalpy \\
$K$ 		& $kg/(m.\,s)$ 	& thermal diffusivity\\
$\lambda$ 	& $W/(m.\,K)$ 	& thermal conductivity\\
$\mu$	 	& $kg/(m.\,s)$ 	& dynamical viscosity\\
$\rho$ 		& $kg/m^3$ 	& density\\
$M$, $M_i$ 	& $kg/mol$ 	& molar mass ($M_i$ for  $i$� constituant)\\
$P$ 		& $Pa$ 		& pressure\\
$R$ 		& $J/(mol.\,K)$ & perfect gas constant\\
$T$ 		& $K$ 		& temperature ($>0$)\\
$Y_i$ 		& 		& mass fraction of constituant $i$ 
					($0 \leqslant Y_i \leqslant 1$)\\
$D^{t}$         & $kg/(m.\,s)$  & turbulent viscosity \\
$\alpha_{i}$    & 		& mass fraction of phase k \\
$t$ 		& $s$ 		& time\\
\end{tabular}
\end{table}

\clearpage

%=================================
\subsection{Budget Equations}
%=================================

The bulk, done of gases and particles, is assumed to be describable
with only one pressure and velocity. The slipping velocitiy between
particles and gases is supposed negligible compared to this mean
velocity.  Scalars for the bulk are :
\vspace{0.5cm}
\begin{itemize}
  \item Bulk density 
     \begin{equation} 
        \rho_{m} = \alpha_{1}\rho_{1} + \alpha_{2}\rho_{2}
     \end{equation} 
  \item Bulk velocity 
     \begin{equation} 
       U_{m} = \frac{ \alpha_{1}\rho_{1} U_{1} 
                    + \alpha_{2}\rho_{2} U_{2} }{\rho_{m}}
     \end{equation} 
  \item Bulk enthalpy 
     \begin{equation} 
        H_{m} = \frac{ \alpha_{1}\rho_{1} H_{1} 
                     + \alpha_{2}\rho_{2} H_{2} }{\rho_{m}}
     \end{equation} 
  \item Bulk pressure 
     \begin{equation} 
       P_{m} = P_{1}
     \end{equation} 
\end{itemize}  

Mass fractions of gaseous medium ($Y_{1}^{*}$) and of partciles are defined by :
\begin{eqnarray}
  Y_{1}^{*} = \frac{\alpha_{1}\rho_{1}}{\rho_{m}} &\\
  Y_{2}^{*} = \frac{\alpha_{2}\rho_{2}}{\rho_{m}} &
\end{eqnarray}

So budget equations for the bulk can be written :

\begin{equation}
  \frac{\partial}{\partial t    } \rho_{m}
 +\frac{\partial}{\partial x_{j}} (\rho_{m}U_{m,j}) = 0 
\end{equation}

\begin{equation}
  \frac{\partial}{\partial t    } (\rho_{m}U_{m,i})
 +\frac{\partial}{\partial x_{j}} (\rho_{m}U_{m,i}U_{m,j})
       =  \frac{\partial}{\partial x_{j}}
              \left[ \rho_{m} \left[ D_{m}^{t}( \frac{\partial U_{m,i}}{\partial x_{j}}
                                  +\frac{\partial U_{m,i}}{\partial x_{j}} ) \\
                      -\frac{2}{3}\delta_{ij}
                            ( q_{m}^{2}
                             +D_{m}^{t}\frac{\partial U_{m,l}}{\partial x_{l}}) \right]  \right]
           - \frac{\partial P_{m}}{\partial x_{i}}+\rho_{m}g_{i}
\end{equation}

\begin{equation}
  \frac{\partial}{\partial t    } (\rho_{m} H_{m})
 +\frac{\partial}{\partial x_{j}} (\rho_{m}U_{m,j}H_{m})
              = \frac{\partial}{\partial x_{j}} 
                       (\rho_{m}D_{m}^{t} \frac{\partial H_{m}}{\partial x_{j}})
               +S_{m,R}
\end{equation}
 
With the ({\small velocity}) homogeneity assumption, mainly budget
equation for bulk caracteristic are pertinent. So transport equation
for the scalar $\Phi_{k}$, where k is the phase, can be written :

\begin{equation}
  \frac{\partial}{\partial t    } (\rho_{m} Y_{k}^{*}\Phi_{k})
 +\frac{\partial}{\partial x_{j}} (\rho_{m} U_{m,j} Y_{k}^{*} \Phi_{k})
              = \frac{\partial}{\partial x_{j}} 
                       (\rho_{m}D_{m}^{t} \frac{\partial Y_{k}^{*} \Phi_{k}}{\partial x_{j}})
               +S_{\Phi_{k}}+\Gamma_{\Phi_{k}}
\end{equation}

%=================================
\subsection{Coal combustion scalars}
%=================================

\subsubsection{Bulk enthalpy : $H_{m}$ }
Budget equation for the specific enthalpy of the mixture ({\small gas
+ particles}) admits only one source term for radiative effects
$S_{m,R}$ :
\begin{equation}
    S_{m,R}= S_{1,R}+ S_{2,R}
\end{equation}
With contributions of each phases liable to be described by different
models ({\small eg : wide band for gases, black body for particles}).

\subsubsection{Particles enthalpy : $Y_{2}^{*}H_{2}$ }
Enthalpy of droplets ({\small J in particles/kg bulk}) is the product
of solid phase mass fraction ({\small kg liq/kg bulk}) by the specific
enthalpy of solid ({\small kg solid/kg bulk}). So the budget equation
for liquid enthalpy has six source terms : :
\begin{equation}
     \Pi_{2}^{'}+S_{2,R}-\Gamma_{evap}H_{H2Ovap}(T_{2})-\Gamma_{devol1}H_{MV1}(T_{2})-\Gamma_{devol2}H_{MV2}(T_{2})
                        +\Gamma_{het}\left( \frac{M_{O}}{M_{C}}H_{O_{2}}(T_{1})
                                      -\frac{M_{CO   }}{M_{C}}H_{CO   }(T_{2})\right) 
\end{equation}
with
\begin{itemize}
  \item $\Pi_{2}^{'}$ : heat flux between phases
  \item $S_{2,R}$ : radiative source term for droplets
  \item $\Gamma_{evap}H_{H2Ovap}(T_{2})$ the vapor flux leaves at particle temperature ($H_{vap}$ includes latent heat)
\item $\Gamma_{dvol1}H_{MV1}(T_{2})$ the light volatile matter flux leaves at particle temperature ($H_{vap}$ includes latent heat)
\item $\Gamma_{dvol2}H_{MV2}(T_{2})$ the heavy volatile matter flux leaves at particle temperature ($H_{vap}$ includes latent heat)
   \item $\Gamma_{het}(...)$ heterogenous combustion induces reciprocal mass flux : oxygen arriving at gas temperature and carbone monoxide leaving at char particle one.

\end{itemize}

\subsubsection{Dispersed phase mass fraction : $Y_{2}^{*}$}
In budget equation for the mass fraction of the dispersed phase
({\small first droplets, then char particles, at last ashes}) the
source terms are interfacial mass fluxes ({\small first evaporation,
then net flux for heterogeneous combustion}):

\begin{equation}
     -\Gamma_{evap}-\Gamma_{het}
\end{equation}
          
\subsubsection{Number of particles : $N_{p}^{*}$}
No source term in the budget equation for number of droplets : a
droplet became a particle ({\small eventually a tiny flying ash}) but
never vanish ({\small particles it have to get out}).

                                
\subsubsection{Mean of the passive scalar for light volatile : $F_{1}$}  
This scalar represent the amount of matter which have leaved the
particle as fuel vapour, whatever it happens after. It's a mass
fraction of gaseous matter ({\small in hydrocarbon form or carbon
oxide ones}). So the source term in its budget is only evaporation
mass flux :
\begin{equation}
   \Gamma_{evap}
\end{equation}     

\subsubsection{ Variance of $F_{1}$ : $F_{1}^{'2}$}
Budget equation for $F_{1}^{'2}$ have three source term :
\begin{equation}
   \Gamma_{F_{1}^{'2}}
   -2\rho_{m}Y_{1}^{*}\frac{F_{1}^{'2}}{\tau_{\chi_{F_{1}^{'2}}}} 
   + \rho_{m}Y_{1}^{*}D_{m}^{t}\frac{\partial F_{1}}{\partial x_{j}} 
                               \frac{\partial F_{1}}{\partial x_{j}}
\end{equation} 
where $\Gamma_{F_{1}^{'2}}$ is due to interfacial mass fluxes ({\small
every interfacial mass fluxes impact gaseous phase variances}).
                                                 
\subsubsection{Mean of the passive scalar for carbon from char : $F_{3}$}  
Budget equation for $F_{3}$ have for lone source term the mass flux
due to heterogeneous combsution ({\small mass flux of carbon monoxide
minus oxygene mass flux}). As for $F_{1}$ oxidation in the gaseous
phase does not modifiy this {\em passive} scalar :
\begin{equation}
   \Gamma_{het}
\end{equation}   
         
                       
\subsubsection{ Variance of the passive scalar for air : $F_{4}^{'2}$} 
 
This passive scalar incomes with air but is'nt destroyed by any
({\small in gaseous phase or heterogeneous}). No budget equation
needed for it, $F_{4}$ can be determined from the wholeness
relation.\\

Budget equation for $F_{4}^{'2}$ have, like other passive scalar
variance budget equaiton, four source terms :
\begin{equation} 
    \Gamma_{F_{4}^{'2}}
   -2\rho_{m}Y_{1}^{*}\frac{F_{4}^{'2}}{\tau_{\chi_{F_{4}^{'2}}}} 
   + \rho_{m}Y_{1}^{*}D_{m}^{t}\frac{\partial F_{4}}{\partial x_{j}} 
                               \frac{\partial F_{4}}{\partial x_{j}}
\end{equation} 
where $\Gamma_{F_{4}^{'2}}$ is due to interfacial mass fluxes ({\small every interfacial mass fluxes impact gaseous phase variances}).



%%%%%%%%%%%%%%%%%%%%%%%%%%%%%%%%%%
%%%%%%%%%%%%%%%%%%%%%%%%%%%%%%%%%%
\section{Discr\'etisation}
%%%%%%%%%%%%%%%%%%%%%%%%%%%%%%%%%%
%%%%%%%%%%%%%%%%%%%%%%%%%%%%%%%%%%

On se reportera aux sections relatives aux sous-programmes
\fort{cfmsvl} (masse volumique), \fort{cfqdmv} 
(quantit\'e de mouvement) et \fort{cfener} (\'energie).  La
documentation du sous-programme
\fort{cfxtcl} fournit des \'el\'ements relatifs aux 
conditions
aux limites. 

\programme{fubase}

%%%%%%%%%%%%%%%%%%%%%%%%%%%%%%%%%%
%%%%%%%%%%%%%%%%%%%%%%%%%%%%%%%%%%
\section{Function}
%%%%%%%%%%%%%%%%%%%%%%%%%%%%%%%%%%
%%%%%%%%%%%%%%%%%%%%%%%%%%%%%%%%%%

The combustion of heavy fuel oil is described. After a droplet enters
the furnace, its temperature increases reaching an evaporating range
({\small heavy fuel oil is'nt a specy with a well defined boiling
point}):
\begin{enumerate}
  \item As droplet's temperature reachs the begin of evaporating
          range, vapour begins to be given off. Gazeous hydrocarbons
          are then involved in a diffusion flame surrouding each
          droplet. The heavy fuel oil beeing constitued of long
          hydrocarbons molecules, cracking mays occur and char is
          leaved. So the complex chemistry inside the droplet is
          summarized by : \\ Heavy fuel oil $\longrightarrow$ gazeous
          hydrocarbons + char
 
  \item After evaporation, when the whole vapour are burnt, oxygen is
  able to reach the surface of the char particle. So heterogenous
  combustion can take place. It is very similar to the oxydation of
  coal char : diffusion of oxygen from bulk, heterogeneous reaction
  ({\small kinetically limited}) and back diffusion of carbone
  monoxide. The heterogenous combustion is achieved if all the carbon
  of the char particle is converted, leaving an ash particle ({\small
  heavy fuel oil can contain inerts, some of which beeing heavy
  metals}). Unburnt carbon can leave the boiler as flying
  particle. The heterogeneous reaction is wrotten : \\
\centerline{$Char + \frac{1}{2} 0_{2} =(k0het, T0het)=> CO$} 
\end{enumerate}

\vspace{0.5cm}
\noindent{} Fuel jet combsution strongly depends on injection conditions and namely on mean diameter of drops (size distribution). In the first (2006) version \CS take in account only one diameter : monodisperse injection.


\vspace{0.5cm}

\noindent{} For heavy fuel oil, like for coal, the french reference [1] can be useful :
 
\noindent{\bf [1]} Escaich, Alain~: ``Mise en oeuvre dans Code\_Saturne des
mod�lisations physiques particuli�res. Tome 2 : Combustion du charbon
pulv�ris�.'', HI-81/02/09/A, Rapport EDF, 2002.

\newpage
%=================================
\subsection{Notations}
%=================================

\begin{table}[h!]
\begin{tabular}{ccp{10,5cm}}

{\bf Symbol} & {\bf Unit} & {\bf Meaning}\\


$H$ 		& $J/kg$ 	& specific enthalpy \\
$K$ 		& $kg/(m.\,s)$ 	& thermal diffusivity\\
$\lambda$ 	& $W/(m.\,K)$ 	& thermal conductivity\\
$\mu$	 	& $kg/(m.\,s)$ 	& dynamical viscosity\\
$\rho$ 		& $kg/m^3$ 	& density\\
$M$, $M_i$ 	& $kg/mol$ 	& molar mass ($M_i$ for  $i$� constituant)\\
$P$ 		& $Pa$ 		& pressure\\
$R$ 		& $J/(mol.\,K)$ & perfect gas constant\\
$T$ 		& $K$ 		& temperature ($>0$)\\
$Y_i$ 		& 		& mass fraction of constituant $i$ 
					($0 \leqslant Y_i \leqslant 1$)\\
$D^{t}$         & $kg/(m.\,s)$  & turbulent viscosity \\
$\alpha_{i}$    & 		& mass fraction of phase k \\
$t$ 		& $s$ 		& time\\
\end{tabular}
\end{table}

\clearpage

%=================================
\subsection{Budget Equations}
%=================================

The bulk, done of gases and droplets, is assumed to be describable
with only one pressure and velocity. The slipping velocitiy between
droplets and gases is supposed negligible compared to this mean
velocity.  Scalars for the bulk are :
\vspace{0.5cm}
\begin{itemize}
  \item Bulk density 
     \begin{equation} 
        \rho_{m} = \alpha_{1}\rho_{1} + \alpha_{2}\rho_{2}
     \end{equation} 
  \item Bulk velocity 
     \begin{equation} 
       U_{m} = \frac{ \alpha_{1}\rho_{1} U_{1} 
                    + \alpha_{2}\rho_{2} U_{2} }{\rho_{m}}
     \end{equation} 
  \item Bulk enthalpy 
     \begin{equation} 
        H_{m} = \frac{ \alpha_{1}\rho_{1} H_{1} 
                     + \alpha_{2}\rho_{2} H_{2} }{\rho_{m}}
     \end{equation} 
  \item Bulk pressure 
     \begin{equation} 
       P_{m} = P_{1}
     \end{equation} 
\end{itemize}  

Mass fractions of gaseous medium ($Y_{1}^{*}$) and of droplets are defined by :
\begin{eqnarray}
  Y_{1}^{*} = \frac{\alpha_{1}\rho_{1}}{\rho_{m}} &\\
  Y_{2}^{*} = \frac{\alpha_{2}\rho_{2}}{\rho_{m}} &
\end{eqnarray}

So budget equations for the bulk can be written :

\begin{equation}
  \frac{\partial}{\partial t    } \rho_{m}
 +\frac{\partial}{\partial x_{j}} (\rho_{m}U_{m,j}) = 0 
\end{equation}

\begin{equation}
  \frac{\partial}{\partial t    } (\rho_{m}U_{m,i})
 +\frac{\partial}{\partial x_{j}} (\rho_{m}U_{m,i}U_{m,j})
       =  \frac{\partial}{\partial x_{j}}
              \left[ \rho_{m} \left[ D_{m}^{t}( \frac{\partial U_{m,i}}{\partial x_{j}}
                                  +\frac{\partial U_{m,i}}{\partial x_{j}} ) \\
                      -\frac{2}{3}\delta_{ij}
                            ( q_{m}^{2}
                             +D_{m}^{t}\frac{\partial U_{m,l}}{\partial x_{l}}) \right]  \right]
           - \frac{\partial P_{m}}{\partial x_{i}}+\rho_{m}g_{i}
\end{equation}

\begin{equation}
  \frac{\partial}{\partial t    } (\rho_{m} H_{m})
 +\frac{\partial}{\partial x_{j}} (\rho_{m}U_{m,j}H_{m})
              = \frac{\partial}{\partial x_{j}} 
                       (\rho_{m}D_{m}^{t} \frac{\partial H_{m}}{\partial x_{j}})
               +S_{m,R}
\end{equation}
 
With the ({\small velocity}) homogeneity assumption, mainly budget equation for bulk caracteristic  are pertinent. So transport equation for the scalar $\Phi_{k}$, where k is the phase, can be written :

\begin{equation}
  \frac{\partial}{\partial t    } (\rho_{m} Y_{k}^{*}\Phi_{k})
 +\frac{\partial}{\partial x_{j}} (\rho_{m} U_{m,j} Y_{k}^{*} \Phi_{k})
              = \frac{\partial}{\partial x_{j}} 
                       (\rho_{m}D_{m}^{t} \frac{\partial Y_{k}^{*} \Phi_{k}}{\partial x_{j}})
               +S_{\Phi_{k}}+\Gamma_{\Phi_{k}}
\end{equation}

%=================================
\subsection{Fuel combustion scalars}
%=================================

\subsubsection{Bulk enthalpy : $H_{m}$ }
Budget equation for the specific enthalpy of the mixture ({\small gas
+ droplets}) admits only one source term for radiative effects
$S_{m,R}$ :
\begin{equation}
    S_{m,R}= S_{1,R}+ S_{2,R}
\end{equation}
With contributions of each phases liable to be described by different
models ({\small eg : wide band for gases, black body for particles}).

\subsubsection{Droplets enthalpy : $Y_{2}^{*}H_{2}$ }
Enthalpy of droplets ({\small J in droplets/kg bulk}) is the product
of liquid phase mass fraction ({\small kg liq/kg bulk}) by the
specific enthalpy of liquid ({\small kg liq/kg bulk}). So the budget
equation for liquid enthalpy has four source terms : :
\begin{equation}
     \Pi_{2}^{'}+S_{2,R}-\Gamma_{evap}H_{vap}(T_{2})
                        +\Gamma_{het}\left( \frac{M_{O}}{M_{C}}H_{O_{2}}(T_{1})
                                      -\frac{M_{CO   }}{M_{C}}H_{CO   }(T_{2})\right) 
\end{equation}
with
\begin{itemize}
  \item $\Pi_{2}^{'}$ : heat flux between phases
  \item $S_{2,R}$ : radiative source term for droplets
  \item $\Gamma_{evap}H_{vap}(T_{2})$ the vapor flux leaves at droplet temperature ($H_{vap}$ includes latent heat)
   \item $\Gamma_{het}(...)$ heterogenous combustion induces reciprocal mass flux : oxygen arriving at gas temperature and carbone monoxide leaving at char particle one.

\end{itemize}
    
\subsubsection{Dispersed phase mass fraction : $Y_{2}^{*}$}
In budget equation for the mass fraction of the dispersed phase
({\small first droplets, then char particles, at last ashes}) the
source terms are interfacial mass fluxes ({\small first evaporation,
then net flux for heterogeneous combustion}):

\begin{equation}
     -\Gamma_{evap}-\Gamma_{het}
\end{equation}
          
\subsubsection{Number of droplets : $N_{p}^{*}$}
No source term in the budget equation for number of droplets : a
droplet became a particle ({\small eventually a tiny flying ash}) but
never vanish ({\small particles it have to get out}).

                                
\subsubsection{Mean of the passive scalar for fuel vapor : $F_{1}$}  
This scalar represent the amount of matter which have leaved the
particle as fuel vapour, whatever it happens after. It's a mass
fraction of gaseous matter ({\small in hydrocarbon form or carbon
oxide ones}). So the source term in its budget is only evaporation
mass flux :
\begin{equation}
   \Gamma_{evap}
\end{equation}     

\subsubsection{ Variance of $F_{1}$ : $F_{1}^{'2}$}
Budget equation for $F_{1}^{'2}$ have three source term :
\begin{equation}
   \Gamma_{F_{1}^{'2}}
   -2\rho_{m}Y_{1}^{*}\frac{F_{1}^{'2}}{\tau_{\chi_{F_{1}^{'2}}}} 
   + \rho_{m}Y_{1}^{*}D_{m}^{t}\frac{\partial F_{1}}{\partial x_{j}} 
                               \frac{\partial F_{1}}{\partial x_{j}}
\end{equation} 
where $\Gamma_{F_{1}^{'2}}$ is due to interfacial mass fluxes ({\small
every interfacial mass fluxes impact gaseous phase variances}).
                                                 
\subsubsection{Mean of the passive scalar for carbon from char : $F_{3}$}  
Budget equation for $F_{3}$ have for lone source term the mass flux
due to heterogeneous combsution ({\small mass flux of carbon monoxide
minus oxygene mass flux}). As for $F_{1}$ oxidation in the gaseous
phase does not modifiy this {\em passive} scalar :
\begin{equation}
   \Gamma_{het}
\end{equation}   
         
                       
\subsubsection{ Variance of the passive scalar for air : $F_{4}^{'2}$} 
 
This passive scalar incomes with air but is'nt destroyed by any
({\small in gaseous phase or heterogeneous}). No budget equation
needed for it, $F_{4}$ can be determined from the wholeness
relation.\\

Budget equation for $F_{4}^{'2}$ have, like other passive scalar
variance budget equaiton, four source terms :
\begin{equation} 
    \Gamma_{F_{4}^{'2}}
   -2\rho_{m}Y_{1}^{*}\frac{F_{4}^{'2}}{\tau_{\chi_{F_{4}^{'2}}}} 
   + \rho_{m}Y_{1}^{*}D_{m}^{t}\frac{\partial F_{4}}{\partial x_{j}} 
                               \frac{\partial F_{4}}{\partial x_{j}}
\end{equation} 
where $\Gamma_{F_{4}^{'2}}$ is due to interfacial mass fluxes ({\small
every interfacial mass fluxes impact gaseous phase variances}).






%%%%%%%%%%%%%%%%%%%%%%%%%%%%%%%%%%
%%%%%%%%%%%%%%%%%%%%%%%%%%%%%%%%%%
\section{Discretisation}
%%%%%%%%%%%%%%%%%%%%%%%%%%%%%%%%%%
%%%%%%%%%%%%%%%%%%%%%%%%%%%%%%%%%%

The discretisation of equation are not problematic. Details are in
sections : \fort{fuflux} (interfacial fluxes of mass and energy),
\fort{futssc} (source term for fuel specific scalars)
and \fort{fucym1} (gas phase combustion). 

%-------------------------------------------------------------------------------

% This file is part of Code_Saturne, a general-purpose CFD tool.
%
% Copyright (C) 1998-2011 EDF S.A.
%
% This program is free software; you can redistribute it and/or modify it under
% the terms of the GNU General Public License as published by the Free Software
% Foundation; either version 2 of the License, or (at your option) any later
% version.
%
% This program is distributed in the hope that it will be useful, but WITHOUT
% ANY WARRANTY; without even the implied warranty of MERCHANTABILITY or FITNESS
% FOR A PARTICULAR PURPOSE.  See the GNU General Public License for more
% details.
%
% You should have received a copy of the GNU General Public License along with
% this program; if not, write to the Free Software Foundation, Inc., 51 Franklin
% Street, Fifth Floor, Boston, MA 02110-1301, USA.

%-------------------------------------------------------------------------------

\programme{Thermodynamics}
{\huge sub-routines : pptbht, cothht, colecd, cplecd, fulecd ...}

%%%%%%%%%%%%%%%%%%%%%%%%%%%%%%%%%%
%%%%%%%%%%%%%%%%%%%%%%%%%%%%%%%%%%
\section{Function}
%%%%%%%%%%%%%%%%%%%%%%%%%%%%%%%%%%
%%%%%%%%%%%%%%%%%%%%%%%%%%%%%%%%%%

The description of the thermodynamical of gases mixture is as close as possible
of the JANAF standard. The gases mixture is, often, considered as composed of
some {\em global} species ({\small eg. oxidizer, products, fuel}) each of them
beeing a mixture ({\small with known ratio}) of {\em elementary} species
({\small oxygen, nitrogen, carbon dioxide, ...}).\\ A tabulation of the enthalpy
of both elementary and global species for some temperatures is constructed
({\small using JANAF polynoms}) or read ({\small if the user found useful to
define a global specie not simply related to elementary ones ; eg. unspecified
hydrocarbon known by C, H, O, N, S analysis and heating value.}).\\ The
thermodynamic properties of condensed phase are more simple : formation enthalpy
is computed using properties of gaseous products of combustion with air ({\small
formation enthalpy of wich is zero valued as O2 and N2 are reference state}) and
the lower heating value. The heat capacity of condensed phase is assumed
constant and it is a data the user have to type ({\small in the corresponding
data file dp\_FCP or dp\_FUE}).


%d
%%%%%%%%%%%%%%%%%%%%%%%%%%%%%%%%%%
%%%%%%%%%%%%%%%%%%%%%%%%%%%%%%%%%%
\section{Gases enthalpy discretisation}
%%%%%%%%%%%%%%%%%%%%%%%%%%%%%%%%%%
%%%%%%%%%%%%%%%%%%%%%%%%%%%%%%%%%%

A table of gases ({\small both elementary species and global ones}) enthalpy for
some temperatures ({\small user choses number of points, temperature in dp\_***
file}) is computed ({\small enthalpy of elementary species is computed using
JANAF polynomia ; enthalpy for global species are computed by weighting of
elementary ones})or read ({\small subroutine PPTBHT}). Then the entahlpy is
supposed to be linear vs. temperature in each temperature gap ({\small
i.e. continuous piece wise linear on the whole temperature range}). As a
consequence, temperature is a linear function of entahlpy ; and a simple
algorithm ({\small subroutine COTHHT}) allows to determine the enthalpy of a
mixture of gases ({\small for inlet conditions it is more useful to indicate
temperature and mass fractions}) or to determine temperature from enthalpy of
the mixture and mass fractions of global species ({\small common use in every
fluid particle, at every time step}).
%%%%%%%%%%%%%%%%%%%%%%%%%%%%%%%%%%
%%%%%%%%%%%%%%%%%%%%%%%%%%%%%%%%%%
\section{Particles enthalpy discretisation}
%%%%%%%%%%%%%%%%%%%%%%%%%%%%%%%%%%
%%%%%%%%%%%%%%%%%%%%%%%%%%%%%%%%%%

Enthalpy of condensed material are rarely known. Commonly, the thermal power and
ultimate analysis are determined. So, using simple assumptions and the enthalpy
of known released species ({\small after burning with air}) the formation
enthalpy of coal or heavy oil can be computed. Assuming the thermal capacity is
constant for every condensed material a table can be build with ... two
temperatures, allowing the use of the same simple algorithm for
temperature-enthalpy conversion. When intermediate gaseous species ({\small
volatile or vapour}) are thermodynamically known, simple assumptions({\small eg
: char is thermodynamically equivalent to pure carbon in reference state ; ashes
are inert}) allow to deduce enthalpy for heterogeneous reactions ({\small these
energies have not to be explicitely taken in account for the energy budget of
particles}).

\part{Mesh Handling Algorithms}
\stepcounter{prog}
\setcounter{section}{0}
\setcounter{equation}{0}
\setcounter{figure}{0}
%-------------------------------------------------------------------------------

% This file is part of Code_Saturne, a general-purpose CFD tool.
%
% Copyright (C) 1998-2019 EDF S.A.
%
% This program is free software; you can redistribute it and/or modify it under
% the terms of the GNU General Public License as published by the Free Software
% Foundation; either version 2 of the License, or (at your option) any later
% version.
%
% This program is distributed in the hope that it will be useful, but WITHOUT
% ANY WARRANTY; without even the implied warranty of MERCHANTABILITY or FITNESS
% FOR A PARTICULAR PURPOSE.  See the GNU General Public License for more
% details.
%
% You should have received a copy of the GNU General Public License along with
% this program; if not, write to the Free Software Foundation, Inc., 51 Franklin
% Street, Fifth Floor, Boston, MA 02110-1301, USA.

%-------------------------------------------------------------------------------

\nopagebreak

In this chapter, we will describe algorithms used for several
operations done by \CS.

\section*{Geometric Quantities\label{sec:geo_quant}}

\hypertarget{meshquantities}{}

See the \doxygenfile{cs__mesh__quantities_8c.html}{programmers reference of the dedicated subroutine} for further details.

\subsection*{Normals and Face Centers%
             \label{sec:geo_quant.normal}}

To calculate face normals, we take care to use an algorithm
that is correct for any planar simple polygon, including non-convex cases.
The principle is as follows: take an arbitrary point $P_a$ in the
same plane as the polygon, then compute the sum of the vector normals
of triangles $\{P_a, P_i, P_{i+1}\}$, where $\{P_1, P_2, ..., P_i, ..., P_n\}$
are the polygon vertices and $P_{n+1} \equiv P_0$. As shown on figure
\ref{fig:algo.norm_fac.principle}, some normals have a ``positive''
contribution while others have a ``negative'' contribution (taking into
account the ordering of the polygon's vertices). The length of the final normal
obtained is equal to the polygon's surface.

\begin{figure}[!h]
\centerline{
\includegraphics*[height=4.5cm]{face_surf}}
\caption{Face normals calculation principle}
\label{fig:algo.norm_fac.principle}
\end{figure}

In our implementation, we take the ``arbitrary'' $P_a$ point as
the center of the polygon's vertices, so as to limit
precision problems due to truncation errors and to ensure that
the chosen point is on the polygon's plane.

\begin{figure}[!h]
\centerline{
\includegraphics*[height=3.5cm]{face_quant}}
\caption{Triangles for calculation of face quantities}
\label{fig:algo.grd_fac.triangles}
\end{figure}

A face's center is defined as the weighted center $G$ of triangles
$T_i$ defined as $\{P_a, P_i, P_{i+1}\}$ and whose centers are
noted $G_i$. Let $O$ be the center of the coordinate system and
$\overrightarrow{n_f}$ the face normal, then:

\begin{displaymath}
\overrightarrow{OG}
= \frac{\sum_{i=1}^n \textrm{surf}(T_i).\overrightarrow{OG_i}}
       {\sum_{i=1}^n \textrm{surf}(T_i)}
\qquad \textrm {avec} \quad
\textrm{surf}(T_i) = \frac{\overrightarrow{n_{T_i}}.\overrightarrow{n_f}}
                          {\parallel \overrightarrow{n_f} \parallel}
\end{displaymath}

It is important to use the signed surface of each triangle so
that this formula remains true in the case of non convex faces.

In real cases, some faces are nor perfectly planar. In this case,
a slight error is introduced, but it is difficult to choose an exact
and practical (implementation and computing cost wise) definition of a
polygon's surface when its edges do not all lie in a same plane.

So as to limit errors due to warped faces, we compare the contribution
of a given face to the the neighboring cell volumes (through
Stoke's formula) with the contribution obtained from the separate
triangles $\{P_a, P_i, P_{i+1}\}$, and we translate the initial center
of gravity along the face normal axis so as to obtain the same contribution.

\subsection*{Cell Centers%
               \label{sec:geo_quant.cdgcel}}

If we consider that in theory, the Finite Volume method uses constant
per-cell values, we can make without a precise knowledge of a given cell's
center, as any point inside the cell could be chosen.
In practice, precision (spatial order) depends on a good choice of
the cell's center, as this point is used for the computation of values
and gradients at mesh faces.
We do not compute the center a the circumscribed sphere as
a cell center, as this notion is usually linked to tetrahedral meshes,
and is not easily defined and calculated for general polyhedra.

Let us consider a cell $\mathcal{C}$ with $p$ faces of centers of gravity
$G_k$ and surfaces of norm $S_k$. If $O$ is the origin of the coordinate
system, $\mathcal{C}$'s center $G$ is defined as:

\begin{displaymath}
\overrightarrow{OG}
= \frac{\sum_{k=1}^p S_k.\overrightarrow{OG_k}}
       {\sum_{k=1}^p S_k}
\end{displaymath}

An older algorithm computed a cell $\mathcal{C}$'s center of gravity as the
center of gravity of its vertices $q$ of coordinates $X_l$:

$$\overrightarrow{OG} = \sum_{l=1}^q \frac{\overrightarrow{OX_l}}{q}$$

In most cases, the newer algorithm gives better results, though there
are exceptions (notably near the axis of partial meshes with rotational symmetry).

On figure \ref{fig:algo.cog_cel.loc}, we show the center of gravity
calculated with both algorithms on a 2D cell. On the left, we have
a simple cell. On the right, we have added additional vertices as they
occur in the case of a joining of non conforming faces. The position
of the center of gravity is stable using the newer algorithm, whilst
this point is shifted towards the joined sub-faces with the older,
vertex-based algorithm (which worsens the mesh quality).

\begin{figure}[!h]
\centerline{
\includegraphics*[width=15.5cm]{cell_cog}}
\caption{Choice of cell center}
\label{fig:algo.cog_cel.loc}
\end{figure}

On figure \ref{fig:algo.cog_cel.nonorth}, we show the possible effect of
the choice of a cell's COG on the position of line segments joining the COG's
of neighboring cells after a joining of non-conforming cells.

\begin{figure}[!h]
\centerline{
\includegraphics*[height=4.5cm]{cell_cog_nonorth}}
\caption{Choice of cell center and face orthogonality}
\label{fig:algo.cog_cel.nonorth}
\end{figure}

We see here that the vertex-based algorithm tends to increase
non-orthogonality of faces, compared to the present
face-based algorithm.

\section*{Conforming Joining\label{sec:join}}

The use of non conforming meshes is one of \CS's key features, and
the associated algorithms constitute the most complex part of the
code's preprocessor. The idea is to build new faces corresponding to
the intersections of the initial faces to be joined.
Those initial faces are then replaced by their covering built
from the new faces, as shown on figure \ref{fig:algo.join.principle}
(from a 2D sideways perspective):

\begin{figure}[!h]
\centerline{
\includegraphics*[height=5cm]{join_principle}}
\caption{Principle of face joinings}
\label{fig:algo.join.principle}
\end{figure}

We speak of \emph{conforming joining}, as the mesh resulting from
this joining is conforming, whereas the initial mesh was not.

The number of faces of a cell of which one side has been joined will
be greater or equal to the initial number of faces, and the new faces
resulting from the joining will not always be triangles or quadrangles,
even if all of the initial faces were of theses types. For this
reason, the data representations of \CS and its preprocessor
are designed with arbitrary simple polygonal faces and polyhedral
cells in mind. We exclude polygons and polyhedra with holes from
this representation, as shown on figure \ref{fig:algo.join.possible}.
Thus the cells shown in figure \ref{fig:algo.join.possible} cannot be
joined, as this would require opening a ``hole'' in one of the larger
cell's faces. In the case of figure \ref{fig:algo.join.possible}b, we
have no such problem, as the addition of another smaller cell
splits the larger face into pieces that do not contain holes.

\begin{figure}[!h]
\centerline{
\includegraphics*[height=5cm]{join_possible}}
\caption{Possible case joinings}
\label{fig:algo.join.possible}
\end{figure}

\subsection*{Robustness Factors%
               \label{sec:join.robust}}

We have sought to build a joining algorithm that could function with
a minimum of user input, on a wide variety of cases.

Several criteria were deemed important:

\begin{enumerate}

\item {\bf determinism}: we want to be able to predict the algorithm's behavior.
We especially want the algorithm to produce the same results whether
some mesh $A$ was joined to mesh $B$ or $B$ to $A$. This might not be perfectly
true in the implementation due to truncation errors, but the main point is
that the user should not have to worry about the order in which he enters
his meshes for the best computational results.
\footnote{The geometry produced by a joining is in theory independent
from the order mesh inputs, but mesh numbering is not. The input order
used for a calculation should thus be kept for any calculation restart.}

\item {\bf non planar surfaces}: We must be able to join both curved surface
meshes and meshes of surfaces assembled from a series of planar sections,
but whose normal is not necessarily a continuous function of space,
as shown on figure \ref{fig:algo.join.non_planar}.

\begin{figure}[!h]
\centerline{
\includegraphics*[height=3cm]{join_non_planar}}
\caption{Initial surfaces}
\label{fig:algo.join.non_planar}
\end{figure}

\item {\bf spacing between meshes}: the surfaces to be joined may
not match perfectly, due to truncation errors or precision differences,
or to the approximation of curved surfaces by a set of planar faces.
The algorithm must not leave gaps where none are desired.

\end{enumerate}

\subsection*{Basic Principle\label{sec:join.principe}}

Let us consider two surfaces to join, as in figure \ref{fig:algo.join.curv}:
We seek to determine the intersections of the edges of the mesh faces,
and to split these edges at those intersections, as shown on figure
\ref{fig:algo.join.curv2}. We will describe more precisely what we
mean by ``intersection'' of two edges in a later section, as
the notion involves spanning of small gaps in our case.

\begin{figure}[!h]
\centerline{
\includegraphics*[height=4cm]{join_overlap_3d_1}}
\caption{Surfaces to be joined}
\label{fig:algo.join.curv}
\end{figure}

\begin{figure}[!h]
\centerline{
\includegraphics*[height=4cm]{join_overlap_3d_2}}
\caption{After edge intersections}
\label{fig:algo.join.curv2}
\end{figure}

The next step consists of reconstructing sub-faces derived from the
initial faces. Starting from an edge of an initial face, we try to find
closed loops, choosing at each vertex the leftmost edge (as seen standing
on that face, normal pointing upwards, facing in the direction of the
current edge), until we have returned to the starting vertex. This way,
we find the shortest loop turning in the trigonometric
direction. Each face to be joined is replaced by its covering of
sub-faces constructed in this manner.

When traversing the loops, we must be careful to stay close to the plane
of the original face. We thus consider only the edges belonging to a face
whose normal has a similar direction to that of the face being subdivided
(i.e. the absolute value of the dot product of the two unitary normals
should be close to 1).

\begin{figure}[!h]
\centerline{
\includegraphics*[height=4.5cm]{join_overlap_3d_3}}
\caption{Sub-face reconstruction.}
\label{fig:algo.join.curv3}
\end{figure}

Once all the sub-faces are built, we should have obtained for two tangent
initial faces two topologically identical sub-faces, each descending
from one of the initial faces and thus belonging to a different cell.
All that is required at this stage is to merge those two sub-faces,
conserving the properties of both. The merged sub-face thus belongs to
two cells, and becomes an internal face. The joining is thus finalized.

\subsection*{Simplification of Face Joinings\label{sec:join.simplif}}

For a finite-volume code such as \CS, it is best that faces belonging
to one same cell have neighboring sizes. This is hard to ensure
when non-conforming boundary faces are split so as to be joined
in a conforming way. On figure \ref{fig:algo.join.simplif},
we see that this can produce faces of highly varying sizes
when splitting a face for conformal joining.

It is possible to simplify the covering of a face so as to limit
this effect, by moving vertices slightly on each side of the covering.
We seek to move vertices so as to simplify the covering while
deforming the mesh as little as possible.

One covering example is presented figure \ref{fig:algo.join.simplif},
where we show several simplification opportunities. We note that all
these possibilities are associated with a given edge. Similar
possibilities associated with other edges are not shown so that the
figure remains readable. After simplification, we obtain the
situation of figure \ref{fig:algo.join.simpl2}.

\begin{figure}[!h]
\centerline{
\includegraphics*[height=5cm]{join_simplify_1}}
\caption{Simplification possibilities}
\label{fig:algo.join.simplif}
\end{figure}

After simplification, we have the following situation:

\begin{figure}[!h]
\centerline{
\includegraphics*[height=5cm]{join_simplify_2}}
\caption{Faces after simplification}
\label{fig:algo.join.simpl2}
\end{figure}

\subsection*{Processing\label{sec:join.process}}

The algorithm's starting point is the search for intersections
of edges belonging to the faces selected for joining. In 3D, we
do not restrict ourselves to ``true'' intersections, but
we consider that two edges intersect as soon as the minimum
distance between those edges is smaller than a certain tolerance.

\begin{figure}[!h]
\centerline{
\includegraphics*[height=3cm]{join_edge_inter_3d}}
\caption{Intersection of edges in 3d space}
\label{fig:algo.join.edge}
\end{figure}

To each vertex we associate a maximum distance, some factor of
the length of the smallest edge incident to that vertex.
This factor is adjustable (we use $0.1$ by default), but
should always be less than $0.5$.
By default, this factor is usually multiplied by the smallest sine
of the angles between the edges considered, so that the
tolerance assigned to a given vertex is a good estimation
of the smallest height/width/depth or the adjacent cells.

On figure \ref{fig:algo.join.tolerance}, we illustrate
this tolerance in 2d with a factor of $0.25$, with red circles
showing the tolerance region without the sine factor
correction, and blue circle showing the tolerance with
the sine correction.

\begin{figure}[!h]
\centerline{
\includegraphics*[height=3cm]{join_tolerance}}
\caption{Join tolerance for vertices of joinable faces}
\label{fig:algo.join.tolerance}
\end{figure}

We consider a neighborhood of an edge defined by the spheres
associated to the minimal distances around each vertex and
the shell joining this two spheres, as shown on figure
\ref{fig:algo.join.edgeint_eps}.
More precisely, at a point on the edge of linear abscissa $s$,
the neighborhood is defined to be the sphere of radius
$d_{max}(s) = (1-s)d_{max}\mid_{s=0} + s.d_{max}\mid_{s=1}$.

We will thus have intersection between edges $E1$ and $E2$ as soon
as the point of $E1$ closest to $E2$ is within the neighborhood
of $E2$, and that simultaneously, the point of $E2$ closest
to $E1$ is inside the neighborhood of $E1$.

\begin{figure}[!h]
\centerline{
\includegraphics*[height=4cm]{join_edge_inter_3d_eps}}
\caption{Tolerances for intersection of edges}
\label{fig:algo.join.edgeint_eps}
\end{figure}

If edge $E2$ cuts the neighborhood of a vertex of edge $E1$,
and this vertex is also in $E2$'s neighborhood, we choose this
vertex to define the intersection rather than the point of
$E1$ closest to $E2$. This avoids needlessly cutting edges.
We thus search for intersections with the following order
of priority: vertex-vertex, vertex-edge, then edge-edge.
If the neighborhoods associated with two edges intersect,
but the criterion:\\
$\exists P1 \in A1, \exists P2 \in A2, d(P1,P2)
 < min(d_{max}(P1), d_{max}(P2))$\\
is not met, we do not have
intersection. These cases are shown on figure
\ref{fig:algo.join.edgeint_type}.

\begin{figure}[!h]
\centerline{
\includegraphics*[height=5cm]{join_edge_inter_3d_type}}
\caption{Tolerances for intersection of edges}
\label{fig:algo.join.edgeint_type}
\end{figure}

\subsection*{Problems Arising From the Merging of Two Neighboring Vertices
                \label{sec:join.pb_merge}}

If we have determined that a vertex $V_1$ should be merged with a
vertex $V_2$ and independently that this vertex $V_2$ should be
merged with a vertex $V_3$, then $V_1$ and $V_3$ should be
merged as a result, even though these vertices share no intersection.
We refer to this problem as merging transitivity and show
theoretical situations leading to it on figure \ref{fig:algo.merging.pb}.

\begin{figure}[!h]
\centerline{
\includegraphics*[height=5cm]{join_merge_1}}
\caption{Merging transitivity.}
\label{fig:algo.merging.pb}
\end{figure}

On figure \ref{fig:algo.merging.pb_1}, we show cases more
characteristic of what we can obtain with real meshes, given
that the definition of local tolerances reduces the risk and
possible cases of this type of situation.

\begin{figure}[!h]
\centerline{
\includegraphics*[height=5cm]{join_merge_2}}
\caption{Merging transitivity (real cases)}
\label{fig:algo.merging.pb_1}
\end{figure}

Figure \ref{fig:algo.merging.pb_2} illustrates the effect of
a combination of merges on vertices belonging to a same edge.
We see that in this case, edges initially going through
vertices $G$ and $J$ are strongly deformed (i.e. cut into
sub-edges that are not well aligned). Without transitivity,
edges going through vertices descended only from the merging
of $(G, H)$ on one hand and $(L, J)$ on the other hand
would be much closer to the initial edges.

To avoid excessive simplifications of the mesh arising from a
combination of merges, we first build ``chains'' of intersections
which should be merged, compute the coordinates of the merged
intersection, and then check for all intersection couples
of a given chain if excessive merging would occur. If this is the case,
we compute a local multiplicative factor ($< 1$) for all the
intersections from this chain, so as to reduce the merge tolerance
and ``break'' this chain into smaller subchains containing only
intersections withing each other's merging distances.

Tolerance reduction may be done in multiple steps, as we
try to break the weakest equivalences (those closest to
the merge tolerance bounds) first.

\begin{figure}[!h]
\centerline{
\includegraphics*[height=6cm]{join_merge_3}}
\caption{Merging transitivity for an edge}
\label{fig:algo.merging.pb_2}
\end{figure}

On figure \ref{fig:algo.merging.pb_3}, we show the possible
effect of merge transitivity limitation on vertices belonging to several edges.
Here, breaking of excessive intersection mergings should lead to
merging of intersections $(G, H, J)$, while $I$ is not merged
with another intersection.

\begin{figure}[!h]
\centerline{
\includegraphics*[height=5cm]{join_merge_4}}
\caption{Limited merging transitivity}
\label{fig:algo.merging.pb_3}
\end{figure}

\subsection*{Algorithm Optimization\label{sec:join.optim}}

Certain factors influence both memory and CPU requirements. We always
try to optimize both, with a slight priority regarding memory
requirements.

When searching for edge intersections, we try to keep the number of intersection
tests to a minimum. We compute coordinate axis-aligned bounding boxes associated
with each joinable face (augmented by the tolerance radii of associated
vertices), and run a full edge intersection test only for
edges belonging to faces whose bounding boxes intersect.

To determine which face bounding boxes intersect, we build a ``bounding
box-tree'' , similar to an octree, but with boxes belonging to all
the octree leaves they overlap. When running in parallel using MPI, a first,
coarser version of the tree with the same maximum depth on all ranks is built
so as to estimate the optimal distribution of data, to balance both the computational
and memory load. Bounding boxes are then migrated to the target ranks,
where the final box-tree (which may have different depth on different ranks)
is built. Assigning the global ids of the matching faces to the bounding boxes
ensures the search results are usable independently of the distribution
across MPI ranks.

\subsection*{Influence on mesh quality\label{sec:join.quality}}

It is preferable for a FV solver such as \CS that the mesh be as
``orthogonal'' as possible (a face is perfectly orthogonal when the
segment joining its center of mass to the center of the other cell to
which it belongs is perfectly aligned with its normal).
It is also important to avoid non planar faces
\footnote {Computation of face COG's includes a correction such that
the contribution of a warped face to a cell's volume is the same
as that of the same face split into triangles joining that face's
COG and outer edges, but this correction may not be enough
for second order values.}.
By the joining algorithm's principle, orthogonal faces are split
into several non-orthogonal sub-faces. In addition, the higher
the tolerance setting, the more merging of neighboring vertices
will tend to warp faces on the sides of cells with joined faces.

It is thus important to know when building a mesh that a mesh
built by joining two perfectly regular hexahedral meshes may
be of poor quality, especially if a high tolerance value
was used and the cell-sizes of each mesh are very different.
It is thus important to use the mesh quality criteria visualizations
available in \CS, and to avoid arbitrary joinings in sensible
areas of the mesh. When possible, joining faces of which one
set is already a subdivision of another (to construct local
refinements for example) is recommended.

\section*{Periodicity\label{sec:algo.perio}}

We use an extension of the non-conforming faces joining algorithm
to build periodic structures. The basic principle is described
figure \ref{fig:algo.rc_perio}:

\begin{itemize}

\item Let us select a set of boundary faces. These faces (and their
      edges and vertices) are duplicated, and the copy is moved
      according to the periodic step (a combination of translation
      and rotation). A link between original and duplicate
      entities is kept.

\item We use a conforming joining on the union of
      selected faces and their duplicates.
      This joining will slightly deform the mesh so that vertices
      very close to each other may be merged, and periodic
      faces may be split into conforming sub-faces (if they are
      not already conforming).

\item If necessary, the splitting of duplicated, moved, and joined
      faces is also applied to the faces from which they were
      descended.

\item Duplicated entities which were not joined (i.e. excess entities)
      are deleted.

\end{itemize}

\begin{figure}[!h]
\centerline{
\includegraphics*[height=8cm]{join_perio}}
\caption{Periodic joining principle (translation example)}
\label{fig:algo.rc_perio}
\end{figure}

It is thus not necessary to use periodic boundary conditions
that the periodic surfaces be meshed in a periodic manner,
though it is always best not to make to much use of the
comfort provided by conforming joinings, as it can lower
mesh quality (and as a consequence, quality of computational
results).

We note that it could seem simpler to separate periodic faces
in two sets of ``base'' and ``periodic'' faces, so as
to duplicate and transform only the first set. This would
allow for some algorithm simplifications and optimizations,
but here we gave a higher priority to the consistency
of user input for specification of face selections, and
the definition of two separate sets of faces would have
made user input more complex. As for the basic
conforming joining, it is usually not necessary to specify
face selections for relatively simple meshes (in which case
all faces on the mesh boundary are selected).

\section*{Triangulation of faces\label{sec:triangle}}

Face triangulation is done in two
stages. The first stage uses an \emph{ear cutting} algorithm, which
can triangulate any planar polygon, whether convex or not.
The triangulation is arbitrary, as it depends on the vertex chosen
to start the loop around the polygon.

The second stage consists of flipping edges so that the final
triangulation is constrained Delaunay, which leads to a
more regular triangulation.

\section*{Initial triangulation\label{sec:triangle_ini}}

\begin{figure}[!h]
\centerline{
\includegraphics*[width=14cm]{face_split_main}}
\caption{Principle of face triangulation}
\label{fig:algo.ear_splitting}
\end{figure}

The algorithm used is based on the one described in \cite{Theussl:1998}.
Its principle is illustrated figure \ref{fig:algo.ear_splitting}.
We start by checking if the triangle defined by the first vertices
of the polygon, $(P_0, P_1, P_2)$ is an ``ear'', that is if it is
interior to the polygon and does not intersect it. As this is the case
on this example, we obtain a first triangle, ad we must then process
the remaining part of the polygon. At the next stage triangle
$(P_0, P_2, P_3)$ is also an ear, and may be removed.

At stage $2$, we see that the next triangle which is a candidate for
removal, $(P_0, P_3, P_4)$ is not an ear, as it is not contained in the
remaining polygon. We thus shift the indexes of vertices to consider,
and see at stage $4$ that triangle $(P_3, P_4, P_5)$
is an ear and may be removed.

The algorithm is built in such a way that a triangle is selected based on
the last vertex of the last triangle considered (starting from triangle
$(P_0, P_1, P_2)$). Thus, we consider at stage $5$ the triangle ending
with $P_5$, that is $(P_0, P_3, P_5)$.
Once this triangle is removed, the remaining polygon is a triangle,
and its handling is trivial.

\section*{Improving the Triangulation\label{sec:triangle_delaunay}}

We show on figures \ref{fig:algo.decoup_ex_1} and \ref{fig:algo.decoup_ex_2}
two examples of a triangulation on similar polygons whose vertices are
numbered in a different manner.

\begin{figure}[!h]
\centerline{
\includegraphics*[height=2.5cm]{face_split_1}}
\caption{Triangulation example (1)}
\label{fig:algo.decoup_ex_1}
\end{figure}

\begin{figure}[!h]
\centerline{
\includegraphics*[height=2.5cm]{face_split_2}}
\caption{Triangulation example (2)}
\label{fig:algo.decoup_ex_2}
\end{figure}

\vfill

Not only is the obtained triangulation different, but it has a tendency
to produce very flat triangles. Once a first triangulation is obtained,
we apply a corrective algorithm, based on edge flips so as to respect
the Delaunay condition \cite{Shewchuck:1999}.

\begin{figure}[!h]
\centerline{
\includegraphics*[height=4cm]{face_split_delaunay_crit}}
\caption{Delaunay condition (2)}
\label{fig:algo.delaunay_cond}
\end{figure}

This condition is illustrated figure \ref{fig:algo.delaunay_cond}.
In the first case, edge $\overline{P_i P_j}$ does not fulfill the
condition, as vertex $P_l$ is contained in the circle going through
$P_i, P_j, P_k$. In the second case, edge $\overline{P_k P_l}$ fulfills
this condition, as $P_i$ is not contained in the circle going through
$P_j, P_k, P_l$.

If triangles $(P_i, P_k, P_j)$ and $(P_i, P_j, P_l)$ originated from
the initial triangulation of a same polygon, they would thus be replaced
by triangles $(P_i, P_k, P_l)$ and $(P_l, P_k, P_j)$, which
fulfill the Delaunay condition. In the case of a quadrangle, we
would be done, but with a more complex initial polygon,
these new triangles could be replaced by others, depending on their
neighbors from the same polygon.

\begin{figure}[!h]
\centerline{
\includegraphics*[height=2.5cm]{face_split_delaunay_1}}
\caption{Edge flip example (1)}
\label{fig:algo.delaunay_ex_1}
\end{figure}

\begin{figure}[!h]
\centerline{
\includegraphics*[height=2cm]{face_split_delaunay_2}}
\caption{Edge flip example (2)}
\label{fig:algo.delaunay_ex_2}
\end{figure}

On figures \ref{fig:algo.delaunay_ex_1} and \ref{fig:algo.delaunay_ex_2},
we illustrate this algorithm on the same examples as before (figures
\ref{fig:algo.decoup_ex_1} and \ref{fig:algo.decoup_ex_2}). We see that
the final result is the same. In theory, the edge flipping algorithm
always converges. To avoid issues due to truncation errors, we allow
a tolerance before deciding to flip two edges. Thus, we allow that the
final triangulation only ``almost'' fulfill the Delaunay condition.

In some cases, especially that of perfect rectangles, two different
triangulations may both fulfill the Delaunay condition, notwithstanding
truncation errors. For periodicity, we must avoid having two periodic
faces triangulated in a non-periodic manner (for example, along
different diagonals of a quadrangle). In this specific case, we
triangulate only one face using the geometric algorithm, and then
apply the same triangulation to it's periodic face, rather then
triangulate both faces independently.

%-------------------------------------------------------------------------------
\section*{Unwarping algorithm\label{sec:unwarping}}
%-------------------------------------------------------------------------------

\hypertarget{unwarp}{}

The unwarping algorithm is a smoother, its
principle is to mitigate the local defects by averaging
the mesh quality. It moves vertices using an iterative
process which is expected to converge to a mesh with better averaged
warping criteria.

See the \doxygenanchor{cs__mesh__smoother_8c.html\#unwarp}{programmers
reference of the dedicated subroutine} for further details.

\section*{Warping criterion in \CS\label{sec:warping_criterion}}
The warp face quality criterion in \CS represents the non coplanarity
in space of $N$ points $P_{i=1:N}$ $(N > 3)$.

Let $f$ be the face defined by $P_{i=1:Nbv(f)}$, the center
of gravity $O_{f}$ and $\overrightarrow{n_{f}}$ the face normal, the warping
criterion is calculated for each face by the ``maximum'' angle between
$\overrightarrow{P_{i}P_{i+1}}$ and $\overrightarrow{n_{f}}^{\perp}$
$\forall i\in [1,Nbv(f)]$ where $P_{Nbv(f)+1} = P_{1}$. For consistency
purposes, the warping criterion $warp_{f}$ is defined in degree for each
face of the mesh $\mathcal{M}$ as:

$$\forall f \in \mathcal{M}, \qquad warp_{f} = 90 -
\arccos\left(\max_{\forall i\in [1,Nbv(f)]}\left(\cos(
\overrightarrow{P_{i}P_{i+1}},\overrightarrow{n_{f}})
\right)\right)\frac{180}{\pi}$$

\section*{Unwarping method\label{sec:unwarping_method}}
The principle of unwarping algorithm is to move vertices in the midplane of the
faces using an iterating process. At each iteration the algorithm tries to
approach vertices from the midplane of the face without increasing the warping
of the neighbours faces.

The displacement is calculated by projecting vertices onto
the plane defined by $\overrightarrow{n_{f}}$ and the center of gravity $O$.

For the face $f$, $\forall i \in [1,Nbv(f)]$, the vertices $P_{i}$ are displaced
by the vector $\lambda^{f}_{P_{i}}$
$$\forall i \in [1,Nbv(f)], \qquad \lambda_{P_{i}}^{f}
= (\overrightarrow{P_{i}O_{f}}.\overrightarrow{n_{f}})
\overrightarrow{n_{f}}$$

In most cases, a vertex is shared by many faces $f_{j=1:Nbf(P_{i})}$, and its final
displacement is:

$$\forall f \in \mathcal{M}, \; \forall i \in [1,Nbv(f)], \qquad
 \lambda_{P_{i}}^{f}=\sum_{j=1:Nbf(P_{i})}\lambda^{f_{j}}_{P_{i}}$$

This displacement technique may cause problems because the contributions
$\lambda^{f_{l}}_{P}$ and $\lambda^{f_{k}}_{P}$ can be ``contradictory''.
Moreover, if a small and a large face are neighbours, the large face contribution
imposes a too big displacement to the shared vertices and the warping criterion
can be deteriorated.

\subsection*{Displacements control\label{sec:unwarping_mvt}}
The weighting coefficients shown below allow us to reduce the conflicting
contributions and equilibrate the contributions between small and large faces.
\paragraph*{Face weighting}
Every iteration, for each face the warping criterion is computed.
It is used to give more weight to the warp faces. After the renormalisation of the
warping criterion, a new displacement formula is obtained:

$$\forall f \in \mathcal{M}, \; \forall i \in [1,Nbv(f)], \qquad
\lambda_{P_{i}}^{f}=\sum_{j=1:Nbf(P_{i})}
\frac{warp_{f_{j}}}{\max_{f \in \mathcal{M}} warp_{f}}\lambda^{f_{j}}_{P_{i}}$$

\paragraph*{Vertex weighting}
Every iteration, for each vertex $P \in \mathcal{M} $,
the vertex tolerance is computed:
$$\forall P \in \mathcal{M}, \qquad
vtxtol_{P} = \frac{\min_{P \in nb(P)} PP\prime}
{\max_{Q\in \mathcal{M}}{\min_{Q\prime\in nb(Q)} QQ\prime}}$$
where $nb(Q)$ are the first neighbors of the point $Q$.

This criterion is used to reduce the vertex displacement when a vertex lies
on an edge much smaller than the average length. Another vertex tolerance
may have been chosen. For example a more ``local coefficient'' can be
defined as:
$$\forall P \in \mathcal{M}, \qquad
vtxtol_{P}^{local} = \frac{\min_{P\prime \in nb(P)} PP\prime}
{\max_{P\prime \in nb(P)} PP\prime}$$
This ``local coefficient'' has been noticed to be less efficient than the first one.
That is why the first definition is used in \CS.

The unwarping displacement is updated as below:

$$\forall f \in \mathcal{M}, \; \forall i \in [1,Nbv(f)], \qquad
\lambda_{P_{i}}^{f}=vtxtol_{P_{i}}\sum_{j=1:Nbf(P_{i})}
\frac{warp_{f_{j}}}{\max_{f \in \mathcal{M}} warp_{f}}\lambda^{f_{j}}_{P_{i}}$$

\paragraph*{Movement coefficient}
To ensure a better convergence, all vertices' movements are multiplied by a scale
factor $Cm$ (where $Cm \leqslant 1$). This coefficient helps to converge to the
best solution by preventing a too big movement in the wrong direction.
In \CS the default value is set to $0.10$.

The unwarping displacement is then:

$$\forall f \in \mathcal{M}, \; \forall i \in [1,Nbv(f)], \qquad
\lambda_{P_{i}}^{f}=Cm*vtxtol_{P_{i}}\sum_{j=1:Nbf(P_{i})}
\frac{warp_{f_{j}}}{\max_{f \in \mathcal{M}} warp_{f}}\lambda^{f_{j}}_{P_{i}}$$

\paragraph*{Maximum displacement}
To reduce the ``cell reversal'' risk, the maximum displacement is limited for
each vertex $P \in \mathcal{M}$ to $Md = 0.10\,\min_{P \in nb(P)} PP\prime$.

Finally, the complete unwarping displacement formula is defined as:

\begin{equation*}
\begin{split}
\forall f \in \mathcal{M},& \; \forall i \in [1,Nbv(f)], \qquad \\
&\lambda_{P_{i}}^{f}=\min(Cm*vtxtol_{P_{i}}\sum_{j=1:Nbf(P_{i})}
\frac{warp_{f_{j}}}{\max_{f \in \mathcal{M}}
warp_{f}}\lambda^{f_{j}}_{P_{i}},\; Md)
\end{split}
\end{equation*}


\subsection*{Stop criterion\label{sec:unwarping_stop}}
The algorithm automatically stops according to the warp face criterion.

\paragraph*{$1^{st}$ case: the algorithm converges}
The algorithm stops when, at the $i^{th}$ iteration, the relation below is verified.

$$ 1 - \frac{\max_{f\in \mathcal{M}} warp_{f}^{i}}{\max_{f\in \mathcal{M}}
warp_{f}^{i-1}} < 1.E-4$$

\paragraph*{$2^{nd}$ case: the algorithm diverges}

The algorithm stops when at the $i^{th}$ iteration the relation below is verified.

$$\frac{\max_{f\in \mathcal{M}} warp_{f}^{i}}{\max_{f\in M}
warp_{f}^{i-1}} > 1.05$$

It means that the current iteration degrades the previous one.
The obtained mesh is the result of the $(i-1)^{th}$ iteration.

\paragraph*{$3^{rd}$ case: the maximum number of iterations is reached}
The algorithm stops after $N_{max}$ iterations ($51$ by default in \CS).

\subsection*{Specific treatment for boundary faces
\label{sec:unwarping_boundary}}

\hypertarget{fixbyfeature}{}

The unwarping algorithm may modify the mesh geometry. The function
\texttt{fix\_by\_feature} allows to fix boundary faces according to a feature angle.
The feature angle between a vertex and one of its adjacent faces is defined
by the angle between the vertex normal and the face normal.

A vertex normal is defined by the average of the normals of the
faces sharing this vertex.

This function fixes a vertex if one of its feature angles is less than
$cos(\theta)$ where $\theta$ is the maximum feature angle (in degrees)
defined by the user.
In fact, if $\theta = 0^{\circ}$ all boundary vertices will be fixed, and
if $\theta = 90^{\circ}$ all boundary vertices will be free.

Fixing all boundary vertices ensures the geometry is preserved, but reduces
the smoothing algorithm's effectiveness.

See the \doxygenanchor{cs__mesh__smoother_8c.html\#fix_by_feature}{programmers
reference of the dedicated subroutine} for further details.

%
%%%%%%%%%%%%%%%%%%%%%%%%%%%%%%%%%%%%%%%%%%%%%%%%%%%%%%%%%%%%%%%%%%%%%%
% FIN DU DOCUMENT
\end{document}
%
%%%%%%%%%%%%%%%%%%%%%%%%%%%%%%%%%%%%%%%%%%%%%%%%%%%%%%%%%%%%%%%%%%%%%%
