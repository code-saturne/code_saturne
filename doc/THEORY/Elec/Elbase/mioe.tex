%                      Code_Saturne version 1.3
%                      ------------------------
%
%     This file is part of the Code_Saturne Kernel, element of the
%     Code_Saturne CFD tool.
%
%     Copyright (C) 1998-2007 EDF S.A., France
%
%     contact: saturne-support@edf.fr
%
%     The Code_Saturne Kernel is free software; you can redistribute it
%     and/or modify it under the terms of the GNU General Public License
%     as published by the Free Software Foundation; either version 2 of
%     the License, or (at your option) any later version.
%
%     The Code_Saturne Kernel is distributed in the hope that it will be
%     useful, but WITHOUT ANY WARRANTY; without even the implied warranty
%     of MERCHANTABILITY or FITNESS FOR A PARTICULAR PURPOSE.  See the
%     GNU General Public License for more details.
%
%     You should have received a copy of the GNU General Public License
%     along with the Code_Saturne Kernel; if not, write to the
%     Free Software Foundation, Inc.,
%     51 Franklin St, Fifth Floor,
%     Boston, MA  02110-1301  USA
%
%-----------------------------------------------------------------------
%
%%%%%%%%%%%%%%%%%%%%%%%%%%%%%%%%%%
%%%%%%%%%%%%%%%%%%%%%%%%%%%%%%%%%%
\section{Mise en \oe uvre}
%%%%%%%%%%%%%%%%%%%%%%%%%%%%%%%%%%
%%%%%%%%%%%%%%%%%%%%%%%%%%%%%%%%%%


\subsection{Introduction}

Le module \'electrique est une ``physique particuli\`ere'' activ\'ee lorsque les
mots-cl\'es \var{IPPMOD(IELARC)} (arc \'electrique) ou \var{IPPMOD(IELJOU)}
(Joule) sont strictement positifs. Les d\'eveloppements concernant la conduction ionique
(mot-cl\'e \var{IPPMOD(IELION)}) ont \'et\'e pr\'evus dans le code mais restent \`a
r\'ealiser. Pour l'arc \'electrique, dans la version actuelle
de \CS, seule est op\'erationnelle l'option \var{IPPMOD(IELARC)}=2~: la version 2D axisym\'etrique qui permettrait de
        s'affranchir du potentiel vecteur (option \var{IPPMOD(IELARC)}=1) n'est pas
        activable.
Pour l'effet Joule, lorsqu'il n'est pas utile d'introduire un potentiel scalaire
complexe
(en courant continu ou alternatif monophas\'e), on utilise
\var{IPPMOD(IELJOU)}=1. Lorsqu'un potentiel scalaire complexe est indispensable (courant
alternatif triphas\'e, par exemple), on utilise
\var{IPPMOD(IELJOU)}=2.

Dans ce qui suit, on pr\'ecise les inconnues et les propri\'et\'es
principales utilis\'ees dans le module.
On fournit \'egalement un arbre d'appel simplifi\'e des sous-programmes du module
(initialisation avec \fort{initi1} puis \fort{inivar} et boucle en temps avec \fort{tridim}).
Les sous-programmes marqu\'es d'un ast\'erisque sont d\'etaill\'es ensuite.

\newpage

\subsection{Inconnues et propri\'et\'es}

Les d\'eveloppements ont \'et\'e r\'ealis\'es pour une unique phase (\var{NPHAS}=1).

Les \var{NSCAPP} inconnues scalaires associ\'ees \`a la physique
particuli\`ere sont d\'efinies dans \fort{elvarp} dans l'ordre
suivant (en particulier afin de limiter le stockage en m\'emoire lors de
la r\'esolution s\'equentielle des
scalaires par \fort{scalai})~:
\begin{itemize}
\item l'enthalpie \var{RTP(*,ISCA(IHM))},
\item un potentiel scalaire r\'eel \var{RTP(*,ISCA(IPOTR))},
\item un potentiel scalaire imaginaire \var{RTP(*,ISCA(IPOTI))} {\it ssi}
          \var{IPPMOD(IELJOU)}=2 (\'etudes Joule en courant alternatif non monophas\'e),
\item les trois composantes d'un potentiel vecteur r\'eel
          \var{RTP(*,ISCA(IPOTVA(i)))} (avec \var{i} variant de 1 \`a 3) {\it ssi}
        \var{IPPMOD(IELARC)}=2 (arc \'electrique),
\item \var{NGAZG}-1 fractions massiques \var{RTP(*,ISCA(IYCOEL(j)))}
        (avec \var{j} variant de 1 \`a \var{NGAZG}-1) pour un fluide \`a \var{NGAZG}
        constituants (avec \var{NGAZG} strictement sup\'erieur \`a 1).
        En arc \'electrique, la composition est fournie dans le fichier de donn\'ees
        \fort{dp\_ELE}. La fraction massique du
        dernier constituant n'est pas stock\'ee en m\'emoire. Elle est
        d\'etermin\'ee chaque fois que n\'ecessaire en calculant le compl\'ement \`a l'unit\'e
        des autres fractions massiques (et, en particulier, lorsque \fort{elthht} est
        utilis\'e pour le calcul des propri\'et\'es physiques).
\end{itemize}

\bigskip
Outre les propri\'et\'es associ\'ees en standard aux variables scalaires
identifi\'ees ci-dessus, le
tableau \var{PROPCE} contient \'egalement~:
 \begin{itemize}
\item la temp\'erature, \var{PROPCE(*,IPPROC(ITEMP))}. En th\'eorie, on
        pourrait \'eviter de stocker cette variable, mais l'utilisateur est
        presque toujours int\'eress\'e par sa valeur en post-traitement et
        les propri\'et\'es physiques sont souvent donn\'ees par des lois qui en
        d\'ependent explicitement.
        Son unit\'e (Kelvin ou Celsius) d\'epend des tables
        enthalpie-temp\'erature fournies par l'utilisateur.
\item la puissance \'electromagn\'etique dissip\'ee par effet Joule,
        \var{PROPCE(*,IPPROC(IEFJOU))} (terme source positif pour l'enthalpie),
\item les trois composantes des forces de Laplace,
        \var{PROPCE(*,IPPROC(ILAPLA(i)))} (avec \var{i} variant de 1 \`a 3)
        en arc \'electrique (\var{IPPMOD(IELARC)}=2).
\end{itemize}

\bigskip
La conductivit\'e \'electrique est {\it a priori} variable et
conserv\'ee dans le tableau de propri\'et\'es aux cellules
\var{PROPCE(*,IPPROC(IVISLS(IPOTR)))}. Elle intervient dans l'\'equation de
Poisson portant sur le potentiel scalaire. Lorsque le potentiel scalaire a une
partie imaginaire, la conductivit\'e n'est pas dupliqu\'ee~:
les entiers \var{IPPROC(IVISLS(IPOTI))} et \var{IPPROC(IVISLS(IPOTR))} pointent sur la
m\^eme case du tableau \var{PROPCE}. La conductivit\'e associ\'ee au potentiel
vecteur est uniforme et de valeur unit\'e (\var{VISLS0(IPOTVA(i))}=1.D0
avec \var{i} variant de 1 \`a 3).

Le champ \'electrique, la densit\'e de courant et le champ magn\'etique ne sont
stock\'es que de mani\`ere temporaire (voir \fort{elflux}).


\newpage

\subsection{Arbre d'appel simplifi\'e}

\begin{table}[htp]
\begin{center}
\begin{tabular}{llllp{10cm}}
\fort{usini1}         &                 &                &
        & Initialisation des mots-cl\'es utilisateur g\'en\'eraux et positionnement des variables\\
                &\fort{usppmo}         &                &
        & D\'efinition du module ``physique particuli\`ere'' employ\'e\\
                &\fort{varpos}         &                &
        & Positionnement des variables \\
                &                 & \fort{pplecd} &
        & Branchement des physiques particuli\`eres pour la lecture de fichier de donn\'ees \\
                &                 &                 & \fort{ellecd}*
        & Lecture du fichier de donn\'ees pour les arcs \'electriques  \fort{dp\_ELE} \\
                &                 & \fort{ppvarp} &
        & Branchement des physiques particuli\`eres pour le positionnement des inconnues \\
                &                 &                 & \fort{elvarp}*
        & Positionnement des inconnues (enthalpie, potentiels, fractions massiques) \\
                &                 & \fort{ppprop} &
        & Branchement des physiques particuli\`eres pour le positionnement des propri\'et\'es\\
                &                 &                 & \fort{elprop}*
        & Positionnement des propri\'et\'es (temp\'erature, effet Joule, forces de Laplace) \\
%
\fort{ppini1}         &                &                &
        & Branchement des physiques particuli\`eres pour l'initialisation des
mots-cl\'es sp\'ecifiques \\
                &\fort{elini1}         &                &
        & Initialisation des mots-cl\'es pour le module \'electrique\\
                &\fort{useli1}         &                &
        & Initialisation des mots-cl\'es utilisateur pour le module \'electrique\\
                &\fort{elveri}         &                &
        & V\'erification des mots-cl\'es pour le module \'electrique\\
\end{tabular}
\caption{Sous-programme \fort{initi1}~: initialisation des mots-cl\'es et
positionnement des variables}
\end{center}
\end{table}


\begin{table}[htp]
\begin{center}
\begin{tabular}{llllp{10cm}}
\fort{ppiniv}         &                &                &
        & Branchement des physiques particuli\`eres pour l'initialisation des variables \\
                & \fort{eliniv}*&                &
        & Initialisation des variables sp\'ecifiques au module \'electrique \\
                 &                 & \fort{elthht}*&
        & Transformation temp\'erature-enthalpie et enthalpie-temp\'erature par
                interpolation sur la base du fichier de donn\'ees \fort{dp\_ELE}
                (arc \'electrique uniquement) \\
                 &                 & \fort{useliv} &
        & Initialisation des variables par l'utilisateur  \\
                 &                 &                 & \fort{elthht}*
        & Transformation temp\'erature-enthalpie et enthalpie-temp\'erature par
                interpolation sur la base du fichier de donn\'ees \fort{dp\_ELE}
                (arc \'electrique uniquement) \\
\end{tabular}
\caption{Sous-programme \fort{inivar}~: initialisation des variables}
\end{center}
\end{table}


\begin{table}[htp]
\begin{center}
\begin{tabular}{llllp{10cm}}
\fort{phyvar}         &                &                &
        & Calcul des propri\'et\'es physiques variables \\
                & \fort{ppphyv} &                &
        & Branchement des physiques particuli\`eres pour le calcul des
                propri\'et\'es physiques variables \\
                & \fort{elphyv} &                &
        & Calcul des propri\'et\'es physiques variables pour le module
                \'electrique. En arc \'electrique, les propri\'et\'es sont
                calcul\'ees par interpolation \`a partir des tables fournies
                dans le fichier de donn\'ees \fort{dp\_ELE}\\
                 &                 & \fort{elthht}*&
        & Transformation temp\'erature-enthalpie et enthalpie-temp\'erature par
                interpolation sur la base du fichier de donn\'ees \fort{dp\_ELE}
                (arc \'electrique uniquement) \\
                 &                 & \fort{uselph} &
        & Calcul par l'utilisateur des propri\'et\'es physiques variables pour le module
                \'electrique. Pour les \'etudes Joule, en particulier, les propri\'et\'es
                doivent \^etre fournies ici sous forme de loi (des exemples sont
                disponibles)\\
                 &                 &                 & \fort{usthht}
        & Transformation temp\'erature-enthalpie et enthalpie-temp\'erature
                fournie par l'utilisateur (plus sp\'ecifiquement pour les
                \'etudes Joule, pour lesquelles on ne dispose pas d'un fichier
                de donn\'ees \`a partir duquel r\'ealiser des interpolations avec \fort{elthht}) \\
\end{tabular}
\caption{Sous-programme \fort{tridim}~: partie 1 (propri\'et\'es physiques)}
\end{center}
\end{table}

\begin{table}[htp]
\begin{center}
\begin{tabular}{llllp{10cm}}
\fort{ppclim}         &                  &                &
        & Branchement des physiques particuli\`eres pour les conditions aux limites\\
                & \fort{uselcl} &                &
        & Intervention de l'utilisateur pour les conditions aux limites (en lieu
                et place de \fort{usclim}, m\^eme pour les variables qui ne sont
                pas sp\'ecifiques au module \'electrique). Si un recalage
                automatique des potentiels est demand\'e (\var{IELCOR=1}), il
                doit \^etre pris en compte ici par le biais des variables
                \var{DPOT} ou \var{COEJOU} (voir la description des
                conditions aux limites).   \\
\fort{navsto}         &                  &                &
        & R\'esolution des \'equations de Navier-Stokes\\
                & \fort{preduv} &                &
        & Pr\'ediction de la vitesse~: prise en compte des forces de Laplace
                calcul\'ees dans \fort{elflux} au pas de temps pr\'ec\'edent\\
\fort{``turb''} &                  &                &
        & Turbulence : r\'esolution des \'equations pour les mod\`eles
                n\'ecessitant des \'equations de convection-diffusion\\
\fort{scalai}*         &                  &                &
        & R\'esolution des \'equations portant sur les scalaires associ\'es aux
                physiques particuli\`eres et des scalaires ``utilisateur''  \\
                & \fort{covofi}         &                &
        & R\'esolution successive de l'enthalpie, du potentiel scalaire
                r\'eel et, si \var{IPPMOD(IELJOU)=2}, de la partie imaginaire du
                potentiel scalaire (appels successifs \`a \fort{covofi} qui appelle
                \fort{eltssc} pour le calcul du terme d'effet Joule au second
                membre de l'\'equation de l'enthalpie)\\
                & \fort{elflux}* &                &
        & Calcul du champ \'electrique, de la densit\'e de courant et de l'effet
                Joule (premier de deux appels au cours du pas de temps courant) \\
                & \fort{uselrc}* &                &
        & Recalage automatique \'eventuel
                de
                la densit\'e de courant, de l'effet Joule, des potentiels et
                des coefficients \var{DPOT} et \var{COEJOU}.
                Ce recalage, s'il a \'et\'e demand\'e
                par l'utilisateur (\var{IELCOR}=1), est effectu\'e \`a partir
                du deuxi\`eme pas de temps. \\
                & \fort{covofi}         &                &
        & R\'esolution successive, si \var{IPPMOD(IELARC)=2}, des trois
                composantes du potentiel vecteur. On proc\`ede par
                appels successifs \`a \fort{covofi} qui appelle
                \fort{eltssc} pour le calcul du second membre de l'\'equation de
                Poisson portant sur chaque composante du potentiel. \\
                & \fort{covofi}         &                &
        & R\'esolution successive des \var{NGAZG-1} fractions massiques
                caract\'erisant la composition du fluide, s'il est
                multiconstituant.
                On proc\`ede par appels successifs \`a \fort{covofi}. \\
                & \fort{elflux}* &                &
        & En arc \'electrique, calcul du champ magn\'etique et
                des trois composantes des forces de
                Laplace (deuxi\`eme et dernier appel au cours du pas de temps courant)\\
                & \fort{covofi}         &                &
        & R\'esolution des scalaires ``utilisateur''\\
\end{tabular}
\caption{Sous-programme \fort{tridim}~: partie 2 (conditions aux limites,
Navier-Stokes, turbulence et scalaires)}
\end{center}
\end{table}

\newpage

\begin{table}[htp]
\begin{center}
\begin{tabular}{llllp{10cm}}
\fort{postlc}         &                  &                &
        & Post-traitement\\
                & \fort{ecrevo}         &                &
        & \'Ecriture des variables \`a post-traiter\\
                &                  & \fort{uselen}&
        & Ajout au post-traitement de variables calcul\'ees par
                l'utilisateur. En exemple activ\'e standard sont post-trait\'es,
                s'ils existent, l'oppos\'e du champ \'electrique ({\it i.e.} le
                gradient du potentiel scalaire, r\'eel ou complexe),  le vecteur
                densit\'e de courant imaginaire (en effet Joule), le champ
                magn\'etique (en arc \'electrique) et enfin le module et l'argument
                du potentiel (en effet Joule, avec \var{IPPMOD(IELJOU)=4})\\
\end{tabular}
\caption{Sous-programme \fort{tridim}~: partie 3 (post-traitement)}
\end{center}
\end{table}

\newpage

\subsection{Pr\'ecisions}

\etape{\fort{ellecd}}

Ce sous-programme r\'ealise la lecture du fichier de donn\'ees sp\'ecifique
aux arcs \'electriques. On donne ci-dessous, \`a titre d'exemple, l'ent\^ete
explicative et deux lignes de donn\'ees d'un fichier type. Ces valeurs sont interpol\'ees chaque
fois que n\'ecessaire par \fort{elthht} pour d\'eterminer les propri\'et\'es
physiques du fluide \`a une temp\'erature (une enthalpie) donn\'ee.

{\scriptsize
\begin{verbatim}
# Nb d'especes NGAZG et Nb de points NPO (le fichier contient NGAZG blocs de NPO lignes chacun)
# NGAZG NPO
   1   238
#
#  Proprietes
#      T           H          ROEL       CPEL           SIGEL        VISEL        XLABEL        XKABEL
#  Temperature  Enthalpie   Masse vol. Chaleur       Conductivite  Viscosite   Conductivite   Coefficient
#                           volumique  massique       electrique   dynamique     thermique   d'absorption
#       K         J/kg         kg/m3     J/(kg K)       Ohm/m        kg/(m s)     W/(m K)         -
#
   300.00       14000.       1.6225       520.33      0.13214E-03  0.34224E-04  0.26712E-01   0.0000
   400.00       65800.       1.2169       520.33      0.13214E-03  0.34224E-04  0.26712E-01   0.0000
\end{verbatim}
}


\etape{\fort{elvarp}}

Ce sous-programme permet de positionner les inconnues de calcul list\'ees
pr\'ec\'edemment. On y pr\'ecise \'egalement que la chaleur massique \`a
pression constante est variable, ainsi que la conductivit\'e de tous les
scalaires associ\'es au module \'electrique, hormis la conductivit\'e de
l'\'eventuel potentiel vecteur (celle-ci est uniforme et de valeur unit\'e).


\etape{\fort{elprop}}

C'est dans ce sous-programme que sont positionn\'ees les propri\'et\'es stock\'ees
dans le tableau \var{PROPCE}, et en particulier la temp\'erature, l'effet Joule
et les forces de Laplace.

\etape{\fort{eliniv}}

Ce sous-programme permet de r\'ealiser les initialisations par d\'efaut
sp\'ecifiques au module.

En particulier, en $k-\varepsilon$, les deux variables
turbulentes sont initialis\'ees \`a $10^{-10}$ (choix historique arbitraire,
mais r\'eput\'e, lors de tests non r\'ef\'erenc\'es, permettre le d\'emarrage de
certains calculs qui \'echouaient avec une initialisation classique).

Les potentiels sont initialis\'es \`a z\'ero, de m\^eme que l'effet Joule. En
arc \'electrique, les forces de Laplace sont initialis\'ees \`a z\'ero.

Le fluide est suppos\'e monoconstituant (seule est pr\'esente la premi\`ere
esp\`ece).

En arc \'electrique, l'enthalpie est initialis\'ee \`a la valeur de l'enthalpie du m\'elange
suppos\'e monoconstituant \`a la temp\'erature \var{T0(IPHAS)} donn\'ee
dans \fort{usini1}.  En effet Joule, l'enthalpie est initialis\'ee \`a z\'ero
(mais l'utilisateur peut fournir une valeur diff\'erente dans \fort{useliv}).

\etape{\fort{elthht}}

Ce sous-programme permet de r\'ealiser (en arc \'electrique) les interpolations
n\'ecessaires \`a la d\'e\-ter\-mi\-na\-tion des propri\'et\'es physiques du fluide, \`a
partir des tables fournies dans le fichier de donn\'ees \fort{dp\_ELE}.

On notera en particulier que ce sous-programme prend en argument le tableau
\var{YESP(NESP)} qui repr\'esente la fraction massique des \var{NGAZG}
constituants du fluide. Dans le code, on ne r\'esout que la fraction massique
des \var{NGAZG}-1 premiers constituants. Avant chaque appel \`a \fort{elthht},
la fraction massique du dernier constituant doit \^etre calcul\'ee comme le
compl\'ement \`a l'unit\'e des autres fractions massiques.

\etape{\fort{scalai}, \fort{elflux}, \fort{uselrc}}

Le sous-programme \fort{scalai} permet de calculer, dans l'ordre souhait\'e,
les \var{NSCAPP} scalaires ``physique particuli\`ere'' associ\'es au module
\'electrique, puis de calculer les grandeurs interm\'ediaires n\'e\-ces\-sai\-res et
enfin de
r\'ealiser les op\'erations qui permettent d'assurer le recalage automatique
des potentiels, lorsqu'il est requis par l'utilisateur ({\it i.e.} si \var{IELCOR=1}).

Les \var{NSCAPP} scalaires ``physique particuli\`ere'' sont calcul\'es successivement par un
appel \`a \fort{covofi} plac\'e dans une boucle portant sur les \var{NSCAPP}
scalaires. L'algorithme tire profit de l'ordre sp\'ecifique dans lequel ils sont d\'efinis et donc
r\'esolus (dans l'ordre~: enthalpie, potentiel scalaire, potentiel vecteur, fractions massiques).

Pour \'eviter des variations trop brutales en d\'ebut de calcul, le terme source
d'effet Joule n'est pris en compte dans l'\'equation de l'enthalpie qu'\`a
partir du troisi\`eme pas de temps.

Apr\`es la r\'esolution de l'enthalpie et du potentiel scalaire
(r\'eel ou complexe), le sous-programme \fort{elflux} permet de calculer
les trois composantes du champ \'electrique
(que l'on stocke dans des tableaux de travail), puis la densit\'e de courant
et enfin l'effet Joule, que l'on conserve dans le tableau \var{PROPCE(*,IPPROC(IEFJOU))}
pour le pas temps suivant (apr\`es recalage \'eventuel dans \fort{uselrc} comme
indiqu\'e ci-apr\`es).
Lorsque \var{IPPMOD(IELJOU)=2},  l'apport de la partie imaginaire est pris en
compte pour le calcul de l'effet Joule. Lorsque \var{IPPMOD(IELARC)=2}
(arc \'electrique), le vecteur densit\'e de courant est
conserv\'e dans \var{PROPCE}, en lieu et place des forces de Laplace
\var{PROPCE(*,IPPROC(ILAPLA(i)))}~: il est utilis\'e pour le calcul du potentiel vecteur dans le
second appel \`a \fort{elflux},
apr\`es recalage \'eventuel par \fort{uselrc} (en effet, il n'est plus
n\'ecessaire de conserver les forces de Laplace \`a ce stade puisque
la seule \'equation dans laquelle elles interviennent est l'\'equation de la
quantit\'e de mouvement et qu'elle a d\'ej\`a \'et\'e r\'esolue).

\`A la suite de \fort{elflux},
le sous-programme \fort{uselrc} effectue le recalage permettant d'adapter automatiquement
les conditions aux limites portant sur les potentiels, si l'utilisateur l'a
demand\'e ({\it i.e.} si \var{IELCOR=1}).
On se reportera au paragraphe relatif aux conditions aux limites. On pr\'ecise
ici que le coefficient de recalage \var{COEPOT} permet d'adapter l'effet Joule
\var{PROPCE(*,IPPROC(IEFJOU))} et la diff\'erence de potentiel \var{DPOT}
(utile pour les conditions aux limites portant sur le potentiel scalaire au pas
de temps suivant\footnote{{\it A priori}, \var{DPOT} n'est pas n\'ecessaire pour les
cas Joule.}).
Pour les cas d'arc \'electrique, \var{COEPOT} permet \'egalement de
recaler le vecteur densit\'e de courant que l'on vient
de stocker temporairement dans \var{PROPCE(*,IPPROC(ILAPLA(i)))} et qui va
servir imm\'ediatement \`a calculer le potentiel vecteur.
Pour les cas Joule,  on recale en outre le coefficient \var{COEJOU} (utile
pour les conditions aux limites portant sur le potentiel scalaire au pas de
temps suivant).

Pour les cas d'arc \'electrique (\var{IPPMOD(IELARC)=2}),
apr\`es \fort{elflux} et \fort{uselrc},
la r\'esolution s\'equentielle des inconnues scalaires se poursuit dans
\fort{scalai} avec le calcul des trois composantes du potentiel vecteur. Le second membre de
l'\'equation de Poisson consid\'er\'ee d\'epend de la densit\'e de courant qui,
dans \fort{elflux}, a \'et\'e temporairement stock\'ee dans le tableau
\var{PROPCE(*,IPPROC(ILAPLA(i)))} et qui,  dans \fort{uselrc}, vient d'\^etre
recal\'ee si \var{IELCOR=1}.
Les valeurs du potentiel vecteur obtenues int\`egrent donc  naturellement l'\'eventuel
recalage.

Pour les cas d'arc \'electrique (\var{IPPMOD(IELARC)=2}), un second appel \`a
\fort{elflux} permet alors de calculer le champ magn\'etique
que l'on stocke dans des tableaux de travail et les forces de Laplace que l'on stocke dans
\var{PROPCE(*,IPPROC(ILAPLA(i)))} pour le pas de temps suivant (la densit\'e de
courant, que l'on avait temporairement conserv\'ee dans ce tableau, ne servait
qu'\`a calculer le second membre de l'\'equation de Poisson portant sur le
potentiel vecteur~: il n'est donc plus n\'ecessaire de la conserver).

La r\'esolution s\'equentielle des inconnues scalaires sp\'ecifiques au module se poursuit dans
\fort{scalai}, avec le calcul des \var{NGAZG}-1 fractions massiques permettant
de d\'efinir la composition du fluide.

Pour terminer, \fort{scalai} permet la r\'esolution des scalaires
``utilisateurs'' (appel \`a \fort{covofi} dans une boucle portant sur les
\var{NSCAUS} scalaires utilisateur).

On peut remarquer pour finir que les termes sources des \'equations de la quantit\'e de
mouvement (forces de Laplace) et de l'enthalpie (effet Joule) sont disponibles
\`a la fin du pas de temps $n$ pour une utilisation au pas de temps $n+1$ (de ce
fait, pour permettre les reprises de calcul, ces termes sources sont stock\'es dans le fichier suite auxiliaire, ainsi que \var{DPOT}
et \var{COEJOU}).

\newpage
%%%%%%%%%%%%%%%%%%%%%%%%%%%%%%%%%%
%%%%%%%%%%%%%%%%%%%%%%%%%%%%%%%%%%
\section{Points \`a traiter}
%%%%%%%%%%%%%%%%%%%%%%%%%%%%%%%%%%
%%%%%%%%%%%%%%%%%%%%%%%%%%%%%%%%%%

\etape{Mobilit\'e ionique} Le module est \`a d\'evelopper.

\etape{Conditions aux limite en Joule} La prise en compte de conditions aux
limites coupl\'ees entre \'electrodes reste  \`a faire.

\etape{Compressible en arc \'electrique} Les  d\'eveloppements du module
compressible de \CS\ doivent \^etre rendus compatibles avec le module arc \'electrique.
