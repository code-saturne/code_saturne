%                      Code_Saturne version 1.3
%                      ------------------------
%
%     This file is part of the Code_Saturne Kernel, element of the
%     Code_Saturne CFD tool.
% 
%     Copyright (C) 1998-2007 EDF S.A., France
%
%     contact: saturne-support@edf.fr
% 
%     The Code_Saturne Kernel is free software; you can redistribute it
%     and/or modify it under the terms of the GNU General Public License
%     as published by the Free Software Foundation; either version 2 of
%     the License, or (at your option) any later version.
% 
%     The Code_Saturne Kernel is distributed in the hope that it will be
%     useful, but WITHOUT ANY WARRANTY; without even the implied warranty
%     of MERCHANTABILITY or FITNESS FOR A PARTICULAR PURPOSE.  See the
%     GNU General Public License for more details.
% 
%     You should have received a copy of the GNU General Public License
%     along with the Code_Saturne Kernel; if not, write to the
%     Free Software Foundation, Inc.,
%     51 Franklin St, Fifth Floor,
%     Boston, MA  02110-1301  USA
%
%-----------------------------------------------------------------------
%
\programme{cfener}
%
\vspace{1cm}
%%%%%%%%%%%%%%%%%%%%%%%%%%%%%%%%%%
%%%%%%%%%%%%%%%%%%%%%%%%%%%%%%%%%%
\section{Fonction}
%%%%%%%%%%%%%%%%%%%%%%%%%%%%%%%%%%
%%%%%%%%%%%%%%%%%%%%%%%%%%%%%%%%%%

Pour les notations et l'algorithme dans son ensemble, 
on se reportera \`a \fort{cfbase}. 

Apr\`es masse (acoustique) et quantit\'e de mouvement, 
on consid\`ere un dernier pas fractionnaire (de $t^{**}$ \`a $t^{***}$)
au cours duquel seule varie l'\'energie totale $E = \rho e$. 

\begin{equation}\label{Cfbl_Cfener_eq_energie_cfener}
\left\{\begin{array}{l}
\rho^{***}=\rho^{**}=\rho^{n+1}\\
\\
\vect{Q}^{***}=\vect{Q}^{**}=\vect{Q}^{n+1}\\
\\
\displaystyle\frac{\partial \rho e}{\partial t}
+ \divs\left( \vect{Q}_{ac} \left(e+\displaystyle\frac{P}{\rho}\right) \right)
= \rho\vect{f}_v\cdot\vect{u}
+ \divs(\tens{\Sigma}^v \vect{u})
- \divs{\,\vect{\Phi}_s} + \rho\Phi_v
\end{array}\right.
\end{equation}
 
Pour conserver la positivit\'e de l'\'energie, il est indispensable ici, 
comme pour les scalaires, d'utiliser le flux de masse convectif acoustique 
$\vect{Q}_{ac}^{n+1}$ compatible avec l'\'equation de la masse. 
De plus, pour obtenir des propri\'et\'es de positivit\'e sur les scalaires,
un sch\'ema upwind pour le terme convectif doit \^etre utilis\'e
(mais les termes sources introduisent des contraintes suppl\'ementaires
qui peuvent \^etre pr\'epond\'erantes et g\^enantes).

\vspace{0.5cm}

\`A la fin de cette \'etape, on actualise �ventuellement 
(mais par d�faut non)
une deuxi\`eme et derni\`ere fois la pression
en utilisant la loi d'\'etat pour obtenir la pression finale~:
\begin{equation}
\displaystyle P^{n+1}=P(\rho^{n+1},\varepsilon^{n+1})
\end{equation}