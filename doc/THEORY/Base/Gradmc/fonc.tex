%                      Code_Saturne version 1.3
%                      ------------------------
%
%     This file is part of the Code_Saturne Kernel, element of the
%     Code_Saturne CFD tool.
% 
%     Copyright (C) 1998-2007 EDF S.A., France
%
%     contact: saturne-support@edf.fr
% 
%     The Code_Saturne Kernel is free software; you can redistribute it
%     and/or modify it under the terms of the GNU General Public License
%     as published by the Free Software Foundation; either version 2 of
%     the License, or (at your option) any later version.
% 
%     The Code_Saturne Kernel is distributed in the hope that it will be
%     useful, but WITHOUT ANY WARRANTY; without even the implied warranty
%     of MERCHANTABILITY or FITNESS FOR A PARTICULAR PURPOSE.  See the
%     GNU General Public License for more details.
% 
%     You should have received a copy of the GNU General Public License
%     along with the Code_Saturne Kernel; if not, write to the
%     Free Software Foundation, Inc.,
%     51 Franklin St, Fifth Floor,
%     Boston, MA  02110-1301  USA
%
%-----------------------------------------------------------------------
%

\programme{gradmc}

\vspace{1cm}
%%%%%%%%%%%%%%%%%%%%%%%%%%%%%%%%%%
%%%%%%%%%%%%%%%%%%%%%%%%%%%%%%%%%%
\section{Fonction}
%%%%%%%%%%%%%%%%%%%%%%%%%%%%%%%%%%
%%%%%%%%%%%%%%%%%%%%%%%%%%%%%%%%%%
Le but de ce sous-programme est de calculer, au centre des cellules, le gradient
d'une fonction scalaire, \'egalement connue au centre des cellules. 
Pour obtenir la valeur de toutes les composantes du gradient, une m\'ethode de
minimisation par moindres carr\'es est mise en  
\oe uvre~: elle utilise l'estimation d'une composante du gradient aux faces, 
obtenue \`a partir des
valeurs de la fonction au centre des cellules voisines. Cette m\'ethode est 
activ\'ee lorsque l'indicateur IMRGRA vaut~1 et on l'utilise alors pour le calcul
des gradients de toutes les grandeurs. Elle est beaucoup plus rapide que la m\'ethode
utilis\'ee par d\'efaut (\var{IMRGRA}=0), mais pr\'esente l'inconv\'enient
d'\^etre moins robuste
sur des maillages non orthogonaux, le gradient produit \'etant moins r\'egulier.
% (bien que de m\^eme ordre en espace) J'en sais rien dans l'absolu !. 


