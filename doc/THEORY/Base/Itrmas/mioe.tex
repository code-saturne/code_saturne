%                      Code_Saturne version 1.3
%                      ------------------------
%
%     This file is part of the Code_Saturne Kernel, element of the
%     Code_Saturne CFD tool.
% 
%     Copyright (C) 1998-2007 EDF S.A., France
%
%     contact: saturne-support@edf.fr
% 
%     The Code_Saturne Kernel is free software; you can redistribute it
%     and/or modify it under the terms of the GNU General Public License
%     as published by the Free Software Foundation; either version 2 of
%     the License, or (at your option) any later version.
% 
%     The Code_Saturne Kernel is distributed in the hope that it will be
%     useful, but WITHOUT ANY WARRANTY; without even the implied warranty
%     of MERCHANTABILITY or FITNESS FOR A PARTICULAR PURPOSE.  See the
%     GNU General Public License for more details.
% 
%     You should have received a copy of the GNU General Public License
%     along with the Code_Saturne Kernel; if not, write to the
%     Free Software Foundation, Inc.,
%     51 Franklin St, Fifth Floor,
%     Boston, MA  02110-1301  USA
%
%-----------------------------------------------------------------------
%

%%%%%%%%%%%%%%%%%%%%%%%%%%%%%%%%%%
%%%%%%%%%%%%%%%%%%%%%%%%%%%%%%%%%%
\section{Mise en \oe uvre}
%%%%%%%%%%%%%%%%%%%%%%%%%%%%%%%%%%
%%%%%%%%%%%%%%%%%%%%%%%%%%%%%%%%%%
Les principaux arguments pass\'es \`a \fort{itrmas} et \fort{itrgrp} sont la
variable trait\'ee \var{PVAR} (la pression), ses conditions aux limites, le pas 
de temps projet\'e aux faces\footnote{%
Plus pr\'ecis\'ement, le pas de temps projet\'e aux faces, multipli\'e par la
surface et divis\'e par $\overline{I^\prime J^\prime}$ ou $\overline{I^\prime F}$, cf. \fort{viscfa}}
(\var{VISCF} et \var{VISCB}), le pas de temps au
centre des cellules, \'eventuellement anisotrope (\var{VISELX}, \var{VISELY},
\var{VISELZ}). \fort{itrmas} retourne les tableaux \var{FLUMAS} et \var{FLUMAB}
(flux de masse aux faces) mis \`a jour. \fort{itrgrp} retourne directement la
divergence du flux de masse mis \`a jour, dans le tableau \var{DIVERG}.

\etape{Initialisation}
Si \var{INIT} vaut 1, les variables \var{FLUMAS} et \var{FLUMAB} ou \var{DIVERG} 
sont mises \`a z\'ero.

\etape{Cas sans reconstruction}
La prise en compte ou non des non orthogonalit\'es est d\'etermin\'ee par
l'indicateur \var{NSWRGR} de la variable trait\'ee (nombre de sweeps de
reconstruction des non orthogonalit\'es dans le calcul des gradients), pass\'e 
par la variable \var{NSWRGP}. Une valeur inf\'erieure ou \'egale \`a 1 enclenche
le traitement sans reconstruction.\\
Des boucles sur les faces internes et les faces de bord utilisent directement
les formules (\ref{Base_Itrmas_eq_intssrec}) et (\ref{Base_Itrmas_eq_brdssrec}) pour remplir les
tableaux \var{FLUMAS} et \var{FLUMAB} (\fort{itrmas}) ou des variables de
travail \var{FLUMAS} et \var{FLUMAB} qui servent \`a mettre \`a jour le tableau
\var{DIVERG} (\fort{itrgrp}).

\`A noter que les tableaux \var{VISCF} et \var{VISCB} contiennent respectivement 
$\displaystyle\frac{\Delta t_{\,ij}S_{\,ij}}{\overline{I^\prime J^\prime}}$ et
$\displaystyle\frac{\Delta t_{\,b_{ik}}S_{\,b_{ik}}}{\overline{I^\prime F}}$.

\etape{Cas avec reconstruction}
Apr\`es un appel \`a \fort{GRDCEL} pour calculer le gradient cellule de
pression, on remplit les tableaux \var{FLUMAS} et \var{FLUMAB} ou \var{DIVERG}
l\`a encore par une application directe des formules (\ref{Base_Itrmas_eq_intavcrec}) et
(\ref{Base_Itrmas_eq_brdavcrec}). 

%%%%%%%%%%%%%%%%%%%%%%%%%%%%%%%%%%
%%%%%%%%%%%%%%%%%%%%%%%%%%%%%%%%%%
\section{Points \`a traiter}
%%%%%%%%%%%%%%%%%%%%%%%%%%%%%%%%%%
%%%%%%%%%%%%%%%%%%%%%%%%%%%%%%%%%%
Il est un peu redondant de passer en argument \`a la fois le pas de temps
projet\'e aux faces et le pas de temps au centre des cellules. Il faudrait
s'assurer de la r\'eelle n\'ecessit\'e de cela, ou alors \'etudier des
formulations plus simples de la partie reconstruction.

