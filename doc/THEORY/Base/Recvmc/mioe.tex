%                      Code_Saturne version 1.3
%                      ------------------------
%
%     This file is part of the Code_Saturne Kernel, element of the
%     Code_Saturne CFD tool.
%
%     Copyright (C) 1998-2007 EDF S.A., France
%
%     contact: saturne-support@edf.fr
%
%     The Code_Saturne Kernel is free software; you can redistribute it
%     and/or modify it under the terms of the GNU General Public License
%     as published by the Free Software Foundation; either version 2 of
%     the License, or (at your option) any later version.
%
%     The Code_Saturne Kernel is distributed in the hope that it will be
%     useful, but WITHOUT ANY WARRANTY; without even the implied warranty
%     of MERCHANTABILITY or FITNESS FOR A PARTICULAR PURPOSE.  See the
%     GNU General Public License for more details.
%
%     You should have received a copy of the GNU General Public License
%     along with the Code_Saturne Kernel; if not, write to the
%     Free Software Foundation, Inc.,
%     51 Franklin St, Fifth Floor,
%     Boston, MA  02110-1301  USA
%
%-----------------------------------------------------------------------
%

%%%%%%%%%%%%%%%%%%%%%%%%%%%%%%%%%%
%%%%%%%%%%%%%%%%%%%%%%%%%%%%%%%%%%
\section{Mise en \oe uvre}
%%%%%%%%%%%%%%%%%%%%%%%%%%%%%%%%%%
%%%%%%%%%%%%%%%%%%%%%%%%%%%%%%%%%%
Le flux de masse est pass\'e par les arguments \var{FLUMAS} et \var{FLUMAB}.

\etape{Calcul de la matrice}
Les \var{NCEL} matrices $3\times 3$ sont stock\'ees dans le tableau de travail
\var{COCG},
de dimension $NCELET\times 3\times 3$. Ce dernier est d'abord mis \`a z\'ero, puis
son remplissage se fait dans des boucles sur les faces internes et les faces de
bord. La matrice \'etant sym\'etrique, ces boucles ne
servent qu'\`a remplir la partie triangulaire sup\'erieure, le reste \'etant
rempli par sym\'etrie \`a la fin.

\etape{Inversion de la matrice}
On calcule les coefficients de la comatrice, puis l'inverse.
Pour des questions de vectorisation, la boucle sur les \var{NCEL} \'el\'ements
est remplac\'ee par une
s\'erie de boucles en vectorisation forc\'ee sur des blocs de \var{NBLOC=1024}
\'el\'ements. Le reliquat ($\var{NCEL}-E(\var{NCEL}/1024)\times 1024$) est
trait\'e apr\`es les boucles.
\`A la fin, la matrice inverse est stock\'ee dans \var{COCG}
(toujours en utilisant sa propri\'et\'e de sym\'etrie).

\etape{Calcul du second membre et r\'esolution}
Le second membre est stock\'e dans \var{BX}, \var{BY} et \var{BZ}. La vitesse
finale est stock\'ee dans \var{UX}, \var{UY} et \var{UZ}.


%%%%%%%%%%%%%%%%%%%%%%%%%%%%%%%%%%
%%%%%%%%%%%%%%%%%%%%%%%%%%%%%%%%%%
\section{Points \`a traiter}
%%%%%%%%%%%%%%%%%%%%%%%%%%%%%%%%%%
%%%%%%%%%%%%%%%%%%%%%%%%%%%%%%%%%%
\etape{Vectorisation forc\'ee}
Le d\'ecoupage en boucles de 1024 est-il vraiment n\'ecessaire ? Les machines
vectorielles et les compilateurs sont-ils aujourd'hui capables
d'effectuer la vectorisation sans cette aide ? On note cependant que ce
d\'ecoupage en boucles de 1024 n'a pas de co\^ut CPU suppl\'ementaire, et un
co\^ut m\'emoire n\'egligeable. Le seul inconv\'enient r\'eside dans la
complexit\'e de l'\'ecriture.

\etape{Suppression de la m�thode \var{IREVMC} = 2}
Sur un maillage ``1D'' d'hexa�dres tous orthogonaux sauf une face, on peut montrer que la m�thode fait appara�tre
une composante de vitesse aberrante non nulle et directement d�termin�e par l'angle de non orthogonalit� de la
face (non consistance). On pourrait donc songer � supprimer purement cette m�thode, dans la mesure o� elle n'est
{\em a priori} consistante que sur une classe r�duite de maillages.

