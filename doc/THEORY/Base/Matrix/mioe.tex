%                      Code_Saturne version 1.3
%                      ------------------------
%
%     This file is part of the Code_Saturne Kernel, element of the
%     Code_Saturne CFD tool.
%
%     Copyright (C) 1998-2007 EDF S.A., France
%
%     contact: saturne-support@edf.fr
%
%     The Code_Saturne Kernel is free software; you can redistribute it
%     and/or modify it under the terms of the GNU General Public License
%     as published by the Free Software Foundation; either version 2 of
%     the License, or (at your option) any later version.
%
%     The Code_Saturne Kernel is distributed in the hope that it will be
%     useful, but WITHOUT ANY WARRANTY; without even the implied warranty
%     of MERCHANTABILITY or FITNESS FOR A PARTICULAR PURPOSE.  See the
%     GNU General Public License for more details.
%
%     You should have received a copy of the GNU General Public License
%     along with the Code_Saturne Kernel; if not, write to the
%     Free Software Foundation, Inc.,
%     51 Franklin St, Fifth Floor,
%     Boston, MA  02110-1301  USA
%
%-----------------------------------------------------------------------
%

%%%%%%%%%%%%%%%%%%%%%%%%%%%%%%%%%%
%%%%%%%%%%%%%%%%%%%%%%%%%%%%%%%%%%
\section{Mise en \oe uvre}
%%%%%%%%%%%%%%%%%%%%%%%%%%%%%%%%%%
%%%%%%%%%%%%%%%%%%%%%%%%%%%%%%%%%%
\subsection{\bf Initialisations}
L'indicateur de sym\'etrie \var{ISYM} de la matrice consid\'er\'ee est affect\'e comme suit :\\
\hspace*{1cm}$\bullet$ $\var{ISYM}$ = 1 , si la matrice est sym\'etrique ;
on travaille en diffusion pure , \var{ICONVP} = 0 et \var{IDIFFP} = 1,\\
\hspace*{1cm}$\bullet$ $\var{ISYM}$ = 2 , si la matrice est non sym\'etrique ;
on travaille soit en convection pure (~\var{ICONVP}~=~1, \var{IDIFFP}~=~0~), soit en
convection/diffusion (~\var{ICONVP}~=~1,~\var{IDIFFP}~=~1~).\\
Les termes diagonaux de la matrice sont stock\'es dans le tableau
$\var{DA(NCEL)}$. Ceux extra-diagonaux le sont dans $\var{XA(NFAC,1)}$ si la
matrice est sym\'etrique, dans $\var{XA(NFAC,2)}$ sinon.


Le tableau $\var{DA}$ est initialis\'e \`a z\'ero pour un calcul avec
$ \var{ISTATP} = 0 $ (en fait, ceci ne concerne que les calculs relatifs \`a la
pression). Sinon, on lui affecte la valeur \var{ROVSDT} comprenant la partie instationnaire, la contribution du terme continu en $-\ a\ \dive(\rho \vect{u})^n$
et la partie diagonale des termes sources implicit\'es. Le tableau
$\var{XA(NFAC,*)}$ est initialis\'e \`a z\'ero.\\
\subsection{\bf Calcul des termes extradiagonaux stock\'es dans \var{XA} }
Ils ne se calculent que pour des faces purement internes \var{IFAC}, les faces
de bord ne contribuant qu'\`a la diagonale.
\subsubsection{matrice non sym\'etrique ( pr\'esence de convection ) }
%\hspace*{1cm}{\tiny$\blacksquare$}\ \underline{pour une face purement interne
%$ij\ ( \var{IFAC} )$} \\
Pour chaque face interne \var{IFAC}, les contributions extradiagonales relatives
au terme $a_I$ et \`a son voisin  associ\'e $a_J$ sont calcul\'ees dans
$\var{XA(IFAC,1)}$ et $\var{XA(IFAC,2)}$ respectivement (pour une face orient�e
de I vers J).\\
On a les relations suivantes :\\
\begin{equation}\label{Base_Matrix_eq_extra}
\begin{array}{ll}
\var{XA(IFAC,1)}& = \var{ICONVP}\,*\,\var{FLUI} - \var{IDIFFP}\,*\,\var{VISCF(IFAC)}\\
\var{XA(IFAC,2)}& = \var{ICONVP}\,*\,\var{FLUJ} - \var{IDIFFP}\,*\,\var{VISCF(IFAC)}\\
\end{array}
\end{equation}
avec $\var{FLUMAS(IFAC)}$  correspondant \`a $\ m_{\,ij}^n$, $\var{FLUI}$ \`a $ \displaystyle\frac{1}{2}\,(\ m_{\,ij}^n - |\
m_{\,ij}^n|\ )$, $\var{VISCF(IFAC)} $ \`a $ \ \displaystyle \beta_{\,ij}\frac {
S_{\,ij}}{\overline{I'J'}} $.\\\\
$\var{XA(IFAC,1)}$ repr\'esente le facteur de $a_J$ dans la
derni\`ere expression de (\ref{Base_Matrix_eq_face_int}).\\

$\var{FLUJ}$ correspond \`a $\ -\displaystyle\frac{1}{2}\,(\ m_{\,ij}^n + |\
m_{\,ij}^n|\ )$. En effet, $\var{XA(IFAC,2)}$ est le facteur de $a_I$ dans l'expression \'equivalente de
la derni\`ere ligne de (\ref{Base_Matrix_eq_face_int}), mais \'ecrite en J.\\
Ce qui donne :\\
\begin{equation}\label{Base_Matrix_eq_extra_J}
\sum\limits_{i\in
Vois(j)}\left[\displaystyle\frac{1}{2}(\ m_{\,ji}^n + |\ m_{\,ji}^n|\ )\,a_J +
\displaystyle\frac{1}{2}(\ m_{\,ji}^n - |\ m_{\,ji}^n|)\,a_I\right]
 - \sum\limits_{i\in Vois(j)}\displaystyle \beta_{\,ji}\frac{a_I - a_J}{\overline{J'I'}} S_{\,ji}
\end{equation}\\
Le terme recherch\'e est donc :
$\ \displaystyle\frac{1}{2}(\ m_{\,ji}^n - |\ m_{\,ji}^n|\ )-\displaystyle
\beta_{\,ji}\frac {S_{\,ji}}{\overline{J'I'}}$ .\\
Or :\\ $ m_{\,ji}^n $ = $\ - m_{\,ij}^n $  ($\vect{S}_{\,ji} = -
\vect{S}_{\,ij}$ et $(\rho \vect{u})_{\,ji}^n\  =\ (\rho \vect{u})_{\,ij}^n\
$), avec $\overline{J'I'}$ mesure alg\'ebrique, orient\'ee relativement \`a la
normale sortante \`a la face, {\it i.e.} allant de $J$ vers $I$. On la note
${\overline{J'I'}^{\tiny {\,J}}}$. \\
On a la relation :\\
\begin{equation}\label{Base_Matrix_Eq_mesure_alg}
\overline{J'I'}^{\tiny {\,J}}=\ \overline{I'J'}^{\tiny {\,I}} = (\ \overline{I'J'})
\end{equation}
d'o\`u la deuxi\`eme \'egalit\'e dans (\ref{Base_Matrix_eq_extra}).
\subsubsection{matrice sym\'etrique ( diffusion pure ) }
Pour chaque face interne \var{IFAC}, la contribution extradiagonale relative au
terme $a_I$ est calcul\'ee dans
$\var{XA(IFAC,1)}$ par la  relation suivante :\\
\begin{equation}
\var{XA(IFAC,1)} =  - \var{IDIFFP}\,*\,\var{VISCF(IFAC)}\\
\end{equation}
avec $\var{VISCF(IFAC)} $ \`a $ \ \displaystyle \beta_{\,ij}\frac {
S_{\,ij}}{\overline{I'J'}} $.
\subsection{\bf Calcul des termes diagonaux stock\'es dans \var{DA} }
\subsubsection{matrice non sym\'etrique ( pr\'esence de convection ) }
Pour chaque face interne $ij\ ( \var{IFAC} )$ s\'eparant les cellules $\Omega_i$
de centre $I$ et $\Omega_j$ de centre $J$, $\var{DA(II)}$ est la quantit\'e en facteur de $a_I$ dans la
derni\`ere expression de (\ref{Base_Matrix_eq_face_int}), soit :
\begin{equation}\label{Base_Matrix_eq_diag_II}
\displaystyle\frac{1}{2}(\ m_{\,ij}^n + |\ m_{\,ij}^n|\ )+\displaystyle
\beta_{\,ij}\frac {S_{\,ij}}{\overline{I'J'}}
\end{equation}
De m\^eme, pour \var{DA(JJ)}, on a :
\begin{equation}\label{Base_Matrix_eq_diag_JJ}
\displaystyle\frac{1}{2}(\ - m_{\,ij}^n + |\ m_{\,ij}^n|\ )+\displaystyle
\beta_{\,ji}\frac {S_{\,ij}}{\overline{I'J'}}
\end{equation}
d'apr\`es l'expression de (\ref{Base_Matrix_eq_extra_J}) et l'\'egalit\'e (\ref{Base_Matrix_Eq_mesure_alg}).\\
L'implantation dans \CS\ associ\'ee est la suivante~:\\
pour toute face \var{IFAC} d'\'el\'ements voisins $\var{II} =
\var{IFACEL(1,IFAC)}$ et $\var{JJ} = \var{IFACEL(2,IFAC)}$,\\
on ajoute \`a
$\var{DA(II)}$ la contribution crois\'ee $-\var{XA(IFAC,2)}$ ({\it cf.}
(\ref{Base_Matrix_eq_diag_II}))  et
\`a
$\var{DA(JJ)}$ la contribution $-\var{XA(IFAC,1)}$ ({\it cf.}
(\ref{Base_Matrix_eq_diag_JJ})).
\subsection{\bf Prise en compte des conditions aux limites}
Elles interviennent juste dans le tableau \var{DA}, compte-tenu de leur
\'ecriture et d\'efinition. Elles se calculent {\it via} la derni\`ere
expression de (\ref{Base_Matrix_eq_face_bord}). Pour chaque face \var{IFAC}, de l'\'el\'ement
de centre $I$, jouxtant le bord, on s'int\'eresse \`a :
\begin{equation}
\begin{array}{ll}
\sum\limits_{k\in {\gamma_b(i)}}\left[\displaystyle\frac{1}{2}(\ m_{\,{b}_{ik}}^n + |\ m_{\,{b}_{ik}}^n|\ )\,a_I +
\displaystyle\frac{1}{2}(\ m_{\,{b}_{ik}}^n -
|m_{\,{b}_{ik}}^n|)\,a_{\,{b}_{ik}}\right] - \sum\limits_{k\in {\gamma_b(i)}}\displaystyle\beta_{\,b_{ik}}
\frac{a_{\,b_{ik}}- a_I}{\overline{I'F}} S_{\,b_{ik}}
\end{array}
\end{equation}
avec~:
\begin{equation}\notag
a_{\,{b}_{ik}} =  B_{\,b,ik}\,a_I\\
\end{equation}
soit :
\begin{equation}
\begin{array}{ll}
\left(\sum\limits_{k\in {\gamma_b(i)}}\left[\displaystyle\frac{1}{2}(\ m_{\,{b}_{ik}}^n + |\ m_{\,{b}_{ik}}^n|\ )\,+
\displaystyle\frac{1}{2}(\ m_{\,{b}_{ik}}^n -
|m_{\,{b}_{ik}}^n|)B_{\,b,ik}\,\right] + \sum\limits_{k\in {\gamma_b(i)}}\displaystyle\beta_{\,b_{ik}}
\frac{1 -\ B_{\,b,ik}}{\overline{I'F}} S_{\,b_{ik}}\right) a_I
\end{array}
\end{equation}
ce qui, pour le terme sur lequel porte la somme, se traduit par :\\
$\var{ICONVP} * (- \var{FLUJ} + \var{FLUI} * \var{COEFBP(IFAC)} + \var{IDIFFP} *
\var{VISCB(IFAC)} * (\ 1 -\ \var{COEFBP(IFAC)})$ \\ avec,
$\ m_{\,{b}_{ik}}^n\ $ repr\'esent\'e par $\ \var{FLUMAB(IFAC)}\ $,
$\ \displaystyle\frac{1}{2}\ (\
m_{\,{b}_{ik}}^n + |\ m_{\,{b}_{ik}}^n|\ )\ $ par $\ \var{-\ FLUJ}\ $,\\
$\ \displaystyle\frac{1}{2}\ (\ m_{\,{b}_{ik}}^n -
|m_{\,{b}_{ik}}^n|)B_{\,b,ik}\ $ par $\ \var{FLUI}\ $,
$B_{\,b,ik}$ par $\var{COEFBP(IFAC)}$, $\beta_{\,b_{ik}}\displaystyle\frac
{S_{\,b_{ik}}}{\overline{I'F}} $ par $\var{VISCB(IFAC)}$.\\
\subsection{\bf D\'ecalage du spectre}
Lorsqu'il n'existe aucune condition \`a la limite de type Dirichlet et que
$\var{ISTATP} = 0 $ (c'est-\`a-dire pour la pression uniquement), on
d\'eplace le spectre de la matrice ${\tens{EM}}_{\,scal}$ de $\var{EPSI}$  ({\it i.e.} on multiplie chaque terme diagonal par $(1 + \var{EPSI})$ ) afin
de la rendre inversible. \var{EPSI} est fix\'e en dur dans \fort{matrix} \`a
 ${10}^{-7}$.
%%%%%%%%%%%%%%%%%%%%%%%%%%%%%%%%%%
%%%%%%%%%%%%%%%%%%%%%%%%%%%%%%%%%%
\section{Points \`a traiter}
%%%%%%%%%%%%%%%%%%%%%%%%%%%%%%%%%%
%%%%%%%%%%%%%%%%%%%%%%%%%%%%%%%%%%
\etape{Initialisation}
Le tableau \var{XA} est initialis\'e \`a z\'ero lorsqu'on veut annuler la
contribution du terme en
$\displaystyle\frac{\rho \ |\Omega_i|}{\Delta t}$, {\it i.e.} $\var{ISTATP} = 0 $ . Ce qui ne permet donc pas la prise en
compte effective des parties diagonales des termes sources \`a impliciter,
d\'ecid\'ee par l'utilisateur. Actuellement, ceci ne sert que pour la variable
pression et reste donc {\it a priori} correct, mais cette d\'emarche est \`a
corriger dans l'absolu.\\\\
\etape{Nettoyage}
La contribution $\var{ICONVP}\ \var{FLUI}$, dans le calcul du terme
\var{XA(IFAC,1)} lorsque la matrice est sym\'etrique est inutile, car
$\var{ICONVP}\ = 0$. \\\\
\etape{Prise en compte du type de sch\'ema de convection dans
${\tens{EM}}_{\,scal}$}
Actuellement, les contributions des  flux convectifs non reconstruits sont
trait\'ees par sch\'ema d\'ecentr\'e amont, quelque soit le sch\'ema choisi par
l'utilisateur. Ceci peut \^etre handicapant. Par exemple, m\^eme sur
maillage orthogonal, on est oblig\'e de faire plusieurs sweeps pour obtenir une
vitesse pr\'edite correcte. Un sch\'ema centr\'e sans test de pente peut
\^etre implant\'e facilement, mais cette \'ecriture pourrait, dans l'\'etat
actuel des connaissances, entra\^\i ner des instabilit\'es
num\'eriques. Il serait souhaitable d'avoir d'autres sch\'emas tout aussi
robustes, mais plus adapt\'es \`a certaines configurations.\\\\
\etape{Maillage localement pathologique}
Il peut arriver, notamment au bord, que les mesures alg\'ebriques,
$\overline{I'J'}$ ou $\overline{I'F}$ soient n\'egatives (en cas de maillages
non convexes par exemple). Ceci peut engendrer des probl\`emes plus ou moins
graves : perte de l'existence et l'unicit\'e de la solution (l'op\'erateur associ\'e n'ayant plus les bonnes propri\'et\'es de r\'egularit\'e
ou de coercivit\'e), d\'egradation de la matrice (perte de la positivit\'e) et donc r\'esolution par solveur lin\'eaire
associ\'e non appropri\'e (gradient conjugu\'e par exemple).\\
Une impression permettant de signaler et de localiser le probl\`eme serait souhaitable.


