%                      Code_Saturne version 1.3
%                      ------------------------
%
%     This file is part of the Code_Saturne Kernel, element of the
%     Code_Saturne CFD tool.
%
%     Copyright (C) 1998-2007 EDF S.A., France
%
%     contact: saturne-support@edf.fr
%
%     The Code_Saturne Kernel is free software; you can redistribute it
%     and/or modify it under the terms of the GNU General Public License
%     as published by the Free Software Foundation; either version 2 of
%     the License, or (at your option) any later version.
%
%     The Code_Saturne Kernel is distributed in the hope that it will be
%     useful, but WITHOUT ANY WARRANTY; without even the implied warranty
%     of MERCHANTABILITY or FITNESS FOR A PARTICULAR PURPOSE.  See the
%     GNU General Public License for more details.
%
%     You should have received a copy of the GNU General Public License
%     along with the Code_Saturne Kernel; if not, write to the
%     Free Software Foundation, Inc.,
%     51 Franklin St, Fifth Floor,
%     Boston, MA  02110-1301  USA
%
%-----------------------------------------------------------------------
%
\programme{codits}
%

\vspace{1cm}
%%%%%%%%%%%%%%%%%%%%%%%%%%%%%%%%%%
%%%%%%%%%%%%%%%%%%%%%%%%%%%%%%%%%%
\section{Fonction}
%%%%%%%%%%%%%%%%%%%%%%%%%%%%%%%%%%
%%%%%%%%%%%%%%%%%%%%%%%%%%%%%%%%%%
Ce sous-programme, appel� entre autre par \fort{preduv}, \fort{turbke}, \fort{covofi},
\fort{resrij}, \fort{reseps}, ..., r�sout les �quations de convection-diffusion
d'un scalaire $a$ avec termes sources du type :
\begin{equation}\label{Base_Codits_eq_ref}
\begin{array}{c}
\displaystyle f_s^{\,imp} (a^{n+1} - a^{n}) +
\theta \ \underbrace{\dive((\rho \underline{u})\,a^{n+1})}_{\text{convection implicite}}
-\theta \ \underbrace{\dive(\mu_{\,tot}\,\grad a^{n+1})}_{\text{diffusion implicite}}
\\\\
= f_s^{\,exp}-(1-\theta) \ \underbrace{\dive((\rho \underline{u})\,a^{n})}_{\text{convection explicite}}
 + (1-\theta) \ \underbrace{\dive(\mu_{\,tot}\,\grad a^{n})}_{\text{diffusion explicite}}
\end{array}
\end{equation}
o� $\rho \underline{u}$, $f_s^{exp}$ et $f_s^{imp}$ d�signent respectivement le flux de masse, les termes sources explicites et les termes lin�aris�s en $a^{n+1}$.
$a$ est un scalaire d�fini sur toutes les cellules\footnote{$a$, sous forme discr\`ete en espace, correspond \`a un vecteur dimensionn\'e \`a \var{NCELET} de composante $a_I$, I d\'ecrivant l'ensemble des cellules.}.
Par souci de clart� on suppose, en l'absence d'indication, les propri�tes
physiques $\Phi$ (viscosit� totale $\mu_{tot}$,...) et le flux de masse $(\rho
\underline{u})$ pris respectivement aux instants $n+\theta_\Phi$ et
$n+\theta_F$, o� $\theta_\Phi$ et $\theta_F$ d�pendent des sch�mas en temps
sp�cifiquement utilis�s pour ces grandeurs\footnote{cf. \fort{introd}}.
\\
L'�criture des termes de convection et diffusion en maillage non orthogonal
engendre des difficult�s (termes de reconstruction et test de pente) qui sont
contourn�es en utilisant une m\'ethode it\'erative dont la limite, si elle
existe, est la solution de l'�quation pr�c�dente.
