%                      Code_Saturne version 1.3
%                      ------------------------
%
%     This file is part of the Code_Saturne Kernel, element of the
%     Code_Saturne CFD tool.
% 
%     Copyright (C) 1998-2007 EDF S.A., France
%
%     contact: saturne-support@edf.fr
% 
%     The Code_Saturne Kernel is free software; you can redistribute it
%     and/or modify it under the terms of the GNU General Public License
%     as published by the Free Software Foundation; either version 2 of
%     the License, or (at your option) any later version.
% 
%     The Code_Saturne Kernel is distributed in the hope that it will be
%     useful, but WITHOUT ANY WARRANTY; without even the implied warranty
%     of MERCHANTABILITY or FITNESS FOR A PARTICULAR PURPOSE.  See the
%     GNU General Public License for more details.
% 
%     You should have received a copy of the GNU General Public License
%     along with the Code_Saturne Kernel; if not, write to the
%     Free Software Foundation, Inc.,
%     51 Franklin St, Fifth Floor,
%     Boston, MA  02110-1301  USA
%
%-----------------------------------------------------------------------
%
\programme{cfbl**}

%%%%%%%%%%%%%%%%%%%%%%%%%%%%%%%%%%
%%%%%%%%%%%%%%%%%%%%%%%%%%%%%%%%%%
\section{Fonction}
%%%%%%%%%%%%%%%%%%%%%%%%%%%%%%%%%%
%%%%%%%%%%%%%%%%%%%%%%%%%%%%%%%%%%


On s'int�resse � la r�solution des �quations de Navier-Stokes en compressible,
en particulier pour des configurations sans choc. Le sch�ma global correspond � une 
extension des algorithmes volumes finis mis en \oe uvre pour simuler les 
�quations de Navier-Stokes en incompressible.

Dans les grandes lignes, le sch\'ema est constitu\'e d'une \'etape
``acoustique'' fournissant la masse volumique (ainsi qu'une pr\'ediction de
pression et un d\'ebit acoustique), suivie de la r\'esolution de l'\'equation de
la quantit\'e de mouvement~; on r\'esout ensuite l'\'equation de l'\'energie 
et, pour terminer, la pression est mise \`a jour. 
Moyennant une contrainte sur la valeur du pas de temps, le sch\'ema permet 
d'assurer la positivit\'e de la masse volumique. 

La thermodynamique prise en compte \`a ce jour est celle des gaz parfaits, mais
l'organisation du code \`a \'et\'e  pr\'evue pour permettre \`a l'utilisateur de
fournir ses propres lois.  

Pour compl�ter la pr�sentation, on pourra se reporter � la r�f�rence 
suivante~: \\
\textbf{[Mathon]} P. Mathon, F. Archambeau, J.-M. H�rard : "Implantation d'un 
algorithme compressible dans \CS", HI-83/03/016/A

Le cas de validation "tube \`a choc" de la version 1.2 de \CS\ permettra 
\'egalement d'apporter quelques compl\'ements (tube \`a choc de Sod, 
discontinuit\'e de contact instationnaire, double d\'etente sym\'etrique, 
double choc sym\'etrique). 

\newpage
%=================================
\subsection{Notations}
%=================================

\begin{table}[h!]
\begin{tabular}{ccp{10,5cm}}

{\bf Symbole} & {\bf Unit\'e} & {\bf Signification}\\

\phantom{$C_v$, ${C_v}_i$} & \phantom{$\lbrack f\rbrack.\,kg/(m^3.\,s)$} & \\

$C_p$, ${C_p}_i$ & $J/(kg.\,K)$	& capacit� calorifique \`a pression constante
	$C_p = \left.\frac{\partial h}{\partial T}\right)_P$\\
$C_v$, ${C_v}_i$ & $J/(kg.\,K)$	& capacit� calorifique \`a volume constant
	$C_v = \left.\frac{\partial \varepsilon}{\partial T}\right)_\rho$\\
$\mathcal{D}_{f/b}$ & $m^2/s$ 	& diffusivit\'e mol\'eculaire du composant $f$
					dans le bain\\
$E$ 		& $J/m^3$	& \'energie totale volumique $E = \rho e$\\
$F$ 		& 	 	& centre de gravit\'e d'une face\\
$H$ 		& $J/kg$ 	& enthalpie totale massique 
					$H = \frac{E+P}{\rho}$\\
$I$ 		& 	 	& point de co-location de la cellule $i$\\
$I'$ 		& 	 	& pour une face $ij$ partag\'ee entre les 
					cellules $i$ et $j$, $I'$ 
					est le projet\'e de $I$ sur la 
					normale \`a la $ij$ passant 
					par $F$, centre de $ij$\\
$K$ 		& $kg/(m.\,s)$ 	& diffusivit\'e thermique\\
$M$, $M_i$ 	& $kg/mol$ 	& masse molaire ($M_i$ pour le constituant $i$)\\
$P$ 		& $Pa$ 		& pression\\
$\vect{Q}$ 	& $kg/(m^2.\,s)$& vecteur quantit\'e de mouvement 
					$\vect{Q} = \rho\vect{u}$\\
$\vect{Q}_{ac}$ & $kg/(m^2.\,s)$& vecteur quantit\'e de mouvement issu 
					de l'\'etape acoustique\\
$Q$ 		& $kg/(m^2.\,s)$& norme de $\vect{Q}$\\
$R$ 		& $J/(mol.\,K)$ & constante universelle des gaz parfaits\\
$S$ 		& $J/(K.\,m^3)$	& entropie volumique\\
$\mathcal{S}$ 	& $\lbrack f\rbrack.\,kg/(m^3.\,s)$
				& Terme de production/dissipation volumique 
					pour le scalaire $f$\\
$T$ 		& $K$ 		& temp\'erature ($>0$)\\
$Y_i$ 		& 		& fraction massique du compos\'e $i$ 
					($0 \leqslant Y_i \leqslant 1$)\\
\end{tabular}
\end{table}

\clearpage 

\begin{table}[htp]
\begin{tabular}{ccp{10,5cm}}

{\bf Symbole} & {\bf Unit\'e} & {\bf Signification}\\

\phantom{$C_v$, ${C_v}_i$} & \phantom{$\lbrack f\rbrack.\,kg/(m^3.\,s)$} & \\

$c^2$ 		& $(m/s)^2$ 	& carr\'e de la vitesse du son
		$c^2 = \left.\frac{\partial P}{\partial \rho}\right)_s$\\
$e$ 		& $J/kg$ 	& \'energie totale massique 
					$e = \varepsilon + \frac{1}{2}u^2$\\
$\vect{f}_v$ 	& $N/kg$ 	& $\rho\vect{f}_v$ repr\'esente le terme 
					source volumique pour la quantit\'e
					de mouvement~: gravit\'e, pertes 
					de charges, tenseurs des contraintes
					turbulentes, forces de Laplace...\\
$\vect{g}$ 	& $m/s^2$ 	& acc\'el\'eration de la pesanteur\\
$h$ 		& $J/kg$ 	& enthalpie massique 
					$h=\varepsilon + \frac{P}{\rho}$\\
$i$ 		&  		& indice faisant r\'ef\'erence \`a la 
					cellule $i$~; $f_i$ est la valeur 
					de la variable $f$ associ\'ee 
					au point de co-location $I$\\
$I'$ 		&  		& indice faisant r\'ef\'erence \`a la 
					cellule $i$~; $f_I'$ est la valeur 
					de la variable $f$ associ\'ee 
					au point $I'$\\
$\vect{j}\wedge\vect{B}$ 
		& $N/m^3$ 	& forces de Laplace\\
$r$, $r_i$ 	& $J/(kg.\,K)$ 	& constante massique des gaz parfaits 
					$r = \frac{R}{M}$
					(pour le constituant $i$, on a $r_i=\frac{R}{M_i}$)\\
$s$ 		& $J/(K.\,kg)$ 	& entropie massique\\
$t$ 		& $s$ 		& temps\\
$\vect{u}$ 	& $m/s$ 	& vecteur vitesse\\
$u$ 		& $m/s$ 	& norme de $\vect{u}$\\

\end{tabular}
\end{table}


\newpage 

\begin{table}[htp]
\begin{tabular}{ccp{10,5cm}}

{\bf Symbole} & {\bf Unit\'e} & {\bf Signification}\\

\phantom{$C_v$, ${C_v}_i$} & \phantom{$\lbrack f\rbrack.\,kg/(m^3.\,s)$} & \\

$\beta$ 	& $kg/(m^3.\,K)$ & 
	$\beta = \left.\frac{\partial P}{\partial s}\right)_\rho$\\
$\gamma$ 	& $kg/(m^3.\,K)$ & constante caract\'eristique 
					d'un gaz parfait 
					$\gamma = \frac{C_p}{C_v}$\\
$\varepsilon$ 	& $J/kg$ 	& \'energie interne massique\\
$\kappa$ 	& $kg/(m.\,s)$ 	& viscosit\'e dynamique en volume\\
$\lambda$ 	& $W/(m.\,K)$ 	& conductivit\'e thermique\\
$\mu$	 	& $kg/(m.\,s)$ 	& viscosit\'e dynamique ordinaire\\
$\rho$ 		& $kg/m^3$ 	& densit\'e\\
$\vect{\varphi}_f$ 
		& $\lbrack f\rbrack.\,kg/(m^2.\,s)$ 		
				& vecteur flux diffusif du compos\'e $f$\\
$\varphi_f$ 	& $\lbrack f\rbrack.\,kg/(m^2.\,s)$ 
				& norme de $\vect{\varphi}_f$\\

\phantom{ouden}	& 		& \\

$\tens{\Sigma}^v$ &$kg/(m^2.\,s^2)$& tenseur des contraintes visqueuses\\
$\vect{\Phi}_s$ & $W/m^2$	& vecteur flux conductif de chaleur\\
$\Phi_s$ 	& $W/m^2$ 	& norme de $\vect{\Phi}_s$\\
$\Phi_v$ 	& $W/kg$ 	& $\rho\Phi_v$ repr\'esente le terme 
					source volumique d'\'energie, 
					comprenant par exemple 
					l'effet Joule $\vect{j}\cdot\vect{E}$,
					le rayonnement...\\
\end{tabular}
\end{table}
\clearpage

%=================================
\subsection{Syst\`eme d'\'equations laminaires de r\'ef\'erence}
%=================================

L'algorithme d\'evelopp\'e propose de r\'esoudre
l'\'equation de continuit\'e, les \'equations de Navier-Stokes
ainsi que l'\'equation d'\'energie totale de mani\`ere conservative,
pour des \'ecoulements compressibles.

\begin{equation}\label{Cfbl_Cfbase_eq_ref_laminaire_cfbase}
\left\{\begin{array}{l}

\displaystyle\frac{\partial\rho}{\partial t} + \divs(\vect{Q}) = 0 \\
\\
\displaystyle\frac{\partial\vect{Q}}{\partial t}
+ \divv(\vect{u} \otimes \vect{Q}) + \gradv{P}
= \rho \vect{f}_v + \divv(\tens{\Sigma}^v) \\
\\
\displaystyle\frac{\partial E}{\partial t} + \divs( \vect{u} (E+P) )
= \rho\vect{f}_v\cdot\vect{u} + \divs(\tens{\Sigma}^v \vect{u})
- \divs{\,\vect{\Phi}_s} + \rho\Phi_v

\end{array}\right.
\end{equation}

Nous avons pr�sent� ici le syst�me d'�quations laminaires, mais il faut pr�ciser
que la turbulence ne pose pas de probl�me particulier dans la mesure o� les 
�quations suppl\'ementaires sont d�coupl�es du syst\`eme~(\ref{Cfbl_Cfbase_eq_ref_laminaire_cfbase}).  

%=================================
\subsection{Expression des termes intervenant dans les \'equations}
%=================================

\begin{itemize}

\item{\'Energie totale volumique :
	\begin{equation}
	E = \rho e = \rho\varepsilon + \frac{1}{2} \rho u^2
	\end{equation}
	avec l'\'energie interne $\varepsilon(P,\rho)$ donn\'ee par l'\'equation d'\'etat}
\\
\item{Forces volumiques : $\rho\vect{f}_v$ (dans la plupart des cas 
                                            $\rho\vect{f}_v= \rho\vect{g}$)}
\\
\item{Tenseur des contraintes visqueuses pour un fluide Newtonien :
	\begin{equation}
	\tens{\Sigma}^v = \mu (\gradt{\vect{u}} +\ ^t\gradt{\vect{u}})
	+ (\kappa - \frac{2}{3}\mu) \divs\,{\vect{u}}\ \tens{Id}
	\end{equation}
	avec $\mu(T,\ldots)$ et $\kappa(T,\ldots)$ mais souvent $\kappa =0$}
\\
\item{Flux de conduction de la chaleur : loi de Fourier
	\begin{equation}
	\vect{\Phi}_s = -\lambda \gradv{T}
	\end{equation}
	avec $\lambda(T,\ldots)$} 
\\
\item{Source de chaleur volumique : $\rho\Phi_v$}

\end{itemize}


%=================================
\subsection{\'Equations d'\'etat et expressions de l'\'energie interne}
\label{Cfbl_Cfbase_equations_etat_cfbase}
%=================================

%---------------------------------
\subsubsection{Gaz parfait}
%---------------------------------

\'Equation d'\'etat : $P = \rho r T$\\


\'Energie interne massique :
$\varepsilon = \displaystyle\frac{P}{(\gamma -1) \rho}$

Soit~: 
\begin{equation}\label{Cfbl_Cfbase_eq_pression_gp_cfbase}
P = (\gamma -1) \rho (e - \frac{1}{2} u^2)
\end{equation}


%---------------------------------
\subsubsection{M\'elange de gaz parfaits}
%---------------------------------

On consid\`ere un m\'elange de $N$ constituants de fractions massiques
$(Y_i)_{i=1 \ldots N}$\\

\'Equation d'\'etat : $P = \rho\ r_{m\acute elange}\ T$\\

\'Energie interne massique :
$\varepsilon = \displaystyle\frac{P}{(\gamma_{m\acute elange} -1)\rho}$

Soit~:
\begin{equation}\label{Cfbl_Cfbase_eq_pression_melange_gp_cfbase}
P = (\gamma_{m\acute elange} -1) \rho (e - \frac{1}{2} u^2)
\end{equation}

avec $\gamma_{m\acute elange}
= \displaystyle\frac{\sum\limits_{i=1}^{N} {Y_i C_{pi}}}
{\sum\limits_{i=1}^{N} {Y_i C_{vi}}}$\ \ 
et\ \ $r_{m\acute elange} = \displaystyle\sum\limits_{i=1}^{N} {Y_i r_i}$


%---------------------------------
\subsubsection{Equation d'\'etat de Van der Waals}
%---------------------------------

Cette \'equation est une correction de l'\'equation d'\'etat
des gaz parfaits pour tenir compte des forces intermol\'eculaires
et du volume des mol\'ecules constitutives du gaz.
On introduit deux coefficients correctifs~:
$a$ [$Pa.\,m^6 / kg^2$] est li\'e aux forces intermol\'eculaires 
et $b$ [$m^3/kg$] est le covolume (volume occup\'e par les mol\'ecules).\\

\'Equation d'\'etat : $(P+a\rho^2)(1-b\rho) = \rho r T$\\

\'Energie interne massique :
$\varepsilon = \displaystyle\frac{(P+a\rho^2)(1-b\rho)}
{(\hat{\gamma} -1)\rho} - a \rho$

Soit~: 
\begin{equation}\label{Cfbl_Cfbase_eq_pression_vdw_cfbase}
P = (\hat{\gamma} -1) \displaystyle\frac{\rho}{(1-b\rho)}
(e - \frac{1}{2} u^2 + a\rho) - a \rho^2
\end{equation}

avec $\hat{\gamma} = 1 + \displaystyle\frac{r}{C_v}
= \displaystyle\frac{C_p}{C_v}
\displaystyle\left(\frac{P-a\rho^2 (1-2b\rho)}{P+a\rho^2}\right)
+ \displaystyle\frac{2a\rho^2 (1-b\rho)}{P+a\rho^2}$



%=================================
\subsection{Calcul des grandeurs thermodymamiques}
%=================================

%---------------------------------
\subsubsection{Pour un gaz parfait \`a $\gamma$ constant}
%---------------------------------

%`````````````````````````````````
\paragraph{Equation d'\'etat~:}
%,,,,,,,,,,,,,,,,,,,,,,,,,,,,,,,,,

$P = \rho r T$

On suppose connues la chaleur massique \`a pression constante $C_p$
et la masse molaire $M$ du gaz, ainsi que les variables d'\'etat.

%`````````````````````````````````
\paragraph{Chaleur massique \`a volume constant~:}
%,,,,,,,,,,,,,,,,,,,,,,,,,,,,,,,,,

$C_v = C_p - \displaystyle\frac{R}{M} = C_p - r$


%`````````````````````````````````
\paragraph{Constante caract\'eristique du gaz~:}
%,,,,,,,,,,,,,,,,,,,,,,,,,,,,,,,,,

$\gamma = \displaystyle\frac{C_p}{C_v} = \displaystyle\frac{C_p}{C_p - r}$


%`````````````````````````````````
\paragraph{Vitesse du son~:}
%,,,,,,,,,,,,,,,,,,,,,,,,,,,,,,,,,

$c^2 = \gamma \displaystyle\frac{P}{\rho}$


%`````````````````````````````````
\paragraph{Entropie~:}
%,,,,,,,,,,,,,,,,,,,,,,,,,,,,,,,,,

$s = \displaystyle\frac{P}{\rho^{\gamma}}$
\quad et
$\beta = \left.\displaystyle\frac{\partial P}{\partial s}\right)_{\rho}
= \rho^{\gamma}$

\noindent\textit{Remarque~:} L'entropie choisie ici n'est pas l'entropie
physique, mais une entropie math\'ematique qui v\'erifie \quad
$c^2 \left.\displaystyle\frac{\partial s}{\partial P}\right)_{\rho}
+ \left.\displaystyle\frac{\partial s}{\partial \rho}\right)_{P} = 0$


%`````````````````````````````````
\paragraph{Pression~:}
%,,,,,,,,,,,,,,,,,,,,,,,,,,,,,,,,,

$P = (\gamma-1) \rho \varepsilon$


%`````````````````````````````````
\paragraph{Energie interne~:}
%,,,,,,,,,,,,,,,,,,,,,,,,,,,,,,,,,

$\varepsilon = C_v T
= \displaystyle\frac{1}{\gamma-1} \displaystyle\frac{P}{\rho}$\qquad\text{\ \ avec\ \ }
$\varepsilon_{sup} = 0$

%`````````````````````````````````
\paragraph{Enthalpie~:}
%,,,,,,,,,,,,,,,,,,,,,,,,,,,,,,,,,

$h = C_p T
= \displaystyle\frac{\gamma}{\gamma-1} \displaystyle\frac{P}{\rho}$


%---------------------------------
\subsubsection{Pour un m\'elange de gaz parfaits}
%---------------------------------

Une intervention de l'utilisateur dans le sous-programme utilisateur 
\fort{uscfth} est n�cessaire pour pouvoir utiliser ces lois.

%`````````````````````````````````
\paragraph{Equation d'\'etat~:}
%,,,,,,,,,,,,,,,,,,,,,,,,,,,,,,,,,

$P = \rho\ r_{m\acute el}\ T$
\quad avec $r_{m\acute el} = \displaystyle\sum\limits_{i=1}^{N} {Y_i r_i}
= \displaystyle\sum\limits_{i=1}^{N} Y_i \displaystyle\frac{R}{M_i}$


On suppose connues la chaleur massique \`a pression constante
des diff\'erents constituants ${C_p}_i$,
la masse molaire $M_i$ des constituants du gaz,
ainsi que les variables d'\'etat (dont les fractions massiques $Y_i$).

%`````````````````````````````````
\paragraph{Masse molaire du m\'elange~:} 
%,,,,,,,,,,,,,,,,,,,,,,,,,,,,,,,,,

$M_{m\acute el} = \left(\displaystyle\sum\limits_{i=1}^{N}
\displaystyle\frac{Y_i}{M_i} \right)^{-1}$

%`````````````````````````````````
\paragraph{Chaleur massique \`a pression constante du m\'elange~:}
%,,,,,,,,,,,,,,,,,,,,,,,,,,,,,,,,,
$\\$
${C_p}_{m\acute el} = \displaystyle\sum\limits_{i=1}^{N} Y_i {C_p}_i$


%`````````````````````````````````
\paragraph{Chaleur massique \`a volume constant du m\'elange~:}
%,,,,,,,,,,,,,,,,,,,,,,,,,,,,,,,,,
$\\$
${C_v}_{m\acute el} = \displaystyle\sum\limits_{i=1}^{N} Y_i {C_v}_i
= {C_p}_{m\acute el} - \displaystyle\frac{R}{M_{m\acute el}}
= {C_p}_{m\acute el} - r_{m\acute el}$


%`````````````````````````````````
\paragraph{Constante caract\'eristique du gaz~:}
%,,,,,,,,,,,,,,,,,,,,,,,,,,,,,,,,,

$\gamma_{m\acute el} = \displaystyle\frac{{C_p}_{m\acute el}}
{{C_v}_{m\acute el}}
= \displaystyle\frac{{C_p}_{m\acute el}}{{C_p}_{m\acute el} - r_{m\acute el}}$


%`````````````````````````````````
\paragraph{Vitesse du son~:}
%,,,,,,,,,,,,,,,,,,,,,,,,,,,,,,,,,

$c^2 = \gamma_{m\acute el} \displaystyle\frac{P}{\rho}$


%`````````````````````````````````
\paragraph{Entropie~:}
%,,,,,,,,,,,,,,,,,,,,,,,,,,,,,,,,,

$s = \displaystyle\frac{P}{\rho^{\gamma_{m\acute el}}}$
\quad et
$\beta = \left.\displaystyle\frac{\partial P}{\partial s}\right)_{\rho}
= \rho^{\gamma_{m\acute el}}$


%`````````````````````````````````
\paragraph{Pression~:}
%,,,,,,,,,,,,,,,,,,,,,,,,,,,,,,,,,

$P = (\gamma_{m\acute el}-1) \rho \varepsilon$


%`````````````````````````````````
\paragraph{Energie interne~:}
%,,,,,,,,,,,,,,,,,,,,,,,,,,,,,,,,,

$\varepsilon = {C_v}_{m\acute el}\ T$\qquad\text{\ \ avec\ \ }
$\varepsilon_{sup} = 0$


%`````````````````````````````````
\paragraph{Enthalpie~:}
%,,,,,,,,,,,,,,,,,,,,,,,,,,,,,,,,,

$h = {C_p}_{m\acute el}\  T
= \displaystyle\frac{\gamma_{m\acute el}}{\gamma_{m\acute el}-1}
\displaystyle\frac{P}{\rho}$

%---------------------------------
\subsubsection{Pour un gaz de Van der Waals}
%---------------------------------

Ces lois n'ont pas �t� programm�es, mais l'utilisateur peut intervenir
dans le sous-programme utilisateur \fort{uscfth} s'il souhaite le faire.

%`````````````````````````````````
\paragraph{Equation d'\'etat~:}
%,,,,,,,,,,,,,,,,,,,,,,,,,,,,,,,,,

$(P+a\rho^2)(1-b\rho) = \rho r T$

avec $a$ [$Pa.\,m^6 / kg^2$] li\'e aux forces intermol\'eculaires 
et $b$ [$m^3/kg$] le covolume (volume occup\'e par les mol\'ecules).

On suppose connus les coefficients $a$ et $b$, 
la chaleur massique \`a pression constante $C_p$,
la masse molaire $M$ du gaz et 
les variables d'\'etat.


%`````````````````````````````````
\paragraph{Chaleur massique \`a volume constant~:}
%,,,,,,,,,,,,,,,,,,,,,,,,,,,,,,,,,

$C_v = C_p - r
\displaystyle\frac{P+a\rho^2}{P-a\rho^2 (1-2b\rho)}$


%`````````````````````````````````
\paragraph{Constante ``\'equivalente'' du gaz~:}
%,,,,,,,,,,,,,,,,,,,,,,,,,,,,,,,,,

$\hat{\gamma} = 1 + \displaystyle\frac{r}{C_v}
= \displaystyle\frac{C_p}{C_v}
\displaystyle\left(\frac{P-a\rho^2 (1-2b\rho)}{P+a\rho^2}\right)
+ \displaystyle\frac{2a\rho^2 (1-b\rho)}{P+a\rho^2}$

%`````````````````````````````````
\paragraph{Vitesse du son~:}
%,,,,,,,,,,,,,,,,,,,,,,,,,,,,,,,,,

$c^2 = \hat{\gamma} \displaystyle\frac{P+a\rho^2}{\rho(1-b\rho)} - 2a\rho$


%`````````````````````````````````
\paragraph{Entropie~:}
%,,,,,,,,,,,,,,,,,,,,,,,,,,,,,,,,,

$s = (P+a\rho^2)
\left(\displaystyle\frac{1-b\rho}{\rho}\right)^{\hat{\gamma}}$
\quad et
$\beta = \left.\displaystyle\frac{\partial P}{\partial s}\right)_{\rho}
= \left(\displaystyle\frac{\rho}{1-b\rho}\right)^{\hat{\gamma}}$


%`````````````````````````````````
\paragraph{Pression~:}
%,,,,,,,,,,,,,,,,,,,,,,,,,,,,,,,,,

$P = (\hat{\gamma} -1) \displaystyle\frac{\rho}{(1-b\rho)}
(\varepsilon + a\rho) - a \rho^2$

%`````````````````````````````````
\paragraph{Energie interne~:}
%,,,,,,,,,,,,,,,,,,,,,,,,,,,,,,,,,

$\varepsilon = C_v T - a \rho$\qquad\text{\ \ avec\ \ }
$\varepsilon_{sup} = - a \rho$


%`````````````````````````````````
\paragraph{Enthalpie~:}
%,,,,,,,,,,,,,,,,,,,,,,,,,,,,,,,,,

$h = \displaystyle\frac{\hat{\gamma}-b\rho}{\hat{\gamma}-1}
 \displaystyle\frac{P+a\rho^2}{\rho} - 2a\rho$


%=================================
\subsection{Algorithme de base}
%=================================

On suppose connues toutes les variables au temps $t^n$ et on cherche
\`a les d\'eterminer \`a l'instant $t^{n+1}$.
On r\'esout en deux blocs principaux~: d'une part le syst\`eme masse-quantit\'e
de mouvement, de l'autre l'\'equation portant sur l'\'energie et les scalaires
transport\'es.
Dans le premier bloc, on distingue le traitement du syst\`eme (coupl\'e)
acoustique et le traitement de l'\'equation de la quantit\'e de mouvement
compl\`ete. 

Au d\'ebut du pas de temps, on commence par mettre \`a jour
les propri\'et\'es physiques variables (par exemple $\mu(T)$, $\kappa(T)$,
$C_p(Y_1,\ldots ,Y_N)$ ou $\lambda(T)$), puis on
r\'esout les \'etapes suivantes~:

\begin{enumerate}

  \item {\bf Acoustique~: sous-programme \fort{cfmsvl}} \\
     R�solution d'une �quation de convection-diffusion portant sur $\rho^{n+1}$.\\
     On obtient � la fin de l'�tape $\rho^{n+1}$, $Q^{n+1}_{ac}$ et 
�ventuellement une
pr\'ediction de la pression $P^{pred}(\rho^{n+1},e^{n})$.\\
    
  \item {\bf Quantit\'e de mouvement~: sous-programme \fort{cfqdmv}}\\
     R�solution d'une �quation de convection-diffusion portant sur $u^{n+1}$ qui
     fait intervenir  $Q^{n+1}_{ac}$ et $P^{pred}$.\\
     On obtient � la fin de l'�tape $u^{n+1}$.\\ 

  \item {\bf \'Energie totale~: sous-programme \fort{cfener}}\\
     R�solution d'une �quation de convection-diffusion portant sur $e^{n+1}$ qui
     fait intervenir  $Q^{n+1}_{ac}$, $P^{pred}$ et  $u^{n+1}$.\\
     On obtient \`a la fin de l'�tape $e^{n+1}$ et une valeur actualis�e de la
pression $P(\rho^{n+1},e^{n+1})$.\\ 

  \item {\bf Scalaires passifs}\\
     R�solution d'une �quation de convection-diffusion standard par     
     scalaire, avec  $Q^{n+1}_{ac}$ pour flux convectif.
\end{enumerate}





