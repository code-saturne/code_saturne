%                      Code_Saturne version 1.3
%                      ------------------------
%
%     This file is part of the Code_Saturne Kernel, element of the
%     Code_Saturne CFD tool.
% 
%     Copyright (C) 1998-2007 EDF S.A., France
%
%     contact: saturne-support@edf.fr
% 
%     The Code_Saturne Kernel is free software; you can redistribute it
%     and/or modify it under the terms of the GNU General Public License
%     as published by the Free Software Foundation; either version 2 of
%     the License, or (at your option) any later version.
% 
%     The Code_Saturne Kernel is distributed in the hope that it will be
%     useful, but WITHOUT ANY WARRANTY; without even the implied warranty
%     of MERCHANTABILITY or FITNESS FOR A PARTICULAR PURPOSE.  See the
%     GNU General Public License for more details.
% 
%     You should have received a copy of the GNU General Public License
%     along with the Code_Saturne Kernel; if not, write to the
%     Free Software Foundation, Inc.,
%     51 Franklin St, Fifth Floor,
%     Boston, MA  02110-1301  USA
%
%-----------------------------------------------------------------------
%
%%%%%%%%%%%%%%%%%%%%%%%%%%%%%%%%%%%%%%%%%%%%%%%%%%%%%%%%%%%%%%%%%%%%%%
%                                                                    %
%                                                                    %
%                                                                    %
% Titre :           Manuel Utilisateur de Code_Saturne               %
%                                                                    %
%                                                                    %
%                                                                    %
%%%%%%%%%%%%%%%%%%%%%%%%%%%%%%%%%%%%%%%%%%%%%%%%%%%%%%%%%%%%%%%%%%%%%%
%\documentclass[a4paper,10pt,twoside]{article}
\documentclass[a4paper,10pt,twoside]{report}
%
%%%%%%%%%%%%%%%%%%%%%%%%%%%%%%%%%%%%%%%%%%%%%%%%%%%%%%%%%%%%%%%%%%%%%%
% PACKAGES OBLIGATOIRES
\usepackage[rddoc]{noteEDF}
%
% Les options de la classe noteEDF.sty disponibles sont :
% - english (page de garde supplementaire en anglais)
% - pdftex
% Les 3 options suivantes sont exclusives mutuellement :
% - confidentiel (orientation dans le fonds documentaire et accesibilit�)
% - rddoc (orientation dans le fonds documentaire)
% - edfdoc (orientation dans le fonds documentaire)
%
%%%%%%%%%%%%%%%%%%%%%%%%%%%%%%%%%%%%%%%%%%%%%%%%%%%%%%%%%%%%%%%%%%%%%%

%
%%%%%%%%%%%%%%%%%%%%%%%%%%%%%%%%%%%%%%%%%%%%%%%%%%%%%%%%%%%%%%%%%%%%%%
% PACKAGES ET COMMANDES POUR LE DOCUMENTS PDF ET LES HYPERLIENS
\usepackage[pdftex,
            bookmarksopen=true,
            colorlinks=true,
            linkcolor=blue,
            filecolor=blue,
            urlcolor=blue,
            citecolor=blue]{hyperref}
\hypersetup{%
  pdftitle = {Code_Saturne version 1.3 Theory and Programmer's Guide},
  pdfauthor = {MFTT},
  pdfpagemode = UseOutlines
}
\pdfinfo{/CreationDate (D:20030429000000-01 00 )}
%
% Pour avoir les Thumbnails a l'ouverture du document sous ACROREAD :
% pdfpagemode = UseThumbs
%%%%%%%%%%%%%%%%%%%%%%%%%%%%%%%%%%%%%%%%%%%%%%%%%%%%%%%%%%%%%%%%%%%%%%

%
%%%%%%%%%%%%%%%%%%%%%%%%%%%%%%%%%%%%%%%%%%%%%%%%%%%%%%%%%%%%%%%%%%%%%%
% PACKAGES ET COMMANDES POUR LA BIBLIOGRAPHIE SI BIBTEX EST CHOISI
% SI L'ENVIRONNEMENT thebibliography EST CHOISI COMMENTER CES 3 LIGNES
% (ET MODIFIER LE MAKEFILE)
%\usepackage{natbib}
%\bibliographystyle{noteEDF_natbib}
%\bibpunct{[}{]}{;}{a}{,}{,}
%
%%%%%%%%%%%%%%%%%%%%%%%%%%%%%%%%%%%%%%%%%%%%%%%%%%%%%%%%%%%%%%%%%%%%%%

%
%%%%%%%%%%%%%%%%%%%%%%%%%%%%%%%%%%%%%%%%%%%%%%%%%%%%%%%%%%%%%%%%%%%%%%

%
%%%%%%%%%%%%%%%%%%%%%%%%%%%%%%%%%%%%%%%%%%%%%%%%%%%%%%%%%%%%%%%%%%%%%%
% PACKAGES POUR LA R�ALISATION D'UNE LISTE DES SYMBOLES UTILISES
%\usepackage[article,french]{symlisteEDF}
%
%%%%%%%%%%%%%%%%%%%%%%%%%%%%%%%%%%%%%%%%%%%%%%%%%%%%%%%%%%%%%%%%%%%%%%

%
%%%%%%%%%%%%%%%%%%%%%%%%%%%%%%%%%%%%%%%%%%%%%%%%%%%%%%%%%%%%%%%%%%%%%%
% PACKAGES SUPPLEMENTAIRES
% \usepackage[...]{.....}
%
% Packages charg�s par noteEDF.sty :
% graphicx, tabularx, amsmath, amssymb, wasysym,
% fancyhdr, lastpage, ifthen, setspace,
% [latin1]{inputenc}, [OT1]{fontenc}, [francais]{babel}
%
%\usepackage{times} % plus joli en pdf (fortement recommand�)
%\usepackage{here} % option [H] pour que les objets flottant ne flottent plus
%
%%%%%%%%%%%%%%%%%%%%%%%%%%%%%%%%%%%%%%%%%%%%%%%%%%%%%%%%%%%%%%%%%%%%%%

%
%%%%%%%%%%%%%%%%%%%%%%%%%%%%%%%%%%%%%%%%%%%%%%%%%%%%%%%%%%%%%%%%%%%%%%
% MACROS SUPPLEMENTAIRES
% \newcommand{/...}{...}
%
\setcounter{tocdepth}{0}
%Compteur de ``programme'' remis a jour dans les part.
\newcounter{prog}[part]
\renewcommand{\theprog}{\arabic{prog}}
\renewcommand{\thesection}{\theprog.\arabic{section}}
\renewcommand{\theequation}{\thepart.\theprog.\arabic{equation}}
\renewcommand{\thefigure}{\thepart.\theprog.\arabic{figure}}
\newcommand{\programme}[1]{%
\passepage
\refstepcounter{prog}
\stepcounter{chapter}
\setcounter{section}{0}
\setcounter{equation}{0}
\setcounter{figure}{0}
\begin{center}
\Huge \bf \theprog - \underline{Sous-programme \fort{#1}}
\end{center}
\addcontentsline{toc}{chapter}{\theprog- Sous-programme #1}}
% repertoire et extension des images
\newcommand{\repgraphics}{../graphics}
\newcommand{\extgraphics}{pdf}
%
\newcommand{\CS}{%
{\fontfamily{ppl}\fontshape{it}\selectfont Code\_Saturne}}
\newcommand{\verscs}{1.3.0}
%
\newcommand{\tildeunix}{%
{\huge$_{_{\widetilde{\ }}}$}\hspace*{0.1mm}}
%
%(Attention : \passepage ne fonctionne pas devant un \chapter (classe 'report'))
%\newcommand{\passepage}{%
%\ifthenelse{\isodd{\arabic{page}}}
%{\newpage\hspace*{6.cm}\newpage}{\newpage}}
\newcommand{\passepage}{%
\newpage\hspace*{6.cm}\ifthenelse{\isodd{\value{page}}}
{\hspace*{-6.cm}}{\newpage}}
\newcommand{\passepart}{%
\hspace*{6.cm}\ifthenelse{\isodd{\value{page}}}
{\newpage}{\hspace*{-6.cm}}}
%
\newcommand{\minititre}[1]{\bigskip\noindent \underline{\sc #1}\\}
\newcommand{\bm}[1]{\text{\boldmath $#1$ \unboldmath} \! \!} 
\newcommand{\vect}[1]{\underline{#1}}
\newcommand{\tens}[1]{\underline{\underline{#1}}}
\newcommand{\grad}{\text{g}\underline{\text{rad}}\ }
\newcommand{\ggrad}{\text{g}\underline{\underline{\text{rad}}}\ }
\newcommand{\rot}{\underline{\text{rot}}\ }
\newcommand{\dive}{\text{div}}
\newcommand{\Min}{\text{Min}}
\newcommand{\Max}{\text{Max}}
\newcommand{\nl}{\vspace{1ex}}
\newcommand{\grandN}{\mbox{I\hspace{-.15em}N}}  
\newcommand{\grandC}{\mbox{l\hspace{-.47em}C}}  
\newcommand{\mat}[1]{\underline{\textrm{#1}}}
\newcommand{\matt}[1]{\underline{\underline{\textrm{#1}}}}
\newcommand{\comp}[1]{\textrm{#1}}
\newcommand{\gradv}{\text{g}\underline{\text{rad}}\ }
\newcommand{\gradt}{\text{g}\underline{\underline{\text{rad}}}\ }
\newcommand{\divs}{\text{div}}
\newcommand{\divv}{\underline{\text{div}}}
\newcommand{\ind}[1]{\text{$_{#1}$}}
\newcommand{\etape}[1]{\vspace{0.3cm}$\bullet\ ${\bf #1}\\}
\newcommand{\fort}[1]{\texttt{#1}}
\newcommand{\var}[1]{\ensuremath{\text{\texttt{#1}}}}
%
%%%%%%%%%%%%%%%%%%%%%%%%%%%%%%%%%%%%%%%%%%%%%%%%%%%%%%%%%%%%%%%%%%%%%%

%
%%%%%%%%%%%%%%%%%%%%%%%%%%%%%%%%%%%%%%%%%%%%%%%%%%%%%%%%%%%%%%%%%%%%%%
% INFO POUR PAGES DE GARDES
\titreEDFfr{Documentation th�orique et informatique du noyau de \CS\ \verscs}
\titreEDFang{\CS\ \verscs\ Theory and Programmer's Guide}
\numeroEDF{contact: saturne-support@edf.fr}
\indiceEDF{}

%\author{SAKIZ M., ARCHAMBEAU F., {\em ET AL.}}
\author{\'Equipe de d�veloppement de \CS}
%\rqauteursEDF{* Sigle de la structure du coauteur EDF, si diff�rente de l'auteur principal.}
%\remauteursEDF{** Nom de la soci�t� (en clair) si le coauteur est ext�rieur.}
\rqauteursEDF{}
\remauteursEDF{}

%\groupeEDFfr{Expertise en bidules et trucs}
%\groupeEDFang{Very strong in something}

\docassociesEDFfr{}
\docassociesEDFang{}
\resumeEDFfr{}
\resumeEDFang{Ce document constitue la documentation th\'eorique et
informatique des parties centrales du noyau de \CS~\verscs.
La documentation est attach\'ee \`a la version du code
correspondante pour favoriser les mises \`a jour. En pratique, les
utilisateurs de \CS\
peuvent acc\'eder \`a la documentation sous
\texttt{\$CS\_HOME/doc/NOYAU/}, information qui leur est rappel\'ee par la  
commande d'information g\'en\'erale \texttt{info\_cs [theory]}.}
\accessibiliteEDFfr{LIBRE} % LIBRE, EDF-GDF ou RESTREINTE
\accessibiliteEDFang{FREE} % FREE, EDF-GDF or RESTRICTED

\actionEDF{Projet JUPITER}
\classementEDF{}
\typerapportEDF{Note technique} % Note d'�tude, Note technique
\motsclesEDF{\CS, CFD, thermohydraulique}
%
%%%%%%%%%%%%%%%%%%%%%%%%%%%%%%%%%%%%%%%%%%%%%%%%%%%%%%%%%%%%%%%%%%%%%%

%
%%%%%%%%%%%%%%%%%%%%%%%%%%%%%%%%%%%%%%%%%%%%%%%%%%%%%%%%%%%%%%%%%%%%%%
% DEBUT DU DOCUMENT
\begin{document}
\pdfgraphics
%\def\bibname{R�f�rences}
\def\contentsname{\textbf{\normalsize SOMMAIRE}\pdfbookmark[1]{Sommaire}{contents}}
\def\indexname{Index des variables principales et des mots cl\'es}

\pdfbookmark[1]{Pages de garde}{pdg}
\large
\makepdgEDF
\normalsize

\passepage
\input synthese

\passepart
\begin{center}\begin{singlespace}
\tableofcontents
\end{singlespace}\end{center}


%\passepage
%\printsymliste\
%
%%%%%%%%%%%%%%%%%%%%%%%%%%%%%%%%%%%%%%%%%%%%%%%%%%%%%%%%%%%%%%%%%%%%%%

%
%%%%%%%%%%%%%%%%%%%%%%%%%%%%%%%%%%%%%%%%%%%%%%%%%%%%%%%%%%%%%%%%%%%%%%
% CORPS DU DOCUMENT
%
\passepart
\part{Introduction}
\stepcounter{prog}
\passepage
%                      Code_Saturne version 1.3
%                      ------------------------
%
%     This file is part of the Code_Saturne Kernel, element of the
%     Code_Saturne CFD tool.
%
%     Copyright (C) 1998-2007 EDF S.A., France
%
%     contact: saturne-support@edf.fr
%
%     The Code_Saturne Kernel is free software; you can redistribute it
%     and/or modify it under the terms of the GNU General Public License
%     as published by the Free Software Foundation; either version 2 of
%     the License, or (at your option) any later version.
%
%     The Code_Saturne Kernel is distributed in the hope that it will be
%     useful, but WITHOUT ANY WARRANTY; without even the implied warranty
%     of MERCHANTABILITY or FITNESS FOR A PARTICULAR PURPOSE.  See the
%     GNU General Public License for more details.
%
%     You should have received a copy of the GNU General Public License
%     along with the Code_Saturne Kernel; if not, write to the
%     Free Software Foundation, Inc.,
%     51 Franklin St, Fifth Floor,
%     Boston, MA  02110-1301  USA
%
%-----------------------------------------------------------------------
%


\programme{navsto}

\vspace{1cm}
On s'int\'eresse \`a la r\'esolution du syst\`eme d'\'equations de Navier-Stokes
tridimensionnelles monophasiques, \`a une pression, instationnaires, en
incompressible ou faiblement dilatable, bas\'ees sur une discr\'etisation
temporelle de type Euler implicite d'ordre 1 ou Crank-Nicolson d'ordre 2 et sur
une discr\'etisation spatiale  par volumes finis colocalis\'ee.


%%%%%%%%%%%%%%%%%%%%%%%%%%%%%%%%%%
%%%%%%%%%%%%%%%%%%%%%%%%%%%%%%%%%%
\section{Fonction}
%%%%%%%%%%%%%%%%%%%%%%%%%%%%%%%%%%
%%%%%%%%%%%%%%%%%%%%%%%%%%%%%%%%%%

  Dans ce sous-programme sont calcul\'ees, \`a un pas de temps donn\'e, les
variables vitesse et pression de ce probl\`eme en proc\'edant en
deux  \'etapes issues d'une d\'ecomposition des op\'erateurs (m\'ethode \`a
pas fractionnaires).\\
Les variables sont donc suppos\'ees connues \`a
l'instant ${t^n}$ et on cherche \`a les d\'eterminer \`a l'instant\footnote{La pression est suppos�e connue � l'instant $t^{n-1+\theta}$ et recherch�e en $t^{n+\theta}$, avec $\theta=1$ ou $1/2$ suivant le sch�ma en temps consid�r�.} ${t^{n+1}}$. Soit ${\Delta {t^n} ={t^{n+1}- {t^n}}}$ le pas de temps associ\'e. Dans un premier temps, on r\'ealise l'\'etape de
pr\'ediction de la vitesse en r\'esolvant l'\'equation de quantit\'e de
mouvement avec une pression explicite. Suit l'\'etape de correction de la
pression (ou projection de la vitesse) qui permet d'obtenir un champ de vitesse \`a divergence nulle.\\\\
Les \'equations en continu sont donc :
\begin{equation}
\left\{\begin{array}{l}
\displaystyle\frac{\partial}{\partial t}(\rho \vect{u}) + \dive(\rho\, \vect{u} \otimes \vect{u})
=\dive(\tens{\sigma}) + \vect{TS} - \tens{K}\,\vect{u}\\
\dive(\rho \vect{u}) = \Gamma
\end{array}\right.
\end{equation}

%(plus tard $\frac{\partial \rho}{\partial t} + \dive(\rho \vect{u}) = \Gamma$)



avec $\rho$ la masse volumique, $\vect{u}$ le champ de vitesse,
$[\,\vect{TS}-\tens{K}\,\vect{u}\,]$ les autres termes sources ($\tens{K}$~est un
tenseur diagonal positif par d\'efinition), $\tens{\sigma}$ le tenseur
de contraintes, $\tens{\tau}$ le tenseur des contraintes visqueuses, $\mu$ la
viscosit\'e dynamique (mol\'eculaire et \'eventuellement turbulente), $\kappa$
la viscosit� de
volume (usuellement nulle et n�glig�e dans le code et dans la suite du document,
sauf en compressible),
$\tens{D}$ le tenseur taux de d\'eformation\footnote{\`A ne pas confondre, malgr\'e la m\^eme notation $D$,
avec les flux diffusifs $\vect{D}_{\,ij}$ et $\vect{D}_{\,{b}_{ik}}$ d\'ecrits par la suite dans ce
sous-programme.}, $\Gamma$ le terme source de masse.
\begin{equation}
\left\{\begin{array}{l}
\tens{\sigma} = \tens{\tau} - P\tens{Id}  \\
\tens{\tau} = 2\,\mu\ \tens{D} +\ (\kappa\ - \frac{2}{3}\mu)\  tr({\tens{D}})\
\tens{Id}  \\
\tens{D} = \frac{1}{2}(\ggrad\vect{u}+\,^{t}\ggrad\vect{u})
\end{array}\right.
\end{equation}
 \\

On rappelle la d\'efinition des notations employ\'ees\footnote{en
utilisant la convention de sommation d'Einstein.}~:
\begin{equation}\notag
\left\{\begin{array}{lll}
\left[\ggrad{\vect{a}}\right]_{ij} &=& \partial_j a_i\\
\left[\dive(\tens{\sigma})\right]_i &=& \partial_j \sigma_{ij}\\
\left[\vect{a}\otimes\vect{b}\right]_{ij} &= &
a_i\,b_j\\
\end{array}\right.
\end{equation}
et donc :
\begin{equation}\notag
\begin{array}{lll}
\left[\dive(\vect{a}\otimes\vect{b})\right]_i &= &
\partial_j (a_i\,b_j)
\end{array}
\end{equation}

\minititre{Remarque}
Dans le cas de la prise en compte d'une masse volumique variable, l'�quation de continuit� s'�crit :
$$\frac{\partial \rho}{\partial t} + \dive{\,(\rho\,\vect{u})} = \Gamma  $$
Cette �quation n'est pas prise en compte dans l'�tape de projection (on continue � r�soudre
seulement
$\displaystyle \dive(\,{\rho\,\vect{u}}) = \Gamma$) alors que le terme
$\displaystyle \frac{\partial \rho}{\partial t}$ appara\^{\i}t lors de l'�tape de pr\'ediction de la vitesse
dans le sous-programme \fort{preduv}. Si ce terme joue un r�le sensible, l'algorithme compressible
de \CS\ (qui r�sout l'�quation compl�te) est alors sans doute plus adapt�.

%                      Code_Saturne version 1.3
%                      ------------------------
%
%     This file is part of the Code_Saturne Kernel, element of the
%     Code_Saturne CFD tool.
% 
%     Copyright (C) 1998-2007 EDF S.A., France
%
%     contact: saturne-support@edf.fr
% 
%     The Code_Saturne Kernel is free software; you can redistribute it
%     and/or modify it under the terms of the GNU General Public License
%     as published by the Free Software Foundation; either version 2 of
%     the License, or (at your option) any later version.
% 
%     The Code_Saturne Kernel is distributed in the hope that it will be
%     useful, but WITHOUT ANY WARRANTY; without even the implied warranty
%     of MERCHANTABILITY or FITNESS FOR A PARTICULAR PURPOSE.  See the
%     GNU General Public License for more details.
% 
%     You should have received a copy of the GNU General Public License
%     along with the Code_Saturne Kernel; if not, write to the
%     Free Software Foundation, Inc.,
%     51 Franklin St, Fifth Floor,
%     Boston, MA  02110-1301  USA
%
%-----------------------------------------------------------------------
%
%%%%%%%%%%%%%%%%%%%%%%%%%%%%%%%%%
%%%%%%%%%%%%%%%%%%%%%%%%%%%%%%%%%%
\section{Discr\'etisation}
%%%%%%%%%%%%%%%%%%%%%%%%%%%%%%%%%%
%%%%%%%%%%%%%%%%%%%%%%%%%%%%%%%%%%

Pour utiliser la m�thode, on se place tout d'abord dans un rep�re local d�fini
de mani�re � ce que le plan $(0yz)$, o� sont inject�s les vortex, soit confondu
avec le plan d'entr�e du calcul (voir figure \ref{Base_Vortex_entree}). 

\begin{figure}[h]
\centerline{\includegraphics[height=6cm]{../Base/Vortex/Images/entree.pdf}}
\caption{\label{Base_Vortex_entree} D�finiton des diff�rentes grandeurs dans le rep�re local
correspondant � l'entr�e d'une conduite de section carr�e.} 
\end{figure}

$u$, $v$ et $w$  sont les composantes de la vitesse fluctuante (principale et
transverse) dans ce plan, et
$\displaystyle \omega(y,z) = \frac{\partial w}{\partial y}-\frac{\partial v}{\partial z}$
la vorticit� dans la direction
normale au plan d'entr�e. $\overline{U}(y,z)$ repr�sente ici la vitesse
principale moyenne impos�e par l'utilisateur dans le plan d'entr�e. 

Chaque vortex $p$ va �tre caract�ris� par sa fonction de forme $\xi$ (identique
pour tous les vortex), sa
circulation $\Gamma_p$, son rayon $\sigma_p$ et les coordonn�es $(y_p,z_p)$ du
point $P$ o� est situ� le vortex dans le plan $(0yz)$. 

Pour cela, on suppose que la vorticit� g�n�r�e par un vortex $p$ au point $M$ de
coordonn�e $(y,z)$ s'�crit : 
\begin{equation}\notag
\omega_p(y,z)= \Gamma_p \, \xi_{\sigma_p}(r)
\end{equation}
o� $r = \sqrt{(y-y_p)^2+(z-z_p)^2}$ est la distance s�parant le point $M$ du point $P$.

Dans la m�thode implant�e, la fonction de forme est de type gaussienne modifi�e :
\begin{equation}\notag
\displaystyle
\xi_\sigma (r) = \frac{1}{2\pi \sigma^2} 
\left(2 e^{-\frac{r^2}{2\sigma^2}}-1\right) e^{-\frac{r^2}{2\sigma^2}}
\end{equation}

Le champ de vitesse induit par cette distribution de vorticit� s'obtient par
inversion des deux �quations de poisson suivantes qui sont d�duites de la
condition d'incompressibilit� dans la plan\footnote{\textit{i.e}
$\displaystyle \frac{\partial v}{\partial y}+\frac{\partial w}{\partial w} = 0$} :
\begin{equation}\notag
\begin{array}{lcr}
\displaystyle
\frac{\partial \omega}{\partial y} = \Delta w
&
\text{    et    }
&
\displaystyle
\frac{\partial \omega}{\partial y} = -\Delta v
\\
\end{array}
\end{equation}

Dans le cas g�n�ral, ce syst�me peut �tre int�gr� � l'aide de la formule de Biot et Savart.

Dans le cas d'une distribution de vorticit� de type gaussienne modifi�e, les
composantes de vitesse v�rifient : 
\begin{equation}\notag
\left\{
\begin{array}{c}
\displaystyle
v_p(y,x) = - \frac{1}{2\pi} \frac{(z-z_p)}{r^2}\left(1 -
e^{-\frac{r^2}{2\sigma^2}}\right)\,e^{-\frac{r^2}{2\sigma^2}} 
\\
\displaystyle
w_p(y,z) = \frac{1}{2\pi} \frac{(y-y_p)}{r^2}\left(1 -e^{-\frac{r^2}{2\sigma^2}}
\right)\,e^{-\frac{r^2}{2\sigma^2}} 
\end{array}
\right.
\end{equation}

Ces relations s'�tendent de fa�on imm�diate au cas de $N$ vortex distincts.
Le champ de vitesse induit par la distribution de vorticit� 
\begin{equation}
\omega(y,z) = \sum_{p=1}^N \Gamma_p \, \xi_{\sigma_p}(r)
\end{equation}
vaut au point $M$ :
\begin{equation}\notag
\begin{array}{lcr}
v(x,y) = \sum_{p=1}^N \Gamma_p\, v_p(y,z) 
&
\text{    et    }
&
w(y,z) = \sum_{p=1}^N \Gamma_p\, w_p(y,z)
\\
\label{Base_Vortex_compvit}
\end{array}
\end{equation}
%================================
\subsection{Param�tres physiques}
%================================

%-------------------------------
\subsubsection{Marche en temps}
%-------------------------------
La position initiale de chaque vortex est tir�e de mani�re al�atoire. On calcul
les d�placements successifs de chacun des vortex dans le plan d'entr�e par
int�gration explicite du champ de vitesse lagrangien : 
\begin{equation}\notag
\begin{array}{lcr}
\displaystyle
\frac{dy_p}{dt} = V(y,z)
&
\text{    et    }
&
\displaystyle
\frac{dz_p}{dt} = W(y,z)
\\
\end{array}
\end{equation}
Les vortex sont alors assimil�s � des particules ponctuelles qui sont convect�es
par le champ $(V,W)$. Ce champ peut �tre impos� par des tirages al�atoires ou
bien d�duit de la vitesse induite par les vortex dans le plan d'entr�e. Dans ce
cas $V(x,y) = \overline{V}(y,z) + v (y,z)$ et $W(y,z)= \overline{W}(y,z) +
w(y,z)$ o� $\overline{V}$ et $\overline{W}$ sont les composantes de la vitesse
transverse moyenne qu'impose l'utilisateur � l'aide des fichiers de donn�es. 

%---------------------------------------------------
\subsubsection{Intensit� et dur�e de vie des vortex}
%---------------------------------------------------
Il serait possible, � partir de l'�quation de transport de la vorticit�,
d'obtenir un mod�le d'�volution pour l'intensit� du vecteur tourbillon
$\omega_p$ associ� � chacun des vortex. En pratique, on pr�f�re utiliser un
mod�le simplifi� dans lequel la circulation des tourbillons ne d�pend que de la
postion de ces derniers � l'instant consid�r�. La circulation initiale de chaque
vortex est alors obtenue � partir de la relation : 
\begin{equation}\notag
|\Gamma_p| = 4 \sqrt{\frac{\pi\,S\,k}{3N\,[2ln(3)-3ln(2)]}}
\end{equation}
o� $S$ est la surface du plan d'entr�e, $N$ le nombre de vortex, et $k$
l'�nergie cin�tique turbulente au point o� se trouve le vortex � l'instant
consid�r�. Le signe de $\Gamma_p$ correspond au sens de rotation du vortex et
est tir� al�atoirement. 

Ce param�tre est celui qui contr�le l'intensit� des fluctuations. Sa d�pendance
en $k$ exprime que, plus l'�coulement est turbulent, plus les vortex sont
intenses. La valeur de $k$ est sp�cifi�e par
l'utilisateur. Elle peut �tre constante ou impos�e � partir de profils d'�nergie
cin�tique turbulente en entr�e. 

Pour �viter que des structures trop allong�es ne se d�veloppent au niveau de
l'entr�e, l'utilisateur doit �galement sp�cifier un temps limites $\tau_p$ au
bout duquel le vortex $p$ va �tre d�truit. Ce temps $\tau_p$ peut �tre pris
constant ou estim� au moyen de la relation : 
\begin{equation}\notag
\tau_p = \frac{5 C_{\mu}k^{\frac{3}{2}}}{\varepsilon\,\overline{U}}
\end{equation}

$\overline{U}$ et $\varepsilon$ repr�sentent respectivement la vitesse moyenne
principale et la dissipation turbulente au point o� est initialement g�n�r� le
vortex ($C_{\mu}=0,09$). 
\\
Lorsque le temps �coul� depuis la cr�ation du vortex $p$ est sup�rieur �
$\tau_p$, le vortex est d�truit et un nouveau vortex g�n�r� (sa position et le
signe de $\Gamma_p$ sont tir�s de fa�on al�atoire). 

%-------------------------------- 
\subsubsection{Taille des vortex}
%--------------------------------
La taille des vortex peut �tre prise constante, ou calcul�e � partir des
relations :
\begin{equation}\notag
\begin{array}{ccc}
\displaystyle
\sigma = \frac{C_{\mu}^{\frac{3}{4}}k^{\frac{3}{2}}}{\varepsilon} 
& \text{    ou    } &
\sigma = max[L_t,L_k]
\\
\end{array}
\end{equation}
avec:
\begin{equation}\notag
\begin{array}{ccc}
\displaystyle
L_t = \sqrt{\left( \frac{5 \nu k}{\varepsilon} \right)} 
& \text{    et    } & 
\displaystyle
L_k = 200\, \left(\frac{\nu^3}{\varepsilon}\right)^{\frac{1}{4}}
\end{array}
\end{equation}
o� $\nu$, $k$ et $\varepsilon$ sont la viscosit� dynamique, l'�nergie cin�tique
turbulente et la dissipation turbulente au point o� se trouve le vortex. 

Dans tous les cas, la taille du vortex doit �tre sup�rieure � la taille des
mailles en entr�e afin que le vortex soit effectivement simul�. 

%==================================
\subsection{Conditions aux limites}
%==================================
Le champ de vitesse g�n�r� � l'aide de la relation \ref{Base_Vortex_compvit} ne tient pas
compte {\em a priori} des conditions aux limites appliqu�es sur les bords du plan
d'entr�e. Pour obtenir des valeurs de la vitesse qui soient coh�rentes sur les
fronti�res du domaine d'entr�e, des ``vortex images'', pseudo-vortex situ�s en
dehors du domaine d'entr�e, sont g�n�r�s � des positions particuli�res et leur
champ de vitesse associ� est superpos� au champ pr�c�demment calcul�.\\
Seuls les cas d'une conduite rectangulaire et d'une conduite circulaire
permettent la g�n�ration de ces pseudo-vortex.
On distingue pour cela trois types de conditions aux limites. 

\begin{figure}[h]
\centerline{\includegraphics[height=6cm]{../Base/Vortex/Images/condlimite.pdf}}
\caption{\label{Base_Vortex_condli} Principe de g�n�ration des ``vortex images'' suivant le
type de conditions aux limites dans une conduite carr�e.} 
\end{figure}

%----------------------------------
\subsubsection{Condition de paroi}
%----------------------------------
On cr�e, pour chaque vortex $P$ contenu dans le plan d'entr�e, un vortex image
$P'$ identique � $P$ (\textit{i.e} de m�me caract�ristiques) et sym�trique de $P$
par rapport au
point $J$ ($J$ �tant la projection orthogonalement � la paroi du point $M$
correspondant au centre de la face o� l'on cherche � calculer la vitesse). La
figure \ref{Base_Vortex_condli} illustre la technique dans le cas d'une conduite
carr�e. Dans ce cas les coordonn�es du vortex situ� en $P'$ v�rifient
$(y_{p'}+y_{p})/2 = y_{J}$ et $(z_{p'}+ z_{p})/2 = z_{J}$. Le champ de vitesse
per�u depuis le point $M$ au niveau du point $J$ est nul, ce qui est bien
l'effet recherch�. 

%------------------------------------
\subsubsection{Condition de sym�trie}
%-------------------------------------
La technique est identique � celle utilis�e pour les conditions de paroi, mais
seule la composante pour la vitesse normale au bord est modifi�e dans ce cas. 

%---------------------------------------
\subsubsection{Condition de p�riodicit�}
%---------------------------------------
On cr�e pour chaque vortex, un vortex images $P'$ identique � $P$ mais translat�
d'une quantit� $L$ correspondant � la longueur qui s�pare les deux plans de la
section d'entr�e o� sont appliqu�es les conditions de p�riodicit�. Dans le cas
o� il y a deux directions de p�riodicit�, on cr�e deux vortex image.

%=============================================
\subsection{Composante de vitesse principale}
%=============================================
La m�thode des vortex ne g�n�rant pas de fluctuation $u$ de la vitesse dans la
direction principale, la derni�re composante est calcul�e � partir d'une
�quation de Langevin. Les coefficients de cette �quation sont d�termin�s par
identification des expressions obtenues pour les contraintes de Reynolds en
$R_{ij}-\varepsilon$. Dans le cas d'un �coulement en canal plan, cette technique
conduit � l'�quation : 
\begin{equation}\notag
\displaystyle
\frac{du}{dt} = - \frac{C_1}{2T} u + \left(\frac{2}{3}C_2-1\right)\frac{\partial
U}{\partial y} v + \sqrt{C_0\varepsilon} dW_i 
\end{equation}

avec $\displaystyle T=\frac{k}{\varepsilon}$, $C_1 = 1,8$, $C_2 = 0,6$,
$C_0=\frac{14}{15}$, et $dW_i$ une variable al�toire Gaussienne de variance
$\sqrt{dt}$. 

En pratique, l'�quation de Langevin n'am�liore pas vraiment les r�sultats. Elle
n'est utilis�e que dans le cas de conduites circulaires. 

%                      Code_Saturne version 1.3
%                      ------------------------
%
%     This file is part of the Code_Saturne Kernel, element of the
%     Code_Saturne CFD tool.
%
%     Copyright (C) 1998-2007 EDF S.A., France
%
%     contact: saturne-support@edf.fr
%
%     The Code_Saturne Kernel is free software; you can redistribute it
%     and/or modify it under the terms of the GNU General Public License
%     as published by the Free Software Foundation; either version 2 of
%     the License, or (at your option) any later version.
%
%     The Code_Saturne Kernel is distributed in the hope that it will be
%     useful, but WITHOUT ANY WARRANTY; without even the implied warranty
%     of MERCHANTABILITY or FITNESS FOR A PARTICULAR PURPOSE.  See the
%     GNU General Public License for more details.
%
%     You should have received a copy of the GNU General Public License
%     along with the Code_Saturne Kernel; if not, write to the
%     Free Software Foundation, Inc.,
%     51 Franklin St, Fifth Floor,
%     Boston, MA  02110-1301  USA
%
%-----------------------------------------------------------------------
%

%%%%%%%%%%%%%%%%%%%%%%%%%%%%%%%%%%
%%%%%%%%%%%%%%%%%%%%%%%%%%%%%%%%%%
\section{Mise en \oe uvre}
%%%%%%%%%%%%%%%%%%%%%%%%%%%%%%%%%%
%%%%%%%%%%%%%%%%%%%%%%%%%%%%%%%%%%
Le syst\`eme (\ref{Cfbl_Cfmsvl_eq_densite_finale_cfmsvl}) est r\'esolu par une m\'ethode
d'incr\'ement et r\'esidu en utilisant
une m\'ethode de Jacobi pour inverser le syst\`eme si le terme convectif
est implicite et en utilisant une m\'ethode de gradient conjugu\'e
si le terme convectif est explicite (qui est le cas par d�faut).

Attention, les valeurs du flux de masse $\rho\,\vect{w}\cdot\vect{S}$ et
de la viscosit\'e $\Delta\,t\,c^2\frac{S}{d}$ aux faces de
bord, qui sont calcul\'ees dans \fort{cfmsfl} et \fort{cfmsvs} respectivement,
sont modifi\'ees imm\'ediatement apr\`es l'appel \`a ces sous-programmes.
En effet, il est indispensable que la contribution de bord de
$\left(\rho\,\vect{w}-\Delta\,t\,(c^2)\,\gradv\,\rho\right)\cdot\vect{S}$
repr\'esente exactement $\vect{Q}_{ac}\cdot\vect{S}$.
Pour cela,
\begin{itemize}
\item imm\'ediatement apr\`es l'appel \`a
\fort{cfmsfl}, on remplace la contribution de bord de
$\rho\,\vect{w}\cdot\vect{S}$
par le flux de masse exact, $\vect{Q}_{ac}\cdot\vect{S}$,
d\'etermin\'e \`a partir des conditions aux limites,
\item puis, imm\'ediatement apr\`es l'appel \`a
\fort{cfmsvs}, on annule la viscosit\'e au bord $\Delta\,t\,(c^2)$ pour
\'eliminer la contribution de $-\Delta\,t\,(c^2)\,(\gradv\,\rho)\cdot\vect{S}$
(l'annulation de la viscosit\'e n'est pas probl\'ematique pour la matrice,
puisqu'elle porte sur des incr\'ements).
\end{itemize}

\bigskip

Une fois qu'on a obtenu $\rho^{n+1}$,
on peut actualiser le flux de masse acoustique
aux faces $(\vect{Q}_{ac}^{n+1})_{ij} \cdot \vect{S}_{ij}$,
qui servira pour la convection des autres variables~:
\begin{equation}\label{Cfbl_Cfmsvl_eq_flux_masse_acoustique_cfmsvl}
\displaystyle(\vect{Q}_{ac}^{n+1})_{ij}\cdot\vect{S}_{ij}=
-\left(\Delta t^n (c^2)^n \gradv(\rho^{n+1})\right)_{ij}\cdot\vect{S}_{ij}
+\left(\rho^{n+\frac{1}{2}} \vect{w}^n\right)_{ij}\cdot\vect{S}_{ij}\\
\end{equation}
Ce calcul de flux est r\'ealis\'e par \fort{cfbsc3}.
Si l'on a choisi l'algorithme standard, \'equation~(\ref{Cfbl_Cfmsvl_eq_densite_cfmsvl}),
on compl\`ete le flux dans \fort{cfmsvl} imm\'ediatement apr\`es l'appel
\`a \fort{cfbsc3}.
En effet, dans ce cas,
la convection est explicite ($\rho^{n+\frac{1}{2}}=\rho^{n}$,
obtenu en imposant \var{ICONV(ISCA(IRHO(IPHAS)))=0})
et le sous-programme \fort{cfbsc3},
qui calcule le flux de masse aux faces,
ne prend pas en compte la contribution du terme
$\rho^{n+\frac{1}{2}}\,\vect{w}^n\cdot\vect{S}$. On ajoute donc cette
contribution dans \fort{cfmsvl}, apr\`es l'appel \`a \fort{cfbsc3}.
Au bord, en particulier, c'est bien le flux de masse calcul\'e \`a partir
des conditions aux limites que l'on obtient.

On actualise la pression \`a la fin de l'\'etape, en utilisant la loi d'\'etat~:
\begin{equation}
\displaystyle P_i^{pred}=P(\rho_i^{n+1},\varepsilon_i^{n})
\end{equation}


%%%%%%%%%%%%%%%%%%%%%%%%%%%%%%%%%%
%%%%%%%%%%%%%%%%%%%%%%%%%%%%%%%%%%
\section{Points \`a traiter}
%%%%%%%%%%%%%%%%%%%%%%%%%%%%%%%%%%
%%%%%%%%%%%%%%%%%%%%%%%%%%%%%%%%%%
Le calcul du flux de masse au  bord n'est pas enti\`erement satisfaisant
si la convection est trait\'ee de mani\`ere implicite
(algorithme non standard, non test\'e,
associ\'e \`a l'\'equation~(\ref{Cfbl_Cfmsvl_eq_densite_bis_cfmsvl}),
correspondant au choix $\rho^{n+\frac{1}{2}}=\rho^{n+1}$ et
obtenu en imposant \var{ICONV(ISCA(IRHO(IPHAS)))=1}).
En effet, apr\`es \fort{cfmsfl}, il faut d\'eterminer la vitesse de
convection $\vect{w}^n$ pour qu'apparaisse
$\rho^{n+1} \vect{w}^n\cdot\vect{n}$
au cours de la r\'esolution par \fort{codits}. De ce fait, on doit d\'eduire
une valeur de $\vect{w}^n$ \`a partir de la valeur
du flux de masse. Au bord, en particulier, il faut
donc diviser le flux de masse
issu des conditions aux limites par la valeur de bord de $\rho^{n+1}$.
Or, lorsque des conditions de Neumann sont appliqu\'ees \`a la
masse volumique,
la valeur de $\rho^{n+1}$ au bord n'est pas connue avant la r\'esolution du
syst\`eme. On utilise donc, au lieu de la valeur de bord inconnue de
$\rho^{n+1}$ la valeur de bord prise au pas de temps
pr\'ec\'edent $\rho^{n}$. Cette approximation est susceptible
d'affecter la valeur du flux de masse au bord.

\passepage
\part{Module de base}
%                      Code_Saturne version 1.3
%                      ------------------------
%
%     This file is part of the Code_Saturne Kernel, element of the
%     Code_Saturne CFD tool.
%
%     Copyright (C) 1998-2007 EDF S.A., France
%
%     contact: saturne-support@edf.fr
%
%     The Code_Saturne Kernel is free software; you can redistribute it
%     and/or modify it under the terms of the GNU General Public License
%     as published by the Free Software Foundation; either version 2 of
%     the License, or (at your option) any later version.
%
%     The Code_Saturne Kernel is distributed in the hope that it will be
%     useful, but WITHOUT ANY WARRANTY; without even the implied warranty
%     of MERCHANTABILITY or FITNESS FOR A PARTICULAR PURPOSE.  See the
%     GNU General Public License for more details.
%
%     You should have received a copy of the GNU General Public License
%     along with the Code_Saturne Kernel; if not, write to the
%     Free Software Foundation, Inc.,
%     51 Franklin St, Fifth Floor,
%     Boston, MA  02110-1301  USA
%
%-----------------------------------------------------------------------
%


\programme{navsto}

\vspace{1cm}
On s'int\'eresse \`a la r\'esolution du syst\`eme d'\'equations de Navier-Stokes
tridimensionnelles monophasiques, \`a une pression, instationnaires, en
incompressible ou faiblement dilatable, bas\'ees sur une discr\'etisation
temporelle de type Euler implicite d'ordre 1 ou Crank-Nicolson d'ordre 2 et sur
une discr\'etisation spatiale  par volumes finis colocalis\'ee.


%%%%%%%%%%%%%%%%%%%%%%%%%%%%%%%%%%
%%%%%%%%%%%%%%%%%%%%%%%%%%%%%%%%%%
\section{Fonction}
%%%%%%%%%%%%%%%%%%%%%%%%%%%%%%%%%%
%%%%%%%%%%%%%%%%%%%%%%%%%%%%%%%%%%

  Dans ce sous-programme sont calcul\'ees, \`a un pas de temps donn\'e, les
variables vitesse et pression de ce probl\`eme en proc\'edant en
deux  \'etapes issues d'une d\'ecomposition des op\'erateurs (m\'ethode \`a
pas fractionnaires).\\
Les variables sont donc suppos\'ees connues \`a
l'instant ${t^n}$ et on cherche \`a les d\'eterminer \`a l'instant\footnote{La pression est suppos�e connue � l'instant $t^{n-1+\theta}$ et recherch�e en $t^{n+\theta}$, avec $\theta=1$ ou $1/2$ suivant le sch�ma en temps consid�r�.} ${t^{n+1}}$. Soit ${\Delta {t^n} ={t^{n+1}- {t^n}}}$ le pas de temps associ\'e. Dans un premier temps, on r\'ealise l'\'etape de
pr\'ediction de la vitesse en r\'esolvant l'\'equation de quantit\'e de
mouvement avec une pression explicite. Suit l'\'etape de correction de la
pression (ou projection de la vitesse) qui permet d'obtenir un champ de vitesse \`a divergence nulle.\\\\
Les \'equations en continu sont donc :
\begin{equation}
\left\{\begin{array}{l}
\displaystyle\frac{\partial}{\partial t}(\rho \vect{u}) + \dive(\rho\, \vect{u} \otimes \vect{u})
=\dive(\tens{\sigma}) + \vect{TS} - \tens{K}\,\vect{u}\\
\dive(\rho \vect{u}) = \Gamma
\end{array}\right.
\end{equation}

%(plus tard $\frac{\partial \rho}{\partial t} + \dive(\rho \vect{u}) = \Gamma$)



avec $\rho$ la masse volumique, $\vect{u}$ le champ de vitesse,
$[\,\vect{TS}-\tens{K}\,\vect{u}\,]$ les autres termes sources ($\tens{K}$~est un
tenseur diagonal positif par d\'efinition), $\tens{\sigma}$ le tenseur
de contraintes, $\tens{\tau}$ le tenseur des contraintes visqueuses, $\mu$ la
viscosit\'e dynamique (mol\'eculaire et \'eventuellement turbulente), $\kappa$
la viscosit� de
volume (usuellement nulle et n�glig�e dans le code et dans la suite du document,
sauf en compressible),
$\tens{D}$ le tenseur taux de d\'eformation\footnote{\`A ne pas confondre, malgr\'e la m\^eme notation $D$,
avec les flux diffusifs $\vect{D}_{\,ij}$ et $\vect{D}_{\,{b}_{ik}}$ d\'ecrits par la suite dans ce
sous-programme.}, $\Gamma$ le terme source de masse.
\begin{equation}
\left\{\begin{array}{l}
\tens{\sigma} = \tens{\tau} - P\tens{Id}  \\
\tens{\tau} = 2\,\mu\ \tens{D} +\ (\kappa\ - \frac{2}{3}\mu)\  tr({\tens{D}})\
\tens{Id}  \\
\tens{D} = \frac{1}{2}(\ggrad\vect{u}+\,^{t}\ggrad\vect{u})
\end{array}\right.
\end{equation}
 \\

On rappelle la d\'efinition des notations employ\'ees\footnote{en
utilisant la convention de sommation d'Einstein.}~:
\begin{equation}\notag
\left\{\begin{array}{lll}
\left[\ggrad{\vect{a}}\right]_{ij} &=& \partial_j a_i\\
\left[\dive(\tens{\sigma})\right]_i &=& \partial_j \sigma_{ij}\\
\left[\vect{a}\otimes\vect{b}\right]_{ij} &= &
a_i\,b_j\\
\end{array}\right.
\end{equation}
et donc :
\begin{equation}\notag
\begin{array}{lll}
\left[\dive(\vect{a}\otimes\vect{b})\right]_i &= &
\partial_j (a_i\,b_j)
\end{array}
\end{equation}

\minititre{Remarque}
Dans le cas de la prise en compte d'une masse volumique variable, l'�quation de continuit� s'�crit :
$$\frac{\partial \rho}{\partial t} + \dive{\,(\rho\,\vect{u})} = \Gamma  $$
Cette �quation n'est pas prise en compte dans l'�tape de projection (on continue � r�soudre
seulement
$\displaystyle \dive(\,{\rho\,\vect{u}}) = \Gamma$) alors que le terme
$\displaystyle \frac{\partial \rho}{\partial t}$ appara\^{\i}t lors de l'�tape de pr\'ediction de la vitesse
dans le sous-programme \fort{preduv}. Si ce terme joue un r�le sensible, l'algorithme compressible
de \CS\ (qui r�sout l'�quation compl�te) est alors sans doute plus adapt�.

%                      Code_Saturne version 1.3
%                      ------------------------
%
%     This file is part of the Code_Saturne Kernel, element of the
%     Code_Saturne CFD tool.
% 
%     Copyright (C) 1998-2007 EDF S.A., France
%
%     contact: saturne-support@edf.fr
% 
%     The Code_Saturne Kernel is free software; you can redistribute it
%     and/or modify it under the terms of the GNU General Public License
%     as published by the Free Software Foundation; either version 2 of
%     the License, or (at your option) any later version.
% 
%     The Code_Saturne Kernel is distributed in the hope that it will be
%     useful, but WITHOUT ANY WARRANTY; without even the implied warranty
%     of MERCHANTABILITY or FITNESS FOR A PARTICULAR PURPOSE.  See the
%     GNU General Public License for more details.
% 
%     You should have received a copy of the GNU General Public License
%     along with the Code_Saturne Kernel; if not, write to the
%     Free Software Foundation, Inc.,
%     51 Franklin St, Fifth Floor,
%     Boston, MA  02110-1301  USA
%
%-----------------------------------------------------------------------
%
%%%%%%%%%%%%%%%%%%%%%%%%%%%%%%%%%
%%%%%%%%%%%%%%%%%%%%%%%%%%%%%%%%%%
\section{Discr\'etisation}
%%%%%%%%%%%%%%%%%%%%%%%%%%%%%%%%%%
%%%%%%%%%%%%%%%%%%%%%%%%%%%%%%%%%%

Pour utiliser la m�thode, on se place tout d'abord dans un rep�re local d�fini
de mani�re � ce que le plan $(0yz)$, o� sont inject�s les vortex, soit confondu
avec le plan d'entr�e du calcul (voir figure \ref{Base_Vortex_entree}). 

\begin{figure}[h]
\centerline{\includegraphics[height=6cm]{../Base/Vortex/Images/entree.pdf}}
\caption{\label{Base_Vortex_entree} D�finiton des diff�rentes grandeurs dans le rep�re local
correspondant � l'entr�e d'une conduite de section carr�e.} 
\end{figure}

$u$, $v$ et $w$  sont les composantes de la vitesse fluctuante (principale et
transverse) dans ce plan, et
$\displaystyle \omega(y,z) = \frac{\partial w}{\partial y}-\frac{\partial v}{\partial z}$
la vorticit� dans la direction
normale au plan d'entr�e. $\overline{U}(y,z)$ repr�sente ici la vitesse
principale moyenne impos�e par l'utilisateur dans le plan d'entr�e. 

Chaque vortex $p$ va �tre caract�ris� par sa fonction de forme $\xi$ (identique
pour tous les vortex), sa
circulation $\Gamma_p$, son rayon $\sigma_p$ et les coordonn�es $(y_p,z_p)$ du
point $P$ o� est situ� le vortex dans le plan $(0yz)$. 

Pour cela, on suppose que la vorticit� g�n�r�e par un vortex $p$ au point $M$ de
coordonn�e $(y,z)$ s'�crit : 
\begin{equation}\notag
\omega_p(y,z)= \Gamma_p \, \xi_{\sigma_p}(r)
\end{equation}
o� $r = \sqrt{(y-y_p)^2+(z-z_p)^2}$ est la distance s�parant le point $M$ du point $P$.

Dans la m�thode implant�e, la fonction de forme est de type gaussienne modifi�e :
\begin{equation}\notag
\displaystyle
\xi_\sigma (r) = \frac{1}{2\pi \sigma^2} 
\left(2 e^{-\frac{r^2}{2\sigma^2}}-1\right) e^{-\frac{r^2}{2\sigma^2}}
\end{equation}

Le champ de vitesse induit par cette distribution de vorticit� s'obtient par
inversion des deux �quations de poisson suivantes qui sont d�duites de la
condition d'incompressibilit� dans la plan\footnote{\textit{i.e}
$\displaystyle \frac{\partial v}{\partial y}+\frac{\partial w}{\partial w} = 0$} :
\begin{equation}\notag
\begin{array}{lcr}
\displaystyle
\frac{\partial \omega}{\partial y} = \Delta w
&
\text{    et    }
&
\displaystyle
\frac{\partial \omega}{\partial y} = -\Delta v
\\
\end{array}
\end{equation}

Dans le cas g�n�ral, ce syst�me peut �tre int�gr� � l'aide de la formule de Biot et Savart.

Dans le cas d'une distribution de vorticit� de type gaussienne modifi�e, les
composantes de vitesse v�rifient : 
\begin{equation}\notag
\left\{
\begin{array}{c}
\displaystyle
v_p(y,x) = - \frac{1}{2\pi} \frac{(z-z_p)}{r^2}\left(1 -
e^{-\frac{r^2}{2\sigma^2}}\right)\,e^{-\frac{r^2}{2\sigma^2}} 
\\
\displaystyle
w_p(y,z) = \frac{1}{2\pi} \frac{(y-y_p)}{r^2}\left(1 -e^{-\frac{r^2}{2\sigma^2}}
\right)\,e^{-\frac{r^2}{2\sigma^2}} 
\end{array}
\right.
\end{equation}

Ces relations s'�tendent de fa�on imm�diate au cas de $N$ vortex distincts.
Le champ de vitesse induit par la distribution de vorticit� 
\begin{equation}
\omega(y,z) = \sum_{p=1}^N \Gamma_p \, \xi_{\sigma_p}(r)
\end{equation}
vaut au point $M$ :
\begin{equation}\notag
\begin{array}{lcr}
v(x,y) = \sum_{p=1}^N \Gamma_p\, v_p(y,z) 
&
\text{    et    }
&
w(y,z) = \sum_{p=1}^N \Gamma_p\, w_p(y,z)
\\
\label{Base_Vortex_compvit}
\end{array}
\end{equation}
%================================
\subsection{Param�tres physiques}
%================================

%-------------------------------
\subsubsection{Marche en temps}
%-------------------------------
La position initiale de chaque vortex est tir�e de mani�re al�atoire. On calcul
les d�placements successifs de chacun des vortex dans le plan d'entr�e par
int�gration explicite du champ de vitesse lagrangien : 
\begin{equation}\notag
\begin{array}{lcr}
\displaystyle
\frac{dy_p}{dt} = V(y,z)
&
\text{    et    }
&
\displaystyle
\frac{dz_p}{dt} = W(y,z)
\\
\end{array}
\end{equation}
Les vortex sont alors assimil�s � des particules ponctuelles qui sont convect�es
par le champ $(V,W)$. Ce champ peut �tre impos� par des tirages al�atoires ou
bien d�duit de la vitesse induite par les vortex dans le plan d'entr�e. Dans ce
cas $V(x,y) = \overline{V}(y,z) + v (y,z)$ et $W(y,z)= \overline{W}(y,z) +
w(y,z)$ o� $\overline{V}$ et $\overline{W}$ sont les composantes de la vitesse
transverse moyenne qu'impose l'utilisateur � l'aide des fichiers de donn�es. 

%---------------------------------------------------
\subsubsection{Intensit� et dur�e de vie des vortex}
%---------------------------------------------------
Il serait possible, � partir de l'�quation de transport de la vorticit�,
d'obtenir un mod�le d'�volution pour l'intensit� du vecteur tourbillon
$\omega_p$ associ� � chacun des vortex. En pratique, on pr�f�re utiliser un
mod�le simplifi� dans lequel la circulation des tourbillons ne d�pend que de la
postion de ces derniers � l'instant consid�r�. La circulation initiale de chaque
vortex est alors obtenue � partir de la relation : 
\begin{equation}\notag
|\Gamma_p| = 4 \sqrt{\frac{\pi\,S\,k}{3N\,[2ln(3)-3ln(2)]}}
\end{equation}
o� $S$ est la surface du plan d'entr�e, $N$ le nombre de vortex, et $k$
l'�nergie cin�tique turbulente au point o� se trouve le vortex � l'instant
consid�r�. Le signe de $\Gamma_p$ correspond au sens de rotation du vortex et
est tir� al�atoirement. 

Ce param�tre est celui qui contr�le l'intensit� des fluctuations. Sa d�pendance
en $k$ exprime que, plus l'�coulement est turbulent, plus les vortex sont
intenses. La valeur de $k$ est sp�cifi�e par
l'utilisateur. Elle peut �tre constante ou impos�e � partir de profils d'�nergie
cin�tique turbulente en entr�e. 

Pour �viter que des structures trop allong�es ne se d�veloppent au niveau de
l'entr�e, l'utilisateur doit �galement sp�cifier un temps limites $\tau_p$ au
bout duquel le vortex $p$ va �tre d�truit. Ce temps $\tau_p$ peut �tre pris
constant ou estim� au moyen de la relation : 
\begin{equation}\notag
\tau_p = \frac{5 C_{\mu}k^{\frac{3}{2}}}{\varepsilon\,\overline{U}}
\end{equation}

$\overline{U}$ et $\varepsilon$ repr�sentent respectivement la vitesse moyenne
principale et la dissipation turbulente au point o� est initialement g�n�r� le
vortex ($C_{\mu}=0,09$). 
\\
Lorsque le temps �coul� depuis la cr�ation du vortex $p$ est sup�rieur �
$\tau_p$, le vortex est d�truit et un nouveau vortex g�n�r� (sa position et le
signe de $\Gamma_p$ sont tir�s de fa�on al�atoire). 

%-------------------------------- 
\subsubsection{Taille des vortex}
%--------------------------------
La taille des vortex peut �tre prise constante, ou calcul�e � partir des
relations :
\begin{equation}\notag
\begin{array}{ccc}
\displaystyle
\sigma = \frac{C_{\mu}^{\frac{3}{4}}k^{\frac{3}{2}}}{\varepsilon} 
& \text{    ou    } &
\sigma = max[L_t,L_k]
\\
\end{array}
\end{equation}
avec:
\begin{equation}\notag
\begin{array}{ccc}
\displaystyle
L_t = \sqrt{\left( \frac{5 \nu k}{\varepsilon} \right)} 
& \text{    et    } & 
\displaystyle
L_k = 200\, \left(\frac{\nu^3}{\varepsilon}\right)^{\frac{1}{4}}
\end{array}
\end{equation}
o� $\nu$, $k$ et $\varepsilon$ sont la viscosit� dynamique, l'�nergie cin�tique
turbulente et la dissipation turbulente au point o� se trouve le vortex. 

Dans tous les cas, la taille du vortex doit �tre sup�rieure � la taille des
mailles en entr�e afin que le vortex soit effectivement simul�. 

%==================================
\subsection{Conditions aux limites}
%==================================
Le champ de vitesse g�n�r� � l'aide de la relation \ref{Base_Vortex_compvit} ne tient pas
compte {\em a priori} des conditions aux limites appliqu�es sur les bords du plan
d'entr�e. Pour obtenir des valeurs de la vitesse qui soient coh�rentes sur les
fronti�res du domaine d'entr�e, des ``vortex images'', pseudo-vortex situ�s en
dehors du domaine d'entr�e, sont g�n�r�s � des positions particuli�res et leur
champ de vitesse associ� est superpos� au champ pr�c�demment calcul�.\\
Seuls les cas d'une conduite rectangulaire et d'une conduite circulaire
permettent la g�n�ration de ces pseudo-vortex.
On distingue pour cela trois types de conditions aux limites. 

\begin{figure}[h]
\centerline{\includegraphics[height=6cm]{../Base/Vortex/Images/condlimite.pdf}}
\caption{\label{Base_Vortex_condli} Principe de g�n�ration des ``vortex images'' suivant le
type de conditions aux limites dans une conduite carr�e.} 
\end{figure}

%----------------------------------
\subsubsection{Condition de paroi}
%----------------------------------
On cr�e, pour chaque vortex $P$ contenu dans le plan d'entr�e, un vortex image
$P'$ identique � $P$ (\textit{i.e} de m�me caract�ristiques) et sym�trique de $P$
par rapport au
point $J$ ($J$ �tant la projection orthogonalement � la paroi du point $M$
correspondant au centre de la face o� l'on cherche � calculer la vitesse). La
figure \ref{Base_Vortex_condli} illustre la technique dans le cas d'une conduite
carr�e. Dans ce cas les coordonn�es du vortex situ� en $P'$ v�rifient
$(y_{p'}+y_{p})/2 = y_{J}$ et $(z_{p'}+ z_{p})/2 = z_{J}$. Le champ de vitesse
per�u depuis le point $M$ au niveau du point $J$ est nul, ce qui est bien
l'effet recherch�. 

%------------------------------------
\subsubsection{Condition de sym�trie}
%-------------------------------------
La technique est identique � celle utilis�e pour les conditions de paroi, mais
seule la composante pour la vitesse normale au bord est modifi�e dans ce cas. 

%---------------------------------------
\subsubsection{Condition de p�riodicit�}
%---------------------------------------
On cr�e pour chaque vortex, un vortex images $P'$ identique � $P$ mais translat�
d'une quantit� $L$ correspondant � la longueur qui s�pare les deux plans de la
section d'entr�e o� sont appliqu�es les conditions de p�riodicit�. Dans le cas
o� il y a deux directions de p�riodicit�, on cr�e deux vortex image.

%=============================================
\subsection{Composante de vitesse principale}
%=============================================
La m�thode des vortex ne g�n�rant pas de fluctuation $u$ de la vitesse dans la
direction principale, la derni�re composante est calcul�e � partir d'une
�quation de Langevin. Les coefficients de cette �quation sont d�termin�s par
identification des expressions obtenues pour les contraintes de Reynolds en
$R_{ij}-\varepsilon$. Dans le cas d'un �coulement en canal plan, cette technique
conduit � l'�quation : 
\begin{equation}\notag
\displaystyle
\frac{du}{dt} = - \frac{C_1}{2T} u + \left(\frac{2}{3}C_2-1\right)\frac{\partial
U}{\partial y} v + \sqrt{C_0\varepsilon} dW_i 
\end{equation}

avec $\displaystyle T=\frac{k}{\varepsilon}$, $C_1 = 1,8$, $C_2 = 0,6$,
$C_0=\frac{14}{15}$, et $dW_i$ une variable al�toire Gaussienne de variance
$\sqrt{dt}$. 

En pratique, l'�quation de Langevin n'am�liore pas vraiment les r�sultats. Elle
n'est utilis�e que dans le cas de conduites circulaires. 

%                      Code_Saturne version 1.3
%                      ------------------------
%
%     This file is part of the Code_Saturne Kernel, element of the
%     Code_Saturne CFD tool.
%
%     Copyright (C) 1998-2007 EDF S.A., France
%
%     contact: saturne-support@edf.fr
%
%     The Code_Saturne Kernel is free software; you can redistribute it
%     and/or modify it under the terms of the GNU General Public License
%     as published by the Free Software Foundation; either version 2 of
%     the License, or (at your option) any later version.
%
%     The Code_Saturne Kernel is distributed in the hope that it will be
%     useful, but WITHOUT ANY WARRANTY; without even the implied warranty
%     of MERCHANTABILITY or FITNESS FOR A PARTICULAR PURPOSE.  See the
%     GNU General Public License for more details.
%
%     You should have received a copy of the GNU General Public License
%     along with the Code_Saturne Kernel; if not, write to the
%     Free Software Foundation, Inc.,
%     51 Franklin St, Fifth Floor,
%     Boston, MA  02110-1301  USA
%
%-----------------------------------------------------------------------
%

%%%%%%%%%%%%%%%%%%%%%%%%%%%%%%%%%%
%%%%%%%%%%%%%%%%%%%%%%%%%%%%%%%%%%
\section{Mise en \oe uvre}
%%%%%%%%%%%%%%%%%%%%%%%%%%%%%%%%%%
%%%%%%%%%%%%%%%%%%%%%%%%%%%%%%%%%%
Le syst\`eme (\ref{Cfbl_Cfmsvl_eq_densite_finale_cfmsvl}) est r\'esolu par une m\'ethode
d'incr\'ement et r\'esidu en utilisant
une m\'ethode de Jacobi pour inverser le syst\`eme si le terme convectif
est implicite et en utilisant une m\'ethode de gradient conjugu\'e
si le terme convectif est explicite (qui est le cas par d�faut).

Attention, les valeurs du flux de masse $\rho\,\vect{w}\cdot\vect{S}$ et
de la viscosit\'e $\Delta\,t\,c^2\frac{S}{d}$ aux faces de
bord, qui sont calcul\'ees dans \fort{cfmsfl} et \fort{cfmsvs} respectivement,
sont modifi\'ees imm\'ediatement apr\`es l'appel \`a ces sous-programmes.
En effet, il est indispensable que la contribution de bord de
$\left(\rho\,\vect{w}-\Delta\,t\,(c^2)\,\gradv\,\rho\right)\cdot\vect{S}$
repr\'esente exactement $\vect{Q}_{ac}\cdot\vect{S}$.
Pour cela,
\begin{itemize}
\item imm\'ediatement apr\`es l'appel \`a
\fort{cfmsfl}, on remplace la contribution de bord de
$\rho\,\vect{w}\cdot\vect{S}$
par le flux de masse exact, $\vect{Q}_{ac}\cdot\vect{S}$,
d\'etermin\'e \`a partir des conditions aux limites,
\item puis, imm\'ediatement apr\`es l'appel \`a
\fort{cfmsvs}, on annule la viscosit\'e au bord $\Delta\,t\,(c^2)$ pour
\'eliminer la contribution de $-\Delta\,t\,(c^2)\,(\gradv\,\rho)\cdot\vect{S}$
(l'annulation de la viscosit\'e n'est pas probl\'ematique pour la matrice,
puisqu'elle porte sur des incr\'ements).
\end{itemize}

\bigskip

Une fois qu'on a obtenu $\rho^{n+1}$,
on peut actualiser le flux de masse acoustique
aux faces $(\vect{Q}_{ac}^{n+1})_{ij} \cdot \vect{S}_{ij}$,
qui servira pour la convection des autres variables~:
\begin{equation}\label{Cfbl_Cfmsvl_eq_flux_masse_acoustique_cfmsvl}
\displaystyle(\vect{Q}_{ac}^{n+1})_{ij}\cdot\vect{S}_{ij}=
-\left(\Delta t^n (c^2)^n \gradv(\rho^{n+1})\right)_{ij}\cdot\vect{S}_{ij}
+\left(\rho^{n+\frac{1}{2}} \vect{w}^n\right)_{ij}\cdot\vect{S}_{ij}\\
\end{equation}
Ce calcul de flux est r\'ealis\'e par \fort{cfbsc3}.
Si l'on a choisi l'algorithme standard, \'equation~(\ref{Cfbl_Cfmsvl_eq_densite_cfmsvl}),
on compl\`ete le flux dans \fort{cfmsvl} imm\'ediatement apr\`es l'appel
\`a \fort{cfbsc3}.
En effet, dans ce cas,
la convection est explicite ($\rho^{n+\frac{1}{2}}=\rho^{n}$,
obtenu en imposant \var{ICONV(ISCA(IRHO(IPHAS)))=0})
et le sous-programme \fort{cfbsc3},
qui calcule le flux de masse aux faces,
ne prend pas en compte la contribution du terme
$\rho^{n+\frac{1}{2}}\,\vect{w}^n\cdot\vect{S}$. On ajoute donc cette
contribution dans \fort{cfmsvl}, apr\`es l'appel \`a \fort{cfbsc3}.
Au bord, en particulier, c'est bien le flux de masse calcul\'e \`a partir
des conditions aux limites que l'on obtient.

On actualise la pression \`a la fin de l'\'etape, en utilisant la loi d'\'etat~:
\begin{equation}
\displaystyle P_i^{pred}=P(\rho_i^{n+1},\varepsilon_i^{n})
\end{equation}


%%%%%%%%%%%%%%%%%%%%%%%%%%%%%%%%%%
%%%%%%%%%%%%%%%%%%%%%%%%%%%%%%%%%%
\section{Points \`a traiter}
%%%%%%%%%%%%%%%%%%%%%%%%%%%%%%%%%%
%%%%%%%%%%%%%%%%%%%%%%%%%%%%%%%%%%
Le calcul du flux de masse au  bord n'est pas enti\`erement satisfaisant
si la convection est trait\'ee de mani\`ere implicite
(algorithme non standard, non test\'e,
associ\'e \`a l'\'equation~(\ref{Cfbl_Cfmsvl_eq_densite_bis_cfmsvl}),
correspondant au choix $\rho^{n+\frac{1}{2}}=\rho^{n+1}$ et
obtenu en imposant \var{ICONV(ISCA(IRHO(IPHAS)))=1}).
En effet, apr\`es \fort{cfmsfl}, il faut d\'eterminer la vitesse de
convection $\vect{w}^n$ pour qu'apparaisse
$\rho^{n+1} \vect{w}^n\cdot\vect{n}$
au cours de la r\'esolution par \fort{codits}. De ce fait, on doit d\'eduire
une valeur de $\vect{w}^n$ \`a partir de la valeur
du flux de masse. Au bord, en particulier, il faut
donc diviser le flux de masse
issu des conditions aux limites par la valeur de bord de $\rho^{n+1}$.
Or, lorsque des conditions de Neumann sont appliqu\'ees \`a la
masse volumique,
la valeur de $\rho^{n+1}$ au bord n'est pas connue avant la r\'esolution du
syst\`eme. On utilise donc, au lieu de la valeur de bord inconnue de
$\rho^{n+1}$ la valeur de bord prise au pas de temps
pr\'ec\'edent $\rho^{n}$. Cette approximation est susceptible
d'affecter la valeur du flux de masse au bord.

%                      Code_Saturne version 1.3
%                      ------------------------
%
%     This file is part of the Code_Saturne Kernel, element of the
%     Code_Saturne CFD tool.
%
%     Copyright (C) 1998-2007 EDF S.A., France
%
%     contact: saturne-support@edf.fr
%
%     The Code_Saturne Kernel is free software; you can redistribute it
%     and/or modify it under the terms of the GNU General Public License
%     as published by the Free Software Foundation; either version 2 of
%     the License, or (at your option) any later version.
%
%     The Code_Saturne Kernel is distributed in the hope that it will be
%     useful, but WITHOUT ANY WARRANTY; without even the implied warranty
%     of MERCHANTABILITY or FITNESS FOR A PARTICULAR PURPOSE.  See the
%     GNU General Public License for more details.
%
%     You should have received a copy of the GNU General Public License
%     along with the Code_Saturne Kernel; if not, write to the
%     Free Software Foundation, Inc.,
%     51 Franklin St, Fifth Floor,
%     Boston, MA  02110-1301  USA
%
%-----------------------------------------------------------------------
%


\programme{navsto}

\vspace{1cm}
On s'int\'eresse \`a la r\'esolution du syst\`eme d'\'equations de Navier-Stokes
tridimensionnelles monophasiques, \`a une pression, instationnaires, en
incompressible ou faiblement dilatable, bas\'ees sur une discr\'etisation
temporelle de type Euler implicite d'ordre 1 ou Crank-Nicolson d'ordre 2 et sur
une discr\'etisation spatiale  par volumes finis colocalis\'ee.


%%%%%%%%%%%%%%%%%%%%%%%%%%%%%%%%%%
%%%%%%%%%%%%%%%%%%%%%%%%%%%%%%%%%%
\section{Fonction}
%%%%%%%%%%%%%%%%%%%%%%%%%%%%%%%%%%
%%%%%%%%%%%%%%%%%%%%%%%%%%%%%%%%%%

  Dans ce sous-programme sont calcul\'ees, \`a un pas de temps donn\'e, les
variables vitesse et pression de ce probl\`eme en proc\'edant en
deux  \'etapes issues d'une d\'ecomposition des op\'erateurs (m\'ethode \`a
pas fractionnaires).\\
Les variables sont donc suppos\'ees connues \`a
l'instant ${t^n}$ et on cherche \`a les d\'eterminer \`a l'instant\footnote{La pression est suppos�e connue � l'instant $t^{n-1+\theta}$ et recherch�e en $t^{n+\theta}$, avec $\theta=1$ ou $1/2$ suivant le sch�ma en temps consid�r�.} ${t^{n+1}}$. Soit ${\Delta {t^n} ={t^{n+1}- {t^n}}}$ le pas de temps associ\'e. Dans un premier temps, on r\'ealise l'\'etape de
pr\'ediction de la vitesse en r\'esolvant l'\'equation de quantit\'e de
mouvement avec une pression explicite. Suit l'\'etape de correction de la
pression (ou projection de la vitesse) qui permet d'obtenir un champ de vitesse \`a divergence nulle.\\\\
Les \'equations en continu sont donc :
\begin{equation}
\left\{\begin{array}{l}
\displaystyle\frac{\partial}{\partial t}(\rho \vect{u}) + \dive(\rho\, \vect{u} \otimes \vect{u})
=\dive(\tens{\sigma}) + \vect{TS} - \tens{K}\,\vect{u}\\
\dive(\rho \vect{u}) = \Gamma
\end{array}\right.
\end{equation}

%(plus tard $\frac{\partial \rho}{\partial t} + \dive(\rho \vect{u}) = \Gamma$)



avec $\rho$ la masse volumique, $\vect{u}$ le champ de vitesse,
$[\,\vect{TS}-\tens{K}\,\vect{u}\,]$ les autres termes sources ($\tens{K}$~est un
tenseur diagonal positif par d\'efinition), $\tens{\sigma}$ le tenseur
de contraintes, $\tens{\tau}$ le tenseur des contraintes visqueuses, $\mu$ la
viscosit\'e dynamique (mol\'eculaire et \'eventuellement turbulente), $\kappa$
la viscosit� de
volume (usuellement nulle et n�glig�e dans le code et dans la suite du document,
sauf en compressible),
$\tens{D}$ le tenseur taux de d\'eformation\footnote{\`A ne pas confondre, malgr\'e la m\^eme notation $D$,
avec les flux diffusifs $\vect{D}_{\,ij}$ et $\vect{D}_{\,{b}_{ik}}$ d\'ecrits par la suite dans ce
sous-programme.}, $\Gamma$ le terme source de masse.
\begin{equation}
\left\{\begin{array}{l}
\tens{\sigma} = \tens{\tau} - P\tens{Id}  \\
\tens{\tau} = 2\,\mu\ \tens{D} +\ (\kappa\ - \frac{2}{3}\mu)\  tr({\tens{D}})\
\tens{Id}  \\
\tens{D} = \frac{1}{2}(\ggrad\vect{u}+\,^{t}\ggrad\vect{u})
\end{array}\right.
\end{equation}
 \\

On rappelle la d\'efinition des notations employ\'ees\footnote{en
utilisant la convention de sommation d'Einstein.}~:
\begin{equation}\notag
\left\{\begin{array}{lll}
\left[\ggrad{\vect{a}}\right]_{ij} &=& \partial_j a_i\\
\left[\dive(\tens{\sigma})\right]_i &=& \partial_j \sigma_{ij}\\
\left[\vect{a}\otimes\vect{b}\right]_{ij} &= &
a_i\,b_j\\
\end{array}\right.
\end{equation}
et donc :
\begin{equation}\notag
\begin{array}{lll}
\left[\dive(\vect{a}\otimes\vect{b})\right]_i &= &
\partial_j (a_i\,b_j)
\end{array}
\end{equation}

\minititre{Remarque}
Dans le cas de la prise en compte d'une masse volumique variable, l'�quation de continuit� s'�crit :
$$\frac{\partial \rho}{\partial t} + \dive{\,(\rho\,\vect{u})} = \Gamma  $$
Cette �quation n'est pas prise en compte dans l'�tape de projection (on continue � r�soudre
seulement
$\displaystyle \dive(\,{\rho\,\vect{u}}) = \Gamma$) alors que le terme
$\displaystyle \frac{\partial \rho}{\partial t}$ appara\^{\i}t lors de l'�tape de pr\'ediction de la vitesse
dans le sous-programme \fort{preduv}. Si ce terme joue un r�le sensible, l'algorithme compressible
de \CS\ (qui r�sout l'�quation compl�te) est alors sans doute plus adapt�.

%                      Code_Saturne version 1.3
%                      ------------------------
%
%     This file is part of the Code_Saturne Kernel, element of the
%     Code_Saturne CFD tool.
% 
%     Copyright (C) 1998-2007 EDF S.A., France
%
%     contact: saturne-support@edf.fr
% 
%     The Code_Saturne Kernel is free software; you can redistribute it
%     and/or modify it under the terms of the GNU General Public License
%     as published by the Free Software Foundation; either version 2 of
%     the License, or (at your option) any later version.
% 
%     The Code_Saturne Kernel is distributed in the hope that it will be
%     useful, but WITHOUT ANY WARRANTY; without even the implied warranty
%     of MERCHANTABILITY or FITNESS FOR A PARTICULAR PURPOSE.  See the
%     GNU General Public License for more details.
% 
%     You should have received a copy of the GNU General Public License
%     along with the Code_Saturne Kernel; if not, write to the
%     Free Software Foundation, Inc.,
%     51 Franklin St, Fifth Floor,
%     Boston, MA  02110-1301  USA
%
%-----------------------------------------------------------------------
%
%%%%%%%%%%%%%%%%%%%%%%%%%%%%%%%%%
%%%%%%%%%%%%%%%%%%%%%%%%%%%%%%%%%%
\section{Discr\'etisation}
%%%%%%%%%%%%%%%%%%%%%%%%%%%%%%%%%%
%%%%%%%%%%%%%%%%%%%%%%%%%%%%%%%%%%

Pour utiliser la m�thode, on se place tout d'abord dans un rep�re local d�fini
de mani�re � ce que le plan $(0yz)$, o� sont inject�s les vortex, soit confondu
avec le plan d'entr�e du calcul (voir figure \ref{Base_Vortex_entree}). 

\begin{figure}[h]
\centerline{\includegraphics[height=6cm]{../Base/Vortex/Images/entree.pdf}}
\caption{\label{Base_Vortex_entree} D�finiton des diff�rentes grandeurs dans le rep�re local
correspondant � l'entr�e d'une conduite de section carr�e.} 
\end{figure}

$u$, $v$ et $w$  sont les composantes de la vitesse fluctuante (principale et
transverse) dans ce plan, et
$\displaystyle \omega(y,z) = \frac{\partial w}{\partial y}-\frac{\partial v}{\partial z}$
la vorticit� dans la direction
normale au plan d'entr�e. $\overline{U}(y,z)$ repr�sente ici la vitesse
principale moyenne impos�e par l'utilisateur dans le plan d'entr�e. 

Chaque vortex $p$ va �tre caract�ris� par sa fonction de forme $\xi$ (identique
pour tous les vortex), sa
circulation $\Gamma_p$, son rayon $\sigma_p$ et les coordonn�es $(y_p,z_p)$ du
point $P$ o� est situ� le vortex dans le plan $(0yz)$. 

Pour cela, on suppose que la vorticit� g�n�r�e par un vortex $p$ au point $M$ de
coordonn�e $(y,z)$ s'�crit : 
\begin{equation}\notag
\omega_p(y,z)= \Gamma_p \, \xi_{\sigma_p}(r)
\end{equation}
o� $r = \sqrt{(y-y_p)^2+(z-z_p)^2}$ est la distance s�parant le point $M$ du point $P$.

Dans la m�thode implant�e, la fonction de forme est de type gaussienne modifi�e :
\begin{equation}\notag
\displaystyle
\xi_\sigma (r) = \frac{1}{2\pi \sigma^2} 
\left(2 e^{-\frac{r^2}{2\sigma^2}}-1\right) e^{-\frac{r^2}{2\sigma^2}}
\end{equation}

Le champ de vitesse induit par cette distribution de vorticit� s'obtient par
inversion des deux �quations de poisson suivantes qui sont d�duites de la
condition d'incompressibilit� dans la plan\footnote{\textit{i.e}
$\displaystyle \frac{\partial v}{\partial y}+\frac{\partial w}{\partial w} = 0$} :
\begin{equation}\notag
\begin{array}{lcr}
\displaystyle
\frac{\partial \omega}{\partial y} = \Delta w
&
\text{    et    }
&
\displaystyle
\frac{\partial \omega}{\partial y} = -\Delta v
\\
\end{array}
\end{equation}

Dans le cas g�n�ral, ce syst�me peut �tre int�gr� � l'aide de la formule de Biot et Savart.

Dans le cas d'une distribution de vorticit� de type gaussienne modifi�e, les
composantes de vitesse v�rifient : 
\begin{equation}\notag
\left\{
\begin{array}{c}
\displaystyle
v_p(y,x) = - \frac{1}{2\pi} \frac{(z-z_p)}{r^2}\left(1 -
e^{-\frac{r^2}{2\sigma^2}}\right)\,e^{-\frac{r^2}{2\sigma^2}} 
\\
\displaystyle
w_p(y,z) = \frac{1}{2\pi} \frac{(y-y_p)}{r^2}\left(1 -e^{-\frac{r^2}{2\sigma^2}}
\right)\,e^{-\frac{r^2}{2\sigma^2}} 
\end{array}
\right.
\end{equation}

Ces relations s'�tendent de fa�on imm�diate au cas de $N$ vortex distincts.
Le champ de vitesse induit par la distribution de vorticit� 
\begin{equation}
\omega(y,z) = \sum_{p=1}^N \Gamma_p \, \xi_{\sigma_p}(r)
\end{equation}
vaut au point $M$ :
\begin{equation}\notag
\begin{array}{lcr}
v(x,y) = \sum_{p=1}^N \Gamma_p\, v_p(y,z) 
&
\text{    et    }
&
w(y,z) = \sum_{p=1}^N \Gamma_p\, w_p(y,z)
\\
\label{Base_Vortex_compvit}
\end{array}
\end{equation}
%================================
\subsection{Param�tres physiques}
%================================

%-------------------------------
\subsubsection{Marche en temps}
%-------------------------------
La position initiale de chaque vortex est tir�e de mani�re al�atoire. On calcul
les d�placements successifs de chacun des vortex dans le plan d'entr�e par
int�gration explicite du champ de vitesse lagrangien : 
\begin{equation}\notag
\begin{array}{lcr}
\displaystyle
\frac{dy_p}{dt} = V(y,z)
&
\text{    et    }
&
\displaystyle
\frac{dz_p}{dt} = W(y,z)
\\
\end{array}
\end{equation}
Les vortex sont alors assimil�s � des particules ponctuelles qui sont convect�es
par le champ $(V,W)$. Ce champ peut �tre impos� par des tirages al�atoires ou
bien d�duit de la vitesse induite par les vortex dans le plan d'entr�e. Dans ce
cas $V(x,y) = \overline{V}(y,z) + v (y,z)$ et $W(y,z)= \overline{W}(y,z) +
w(y,z)$ o� $\overline{V}$ et $\overline{W}$ sont les composantes de la vitesse
transverse moyenne qu'impose l'utilisateur � l'aide des fichiers de donn�es. 

%---------------------------------------------------
\subsubsection{Intensit� et dur�e de vie des vortex}
%---------------------------------------------------
Il serait possible, � partir de l'�quation de transport de la vorticit�,
d'obtenir un mod�le d'�volution pour l'intensit� du vecteur tourbillon
$\omega_p$ associ� � chacun des vortex. En pratique, on pr�f�re utiliser un
mod�le simplifi� dans lequel la circulation des tourbillons ne d�pend que de la
postion de ces derniers � l'instant consid�r�. La circulation initiale de chaque
vortex est alors obtenue � partir de la relation : 
\begin{equation}\notag
|\Gamma_p| = 4 \sqrt{\frac{\pi\,S\,k}{3N\,[2ln(3)-3ln(2)]}}
\end{equation}
o� $S$ est la surface du plan d'entr�e, $N$ le nombre de vortex, et $k$
l'�nergie cin�tique turbulente au point o� se trouve le vortex � l'instant
consid�r�. Le signe de $\Gamma_p$ correspond au sens de rotation du vortex et
est tir� al�atoirement. 

Ce param�tre est celui qui contr�le l'intensit� des fluctuations. Sa d�pendance
en $k$ exprime que, plus l'�coulement est turbulent, plus les vortex sont
intenses. La valeur de $k$ est sp�cifi�e par
l'utilisateur. Elle peut �tre constante ou impos�e � partir de profils d'�nergie
cin�tique turbulente en entr�e. 

Pour �viter que des structures trop allong�es ne se d�veloppent au niveau de
l'entr�e, l'utilisateur doit �galement sp�cifier un temps limites $\tau_p$ au
bout duquel le vortex $p$ va �tre d�truit. Ce temps $\tau_p$ peut �tre pris
constant ou estim� au moyen de la relation : 
\begin{equation}\notag
\tau_p = \frac{5 C_{\mu}k^{\frac{3}{2}}}{\varepsilon\,\overline{U}}
\end{equation}

$\overline{U}$ et $\varepsilon$ repr�sentent respectivement la vitesse moyenne
principale et la dissipation turbulente au point o� est initialement g�n�r� le
vortex ($C_{\mu}=0,09$). 
\\
Lorsque le temps �coul� depuis la cr�ation du vortex $p$ est sup�rieur �
$\tau_p$, le vortex est d�truit et un nouveau vortex g�n�r� (sa position et le
signe de $\Gamma_p$ sont tir�s de fa�on al�atoire). 

%-------------------------------- 
\subsubsection{Taille des vortex}
%--------------------------------
La taille des vortex peut �tre prise constante, ou calcul�e � partir des
relations :
\begin{equation}\notag
\begin{array}{ccc}
\displaystyle
\sigma = \frac{C_{\mu}^{\frac{3}{4}}k^{\frac{3}{2}}}{\varepsilon} 
& \text{    ou    } &
\sigma = max[L_t,L_k]
\\
\end{array}
\end{equation}
avec:
\begin{equation}\notag
\begin{array}{ccc}
\displaystyle
L_t = \sqrt{\left( \frac{5 \nu k}{\varepsilon} \right)} 
& \text{    et    } & 
\displaystyle
L_k = 200\, \left(\frac{\nu^3}{\varepsilon}\right)^{\frac{1}{4}}
\end{array}
\end{equation}
o� $\nu$, $k$ et $\varepsilon$ sont la viscosit� dynamique, l'�nergie cin�tique
turbulente et la dissipation turbulente au point o� se trouve le vortex. 

Dans tous les cas, la taille du vortex doit �tre sup�rieure � la taille des
mailles en entr�e afin que le vortex soit effectivement simul�. 

%==================================
\subsection{Conditions aux limites}
%==================================
Le champ de vitesse g�n�r� � l'aide de la relation \ref{Base_Vortex_compvit} ne tient pas
compte {\em a priori} des conditions aux limites appliqu�es sur les bords du plan
d'entr�e. Pour obtenir des valeurs de la vitesse qui soient coh�rentes sur les
fronti�res du domaine d'entr�e, des ``vortex images'', pseudo-vortex situ�s en
dehors du domaine d'entr�e, sont g�n�r�s � des positions particuli�res et leur
champ de vitesse associ� est superpos� au champ pr�c�demment calcul�.\\
Seuls les cas d'une conduite rectangulaire et d'une conduite circulaire
permettent la g�n�ration de ces pseudo-vortex.
On distingue pour cela trois types de conditions aux limites. 

\begin{figure}[h]
\centerline{\includegraphics[height=6cm]{../Base/Vortex/Images/condlimite.pdf}}
\caption{\label{Base_Vortex_condli} Principe de g�n�ration des ``vortex images'' suivant le
type de conditions aux limites dans une conduite carr�e.} 
\end{figure}

%----------------------------------
\subsubsection{Condition de paroi}
%----------------------------------
On cr�e, pour chaque vortex $P$ contenu dans le plan d'entr�e, un vortex image
$P'$ identique � $P$ (\textit{i.e} de m�me caract�ristiques) et sym�trique de $P$
par rapport au
point $J$ ($J$ �tant la projection orthogonalement � la paroi du point $M$
correspondant au centre de la face o� l'on cherche � calculer la vitesse). La
figure \ref{Base_Vortex_condli} illustre la technique dans le cas d'une conduite
carr�e. Dans ce cas les coordonn�es du vortex situ� en $P'$ v�rifient
$(y_{p'}+y_{p})/2 = y_{J}$ et $(z_{p'}+ z_{p})/2 = z_{J}$. Le champ de vitesse
per�u depuis le point $M$ au niveau du point $J$ est nul, ce qui est bien
l'effet recherch�. 

%------------------------------------
\subsubsection{Condition de sym�trie}
%-------------------------------------
La technique est identique � celle utilis�e pour les conditions de paroi, mais
seule la composante pour la vitesse normale au bord est modifi�e dans ce cas. 

%---------------------------------------
\subsubsection{Condition de p�riodicit�}
%---------------------------------------
On cr�e pour chaque vortex, un vortex images $P'$ identique � $P$ mais translat�
d'une quantit� $L$ correspondant � la longueur qui s�pare les deux plans de la
section d'entr�e o� sont appliqu�es les conditions de p�riodicit�. Dans le cas
o� il y a deux directions de p�riodicit�, on cr�e deux vortex image.

%=============================================
\subsection{Composante de vitesse principale}
%=============================================
La m�thode des vortex ne g�n�rant pas de fluctuation $u$ de la vitesse dans la
direction principale, la derni�re composante est calcul�e � partir d'une
�quation de Langevin. Les coefficients de cette �quation sont d�termin�s par
identification des expressions obtenues pour les contraintes de Reynolds en
$R_{ij}-\varepsilon$. Dans le cas d'un �coulement en canal plan, cette technique
conduit � l'�quation : 
\begin{equation}\notag
\displaystyle
\frac{du}{dt} = - \frac{C_1}{2T} u + \left(\frac{2}{3}C_2-1\right)\frac{\partial
U}{\partial y} v + \sqrt{C_0\varepsilon} dW_i 
\end{equation}

avec $\displaystyle T=\frac{k}{\varepsilon}$, $C_1 = 1,8$, $C_2 = 0,6$,
$C_0=\frac{14}{15}$, et $dW_i$ une variable al�toire Gaussienne de variance
$\sqrt{dt}$. 

En pratique, l'�quation de Langevin n'am�liore pas vraiment les r�sultats. Elle
n'est utilis�e que dans le cas de conduites circulaires. 

%                      Code_Saturne version 1.3
%                      ------------------------
%
%     This file is part of the Code_Saturne Kernel, element of the
%     Code_Saturne CFD tool.
%
%     Copyright (C) 1998-2007 EDF S.A., France
%
%     contact: saturne-support@edf.fr
%
%     The Code_Saturne Kernel is free software; you can redistribute it
%     and/or modify it under the terms of the GNU General Public License
%     as published by the Free Software Foundation; either version 2 of
%     the License, or (at your option) any later version.
%
%     The Code_Saturne Kernel is distributed in the hope that it will be
%     useful, but WITHOUT ANY WARRANTY; without even the implied warranty
%     of MERCHANTABILITY or FITNESS FOR A PARTICULAR PURPOSE.  See the
%     GNU General Public License for more details.
%
%     You should have received a copy of the GNU General Public License
%     along with the Code_Saturne Kernel; if not, write to the
%     Free Software Foundation, Inc.,
%     51 Franklin St, Fifth Floor,
%     Boston, MA  02110-1301  USA
%
%-----------------------------------------------------------------------
%

%%%%%%%%%%%%%%%%%%%%%%%%%%%%%%%%%%
%%%%%%%%%%%%%%%%%%%%%%%%%%%%%%%%%%
\section{Mise en \oe uvre}
%%%%%%%%%%%%%%%%%%%%%%%%%%%%%%%%%%
%%%%%%%%%%%%%%%%%%%%%%%%%%%%%%%%%%
Le syst\`eme (\ref{Cfbl_Cfmsvl_eq_densite_finale_cfmsvl}) est r\'esolu par une m\'ethode
d'incr\'ement et r\'esidu en utilisant
une m\'ethode de Jacobi pour inverser le syst\`eme si le terme convectif
est implicite et en utilisant une m\'ethode de gradient conjugu\'e
si le terme convectif est explicite (qui est le cas par d�faut).

Attention, les valeurs du flux de masse $\rho\,\vect{w}\cdot\vect{S}$ et
de la viscosit\'e $\Delta\,t\,c^2\frac{S}{d}$ aux faces de
bord, qui sont calcul\'ees dans \fort{cfmsfl} et \fort{cfmsvs} respectivement,
sont modifi\'ees imm\'ediatement apr\`es l'appel \`a ces sous-programmes.
En effet, il est indispensable que la contribution de bord de
$\left(\rho\,\vect{w}-\Delta\,t\,(c^2)\,\gradv\,\rho\right)\cdot\vect{S}$
repr\'esente exactement $\vect{Q}_{ac}\cdot\vect{S}$.
Pour cela,
\begin{itemize}
\item imm\'ediatement apr\`es l'appel \`a
\fort{cfmsfl}, on remplace la contribution de bord de
$\rho\,\vect{w}\cdot\vect{S}$
par le flux de masse exact, $\vect{Q}_{ac}\cdot\vect{S}$,
d\'etermin\'e \`a partir des conditions aux limites,
\item puis, imm\'ediatement apr\`es l'appel \`a
\fort{cfmsvs}, on annule la viscosit\'e au bord $\Delta\,t\,(c^2)$ pour
\'eliminer la contribution de $-\Delta\,t\,(c^2)\,(\gradv\,\rho)\cdot\vect{S}$
(l'annulation de la viscosit\'e n'est pas probl\'ematique pour la matrice,
puisqu'elle porte sur des incr\'ements).
\end{itemize}

\bigskip

Une fois qu'on a obtenu $\rho^{n+1}$,
on peut actualiser le flux de masse acoustique
aux faces $(\vect{Q}_{ac}^{n+1})_{ij} \cdot \vect{S}_{ij}$,
qui servira pour la convection des autres variables~:
\begin{equation}\label{Cfbl_Cfmsvl_eq_flux_masse_acoustique_cfmsvl}
\displaystyle(\vect{Q}_{ac}^{n+1})_{ij}\cdot\vect{S}_{ij}=
-\left(\Delta t^n (c^2)^n \gradv(\rho^{n+1})\right)_{ij}\cdot\vect{S}_{ij}
+\left(\rho^{n+\frac{1}{2}} \vect{w}^n\right)_{ij}\cdot\vect{S}_{ij}\\
\end{equation}
Ce calcul de flux est r\'ealis\'e par \fort{cfbsc3}.
Si l'on a choisi l'algorithme standard, \'equation~(\ref{Cfbl_Cfmsvl_eq_densite_cfmsvl}),
on compl\`ete le flux dans \fort{cfmsvl} imm\'ediatement apr\`es l'appel
\`a \fort{cfbsc3}.
En effet, dans ce cas,
la convection est explicite ($\rho^{n+\frac{1}{2}}=\rho^{n}$,
obtenu en imposant \var{ICONV(ISCA(IRHO(IPHAS)))=0})
et le sous-programme \fort{cfbsc3},
qui calcule le flux de masse aux faces,
ne prend pas en compte la contribution du terme
$\rho^{n+\frac{1}{2}}\,\vect{w}^n\cdot\vect{S}$. On ajoute donc cette
contribution dans \fort{cfmsvl}, apr\`es l'appel \`a \fort{cfbsc3}.
Au bord, en particulier, c'est bien le flux de masse calcul\'e \`a partir
des conditions aux limites que l'on obtient.

On actualise la pression \`a la fin de l'\'etape, en utilisant la loi d'\'etat~:
\begin{equation}
\displaystyle P_i^{pred}=P(\rho_i^{n+1},\varepsilon_i^{n})
\end{equation}


%%%%%%%%%%%%%%%%%%%%%%%%%%%%%%%%%%
%%%%%%%%%%%%%%%%%%%%%%%%%%%%%%%%%%
\section{Points \`a traiter}
%%%%%%%%%%%%%%%%%%%%%%%%%%%%%%%%%%
%%%%%%%%%%%%%%%%%%%%%%%%%%%%%%%%%%
Le calcul du flux de masse au  bord n'est pas enti\`erement satisfaisant
si la convection est trait\'ee de mani\`ere implicite
(algorithme non standard, non test\'e,
associ\'e \`a l'\'equation~(\ref{Cfbl_Cfmsvl_eq_densite_bis_cfmsvl}),
correspondant au choix $\rho^{n+\frac{1}{2}}=\rho^{n+1}$ et
obtenu en imposant \var{ICONV(ISCA(IRHO(IPHAS)))=1}).
En effet, apr\`es \fort{cfmsfl}, il faut d\'eterminer la vitesse de
convection $\vect{w}^n$ pour qu'apparaisse
$\rho^{n+1} \vect{w}^n\cdot\vect{n}$
au cours de la r\'esolution par \fort{codits}. De ce fait, on doit d\'eduire
une valeur de $\vect{w}^n$ \`a partir de la valeur
du flux de masse. Au bord, en particulier, il faut
donc diviser le flux de masse
issu des conditions aux limites par la valeur de bord de $\rho^{n+1}$.
Or, lorsque des conditions de Neumann sont appliqu\'ees \`a la
masse volumique,
la valeur de $\rho^{n+1}$ au bord n'est pas connue avant la r\'esolution du
syst\`eme. On utilise donc, au lieu de la valeur de bord inconnue de
$\rho^{n+1}$ la valeur de bord prise au pas de temps
pr\'ec\'edent $\rho^{n}$. Cette approximation est susceptible
d'affecter la valeur du flux de masse au bord.

%                      Code_Saturne version 1.3
%                      ------------------------
%
%     This file is part of the Code_Saturne Kernel, element of the
%     Code_Saturne CFD tool.
%
%     Copyright (C) 1998-2007 EDF S.A., France
%
%     contact: saturne-support@edf.fr
%
%     The Code_Saturne Kernel is free software; you can redistribute it
%     and/or modify it under the terms of the GNU General Public License
%     as published by the Free Software Foundation; either version 2 of
%     the License, or (at your option) any later version.
%
%     The Code_Saturne Kernel is distributed in the hope that it will be
%     useful, but WITHOUT ANY WARRANTY; without even the implied warranty
%     of MERCHANTABILITY or FITNESS FOR A PARTICULAR PURPOSE.  See the
%     GNU General Public License for more details.
%
%     You should have received a copy of the GNU General Public License
%     along with the Code_Saturne Kernel; if not, write to the
%     Free Software Foundation, Inc.,
%     51 Franklin St, Fifth Floor,
%     Boston, MA  02110-1301  USA
%
%-----------------------------------------------------------------------
%


\programme{navsto}

\vspace{1cm}
On s'int\'eresse \`a la r\'esolution du syst\`eme d'\'equations de Navier-Stokes
tridimensionnelles monophasiques, \`a une pression, instationnaires, en
incompressible ou faiblement dilatable, bas\'ees sur une discr\'etisation
temporelle de type Euler implicite d'ordre 1 ou Crank-Nicolson d'ordre 2 et sur
une discr\'etisation spatiale  par volumes finis colocalis\'ee.


%%%%%%%%%%%%%%%%%%%%%%%%%%%%%%%%%%
%%%%%%%%%%%%%%%%%%%%%%%%%%%%%%%%%%
\section{Fonction}
%%%%%%%%%%%%%%%%%%%%%%%%%%%%%%%%%%
%%%%%%%%%%%%%%%%%%%%%%%%%%%%%%%%%%

  Dans ce sous-programme sont calcul\'ees, \`a un pas de temps donn\'e, les
variables vitesse et pression de ce probl\`eme en proc\'edant en
deux  \'etapes issues d'une d\'ecomposition des op\'erateurs (m\'ethode \`a
pas fractionnaires).\\
Les variables sont donc suppos\'ees connues \`a
l'instant ${t^n}$ et on cherche \`a les d\'eterminer \`a l'instant\footnote{La pression est suppos�e connue � l'instant $t^{n-1+\theta}$ et recherch�e en $t^{n+\theta}$, avec $\theta=1$ ou $1/2$ suivant le sch�ma en temps consid�r�.} ${t^{n+1}}$. Soit ${\Delta {t^n} ={t^{n+1}- {t^n}}}$ le pas de temps associ\'e. Dans un premier temps, on r\'ealise l'\'etape de
pr\'ediction de la vitesse en r\'esolvant l'\'equation de quantit\'e de
mouvement avec une pression explicite. Suit l'\'etape de correction de la
pression (ou projection de la vitesse) qui permet d'obtenir un champ de vitesse \`a divergence nulle.\\\\
Les \'equations en continu sont donc :
\begin{equation}
\left\{\begin{array}{l}
\displaystyle\frac{\partial}{\partial t}(\rho \vect{u}) + \dive(\rho\, \vect{u} \otimes \vect{u})
=\dive(\tens{\sigma}) + \vect{TS} - \tens{K}\,\vect{u}\\
\dive(\rho \vect{u}) = \Gamma
\end{array}\right.
\end{equation}

%(plus tard $\frac{\partial \rho}{\partial t} + \dive(\rho \vect{u}) = \Gamma$)



avec $\rho$ la masse volumique, $\vect{u}$ le champ de vitesse,
$[\,\vect{TS}-\tens{K}\,\vect{u}\,]$ les autres termes sources ($\tens{K}$~est un
tenseur diagonal positif par d\'efinition), $\tens{\sigma}$ le tenseur
de contraintes, $\tens{\tau}$ le tenseur des contraintes visqueuses, $\mu$ la
viscosit\'e dynamique (mol\'eculaire et \'eventuellement turbulente), $\kappa$
la viscosit� de
volume (usuellement nulle et n�glig�e dans le code et dans la suite du document,
sauf en compressible),
$\tens{D}$ le tenseur taux de d\'eformation\footnote{\`A ne pas confondre, malgr\'e la m\^eme notation $D$,
avec les flux diffusifs $\vect{D}_{\,ij}$ et $\vect{D}_{\,{b}_{ik}}$ d\'ecrits par la suite dans ce
sous-programme.}, $\Gamma$ le terme source de masse.
\begin{equation}
\left\{\begin{array}{l}
\tens{\sigma} = \tens{\tau} - P\tens{Id}  \\
\tens{\tau} = 2\,\mu\ \tens{D} +\ (\kappa\ - \frac{2}{3}\mu)\  tr({\tens{D}})\
\tens{Id}  \\
\tens{D} = \frac{1}{2}(\ggrad\vect{u}+\,^{t}\ggrad\vect{u})
\end{array}\right.
\end{equation}
 \\

On rappelle la d\'efinition des notations employ\'ees\footnote{en
utilisant la convention de sommation d'Einstein.}~:
\begin{equation}\notag
\left\{\begin{array}{lll}
\left[\ggrad{\vect{a}}\right]_{ij} &=& \partial_j a_i\\
\left[\dive(\tens{\sigma})\right]_i &=& \partial_j \sigma_{ij}\\
\left[\vect{a}\otimes\vect{b}\right]_{ij} &= &
a_i\,b_j\\
\end{array}\right.
\end{equation}
et donc :
\begin{equation}\notag
\begin{array}{lll}
\left[\dive(\vect{a}\otimes\vect{b})\right]_i &= &
\partial_j (a_i\,b_j)
\end{array}
\end{equation}

\minititre{Remarque}
Dans le cas de la prise en compte d'une masse volumique variable, l'�quation de continuit� s'�crit :
$$\frac{\partial \rho}{\partial t} + \dive{\,(\rho\,\vect{u})} = \Gamma  $$
Cette �quation n'est pas prise en compte dans l'�tape de projection (on continue � r�soudre
seulement
$\displaystyle \dive(\,{\rho\,\vect{u}}) = \Gamma$) alors que le terme
$\displaystyle \frac{\partial \rho}{\partial t}$ appara\^{\i}t lors de l'�tape de pr\'ediction de la vitesse
dans le sous-programme \fort{preduv}. Si ce terme joue un r�le sensible, l'algorithme compressible
de \CS\ (qui r�sout l'�quation compl�te) est alors sans doute plus adapt�.

%                      Code_Saturne version 1.3
%                      ------------------------
%
%     This file is part of the Code_Saturne Kernel, element of the
%     Code_Saturne CFD tool.
% 
%     Copyright (C) 1998-2007 EDF S.A., France
%
%     contact: saturne-support@edf.fr
% 
%     The Code_Saturne Kernel is free software; you can redistribute it
%     and/or modify it under the terms of the GNU General Public License
%     as published by the Free Software Foundation; either version 2 of
%     the License, or (at your option) any later version.
% 
%     The Code_Saturne Kernel is distributed in the hope that it will be
%     useful, but WITHOUT ANY WARRANTY; without even the implied warranty
%     of MERCHANTABILITY or FITNESS FOR A PARTICULAR PURPOSE.  See the
%     GNU General Public License for more details.
% 
%     You should have received a copy of the GNU General Public License
%     along with the Code_Saturne Kernel; if not, write to the
%     Free Software Foundation, Inc.,
%     51 Franklin St, Fifth Floor,
%     Boston, MA  02110-1301  USA
%
%-----------------------------------------------------------------------
%
%%%%%%%%%%%%%%%%%%%%%%%%%%%%%%%%%
%%%%%%%%%%%%%%%%%%%%%%%%%%%%%%%%%%
\section{Discr\'etisation}
%%%%%%%%%%%%%%%%%%%%%%%%%%%%%%%%%%
%%%%%%%%%%%%%%%%%%%%%%%%%%%%%%%%%%

Pour utiliser la m�thode, on se place tout d'abord dans un rep�re local d�fini
de mani�re � ce que le plan $(0yz)$, o� sont inject�s les vortex, soit confondu
avec le plan d'entr�e du calcul (voir figure \ref{Base_Vortex_entree}). 

\begin{figure}[h]
\centerline{\includegraphics[height=6cm]{../Base/Vortex/Images/entree.pdf}}
\caption{\label{Base_Vortex_entree} D�finiton des diff�rentes grandeurs dans le rep�re local
correspondant � l'entr�e d'une conduite de section carr�e.} 
\end{figure}

$u$, $v$ et $w$  sont les composantes de la vitesse fluctuante (principale et
transverse) dans ce plan, et
$\displaystyle \omega(y,z) = \frac{\partial w}{\partial y}-\frac{\partial v}{\partial z}$
la vorticit� dans la direction
normale au plan d'entr�e. $\overline{U}(y,z)$ repr�sente ici la vitesse
principale moyenne impos�e par l'utilisateur dans le plan d'entr�e. 

Chaque vortex $p$ va �tre caract�ris� par sa fonction de forme $\xi$ (identique
pour tous les vortex), sa
circulation $\Gamma_p$, son rayon $\sigma_p$ et les coordonn�es $(y_p,z_p)$ du
point $P$ o� est situ� le vortex dans le plan $(0yz)$. 

Pour cela, on suppose que la vorticit� g�n�r�e par un vortex $p$ au point $M$ de
coordonn�e $(y,z)$ s'�crit : 
\begin{equation}\notag
\omega_p(y,z)= \Gamma_p \, \xi_{\sigma_p}(r)
\end{equation}
o� $r = \sqrt{(y-y_p)^2+(z-z_p)^2}$ est la distance s�parant le point $M$ du point $P$.

Dans la m�thode implant�e, la fonction de forme est de type gaussienne modifi�e :
\begin{equation}\notag
\displaystyle
\xi_\sigma (r) = \frac{1}{2\pi \sigma^2} 
\left(2 e^{-\frac{r^2}{2\sigma^2}}-1\right) e^{-\frac{r^2}{2\sigma^2}}
\end{equation}

Le champ de vitesse induit par cette distribution de vorticit� s'obtient par
inversion des deux �quations de poisson suivantes qui sont d�duites de la
condition d'incompressibilit� dans la plan\footnote{\textit{i.e}
$\displaystyle \frac{\partial v}{\partial y}+\frac{\partial w}{\partial w} = 0$} :
\begin{equation}\notag
\begin{array}{lcr}
\displaystyle
\frac{\partial \omega}{\partial y} = \Delta w
&
\text{    et    }
&
\displaystyle
\frac{\partial \omega}{\partial y} = -\Delta v
\\
\end{array}
\end{equation}

Dans le cas g�n�ral, ce syst�me peut �tre int�gr� � l'aide de la formule de Biot et Savart.

Dans le cas d'une distribution de vorticit� de type gaussienne modifi�e, les
composantes de vitesse v�rifient : 
\begin{equation}\notag
\left\{
\begin{array}{c}
\displaystyle
v_p(y,x) = - \frac{1}{2\pi} \frac{(z-z_p)}{r^2}\left(1 -
e^{-\frac{r^2}{2\sigma^2}}\right)\,e^{-\frac{r^2}{2\sigma^2}} 
\\
\displaystyle
w_p(y,z) = \frac{1}{2\pi} \frac{(y-y_p)}{r^2}\left(1 -e^{-\frac{r^2}{2\sigma^2}}
\right)\,e^{-\frac{r^2}{2\sigma^2}} 
\end{array}
\right.
\end{equation}

Ces relations s'�tendent de fa�on imm�diate au cas de $N$ vortex distincts.
Le champ de vitesse induit par la distribution de vorticit� 
\begin{equation}
\omega(y,z) = \sum_{p=1}^N \Gamma_p \, \xi_{\sigma_p}(r)
\end{equation}
vaut au point $M$ :
\begin{equation}\notag
\begin{array}{lcr}
v(x,y) = \sum_{p=1}^N \Gamma_p\, v_p(y,z) 
&
\text{    et    }
&
w(y,z) = \sum_{p=1}^N \Gamma_p\, w_p(y,z)
\\
\label{Base_Vortex_compvit}
\end{array}
\end{equation}
%================================
\subsection{Param�tres physiques}
%================================

%-------------------------------
\subsubsection{Marche en temps}
%-------------------------------
La position initiale de chaque vortex est tir�e de mani�re al�atoire. On calcul
les d�placements successifs de chacun des vortex dans le plan d'entr�e par
int�gration explicite du champ de vitesse lagrangien : 
\begin{equation}\notag
\begin{array}{lcr}
\displaystyle
\frac{dy_p}{dt} = V(y,z)
&
\text{    et    }
&
\displaystyle
\frac{dz_p}{dt} = W(y,z)
\\
\end{array}
\end{equation}
Les vortex sont alors assimil�s � des particules ponctuelles qui sont convect�es
par le champ $(V,W)$. Ce champ peut �tre impos� par des tirages al�atoires ou
bien d�duit de la vitesse induite par les vortex dans le plan d'entr�e. Dans ce
cas $V(x,y) = \overline{V}(y,z) + v (y,z)$ et $W(y,z)= \overline{W}(y,z) +
w(y,z)$ o� $\overline{V}$ et $\overline{W}$ sont les composantes de la vitesse
transverse moyenne qu'impose l'utilisateur � l'aide des fichiers de donn�es. 

%---------------------------------------------------
\subsubsection{Intensit� et dur�e de vie des vortex}
%---------------------------------------------------
Il serait possible, � partir de l'�quation de transport de la vorticit�,
d'obtenir un mod�le d'�volution pour l'intensit� du vecteur tourbillon
$\omega_p$ associ� � chacun des vortex. En pratique, on pr�f�re utiliser un
mod�le simplifi� dans lequel la circulation des tourbillons ne d�pend que de la
postion de ces derniers � l'instant consid�r�. La circulation initiale de chaque
vortex est alors obtenue � partir de la relation : 
\begin{equation}\notag
|\Gamma_p| = 4 \sqrt{\frac{\pi\,S\,k}{3N\,[2ln(3)-3ln(2)]}}
\end{equation}
o� $S$ est la surface du plan d'entr�e, $N$ le nombre de vortex, et $k$
l'�nergie cin�tique turbulente au point o� se trouve le vortex � l'instant
consid�r�. Le signe de $\Gamma_p$ correspond au sens de rotation du vortex et
est tir� al�atoirement. 

Ce param�tre est celui qui contr�le l'intensit� des fluctuations. Sa d�pendance
en $k$ exprime que, plus l'�coulement est turbulent, plus les vortex sont
intenses. La valeur de $k$ est sp�cifi�e par
l'utilisateur. Elle peut �tre constante ou impos�e � partir de profils d'�nergie
cin�tique turbulente en entr�e. 

Pour �viter que des structures trop allong�es ne se d�veloppent au niveau de
l'entr�e, l'utilisateur doit �galement sp�cifier un temps limites $\tau_p$ au
bout duquel le vortex $p$ va �tre d�truit. Ce temps $\tau_p$ peut �tre pris
constant ou estim� au moyen de la relation : 
\begin{equation}\notag
\tau_p = \frac{5 C_{\mu}k^{\frac{3}{2}}}{\varepsilon\,\overline{U}}
\end{equation}

$\overline{U}$ et $\varepsilon$ repr�sentent respectivement la vitesse moyenne
principale et la dissipation turbulente au point o� est initialement g�n�r� le
vortex ($C_{\mu}=0,09$). 
\\
Lorsque le temps �coul� depuis la cr�ation du vortex $p$ est sup�rieur �
$\tau_p$, le vortex est d�truit et un nouveau vortex g�n�r� (sa position et le
signe de $\Gamma_p$ sont tir�s de fa�on al�atoire). 

%-------------------------------- 
\subsubsection{Taille des vortex}
%--------------------------------
La taille des vortex peut �tre prise constante, ou calcul�e � partir des
relations :
\begin{equation}\notag
\begin{array}{ccc}
\displaystyle
\sigma = \frac{C_{\mu}^{\frac{3}{4}}k^{\frac{3}{2}}}{\varepsilon} 
& \text{    ou    } &
\sigma = max[L_t,L_k]
\\
\end{array}
\end{equation}
avec:
\begin{equation}\notag
\begin{array}{ccc}
\displaystyle
L_t = \sqrt{\left( \frac{5 \nu k}{\varepsilon} \right)} 
& \text{    et    } & 
\displaystyle
L_k = 200\, \left(\frac{\nu^3}{\varepsilon}\right)^{\frac{1}{4}}
\end{array}
\end{equation}
o� $\nu$, $k$ et $\varepsilon$ sont la viscosit� dynamique, l'�nergie cin�tique
turbulente et la dissipation turbulente au point o� se trouve le vortex. 

Dans tous les cas, la taille du vortex doit �tre sup�rieure � la taille des
mailles en entr�e afin que le vortex soit effectivement simul�. 

%==================================
\subsection{Conditions aux limites}
%==================================
Le champ de vitesse g�n�r� � l'aide de la relation \ref{Base_Vortex_compvit} ne tient pas
compte {\em a priori} des conditions aux limites appliqu�es sur les bords du plan
d'entr�e. Pour obtenir des valeurs de la vitesse qui soient coh�rentes sur les
fronti�res du domaine d'entr�e, des ``vortex images'', pseudo-vortex situ�s en
dehors du domaine d'entr�e, sont g�n�r�s � des positions particuli�res et leur
champ de vitesse associ� est superpos� au champ pr�c�demment calcul�.\\
Seuls les cas d'une conduite rectangulaire et d'une conduite circulaire
permettent la g�n�ration de ces pseudo-vortex.
On distingue pour cela trois types de conditions aux limites. 

\begin{figure}[h]
\centerline{\includegraphics[height=6cm]{../Base/Vortex/Images/condlimite.pdf}}
\caption{\label{Base_Vortex_condli} Principe de g�n�ration des ``vortex images'' suivant le
type de conditions aux limites dans une conduite carr�e.} 
\end{figure}

%----------------------------------
\subsubsection{Condition de paroi}
%----------------------------------
On cr�e, pour chaque vortex $P$ contenu dans le plan d'entr�e, un vortex image
$P'$ identique � $P$ (\textit{i.e} de m�me caract�ristiques) et sym�trique de $P$
par rapport au
point $J$ ($J$ �tant la projection orthogonalement � la paroi du point $M$
correspondant au centre de la face o� l'on cherche � calculer la vitesse). La
figure \ref{Base_Vortex_condli} illustre la technique dans le cas d'une conduite
carr�e. Dans ce cas les coordonn�es du vortex situ� en $P'$ v�rifient
$(y_{p'}+y_{p})/2 = y_{J}$ et $(z_{p'}+ z_{p})/2 = z_{J}$. Le champ de vitesse
per�u depuis le point $M$ au niveau du point $J$ est nul, ce qui est bien
l'effet recherch�. 

%------------------------------------
\subsubsection{Condition de sym�trie}
%-------------------------------------
La technique est identique � celle utilis�e pour les conditions de paroi, mais
seule la composante pour la vitesse normale au bord est modifi�e dans ce cas. 

%---------------------------------------
\subsubsection{Condition de p�riodicit�}
%---------------------------------------
On cr�e pour chaque vortex, un vortex images $P'$ identique � $P$ mais translat�
d'une quantit� $L$ correspondant � la longueur qui s�pare les deux plans de la
section d'entr�e o� sont appliqu�es les conditions de p�riodicit�. Dans le cas
o� il y a deux directions de p�riodicit�, on cr�e deux vortex image.

%=============================================
\subsection{Composante de vitesse principale}
%=============================================
La m�thode des vortex ne g�n�rant pas de fluctuation $u$ de la vitesse dans la
direction principale, la derni�re composante est calcul�e � partir d'une
�quation de Langevin. Les coefficients de cette �quation sont d�termin�s par
identification des expressions obtenues pour les contraintes de Reynolds en
$R_{ij}-\varepsilon$. Dans le cas d'un �coulement en canal plan, cette technique
conduit � l'�quation : 
\begin{equation}\notag
\displaystyle
\frac{du}{dt} = - \frac{C_1}{2T} u + \left(\frac{2}{3}C_2-1\right)\frac{\partial
U}{\partial y} v + \sqrt{C_0\varepsilon} dW_i 
\end{equation}

avec $\displaystyle T=\frac{k}{\varepsilon}$, $C_1 = 1,8$, $C_2 = 0,6$,
$C_0=\frac{14}{15}$, et $dW_i$ une variable al�toire Gaussienne de variance
$\sqrt{dt}$. 

En pratique, l'�quation de Langevin n'am�liore pas vraiment les r�sultats. Elle
n'est utilis�e que dans le cas de conduites circulaires. 

%                      Code_Saturne version 1.3
%                      ------------------------
%
%     This file is part of the Code_Saturne Kernel, element of the
%     Code_Saturne CFD tool.
%
%     Copyright (C) 1998-2007 EDF S.A., France
%
%     contact: saturne-support@edf.fr
%
%     The Code_Saturne Kernel is free software; you can redistribute it
%     and/or modify it under the terms of the GNU General Public License
%     as published by the Free Software Foundation; either version 2 of
%     the License, or (at your option) any later version.
%
%     The Code_Saturne Kernel is distributed in the hope that it will be
%     useful, but WITHOUT ANY WARRANTY; without even the implied warranty
%     of MERCHANTABILITY or FITNESS FOR A PARTICULAR PURPOSE.  See the
%     GNU General Public License for more details.
%
%     You should have received a copy of the GNU General Public License
%     along with the Code_Saturne Kernel; if not, write to the
%     Free Software Foundation, Inc.,
%     51 Franklin St, Fifth Floor,
%     Boston, MA  02110-1301  USA
%
%-----------------------------------------------------------------------
%

%%%%%%%%%%%%%%%%%%%%%%%%%%%%%%%%%%
%%%%%%%%%%%%%%%%%%%%%%%%%%%%%%%%%%
\section{Mise en \oe uvre}
%%%%%%%%%%%%%%%%%%%%%%%%%%%%%%%%%%
%%%%%%%%%%%%%%%%%%%%%%%%%%%%%%%%%%
Le syst\`eme (\ref{Cfbl_Cfmsvl_eq_densite_finale_cfmsvl}) est r\'esolu par une m\'ethode
d'incr\'ement et r\'esidu en utilisant
une m\'ethode de Jacobi pour inverser le syst\`eme si le terme convectif
est implicite et en utilisant une m\'ethode de gradient conjugu\'e
si le terme convectif est explicite (qui est le cas par d�faut).

Attention, les valeurs du flux de masse $\rho\,\vect{w}\cdot\vect{S}$ et
de la viscosit\'e $\Delta\,t\,c^2\frac{S}{d}$ aux faces de
bord, qui sont calcul\'ees dans \fort{cfmsfl} et \fort{cfmsvs} respectivement,
sont modifi\'ees imm\'ediatement apr\`es l'appel \`a ces sous-programmes.
En effet, il est indispensable que la contribution de bord de
$\left(\rho\,\vect{w}-\Delta\,t\,(c^2)\,\gradv\,\rho\right)\cdot\vect{S}$
repr\'esente exactement $\vect{Q}_{ac}\cdot\vect{S}$.
Pour cela,
\begin{itemize}
\item imm\'ediatement apr\`es l'appel \`a
\fort{cfmsfl}, on remplace la contribution de bord de
$\rho\,\vect{w}\cdot\vect{S}$
par le flux de masse exact, $\vect{Q}_{ac}\cdot\vect{S}$,
d\'etermin\'e \`a partir des conditions aux limites,
\item puis, imm\'ediatement apr\`es l'appel \`a
\fort{cfmsvs}, on annule la viscosit\'e au bord $\Delta\,t\,(c^2)$ pour
\'eliminer la contribution de $-\Delta\,t\,(c^2)\,(\gradv\,\rho)\cdot\vect{S}$
(l'annulation de la viscosit\'e n'est pas probl\'ematique pour la matrice,
puisqu'elle porte sur des incr\'ements).
\end{itemize}

\bigskip

Une fois qu'on a obtenu $\rho^{n+1}$,
on peut actualiser le flux de masse acoustique
aux faces $(\vect{Q}_{ac}^{n+1})_{ij} \cdot \vect{S}_{ij}$,
qui servira pour la convection des autres variables~:
\begin{equation}\label{Cfbl_Cfmsvl_eq_flux_masse_acoustique_cfmsvl}
\displaystyle(\vect{Q}_{ac}^{n+1})_{ij}\cdot\vect{S}_{ij}=
-\left(\Delta t^n (c^2)^n \gradv(\rho^{n+1})\right)_{ij}\cdot\vect{S}_{ij}
+\left(\rho^{n+\frac{1}{2}} \vect{w}^n\right)_{ij}\cdot\vect{S}_{ij}\\
\end{equation}
Ce calcul de flux est r\'ealis\'e par \fort{cfbsc3}.
Si l'on a choisi l'algorithme standard, \'equation~(\ref{Cfbl_Cfmsvl_eq_densite_cfmsvl}),
on compl\`ete le flux dans \fort{cfmsvl} imm\'ediatement apr\`es l'appel
\`a \fort{cfbsc3}.
En effet, dans ce cas,
la convection est explicite ($\rho^{n+\frac{1}{2}}=\rho^{n}$,
obtenu en imposant \var{ICONV(ISCA(IRHO(IPHAS)))=0})
et le sous-programme \fort{cfbsc3},
qui calcule le flux de masse aux faces,
ne prend pas en compte la contribution du terme
$\rho^{n+\frac{1}{2}}\,\vect{w}^n\cdot\vect{S}$. On ajoute donc cette
contribution dans \fort{cfmsvl}, apr\`es l'appel \`a \fort{cfbsc3}.
Au bord, en particulier, c'est bien le flux de masse calcul\'e \`a partir
des conditions aux limites que l'on obtient.

On actualise la pression \`a la fin de l'\'etape, en utilisant la loi d'\'etat~:
\begin{equation}
\displaystyle P_i^{pred}=P(\rho_i^{n+1},\varepsilon_i^{n})
\end{equation}


%%%%%%%%%%%%%%%%%%%%%%%%%%%%%%%%%%
%%%%%%%%%%%%%%%%%%%%%%%%%%%%%%%%%%
\section{Points \`a traiter}
%%%%%%%%%%%%%%%%%%%%%%%%%%%%%%%%%%
%%%%%%%%%%%%%%%%%%%%%%%%%%%%%%%%%%
Le calcul du flux de masse au  bord n'est pas enti\`erement satisfaisant
si la convection est trait\'ee de mani\`ere implicite
(algorithme non standard, non test\'e,
associ\'e \`a l'\'equation~(\ref{Cfbl_Cfmsvl_eq_densite_bis_cfmsvl}),
correspondant au choix $\rho^{n+\frac{1}{2}}=\rho^{n+1}$ et
obtenu en imposant \var{ICONV(ISCA(IRHO(IPHAS)))=1}).
En effet, apr\`es \fort{cfmsfl}, il faut d\'eterminer la vitesse de
convection $\vect{w}^n$ pour qu'apparaisse
$\rho^{n+1} \vect{w}^n\cdot\vect{n}$
au cours de la r\'esolution par \fort{codits}. De ce fait, on doit d\'eduire
une valeur de $\vect{w}^n$ \`a partir de la valeur
du flux de masse. Au bord, en particulier, il faut
donc diviser le flux de masse
issu des conditions aux limites par la valeur de bord de $\rho^{n+1}$.
Or, lorsque des conditions de Neumann sont appliqu\'ees \`a la
masse volumique,
la valeur de $\rho^{n+1}$ au bord n'est pas connue avant la r\'esolution du
syst\`eme. On utilise donc, au lieu de la valeur de bord inconnue de
$\rho^{n+1}$ la valeur de bord prise au pas de temps
pr\'ec\'edent $\rho^{n}$. Cette approximation est susceptible
d'affecter la valeur du flux de masse au bord.

%                      Code_Saturne version 1.3
%                      ------------------------
%
%     This file is part of the Code_Saturne Kernel, element of the
%     Code_Saturne CFD tool.
%
%     Copyright (C) 1998-2007 EDF S.A., France
%
%     contact: saturne-support@edf.fr
%
%     The Code_Saturne Kernel is free software; you can redistribute it
%     and/or modify it under the terms of the GNU General Public License
%     as published by the Free Software Foundation; either version 2 of
%     the License, or (at your option) any later version.
%
%     The Code_Saturne Kernel is distributed in the hope that it will be
%     useful, but WITHOUT ANY WARRANTY; without even the implied warranty
%     of MERCHANTABILITY or FITNESS FOR A PARTICULAR PURPOSE.  See the
%     GNU General Public License for more details.
%
%     You should have received a copy of the GNU General Public License
%     along with the Code_Saturne Kernel; if not, write to the
%     Free Software Foundation, Inc.,
%     51 Franklin St, Fifth Floor,
%     Boston, MA  02110-1301  USA
%
%-----------------------------------------------------------------------
%


\programme{navsto}

\vspace{1cm}
On s'int\'eresse \`a la r\'esolution du syst\`eme d'\'equations de Navier-Stokes
tridimensionnelles monophasiques, \`a une pression, instationnaires, en
incompressible ou faiblement dilatable, bas\'ees sur une discr\'etisation
temporelle de type Euler implicite d'ordre 1 ou Crank-Nicolson d'ordre 2 et sur
une discr\'etisation spatiale  par volumes finis colocalis\'ee.


%%%%%%%%%%%%%%%%%%%%%%%%%%%%%%%%%%
%%%%%%%%%%%%%%%%%%%%%%%%%%%%%%%%%%
\section{Fonction}
%%%%%%%%%%%%%%%%%%%%%%%%%%%%%%%%%%
%%%%%%%%%%%%%%%%%%%%%%%%%%%%%%%%%%

  Dans ce sous-programme sont calcul\'ees, \`a un pas de temps donn\'e, les
variables vitesse et pression de ce probl\`eme en proc\'edant en
deux  \'etapes issues d'une d\'ecomposition des op\'erateurs (m\'ethode \`a
pas fractionnaires).\\
Les variables sont donc suppos\'ees connues \`a
l'instant ${t^n}$ et on cherche \`a les d\'eterminer \`a l'instant\footnote{La pression est suppos�e connue � l'instant $t^{n-1+\theta}$ et recherch�e en $t^{n+\theta}$, avec $\theta=1$ ou $1/2$ suivant le sch�ma en temps consid�r�.} ${t^{n+1}}$. Soit ${\Delta {t^n} ={t^{n+1}- {t^n}}}$ le pas de temps associ\'e. Dans un premier temps, on r\'ealise l'\'etape de
pr\'ediction de la vitesse en r\'esolvant l'\'equation de quantit\'e de
mouvement avec une pression explicite. Suit l'\'etape de correction de la
pression (ou projection de la vitesse) qui permet d'obtenir un champ de vitesse \`a divergence nulle.\\\\
Les \'equations en continu sont donc :
\begin{equation}
\left\{\begin{array}{l}
\displaystyle\frac{\partial}{\partial t}(\rho \vect{u}) + \dive(\rho\, \vect{u} \otimes \vect{u})
=\dive(\tens{\sigma}) + \vect{TS} - \tens{K}\,\vect{u}\\
\dive(\rho \vect{u}) = \Gamma
\end{array}\right.
\end{equation}

%(plus tard $\frac{\partial \rho}{\partial t} + \dive(\rho \vect{u}) = \Gamma$)



avec $\rho$ la masse volumique, $\vect{u}$ le champ de vitesse,
$[\,\vect{TS}-\tens{K}\,\vect{u}\,]$ les autres termes sources ($\tens{K}$~est un
tenseur diagonal positif par d\'efinition), $\tens{\sigma}$ le tenseur
de contraintes, $\tens{\tau}$ le tenseur des contraintes visqueuses, $\mu$ la
viscosit\'e dynamique (mol\'eculaire et \'eventuellement turbulente), $\kappa$
la viscosit� de
volume (usuellement nulle et n�glig�e dans le code et dans la suite du document,
sauf en compressible),
$\tens{D}$ le tenseur taux de d\'eformation\footnote{\`A ne pas confondre, malgr\'e la m\^eme notation $D$,
avec les flux diffusifs $\vect{D}_{\,ij}$ et $\vect{D}_{\,{b}_{ik}}$ d\'ecrits par la suite dans ce
sous-programme.}, $\Gamma$ le terme source de masse.
\begin{equation}
\left\{\begin{array}{l}
\tens{\sigma} = \tens{\tau} - P\tens{Id}  \\
\tens{\tau} = 2\,\mu\ \tens{D} +\ (\kappa\ - \frac{2}{3}\mu)\  tr({\tens{D}})\
\tens{Id}  \\
\tens{D} = \frac{1}{2}(\ggrad\vect{u}+\,^{t}\ggrad\vect{u})
\end{array}\right.
\end{equation}
 \\

On rappelle la d\'efinition des notations employ\'ees\footnote{en
utilisant la convention de sommation d'Einstein.}~:
\begin{equation}\notag
\left\{\begin{array}{lll}
\left[\ggrad{\vect{a}}\right]_{ij} &=& \partial_j a_i\\
\left[\dive(\tens{\sigma})\right]_i &=& \partial_j \sigma_{ij}\\
\left[\vect{a}\otimes\vect{b}\right]_{ij} &= &
a_i\,b_j\\
\end{array}\right.
\end{equation}
et donc :
\begin{equation}\notag
\begin{array}{lll}
\left[\dive(\vect{a}\otimes\vect{b})\right]_i &= &
\partial_j (a_i\,b_j)
\end{array}
\end{equation}

\minititre{Remarque}
Dans le cas de la prise en compte d'une masse volumique variable, l'�quation de continuit� s'�crit :
$$\frac{\partial \rho}{\partial t} + \dive{\,(\rho\,\vect{u})} = \Gamma  $$
Cette �quation n'est pas prise en compte dans l'�tape de projection (on continue � r�soudre
seulement
$\displaystyle \dive(\,{\rho\,\vect{u}}) = \Gamma$) alors que le terme
$\displaystyle \frac{\partial \rho}{\partial t}$ appara\^{\i}t lors de l'�tape de pr\'ediction de la vitesse
dans le sous-programme \fort{preduv}. Si ce terme joue un r�le sensible, l'algorithme compressible
de \CS\ (qui r�sout l'�quation compl�te) est alors sans doute plus adapt�.

%                      Code_Saturne version 1.3
%                      ------------------------
%
%     This file is part of the Code_Saturne Kernel, element of the
%     Code_Saturne CFD tool.
% 
%     Copyright (C) 1998-2007 EDF S.A., France
%
%     contact: saturne-support@edf.fr
% 
%     The Code_Saturne Kernel is free software; you can redistribute it
%     and/or modify it under the terms of the GNU General Public License
%     as published by the Free Software Foundation; either version 2 of
%     the License, or (at your option) any later version.
% 
%     The Code_Saturne Kernel is distributed in the hope that it will be
%     useful, but WITHOUT ANY WARRANTY; without even the implied warranty
%     of MERCHANTABILITY or FITNESS FOR A PARTICULAR PURPOSE.  See the
%     GNU General Public License for more details.
% 
%     You should have received a copy of the GNU General Public License
%     along with the Code_Saturne Kernel; if not, write to the
%     Free Software Foundation, Inc.,
%     51 Franklin St, Fifth Floor,
%     Boston, MA  02110-1301  USA
%
%-----------------------------------------------------------------------
%
%%%%%%%%%%%%%%%%%%%%%%%%%%%%%%%%%
%%%%%%%%%%%%%%%%%%%%%%%%%%%%%%%%%%
\section{Discr\'etisation}
%%%%%%%%%%%%%%%%%%%%%%%%%%%%%%%%%%
%%%%%%%%%%%%%%%%%%%%%%%%%%%%%%%%%%

Pour utiliser la m�thode, on se place tout d'abord dans un rep�re local d�fini
de mani�re � ce que le plan $(0yz)$, o� sont inject�s les vortex, soit confondu
avec le plan d'entr�e du calcul (voir figure \ref{Base_Vortex_entree}). 

\begin{figure}[h]
\centerline{\includegraphics[height=6cm]{../Base/Vortex/Images/entree.pdf}}
\caption{\label{Base_Vortex_entree} D�finiton des diff�rentes grandeurs dans le rep�re local
correspondant � l'entr�e d'une conduite de section carr�e.} 
\end{figure}

$u$, $v$ et $w$  sont les composantes de la vitesse fluctuante (principale et
transverse) dans ce plan, et
$\displaystyle \omega(y,z) = \frac{\partial w}{\partial y}-\frac{\partial v}{\partial z}$
la vorticit� dans la direction
normale au plan d'entr�e. $\overline{U}(y,z)$ repr�sente ici la vitesse
principale moyenne impos�e par l'utilisateur dans le plan d'entr�e. 

Chaque vortex $p$ va �tre caract�ris� par sa fonction de forme $\xi$ (identique
pour tous les vortex), sa
circulation $\Gamma_p$, son rayon $\sigma_p$ et les coordonn�es $(y_p,z_p)$ du
point $P$ o� est situ� le vortex dans le plan $(0yz)$. 

Pour cela, on suppose que la vorticit� g�n�r�e par un vortex $p$ au point $M$ de
coordonn�e $(y,z)$ s'�crit : 
\begin{equation}\notag
\omega_p(y,z)= \Gamma_p \, \xi_{\sigma_p}(r)
\end{equation}
o� $r = \sqrt{(y-y_p)^2+(z-z_p)^2}$ est la distance s�parant le point $M$ du point $P$.

Dans la m�thode implant�e, la fonction de forme est de type gaussienne modifi�e :
\begin{equation}\notag
\displaystyle
\xi_\sigma (r) = \frac{1}{2\pi \sigma^2} 
\left(2 e^{-\frac{r^2}{2\sigma^2}}-1\right) e^{-\frac{r^2}{2\sigma^2}}
\end{equation}

Le champ de vitesse induit par cette distribution de vorticit� s'obtient par
inversion des deux �quations de poisson suivantes qui sont d�duites de la
condition d'incompressibilit� dans la plan\footnote{\textit{i.e}
$\displaystyle \frac{\partial v}{\partial y}+\frac{\partial w}{\partial w} = 0$} :
\begin{equation}\notag
\begin{array}{lcr}
\displaystyle
\frac{\partial \omega}{\partial y} = \Delta w
&
\text{    et    }
&
\displaystyle
\frac{\partial \omega}{\partial y} = -\Delta v
\\
\end{array}
\end{equation}

Dans le cas g�n�ral, ce syst�me peut �tre int�gr� � l'aide de la formule de Biot et Savart.

Dans le cas d'une distribution de vorticit� de type gaussienne modifi�e, les
composantes de vitesse v�rifient : 
\begin{equation}\notag
\left\{
\begin{array}{c}
\displaystyle
v_p(y,x) = - \frac{1}{2\pi} \frac{(z-z_p)}{r^2}\left(1 -
e^{-\frac{r^2}{2\sigma^2}}\right)\,e^{-\frac{r^2}{2\sigma^2}} 
\\
\displaystyle
w_p(y,z) = \frac{1}{2\pi} \frac{(y-y_p)}{r^2}\left(1 -e^{-\frac{r^2}{2\sigma^2}}
\right)\,e^{-\frac{r^2}{2\sigma^2}} 
\end{array}
\right.
\end{equation}

Ces relations s'�tendent de fa�on imm�diate au cas de $N$ vortex distincts.
Le champ de vitesse induit par la distribution de vorticit� 
\begin{equation}
\omega(y,z) = \sum_{p=1}^N \Gamma_p \, \xi_{\sigma_p}(r)
\end{equation}
vaut au point $M$ :
\begin{equation}\notag
\begin{array}{lcr}
v(x,y) = \sum_{p=1}^N \Gamma_p\, v_p(y,z) 
&
\text{    et    }
&
w(y,z) = \sum_{p=1}^N \Gamma_p\, w_p(y,z)
\\
\label{Base_Vortex_compvit}
\end{array}
\end{equation}
%================================
\subsection{Param�tres physiques}
%================================

%-------------------------------
\subsubsection{Marche en temps}
%-------------------------------
La position initiale de chaque vortex est tir�e de mani�re al�atoire. On calcul
les d�placements successifs de chacun des vortex dans le plan d'entr�e par
int�gration explicite du champ de vitesse lagrangien : 
\begin{equation}\notag
\begin{array}{lcr}
\displaystyle
\frac{dy_p}{dt} = V(y,z)
&
\text{    et    }
&
\displaystyle
\frac{dz_p}{dt} = W(y,z)
\\
\end{array}
\end{equation}
Les vortex sont alors assimil�s � des particules ponctuelles qui sont convect�es
par le champ $(V,W)$. Ce champ peut �tre impos� par des tirages al�atoires ou
bien d�duit de la vitesse induite par les vortex dans le plan d'entr�e. Dans ce
cas $V(x,y) = \overline{V}(y,z) + v (y,z)$ et $W(y,z)= \overline{W}(y,z) +
w(y,z)$ o� $\overline{V}$ et $\overline{W}$ sont les composantes de la vitesse
transverse moyenne qu'impose l'utilisateur � l'aide des fichiers de donn�es. 

%---------------------------------------------------
\subsubsection{Intensit� et dur�e de vie des vortex}
%---------------------------------------------------
Il serait possible, � partir de l'�quation de transport de la vorticit�,
d'obtenir un mod�le d'�volution pour l'intensit� du vecteur tourbillon
$\omega_p$ associ� � chacun des vortex. En pratique, on pr�f�re utiliser un
mod�le simplifi� dans lequel la circulation des tourbillons ne d�pend que de la
postion de ces derniers � l'instant consid�r�. La circulation initiale de chaque
vortex est alors obtenue � partir de la relation : 
\begin{equation}\notag
|\Gamma_p| = 4 \sqrt{\frac{\pi\,S\,k}{3N\,[2ln(3)-3ln(2)]}}
\end{equation}
o� $S$ est la surface du plan d'entr�e, $N$ le nombre de vortex, et $k$
l'�nergie cin�tique turbulente au point o� se trouve le vortex � l'instant
consid�r�. Le signe de $\Gamma_p$ correspond au sens de rotation du vortex et
est tir� al�atoirement. 

Ce param�tre est celui qui contr�le l'intensit� des fluctuations. Sa d�pendance
en $k$ exprime que, plus l'�coulement est turbulent, plus les vortex sont
intenses. La valeur de $k$ est sp�cifi�e par
l'utilisateur. Elle peut �tre constante ou impos�e � partir de profils d'�nergie
cin�tique turbulente en entr�e. 

Pour �viter que des structures trop allong�es ne se d�veloppent au niveau de
l'entr�e, l'utilisateur doit �galement sp�cifier un temps limites $\tau_p$ au
bout duquel le vortex $p$ va �tre d�truit. Ce temps $\tau_p$ peut �tre pris
constant ou estim� au moyen de la relation : 
\begin{equation}\notag
\tau_p = \frac{5 C_{\mu}k^{\frac{3}{2}}}{\varepsilon\,\overline{U}}
\end{equation}

$\overline{U}$ et $\varepsilon$ repr�sentent respectivement la vitesse moyenne
principale et la dissipation turbulente au point o� est initialement g�n�r� le
vortex ($C_{\mu}=0,09$). 
\\
Lorsque le temps �coul� depuis la cr�ation du vortex $p$ est sup�rieur �
$\tau_p$, le vortex est d�truit et un nouveau vortex g�n�r� (sa position et le
signe de $\Gamma_p$ sont tir�s de fa�on al�atoire). 

%-------------------------------- 
\subsubsection{Taille des vortex}
%--------------------------------
La taille des vortex peut �tre prise constante, ou calcul�e � partir des
relations :
\begin{equation}\notag
\begin{array}{ccc}
\displaystyle
\sigma = \frac{C_{\mu}^{\frac{3}{4}}k^{\frac{3}{2}}}{\varepsilon} 
& \text{    ou    } &
\sigma = max[L_t,L_k]
\\
\end{array}
\end{equation}
avec:
\begin{equation}\notag
\begin{array}{ccc}
\displaystyle
L_t = \sqrt{\left( \frac{5 \nu k}{\varepsilon} \right)} 
& \text{    et    } & 
\displaystyle
L_k = 200\, \left(\frac{\nu^3}{\varepsilon}\right)^{\frac{1}{4}}
\end{array}
\end{equation}
o� $\nu$, $k$ et $\varepsilon$ sont la viscosit� dynamique, l'�nergie cin�tique
turbulente et la dissipation turbulente au point o� se trouve le vortex. 

Dans tous les cas, la taille du vortex doit �tre sup�rieure � la taille des
mailles en entr�e afin que le vortex soit effectivement simul�. 

%==================================
\subsection{Conditions aux limites}
%==================================
Le champ de vitesse g�n�r� � l'aide de la relation \ref{Base_Vortex_compvit} ne tient pas
compte {\em a priori} des conditions aux limites appliqu�es sur les bords du plan
d'entr�e. Pour obtenir des valeurs de la vitesse qui soient coh�rentes sur les
fronti�res du domaine d'entr�e, des ``vortex images'', pseudo-vortex situ�s en
dehors du domaine d'entr�e, sont g�n�r�s � des positions particuli�res et leur
champ de vitesse associ� est superpos� au champ pr�c�demment calcul�.\\
Seuls les cas d'une conduite rectangulaire et d'une conduite circulaire
permettent la g�n�ration de ces pseudo-vortex.
On distingue pour cela trois types de conditions aux limites. 

\begin{figure}[h]
\centerline{\includegraphics[height=6cm]{../Base/Vortex/Images/condlimite.pdf}}
\caption{\label{Base_Vortex_condli} Principe de g�n�ration des ``vortex images'' suivant le
type de conditions aux limites dans une conduite carr�e.} 
\end{figure}

%----------------------------------
\subsubsection{Condition de paroi}
%----------------------------------
On cr�e, pour chaque vortex $P$ contenu dans le plan d'entr�e, un vortex image
$P'$ identique � $P$ (\textit{i.e} de m�me caract�ristiques) et sym�trique de $P$
par rapport au
point $J$ ($J$ �tant la projection orthogonalement � la paroi du point $M$
correspondant au centre de la face o� l'on cherche � calculer la vitesse). La
figure \ref{Base_Vortex_condli} illustre la technique dans le cas d'une conduite
carr�e. Dans ce cas les coordonn�es du vortex situ� en $P'$ v�rifient
$(y_{p'}+y_{p})/2 = y_{J}$ et $(z_{p'}+ z_{p})/2 = z_{J}$. Le champ de vitesse
per�u depuis le point $M$ au niveau du point $J$ est nul, ce qui est bien
l'effet recherch�. 

%------------------------------------
\subsubsection{Condition de sym�trie}
%-------------------------------------
La technique est identique � celle utilis�e pour les conditions de paroi, mais
seule la composante pour la vitesse normale au bord est modifi�e dans ce cas. 

%---------------------------------------
\subsubsection{Condition de p�riodicit�}
%---------------------------------------
On cr�e pour chaque vortex, un vortex images $P'$ identique � $P$ mais translat�
d'une quantit� $L$ correspondant � la longueur qui s�pare les deux plans de la
section d'entr�e o� sont appliqu�es les conditions de p�riodicit�. Dans le cas
o� il y a deux directions de p�riodicit�, on cr�e deux vortex image.

%=============================================
\subsection{Composante de vitesse principale}
%=============================================
La m�thode des vortex ne g�n�rant pas de fluctuation $u$ de la vitesse dans la
direction principale, la derni�re composante est calcul�e � partir d'une
�quation de Langevin. Les coefficients de cette �quation sont d�termin�s par
identification des expressions obtenues pour les contraintes de Reynolds en
$R_{ij}-\varepsilon$. Dans le cas d'un �coulement en canal plan, cette technique
conduit � l'�quation : 
\begin{equation}\notag
\displaystyle
\frac{du}{dt} = - \frac{C_1}{2T} u + \left(\frac{2}{3}C_2-1\right)\frac{\partial
U}{\partial y} v + \sqrt{C_0\varepsilon} dW_i 
\end{equation}

avec $\displaystyle T=\frac{k}{\varepsilon}$, $C_1 = 1,8$, $C_2 = 0,6$,
$C_0=\frac{14}{15}$, et $dW_i$ une variable al�toire Gaussienne de variance
$\sqrt{dt}$. 

En pratique, l'�quation de Langevin n'am�liore pas vraiment les r�sultats. Elle
n'est utilis�e que dans le cas de conduites circulaires. 

%                      Code_Saturne version 1.3
%                      ------------------------
%
%     This file is part of the Code_Saturne Kernel, element of the
%     Code_Saturne CFD tool.
%
%     Copyright (C) 1998-2007 EDF S.A., France
%
%     contact: saturne-support@edf.fr
%
%     The Code_Saturne Kernel is free software; you can redistribute it
%     and/or modify it under the terms of the GNU General Public License
%     as published by the Free Software Foundation; either version 2 of
%     the License, or (at your option) any later version.
%
%     The Code_Saturne Kernel is distributed in the hope that it will be
%     useful, but WITHOUT ANY WARRANTY; without even the implied warranty
%     of MERCHANTABILITY or FITNESS FOR A PARTICULAR PURPOSE.  See the
%     GNU General Public License for more details.
%
%     You should have received a copy of the GNU General Public License
%     along with the Code_Saturne Kernel; if not, write to the
%     Free Software Foundation, Inc.,
%     51 Franklin St, Fifth Floor,
%     Boston, MA  02110-1301  USA
%
%-----------------------------------------------------------------------
%

%%%%%%%%%%%%%%%%%%%%%%%%%%%%%%%%%%
%%%%%%%%%%%%%%%%%%%%%%%%%%%%%%%%%%
\section{Mise en \oe uvre}
%%%%%%%%%%%%%%%%%%%%%%%%%%%%%%%%%%
%%%%%%%%%%%%%%%%%%%%%%%%%%%%%%%%%%
Le syst\`eme (\ref{Cfbl_Cfmsvl_eq_densite_finale_cfmsvl}) est r\'esolu par une m\'ethode
d'incr\'ement et r\'esidu en utilisant
une m\'ethode de Jacobi pour inverser le syst\`eme si le terme convectif
est implicite et en utilisant une m\'ethode de gradient conjugu\'e
si le terme convectif est explicite (qui est le cas par d�faut).

Attention, les valeurs du flux de masse $\rho\,\vect{w}\cdot\vect{S}$ et
de la viscosit\'e $\Delta\,t\,c^2\frac{S}{d}$ aux faces de
bord, qui sont calcul\'ees dans \fort{cfmsfl} et \fort{cfmsvs} respectivement,
sont modifi\'ees imm\'ediatement apr\`es l'appel \`a ces sous-programmes.
En effet, il est indispensable que la contribution de bord de
$\left(\rho\,\vect{w}-\Delta\,t\,(c^2)\,\gradv\,\rho\right)\cdot\vect{S}$
repr\'esente exactement $\vect{Q}_{ac}\cdot\vect{S}$.
Pour cela,
\begin{itemize}
\item imm\'ediatement apr\`es l'appel \`a
\fort{cfmsfl}, on remplace la contribution de bord de
$\rho\,\vect{w}\cdot\vect{S}$
par le flux de masse exact, $\vect{Q}_{ac}\cdot\vect{S}$,
d\'etermin\'e \`a partir des conditions aux limites,
\item puis, imm\'ediatement apr\`es l'appel \`a
\fort{cfmsvs}, on annule la viscosit\'e au bord $\Delta\,t\,(c^2)$ pour
\'eliminer la contribution de $-\Delta\,t\,(c^2)\,(\gradv\,\rho)\cdot\vect{S}$
(l'annulation de la viscosit\'e n'est pas probl\'ematique pour la matrice,
puisqu'elle porte sur des incr\'ements).
\end{itemize}

\bigskip

Une fois qu'on a obtenu $\rho^{n+1}$,
on peut actualiser le flux de masse acoustique
aux faces $(\vect{Q}_{ac}^{n+1})_{ij} \cdot \vect{S}_{ij}$,
qui servira pour la convection des autres variables~:
\begin{equation}\label{Cfbl_Cfmsvl_eq_flux_masse_acoustique_cfmsvl}
\displaystyle(\vect{Q}_{ac}^{n+1})_{ij}\cdot\vect{S}_{ij}=
-\left(\Delta t^n (c^2)^n \gradv(\rho^{n+1})\right)_{ij}\cdot\vect{S}_{ij}
+\left(\rho^{n+\frac{1}{2}} \vect{w}^n\right)_{ij}\cdot\vect{S}_{ij}\\
\end{equation}
Ce calcul de flux est r\'ealis\'e par \fort{cfbsc3}.
Si l'on a choisi l'algorithme standard, \'equation~(\ref{Cfbl_Cfmsvl_eq_densite_cfmsvl}),
on compl\`ete le flux dans \fort{cfmsvl} imm\'ediatement apr\`es l'appel
\`a \fort{cfbsc3}.
En effet, dans ce cas,
la convection est explicite ($\rho^{n+\frac{1}{2}}=\rho^{n}$,
obtenu en imposant \var{ICONV(ISCA(IRHO(IPHAS)))=0})
et le sous-programme \fort{cfbsc3},
qui calcule le flux de masse aux faces,
ne prend pas en compte la contribution du terme
$\rho^{n+\frac{1}{2}}\,\vect{w}^n\cdot\vect{S}$. On ajoute donc cette
contribution dans \fort{cfmsvl}, apr\`es l'appel \`a \fort{cfbsc3}.
Au bord, en particulier, c'est bien le flux de masse calcul\'e \`a partir
des conditions aux limites que l'on obtient.

On actualise la pression \`a la fin de l'\'etape, en utilisant la loi d'\'etat~:
\begin{equation}
\displaystyle P_i^{pred}=P(\rho_i^{n+1},\varepsilon_i^{n})
\end{equation}


%%%%%%%%%%%%%%%%%%%%%%%%%%%%%%%%%%
%%%%%%%%%%%%%%%%%%%%%%%%%%%%%%%%%%
\section{Points \`a traiter}
%%%%%%%%%%%%%%%%%%%%%%%%%%%%%%%%%%
%%%%%%%%%%%%%%%%%%%%%%%%%%%%%%%%%%
Le calcul du flux de masse au  bord n'est pas enti\`erement satisfaisant
si la convection est trait\'ee de mani\`ere implicite
(algorithme non standard, non test\'e,
associ\'e \`a l'\'equation~(\ref{Cfbl_Cfmsvl_eq_densite_bis_cfmsvl}),
correspondant au choix $\rho^{n+\frac{1}{2}}=\rho^{n+1}$ et
obtenu en imposant \var{ICONV(ISCA(IRHO(IPHAS)))=1}).
En effet, apr\`es \fort{cfmsfl}, il faut d\'eterminer la vitesse de
convection $\vect{w}^n$ pour qu'apparaisse
$\rho^{n+1} \vect{w}^n\cdot\vect{n}$
au cours de la r\'esolution par \fort{codits}. De ce fait, on doit d\'eduire
une valeur de $\vect{w}^n$ \`a partir de la valeur
du flux de masse. Au bord, en particulier, il faut
donc diviser le flux de masse
issu des conditions aux limites par la valeur de bord de $\rho^{n+1}$.
Or, lorsque des conditions de Neumann sont appliqu\'ees \`a la
masse volumique,
la valeur de $\rho^{n+1}$ au bord n'est pas connue avant la r\'esolution du
syst\`eme. On utilise donc, au lieu de la valeur de bord inconnue de
$\rho^{n+1}$ la valeur de bord prise au pas de temps
pr\'ec\'edent $\rho^{n}$. Cette approximation est susceptible
d'affecter la valeur du flux de masse au bord.

%                      Code_Saturne version 1.3
%                      ------------------------
%
%     This file is part of the Code_Saturne Kernel, element of the
%     Code_Saturne CFD tool.
%
%     Copyright (C) 1998-2007 EDF S.A., France
%
%     contact: saturne-support@edf.fr
%
%     The Code_Saturne Kernel is free software; you can redistribute it
%     and/or modify it under the terms of the GNU General Public License
%     as published by the Free Software Foundation; either version 2 of
%     the License, or (at your option) any later version.
%
%     The Code_Saturne Kernel is distributed in the hope that it will be
%     useful, but WITHOUT ANY WARRANTY; without even the implied warranty
%     of MERCHANTABILITY or FITNESS FOR A PARTICULAR PURPOSE.  See the
%     GNU General Public License for more details.
%
%     You should have received a copy of the GNU General Public License
%     along with the Code_Saturne Kernel; if not, write to the
%     Free Software Foundation, Inc.,
%     51 Franklin St, Fifth Floor,
%     Boston, MA  02110-1301  USA
%
%-----------------------------------------------------------------------
%


\programme{navsto}

\vspace{1cm}
On s'int\'eresse \`a la r\'esolution du syst\`eme d'\'equations de Navier-Stokes
tridimensionnelles monophasiques, \`a une pression, instationnaires, en
incompressible ou faiblement dilatable, bas\'ees sur une discr\'etisation
temporelle de type Euler implicite d'ordre 1 ou Crank-Nicolson d'ordre 2 et sur
une discr\'etisation spatiale  par volumes finis colocalis\'ee.


%%%%%%%%%%%%%%%%%%%%%%%%%%%%%%%%%%
%%%%%%%%%%%%%%%%%%%%%%%%%%%%%%%%%%
\section{Fonction}
%%%%%%%%%%%%%%%%%%%%%%%%%%%%%%%%%%
%%%%%%%%%%%%%%%%%%%%%%%%%%%%%%%%%%

  Dans ce sous-programme sont calcul\'ees, \`a un pas de temps donn\'e, les
variables vitesse et pression de ce probl\`eme en proc\'edant en
deux  \'etapes issues d'une d\'ecomposition des op\'erateurs (m\'ethode \`a
pas fractionnaires).\\
Les variables sont donc suppos\'ees connues \`a
l'instant ${t^n}$ et on cherche \`a les d\'eterminer \`a l'instant\footnote{La pression est suppos�e connue � l'instant $t^{n-1+\theta}$ et recherch�e en $t^{n+\theta}$, avec $\theta=1$ ou $1/2$ suivant le sch�ma en temps consid�r�.} ${t^{n+1}}$. Soit ${\Delta {t^n} ={t^{n+1}- {t^n}}}$ le pas de temps associ\'e. Dans un premier temps, on r\'ealise l'\'etape de
pr\'ediction de la vitesse en r\'esolvant l'\'equation de quantit\'e de
mouvement avec une pression explicite. Suit l'\'etape de correction de la
pression (ou projection de la vitesse) qui permet d'obtenir un champ de vitesse \`a divergence nulle.\\\\
Les \'equations en continu sont donc :
\begin{equation}
\left\{\begin{array}{l}
\displaystyle\frac{\partial}{\partial t}(\rho \vect{u}) + \dive(\rho\, \vect{u} \otimes \vect{u})
=\dive(\tens{\sigma}) + \vect{TS} - \tens{K}\,\vect{u}\\
\dive(\rho \vect{u}) = \Gamma
\end{array}\right.
\end{equation}

%(plus tard $\frac{\partial \rho}{\partial t} + \dive(\rho \vect{u}) = \Gamma$)



avec $\rho$ la masse volumique, $\vect{u}$ le champ de vitesse,
$[\,\vect{TS}-\tens{K}\,\vect{u}\,]$ les autres termes sources ($\tens{K}$~est un
tenseur diagonal positif par d\'efinition), $\tens{\sigma}$ le tenseur
de contraintes, $\tens{\tau}$ le tenseur des contraintes visqueuses, $\mu$ la
viscosit\'e dynamique (mol\'eculaire et \'eventuellement turbulente), $\kappa$
la viscosit� de
volume (usuellement nulle et n�glig�e dans le code et dans la suite du document,
sauf en compressible),
$\tens{D}$ le tenseur taux de d\'eformation\footnote{\`A ne pas confondre, malgr\'e la m\^eme notation $D$,
avec les flux diffusifs $\vect{D}_{\,ij}$ et $\vect{D}_{\,{b}_{ik}}$ d\'ecrits par la suite dans ce
sous-programme.}, $\Gamma$ le terme source de masse.
\begin{equation}
\left\{\begin{array}{l}
\tens{\sigma} = \tens{\tau} - P\tens{Id}  \\
\tens{\tau} = 2\,\mu\ \tens{D} +\ (\kappa\ - \frac{2}{3}\mu)\  tr({\tens{D}})\
\tens{Id}  \\
\tens{D} = \frac{1}{2}(\ggrad\vect{u}+\,^{t}\ggrad\vect{u})
\end{array}\right.
\end{equation}
 \\

On rappelle la d\'efinition des notations employ\'ees\footnote{en
utilisant la convention de sommation d'Einstein.}~:
\begin{equation}\notag
\left\{\begin{array}{lll}
\left[\ggrad{\vect{a}}\right]_{ij} &=& \partial_j a_i\\
\left[\dive(\tens{\sigma})\right]_i &=& \partial_j \sigma_{ij}\\
\left[\vect{a}\otimes\vect{b}\right]_{ij} &= &
a_i\,b_j\\
\end{array}\right.
\end{equation}
et donc :
\begin{equation}\notag
\begin{array}{lll}
\left[\dive(\vect{a}\otimes\vect{b})\right]_i &= &
\partial_j (a_i\,b_j)
\end{array}
\end{equation}

\minititre{Remarque}
Dans le cas de la prise en compte d'une masse volumique variable, l'�quation de continuit� s'�crit :
$$\frac{\partial \rho}{\partial t} + \dive{\,(\rho\,\vect{u})} = \Gamma  $$
Cette �quation n'est pas prise en compte dans l'�tape de projection (on continue � r�soudre
seulement
$\displaystyle \dive(\,{\rho\,\vect{u}}) = \Gamma$) alors que le terme
$\displaystyle \frac{\partial \rho}{\partial t}$ appara\^{\i}t lors de l'�tape de pr\'ediction de la vitesse
dans le sous-programme \fort{preduv}. Si ce terme joue un r�le sensible, l'algorithme compressible
de \CS\ (qui r�sout l'�quation compl�te) est alors sans doute plus adapt�.

%                      Code_Saturne version 1.3
%                      ------------------------
%
%     This file is part of the Code_Saturne Kernel, element of the
%     Code_Saturne CFD tool.
% 
%     Copyright (C) 1998-2007 EDF S.A., France
%
%     contact: saturne-support@edf.fr
% 
%     The Code_Saturne Kernel is free software; you can redistribute it
%     and/or modify it under the terms of the GNU General Public License
%     as published by the Free Software Foundation; either version 2 of
%     the License, or (at your option) any later version.
% 
%     The Code_Saturne Kernel is distributed in the hope that it will be
%     useful, but WITHOUT ANY WARRANTY; without even the implied warranty
%     of MERCHANTABILITY or FITNESS FOR A PARTICULAR PURPOSE.  See the
%     GNU General Public License for more details.
% 
%     You should have received a copy of the GNU General Public License
%     along with the Code_Saturne Kernel; if not, write to the
%     Free Software Foundation, Inc.,
%     51 Franklin St, Fifth Floor,
%     Boston, MA  02110-1301  USA
%
%-----------------------------------------------------------------------
%
%%%%%%%%%%%%%%%%%%%%%%%%%%%%%%%%%
%%%%%%%%%%%%%%%%%%%%%%%%%%%%%%%%%%
\section{Discr\'etisation}
%%%%%%%%%%%%%%%%%%%%%%%%%%%%%%%%%%
%%%%%%%%%%%%%%%%%%%%%%%%%%%%%%%%%%

Pour utiliser la m�thode, on se place tout d'abord dans un rep�re local d�fini
de mani�re � ce que le plan $(0yz)$, o� sont inject�s les vortex, soit confondu
avec le plan d'entr�e du calcul (voir figure \ref{Base_Vortex_entree}). 

\begin{figure}[h]
\centerline{\includegraphics[height=6cm]{../Base/Vortex/Images/entree.pdf}}
\caption{\label{Base_Vortex_entree} D�finiton des diff�rentes grandeurs dans le rep�re local
correspondant � l'entr�e d'une conduite de section carr�e.} 
\end{figure}

$u$, $v$ et $w$  sont les composantes de la vitesse fluctuante (principale et
transverse) dans ce plan, et
$\displaystyle \omega(y,z) = \frac{\partial w}{\partial y}-\frac{\partial v}{\partial z}$
la vorticit� dans la direction
normale au plan d'entr�e. $\overline{U}(y,z)$ repr�sente ici la vitesse
principale moyenne impos�e par l'utilisateur dans le plan d'entr�e. 

Chaque vortex $p$ va �tre caract�ris� par sa fonction de forme $\xi$ (identique
pour tous les vortex), sa
circulation $\Gamma_p$, son rayon $\sigma_p$ et les coordonn�es $(y_p,z_p)$ du
point $P$ o� est situ� le vortex dans le plan $(0yz)$. 

Pour cela, on suppose que la vorticit� g�n�r�e par un vortex $p$ au point $M$ de
coordonn�e $(y,z)$ s'�crit : 
\begin{equation}\notag
\omega_p(y,z)= \Gamma_p \, \xi_{\sigma_p}(r)
\end{equation}
o� $r = \sqrt{(y-y_p)^2+(z-z_p)^2}$ est la distance s�parant le point $M$ du point $P$.

Dans la m�thode implant�e, la fonction de forme est de type gaussienne modifi�e :
\begin{equation}\notag
\displaystyle
\xi_\sigma (r) = \frac{1}{2\pi \sigma^2} 
\left(2 e^{-\frac{r^2}{2\sigma^2}}-1\right) e^{-\frac{r^2}{2\sigma^2}}
\end{equation}

Le champ de vitesse induit par cette distribution de vorticit� s'obtient par
inversion des deux �quations de poisson suivantes qui sont d�duites de la
condition d'incompressibilit� dans la plan\footnote{\textit{i.e}
$\displaystyle \frac{\partial v}{\partial y}+\frac{\partial w}{\partial w} = 0$} :
\begin{equation}\notag
\begin{array}{lcr}
\displaystyle
\frac{\partial \omega}{\partial y} = \Delta w
&
\text{    et    }
&
\displaystyle
\frac{\partial \omega}{\partial y} = -\Delta v
\\
\end{array}
\end{equation}

Dans le cas g�n�ral, ce syst�me peut �tre int�gr� � l'aide de la formule de Biot et Savart.

Dans le cas d'une distribution de vorticit� de type gaussienne modifi�e, les
composantes de vitesse v�rifient : 
\begin{equation}\notag
\left\{
\begin{array}{c}
\displaystyle
v_p(y,x) = - \frac{1}{2\pi} \frac{(z-z_p)}{r^2}\left(1 -
e^{-\frac{r^2}{2\sigma^2}}\right)\,e^{-\frac{r^2}{2\sigma^2}} 
\\
\displaystyle
w_p(y,z) = \frac{1}{2\pi} \frac{(y-y_p)}{r^2}\left(1 -e^{-\frac{r^2}{2\sigma^2}}
\right)\,e^{-\frac{r^2}{2\sigma^2}} 
\end{array}
\right.
\end{equation}

Ces relations s'�tendent de fa�on imm�diate au cas de $N$ vortex distincts.
Le champ de vitesse induit par la distribution de vorticit� 
\begin{equation}
\omega(y,z) = \sum_{p=1}^N \Gamma_p \, \xi_{\sigma_p}(r)
\end{equation}
vaut au point $M$ :
\begin{equation}\notag
\begin{array}{lcr}
v(x,y) = \sum_{p=1}^N \Gamma_p\, v_p(y,z) 
&
\text{    et    }
&
w(y,z) = \sum_{p=1}^N \Gamma_p\, w_p(y,z)
\\
\label{Base_Vortex_compvit}
\end{array}
\end{equation}
%================================
\subsection{Param�tres physiques}
%================================

%-------------------------------
\subsubsection{Marche en temps}
%-------------------------------
La position initiale de chaque vortex est tir�e de mani�re al�atoire. On calcul
les d�placements successifs de chacun des vortex dans le plan d'entr�e par
int�gration explicite du champ de vitesse lagrangien : 
\begin{equation}\notag
\begin{array}{lcr}
\displaystyle
\frac{dy_p}{dt} = V(y,z)
&
\text{    et    }
&
\displaystyle
\frac{dz_p}{dt} = W(y,z)
\\
\end{array}
\end{equation}
Les vortex sont alors assimil�s � des particules ponctuelles qui sont convect�es
par le champ $(V,W)$. Ce champ peut �tre impos� par des tirages al�atoires ou
bien d�duit de la vitesse induite par les vortex dans le plan d'entr�e. Dans ce
cas $V(x,y) = \overline{V}(y,z) + v (y,z)$ et $W(y,z)= \overline{W}(y,z) +
w(y,z)$ o� $\overline{V}$ et $\overline{W}$ sont les composantes de la vitesse
transverse moyenne qu'impose l'utilisateur � l'aide des fichiers de donn�es. 

%---------------------------------------------------
\subsubsection{Intensit� et dur�e de vie des vortex}
%---------------------------------------------------
Il serait possible, � partir de l'�quation de transport de la vorticit�,
d'obtenir un mod�le d'�volution pour l'intensit� du vecteur tourbillon
$\omega_p$ associ� � chacun des vortex. En pratique, on pr�f�re utiliser un
mod�le simplifi� dans lequel la circulation des tourbillons ne d�pend que de la
postion de ces derniers � l'instant consid�r�. La circulation initiale de chaque
vortex est alors obtenue � partir de la relation : 
\begin{equation}\notag
|\Gamma_p| = 4 \sqrt{\frac{\pi\,S\,k}{3N\,[2ln(3)-3ln(2)]}}
\end{equation}
o� $S$ est la surface du plan d'entr�e, $N$ le nombre de vortex, et $k$
l'�nergie cin�tique turbulente au point o� se trouve le vortex � l'instant
consid�r�. Le signe de $\Gamma_p$ correspond au sens de rotation du vortex et
est tir� al�atoirement. 

Ce param�tre est celui qui contr�le l'intensit� des fluctuations. Sa d�pendance
en $k$ exprime que, plus l'�coulement est turbulent, plus les vortex sont
intenses. La valeur de $k$ est sp�cifi�e par
l'utilisateur. Elle peut �tre constante ou impos�e � partir de profils d'�nergie
cin�tique turbulente en entr�e. 

Pour �viter que des structures trop allong�es ne se d�veloppent au niveau de
l'entr�e, l'utilisateur doit �galement sp�cifier un temps limites $\tau_p$ au
bout duquel le vortex $p$ va �tre d�truit. Ce temps $\tau_p$ peut �tre pris
constant ou estim� au moyen de la relation : 
\begin{equation}\notag
\tau_p = \frac{5 C_{\mu}k^{\frac{3}{2}}}{\varepsilon\,\overline{U}}
\end{equation}

$\overline{U}$ et $\varepsilon$ repr�sentent respectivement la vitesse moyenne
principale et la dissipation turbulente au point o� est initialement g�n�r� le
vortex ($C_{\mu}=0,09$). 
\\
Lorsque le temps �coul� depuis la cr�ation du vortex $p$ est sup�rieur �
$\tau_p$, le vortex est d�truit et un nouveau vortex g�n�r� (sa position et le
signe de $\Gamma_p$ sont tir�s de fa�on al�atoire). 

%-------------------------------- 
\subsubsection{Taille des vortex}
%--------------------------------
La taille des vortex peut �tre prise constante, ou calcul�e � partir des
relations :
\begin{equation}\notag
\begin{array}{ccc}
\displaystyle
\sigma = \frac{C_{\mu}^{\frac{3}{4}}k^{\frac{3}{2}}}{\varepsilon} 
& \text{    ou    } &
\sigma = max[L_t,L_k]
\\
\end{array}
\end{equation}
avec:
\begin{equation}\notag
\begin{array}{ccc}
\displaystyle
L_t = \sqrt{\left( \frac{5 \nu k}{\varepsilon} \right)} 
& \text{    et    } & 
\displaystyle
L_k = 200\, \left(\frac{\nu^3}{\varepsilon}\right)^{\frac{1}{4}}
\end{array}
\end{equation}
o� $\nu$, $k$ et $\varepsilon$ sont la viscosit� dynamique, l'�nergie cin�tique
turbulente et la dissipation turbulente au point o� se trouve le vortex. 

Dans tous les cas, la taille du vortex doit �tre sup�rieure � la taille des
mailles en entr�e afin que le vortex soit effectivement simul�. 

%==================================
\subsection{Conditions aux limites}
%==================================
Le champ de vitesse g�n�r� � l'aide de la relation \ref{Base_Vortex_compvit} ne tient pas
compte {\em a priori} des conditions aux limites appliqu�es sur les bords du plan
d'entr�e. Pour obtenir des valeurs de la vitesse qui soient coh�rentes sur les
fronti�res du domaine d'entr�e, des ``vortex images'', pseudo-vortex situ�s en
dehors du domaine d'entr�e, sont g�n�r�s � des positions particuli�res et leur
champ de vitesse associ� est superpos� au champ pr�c�demment calcul�.\\
Seuls les cas d'une conduite rectangulaire et d'une conduite circulaire
permettent la g�n�ration de ces pseudo-vortex.
On distingue pour cela trois types de conditions aux limites. 

\begin{figure}[h]
\centerline{\includegraphics[height=6cm]{../Base/Vortex/Images/condlimite.pdf}}
\caption{\label{Base_Vortex_condli} Principe de g�n�ration des ``vortex images'' suivant le
type de conditions aux limites dans une conduite carr�e.} 
\end{figure}

%----------------------------------
\subsubsection{Condition de paroi}
%----------------------------------
On cr�e, pour chaque vortex $P$ contenu dans le plan d'entr�e, un vortex image
$P'$ identique � $P$ (\textit{i.e} de m�me caract�ristiques) et sym�trique de $P$
par rapport au
point $J$ ($J$ �tant la projection orthogonalement � la paroi du point $M$
correspondant au centre de la face o� l'on cherche � calculer la vitesse). La
figure \ref{Base_Vortex_condli} illustre la technique dans le cas d'une conduite
carr�e. Dans ce cas les coordonn�es du vortex situ� en $P'$ v�rifient
$(y_{p'}+y_{p})/2 = y_{J}$ et $(z_{p'}+ z_{p})/2 = z_{J}$. Le champ de vitesse
per�u depuis le point $M$ au niveau du point $J$ est nul, ce qui est bien
l'effet recherch�. 

%------------------------------------
\subsubsection{Condition de sym�trie}
%-------------------------------------
La technique est identique � celle utilis�e pour les conditions de paroi, mais
seule la composante pour la vitesse normale au bord est modifi�e dans ce cas. 

%---------------------------------------
\subsubsection{Condition de p�riodicit�}
%---------------------------------------
On cr�e pour chaque vortex, un vortex images $P'$ identique � $P$ mais translat�
d'une quantit� $L$ correspondant � la longueur qui s�pare les deux plans de la
section d'entr�e o� sont appliqu�es les conditions de p�riodicit�. Dans le cas
o� il y a deux directions de p�riodicit�, on cr�e deux vortex image.

%=============================================
\subsection{Composante de vitesse principale}
%=============================================
La m�thode des vortex ne g�n�rant pas de fluctuation $u$ de la vitesse dans la
direction principale, la derni�re composante est calcul�e � partir d'une
�quation de Langevin. Les coefficients de cette �quation sont d�termin�s par
identification des expressions obtenues pour les contraintes de Reynolds en
$R_{ij}-\varepsilon$. Dans le cas d'un �coulement en canal plan, cette technique
conduit � l'�quation : 
\begin{equation}\notag
\displaystyle
\frac{du}{dt} = - \frac{C_1}{2T} u + \left(\frac{2}{3}C_2-1\right)\frac{\partial
U}{\partial y} v + \sqrt{C_0\varepsilon} dW_i 
\end{equation}

avec $\displaystyle T=\frac{k}{\varepsilon}$, $C_1 = 1,8$, $C_2 = 0,6$,
$C_0=\frac{14}{15}$, et $dW_i$ une variable al�toire Gaussienne de variance
$\sqrt{dt}$. 

En pratique, l'�quation de Langevin n'am�liore pas vraiment les r�sultats. Elle
n'est utilis�e que dans le cas de conduites circulaires. 

%                      Code_Saturne version 1.3
%                      ------------------------
%
%     This file is part of the Code_Saturne Kernel, element of the
%     Code_Saturne CFD tool.
%
%     Copyright (C) 1998-2007 EDF S.A., France
%
%     contact: saturne-support@edf.fr
%
%     The Code_Saturne Kernel is free software; you can redistribute it
%     and/or modify it under the terms of the GNU General Public License
%     as published by the Free Software Foundation; either version 2 of
%     the License, or (at your option) any later version.
%
%     The Code_Saturne Kernel is distributed in the hope that it will be
%     useful, but WITHOUT ANY WARRANTY; without even the implied warranty
%     of MERCHANTABILITY or FITNESS FOR A PARTICULAR PURPOSE.  See the
%     GNU General Public License for more details.
%
%     You should have received a copy of the GNU General Public License
%     along with the Code_Saturne Kernel; if not, write to the
%     Free Software Foundation, Inc.,
%     51 Franklin St, Fifth Floor,
%     Boston, MA  02110-1301  USA
%
%-----------------------------------------------------------------------
%

%%%%%%%%%%%%%%%%%%%%%%%%%%%%%%%%%%
%%%%%%%%%%%%%%%%%%%%%%%%%%%%%%%%%%
\section{Mise en \oe uvre}
%%%%%%%%%%%%%%%%%%%%%%%%%%%%%%%%%%
%%%%%%%%%%%%%%%%%%%%%%%%%%%%%%%%%%
Le syst\`eme (\ref{Cfbl_Cfmsvl_eq_densite_finale_cfmsvl}) est r\'esolu par une m\'ethode
d'incr\'ement et r\'esidu en utilisant
une m\'ethode de Jacobi pour inverser le syst\`eme si le terme convectif
est implicite et en utilisant une m\'ethode de gradient conjugu\'e
si le terme convectif est explicite (qui est le cas par d�faut).

Attention, les valeurs du flux de masse $\rho\,\vect{w}\cdot\vect{S}$ et
de la viscosit\'e $\Delta\,t\,c^2\frac{S}{d}$ aux faces de
bord, qui sont calcul\'ees dans \fort{cfmsfl} et \fort{cfmsvs} respectivement,
sont modifi\'ees imm\'ediatement apr\`es l'appel \`a ces sous-programmes.
En effet, il est indispensable que la contribution de bord de
$\left(\rho\,\vect{w}-\Delta\,t\,(c^2)\,\gradv\,\rho\right)\cdot\vect{S}$
repr\'esente exactement $\vect{Q}_{ac}\cdot\vect{S}$.
Pour cela,
\begin{itemize}
\item imm\'ediatement apr\`es l'appel \`a
\fort{cfmsfl}, on remplace la contribution de bord de
$\rho\,\vect{w}\cdot\vect{S}$
par le flux de masse exact, $\vect{Q}_{ac}\cdot\vect{S}$,
d\'etermin\'e \`a partir des conditions aux limites,
\item puis, imm\'ediatement apr\`es l'appel \`a
\fort{cfmsvs}, on annule la viscosit\'e au bord $\Delta\,t\,(c^2)$ pour
\'eliminer la contribution de $-\Delta\,t\,(c^2)\,(\gradv\,\rho)\cdot\vect{S}$
(l'annulation de la viscosit\'e n'est pas probl\'ematique pour la matrice,
puisqu'elle porte sur des incr\'ements).
\end{itemize}

\bigskip

Une fois qu'on a obtenu $\rho^{n+1}$,
on peut actualiser le flux de masse acoustique
aux faces $(\vect{Q}_{ac}^{n+1})_{ij} \cdot \vect{S}_{ij}$,
qui servira pour la convection des autres variables~:
\begin{equation}\label{Cfbl_Cfmsvl_eq_flux_masse_acoustique_cfmsvl}
\displaystyle(\vect{Q}_{ac}^{n+1})_{ij}\cdot\vect{S}_{ij}=
-\left(\Delta t^n (c^2)^n \gradv(\rho^{n+1})\right)_{ij}\cdot\vect{S}_{ij}
+\left(\rho^{n+\frac{1}{2}} \vect{w}^n\right)_{ij}\cdot\vect{S}_{ij}\\
\end{equation}
Ce calcul de flux est r\'ealis\'e par \fort{cfbsc3}.
Si l'on a choisi l'algorithme standard, \'equation~(\ref{Cfbl_Cfmsvl_eq_densite_cfmsvl}),
on compl\`ete le flux dans \fort{cfmsvl} imm\'ediatement apr\`es l'appel
\`a \fort{cfbsc3}.
En effet, dans ce cas,
la convection est explicite ($\rho^{n+\frac{1}{2}}=\rho^{n}$,
obtenu en imposant \var{ICONV(ISCA(IRHO(IPHAS)))=0})
et le sous-programme \fort{cfbsc3},
qui calcule le flux de masse aux faces,
ne prend pas en compte la contribution du terme
$\rho^{n+\frac{1}{2}}\,\vect{w}^n\cdot\vect{S}$. On ajoute donc cette
contribution dans \fort{cfmsvl}, apr\`es l'appel \`a \fort{cfbsc3}.
Au bord, en particulier, c'est bien le flux de masse calcul\'e \`a partir
des conditions aux limites que l'on obtient.

On actualise la pression \`a la fin de l'\'etape, en utilisant la loi d'\'etat~:
\begin{equation}
\displaystyle P_i^{pred}=P(\rho_i^{n+1},\varepsilon_i^{n})
\end{equation}


%%%%%%%%%%%%%%%%%%%%%%%%%%%%%%%%%%
%%%%%%%%%%%%%%%%%%%%%%%%%%%%%%%%%%
\section{Points \`a traiter}
%%%%%%%%%%%%%%%%%%%%%%%%%%%%%%%%%%
%%%%%%%%%%%%%%%%%%%%%%%%%%%%%%%%%%
Le calcul du flux de masse au  bord n'est pas enti\`erement satisfaisant
si la convection est trait\'ee de mani\`ere implicite
(algorithme non standard, non test\'e,
associ\'e \`a l'\'equation~(\ref{Cfbl_Cfmsvl_eq_densite_bis_cfmsvl}),
correspondant au choix $\rho^{n+\frac{1}{2}}=\rho^{n+1}$ et
obtenu en imposant \var{ICONV(ISCA(IRHO(IPHAS)))=1}).
En effet, apr\`es \fort{cfmsfl}, il faut d\'eterminer la vitesse de
convection $\vect{w}^n$ pour qu'apparaisse
$\rho^{n+1} \vect{w}^n\cdot\vect{n}$
au cours de la r\'esolution par \fort{codits}. De ce fait, on doit d\'eduire
une valeur de $\vect{w}^n$ \`a partir de la valeur
du flux de masse. Au bord, en particulier, il faut
donc diviser le flux de masse
issu des conditions aux limites par la valeur de bord de $\rho^{n+1}$.
Or, lorsque des conditions de Neumann sont appliqu\'ees \`a la
masse volumique,
la valeur de $\rho^{n+1}$ au bord n'est pas connue avant la r\'esolution du
syst\`eme. On utilise donc, au lieu de la valeur de bord inconnue de
$\rho^{n+1}$ la valeur de bord prise au pas de temps
pr\'ec\'edent $\rho^{n}$. Cette approximation est susceptible
d'affecter la valeur du flux de masse au bord.

%                      Code_Saturne version 1.3
%                      ------------------------
%
%     This file is part of the Code_Saturne Kernel, element of the
%     Code_Saturne CFD tool.
%
%     Copyright (C) 1998-2007 EDF S.A., France
%
%     contact: saturne-support@edf.fr
%
%     The Code_Saturne Kernel is free software; you can redistribute it
%     and/or modify it under the terms of the GNU General Public License
%     as published by the Free Software Foundation; either version 2 of
%     the License, or (at your option) any later version.
%
%     The Code_Saturne Kernel is distributed in the hope that it will be
%     useful, but WITHOUT ANY WARRANTY; without even the implied warranty
%     of MERCHANTABILITY or FITNESS FOR A PARTICULAR PURPOSE.  See the
%     GNU General Public License for more details.
%
%     You should have received a copy of the GNU General Public License
%     along with the Code_Saturne Kernel; if not, write to the
%     Free Software Foundation, Inc.,
%     51 Franklin St, Fifth Floor,
%     Boston, MA  02110-1301  USA
%
%-----------------------------------------------------------------------
%


\programme{navsto}

\vspace{1cm}
On s'int\'eresse \`a la r\'esolution du syst\`eme d'\'equations de Navier-Stokes
tridimensionnelles monophasiques, \`a une pression, instationnaires, en
incompressible ou faiblement dilatable, bas\'ees sur une discr\'etisation
temporelle de type Euler implicite d'ordre 1 ou Crank-Nicolson d'ordre 2 et sur
une discr\'etisation spatiale  par volumes finis colocalis\'ee.


%%%%%%%%%%%%%%%%%%%%%%%%%%%%%%%%%%
%%%%%%%%%%%%%%%%%%%%%%%%%%%%%%%%%%
\section{Fonction}
%%%%%%%%%%%%%%%%%%%%%%%%%%%%%%%%%%
%%%%%%%%%%%%%%%%%%%%%%%%%%%%%%%%%%

  Dans ce sous-programme sont calcul\'ees, \`a un pas de temps donn\'e, les
variables vitesse et pression de ce probl\`eme en proc\'edant en
deux  \'etapes issues d'une d\'ecomposition des op\'erateurs (m\'ethode \`a
pas fractionnaires).\\
Les variables sont donc suppos\'ees connues \`a
l'instant ${t^n}$ et on cherche \`a les d\'eterminer \`a l'instant\footnote{La pression est suppos�e connue � l'instant $t^{n-1+\theta}$ et recherch�e en $t^{n+\theta}$, avec $\theta=1$ ou $1/2$ suivant le sch�ma en temps consid�r�.} ${t^{n+1}}$. Soit ${\Delta {t^n} ={t^{n+1}- {t^n}}}$ le pas de temps associ\'e. Dans un premier temps, on r\'ealise l'\'etape de
pr\'ediction de la vitesse en r\'esolvant l'\'equation de quantit\'e de
mouvement avec une pression explicite. Suit l'\'etape de correction de la
pression (ou projection de la vitesse) qui permet d'obtenir un champ de vitesse \`a divergence nulle.\\\\
Les \'equations en continu sont donc :
\begin{equation}
\left\{\begin{array}{l}
\displaystyle\frac{\partial}{\partial t}(\rho \vect{u}) + \dive(\rho\, \vect{u} \otimes \vect{u})
=\dive(\tens{\sigma}) + \vect{TS} - \tens{K}\,\vect{u}\\
\dive(\rho \vect{u}) = \Gamma
\end{array}\right.
\end{equation}

%(plus tard $\frac{\partial \rho}{\partial t} + \dive(\rho \vect{u}) = \Gamma$)



avec $\rho$ la masse volumique, $\vect{u}$ le champ de vitesse,
$[\,\vect{TS}-\tens{K}\,\vect{u}\,]$ les autres termes sources ($\tens{K}$~est un
tenseur diagonal positif par d\'efinition), $\tens{\sigma}$ le tenseur
de contraintes, $\tens{\tau}$ le tenseur des contraintes visqueuses, $\mu$ la
viscosit\'e dynamique (mol\'eculaire et \'eventuellement turbulente), $\kappa$
la viscosit� de
volume (usuellement nulle et n�glig�e dans le code et dans la suite du document,
sauf en compressible),
$\tens{D}$ le tenseur taux de d\'eformation\footnote{\`A ne pas confondre, malgr\'e la m\^eme notation $D$,
avec les flux diffusifs $\vect{D}_{\,ij}$ et $\vect{D}_{\,{b}_{ik}}$ d\'ecrits par la suite dans ce
sous-programme.}, $\Gamma$ le terme source de masse.
\begin{equation}
\left\{\begin{array}{l}
\tens{\sigma} = \tens{\tau} - P\tens{Id}  \\
\tens{\tau} = 2\,\mu\ \tens{D} +\ (\kappa\ - \frac{2}{3}\mu)\  tr({\tens{D}})\
\tens{Id}  \\
\tens{D} = \frac{1}{2}(\ggrad\vect{u}+\,^{t}\ggrad\vect{u})
\end{array}\right.
\end{equation}
 \\

On rappelle la d\'efinition des notations employ\'ees\footnote{en
utilisant la convention de sommation d'Einstein.}~:
\begin{equation}\notag
\left\{\begin{array}{lll}
\left[\ggrad{\vect{a}}\right]_{ij} &=& \partial_j a_i\\
\left[\dive(\tens{\sigma})\right]_i &=& \partial_j \sigma_{ij}\\
\left[\vect{a}\otimes\vect{b}\right]_{ij} &= &
a_i\,b_j\\
\end{array}\right.
\end{equation}
et donc :
\begin{equation}\notag
\begin{array}{lll}
\left[\dive(\vect{a}\otimes\vect{b})\right]_i &= &
\partial_j (a_i\,b_j)
\end{array}
\end{equation}

\minititre{Remarque}
Dans le cas de la prise en compte d'une masse volumique variable, l'�quation de continuit� s'�crit :
$$\frac{\partial \rho}{\partial t} + \dive{\,(\rho\,\vect{u})} = \Gamma  $$
Cette �quation n'est pas prise en compte dans l'�tape de projection (on continue � r�soudre
seulement
$\displaystyle \dive(\,{\rho\,\vect{u}}) = \Gamma$) alors que le terme
$\displaystyle \frac{\partial \rho}{\partial t}$ appara\^{\i}t lors de l'�tape de pr\'ediction de la vitesse
dans le sous-programme \fort{preduv}. Si ce terme joue un r�le sensible, l'algorithme compressible
de \CS\ (qui r�sout l'�quation compl�te) est alors sans doute plus adapt�.

%                      Code_Saturne version 1.3
%                      ------------------------
%
%     This file is part of the Code_Saturne Kernel, element of the
%     Code_Saturne CFD tool.
% 
%     Copyright (C) 1998-2007 EDF S.A., France
%
%     contact: saturne-support@edf.fr
% 
%     The Code_Saturne Kernel is free software; you can redistribute it
%     and/or modify it under the terms of the GNU General Public License
%     as published by the Free Software Foundation; either version 2 of
%     the License, or (at your option) any later version.
% 
%     The Code_Saturne Kernel is distributed in the hope that it will be
%     useful, but WITHOUT ANY WARRANTY; without even the implied warranty
%     of MERCHANTABILITY or FITNESS FOR A PARTICULAR PURPOSE.  See the
%     GNU General Public License for more details.
% 
%     You should have received a copy of the GNU General Public License
%     along with the Code_Saturne Kernel; if not, write to the
%     Free Software Foundation, Inc.,
%     51 Franklin St, Fifth Floor,
%     Boston, MA  02110-1301  USA
%
%-----------------------------------------------------------------------
%
%%%%%%%%%%%%%%%%%%%%%%%%%%%%%%%%%
%%%%%%%%%%%%%%%%%%%%%%%%%%%%%%%%%%
\section{Discr\'etisation}
%%%%%%%%%%%%%%%%%%%%%%%%%%%%%%%%%%
%%%%%%%%%%%%%%%%%%%%%%%%%%%%%%%%%%

Pour utiliser la m�thode, on se place tout d'abord dans un rep�re local d�fini
de mani�re � ce que le plan $(0yz)$, o� sont inject�s les vortex, soit confondu
avec le plan d'entr�e du calcul (voir figure \ref{Base_Vortex_entree}). 

\begin{figure}[h]
\centerline{\includegraphics[height=6cm]{../Base/Vortex/Images/entree.pdf}}
\caption{\label{Base_Vortex_entree} D�finiton des diff�rentes grandeurs dans le rep�re local
correspondant � l'entr�e d'une conduite de section carr�e.} 
\end{figure}

$u$, $v$ et $w$  sont les composantes de la vitesse fluctuante (principale et
transverse) dans ce plan, et
$\displaystyle \omega(y,z) = \frac{\partial w}{\partial y}-\frac{\partial v}{\partial z}$
la vorticit� dans la direction
normale au plan d'entr�e. $\overline{U}(y,z)$ repr�sente ici la vitesse
principale moyenne impos�e par l'utilisateur dans le plan d'entr�e. 

Chaque vortex $p$ va �tre caract�ris� par sa fonction de forme $\xi$ (identique
pour tous les vortex), sa
circulation $\Gamma_p$, son rayon $\sigma_p$ et les coordonn�es $(y_p,z_p)$ du
point $P$ o� est situ� le vortex dans le plan $(0yz)$. 

Pour cela, on suppose que la vorticit� g�n�r�e par un vortex $p$ au point $M$ de
coordonn�e $(y,z)$ s'�crit : 
\begin{equation}\notag
\omega_p(y,z)= \Gamma_p \, \xi_{\sigma_p}(r)
\end{equation}
o� $r = \sqrt{(y-y_p)^2+(z-z_p)^2}$ est la distance s�parant le point $M$ du point $P$.

Dans la m�thode implant�e, la fonction de forme est de type gaussienne modifi�e :
\begin{equation}\notag
\displaystyle
\xi_\sigma (r) = \frac{1}{2\pi \sigma^2} 
\left(2 e^{-\frac{r^2}{2\sigma^2}}-1\right) e^{-\frac{r^2}{2\sigma^2}}
\end{equation}

Le champ de vitesse induit par cette distribution de vorticit� s'obtient par
inversion des deux �quations de poisson suivantes qui sont d�duites de la
condition d'incompressibilit� dans la plan\footnote{\textit{i.e}
$\displaystyle \frac{\partial v}{\partial y}+\frac{\partial w}{\partial w} = 0$} :
\begin{equation}\notag
\begin{array}{lcr}
\displaystyle
\frac{\partial \omega}{\partial y} = \Delta w
&
\text{    et    }
&
\displaystyle
\frac{\partial \omega}{\partial y} = -\Delta v
\\
\end{array}
\end{equation}

Dans le cas g�n�ral, ce syst�me peut �tre int�gr� � l'aide de la formule de Biot et Savart.

Dans le cas d'une distribution de vorticit� de type gaussienne modifi�e, les
composantes de vitesse v�rifient : 
\begin{equation}\notag
\left\{
\begin{array}{c}
\displaystyle
v_p(y,x) = - \frac{1}{2\pi} \frac{(z-z_p)}{r^2}\left(1 -
e^{-\frac{r^2}{2\sigma^2}}\right)\,e^{-\frac{r^2}{2\sigma^2}} 
\\
\displaystyle
w_p(y,z) = \frac{1}{2\pi} \frac{(y-y_p)}{r^2}\left(1 -e^{-\frac{r^2}{2\sigma^2}}
\right)\,e^{-\frac{r^2}{2\sigma^2}} 
\end{array}
\right.
\end{equation}

Ces relations s'�tendent de fa�on imm�diate au cas de $N$ vortex distincts.
Le champ de vitesse induit par la distribution de vorticit� 
\begin{equation}
\omega(y,z) = \sum_{p=1}^N \Gamma_p \, \xi_{\sigma_p}(r)
\end{equation}
vaut au point $M$ :
\begin{equation}\notag
\begin{array}{lcr}
v(x,y) = \sum_{p=1}^N \Gamma_p\, v_p(y,z) 
&
\text{    et    }
&
w(y,z) = \sum_{p=1}^N \Gamma_p\, w_p(y,z)
\\
\label{Base_Vortex_compvit}
\end{array}
\end{equation}
%================================
\subsection{Param�tres physiques}
%================================

%-------------------------------
\subsubsection{Marche en temps}
%-------------------------------
La position initiale de chaque vortex est tir�e de mani�re al�atoire. On calcul
les d�placements successifs de chacun des vortex dans le plan d'entr�e par
int�gration explicite du champ de vitesse lagrangien : 
\begin{equation}\notag
\begin{array}{lcr}
\displaystyle
\frac{dy_p}{dt} = V(y,z)
&
\text{    et    }
&
\displaystyle
\frac{dz_p}{dt} = W(y,z)
\\
\end{array}
\end{equation}
Les vortex sont alors assimil�s � des particules ponctuelles qui sont convect�es
par le champ $(V,W)$. Ce champ peut �tre impos� par des tirages al�atoires ou
bien d�duit de la vitesse induite par les vortex dans le plan d'entr�e. Dans ce
cas $V(x,y) = \overline{V}(y,z) + v (y,z)$ et $W(y,z)= \overline{W}(y,z) +
w(y,z)$ o� $\overline{V}$ et $\overline{W}$ sont les composantes de la vitesse
transverse moyenne qu'impose l'utilisateur � l'aide des fichiers de donn�es. 

%---------------------------------------------------
\subsubsection{Intensit� et dur�e de vie des vortex}
%---------------------------------------------------
Il serait possible, � partir de l'�quation de transport de la vorticit�,
d'obtenir un mod�le d'�volution pour l'intensit� du vecteur tourbillon
$\omega_p$ associ� � chacun des vortex. En pratique, on pr�f�re utiliser un
mod�le simplifi� dans lequel la circulation des tourbillons ne d�pend que de la
postion de ces derniers � l'instant consid�r�. La circulation initiale de chaque
vortex est alors obtenue � partir de la relation : 
\begin{equation}\notag
|\Gamma_p| = 4 \sqrt{\frac{\pi\,S\,k}{3N\,[2ln(3)-3ln(2)]}}
\end{equation}
o� $S$ est la surface du plan d'entr�e, $N$ le nombre de vortex, et $k$
l'�nergie cin�tique turbulente au point o� se trouve le vortex � l'instant
consid�r�. Le signe de $\Gamma_p$ correspond au sens de rotation du vortex et
est tir� al�atoirement. 

Ce param�tre est celui qui contr�le l'intensit� des fluctuations. Sa d�pendance
en $k$ exprime que, plus l'�coulement est turbulent, plus les vortex sont
intenses. La valeur de $k$ est sp�cifi�e par
l'utilisateur. Elle peut �tre constante ou impos�e � partir de profils d'�nergie
cin�tique turbulente en entr�e. 

Pour �viter que des structures trop allong�es ne se d�veloppent au niveau de
l'entr�e, l'utilisateur doit �galement sp�cifier un temps limites $\tau_p$ au
bout duquel le vortex $p$ va �tre d�truit. Ce temps $\tau_p$ peut �tre pris
constant ou estim� au moyen de la relation : 
\begin{equation}\notag
\tau_p = \frac{5 C_{\mu}k^{\frac{3}{2}}}{\varepsilon\,\overline{U}}
\end{equation}

$\overline{U}$ et $\varepsilon$ repr�sentent respectivement la vitesse moyenne
principale et la dissipation turbulente au point o� est initialement g�n�r� le
vortex ($C_{\mu}=0,09$). 
\\
Lorsque le temps �coul� depuis la cr�ation du vortex $p$ est sup�rieur �
$\tau_p$, le vortex est d�truit et un nouveau vortex g�n�r� (sa position et le
signe de $\Gamma_p$ sont tir�s de fa�on al�atoire). 

%-------------------------------- 
\subsubsection{Taille des vortex}
%--------------------------------
La taille des vortex peut �tre prise constante, ou calcul�e � partir des
relations :
\begin{equation}\notag
\begin{array}{ccc}
\displaystyle
\sigma = \frac{C_{\mu}^{\frac{3}{4}}k^{\frac{3}{2}}}{\varepsilon} 
& \text{    ou    } &
\sigma = max[L_t,L_k]
\\
\end{array}
\end{equation}
avec:
\begin{equation}\notag
\begin{array}{ccc}
\displaystyle
L_t = \sqrt{\left( \frac{5 \nu k}{\varepsilon} \right)} 
& \text{    et    } & 
\displaystyle
L_k = 200\, \left(\frac{\nu^3}{\varepsilon}\right)^{\frac{1}{4}}
\end{array}
\end{equation}
o� $\nu$, $k$ et $\varepsilon$ sont la viscosit� dynamique, l'�nergie cin�tique
turbulente et la dissipation turbulente au point o� se trouve le vortex. 

Dans tous les cas, la taille du vortex doit �tre sup�rieure � la taille des
mailles en entr�e afin que le vortex soit effectivement simul�. 

%==================================
\subsection{Conditions aux limites}
%==================================
Le champ de vitesse g�n�r� � l'aide de la relation \ref{Base_Vortex_compvit} ne tient pas
compte {\em a priori} des conditions aux limites appliqu�es sur les bords du plan
d'entr�e. Pour obtenir des valeurs de la vitesse qui soient coh�rentes sur les
fronti�res du domaine d'entr�e, des ``vortex images'', pseudo-vortex situ�s en
dehors du domaine d'entr�e, sont g�n�r�s � des positions particuli�res et leur
champ de vitesse associ� est superpos� au champ pr�c�demment calcul�.\\
Seuls les cas d'une conduite rectangulaire et d'une conduite circulaire
permettent la g�n�ration de ces pseudo-vortex.
On distingue pour cela trois types de conditions aux limites. 

\begin{figure}[h]
\centerline{\includegraphics[height=6cm]{../Base/Vortex/Images/condlimite.pdf}}
\caption{\label{Base_Vortex_condli} Principe de g�n�ration des ``vortex images'' suivant le
type de conditions aux limites dans une conduite carr�e.} 
\end{figure}

%----------------------------------
\subsubsection{Condition de paroi}
%----------------------------------
On cr�e, pour chaque vortex $P$ contenu dans le plan d'entr�e, un vortex image
$P'$ identique � $P$ (\textit{i.e} de m�me caract�ristiques) et sym�trique de $P$
par rapport au
point $J$ ($J$ �tant la projection orthogonalement � la paroi du point $M$
correspondant au centre de la face o� l'on cherche � calculer la vitesse). La
figure \ref{Base_Vortex_condli} illustre la technique dans le cas d'une conduite
carr�e. Dans ce cas les coordonn�es du vortex situ� en $P'$ v�rifient
$(y_{p'}+y_{p})/2 = y_{J}$ et $(z_{p'}+ z_{p})/2 = z_{J}$. Le champ de vitesse
per�u depuis le point $M$ au niveau du point $J$ est nul, ce qui est bien
l'effet recherch�. 

%------------------------------------
\subsubsection{Condition de sym�trie}
%-------------------------------------
La technique est identique � celle utilis�e pour les conditions de paroi, mais
seule la composante pour la vitesse normale au bord est modifi�e dans ce cas. 

%---------------------------------------
\subsubsection{Condition de p�riodicit�}
%---------------------------------------
On cr�e pour chaque vortex, un vortex images $P'$ identique � $P$ mais translat�
d'une quantit� $L$ correspondant � la longueur qui s�pare les deux plans de la
section d'entr�e o� sont appliqu�es les conditions de p�riodicit�. Dans le cas
o� il y a deux directions de p�riodicit�, on cr�e deux vortex image.

%=============================================
\subsection{Composante de vitesse principale}
%=============================================
La m�thode des vortex ne g�n�rant pas de fluctuation $u$ de la vitesse dans la
direction principale, la derni�re composante est calcul�e � partir d'une
�quation de Langevin. Les coefficients de cette �quation sont d�termin�s par
identification des expressions obtenues pour les contraintes de Reynolds en
$R_{ij}-\varepsilon$. Dans le cas d'un �coulement en canal plan, cette technique
conduit � l'�quation : 
\begin{equation}\notag
\displaystyle
\frac{du}{dt} = - \frac{C_1}{2T} u + \left(\frac{2}{3}C_2-1\right)\frac{\partial
U}{\partial y} v + \sqrt{C_0\varepsilon} dW_i 
\end{equation}

avec $\displaystyle T=\frac{k}{\varepsilon}$, $C_1 = 1,8$, $C_2 = 0,6$,
$C_0=\frac{14}{15}$, et $dW_i$ une variable al�toire Gaussienne de variance
$\sqrt{dt}$. 

En pratique, l'�quation de Langevin n'am�liore pas vraiment les r�sultats. Elle
n'est utilis�e que dans le cas de conduites circulaires. 

%                      Code_Saturne version 1.3
%                      ------------------------
%
%     This file is part of the Code_Saturne Kernel, element of the
%     Code_Saturne CFD tool.
%
%     Copyright (C) 1998-2007 EDF S.A., France
%
%     contact: saturne-support@edf.fr
%
%     The Code_Saturne Kernel is free software; you can redistribute it
%     and/or modify it under the terms of the GNU General Public License
%     as published by the Free Software Foundation; either version 2 of
%     the License, or (at your option) any later version.
%
%     The Code_Saturne Kernel is distributed in the hope that it will be
%     useful, but WITHOUT ANY WARRANTY; without even the implied warranty
%     of MERCHANTABILITY or FITNESS FOR A PARTICULAR PURPOSE.  See the
%     GNU General Public License for more details.
%
%     You should have received a copy of the GNU General Public License
%     along with the Code_Saturne Kernel; if not, write to the
%     Free Software Foundation, Inc.,
%     51 Franklin St, Fifth Floor,
%     Boston, MA  02110-1301  USA
%
%-----------------------------------------------------------------------
%

%%%%%%%%%%%%%%%%%%%%%%%%%%%%%%%%%%
%%%%%%%%%%%%%%%%%%%%%%%%%%%%%%%%%%
\section{Mise en \oe uvre}
%%%%%%%%%%%%%%%%%%%%%%%%%%%%%%%%%%
%%%%%%%%%%%%%%%%%%%%%%%%%%%%%%%%%%
Le syst\`eme (\ref{Cfbl_Cfmsvl_eq_densite_finale_cfmsvl}) est r\'esolu par une m\'ethode
d'incr\'ement et r\'esidu en utilisant
une m\'ethode de Jacobi pour inverser le syst\`eme si le terme convectif
est implicite et en utilisant une m\'ethode de gradient conjugu\'e
si le terme convectif est explicite (qui est le cas par d�faut).

Attention, les valeurs du flux de masse $\rho\,\vect{w}\cdot\vect{S}$ et
de la viscosit\'e $\Delta\,t\,c^2\frac{S}{d}$ aux faces de
bord, qui sont calcul\'ees dans \fort{cfmsfl} et \fort{cfmsvs} respectivement,
sont modifi\'ees imm\'ediatement apr\`es l'appel \`a ces sous-programmes.
En effet, il est indispensable que la contribution de bord de
$\left(\rho\,\vect{w}-\Delta\,t\,(c^2)\,\gradv\,\rho\right)\cdot\vect{S}$
repr\'esente exactement $\vect{Q}_{ac}\cdot\vect{S}$.
Pour cela,
\begin{itemize}
\item imm\'ediatement apr\`es l'appel \`a
\fort{cfmsfl}, on remplace la contribution de bord de
$\rho\,\vect{w}\cdot\vect{S}$
par le flux de masse exact, $\vect{Q}_{ac}\cdot\vect{S}$,
d\'etermin\'e \`a partir des conditions aux limites,
\item puis, imm\'ediatement apr\`es l'appel \`a
\fort{cfmsvs}, on annule la viscosit\'e au bord $\Delta\,t\,(c^2)$ pour
\'eliminer la contribution de $-\Delta\,t\,(c^2)\,(\gradv\,\rho)\cdot\vect{S}$
(l'annulation de la viscosit\'e n'est pas probl\'ematique pour la matrice,
puisqu'elle porte sur des incr\'ements).
\end{itemize}

\bigskip

Une fois qu'on a obtenu $\rho^{n+1}$,
on peut actualiser le flux de masse acoustique
aux faces $(\vect{Q}_{ac}^{n+1})_{ij} \cdot \vect{S}_{ij}$,
qui servira pour la convection des autres variables~:
\begin{equation}\label{Cfbl_Cfmsvl_eq_flux_masse_acoustique_cfmsvl}
\displaystyle(\vect{Q}_{ac}^{n+1})_{ij}\cdot\vect{S}_{ij}=
-\left(\Delta t^n (c^2)^n \gradv(\rho^{n+1})\right)_{ij}\cdot\vect{S}_{ij}
+\left(\rho^{n+\frac{1}{2}} \vect{w}^n\right)_{ij}\cdot\vect{S}_{ij}\\
\end{equation}
Ce calcul de flux est r\'ealis\'e par \fort{cfbsc3}.
Si l'on a choisi l'algorithme standard, \'equation~(\ref{Cfbl_Cfmsvl_eq_densite_cfmsvl}),
on compl\`ete le flux dans \fort{cfmsvl} imm\'ediatement apr\`es l'appel
\`a \fort{cfbsc3}.
En effet, dans ce cas,
la convection est explicite ($\rho^{n+\frac{1}{2}}=\rho^{n}$,
obtenu en imposant \var{ICONV(ISCA(IRHO(IPHAS)))=0})
et le sous-programme \fort{cfbsc3},
qui calcule le flux de masse aux faces,
ne prend pas en compte la contribution du terme
$\rho^{n+\frac{1}{2}}\,\vect{w}^n\cdot\vect{S}$. On ajoute donc cette
contribution dans \fort{cfmsvl}, apr\`es l'appel \`a \fort{cfbsc3}.
Au bord, en particulier, c'est bien le flux de masse calcul\'e \`a partir
des conditions aux limites que l'on obtient.

On actualise la pression \`a la fin de l'\'etape, en utilisant la loi d'\'etat~:
\begin{equation}
\displaystyle P_i^{pred}=P(\rho_i^{n+1},\varepsilon_i^{n})
\end{equation}


%%%%%%%%%%%%%%%%%%%%%%%%%%%%%%%%%%
%%%%%%%%%%%%%%%%%%%%%%%%%%%%%%%%%%
\section{Points \`a traiter}
%%%%%%%%%%%%%%%%%%%%%%%%%%%%%%%%%%
%%%%%%%%%%%%%%%%%%%%%%%%%%%%%%%%%%
Le calcul du flux de masse au  bord n'est pas enti\`erement satisfaisant
si la convection est trait\'ee de mani\`ere implicite
(algorithme non standard, non test\'e,
associ\'e \`a l'\'equation~(\ref{Cfbl_Cfmsvl_eq_densite_bis_cfmsvl}),
correspondant au choix $\rho^{n+\frac{1}{2}}=\rho^{n+1}$ et
obtenu en imposant \var{ICONV(ISCA(IRHO(IPHAS)))=1}).
En effet, apr\`es \fort{cfmsfl}, il faut d\'eterminer la vitesse de
convection $\vect{w}^n$ pour qu'apparaisse
$\rho^{n+1} \vect{w}^n\cdot\vect{n}$
au cours de la r\'esolution par \fort{codits}. De ce fait, on doit d\'eduire
une valeur de $\vect{w}^n$ \`a partir de la valeur
du flux de masse. Au bord, en particulier, il faut
donc diviser le flux de masse
issu des conditions aux limites par la valeur de bord de $\rho^{n+1}$.
Or, lorsque des conditions de Neumann sont appliqu\'ees \`a la
masse volumique,
la valeur de $\rho^{n+1}$ au bord n'est pas connue avant la r\'esolution du
syst\`eme. On utilise donc, au lieu de la valeur de bord inconnue de
$\rho^{n+1}$ la valeur de bord prise au pas de temps
pr\'ec\'edent $\rho^{n}$. Cette approximation est susceptible
d'affecter la valeur du flux de masse au bord.

%                      Code_Saturne version 1.3
%                      ------------------------
%
%     This file is part of the Code_Saturne Kernel, element of the
%     Code_Saturne CFD tool.
%
%     Copyright (C) 1998-2007 EDF S.A., France
%
%     contact: saturne-support@edf.fr
%
%     The Code_Saturne Kernel is free software; you can redistribute it
%     and/or modify it under the terms of the GNU General Public License
%     as published by the Free Software Foundation; either version 2 of
%     the License, or (at your option) any later version.
%
%     The Code_Saturne Kernel is distributed in the hope that it will be
%     useful, but WITHOUT ANY WARRANTY; without even the implied warranty
%     of MERCHANTABILITY or FITNESS FOR A PARTICULAR PURPOSE.  See the
%     GNU General Public License for more details.
%
%     You should have received a copy of the GNU General Public License
%     along with the Code_Saturne Kernel; if not, write to the
%     Free Software Foundation, Inc.,
%     51 Franklin St, Fifth Floor,
%     Boston, MA  02110-1301  USA
%
%-----------------------------------------------------------------------
%


\programme{navsto}

\vspace{1cm}
On s'int\'eresse \`a la r\'esolution du syst\`eme d'\'equations de Navier-Stokes
tridimensionnelles monophasiques, \`a une pression, instationnaires, en
incompressible ou faiblement dilatable, bas\'ees sur une discr\'etisation
temporelle de type Euler implicite d'ordre 1 ou Crank-Nicolson d'ordre 2 et sur
une discr\'etisation spatiale  par volumes finis colocalis\'ee.


%%%%%%%%%%%%%%%%%%%%%%%%%%%%%%%%%%
%%%%%%%%%%%%%%%%%%%%%%%%%%%%%%%%%%
\section{Fonction}
%%%%%%%%%%%%%%%%%%%%%%%%%%%%%%%%%%
%%%%%%%%%%%%%%%%%%%%%%%%%%%%%%%%%%

  Dans ce sous-programme sont calcul\'ees, \`a un pas de temps donn\'e, les
variables vitesse et pression de ce probl\`eme en proc\'edant en
deux  \'etapes issues d'une d\'ecomposition des op\'erateurs (m\'ethode \`a
pas fractionnaires).\\
Les variables sont donc suppos\'ees connues \`a
l'instant ${t^n}$ et on cherche \`a les d\'eterminer \`a l'instant\footnote{La pression est suppos�e connue � l'instant $t^{n-1+\theta}$ et recherch�e en $t^{n+\theta}$, avec $\theta=1$ ou $1/2$ suivant le sch�ma en temps consid�r�.} ${t^{n+1}}$. Soit ${\Delta {t^n} ={t^{n+1}- {t^n}}}$ le pas de temps associ\'e. Dans un premier temps, on r\'ealise l'\'etape de
pr\'ediction de la vitesse en r\'esolvant l'\'equation de quantit\'e de
mouvement avec une pression explicite. Suit l'\'etape de correction de la
pression (ou projection de la vitesse) qui permet d'obtenir un champ de vitesse \`a divergence nulle.\\\\
Les \'equations en continu sont donc :
\begin{equation}
\left\{\begin{array}{l}
\displaystyle\frac{\partial}{\partial t}(\rho \vect{u}) + \dive(\rho\, \vect{u} \otimes \vect{u})
=\dive(\tens{\sigma}) + \vect{TS} - \tens{K}\,\vect{u}\\
\dive(\rho \vect{u}) = \Gamma
\end{array}\right.
\end{equation}

%(plus tard $\frac{\partial \rho}{\partial t} + \dive(\rho \vect{u}) = \Gamma$)



avec $\rho$ la masse volumique, $\vect{u}$ le champ de vitesse,
$[\,\vect{TS}-\tens{K}\,\vect{u}\,]$ les autres termes sources ($\tens{K}$~est un
tenseur diagonal positif par d\'efinition), $\tens{\sigma}$ le tenseur
de contraintes, $\tens{\tau}$ le tenseur des contraintes visqueuses, $\mu$ la
viscosit\'e dynamique (mol\'eculaire et \'eventuellement turbulente), $\kappa$
la viscosit� de
volume (usuellement nulle et n�glig�e dans le code et dans la suite du document,
sauf en compressible),
$\tens{D}$ le tenseur taux de d\'eformation\footnote{\`A ne pas confondre, malgr\'e la m\^eme notation $D$,
avec les flux diffusifs $\vect{D}_{\,ij}$ et $\vect{D}_{\,{b}_{ik}}$ d\'ecrits par la suite dans ce
sous-programme.}, $\Gamma$ le terme source de masse.
\begin{equation}
\left\{\begin{array}{l}
\tens{\sigma} = \tens{\tau} - P\tens{Id}  \\
\tens{\tau} = 2\,\mu\ \tens{D} +\ (\kappa\ - \frac{2}{3}\mu)\  tr({\tens{D}})\
\tens{Id}  \\
\tens{D} = \frac{1}{2}(\ggrad\vect{u}+\,^{t}\ggrad\vect{u})
\end{array}\right.
\end{equation}
 \\

On rappelle la d\'efinition des notations employ\'ees\footnote{en
utilisant la convention de sommation d'Einstein.}~:
\begin{equation}\notag
\left\{\begin{array}{lll}
\left[\ggrad{\vect{a}}\right]_{ij} &=& \partial_j a_i\\
\left[\dive(\tens{\sigma})\right]_i &=& \partial_j \sigma_{ij}\\
\left[\vect{a}\otimes\vect{b}\right]_{ij} &= &
a_i\,b_j\\
\end{array}\right.
\end{equation}
et donc :
\begin{equation}\notag
\begin{array}{lll}
\left[\dive(\vect{a}\otimes\vect{b})\right]_i &= &
\partial_j (a_i\,b_j)
\end{array}
\end{equation}

\minititre{Remarque}
Dans le cas de la prise en compte d'une masse volumique variable, l'�quation de continuit� s'�crit :
$$\frac{\partial \rho}{\partial t} + \dive{\,(\rho\,\vect{u})} = \Gamma  $$
Cette �quation n'est pas prise en compte dans l'�tape de projection (on continue � r�soudre
seulement
$\displaystyle \dive(\,{\rho\,\vect{u}}) = \Gamma$) alors que le terme
$\displaystyle \frac{\partial \rho}{\partial t}$ appara\^{\i}t lors de l'�tape de pr\'ediction de la vitesse
dans le sous-programme \fort{preduv}. Si ce terme joue un r�le sensible, l'algorithme compressible
de \CS\ (qui r�sout l'�quation compl�te) est alors sans doute plus adapt�.

%                      Code_Saturne version 1.3
%                      ------------------------
%
%     This file is part of the Code_Saturne Kernel, element of the
%     Code_Saturne CFD tool.
% 
%     Copyright (C) 1998-2007 EDF S.A., France
%
%     contact: saturne-support@edf.fr
% 
%     The Code_Saturne Kernel is free software; you can redistribute it
%     and/or modify it under the terms of the GNU General Public License
%     as published by the Free Software Foundation; either version 2 of
%     the License, or (at your option) any later version.
% 
%     The Code_Saturne Kernel is distributed in the hope that it will be
%     useful, but WITHOUT ANY WARRANTY; without even the implied warranty
%     of MERCHANTABILITY or FITNESS FOR A PARTICULAR PURPOSE.  See the
%     GNU General Public License for more details.
% 
%     You should have received a copy of the GNU General Public License
%     along with the Code_Saturne Kernel; if not, write to the
%     Free Software Foundation, Inc.,
%     51 Franklin St, Fifth Floor,
%     Boston, MA  02110-1301  USA
%
%-----------------------------------------------------------------------
%
%%%%%%%%%%%%%%%%%%%%%%%%%%%%%%%%%
%%%%%%%%%%%%%%%%%%%%%%%%%%%%%%%%%%
\section{Discr\'etisation}
%%%%%%%%%%%%%%%%%%%%%%%%%%%%%%%%%%
%%%%%%%%%%%%%%%%%%%%%%%%%%%%%%%%%%

Pour utiliser la m�thode, on se place tout d'abord dans un rep�re local d�fini
de mani�re � ce que le plan $(0yz)$, o� sont inject�s les vortex, soit confondu
avec le plan d'entr�e du calcul (voir figure \ref{Base_Vortex_entree}). 

\begin{figure}[h]
\centerline{\includegraphics[height=6cm]{../Base/Vortex/Images/entree.pdf}}
\caption{\label{Base_Vortex_entree} D�finiton des diff�rentes grandeurs dans le rep�re local
correspondant � l'entr�e d'une conduite de section carr�e.} 
\end{figure}

$u$, $v$ et $w$  sont les composantes de la vitesse fluctuante (principale et
transverse) dans ce plan, et
$\displaystyle \omega(y,z) = \frac{\partial w}{\partial y}-\frac{\partial v}{\partial z}$
la vorticit� dans la direction
normale au plan d'entr�e. $\overline{U}(y,z)$ repr�sente ici la vitesse
principale moyenne impos�e par l'utilisateur dans le plan d'entr�e. 

Chaque vortex $p$ va �tre caract�ris� par sa fonction de forme $\xi$ (identique
pour tous les vortex), sa
circulation $\Gamma_p$, son rayon $\sigma_p$ et les coordonn�es $(y_p,z_p)$ du
point $P$ o� est situ� le vortex dans le plan $(0yz)$. 

Pour cela, on suppose que la vorticit� g�n�r�e par un vortex $p$ au point $M$ de
coordonn�e $(y,z)$ s'�crit : 
\begin{equation}\notag
\omega_p(y,z)= \Gamma_p \, \xi_{\sigma_p}(r)
\end{equation}
o� $r = \sqrt{(y-y_p)^2+(z-z_p)^2}$ est la distance s�parant le point $M$ du point $P$.

Dans la m�thode implant�e, la fonction de forme est de type gaussienne modifi�e :
\begin{equation}\notag
\displaystyle
\xi_\sigma (r) = \frac{1}{2\pi \sigma^2} 
\left(2 e^{-\frac{r^2}{2\sigma^2}}-1\right) e^{-\frac{r^2}{2\sigma^2}}
\end{equation}

Le champ de vitesse induit par cette distribution de vorticit� s'obtient par
inversion des deux �quations de poisson suivantes qui sont d�duites de la
condition d'incompressibilit� dans la plan\footnote{\textit{i.e}
$\displaystyle \frac{\partial v}{\partial y}+\frac{\partial w}{\partial w} = 0$} :
\begin{equation}\notag
\begin{array}{lcr}
\displaystyle
\frac{\partial \omega}{\partial y} = \Delta w
&
\text{    et    }
&
\displaystyle
\frac{\partial \omega}{\partial y} = -\Delta v
\\
\end{array}
\end{equation}

Dans le cas g�n�ral, ce syst�me peut �tre int�gr� � l'aide de la formule de Biot et Savart.

Dans le cas d'une distribution de vorticit� de type gaussienne modifi�e, les
composantes de vitesse v�rifient : 
\begin{equation}\notag
\left\{
\begin{array}{c}
\displaystyle
v_p(y,x) = - \frac{1}{2\pi} \frac{(z-z_p)}{r^2}\left(1 -
e^{-\frac{r^2}{2\sigma^2}}\right)\,e^{-\frac{r^2}{2\sigma^2}} 
\\
\displaystyle
w_p(y,z) = \frac{1}{2\pi} \frac{(y-y_p)}{r^2}\left(1 -e^{-\frac{r^2}{2\sigma^2}}
\right)\,e^{-\frac{r^2}{2\sigma^2}} 
\end{array}
\right.
\end{equation}

Ces relations s'�tendent de fa�on imm�diate au cas de $N$ vortex distincts.
Le champ de vitesse induit par la distribution de vorticit� 
\begin{equation}
\omega(y,z) = \sum_{p=1}^N \Gamma_p \, \xi_{\sigma_p}(r)
\end{equation}
vaut au point $M$ :
\begin{equation}\notag
\begin{array}{lcr}
v(x,y) = \sum_{p=1}^N \Gamma_p\, v_p(y,z) 
&
\text{    et    }
&
w(y,z) = \sum_{p=1}^N \Gamma_p\, w_p(y,z)
\\
\label{Base_Vortex_compvit}
\end{array}
\end{equation}
%================================
\subsection{Param�tres physiques}
%================================

%-------------------------------
\subsubsection{Marche en temps}
%-------------------------------
La position initiale de chaque vortex est tir�e de mani�re al�atoire. On calcul
les d�placements successifs de chacun des vortex dans le plan d'entr�e par
int�gration explicite du champ de vitesse lagrangien : 
\begin{equation}\notag
\begin{array}{lcr}
\displaystyle
\frac{dy_p}{dt} = V(y,z)
&
\text{    et    }
&
\displaystyle
\frac{dz_p}{dt} = W(y,z)
\\
\end{array}
\end{equation}
Les vortex sont alors assimil�s � des particules ponctuelles qui sont convect�es
par le champ $(V,W)$. Ce champ peut �tre impos� par des tirages al�atoires ou
bien d�duit de la vitesse induite par les vortex dans le plan d'entr�e. Dans ce
cas $V(x,y) = \overline{V}(y,z) + v (y,z)$ et $W(y,z)= \overline{W}(y,z) +
w(y,z)$ o� $\overline{V}$ et $\overline{W}$ sont les composantes de la vitesse
transverse moyenne qu'impose l'utilisateur � l'aide des fichiers de donn�es. 

%---------------------------------------------------
\subsubsection{Intensit� et dur�e de vie des vortex}
%---------------------------------------------------
Il serait possible, � partir de l'�quation de transport de la vorticit�,
d'obtenir un mod�le d'�volution pour l'intensit� du vecteur tourbillon
$\omega_p$ associ� � chacun des vortex. En pratique, on pr�f�re utiliser un
mod�le simplifi� dans lequel la circulation des tourbillons ne d�pend que de la
postion de ces derniers � l'instant consid�r�. La circulation initiale de chaque
vortex est alors obtenue � partir de la relation : 
\begin{equation}\notag
|\Gamma_p| = 4 \sqrt{\frac{\pi\,S\,k}{3N\,[2ln(3)-3ln(2)]}}
\end{equation}
o� $S$ est la surface du plan d'entr�e, $N$ le nombre de vortex, et $k$
l'�nergie cin�tique turbulente au point o� se trouve le vortex � l'instant
consid�r�. Le signe de $\Gamma_p$ correspond au sens de rotation du vortex et
est tir� al�atoirement. 

Ce param�tre est celui qui contr�le l'intensit� des fluctuations. Sa d�pendance
en $k$ exprime que, plus l'�coulement est turbulent, plus les vortex sont
intenses. La valeur de $k$ est sp�cifi�e par
l'utilisateur. Elle peut �tre constante ou impos�e � partir de profils d'�nergie
cin�tique turbulente en entr�e. 

Pour �viter que des structures trop allong�es ne se d�veloppent au niveau de
l'entr�e, l'utilisateur doit �galement sp�cifier un temps limites $\tau_p$ au
bout duquel le vortex $p$ va �tre d�truit. Ce temps $\tau_p$ peut �tre pris
constant ou estim� au moyen de la relation : 
\begin{equation}\notag
\tau_p = \frac{5 C_{\mu}k^{\frac{3}{2}}}{\varepsilon\,\overline{U}}
\end{equation}

$\overline{U}$ et $\varepsilon$ repr�sentent respectivement la vitesse moyenne
principale et la dissipation turbulente au point o� est initialement g�n�r� le
vortex ($C_{\mu}=0,09$). 
\\
Lorsque le temps �coul� depuis la cr�ation du vortex $p$ est sup�rieur �
$\tau_p$, le vortex est d�truit et un nouveau vortex g�n�r� (sa position et le
signe de $\Gamma_p$ sont tir�s de fa�on al�atoire). 

%-------------------------------- 
\subsubsection{Taille des vortex}
%--------------------------------
La taille des vortex peut �tre prise constante, ou calcul�e � partir des
relations :
\begin{equation}\notag
\begin{array}{ccc}
\displaystyle
\sigma = \frac{C_{\mu}^{\frac{3}{4}}k^{\frac{3}{2}}}{\varepsilon} 
& \text{    ou    } &
\sigma = max[L_t,L_k]
\\
\end{array}
\end{equation}
avec:
\begin{equation}\notag
\begin{array}{ccc}
\displaystyle
L_t = \sqrt{\left( \frac{5 \nu k}{\varepsilon} \right)} 
& \text{    et    } & 
\displaystyle
L_k = 200\, \left(\frac{\nu^3}{\varepsilon}\right)^{\frac{1}{4}}
\end{array}
\end{equation}
o� $\nu$, $k$ et $\varepsilon$ sont la viscosit� dynamique, l'�nergie cin�tique
turbulente et la dissipation turbulente au point o� se trouve le vortex. 

Dans tous les cas, la taille du vortex doit �tre sup�rieure � la taille des
mailles en entr�e afin que le vortex soit effectivement simul�. 

%==================================
\subsection{Conditions aux limites}
%==================================
Le champ de vitesse g�n�r� � l'aide de la relation \ref{Base_Vortex_compvit} ne tient pas
compte {\em a priori} des conditions aux limites appliqu�es sur les bords du plan
d'entr�e. Pour obtenir des valeurs de la vitesse qui soient coh�rentes sur les
fronti�res du domaine d'entr�e, des ``vortex images'', pseudo-vortex situ�s en
dehors du domaine d'entr�e, sont g�n�r�s � des positions particuli�res et leur
champ de vitesse associ� est superpos� au champ pr�c�demment calcul�.\\
Seuls les cas d'une conduite rectangulaire et d'une conduite circulaire
permettent la g�n�ration de ces pseudo-vortex.
On distingue pour cela trois types de conditions aux limites. 

\begin{figure}[h]
\centerline{\includegraphics[height=6cm]{../Base/Vortex/Images/condlimite.pdf}}
\caption{\label{Base_Vortex_condli} Principe de g�n�ration des ``vortex images'' suivant le
type de conditions aux limites dans une conduite carr�e.} 
\end{figure}

%----------------------------------
\subsubsection{Condition de paroi}
%----------------------------------
On cr�e, pour chaque vortex $P$ contenu dans le plan d'entr�e, un vortex image
$P'$ identique � $P$ (\textit{i.e} de m�me caract�ristiques) et sym�trique de $P$
par rapport au
point $J$ ($J$ �tant la projection orthogonalement � la paroi du point $M$
correspondant au centre de la face o� l'on cherche � calculer la vitesse). La
figure \ref{Base_Vortex_condli} illustre la technique dans le cas d'une conduite
carr�e. Dans ce cas les coordonn�es du vortex situ� en $P'$ v�rifient
$(y_{p'}+y_{p})/2 = y_{J}$ et $(z_{p'}+ z_{p})/2 = z_{J}$. Le champ de vitesse
per�u depuis le point $M$ au niveau du point $J$ est nul, ce qui est bien
l'effet recherch�. 

%------------------------------------
\subsubsection{Condition de sym�trie}
%-------------------------------------
La technique est identique � celle utilis�e pour les conditions de paroi, mais
seule la composante pour la vitesse normale au bord est modifi�e dans ce cas. 

%---------------------------------------
\subsubsection{Condition de p�riodicit�}
%---------------------------------------
On cr�e pour chaque vortex, un vortex images $P'$ identique � $P$ mais translat�
d'une quantit� $L$ correspondant � la longueur qui s�pare les deux plans de la
section d'entr�e o� sont appliqu�es les conditions de p�riodicit�. Dans le cas
o� il y a deux directions de p�riodicit�, on cr�e deux vortex image.

%=============================================
\subsection{Composante de vitesse principale}
%=============================================
La m�thode des vortex ne g�n�rant pas de fluctuation $u$ de la vitesse dans la
direction principale, la derni�re composante est calcul�e � partir d'une
�quation de Langevin. Les coefficients de cette �quation sont d�termin�s par
identification des expressions obtenues pour les contraintes de Reynolds en
$R_{ij}-\varepsilon$. Dans le cas d'un �coulement en canal plan, cette technique
conduit � l'�quation : 
\begin{equation}\notag
\displaystyle
\frac{du}{dt} = - \frac{C_1}{2T} u + \left(\frac{2}{3}C_2-1\right)\frac{\partial
U}{\partial y} v + \sqrt{C_0\varepsilon} dW_i 
\end{equation}

avec $\displaystyle T=\frac{k}{\varepsilon}$, $C_1 = 1,8$, $C_2 = 0,6$,
$C_0=\frac{14}{15}$, et $dW_i$ une variable al�toire Gaussienne de variance
$\sqrt{dt}$. 

En pratique, l'�quation de Langevin n'am�liore pas vraiment les r�sultats. Elle
n'est utilis�e que dans le cas de conduites circulaires. 

%                      Code_Saturne version 1.3
%                      ------------------------
%
%     This file is part of the Code_Saturne Kernel, element of the
%     Code_Saturne CFD tool.
%
%     Copyright (C) 1998-2007 EDF S.A., France
%
%     contact: saturne-support@edf.fr
%
%     The Code_Saturne Kernel is free software; you can redistribute it
%     and/or modify it under the terms of the GNU General Public License
%     as published by the Free Software Foundation; either version 2 of
%     the License, or (at your option) any later version.
%
%     The Code_Saturne Kernel is distributed in the hope that it will be
%     useful, but WITHOUT ANY WARRANTY; without even the implied warranty
%     of MERCHANTABILITY or FITNESS FOR A PARTICULAR PURPOSE.  See the
%     GNU General Public License for more details.
%
%     You should have received a copy of the GNU General Public License
%     along with the Code_Saturne Kernel; if not, write to the
%     Free Software Foundation, Inc.,
%     51 Franklin St, Fifth Floor,
%     Boston, MA  02110-1301  USA
%
%-----------------------------------------------------------------------
%

%%%%%%%%%%%%%%%%%%%%%%%%%%%%%%%%%%
%%%%%%%%%%%%%%%%%%%%%%%%%%%%%%%%%%
\section{Mise en \oe uvre}
%%%%%%%%%%%%%%%%%%%%%%%%%%%%%%%%%%
%%%%%%%%%%%%%%%%%%%%%%%%%%%%%%%%%%
Le syst\`eme (\ref{Cfbl_Cfmsvl_eq_densite_finale_cfmsvl}) est r\'esolu par une m\'ethode
d'incr\'ement et r\'esidu en utilisant
une m\'ethode de Jacobi pour inverser le syst\`eme si le terme convectif
est implicite et en utilisant une m\'ethode de gradient conjugu\'e
si le terme convectif est explicite (qui est le cas par d�faut).

Attention, les valeurs du flux de masse $\rho\,\vect{w}\cdot\vect{S}$ et
de la viscosit\'e $\Delta\,t\,c^2\frac{S}{d}$ aux faces de
bord, qui sont calcul\'ees dans \fort{cfmsfl} et \fort{cfmsvs} respectivement,
sont modifi\'ees imm\'ediatement apr\`es l'appel \`a ces sous-programmes.
En effet, il est indispensable que la contribution de bord de
$\left(\rho\,\vect{w}-\Delta\,t\,(c^2)\,\gradv\,\rho\right)\cdot\vect{S}$
repr\'esente exactement $\vect{Q}_{ac}\cdot\vect{S}$.
Pour cela,
\begin{itemize}
\item imm\'ediatement apr\`es l'appel \`a
\fort{cfmsfl}, on remplace la contribution de bord de
$\rho\,\vect{w}\cdot\vect{S}$
par le flux de masse exact, $\vect{Q}_{ac}\cdot\vect{S}$,
d\'etermin\'e \`a partir des conditions aux limites,
\item puis, imm\'ediatement apr\`es l'appel \`a
\fort{cfmsvs}, on annule la viscosit\'e au bord $\Delta\,t\,(c^2)$ pour
\'eliminer la contribution de $-\Delta\,t\,(c^2)\,(\gradv\,\rho)\cdot\vect{S}$
(l'annulation de la viscosit\'e n'est pas probl\'ematique pour la matrice,
puisqu'elle porte sur des incr\'ements).
\end{itemize}

\bigskip

Une fois qu'on a obtenu $\rho^{n+1}$,
on peut actualiser le flux de masse acoustique
aux faces $(\vect{Q}_{ac}^{n+1})_{ij} \cdot \vect{S}_{ij}$,
qui servira pour la convection des autres variables~:
\begin{equation}\label{Cfbl_Cfmsvl_eq_flux_masse_acoustique_cfmsvl}
\displaystyle(\vect{Q}_{ac}^{n+1})_{ij}\cdot\vect{S}_{ij}=
-\left(\Delta t^n (c^2)^n \gradv(\rho^{n+1})\right)_{ij}\cdot\vect{S}_{ij}
+\left(\rho^{n+\frac{1}{2}} \vect{w}^n\right)_{ij}\cdot\vect{S}_{ij}\\
\end{equation}
Ce calcul de flux est r\'ealis\'e par \fort{cfbsc3}.
Si l'on a choisi l'algorithme standard, \'equation~(\ref{Cfbl_Cfmsvl_eq_densite_cfmsvl}),
on compl\`ete le flux dans \fort{cfmsvl} imm\'ediatement apr\`es l'appel
\`a \fort{cfbsc3}.
En effet, dans ce cas,
la convection est explicite ($\rho^{n+\frac{1}{2}}=\rho^{n}$,
obtenu en imposant \var{ICONV(ISCA(IRHO(IPHAS)))=0})
et le sous-programme \fort{cfbsc3},
qui calcule le flux de masse aux faces,
ne prend pas en compte la contribution du terme
$\rho^{n+\frac{1}{2}}\,\vect{w}^n\cdot\vect{S}$. On ajoute donc cette
contribution dans \fort{cfmsvl}, apr\`es l'appel \`a \fort{cfbsc3}.
Au bord, en particulier, c'est bien le flux de masse calcul\'e \`a partir
des conditions aux limites que l'on obtient.

On actualise la pression \`a la fin de l'\'etape, en utilisant la loi d'\'etat~:
\begin{equation}
\displaystyle P_i^{pred}=P(\rho_i^{n+1},\varepsilon_i^{n})
\end{equation}


%%%%%%%%%%%%%%%%%%%%%%%%%%%%%%%%%%
%%%%%%%%%%%%%%%%%%%%%%%%%%%%%%%%%%
\section{Points \`a traiter}
%%%%%%%%%%%%%%%%%%%%%%%%%%%%%%%%%%
%%%%%%%%%%%%%%%%%%%%%%%%%%%%%%%%%%
Le calcul du flux de masse au  bord n'est pas enti\`erement satisfaisant
si la convection est trait\'ee de mani\`ere implicite
(algorithme non standard, non test\'e,
associ\'e \`a l'\'equation~(\ref{Cfbl_Cfmsvl_eq_densite_bis_cfmsvl}),
correspondant au choix $\rho^{n+\frac{1}{2}}=\rho^{n+1}$ et
obtenu en imposant \var{ICONV(ISCA(IRHO(IPHAS)))=1}).
En effet, apr\`es \fort{cfmsfl}, il faut d\'eterminer la vitesse de
convection $\vect{w}^n$ pour qu'apparaisse
$\rho^{n+1} \vect{w}^n\cdot\vect{n}$
au cours de la r\'esolution par \fort{codits}. De ce fait, on doit d\'eduire
une valeur de $\vect{w}^n$ \`a partir de la valeur
du flux de masse. Au bord, en particulier, il faut
donc diviser le flux de masse
issu des conditions aux limites par la valeur de bord de $\rho^{n+1}$.
Or, lorsque des conditions de Neumann sont appliqu\'ees \`a la
masse volumique,
la valeur de $\rho^{n+1}$ au bord n'est pas connue avant la r\'esolution du
syst\`eme. On utilise donc, au lieu de la valeur de bord inconnue de
$\rho^{n+1}$ la valeur de bord prise au pas de temps
pr\'ec\'edent $\rho^{n}$. Cette approximation est susceptible
d'affecter la valeur du flux de masse au bord.

%                      Code_Saturne version 1.3
%                      ------------------------
%
%     This file is part of the Code_Saturne Kernel, element of the
%     Code_Saturne CFD tool.
%
%     Copyright (C) 1998-2007 EDF S.A., France
%
%     contact: saturne-support@edf.fr
%
%     The Code_Saturne Kernel is free software; you can redistribute it
%     and/or modify it under the terms of the GNU General Public License
%     as published by the Free Software Foundation; either version 2 of
%     the License, or (at your option) any later version.
%
%     The Code_Saturne Kernel is distributed in the hope that it will be
%     useful, but WITHOUT ANY WARRANTY; without even the implied warranty
%     of MERCHANTABILITY or FITNESS FOR A PARTICULAR PURPOSE.  See the
%     GNU General Public License for more details.
%
%     You should have received a copy of the GNU General Public License
%     along with the Code_Saturne Kernel; if not, write to the
%     Free Software Foundation, Inc.,
%     51 Franklin St, Fifth Floor,
%     Boston, MA  02110-1301  USA
%
%-----------------------------------------------------------------------
%


\programme{navsto}

\vspace{1cm}
On s'int\'eresse \`a la r\'esolution du syst\`eme d'\'equations de Navier-Stokes
tridimensionnelles monophasiques, \`a une pression, instationnaires, en
incompressible ou faiblement dilatable, bas\'ees sur une discr\'etisation
temporelle de type Euler implicite d'ordre 1 ou Crank-Nicolson d'ordre 2 et sur
une discr\'etisation spatiale  par volumes finis colocalis\'ee.


%%%%%%%%%%%%%%%%%%%%%%%%%%%%%%%%%%
%%%%%%%%%%%%%%%%%%%%%%%%%%%%%%%%%%
\section{Fonction}
%%%%%%%%%%%%%%%%%%%%%%%%%%%%%%%%%%
%%%%%%%%%%%%%%%%%%%%%%%%%%%%%%%%%%

  Dans ce sous-programme sont calcul\'ees, \`a un pas de temps donn\'e, les
variables vitesse et pression de ce probl\`eme en proc\'edant en
deux  \'etapes issues d'une d\'ecomposition des op\'erateurs (m\'ethode \`a
pas fractionnaires).\\
Les variables sont donc suppos\'ees connues \`a
l'instant ${t^n}$ et on cherche \`a les d\'eterminer \`a l'instant\footnote{La pression est suppos�e connue � l'instant $t^{n-1+\theta}$ et recherch�e en $t^{n+\theta}$, avec $\theta=1$ ou $1/2$ suivant le sch�ma en temps consid�r�.} ${t^{n+1}}$. Soit ${\Delta {t^n} ={t^{n+1}- {t^n}}}$ le pas de temps associ\'e. Dans un premier temps, on r\'ealise l'\'etape de
pr\'ediction de la vitesse en r\'esolvant l'\'equation de quantit\'e de
mouvement avec une pression explicite. Suit l'\'etape de correction de la
pression (ou projection de la vitesse) qui permet d'obtenir un champ de vitesse \`a divergence nulle.\\\\
Les \'equations en continu sont donc :
\begin{equation}
\left\{\begin{array}{l}
\displaystyle\frac{\partial}{\partial t}(\rho \vect{u}) + \dive(\rho\, \vect{u} \otimes \vect{u})
=\dive(\tens{\sigma}) + \vect{TS} - \tens{K}\,\vect{u}\\
\dive(\rho \vect{u}) = \Gamma
\end{array}\right.
\end{equation}

%(plus tard $\frac{\partial \rho}{\partial t} + \dive(\rho \vect{u}) = \Gamma$)



avec $\rho$ la masse volumique, $\vect{u}$ le champ de vitesse,
$[\,\vect{TS}-\tens{K}\,\vect{u}\,]$ les autres termes sources ($\tens{K}$~est un
tenseur diagonal positif par d\'efinition), $\tens{\sigma}$ le tenseur
de contraintes, $\tens{\tau}$ le tenseur des contraintes visqueuses, $\mu$ la
viscosit\'e dynamique (mol\'eculaire et \'eventuellement turbulente), $\kappa$
la viscosit� de
volume (usuellement nulle et n�glig�e dans le code et dans la suite du document,
sauf en compressible),
$\tens{D}$ le tenseur taux de d\'eformation\footnote{\`A ne pas confondre, malgr\'e la m\^eme notation $D$,
avec les flux diffusifs $\vect{D}_{\,ij}$ et $\vect{D}_{\,{b}_{ik}}$ d\'ecrits par la suite dans ce
sous-programme.}, $\Gamma$ le terme source de masse.
\begin{equation}
\left\{\begin{array}{l}
\tens{\sigma} = \tens{\tau} - P\tens{Id}  \\
\tens{\tau} = 2\,\mu\ \tens{D} +\ (\kappa\ - \frac{2}{3}\mu)\  tr({\tens{D}})\
\tens{Id}  \\
\tens{D} = \frac{1}{2}(\ggrad\vect{u}+\,^{t}\ggrad\vect{u})
\end{array}\right.
\end{equation}
 \\

On rappelle la d\'efinition des notations employ\'ees\footnote{en
utilisant la convention de sommation d'Einstein.}~:
\begin{equation}\notag
\left\{\begin{array}{lll}
\left[\ggrad{\vect{a}}\right]_{ij} &=& \partial_j a_i\\
\left[\dive(\tens{\sigma})\right]_i &=& \partial_j \sigma_{ij}\\
\left[\vect{a}\otimes\vect{b}\right]_{ij} &= &
a_i\,b_j\\
\end{array}\right.
\end{equation}
et donc :
\begin{equation}\notag
\begin{array}{lll}
\left[\dive(\vect{a}\otimes\vect{b})\right]_i &= &
\partial_j (a_i\,b_j)
\end{array}
\end{equation}

\minititre{Remarque}
Dans le cas de la prise en compte d'une masse volumique variable, l'�quation de continuit� s'�crit :
$$\frac{\partial \rho}{\partial t} + \dive{\,(\rho\,\vect{u})} = \Gamma  $$
Cette �quation n'est pas prise en compte dans l'�tape de projection (on continue � r�soudre
seulement
$\displaystyle \dive(\,{\rho\,\vect{u}}) = \Gamma$) alors que le terme
$\displaystyle \frac{\partial \rho}{\partial t}$ appara\^{\i}t lors de l'�tape de pr\'ediction de la vitesse
dans le sous-programme \fort{preduv}. Si ce terme joue un r�le sensible, l'algorithme compressible
de \CS\ (qui r�sout l'�quation compl�te) est alors sans doute plus adapt�.

%                      Code_Saturne version 1.3
%                      ------------------------
%
%     This file is part of the Code_Saturne Kernel, element of the
%     Code_Saturne CFD tool.
% 
%     Copyright (C) 1998-2007 EDF S.A., France
%
%     contact: saturne-support@edf.fr
% 
%     The Code_Saturne Kernel is free software; you can redistribute it
%     and/or modify it under the terms of the GNU General Public License
%     as published by the Free Software Foundation; either version 2 of
%     the License, or (at your option) any later version.
% 
%     The Code_Saturne Kernel is distributed in the hope that it will be
%     useful, but WITHOUT ANY WARRANTY; without even the implied warranty
%     of MERCHANTABILITY or FITNESS FOR A PARTICULAR PURPOSE.  See the
%     GNU General Public License for more details.
% 
%     You should have received a copy of the GNU General Public License
%     along with the Code_Saturne Kernel; if not, write to the
%     Free Software Foundation, Inc.,
%     51 Franklin St, Fifth Floor,
%     Boston, MA  02110-1301  USA
%
%-----------------------------------------------------------------------
%
%%%%%%%%%%%%%%%%%%%%%%%%%%%%%%%%%
%%%%%%%%%%%%%%%%%%%%%%%%%%%%%%%%%%
\section{Discr\'etisation}
%%%%%%%%%%%%%%%%%%%%%%%%%%%%%%%%%%
%%%%%%%%%%%%%%%%%%%%%%%%%%%%%%%%%%

Pour utiliser la m�thode, on se place tout d'abord dans un rep�re local d�fini
de mani�re � ce que le plan $(0yz)$, o� sont inject�s les vortex, soit confondu
avec le plan d'entr�e du calcul (voir figure \ref{Base_Vortex_entree}). 

\begin{figure}[h]
\centerline{\includegraphics[height=6cm]{../Base/Vortex/Images/entree.pdf}}
\caption{\label{Base_Vortex_entree} D�finiton des diff�rentes grandeurs dans le rep�re local
correspondant � l'entr�e d'une conduite de section carr�e.} 
\end{figure}

$u$, $v$ et $w$  sont les composantes de la vitesse fluctuante (principale et
transverse) dans ce plan, et
$\displaystyle \omega(y,z) = \frac{\partial w}{\partial y}-\frac{\partial v}{\partial z}$
la vorticit� dans la direction
normale au plan d'entr�e. $\overline{U}(y,z)$ repr�sente ici la vitesse
principale moyenne impos�e par l'utilisateur dans le plan d'entr�e. 

Chaque vortex $p$ va �tre caract�ris� par sa fonction de forme $\xi$ (identique
pour tous les vortex), sa
circulation $\Gamma_p$, son rayon $\sigma_p$ et les coordonn�es $(y_p,z_p)$ du
point $P$ o� est situ� le vortex dans le plan $(0yz)$. 

Pour cela, on suppose que la vorticit� g�n�r�e par un vortex $p$ au point $M$ de
coordonn�e $(y,z)$ s'�crit : 
\begin{equation}\notag
\omega_p(y,z)= \Gamma_p \, \xi_{\sigma_p}(r)
\end{equation}
o� $r = \sqrt{(y-y_p)^2+(z-z_p)^2}$ est la distance s�parant le point $M$ du point $P$.

Dans la m�thode implant�e, la fonction de forme est de type gaussienne modifi�e :
\begin{equation}\notag
\displaystyle
\xi_\sigma (r) = \frac{1}{2\pi \sigma^2} 
\left(2 e^{-\frac{r^2}{2\sigma^2}}-1\right) e^{-\frac{r^2}{2\sigma^2}}
\end{equation}

Le champ de vitesse induit par cette distribution de vorticit� s'obtient par
inversion des deux �quations de poisson suivantes qui sont d�duites de la
condition d'incompressibilit� dans la plan\footnote{\textit{i.e}
$\displaystyle \frac{\partial v}{\partial y}+\frac{\partial w}{\partial w} = 0$} :
\begin{equation}\notag
\begin{array}{lcr}
\displaystyle
\frac{\partial \omega}{\partial y} = \Delta w
&
\text{    et    }
&
\displaystyle
\frac{\partial \omega}{\partial y} = -\Delta v
\\
\end{array}
\end{equation}

Dans le cas g�n�ral, ce syst�me peut �tre int�gr� � l'aide de la formule de Biot et Savart.

Dans le cas d'une distribution de vorticit� de type gaussienne modifi�e, les
composantes de vitesse v�rifient : 
\begin{equation}\notag
\left\{
\begin{array}{c}
\displaystyle
v_p(y,x) = - \frac{1}{2\pi} \frac{(z-z_p)}{r^2}\left(1 -
e^{-\frac{r^2}{2\sigma^2}}\right)\,e^{-\frac{r^2}{2\sigma^2}} 
\\
\displaystyle
w_p(y,z) = \frac{1}{2\pi} \frac{(y-y_p)}{r^2}\left(1 -e^{-\frac{r^2}{2\sigma^2}}
\right)\,e^{-\frac{r^2}{2\sigma^2}} 
\end{array}
\right.
\end{equation}

Ces relations s'�tendent de fa�on imm�diate au cas de $N$ vortex distincts.
Le champ de vitesse induit par la distribution de vorticit� 
\begin{equation}
\omega(y,z) = \sum_{p=1}^N \Gamma_p \, \xi_{\sigma_p}(r)
\end{equation}
vaut au point $M$ :
\begin{equation}\notag
\begin{array}{lcr}
v(x,y) = \sum_{p=1}^N \Gamma_p\, v_p(y,z) 
&
\text{    et    }
&
w(y,z) = \sum_{p=1}^N \Gamma_p\, w_p(y,z)
\\
\label{Base_Vortex_compvit}
\end{array}
\end{equation}
%================================
\subsection{Param�tres physiques}
%================================

%-------------------------------
\subsubsection{Marche en temps}
%-------------------------------
La position initiale de chaque vortex est tir�e de mani�re al�atoire. On calcul
les d�placements successifs de chacun des vortex dans le plan d'entr�e par
int�gration explicite du champ de vitesse lagrangien : 
\begin{equation}\notag
\begin{array}{lcr}
\displaystyle
\frac{dy_p}{dt} = V(y,z)
&
\text{    et    }
&
\displaystyle
\frac{dz_p}{dt} = W(y,z)
\\
\end{array}
\end{equation}
Les vortex sont alors assimil�s � des particules ponctuelles qui sont convect�es
par le champ $(V,W)$. Ce champ peut �tre impos� par des tirages al�atoires ou
bien d�duit de la vitesse induite par les vortex dans le plan d'entr�e. Dans ce
cas $V(x,y) = \overline{V}(y,z) + v (y,z)$ et $W(y,z)= \overline{W}(y,z) +
w(y,z)$ o� $\overline{V}$ et $\overline{W}$ sont les composantes de la vitesse
transverse moyenne qu'impose l'utilisateur � l'aide des fichiers de donn�es. 

%---------------------------------------------------
\subsubsection{Intensit� et dur�e de vie des vortex}
%---------------------------------------------------
Il serait possible, � partir de l'�quation de transport de la vorticit�,
d'obtenir un mod�le d'�volution pour l'intensit� du vecteur tourbillon
$\omega_p$ associ� � chacun des vortex. En pratique, on pr�f�re utiliser un
mod�le simplifi� dans lequel la circulation des tourbillons ne d�pend que de la
postion de ces derniers � l'instant consid�r�. La circulation initiale de chaque
vortex est alors obtenue � partir de la relation : 
\begin{equation}\notag
|\Gamma_p| = 4 \sqrt{\frac{\pi\,S\,k}{3N\,[2ln(3)-3ln(2)]}}
\end{equation}
o� $S$ est la surface du plan d'entr�e, $N$ le nombre de vortex, et $k$
l'�nergie cin�tique turbulente au point o� se trouve le vortex � l'instant
consid�r�. Le signe de $\Gamma_p$ correspond au sens de rotation du vortex et
est tir� al�atoirement. 

Ce param�tre est celui qui contr�le l'intensit� des fluctuations. Sa d�pendance
en $k$ exprime que, plus l'�coulement est turbulent, plus les vortex sont
intenses. La valeur de $k$ est sp�cifi�e par
l'utilisateur. Elle peut �tre constante ou impos�e � partir de profils d'�nergie
cin�tique turbulente en entr�e. 

Pour �viter que des structures trop allong�es ne se d�veloppent au niveau de
l'entr�e, l'utilisateur doit �galement sp�cifier un temps limites $\tau_p$ au
bout duquel le vortex $p$ va �tre d�truit. Ce temps $\tau_p$ peut �tre pris
constant ou estim� au moyen de la relation : 
\begin{equation}\notag
\tau_p = \frac{5 C_{\mu}k^{\frac{3}{2}}}{\varepsilon\,\overline{U}}
\end{equation}

$\overline{U}$ et $\varepsilon$ repr�sentent respectivement la vitesse moyenne
principale et la dissipation turbulente au point o� est initialement g�n�r� le
vortex ($C_{\mu}=0,09$). 
\\
Lorsque le temps �coul� depuis la cr�ation du vortex $p$ est sup�rieur �
$\tau_p$, le vortex est d�truit et un nouveau vortex g�n�r� (sa position et le
signe de $\Gamma_p$ sont tir�s de fa�on al�atoire). 

%-------------------------------- 
\subsubsection{Taille des vortex}
%--------------------------------
La taille des vortex peut �tre prise constante, ou calcul�e � partir des
relations :
\begin{equation}\notag
\begin{array}{ccc}
\displaystyle
\sigma = \frac{C_{\mu}^{\frac{3}{4}}k^{\frac{3}{2}}}{\varepsilon} 
& \text{    ou    } &
\sigma = max[L_t,L_k]
\\
\end{array}
\end{equation}
avec:
\begin{equation}\notag
\begin{array}{ccc}
\displaystyle
L_t = \sqrt{\left( \frac{5 \nu k}{\varepsilon} \right)} 
& \text{    et    } & 
\displaystyle
L_k = 200\, \left(\frac{\nu^3}{\varepsilon}\right)^{\frac{1}{4}}
\end{array}
\end{equation}
o� $\nu$, $k$ et $\varepsilon$ sont la viscosit� dynamique, l'�nergie cin�tique
turbulente et la dissipation turbulente au point o� se trouve le vortex. 

Dans tous les cas, la taille du vortex doit �tre sup�rieure � la taille des
mailles en entr�e afin que le vortex soit effectivement simul�. 

%==================================
\subsection{Conditions aux limites}
%==================================
Le champ de vitesse g�n�r� � l'aide de la relation \ref{Base_Vortex_compvit} ne tient pas
compte {\em a priori} des conditions aux limites appliqu�es sur les bords du plan
d'entr�e. Pour obtenir des valeurs de la vitesse qui soient coh�rentes sur les
fronti�res du domaine d'entr�e, des ``vortex images'', pseudo-vortex situ�s en
dehors du domaine d'entr�e, sont g�n�r�s � des positions particuli�res et leur
champ de vitesse associ� est superpos� au champ pr�c�demment calcul�.\\
Seuls les cas d'une conduite rectangulaire et d'une conduite circulaire
permettent la g�n�ration de ces pseudo-vortex.
On distingue pour cela trois types de conditions aux limites. 

\begin{figure}[h]
\centerline{\includegraphics[height=6cm]{../Base/Vortex/Images/condlimite.pdf}}
\caption{\label{Base_Vortex_condli} Principe de g�n�ration des ``vortex images'' suivant le
type de conditions aux limites dans une conduite carr�e.} 
\end{figure}

%----------------------------------
\subsubsection{Condition de paroi}
%----------------------------------
On cr�e, pour chaque vortex $P$ contenu dans le plan d'entr�e, un vortex image
$P'$ identique � $P$ (\textit{i.e} de m�me caract�ristiques) et sym�trique de $P$
par rapport au
point $J$ ($J$ �tant la projection orthogonalement � la paroi du point $M$
correspondant au centre de la face o� l'on cherche � calculer la vitesse). La
figure \ref{Base_Vortex_condli} illustre la technique dans le cas d'une conduite
carr�e. Dans ce cas les coordonn�es du vortex situ� en $P'$ v�rifient
$(y_{p'}+y_{p})/2 = y_{J}$ et $(z_{p'}+ z_{p})/2 = z_{J}$. Le champ de vitesse
per�u depuis le point $M$ au niveau du point $J$ est nul, ce qui est bien
l'effet recherch�. 

%------------------------------------
\subsubsection{Condition de sym�trie}
%-------------------------------------
La technique est identique � celle utilis�e pour les conditions de paroi, mais
seule la composante pour la vitesse normale au bord est modifi�e dans ce cas. 

%---------------------------------------
\subsubsection{Condition de p�riodicit�}
%---------------------------------------
On cr�e pour chaque vortex, un vortex images $P'$ identique � $P$ mais translat�
d'une quantit� $L$ correspondant � la longueur qui s�pare les deux plans de la
section d'entr�e o� sont appliqu�es les conditions de p�riodicit�. Dans le cas
o� il y a deux directions de p�riodicit�, on cr�e deux vortex image.

%=============================================
\subsection{Composante de vitesse principale}
%=============================================
La m�thode des vortex ne g�n�rant pas de fluctuation $u$ de la vitesse dans la
direction principale, la derni�re composante est calcul�e � partir d'une
�quation de Langevin. Les coefficients de cette �quation sont d�termin�s par
identification des expressions obtenues pour les contraintes de Reynolds en
$R_{ij}-\varepsilon$. Dans le cas d'un �coulement en canal plan, cette technique
conduit � l'�quation : 
\begin{equation}\notag
\displaystyle
\frac{du}{dt} = - \frac{C_1}{2T} u + \left(\frac{2}{3}C_2-1\right)\frac{\partial
U}{\partial y} v + \sqrt{C_0\varepsilon} dW_i 
\end{equation}

avec $\displaystyle T=\frac{k}{\varepsilon}$, $C_1 = 1,8$, $C_2 = 0,6$,
$C_0=\frac{14}{15}$, et $dW_i$ une variable al�toire Gaussienne de variance
$\sqrt{dt}$. 

En pratique, l'�quation de Langevin n'am�liore pas vraiment les r�sultats. Elle
n'est utilis�e que dans le cas de conduites circulaires. 

%                      Code_Saturne version 1.3
%                      ------------------------
%
%     This file is part of the Code_Saturne Kernel, element of the
%     Code_Saturne CFD tool.
%
%     Copyright (C) 1998-2007 EDF S.A., France
%
%     contact: saturne-support@edf.fr
%
%     The Code_Saturne Kernel is free software; you can redistribute it
%     and/or modify it under the terms of the GNU General Public License
%     as published by the Free Software Foundation; either version 2 of
%     the License, or (at your option) any later version.
%
%     The Code_Saturne Kernel is distributed in the hope that it will be
%     useful, but WITHOUT ANY WARRANTY; without even the implied warranty
%     of MERCHANTABILITY or FITNESS FOR A PARTICULAR PURPOSE.  See the
%     GNU General Public License for more details.
%
%     You should have received a copy of the GNU General Public License
%     along with the Code_Saturne Kernel; if not, write to the
%     Free Software Foundation, Inc.,
%     51 Franklin St, Fifth Floor,
%     Boston, MA  02110-1301  USA
%
%-----------------------------------------------------------------------
%

%%%%%%%%%%%%%%%%%%%%%%%%%%%%%%%%%%
%%%%%%%%%%%%%%%%%%%%%%%%%%%%%%%%%%
\section{Mise en \oe uvre}
%%%%%%%%%%%%%%%%%%%%%%%%%%%%%%%%%%
%%%%%%%%%%%%%%%%%%%%%%%%%%%%%%%%%%
Le syst\`eme (\ref{Cfbl_Cfmsvl_eq_densite_finale_cfmsvl}) est r\'esolu par une m\'ethode
d'incr\'ement et r\'esidu en utilisant
une m\'ethode de Jacobi pour inverser le syst\`eme si le terme convectif
est implicite et en utilisant une m\'ethode de gradient conjugu\'e
si le terme convectif est explicite (qui est le cas par d�faut).

Attention, les valeurs du flux de masse $\rho\,\vect{w}\cdot\vect{S}$ et
de la viscosit\'e $\Delta\,t\,c^2\frac{S}{d}$ aux faces de
bord, qui sont calcul\'ees dans \fort{cfmsfl} et \fort{cfmsvs} respectivement,
sont modifi\'ees imm\'ediatement apr\`es l'appel \`a ces sous-programmes.
En effet, il est indispensable que la contribution de bord de
$\left(\rho\,\vect{w}-\Delta\,t\,(c^2)\,\gradv\,\rho\right)\cdot\vect{S}$
repr\'esente exactement $\vect{Q}_{ac}\cdot\vect{S}$.
Pour cela,
\begin{itemize}
\item imm\'ediatement apr\`es l'appel \`a
\fort{cfmsfl}, on remplace la contribution de bord de
$\rho\,\vect{w}\cdot\vect{S}$
par le flux de masse exact, $\vect{Q}_{ac}\cdot\vect{S}$,
d\'etermin\'e \`a partir des conditions aux limites,
\item puis, imm\'ediatement apr\`es l'appel \`a
\fort{cfmsvs}, on annule la viscosit\'e au bord $\Delta\,t\,(c^2)$ pour
\'eliminer la contribution de $-\Delta\,t\,(c^2)\,(\gradv\,\rho)\cdot\vect{S}$
(l'annulation de la viscosit\'e n'est pas probl\'ematique pour la matrice,
puisqu'elle porte sur des incr\'ements).
\end{itemize}

\bigskip

Une fois qu'on a obtenu $\rho^{n+1}$,
on peut actualiser le flux de masse acoustique
aux faces $(\vect{Q}_{ac}^{n+1})_{ij} \cdot \vect{S}_{ij}$,
qui servira pour la convection des autres variables~:
\begin{equation}\label{Cfbl_Cfmsvl_eq_flux_masse_acoustique_cfmsvl}
\displaystyle(\vect{Q}_{ac}^{n+1})_{ij}\cdot\vect{S}_{ij}=
-\left(\Delta t^n (c^2)^n \gradv(\rho^{n+1})\right)_{ij}\cdot\vect{S}_{ij}
+\left(\rho^{n+\frac{1}{2}} \vect{w}^n\right)_{ij}\cdot\vect{S}_{ij}\\
\end{equation}
Ce calcul de flux est r\'ealis\'e par \fort{cfbsc3}.
Si l'on a choisi l'algorithme standard, \'equation~(\ref{Cfbl_Cfmsvl_eq_densite_cfmsvl}),
on compl\`ete le flux dans \fort{cfmsvl} imm\'ediatement apr\`es l'appel
\`a \fort{cfbsc3}.
En effet, dans ce cas,
la convection est explicite ($\rho^{n+\frac{1}{2}}=\rho^{n}$,
obtenu en imposant \var{ICONV(ISCA(IRHO(IPHAS)))=0})
et le sous-programme \fort{cfbsc3},
qui calcule le flux de masse aux faces,
ne prend pas en compte la contribution du terme
$\rho^{n+\frac{1}{2}}\,\vect{w}^n\cdot\vect{S}$. On ajoute donc cette
contribution dans \fort{cfmsvl}, apr\`es l'appel \`a \fort{cfbsc3}.
Au bord, en particulier, c'est bien le flux de masse calcul\'e \`a partir
des conditions aux limites que l'on obtient.

On actualise la pression \`a la fin de l'\'etape, en utilisant la loi d'\'etat~:
\begin{equation}
\displaystyle P_i^{pred}=P(\rho_i^{n+1},\varepsilon_i^{n})
\end{equation}


%%%%%%%%%%%%%%%%%%%%%%%%%%%%%%%%%%
%%%%%%%%%%%%%%%%%%%%%%%%%%%%%%%%%%
\section{Points \`a traiter}
%%%%%%%%%%%%%%%%%%%%%%%%%%%%%%%%%%
%%%%%%%%%%%%%%%%%%%%%%%%%%%%%%%%%%
Le calcul du flux de masse au  bord n'est pas enti\`erement satisfaisant
si la convection est trait\'ee de mani\`ere implicite
(algorithme non standard, non test\'e,
associ\'e \`a l'\'equation~(\ref{Cfbl_Cfmsvl_eq_densite_bis_cfmsvl}),
correspondant au choix $\rho^{n+\frac{1}{2}}=\rho^{n+1}$ et
obtenu en imposant \var{ICONV(ISCA(IRHO(IPHAS)))=1}).
En effet, apr\`es \fort{cfmsfl}, il faut d\'eterminer la vitesse de
convection $\vect{w}^n$ pour qu'apparaisse
$\rho^{n+1} \vect{w}^n\cdot\vect{n}$
au cours de la r\'esolution par \fort{codits}. De ce fait, on doit d\'eduire
une valeur de $\vect{w}^n$ \`a partir de la valeur
du flux de masse. Au bord, en particulier, il faut
donc diviser le flux de masse
issu des conditions aux limites par la valeur de bord de $\rho^{n+1}$.
Or, lorsque des conditions de Neumann sont appliqu\'ees \`a la
masse volumique,
la valeur de $\rho^{n+1}$ au bord n'est pas connue avant la r\'esolution du
syst\`eme. On utilise donc, au lieu de la valeur de bord inconnue de
$\rho^{n+1}$ la valeur de bord prise au pas de temps
pr\'ec\'edent $\rho^{n}$. Cette approximation est susceptible
d'affecter la valeur du flux de masse au bord.

%                      Code_Saturne version 1.3
%                      ------------------------
%
%     This file is part of the Code_Saturne Kernel, element of the
%     Code_Saturne CFD tool.
%
%     Copyright (C) 1998-2007 EDF S.A., France
%
%     contact: saturne-support@edf.fr
%
%     The Code_Saturne Kernel is free software; you can redistribute it
%     and/or modify it under the terms of the GNU General Public License
%     as published by the Free Software Foundation; either version 2 of
%     the License, or (at your option) any later version.
%
%     The Code_Saturne Kernel is distributed in the hope that it will be
%     useful, but WITHOUT ANY WARRANTY; without even the implied warranty
%     of MERCHANTABILITY or FITNESS FOR A PARTICULAR PURPOSE.  See the
%     GNU General Public License for more details.
%
%     You should have received a copy of the GNU General Public License
%     along with the Code_Saturne Kernel; if not, write to the
%     Free Software Foundation, Inc.,
%     51 Franklin St, Fifth Floor,
%     Boston, MA  02110-1301  USA
%
%-----------------------------------------------------------------------
%


\programme{navsto}

\vspace{1cm}
On s'int\'eresse \`a la r\'esolution du syst\`eme d'\'equations de Navier-Stokes
tridimensionnelles monophasiques, \`a une pression, instationnaires, en
incompressible ou faiblement dilatable, bas\'ees sur une discr\'etisation
temporelle de type Euler implicite d'ordre 1 ou Crank-Nicolson d'ordre 2 et sur
une discr\'etisation spatiale  par volumes finis colocalis\'ee.


%%%%%%%%%%%%%%%%%%%%%%%%%%%%%%%%%%
%%%%%%%%%%%%%%%%%%%%%%%%%%%%%%%%%%
\section{Fonction}
%%%%%%%%%%%%%%%%%%%%%%%%%%%%%%%%%%
%%%%%%%%%%%%%%%%%%%%%%%%%%%%%%%%%%

  Dans ce sous-programme sont calcul\'ees, \`a un pas de temps donn\'e, les
variables vitesse et pression de ce probl\`eme en proc\'edant en
deux  \'etapes issues d'une d\'ecomposition des op\'erateurs (m\'ethode \`a
pas fractionnaires).\\
Les variables sont donc suppos\'ees connues \`a
l'instant ${t^n}$ et on cherche \`a les d\'eterminer \`a l'instant\footnote{La pression est suppos�e connue � l'instant $t^{n-1+\theta}$ et recherch�e en $t^{n+\theta}$, avec $\theta=1$ ou $1/2$ suivant le sch�ma en temps consid�r�.} ${t^{n+1}}$. Soit ${\Delta {t^n} ={t^{n+1}- {t^n}}}$ le pas de temps associ\'e. Dans un premier temps, on r\'ealise l'\'etape de
pr\'ediction de la vitesse en r\'esolvant l'\'equation de quantit\'e de
mouvement avec une pression explicite. Suit l'\'etape de correction de la
pression (ou projection de la vitesse) qui permet d'obtenir un champ de vitesse \`a divergence nulle.\\\\
Les \'equations en continu sont donc :
\begin{equation}
\left\{\begin{array}{l}
\displaystyle\frac{\partial}{\partial t}(\rho \vect{u}) + \dive(\rho\, \vect{u} \otimes \vect{u})
=\dive(\tens{\sigma}) + \vect{TS} - \tens{K}\,\vect{u}\\
\dive(\rho \vect{u}) = \Gamma
\end{array}\right.
\end{equation}

%(plus tard $\frac{\partial \rho}{\partial t} + \dive(\rho \vect{u}) = \Gamma$)



avec $\rho$ la masse volumique, $\vect{u}$ le champ de vitesse,
$[\,\vect{TS}-\tens{K}\,\vect{u}\,]$ les autres termes sources ($\tens{K}$~est un
tenseur diagonal positif par d\'efinition), $\tens{\sigma}$ le tenseur
de contraintes, $\tens{\tau}$ le tenseur des contraintes visqueuses, $\mu$ la
viscosit\'e dynamique (mol\'eculaire et \'eventuellement turbulente), $\kappa$
la viscosit� de
volume (usuellement nulle et n�glig�e dans le code et dans la suite du document,
sauf en compressible),
$\tens{D}$ le tenseur taux de d\'eformation\footnote{\`A ne pas confondre, malgr\'e la m\^eme notation $D$,
avec les flux diffusifs $\vect{D}_{\,ij}$ et $\vect{D}_{\,{b}_{ik}}$ d\'ecrits par la suite dans ce
sous-programme.}, $\Gamma$ le terme source de masse.
\begin{equation}
\left\{\begin{array}{l}
\tens{\sigma} = \tens{\tau} - P\tens{Id}  \\
\tens{\tau} = 2\,\mu\ \tens{D} +\ (\kappa\ - \frac{2}{3}\mu)\  tr({\tens{D}})\
\tens{Id}  \\
\tens{D} = \frac{1}{2}(\ggrad\vect{u}+\,^{t}\ggrad\vect{u})
\end{array}\right.
\end{equation}
 \\

On rappelle la d\'efinition des notations employ\'ees\footnote{en
utilisant la convention de sommation d'Einstein.}~:
\begin{equation}\notag
\left\{\begin{array}{lll}
\left[\ggrad{\vect{a}}\right]_{ij} &=& \partial_j a_i\\
\left[\dive(\tens{\sigma})\right]_i &=& \partial_j \sigma_{ij}\\
\left[\vect{a}\otimes\vect{b}\right]_{ij} &= &
a_i\,b_j\\
\end{array}\right.
\end{equation}
et donc :
\begin{equation}\notag
\begin{array}{lll}
\left[\dive(\vect{a}\otimes\vect{b})\right]_i &= &
\partial_j (a_i\,b_j)
\end{array}
\end{equation}

\minititre{Remarque}
Dans le cas de la prise en compte d'une masse volumique variable, l'�quation de continuit� s'�crit :
$$\frac{\partial \rho}{\partial t} + \dive{\,(\rho\,\vect{u})} = \Gamma  $$
Cette �quation n'est pas prise en compte dans l'�tape de projection (on continue � r�soudre
seulement
$\displaystyle \dive(\,{\rho\,\vect{u}}) = \Gamma$) alors que le terme
$\displaystyle \frac{\partial \rho}{\partial t}$ appara\^{\i}t lors de l'�tape de pr\'ediction de la vitesse
dans le sous-programme \fort{preduv}. Si ce terme joue un r�le sensible, l'algorithme compressible
de \CS\ (qui r�sout l'�quation compl�te) est alors sans doute plus adapt�.

%                      Code_Saturne version 1.3
%                      ------------------------
%
%     This file is part of the Code_Saturne Kernel, element of the
%     Code_Saturne CFD tool.
% 
%     Copyright (C) 1998-2007 EDF S.A., France
%
%     contact: saturne-support@edf.fr
% 
%     The Code_Saturne Kernel is free software; you can redistribute it
%     and/or modify it under the terms of the GNU General Public License
%     as published by the Free Software Foundation; either version 2 of
%     the License, or (at your option) any later version.
% 
%     The Code_Saturne Kernel is distributed in the hope that it will be
%     useful, but WITHOUT ANY WARRANTY; without even the implied warranty
%     of MERCHANTABILITY or FITNESS FOR A PARTICULAR PURPOSE.  See the
%     GNU General Public License for more details.
% 
%     You should have received a copy of the GNU General Public License
%     along with the Code_Saturne Kernel; if not, write to the
%     Free Software Foundation, Inc.,
%     51 Franklin St, Fifth Floor,
%     Boston, MA  02110-1301  USA
%
%-----------------------------------------------------------------------
%
%%%%%%%%%%%%%%%%%%%%%%%%%%%%%%%%%
%%%%%%%%%%%%%%%%%%%%%%%%%%%%%%%%%%
\section{Discr\'etisation}
%%%%%%%%%%%%%%%%%%%%%%%%%%%%%%%%%%
%%%%%%%%%%%%%%%%%%%%%%%%%%%%%%%%%%

Pour utiliser la m�thode, on se place tout d'abord dans un rep�re local d�fini
de mani�re � ce que le plan $(0yz)$, o� sont inject�s les vortex, soit confondu
avec le plan d'entr�e du calcul (voir figure \ref{Base_Vortex_entree}). 

\begin{figure}[h]
\centerline{\includegraphics[height=6cm]{../Base/Vortex/Images/entree.pdf}}
\caption{\label{Base_Vortex_entree} D�finiton des diff�rentes grandeurs dans le rep�re local
correspondant � l'entr�e d'une conduite de section carr�e.} 
\end{figure}

$u$, $v$ et $w$  sont les composantes de la vitesse fluctuante (principale et
transverse) dans ce plan, et
$\displaystyle \omega(y,z) = \frac{\partial w}{\partial y}-\frac{\partial v}{\partial z}$
la vorticit� dans la direction
normale au plan d'entr�e. $\overline{U}(y,z)$ repr�sente ici la vitesse
principale moyenne impos�e par l'utilisateur dans le plan d'entr�e. 

Chaque vortex $p$ va �tre caract�ris� par sa fonction de forme $\xi$ (identique
pour tous les vortex), sa
circulation $\Gamma_p$, son rayon $\sigma_p$ et les coordonn�es $(y_p,z_p)$ du
point $P$ o� est situ� le vortex dans le plan $(0yz)$. 

Pour cela, on suppose que la vorticit� g�n�r�e par un vortex $p$ au point $M$ de
coordonn�e $(y,z)$ s'�crit : 
\begin{equation}\notag
\omega_p(y,z)= \Gamma_p \, \xi_{\sigma_p}(r)
\end{equation}
o� $r = \sqrt{(y-y_p)^2+(z-z_p)^2}$ est la distance s�parant le point $M$ du point $P$.

Dans la m�thode implant�e, la fonction de forme est de type gaussienne modifi�e :
\begin{equation}\notag
\displaystyle
\xi_\sigma (r) = \frac{1}{2\pi \sigma^2} 
\left(2 e^{-\frac{r^2}{2\sigma^2}}-1\right) e^{-\frac{r^2}{2\sigma^2}}
\end{equation}

Le champ de vitesse induit par cette distribution de vorticit� s'obtient par
inversion des deux �quations de poisson suivantes qui sont d�duites de la
condition d'incompressibilit� dans la plan\footnote{\textit{i.e}
$\displaystyle \frac{\partial v}{\partial y}+\frac{\partial w}{\partial w} = 0$} :
\begin{equation}\notag
\begin{array}{lcr}
\displaystyle
\frac{\partial \omega}{\partial y} = \Delta w
&
\text{    et    }
&
\displaystyle
\frac{\partial \omega}{\partial y} = -\Delta v
\\
\end{array}
\end{equation}

Dans le cas g�n�ral, ce syst�me peut �tre int�gr� � l'aide de la formule de Biot et Savart.

Dans le cas d'une distribution de vorticit� de type gaussienne modifi�e, les
composantes de vitesse v�rifient : 
\begin{equation}\notag
\left\{
\begin{array}{c}
\displaystyle
v_p(y,x) = - \frac{1}{2\pi} \frac{(z-z_p)}{r^2}\left(1 -
e^{-\frac{r^2}{2\sigma^2}}\right)\,e^{-\frac{r^2}{2\sigma^2}} 
\\
\displaystyle
w_p(y,z) = \frac{1}{2\pi} \frac{(y-y_p)}{r^2}\left(1 -e^{-\frac{r^2}{2\sigma^2}}
\right)\,e^{-\frac{r^2}{2\sigma^2}} 
\end{array}
\right.
\end{equation}

Ces relations s'�tendent de fa�on imm�diate au cas de $N$ vortex distincts.
Le champ de vitesse induit par la distribution de vorticit� 
\begin{equation}
\omega(y,z) = \sum_{p=1}^N \Gamma_p \, \xi_{\sigma_p}(r)
\end{equation}
vaut au point $M$ :
\begin{equation}\notag
\begin{array}{lcr}
v(x,y) = \sum_{p=1}^N \Gamma_p\, v_p(y,z) 
&
\text{    et    }
&
w(y,z) = \sum_{p=1}^N \Gamma_p\, w_p(y,z)
\\
\label{Base_Vortex_compvit}
\end{array}
\end{equation}
%================================
\subsection{Param�tres physiques}
%================================

%-------------------------------
\subsubsection{Marche en temps}
%-------------------------------
La position initiale de chaque vortex est tir�e de mani�re al�atoire. On calcul
les d�placements successifs de chacun des vortex dans le plan d'entr�e par
int�gration explicite du champ de vitesse lagrangien : 
\begin{equation}\notag
\begin{array}{lcr}
\displaystyle
\frac{dy_p}{dt} = V(y,z)
&
\text{    et    }
&
\displaystyle
\frac{dz_p}{dt} = W(y,z)
\\
\end{array}
\end{equation}
Les vortex sont alors assimil�s � des particules ponctuelles qui sont convect�es
par le champ $(V,W)$. Ce champ peut �tre impos� par des tirages al�atoires ou
bien d�duit de la vitesse induite par les vortex dans le plan d'entr�e. Dans ce
cas $V(x,y) = \overline{V}(y,z) + v (y,z)$ et $W(y,z)= \overline{W}(y,z) +
w(y,z)$ o� $\overline{V}$ et $\overline{W}$ sont les composantes de la vitesse
transverse moyenne qu'impose l'utilisateur � l'aide des fichiers de donn�es. 

%---------------------------------------------------
\subsubsection{Intensit� et dur�e de vie des vortex}
%---------------------------------------------------
Il serait possible, � partir de l'�quation de transport de la vorticit�,
d'obtenir un mod�le d'�volution pour l'intensit� du vecteur tourbillon
$\omega_p$ associ� � chacun des vortex. En pratique, on pr�f�re utiliser un
mod�le simplifi� dans lequel la circulation des tourbillons ne d�pend que de la
postion de ces derniers � l'instant consid�r�. La circulation initiale de chaque
vortex est alors obtenue � partir de la relation : 
\begin{equation}\notag
|\Gamma_p| = 4 \sqrt{\frac{\pi\,S\,k}{3N\,[2ln(3)-3ln(2)]}}
\end{equation}
o� $S$ est la surface du plan d'entr�e, $N$ le nombre de vortex, et $k$
l'�nergie cin�tique turbulente au point o� se trouve le vortex � l'instant
consid�r�. Le signe de $\Gamma_p$ correspond au sens de rotation du vortex et
est tir� al�atoirement. 

Ce param�tre est celui qui contr�le l'intensit� des fluctuations. Sa d�pendance
en $k$ exprime que, plus l'�coulement est turbulent, plus les vortex sont
intenses. La valeur de $k$ est sp�cifi�e par
l'utilisateur. Elle peut �tre constante ou impos�e � partir de profils d'�nergie
cin�tique turbulente en entr�e. 

Pour �viter que des structures trop allong�es ne se d�veloppent au niveau de
l'entr�e, l'utilisateur doit �galement sp�cifier un temps limites $\tau_p$ au
bout duquel le vortex $p$ va �tre d�truit. Ce temps $\tau_p$ peut �tre pris
constant ou estim� au moyen de la relation : 
\begin{equation}\notag
\tau_p = \frac{5 C_{\mu}k^{\frac{3}{2}}}{\varepsilon\,\overline{U}}
\end{equation}

$\overline{U}$ et $\varepsilon$ repr�sentent respectivement la vitesse moyenne
principale et la dissipation turbulente au point o� est initialement g�n�r� le
vortex ($C_{\mu}=0,09$). 
\\
Lorsque le temps �coul� depuis la cr�ation du vortex $p$ est sup�rieur �
$\tau_p$, le vortex est d�truit et un nouveau vortex g�n�r� (sa position et le
signe de $\Gamma_p$ sont tir�s de fa�on al�atoire). 

%-------------------------------- 
\subsubsection{Taille des vortex}
%--------------------------------
La taille des vortex peut �tre prise constante, ou calcul�e � partir des
relations :
\begin{equation}\notag
\begin{array}{ccc}
\displaystyle
\sigma = \frac{C_{\mu}^{\frac{3}{4}}k^{\frac{3}{2}}}{\varepsilon} 
& \text{    ou    } &
\sigma = max[L_t,L_k]
\\
\end{array}
\end{equation}
avec:
\begin{equation}\notag
\begin{array}{ccc}
\displaystyle
L_t = \sqrt{\left( \frac{5 \nu k}{\varepsilon} \right)} 
& \text{    et    } & 
\displaystyle
L_k = 200\, \left(\frac{\nu^3}{\varepsilon}\right)^{\frac{1}{4}}
\end{array}
\end{equation}
o� $\nu$, $k$ et $\varepsilon$ sont la viscosit� dynamique, l'�nergie cin�tique
turbulente et la dissipation turbulente au point o� se trouve le vortex. 

Dans tous les cas, la taille du vortex doit �tre sup�rieure � la taille des
mailles en entr�e afin que le vortex soit effectivement simul�. 

%==================================
\subsection{Conditions aux limites}
%==================================
Le champ de vitesse g�n�r� � l'aide de la relation \ref{Base_Vortex_compvit} ne tient pas
compte {\em a priori} des conditions aux limites appliqu�es sur les bords du plan
d'entr�e. Pour obtenir des valeurs de la vitesse qui soient coh�rentes sur les
fronti�res du domaine d'entr�e, des ``vortex images'', pseudo-vortex situ�s en
dehors du domaine d'entr�e, sont g�n�r�s � des positions particuli�res et leur
champ de vitesse associ� est superpos� au champ pr�c�demment calcul�.\\
Seuls les cas d'une conduite rectangulaire et d'une conduite circulaire
permettent la g�n�ration de ces pseudo-vortex.
On distingue pour cela trois types de conditions aux limites. 

\begin{figure}[h]
\centerline{\includegraphics[height=6cm]{../Base/Vortex/Images/condlimite.pdf}}
\caption{\label{Base_Vortex_condli} Principe de g�n�ration des ``vortex images'' suivant le
type de conditions aux limites dans une conduite carr�e.} 
\end{figure}

%----------------------------------
\subsubsection{Condition de paroi}
%----------------------------------
On cr�e, pour chaque vortex $P$ contenu dans le plan d'entr�e, un vortex image
$P'$ identique � $P$ (\textit{i.e} de m�me caract�ristiques) et sym�trique de $P$
par rapport au
point $J$ ($J$ �tant la projection orthogonalement � la paroi du point $M$
correspondant au centre de la face o� l'on cherche � calculer la vitesse). La
figure \ref{Base_Vortex_condli} illustre la technique dans le cas d'une conduite
carr�e. Dans ce cas les coordonn�es du vortex situ� en $P'$ v�rifient
$(y_{p'}+y_{p})/2 = y_{J}$ et $(z_{p'}+ z_{p})/2 = z_{J}$. Le champ de vitesse
per�u depuis le point $M$ au niveau du point $J$ est nul, ce qui est bien
l'effet recherch�. 

%------------------------------------
\subsubsection{Condition de sym�trie}
%-------------------------------------
La technique est identique � celle utilis�e pour les conditions de paroi, mais
seule la composante pour la vitesse normale au bord est modifi�e dans ce cas. 

%---------------------------------------
\subsubsection{Condition de p�riodicit�}
%---------------------------------------
On cr�e pour chaque vortex, un vortex images $P'$ identique � $P$ mais translat�
d'une quantit� $L$ correspondant � la longueur qui s�pare les deux plans de la
section d'entr�e o� sont appliqu�es les conditions de p�riodicit�. Dans le cas
o� il y a deux directions de p�riodicit�, on cr�e deux vortex image.

%=============================================
\subsection{Composante de vitesse principale}
%=============================================
La m�thode des vortex ne g�n�rant pas de fluctuation $u$ de la vitesse dans la
direction principale, la derni�re composante est calcul�e � partir d'une
�quation de Langevin. Les coefficients de cette �quation sont d�termin�s par
identification des expressions obtenues pour les contraintes de Reynolds en
$R_{ij}-\varepsilon$. Dans le cas d'un �coulement en canal plan, cette technique
conduit � l'�quation : 
\begin{equation}\notag
\displaystyle
\frac{du}{dt} = - \frac{C_1}{2T} u + \left(\frac{2}{3}C_2-1\right)\frac{\partial
U}{\partial y} v + \sqrt{C_0\varepsilon} dW_i 
\end{equation}

avec $\displaystyle T=\frac{k}{\varepsilon}$, $C_1 = 1,8$, $C_2 = 0,6$,
$C_0=\frac{14}{15}$, et $dW_i$ une variable al�toire Gaussienne de variance
$\sqrt{dt}$. 

En pratique, l'�quation de Langevin n'am�liore pas vraiment les r�sultats. Elle
n'est utilis�e que dans le cas de conduites circulaires. 

%                      Code_Saturne version 1.3
%                      ------------------------
%
%     This file is part of the Code_Saturne Kernel, element of the
%     Code_Saturne CFD tool.
%
%     Copyright (C) 1998-2007 EDF S.A., France
%
%     contact: saturne-support@edf.fr
%
%     The Code_Saturne Kernel is free software; you can redistribute it
%     and/or modify it under the terms of the GNU General Public License
%     as published by the Free Software Foundation; either version 2 of
%     the License, or (at your option) any later version.
%
%     The Code_Saturne Kernel is distributed in the hope that it will be
%     useful, but WITHOUT ANY WARRANTY; without even the implied warranty
%     of MERCHANTABILITY or FITNESS FOR A PARTICULAR PURPOSE.  See the
%     GNU General Public License for more details.
%
%     You should have received a copy of the GNU General Public License
%     along with the Code_Saturne Kernel; if not, write to the
%     Free Software Foundation, Inc.,
%     51 Franklin St, Fifth Floor,
%     Boston, MA  02110-1301  USA
%
%-----------------------------------------------------------------------
%

%%%%%%%%%%%%%%%%%%%%%%%%%%%%%%%%%%
%%%%%%%%%%%%%%%%%%%%%%%%%%%%%%%%%%
\section{Mise en \oe uvre}
%%%%%%%%%%%%%%%%%%%%%%%%%%%%%%%%%%
%%%%%%%%%%%%%%%%%%%%%%%%%%%%%%%%%%
Le syst\`eme (\ref{Cfbl_Cfmsvl_eq_densite_finale_cfmsvl}) est r\'esolu par une m\'ethode
d'incr\'ement et r\'esidu en utilisant
une m\'ethode de Jacobi pour inverser le syst\`eme si le terme convectif
est implicite et en utilisant une m\'ethode de gradient conjugu\'e
si le terme convectif est explicite (qui est le cas par d�faut).

Attention, les valeurs du flux de masse $\rho\,\vect{w}\cdot\vect{S}$ et
de la viscosit\'e $\Delta\,t\,c^2\frac{S}{d}$ aux faces de
bord, qui sont calcul\'ees dans \fort{cfmsfl} et \fort{cfmsvs} respectivement,
sont modifi\'ees imm\'ediatement apr\`es l'appel \`a ces sous-programmes.
En effet, il est indispensable que la contribution de bord de
$\left(\rho\,\vect{w}-\Delta\,t\,(c^2)\,\gradv\,\rho\right)\cdot\vect{S}$
repr\'esente exactement $\vect{Q}_{ac}\cdot\vect{S}$.
Pour cela,
\begin{itemize}
\item imm\'ediatement apr\`es l'appel \`a
\fort{cfmsfl}, on remplace la contribution de bord de
$\rho\,\vect{w}\cdot\vect{S}$
par le flux de masse exact, $\vect{Q}_{ac}\cdot\vect{S}$,
d\'etermin\'e \`a partir des conditions aux limites,
\item puis, imm\'ediatement apr\`es l'appel \`a
\fort{cfmsvs}, on annule la viscosit\'e au bord $\Delta\,t\,(c^2)$ pour
\'eliminer la contribution de $-\Delta\,t\,(c^2)\,(\gradv\,\rho)\cdot\vect{S}$
(l'annulation de la viscosit\'e n'est pas probl\'ematique pour la matrice,
puisqu'elle porte sur des incr\'ements).
\end{itemize}

\bigskip

Une fois qu'on a obtenu $\rho^{n+1}$,
on peut actualiser le flux de masse acoustique
aux faces $(\vect{Q}_{ac}^{n+1})_{ij} \cdot \vect{S}_{ij}$,
qui servira pour la convection des autres variables~:
\begin{equation}\label{Cfbl_Cfmsvl_eq_flux_masse_acoustique_cfmsvl}
\displaystyle(\vect{Q}_{ac}^{n+1})_{ij}\cdot\vect{S}_{ij}=
-\left(\Delta t^n (c^2)^n \gradv(\rho^{n+1})\right)_{ij}\cdot\vect{S}_{ij}
+\left(\rho^{n+\frac{1}{2}} \vect{w}^n\right)_{ij}\cdot\vect{S}_{ij}\\
\end{equation}
Ce calcul de flux est r\'ealis\'e par \fort{cfbsc3}.
Si l'on a choisi l'algorithme standard, \'equation~(\ref{Cfbl_Cfmsvl_eq_densite_cfmsvl}),
on compl\`ete le flux dans \fort{cfmsvl} imm\'ediatement apr\`es l'appel
\`a \fort{cfbsc3}.
En effet, dans ce cas,
la convection est explicite ($\rho^{n+\frac{1}{2}}=\rho^{n}$,
obtenu en imposant \var{ICONV(ISCA(IRHO(IPHAS)))=0})
et le sous-programme \fort{cfbsc3},
qui calcule le flux de masse aux faces,
ne prend pas en compte la contribution du terme
$\rho^{n+\frac{1}{2}}\,\vect{w}^n\cdot\vect{S}$. On ajoute donc cette
contribution dans \fort{cfmsvl}, apr\`es l'appel \`a \fort{cfbsc3}.
Au bord, en particulier, c'est bien le flux de masse calcul\'e \`a partir
des conditions aux limites que l'on obtient.

On actualise la pression \`a la fin de l'\'etape, en utilisant la loi d'\'etat~:
\begin{equation}
\displaystyle P_i^{pred}=P(\rho_i^{n+1},\varepsilon_i^{n})
\end{equation}


%%%%%%%%%%%%%%%%%%%%%%%%%%%%%%%%%%
%%%%%%%%%%%%%%%%%%%%%%%%%%%%%%%%%%
\section{Points \`a traiter}
%%%%%%%%%%%%%%%%%%%%%%%%%%%%%%%%%%
%%%%%%%%%%%%%%%%%%%%%%%%%%%%%%%%%%
Le calcul du flux de masse au  bord n'est pas enti\`erement satisfaisant
si la convection est trait\'ee de mani\`ere implicite
(algorithme non standard, non test\'e,
associ\'e \`a l'\'equation~(\ref{Cfbl_Cfmsvl_eq_densite_bis_cfmsvl}),
correspondant au choix $\rho^{n+\frac{1}{2}}=\rho^{n+1}$ et
obtenu en imposant \var{ICONV(ISCA(IRHO(IPHAS)))=1}).
En effet, apr\`es \fort{cfmsfl}, il faut d\'eterminer la vitesse de
convection $\vect{w}^n$ pour qu'apparaisse
$\rho^{n+1} \vect{w}^n\cdot\vect{n}$
au cours de la r\'esolution par \fort{codits}. De ce fait, on doit d\'eduire
une valeur de $\vect{w}^n$ \`a partir de la valeur
du flux de masse. Au bord, en particulier, il faut
donc diviser le flux de masse
issu des conditions aux limites par la valeur de bord de $\rho^{n+1}$.
Or, lorsque des conditions de Neumann sont appliqu\'ees \`a la
masse volumique,
la valeur de $\rho^{n+1}$ au bord n'est pas connue avant la r\'esolution du
syst\`eme. On utilise donc, au lieu de la valeur de bord inconnue de
$\rho^{n+1}$ la valeur de bord prise au pas de temps
pr\'ec\'edent $\rho^{n}$. Cette approximation est susceptible
d'affecter la valeur du flux de masse au bord.

%                      Code_Saturne version 1.3
%                      ------------------------
%
%     This file is part of the Code_Saturne Kernel, element of the
%     Code_Saturne CFD tool.
%
%     Copyright (C) 1998-2007 EDF S.A., France
%
%     contact: saturne-support@edf.fr
%
%     The Code_Saturne Kernel is free software; you can redistribute it
%     and/or modify it under the terms of the GNU General Public License
%     as published by the Free Software Foundation; either version 2 of
%     the License, or (at your option) any later version.
%
%     The Code_Saturne Kernel is distributed in the hope that it will be
%     useful, but WITHOUT ANY WARRANTY; without even the implied warranty
%     of MERCHANTABILITY or FITNESS FOR A PARTICULAR PURPOSE.  See the
%     GNU General Public License for more details.
%
%     You should have received a copy of the GNU General Public License
%     along with the Code_Saturne Kernel; if not, write to the
%     Free Software Foundation, Inc.,
%     51 Franklin St, Fifth Floor,
%     Boston, MA  02110-1301  USA
%
%-----------------------------------------------------------------------
%


\programme{navsto}

\vspace{1cm}
On s'int\'eresse \`a la r\'esolution du syst\`eme d'\'equations de Navier-Stokes
tridimensionnelles monophasiques, \`a une pression, instationnaires, en
incompressible ou faiblement dilatable, bas\'ees sur une discr\'etisation
temporelle de type Euler implicite d'ordre 1 ou Crank-Nicolson d'ordre 2 et sur
une discr\'etisation spatiale  par volumes finis colocalis\'ee.


%%%%%%%%%%%%%%%%%%%%%%%%%%%%%%%%%%
%%%%%%%%%%%%%%%%%%%%%%%%%%%%%%%%%%
\section{Fonction}
%%%%%%%%%%%%%%%%%%%%%%%%%%%%%%%%%%
%%%%%%%%%%%%%%%%%%%%%%%%%%%%%%%%%%

  Dans ce sous-programme sont calcul\'ees, \`a un pas de temps donn\'e, les
variables vitesse et pression de ce probl\`eme en proc\'edant en
deux  \'etapes issues d'une d\'ecomposition des op\'erateurs (m\'ethode \`a
pas fractionnaires).\\
Les variables sont donc suppos\'ees connues \`a
l'instant ${t^n}$ et on cherche \`a les d\'eterminer \`a l'instant\footnote{La pression est suppos�e connue � l'instant $t^{n-1+\theta}$ et recherch�e en $t^{n+\theta}$, avec $\theta=1$ ou $1/2$ suivant le sch�ma en temps consid�r�.} ${t^{n+1}}$. Soit ${\Delta {t^n} ={t^{n+1}- {t^n}}}$ le pas de temps associ\'e. Dans un premier temps, on r\'ealise l'\'etape de
pr\'ediction de la vitesse en r\'esolvant l'\'equation de quantit\'e de
mouvement avec une pression explicite. Suit l'\'etape de correction de la
pression (ou projection de la vitesse) qui permet d'obtenir un champ de vitesse \`a divergence nulle.\\\\
Les \'equations en continu sont donc :
\begin{equation}
\left\{\begin{array}{l}
\displaystyle\frac{\partial}{\partial t}(\rho \vect{u}) + \dive(\rho\, \vect{u} \otimes \vect{u})
=\dive(\tens{\sigma}) + \vect{TS} - \tens{K}\,\vect{u}\\
\dive(\rho \vect{u}) = \Gamma
\end{array}\right.
\end{equation}

%(plus tard $\frac{\partial \rho}{\partial t} + \dive(\rho \vect{u}) = \Gamma$)



avec $\rho$ la masse volumique, $\vect{u}$ le champ de vitesse,
$[\,\vect{TS}-\tens{K}\,\vect{u}\,]$ les autres termes sources ($\tens{K}$~est un
tenseur diagonal positif par d\'efinition), $\tens{\sigma}$ le tenseur
de contraintes, $\tens{\tau}$ le tenseur des contraintes visqueuses, $\mu$ la
viscosit\'e dynamique (mol\'eculaire et \'eventuellement turbulente), $\kappa$
la viscosit� de
volume (usuellement nulle et n�glig�e dans le code et dans la suite du document,
sauf en compressible),
$\tens{D}$ le tenseur taux de d\'eformation\footnote{\`A ne pas confondre, malgr\'e la m\^eme notation $D$,
avec les flux diffusifs $\vect{D}_{\,ij}$ et $\vect{D}_{\,{b}_{ik}}$ d\'ecrits par la suite dans ce
sous-programme.}, $\Gamma$ le terme source de masse.
\begin{equation}
\left\{\begin{array}{l}
\tens{\sigma} = \tens{\tau} - P\tens{Id}  \\
\tens{\tau} = 2\,\mu\ \tens{D} +\ (\kappa\ - \frac{2}{3}\mu)\  tr({\tens{D}})\
\tens{Id}  \\
\tens{D} = \frac{1}{2}(\ggrad\vect{u}+\,^{t}\ggrad\vect{u})
\end{array}\right.
\end{equation}
 \\

On rappelle la d\'efinition des notations employ\'ees\footnote{en
utilisant la convention de sommation d'Einstein.}~:
\begin{equation}\notag
\left\{\begin{array}{lll}
\left[\ggrad{\vect{a}}\right]_{ij} &=& \partial_j a_i\\
\left[\dive(\tens{\sigma})\right]_i &=& \partial_j \sigma_{ij}\\
\left[\vect{a}\otimes\vect{b}\right]_{ij} &= &
a_i\,b_j\\
\end{array}\right.
\end{equation}
et donc :
\begin{equation}\notag
\begin{array}{lll}
\left[\dive(\vect{a}\otimes\vect{b})\right]_i &= &
\partial_j (a_i\,b_j)
\end{array}
\end{equation}

\minititre{Remarque}
Dans le cas de la prise en compte d'une masse volumique variable, l'�quation de continuit� s'�crit :
$$\frac{\partial \rho}{\partial t} + \dive{\,(\rho\,\vect{u})} = \Gamma  $$
Cette �quation n'est pas prise en compte dans l'�tape de projection (on continue � r�soudre
seulement
$\displaystyle \dive(\,{\rho\,\vect{u}}) = \Gamma$) alors que le terme
$\displaystyle \frac{\partial \rho}{\partial t}$ appara\^{\i}t lors de l'�tape de pr\'ediction de la vitesse
dans le sous-programme \fort{preduv}. Si ce terme joue un r�le sensible, l'algorithme compressible
de \CS\ (qui r�sout l'�quation compl�te) est alors sans doute plus adapt�.

%                      Code_Saturne version 1.3
%                      ------------------------
%
%     This file is part of the Code_Saturne Kernel, element of the
%     Code_Saturne CFD tool.
% 
%     Copyright (C) 1998-2007 EDF S.A., France
%
%     contact: saturne-support@edf.fr
% 
%     The Code_Saturne Kernel is free software; you can redistribute it
%     and/or modify it under the terms of the GNU General Public License
%     as published by the Free Software Foundation; either version 2 of
%     the License, or (at your option) any later version.
% 
%     The Code_Saturne Kernel is distributed in the hope that it will be
%     useful, but WITHOUT ANY WARRANTY; without even the implied warranty
%     of MERCHANTABILITY or FITNESS FOR A PARTICULAR PURPOSE.  See the
%     GNU General Public License for more details.
% 
%     You should have received a copy of the GNU General Public License
%     along with the Code_Saturne Kernel; if not, write to the
%     Free Software Foundation, Inc.,
%     51 Franklin St, Fifth Floor,
%     Boston, MA  02110-1301  USA
%
%-----------------------------------------------------------------------
%
%%%%%%%%%%%%%%%%%%%%%%%%%%%%%%%%%
%%%%%%%%%%%%%%%%%%%%%%%%%%%%%%%%%%
\section{Discr\'etisation}
%%%%%%%%%%%%%%%%%%%%%%%%%%%%%%%%%%
%%%%%%%%%%%%%%%%%%%%%%%%%%%%%%%%%%

Pour utiliser la m�thode, on se place tout d'abord dans un rep�re local d�fini
de mani�re � ce que le plan $(0yz)$, o� sont inject�s les vortex, soit confondu
avec le plan d'entr�e du calcul (voir figure \ref{Base_Vortex_entree}). 

\begin{figure}[h]
\centerline{\includegraphics[height=6cm]{../Base/Vortex/Images/entree.pdf}}
\caption{\label{Base_Vortex_entree} D�finiton des diff�rentes grandeurs dans le rep�re local
correspondant � l'entr�e d'une conduite de section carr�e.} 
\end{figure}

$u$, $v$ et $w$  sont les composantes de la vitesse fluctuante (principale et
transverse) dans ce plan, et
$\displaystyle \omega(y,z) = \frac{\partial w}{\partial y}-\frac{\partial v}{\partial z}$
la vorticit� dans la direction
normale au plan d'entr�e. $\overline{U}(y,z)$ repr�sente ici la vitesse
principale moyenne impos�e par l'utilisateur dans le plan d'entr�e. 

Chaque vortex $p$ va �tre caract�ris� par sa fonction de forme $\xi$ (identique
pour tous les vortex), sa
circulation $\Gamma_p$, son rayon $\sigma_p$ et les coordonn�es $(y_p,z_p)$ du
point $P$ o� est situ� le vortex dans le plan $(0yz)$. 

Pour cela, on suppose que la vorticit� g�n�r�e par un vortex $p$ au point $M$ de
coordonn�e $(y,z)$ s'�crit : 
\begin{equation}\notag
\omega_p(y,z)= \Gamma_p \, \xi_{\sigma_p}(r)
\end{equation}
o� $r = \sqrt{(y-y_p)^2+(z-z_p)^2}$ est la distance s�parant le point $M$ du point $P$.

Dans la m�thode implant�e, la fonction de forme est de type gaussienne modifi�e :
\begin{equation}\notag
\displaystyle
\xi_\sigma (r) = \frac{1}{2\pi \sigma^2} 
\left(2 e^{-\frac{r^2}{2\sigma^2}}-1\right) e^{-\frac{r^2}{2\sigma^2}}
\end{equation}

Le champ de vitesse induit par cette distribution de vorticit� s'obtient par
inversion des deux �quations de poisson suivantes qui sont d�duites de la
condition d'incompressibilit� dans la plan\footnote{\textit{i.e}
$\displaystyle \frac{\partial v}{\partial y}+\frac{\partial w}{\partial w} = 0$} :
\begin{equation}\notag
\begin{array}{lcr}
\displaystyle
\frac{\partial \omega}{\partial y} = \Delta w
&
\text{    et    }
&
\displaystyle
\frac{\partial \omega}{\partial y} = -\Delta v
\\
\end{array}
\end{equation}

Dans le cas g�n�ral, ce syst�me peut �tre int�gr� � l'aide de la formule de Biot et Savart.

Dans le cas d'une distribution de vorticit� de type gaussienne modifi�e, les
composantes de vitesse v�rifient : 
\begin{equation}\notag
\left\{
\begin{array}{c}
\displaystyle
v_p(y,x) = - \frac{1}{2\pi} \frac{(z-z_p)}{r^2}\left(1 -
e^{-\frac{r^2}{2\sigma^2}}\right)\,e^{-\frac{r^2}{2\sigma^2}} 
\\
\displaystyle
w_p(y,z) = \frac{1}{2\pi} \frac{(y-y_p)}{r^2}\left(1 -e^{-\frac{r^2}{2\sigma^2}}
\right)\,e^{-\frac{r^2}{2\sigma^2}} 
\end{array}
\right.
\end{equation}

Ces relations s'�tendent de fa�on imm�diate au cas de $N$ vortex distincts.
Le champ de vitesse induit par la distribution de vorticit� 
\begin{equation}
\omega(y,z) = \sum_{p=1}^N \Gamma_p \, \xi_{\sigma_p}(r)
\end{equation}
vaut au point $M$ :
\begin{equation}\notag
\begin{array}{lcr}
v(x,y) = \sum_{p=1}^N \Gamma_p\, v_p(y,z) 
&
\text{    et    }
&
w(y,z) = \sum_{p=1}^N \Gamma_p\, w_p(y,z)
\\
\label{Base_Vortex_compvit}
\end{array}
\end{equation}
%================================
\subsection{Param�tres physiques}
%================================

%-------------------------------
\subsubsection{Marche en temps}
%-------------------------------
La position initiale de chaque vortex est tir�e de mani�re al�atoire. On calcul
les d�placements successifs de chacun des vortex dans le plan d'entr�e par
int�gration explicite du champ de vitesse lagrangien : 
\begin{equation}\notag
\begin{array}{lcr}
\displaystyle
\frac{dy_p}{dt} = V(y,z)
&
\text{    et    }
&
\displaystyle
\frac{dz_p}{dt} = W(y,z)
\\
\end{array}
\end{equation}
Les vortex sont alors assimil�s � des particules ponctuelles qui sont convect�es
par le champ $(V,W)$. Ce champ peut �tre impos� par des tirages al�atoires ou
bien d�duit de la vitesse induite par les vortex dans le plan d'entr�e. Dans ce
cas $V(x,y) = \overline{V}(y,z) + v (y,z)$ et $W(y,z)= \overline{W}(y,z) +
w(y,z)$ o� $\overline{V}$ et $\overline{W}$ sont les composantes de la vitesse
transverse moyenne qu'impose l'utilisateur � l'aide des fichiers de donn�es. 

%---------------------------------------------------
\subsubsection{Intensit� et dur�e de vie des vortex}
%---------------------------------------------------
Il serait possible, � partir de l'�quation de transport de la vorticit�,
d'obtenir un mod�le d'�volution pour l'intensit� du vecteur tourbillon
$\omega_p$ associ� � chacun des vortex. En pratique, on pr�f�re utiliser un
mod�le simplifi� dans lequel la circulation des tourbillons ne d�pend que de la
postion de ces derniers � l'instant consid�r�. La circulation initiale de chaque
vortex est alors obtenue � partir de la relation : 
\begin{equation}\notag
|\Gamma_p| = 4 \sqrt{\frac{\pi\,S\,k}{3N\,[2ln(3)-3ln(2)]}}
\end{equation}
o� $S$ est la surface du plan d'entr�e, $N$ le nombre de vortex, et $k$
l'�nergie cin�tique turbulente au point o� se trouve le vortex � l'instant
consid�r�. Le signe de $\Gamma_p$ correspond au sens de rotation du vortex et
est tir� al�atoirement. 

Ce param�tre est celui qui contr�le l'intensit� des fluctuations. Sa d�pendance
en $k$ exprime que, plus l'�coulement est turbulent, plus les vortex sont
intenses. La valeur de $k$ est sp�cifi�e par
l'utilisateur. Elle peut �tre constante ou impos�e � partir de profils d'�nergie
cin�tique turbulente en entr�e. 

Pour �viter que des structures trop allong�es ne se d�veloppent au niveau de
l'entr�e, l'utilisateur doit �galement sp�cifier un temps limites $\tau_p$ au
bout duquel le vortex $p$ va �tre d�truit. Ce temps $\tau_p$ peut �tre pris
constant ou estim� au moyen de la relation : 
\begin{equation}\notag
\tau_p = \frac{5 C_{\mu}k^{\frac{3}{2}}}{\varepsilon\,\overline{U}}
\end{equation}

$\overline{U}$ et $\varepsilon$ repr�sentent respectivement la vitesse moyenne
principale et la dissipation turbulente au point o� est initialement g�n�r� le
vortex ($C_{\mu}=0,09$). 
\\
Lorsque le temps �coul� depuis la cr�ation du vortex $p$ est sup�rieur �
$\tau_p$, le vortex est d�truit et un nouveau vortex g�n�r� (sa position et le
signe de $\Gamma_p$ sont tir�s de fa�on al�atoire). 

%-------------------------------- 
\subsubsection{Taille des vortex}
%--------------------------------
La taille des vortex peut �tre prise constante, ou calcul�e � partir des
relations :
\begin{equation}\notag
\begin{array}{ccc}
\displaystyle
\sigma = \frac{C_{\mu}^{\frac{3}{4}}k^{\frac{3}{2}}}{\varepsilon} 
& \text{    ou    } &
\sigma = max[L_t,L_k]
\\
\end{array}
\end{equation}
avec:
\begin{equation}\notag
\begin{array}{ccc}
\displaystyle
L_t = \sqrt{\left( \frac{5 \nu k}{\varepsilon} \right)} 
& \text{    et    } & 
\displaystyle
L_k = 200\, \left(\frac{\nu^3}{\varepsilon}\right)^{\frac{1}{4}}
\end{array}
\end{equation}
o� $\nu$, $k$ et $\varepsilon$ sont la viscosit� dynamique, l'�nergie cin�tique
turbulente et la dissipation turbulente au point o� se trouve le vortex. 

Dans tous les cas, la taille du vortex doit �tre sup�rieure � la taille des
mailles en entr�e afin que le vortex soit effectivement simul�. 

%==================================
\subsection{Conditions aux limites}
%==================================
Le champ de vitesse g�n�r� � l'aide de la relation \ref{Base_Vortex_compvit} ne tient pas
compte {\em a priori} des conditions aux limites appliqu�es sur les bords du plan
d'entr�e. Pour obtenir des valeurs de la vitesse qui soient coh�rentes sur les
fronti�res du domaine d'entr�e, des ``vortex images'', pseudo-vortex situ�s en
dehors du domaine d'entr�e, sont g�n�r�s � des positions particuli�res et leur
champ de vitesse associ� est superpos� au champ pr�c�demment calcul�.\\
Seuls les cas d'une conduite rectangulaire et d'une conduite circulaire
permettent la g�n�ration de ces pseudo-vortex.
On distingue pour cela trois types de conditions aux limites. 

\begin{figure}[h]
\centerline{\includegraphics[height=6cm]{../Base/Vortex/Images/condlimite.pdf}}
\caption{\label{Base_Vortex_condli} Principe de g�n�ration des ``vortex images'' suivant le
type de conditions aux limites dans une conduite carr�e.} 
\end{figure}

%----------------------------------
\subsubsection{Condition de paroi}
%----------------------------------
On cr�e, pour chaque vortex $P$ contenu dans le plan d'entr�e, un vortex image
$P'$ identique � $P$ (\textit{i.e} de m�me caract�ristiques) et sym�trique de $P$
par rapport au
point $J$ ($J$ �tant la projection orthogonalement � la paroi du point $M$
correspondant au centre de la face o� l'on cherche � calculer la vitesse). La
figure \ref{Base_Vortex_condli} illustre la technique dans le cas d'une conduite
carr�e. Dans ce cas les coordonn�es du vortex situ� en $P'$ v�rifient
$(y_{p'}+y_{p})/2 = y_{J}$ et $(z_{p'}+ z_{p})/2 = z_{J}$. Le champ de vitesse
per�u depuis le point $M$ au niveau du point $J$ est nul, ce qui est bien
l'effet recherch�. 

%------------------------------------
\subsubsection{Condition de sym�trie}
%-------------------------------------
La technique est identique � celle utilis�e pour les conditions de paroi, mais
seule la composante pour la vitesse normale au bord est modifi�e dans ce cas. 

%---------------------------------------
\subsubsection{Condition de p�riodicit�}
%---------------------------------------
On cr�e pour chaque vortex, un vortex images $P'$ identique � $P$ mais translat�
d'une quantit� $L$ correspondant � la longueur qui s�pare les deux plans de la
section d'entr�e o� sont appliqu�es les conditions de p�riodicit�. Dans le cas
o� il y a deux directions de p�riodicit�, on cr�e deux vortex image.

%=============================================
\subsection{Composante de vitesse principale}
%=============================================
La m�thode des vortex ne g�n�rant pas de fluctuation $u$ de la vitesse dans la
direction principale, la derni�re composante est calcul�e � partir d'une
�quation de Langevin. Les coefficients de cette �quation sont d�termin�s par
identification des expressions obtenues pour les contraintes de Reynolds en
$R_{ij}-\varepsilon$. Dans le cas d'un �coulement en canal plan, cette technique
conduit � l'�quation : 
\begin{equation}\notag
\displaystyle
\frac{du}{dt} = - \frac{C_1}{2T} u + \left(\frac{2}{3}C_2-1\right)\frac{\partial
U}{\partial y} v + \sqrt{C_0\varepsilon} dW_i 
\end{equation}

avec $\displaystyle T=\frac{k}{\varepsilon}$, $C_1 = 1,8$, $C_2 = 0,6$,
$C_0=\frac{14}{15}$, et $dW_i$ une variable al�toire Gaussienne de variance
$\sqrt{dt}$. 

En pratique, l'�quation de Langevin n'am�liore pas vraiment les r�sultats. Elle
n'est utilis�e que dans le cas de conduites circulaires. 

%                      Code_Saturne version 1.3
%                      ------------------------
%
%     This file is part of the Code_Saturne Kernel, element of the
%     Code_Saturne CFD tool.
%
%     Copyright (C) 1998-2007 EDF S.A., France
%
%     contact: saturne-support@edf.fr
%
%     The Code_Saturne Kernel is free software; you can redistribute it
%     and/or modify it under the terms of the GNU General Public License
%     as published by the Free Software Foundation; either version 2 of
%     the License, or (at your option) any later version.
%
%     The Code_Saturne Kernel is distributed in the hope that it will be
%     useful, but WITHOUT ANY WARRANTY; without even the implied warranty
%     of MERCHANTABILITY or FITNESS FOR A PARTICULAR PURPOSE.  See the
%     GNU General Public License for more details.
%
%     You should have received a copy of the GNU General Public License
%     along with the Code_Saturne Kernel; if not, write to the
%     Free Software Foundation, Inc.,
%     51 Franklin St, Fifth Floor,
%     Boston, MA  02110-1301  USA
%
%-----------------------------------------------------------------------
%

%%%%%%%%%%%%%%%%%%%%%%%%%%%%%%%%%%
%%%%%%%%%%%%%%%%%%%%%%%%%%%%%%%%%%
\section{Mise en \oe uvre}
%%%%%%%%%%%%%%%%%%%%%%%%%%%%%%%%%%
%%%%%%%%%%%%%%%%%%%%%%%%%%%%%%%%%%
Le syst\`eme (\ref{Cfbl_Cfmsvl_eq_densite_finale_cfmsvl}) est r\'esolu par une m\'ethode
d'incr\'ement et r\'esidu en utilisant
une m\'ethode de Jacobi pour inverser le syst\`eme si le terme convectif
est implicite et en utilisant une m\'ethode de gradient conjugu\'e
si le terme convectif est explicite (qui est le cas par d�faut).

Attention, les valeurs du flux de masse $\rho\,\vect{w}\cdot\vect{S}$ et
de la viscosit\'e $\Delta\,t\,c^2\frac{S}{d}$ aux faces de
bord, qui sont calcul\'ees dans \fort{cfmsfl} et \fort{cfmsvs} respectivement,
sont modifi\'ees imm\'ediatement apr\`es l'appel \`a ces sous-programmes.
En effet, il est indispensable que la contribution de bord de
$\left(\rho\,\vect{w}-\Delta\,t\,(c^2)\,\gradv\,\rho\right)\cdot\vect{S}$
repr\'esente exactement $\vect{Q}_{ac}\cdot\vect{S}$.
Pour cela,
\begin{itemize}
\item imm\'ediatement apr\`es l'appel \`a
\fort{cfmsfl}, on remplace la contribution de bord de
$\rho\,\vect{w}\cdot\vect{S}$
par le flux de masse exact, $\vect{Q}_{ac}\cdot\vect{S}$,
d\'etermin\'e \`a partir des conditions aux limites,
\item puis, imm\'ediatement apr\`es l'appel \`a
\fort{cfmsvs}, on annule la viscosit\'e au bord $\Delta\,t\,(c^2)$ pour
\'eliminer la contribution de $-\Delta\,t\,(c^2)\,(\gradv\,\rho)\cdot\vect{S}$
(l'annulation de la viscosit\'e n'est pas probl\'ematique pour la matrice,
puisqu'elle porte sur des incr\'ements).
\end{itemize}

\bigskip

Une fois qu'on a obtenu $\rho^{n+1}$,
on peut actualiser le flux de masse acoustique
aux faces $(\vect{Q}_{ac}^{n+1})_{ij} \cdot \vect{S}_{ij}$,
qui servira pour la convection des autres variables~:
\begin{equation}\label{Cfbl_Cfmsvl_eq_flux_masse_acoustique_cfmsvl}
\displaystyle(\vect{Q}_{ac}^{n+1})_{ij}\cdot\vect{S}_{ij}=
-\left(\Delta t^n (c^2)^n \gradv(\rho^{n+1})\right)_{ij}\cdot\vect{S}_{ij}
+\left(\rho^{n+\frac{1}{2}} \vect{w}^n\right)_{ij}\cdot\vect{S}_{ij}\\
\end{equation}
Ce calcul de flux est r\'ealis\'e par \fort{cfbsc3}.
Si l'on a choisi l'algorithme standard, \'equation~(\ref{Cfbl_Cfmsvl_eq_densite_cfmsvl}),
on compl\`ete le flux dans \fort{cfmsvl} imm\'ediatement apr\`es l'appel
\`a \fort{cfbsc3}.
En effet, dans ce cas,
la convection est explicite ($\rho^{n+\frac{1}{2}}=\rho^{n}$,
obtenu en imposant \var{ICONV(ISCA(IRHO(IPHAS)))=0})
et le sous-programme \fort{cfbsc3},
qui calcule le flux de masse aux faces,
ne prend pas en compte la contribution du terme
$\rho^{n+\frac{1}{2}}\,\vect{w}^n\cdot\vect{S}$. On ajoute donc cette
contribution dans \fort{cfmsvl}, apr\`es l'appel \`a \fort{cfbsc3}.
Au bord, en particulier, c'est bien le flux de masse calcul\'e \`a partir
des conditions aux limites que l'on obtient.

On actualise la pression \`a la fin de l'\'etape, en utilisant la loi d'\'etat~:
\begin{equation}
\displaystyle P_i^{pred}=P(\rho_i^{n+1},\varepsilon_i^{n})
\end{equation}


%%%%%%%%%%%%%%%%%%%%%%%%%%%%%%%%%%
%%%%%%%%%%%%%%%%%%%%%%%%%%%%%%%%%%
\section{Points \`a traiter}
%%%%%%%%%%%%%%%%%%%%%%%%%%%%%%%%%%
%%%%%%%%%%%%%%%%%%%%%%%%%%%%%%%%%%
Le calcul du flux de masse au  bord n'est pas enti\`erement satisfaisant
si la convection est trait\'ee de mani\`ere implicite
(algorithme non standard, non test\'e,
associ\'e \`a l'\'equation~(\ref{Cfbl_Cfmsvl_eq_densite_bis_cfmsvl}),
correspondant au choix $\rho^{n+\frac{1}{2}}=\rho^{n+1}$ et
obtenu en imposant \var{ICONV(ISCA(IRHO(IPHAS)))=1}).
En effet, apr\`es \fort{cfmsfl}, il faut d\'eterminer la vitesse de
convection $\vect{w}^n$ pour qu'apparaisse
$\rho^{n+1} \vect{w}^n\cdot\vect{n}$
au cours de la r\'esolution par \fort{codits}. De ce fait, on doit d\'eduire
une valeur de $\vect{w}^n$ \`a partir de la valeur
du flux de masse. Au bord, en particulier, il faut
donc diviser le flux de masse
issu des conditions aux limites par la valeur de bord de $\rho^{n+1}$.
Or, lorsque des conditions de Neumann sont appliqu\'ees \`a la
masse volumique,
la valeur de $\rho^{n+1}$ au bord n'est pas connue avant la r\'esolution du
syst\`eme. On utilise donc, au lieu de la valeur de bord inconnue de
$\rho^{n+1}$ la valeur de bord prise au pas de temps
pr\'ec\'edent $\rho^{n}$. Cette approximation est susceptible
d'affecter la valeur du flux de masse au bord.

%                      Code_Saturne version 1.3
%                      ------------------------
%
%     This file is part of the Code_Saturne Kernel, element of the
%     Code_Saturne CFD tool.
%
%     Copyright (C) 1998-2007 EDF S.A., France
%
%     contact: saturne-support@edf.fr
%
%     The Code_Saturne Kernel is free software; you can redistribute it
%     and/or modify it under the terms of the GNU General Public License
%     as published by the Free Software Foundation; either version 2 of
%     the License, or (at your option) any later version.
%
%     The Code_Saturne Kernel is distributed in the hope that it will be
%     useful, but WITHOUT ANY WARRANTY; without even the implied warranty
%     of MERCHANTABILITY or FITNESS FOR A PARTICULAR PURPOSE.  See the
%     GNU General Public License for more details.
%
%     You should have received a copy of the GNU General Public License
%     along with the Code_Saturne Kernel; if not, write to the
%     Free Software Foundation, Inc.,
%     51 Franklin St, Fifth Floor,
%     Boston, MA  02110-1301  USA
%
%-----------------------------------------------------------------------
%


\programme{navsto}

\vspace{1cm}
On s'int\'eresse \`a la r\'esolution du syst\`eme d'\'equations de Navier-Stokes
tridimensionnelles monophasiques, \`a une pression, instationnaires, en
incompressible ou faiblement dilatable, bas\'ees sur une discr\'etisation
temporelle de type Euler implicite d'ordre 1 ou Crank-Nicolson d'ordre 2 et sur
une discr\'etisation spatiale  par volumes finis colocalis\'ee.


%%%%%%%%%%%%%%%%%%%%%%%%%%%%%%%%%%
%%%%%%%%%%%%%%%%%%%%%%%%%%%%%%%%%%
\section{Fonction}
%%%%%%%%%%%%%%%%%%%%%%%%%%%%%%%%%%
%%%%%%%%%%%%%%%%%%%%%%%%%%%%%%%%%%

  Dans ce sous-programme sont calcul\'ees, \`a un pas de temps donn\'e, les
variables vitesse et pression de ce probl\`eme en proc\'edant en
deux  \'etapes issues d'une d\'ecomposition des op\'erateurs (m\'ethode \`a
pas fractionnaires).\\
Les variables sont donc suppos\'ees connues \`a
l'instant ${t^n}$ et on cherche \`a les d\'eterminer \`a l'instant\footnote{La pression est suppos�e connue � l'instant $t^{n-1+\theta}$ et recherch�e en $t^{n+\theta}$, avec $\theta=1$ ou $1/2$ suivant le sch�ma en temps consid�r�.} ${t^{n+1}}$. Soit ${\Delta {t^n} ={t^{n+1}- {t^n}}}$ le pas de temps associ\'e. Dans un premier temps, on r\'ealise l'\'etape de
pr\'ediction de la vitesse en r\'esolvant l'\'equation de quantit\'e de
mouvement avec une pression explicite. Suit l'\'etape de correction de la
pression (ou projection de la vitesse) qui permet d'obtenir un champ de vitesse \`a divergence nulle.\\\\
Les \'equations en continu sont donc :
\begin{equation}
\left\{\begin{array}{l}
\displaystyle\frac{\partial}{\partial t}(\rho \vect{u}) + \dive(\rho\, \vect{u} \otimes \vect{u})
=\dive(\tens{\sigma}) + \vect{TS} - \tens{K}\,\vect{u}\\
\dive(\rho \vect{u}) = \Gamma
\end{array}\right.
\end{equation}

%(plus tard $\frac{\partial \rho}{\partial t} + \dive(\rho \vect{u}) = \Gamma$)



avec $\rho$ la masse volumique, $\vect{u}$ le champ de vitesse,
$[\,\vect{TS}-\tens{K}\,\vect{u}\,]$ les autres termes sources ($\tens{K}$~est un
tenseur diagonal positif par d\'efinition), $\tens{\sigma}$ le tenseur
de contraintes, $\tens{\tau}$ le tenseur des contraintes visqueuses, $\mu$ la
viscosit\'e dynamique (mol\'eculaire et \'eventuellement turbulente), $\kappa$
la viscosit� de
volume (usuellement nulle et n�glig�e dans le code et dans la suite du document,
sauf en compressible),
$\tens{D}$ le tenseur taux de d\'eformation\footnote{\`A ne pas confondre, malgr\'e la m\^eme notation $D$,
avec les flux diffusifs $\vect{D}_{\,ij}$ et $\vect{D}_{\,{b}_{ik}}$ d\'ecrits par la suite dans ce
sous-programme.}, $\Gamma$ le terme source de masse.
\begin{equation}
\left\{\begin{array}{l}
\tens{\sigma} = \tens{\tau} - P\tens{Id}  \\
\tens{\tau} = 2\,\mu\ \tens{D} +\ (\kappa\ - \frac{2}{3}\mu)\  tr({\tens{D}})\
\tens{Id}  \\
\tens{D} = \frac{1}{2}(\ggrad\vect{u}+\,^{t}\ggrad\vect{u})
\end{array}\right.
\end{equation}
 \\

On rappelle la d\'efinition des notations employ\'ees\footnote{en
utilisant la convention de sommation d'Einstein.}~:
\begin{equation}\notag
\left\{\begin{array}{lll}
\left[\ggrad{\vect{a}}\right]_{ij} &=& \partial_j a_i\\
\left[\dive(\tens{\sigma})\right]_i &=& \partial_j \sigma_{ij}\\
\left[\vect{a}\otimes\vect{b}\right]_{ij} &= &
a_i\,b_j\\
\end{array}\right.
\end{equation}
et donc :
\begin{equation}\notag
\begin{array}{lll}
\left[\dive(\vect{a}\otimes\vect{b})\right]_i &= &
\partial_j (a_i\,b_j)
\end{array}
\end{equation}

\minititre{Remarque}
Dans le cas de la prise en compte d'une masse volumique variable, l'�quation de continuit� s'�crit :
$$\frac{\partial \rho}{\partial t} + \dive{\,(\rho\,\vect{u})} = \Gamma  $$
Cette �quation n'est pas prise en compte dans l'�tape de projection (on continue � r�soudre
seulement
$\displaystyle \dive(\,{\rho\,\vect{u}}) = \Gamma$) alors que le terme
$\displaystyle \frac{\partial \rho}{\partial t}$ appara\^{\i}t lors de l'�tape de pr\'ediction de la vitesse
dans le sous-programme \fort{preduv}. Si ce terme joue un r�le sensible, l'algorithme compressible
de \CS\ (qui r�sout l'�quation compl�te) est alors sans doute plus adapt�.

%                      Code_Saturne version 1.3
%                      ------------------------
%
%     This file is part of the Code_Saturne Kernel, element of the
%     Code_Saturne CFD tool.
% 
%     Copyright (C) 1998-2007 EDF S.A., France
%
%     contact: saturne-support@edf.fr
% 
%     The Code_Saturne Kernel is free software; you can redistribute it
%     and/or modify it under the terms of the GNU General Public License
%     as published by the Free Software Foundation; either version 2 of
%     the License, or (at your option) any later version.
% 
%     The Code_Saturne Kernel is distributed in the hope that it will be
%     useful, but WITHOUT ANY WARRANTY; without even the implied warranty
%     of MERCHANTABILITY or FITNESS FOR A PARTICULAR PURPOSE.  See the
%     GNU General Public License for more details.
% 
%     You should have received a copy of the GNU General Public License
%     along with the Code_Saturne Kernel; if not, write to the
%     Free Software Foundation, Inc.,
%     51 Franklin St, Fifth Floor,
%     Boston, MA  02110-1301  USA
%
%-----------------------------------------------------------------------
%
%%%%%%%%%%%%%%%%%%%%%%%%%%%%%%%%%
%%%%%%%%%%%%%%%%%%%%%%%%%%%%%%%%%%
\section{Discr\'etisation}
%%%%%%%%%%%%%%%%%%%%%%%%%%%%%%%%%%
%%%%%%%%%%%%%%%%%%%%%%%%%%%%%%%%%%

Pour utiliser la m�thode, on se place tout d'abord dans un rep�re local d�fini
de mani�re � ce que le plan $(0yz)$, o� sont inject�s les vortex, soit confondu
avec le plan d'entr�e du calcul (voir figure \ref{Base_Vortex_entree}). 

\begin{figure}[h]
\centerline{\includegraphics[height=6cm]{../Base/Vortex/Images/entree.pdf}}
\caption{\label{Base_Vortex_entree} D�finiton des diff�rentes grandeurs dans le rep�re local
correspondant � l'entr�e d'une conduite de section carr�e.} 
\end{figure}

$u$, $v$ et $w$  sont les composantes de la vitesse fluctuante (principale et
transverse) dans ce plan, et
$\displaystyle \omega(y,z) = \frac{\partial w}{\partial y}-\frac{\partial v}{\partial z}$
la vorticit� dans la direction
normale au plan d'entr�e. $\overline{U}(y,z)$ repr�sente ici la vitesse
principale moyenne impos�e par l'utilisateur dans le plan d'entr�e. 

Chaque vortex $p$ va �tre caract�ris� par sa fonction de forme $\xi$ (identique
pour tous les vortex), sa
circulation $\Gamma_p$, son rayon $\sigma_p$ et les coordonn�es $(y_p,z_p)$ du
point $P$ o� est situ� le vortex dans le plan $(0yz)$. 

Pour cela, on suppose que la vorticit� g�n�r�e par un vortex $p$ au point $M$ de
coordonn�e $(y,z)$ s'�crit : 
\begin{equation}\notag
\omega_p(y,z)= \Gamma_p \, \xi_{\sigma_p}(r)
\end{equation}
o� $r = \sqrt{(y-y_p)^2+(z-z_p)^2}$ est la distance s�parant le point $M$ du point $P$.

Dans la m�thode implant�e, la fonction de forme est de type gaussienne modifi�e :
\begin{equation}\notag
\displaystyle
\xi_\sigma (r) = \frac{1}{2\pi \sigma^2} 
\left(2 e^{-\frac{r^2}{2\sigma^2}}-1\right) e^{-\frac{r^2}{2\sigma^2}}
\end{equation}

Le champ de vitesse induit par cette distribution de vorticit� s'obtient par
inversion des deux �quations de poisson suivantes qui sont d�duites de la
condition d'incompressibilit� dans la plan\footnote{\textit{i.e}
$\displaystyle \frac{\partial v}{\partial y}+\frac{\partial w}{\partial w} = 0$} :
\begin{equation}\notag
\begin{array}{lcr}
\displaystyle
\frac{\partial \omega}{\partial y} = \Delta w
&
\text{    et    }
&
\displaystyle
\frac{\partial \omega}{\partial y} = -\Delta v
\\
\end{array}
\end{equation}

Dans le cas g�n�ral, ce syst�me peut �tre int�gr� � l'aide de la formule de Biot et Savart.

Dans le cas d'une distribution de vorticit� de type gaussienne modifi�e, les
composantes de vitesse v�rifient : 
\begin{equation}\notag
\left\{
\begin{array}{c}
\displaystyle
v_p(y,x) = - \frac{1}{2\pi} \frac{(z-z_p)}{r^2}\left(1 -
e^{-\frac{r^2}{2\sigma^2}}\right)\,e^{-\frac{r^2}{2\sigma^2}} 
\\
\displaystyle
w_p(y,z) = \frac{1}{2\pi} \frac{(y-y_p)}{r^2}\left(1 -e^{-\frac{r^2}{2\sigma^2}}
\right)\,e^{-\frac{r^2}{2\sigma^2}} 
\end{array}
\right.
\end{equation}

Ces relations s'�tendent de fa�on imm�diate au cas de $N$ vortex distincts.
Le champ de vitesse induit par la distribution de vorticit� 
\begin{equation}
\omega(y,z) = \sum_{p=1}^N \Gamma_p \, \xi_{\sigma_p}(r)
\end{equation}
vaut au point $M$ :
\begin{equation}\notag
\begin{array}{lcr}
v(x,y) = \sum_{p=1}^N \Gamma_p\, v_p(y,z) 
&
\text{    et    }
&
w(y,z) = \sum_{p=1}^N \Gamma_p\, w_p(y,z)
\\
\label{Base_Vortex_compvit}
\end{array}
\end{equation}
%================================
\subsection{Param�tres physiques}
%================================

%-------------------------------
\subsubsection{Marche en temps}
%-------------------------------
La position initiale de chaque vortex est tir�e de mani�re al�atoire. On calcul
les d�placements successifs de chacun des vortex dans le plan d'entr�e par
int�gration explicite du champ de vitesse lagrangien : 
\begin{equation}\notag
\begin{array}{lcr}
\displaystyle
\frac{dy_p}{dt} = V(y,z)
&
\text{    et    }
&
\displaystyle
\frac{dz_p}{dt} = W(y,z)
\\
\end{array}
\end{equation}
Les vortex sont alors assimil�s � des particules ponctuelles qui sont convect�es
par le champ $(V,W)$. Ce champ peut �tre impos� par des tirages al�atoires ou
bien d�duit de la vitesse induite par les vortex dans le plan d'entr�e. Dans ce
cas $V(x,y) = \overline{V}(y,z) + v (y,z)$ et $W(y,z)= \overline{W}(y,z) +
w(y,z)$ o� $\overline{V}$ et $\overline{W}$ sont les composantes de la vitesse
transverse moyenne qu'impose l'utilisateur � l'aide des fichiers de donn�es. 

%---------------------------------------------------
\subsubsection{Intensit� et dur�e de vie des vortex}
%---------------------------------------------------
Il serait possible, � partir de l'�quation de transport de la vorticit�,
d'obtenir un mod�le d'�volution pour l'intensit� du vecteur tourbillon
$\omega_p$ associ� � chacun des vortex. En pratique, on pr�f�re utiliser un
mod�le simplifi� dans lequel la circulation des tourbillons ne d�pend que de la
postion de ces derniers � l'instant consid�r�. La circulation initiale de chaque
vortex est alors obtenue � partir de la relation : 
\begin{equation}\notag
|\Gamma_p| = 4 \sqrt{\frac{\pi\,S\,k}{3N\,[2ln(3)-3ln(2)]}}
\end{equation}
o� $S$ est la surface du plan d'entr�e, $N$ le nombre de vortex, et $k$
l'�nergie cin�tique turbulente au point o� se trouve le vortex � l'instant
consid�r�. Le signe de $\Gamma_p$ correspond au sens de rotation du vortex et
est tir� al�atoirement. 

Ce param�tre est celui qui contr�le l'intensit� des fluctuations. Sa d�pendance
en $k$ exprime que, plus l'�coulement est turbulent, plus les vortex sont
intenses. La valeur de $k$ est sp�cifi�e par
l'utilisateur. Elle peut �tre constante ou impos�e � partir de profils d'�nergie
cin�tique turbulente en entr�e. 

Pour �viter que des structures trop allong�es ne se d�veloppent au niveau de
l'entr�e, l'utilisateur doit �galement sp�cifier un temps limites $\tau_p$ au
bout duquel le vortex $p$ va �tre d�truit. Ce temps $\tau_p$ peut �tre pris
constant ou estim� au moyen de la relation : 
\begin{equation}\notag
\tau_p = \frac{5 C_{\mu}k^{\frac{3}{2}}}{\varepsilon\,\overline{U}}
\end{equation}

$\overline{U}$ et $\varepsilon$ repr�sentent respectivement la vitesse moyenne
principale et la dissipation turbulente au point o� est initialement g�n�r� le
vortex ($C_{\mu}=0,09$). 
\\
Lorsque le temps �coul� depuis la cr�ation du vortex $p$ est sup�rieur �
$\tau_p$, le vortex est d�truit et un nouveau vortex g�n�r� (sa position et le
signe de $\Gamma_p$ sont tir�s de fa�on al�atoire). 

%-------------------------------- 
\subsubsection{Taille des vortex}
%--------------------------------
La taille des vortex peut �tre prise constante, ou calcul�e � partir des
relations :
\begin{equation}\notag
\begin{array}{ccc}
\displaystyle
\sigma = \frac{C_{\mu}^{\frac{3}{4}}k^{\frac{3}{2}}}{\varepsilon} 
& \text{    ou    } &
\sigma = max[L_t,L_k]
\\
\end{array}
\end{equation}
avec:
\begin{equation}\notag
\begin{array}{ccc}
\displaystyle
L_t = \sqrt{\left( \frac{5 \nu k}{\varepsilon} \right)} 
& \text{    et    } & 
\displaystyle
L_k = 200\, \left(\frac{\nu^3}{\varepsilon}\right)^{\frac{1}{4}}
\end{array}
\end{equation}
o� $\nu$, $k$ et $\varepsilon$ sont la viscosit� dynamique, l'�nergie cin�tique
turbulente et la dissipation turbulente au point o� se trouve le vortex. 

Dans tous les cas, la taille du vortex doit �tre sup�rieure � la taille des
mailles en entr�e afin que le vortex soit effectivement simul�. 

%==================================
\subsection{Conditions aux limites}
%==================================
Le champ de vitesse g�n�r� � l'aide de la relation \ref{Base_Vortex_compvit} ne tient pas
compte {\em a priori} des conditions aux limites appliqu�es sur les bords du plan
d'entr�e. Pour obtenir des valeurs de la vitesse qui soient coh�rentes sur les
fronti�res du domaine d'entr�e, des ``vortex images'', pseudo-vortex situ�s en
dehors du domaine d'entr�e, sont g�n�r�s � des positions particuli�res et leur
champ de vitesse associ� est superpos� au champ pr�c�demment calcul�.\\
Seuls les cas d'une conduite rectangulaire et d'une conduite circulaire
permettent la g�n�ration de ces pseudo-vortex.
On distingue pour cela trois types de conditions aux limites. 

\begin{figure}[h]
\centerline{\includegraphics[height=6cm]{../Base/Vortex/Images/condlimite.pdf}}
\caption{\label{Base_Vortex_condli} Principe de g�n�ration des ``vortex images'' suivant le
type de conditions aux limites dans une conduite carr�e.} 
\end{figure}

%----------------------------------
\subsubsection{Condition de paroi}
%----------------------------------
On cr�e, pour chaque vortex $P$ contenu dans le plan d'entr�e, un vortex image
$P'$ identique � $P$ (\textit{i.e} de m�me caract�ristiques) et sym�trique de $P$
par rapport au
point $J$ ($J$ �tant la projection orthogonalement � la paroi du point $M$
correspondant au centre de la face o� l'on cherche � calculer la vitesse). La
figure \ref{Base_Vortex_condli} illustre la technique dans le cas d'une conduite
carr�e. Dans ce cas les coordonn�es du vortex situ� en $P'$ v�rifient
$(y_{p'}+y_{p})/2 = y_{J}$ et $(z_{p'}+ z_{p})/2 = z_{J}$. Le champ de vitesse
per�u depuis le point $M$ au niveau du point $J$ est nul, ce qui est bien
l'effet recherch�. 

%------------------------------------
\subsubsection{Condition de sym�trie}
%-------------------------------------
La technique est identique � celle utilis�e pour les conditions de paroi, mais
seule la composante pour la vitesse normale au bord est modifi�e dans ce cas. 

%---------------------------------------
\subsubsection{Condition de p�riodicit�}
%---------------------------------------
On cr�e pour chaque vortex, un vortex images $P'$ identique � $P$ mais translat�
d'une quantit� $L$ correspondant � la longueur qui s�pare les deux plans de la
section d'entr�e o� sont appliqu�es les conditions de p�riodicit�. Dans le cas
o� il y a deux directions de p�riodicit�, on cr�e deux vortex image.

%=============================================
\subsection{Composante de vitesse principale}
%=============================================
La m�thode des vortex ne g�n�rant pas de fluctuation $u$ de la vitesse dans la
direction principale, la derni�re composante est calcul�e � partir d'une
�quation de Langevin. Les coefficients de cette �quation sont d�termin�s par
identification des expressions obtenues pour les contraintes de Reynolds en
$R_{ij}-\varepsilon$. Dans le cas d'un �coulement en canal plan, cette technique
conduit � l'�quation : 
\begin{equation}\notag
\displaystyle
\frac{du}{dt} = - \frac{C_1}{2T} u + \left(\frac{2}{3}C_2-1\right)\frac{\partial
U}{\partial y} v + \sqrt{C_0\varepsilon} dW_i 
\end{equation}

avec $\displaystyle T=\frac{k}{\varepsilon}$, $C_1 = 1,8$, $C_2 = 0,6$,
$C_0=\frac{14}{15}$, et $dW_i$ une variable al�toire Gaussienne de variance
$\sqrt{dt}$. 

En pratique, l'�quation de Langevin n'am�liore pas vraiment les r�sultats. Elle
n'est utilis�e que dans le cas de conduites circulaires. 

%                      Code_Saturne version 1.3
%                      ------------------------
%
%     This file is part of the Code_Saturne Kernel, element of the
%     Code_Saturne CFD tool.
%
%     Copyright (C) 1998-2007 EDF S.A., France
%
%     contact: saturne-support@edf.fr
%
%     The Code_Saturne Kernel is free software; you can redistribute it
%     and/or modify it under the terms of the GNU General Public License
%     as published by the Free Software Foundation; either version 2 of
%     the License, or (at your option) any later version.
%
%     The Code_Saturne Kernel is distributed in the hope that it will be
%     useful, but WITHOUT ANY WARRANTY; without even the implied warranty
%     of MERCHANTABILITY or FITNESS FOR A PARTICULAR PURPOSE.  See the
%     GNU General Public License for more details.
%
%     You should have received a copy of the GNU General Public License
%     along with the Code_Saturne Kernel; if not, write to the
%     Free Software Foundation, Inc.,
%     51 Franklin St, Fifth Floor,
%     Boston, MA  02110-1301  USA
%
%-----------------------------------------------------------------------
%

%%%%%%%%%%%%%%%%%%%%%%%%%%%%%%%%%%
%%%%%%%%%%%%%%%%%%%%%%%%%%%%%%%%%%
\section{Mise en \oe uvre}
%%%%%%%%%%%%%%%%%%%%%%%%%%%%%%%%%%
%%%%%%%%%%%%%%%%%%%%%%%%%%%%%%%%%%
Le syst\`eme (\ref{Cfbl_Cfmsvl_eq_densite_finale_cfmsvl}) est r\'esolu par une m\'ethode
d'incr\'ement et r\'esidu en utilisant
une m\'ethode de Jacobi pour inverser le syst\`eme si le terme convectif
est implicite et en utilisant une m\'ethode de gradient conjugu\'e
si le terme convectif est explicite (qui est le cas par d�faut).

Attention, les valeurs du flux de masse $\rho\,\vect{w}\cdot\vect{S}$ et
de la viscosit\'e $\Delta\,t\,c^2\frac{S}{d}$ aux faces de
bord, qui sont calcul\'ees dans \fort{cfmsfl} et \fort{cfmsvs} respectivement,
sont modifi\'ees imm\'ediatement apr\`es l'appel \`a ces sous-programmes.
En effet, il est indispensable que la contribution de bord de
$\left(\rho\,\vect{w}-\Delta\,t\,(c^2)\,\gradv\,\rho\right)\cdot\vect{S}$
repr\'esente exactement $\vect{Q}_{ac}\cdot\vect{S}$.
Pour cela,
\begin{itemize}
\item imm\'ediatement apr\`es l'appel \`a
\fort{cfmsfl}, on remplace la contribution de bord de
$\rho\,\vect{w}\cdot\vect{S}$
par le flux de masse exact, $\vect{Q}_{ac}\cdot\vect{S}$,
d\'etermin\'e \`a partir des conditions aux limites,
\item puis, imm\'ediatement apr\`es l'appel \`a
\fort{cfmsvs}, on annule la viscosit\'e au bord $\Delta\,t\,(c^2)$ pour
\'eliminer la contribution de $-\Delta\,t\,(c^2)\,(\gradv\,\rho)\cdot\vect{S}$
(l'annulation de la viscosit\'e n'est pas probl\'ematique pour la matrice,
puisqu'elle porte sur des incr\'ements).
\end{itemize}

\bigskip

Une fois qu'on a obtenu $\rho^{n+1}$,
on peut actualiser le flux de masse acoustique
aux faces $(\vect{Q}_{ac}^{n+1})_{ij} \cdot \vect{S}_{ij}$,
qui servira pour la convection des autres variables~:
\begin{equation}\label{Cfbl_Cfmsvl_eq_flux_masse_acoustique_cfmsvl}
\displaystyle(\vect{Q}_{ac}^{n+1})_{ij}\cdot\vect{S}_{ij}=
-\left(\Delta t^n (c^2)^n \gradv(\rho^{n+1})\right)_{ij}\cdot\vect{S}_{ij}
+\left(\rho^{n+\frac{1}{2}} \vect{w}^n\right)_{ij}\cdot\vect{S}_{ij}\\
\end{equation}
Ce calcul de flux est r\'ealis\'e par \fort{cfbsc3}.
Si l'on a choisi l'algorithme standard, \'equation~(\ref{Cfbl_Cfmsvl_eq_densite_cfmsvl}),
on compl\`ete le flux dans \fort{cfmsvl} imm\'ediatement apr\`es l'appel
\`a \fort{cfbsc3}.
En effet, dans ce cas,
la convection est explicite ($\rho^{n+\frac{1}{2}}=\rho^{n}$,
obtenu en imposant \var{ICONV(ISCA(IRHO(IPHAS)))=0})
et le sous-programme \fort{cfbsc3},
qui calcule le flux de masse aux faces,
ne prend pas en compte la contribution du terme
$\rho^{n+\frac{1}{2}}\,\vect{w}^n\cdot\vect{S}$. On ajoute donc cette
contribution dans \fort{cfmsvl}, apr\`es l'appel \`a \fort{cfbsc3}.
Au bord, en particulier, c'est bien le flux de masse calcul\'e \`a partir
des conditions aux limites que l'on obtient.

On actualise la pression \`a la fin de l'\'etape, en utilisant la loi d'\'etat~:
\begin{equation}
\displaystyle P_i^{pred}=P(\rho_i^{n+1},\varepsilon_i^{n})
\end{equation}


%%%%%%%%%%%%%%%%%%%%%%%%%%%%%%%%%%
%%%%%%%%%%%%%%%%%%%%%%%%%%%%%%%%%%
\section{Points \`a traiter}
%%%%%%%%%%%%%%%%%%%%%%%%%%%%%%%%%%
%%%%%%%%%%%%%%%%%%%%%%%%%%%%%%%%%%
Le calcul du flux de masse au  bord n'est pas enti\`erement satisfaisant
si la convection est trait\'ee de mani\`ere implicite
(algorithme non standard, non test\'e,
associ\'e \`a l'\'equation~(\ref{Cfbl_Cfmsvl_eq_densite_bis_cfmsvl}),
correspondant au choix $\rho^{n+\frac{1}{2}}=\rho^{n+1}$ et
obtenu en imposant \var{ICONV(ISCA(IRHO(IPHAS)))=1}).
En effet, apr\`es \fort{cfmsfl}, il faut d\'eterminer la vitesse de
convection $\vect{w}^n$ pour qu'apparaisse
$\rho^{n+1} \vect{w}^n\cdot\vect{n}$
au cours de la r\'esolution par \fort{codits}. De ce fait, on doit d\'eduire
une valeur de $\vect{w}^n$ \`a partir de la valeur
du flux de masse. Au bord, en particulier, il faut
donc diviser le flux de masse
issu des conditions aux limites par la valeur de bord de $\rho^{n+1}$.
Or, lorsque des conditions de Neumann sont appliqu\'ees \`a la
masse volumique,
la valeur de $\rho^{n+1}$ au bord n'est pas connue avant la r\'esolution du
syst\`eme. On utilise donc, au lieu de la valeur de bord inconnue de
$\rho^{n+1}$ la valeur de bord prise au pas de temps
pr\'ec\'edent $\rho^{n}$. Cette approximation est susceptible
d'affecter la valeur du flux de masse au bord.

%                      Code_Saturne version 1.3
%                      ------------------------
%
%     This file is part of the Code_Saturne Kernel, element of the
%     Code_Saturne CFD tool.
%
%     Copyright (C) 1998-2007 EDF S.A., France
%
%     contact: saturne-support@edf.fr
%
%     The Code_Saturne Kernel is free software; you can redistribute it
%     and/or modify it under the terms of the GNU General Public License
%     as published by the Free Software Foundation; either version 2 of
%     the License, or (at your option) any later version.
%
%     The Code_Saturne Kernel is distributed in the hope that it will be
%     useful, but WITHOUT ANY WARRANTY; without even the implied warranty
%     of MERCHANTABILITY or FITNESS FOR A PARTICULAR PURPOSE.  See the
%     GNU General Public License for more details.
%
%     You should have received a copy of the GNU General Public License
%     along with the Code_Saturne Kernel; if not, write to the
%     Free Software Foundation, Inc.,
%     51 Franklin St, Fifth Floor,
%     Boston, MA  02110-1301  USA
%
%-----------------------------------------------------------------------
%


\programme{navsto}

\vspace{1cm}
On s'int\'eresse \`a la r\'esolution du syst\`eme d'\'equations de Navier-Stokes
tridimensionnelles monophasiques, \`a une pression, instationnaires, en
incompressible ou faiblement dilatable, bas\'ees sur une discr\'etisation
temporelle de type Euler implicite d'ordre 1 ou Crank-Nicolson d'ordre 2 et sur
une discr\'etisation spatiale  par volumes finis colocalis\'ee.


%%%%%%%%%%%%%%%%%%%%%%%%%%%%%%%%%%
%%%%%%%%%%%%%%%%%%%%%%%%%%%%%%%%%%
\section{Fonction}
%%%%%%%%%%%%%%%%%%%%%%%%%%%%%%%%%%
%%%%%%%%%%%%%%%%%%%%%%%%%%%%%%%%%%

  Dans ce sous-programme sont calcul\'ees, \`a un pas de temps donn\'e, les
variables vitesse et pression de ce probl\`eme en proc\'edant en
deux  \'etapes issues d'une d\'ecomposition des op\'erateurs (m\'ethode \`a
pas fractionnaires).\\
Les variables sont donc suppos\'ees connues \`a
l'instant ${t^n}$ et on cherche \`a les d\'eterminer \`a l'instant\footnote{La pression est suppos�e connue � l'instant $t^{n-1+\theta}$ et recherch�e en $t^{n+\theta}$, avec $\theta=1$ ou $1/2$ suivant le sch�ma en temps consid�r�.} ${t^{n+1}}$. Soit ${\Delta {t^n} ={t^{n+1}- {t^n}}}$ le pas de temps associ\'e. Dans un premier temps, on r\'ealise l'\'etape de
pr\'ediction de la vitesse en r\'esolvant l'\'equation de quantit\'e de
mouvement avec une pression explicite. Suit l'\'etape de correction de la
pression (ou projection de la vitesse) qui permet d'obtenir un champ de vitesse \`a divergence nulle.\\\\
Les \'equations en continu sont donc :
\begin{equation}
\left\{\begin{array}{l}
\displaystyle\frac{\partial}{\partial t}(\rho \vect{u}) + \dive(\rho\, \vect{u} \otimes \vect{u})
=\dive(\tens{\sigma}) + \vect{TS} - \tens{K}\,\vect{u}\\
\dive(\rho \vect{u}) = \Gamma
\end{array}\right.
\end{equation}

%(plus tard $\frac{\partial \rho}{\partial t} + \dive(\rho \vect{u}) = \Gamma$)



avec $\rho$ la masse volumique, $\vect{u}$ le champ de vitesse,
$[\,\vect{TS}-\tens{K}\,\vect{u}\,]$ les autres termes sources ($\tens{K}$~est un
tenseur diagonal positif par d\'efinition), $\tens{\sigma}$ le tenseur
de contraintes, $\tens{\tau}$ le tenseur des contraintes visqueuses, $\mu$ la
viscosit\'e dynamique (mol\'eculaire et \'eventuellement turbulente), $\kappa$
la viscosit� de
volume (usuellement nulle et n�glig�e dans le code et dans la suite du document,
sauf en compressible),
$\tens{D}$ le tenseur taux de d\'eformation\footnote{\`A ne pas confondre, malgr\'e la m\^eme notation $D$,
avec les flux diffusifs $\vect{D}_{\,ij}$ et $\vect{D}_{\,{b}_{ik}}$ d\'ecrits par la suite dans ce
sous-programme.}, $\Gamma$ le terme source de masse.
\begin{equation}
\left\{\begin{array}{l}
\tens{\sigma} = \tens{\tau} - P\tens{Id}  \\
\tens{\tau} = 2\,\mu\ \tens{D} +\ (\kappa\ - \frac{2}{3}\mu)\  tr({\tens{D}})\
\tens{Id}  \\
\tens{D} = \frac{1}{2}(\ggrad\vect{u}+\,^{t}\ggrad\vect{u})
\end{array}\right.
\end{equation}
 \\

On rappelle la d\'efinition des notations employ\'ees\footnote{en
utilisant la convention de sommation d'Einstein.}~:
\begin{equation}\notag
\left\{\begin{array}{lll}
\left[\ggrad{\vect{a}}\right]_{ij} &=& \partial_j a_i\\
\left[\dive(\tens{\sigma})\right]_i &=& \partial_j \sigma_{ij}\\
\left[\vect{a}\otimes\vect{b}\right]_{ij} &= &
a_i\,b_j\\
\end{array}\right.
\end{equation}
et donc :
\begin{equation}\notag
\begin{array}{lll}
\left[\dive(\vect{a}\otimes\vect{b})\right]_i &= &
\partial_j (a_i\,b_j)
\end{array}
\end{equation}

\minititre{Remarque}
Dans le cas de la prise en compte d'une masse volumique variable, l'�quation de continuit� s'�crit :
$$\frac{\partial \rho}{\partial t} + \dive{\,(\rho\,\vect{u})} = \Gamma  $$
Cette �quation n'est pas prise en compte dans l'�tape de projection (on continue � r�soudre
seulement
$\displaystyle \dive(\,{\rho\,\vect{u}}) = \Gamma$) alors que le terme
$\displaystyle \frac{\partial \rho}{\partial t}$ appara\^{\i}t lors de l'�tape de pr\'ediction de la vitesse
dans le sous-programme \fort{preduv}. Si ce terme joue un r�le sensible, l'algorithme compressible
de \CS\ (qui r�sout l'�quation compl�te) est alors sans doute plus adapt�.

%                      Code_Saturne version 1.3
%                      ------------------------
%
%     This file is part of the Code_Saturne Kernel, element of the
%     Code_Saturne CFD tool.
% 
%     Copyright (C) 1998-2007 EDF S.A., France
%
%     contact: saturne-support@edf.fr
% 
%     The Code_Saturne Kernel is free software; you can redistribute it
%     and/or modify it under the terms of the GNU General Public License
%     as published by the Free Software Foundation; either version 2 of
%     the License, or (at your option) any later version.
% 
%     The Code_Saturne Kernel is distributed in the hope that it will be
%     useful, but WITHOUT ANY WARRANTY; without even the implied warranty
%     of MERCHANTABILITY or FITNESS FOR A PARTICULAR PURPOSE.  See the
%     GNU General Public License for more details.
% 
%     You should have received a copy of the GNU General Public License
%     along with the Code_Saturne Kernel; if not, write to the
%     Free Software Foundation, Inc.,
%     51 Franklin St, Fifth Floor,
%     Boston, MA  02110-1301  USA
%
%-----------------------------------------------------------------------
%
%%%%%%%%%%%%%%%%%%%%%%%%%%%%%%%%%
%%%%%%%%%%%%%%%%%%%%%%%%%%%%%%%%%%
\section{Discr\'etisation}
%%%%%%%%%%%%%%%%%%%%%%%%%%%%%%%%%%
%%%%%%%%%%%%%%%%%%%%%%%%%%%%%%%%%%

Pour utiliser la m�thode, on se place tout d'abord dans un rep�re local d�fini
de mani�re � ce que le plan $(0yz)$, o� sont inject�s les vortex, soit confondu
avec le plan d'entr�e du calcul (voir figure \ref{Base_Vortex_entree}). 

\begin{figure}[h]
\centerline{\includegraphics[height=6cm]{../Base/Vortex/Images/entree.pdf}}
\caption{\label{Base_Vortex_entree} D�finiton des diff�rentes grandeurs dans le rep�re local
correspondant � l'entr�e d'une conduite de section carr�e.} 
\end{figure}

$u$, $v$ et $w$  sont les composantes de la vitesse fluctuante (principale et
transverse) dans ce plan, et
$\displaystyle \omega(y,z) = \frac{\partial w}{\partial y}-\frac{\partial v}{\partial z}$
la vorticit� dans la direction
normale au plan d'entr�e. $\overline{U}(y,z)$ repr�sente ici la vitesse
principale moyenne impos�e par l'utilisateur dans le plan d'entr�e. 

Chaque vortex $p$ va �tre caract�ris� par sa fonction de forme $\xi$ (identique
pour tous les vortex), sa
circulation $\Gamma_p$, son rayon $\sigma_p$ et les coordonn�es $(y_p,z_p)$ du
point $P$ o� est situ� le vortex dans le plan $(0yz)$. 

Pour cela, on suppose que la vorticit� g�n�r�e par un vortex $p$ au point $M$ de
coordonn�e $(y,z)$ s'�crit : 
\begin{equation}\notag
\omega_p(y,z)= \Gamma_p \, \xi_{\sigma_p}(r)
\end{equation}
o� $r = \sqrt{(y-y_p)^2+(z-z_p)^2}$ est la distance s�parant le point $M$ du point $P$.

Dans la m�thode implant�e, la fonction de forme est de type gaussienne modifi�e :
\begin{equation}\notag
\displaystyle
\xi_\sigma (r) = \frac{1}{2\pi \sigma^2} 
\left(2 e^{-\frac{r^2}{2\sigma^2}}-1\right) e^{-\frac{r^2}{2\sigma^2}}
\end{equation}

Le champ de vitesse induit par cette distribution de vorticit� s'obtient par
inversion des deux �quations de poisson suivantes qui sont d�duites de la
condition d'incompressibilit� dans la plan\footnote{\textit{i.e}
$\displaystyle \frac{\partial v}{\partial y}+\frac{\partial w}{\partial w} = 0$} :
\begin{equation}\notag
\begin{array}{lcr}
\displaystyle
\frac{\partial \omega}{\partial y} = \Delta w
&
\text{    et    }
&
\displaystyle
\frac{\partial \omega}{\partial y} = -\Delta v
\\
\end{array}
\end{equation}

Dans le cas g�n�ral, ce syst�me peut �tre int�gr� � l'aide de la formule de Biot et Savart.

Dans le cas d'une distribution de vorticit� de type gaussienne modifi�e, les
composantes de vitesse v�rifient : 
\begin{equation}\notag
\left\{
\begin{array}{c}
\displaystyle
v_p(y,x) = - \frac{1}{2\pi} \frac{(z-z_p)}{r^2}\left(1 -
e^{-\frac{r^2}{2\sigma^2}}\right)\,e^{-\frac{r^2}{2\sigma^2}} 
\\
\displaystyle
w_p(y,z) = \frac{1}{2\pi} \frac{(y-y_p)}{r^2}\left(1 -e^{-\frac{r^2}{2\sigma^2}}
\right)\,e^{-\frac{r^2}{2\sigma^2}} 
\end{array}
\right.
\end{equation}

Ces relations s'�tendent de fa�on imm�diate au cas de $N$ vortex distincts.
Le champ de vitesse induit par la distribution de vorticit� 
\begin{equation}
\omega(y,z) = \sum_{p=1}^N \Gamma_p \, \xi_{\sigma_p}(r)
\end{equation}
vaut au point $M$ :
\begin{equation}\notag
\begin{array}{lcr}
v(x,y) = \sum_{p=1}^N \Gamma_p\, v_p(y,z) 
&
\text{    et    }
&
w(y,z) = \sum_{p=1}^N \Gamma_p\, w_p(y,z)
\\
\label{Base_Vortex_compvit}
\end{array}
\end{equation}
%================================
\subsection{Param�tres physiques}
%================================

%-------------------------------
\subsubsection{Marche en temps}
%-------------------------------
La position initiale de chaque vortex est tir�e de mani�re al�atoire. On calcul
les d�placements successifs de chacun des vortex dans le plan d'entr�e par
int�gration explicite du champ de vitesse lagrangien : 
\begin{equation}\notag
\begin{array}{lcr}
\displaystyle
\frac{dy_p}{dt} = V(y,z)
&
\text{    et    }
&
\displaystyle
\frac{dz_p}{dt} = W(y,z)
\\
\end{array}
\end{equation}
Les vortex sont alors assimil�s � des particules ponctuelles qui sont convect�es
par le champ $(V,W)$. Ce champ peut �tre impos� par des tirages al�atoires ou
bien d�duit de la vitesse induite par les vortex dans le plan d'entr�e. Dans ce
cas $V(x,y) = \overline{V}(y,z) + v (y,z)$ et $W(y,z)= \overline{W}(y,z) +
w(y,z)$ o� $\overline{V}$ et $\overline{W}$ sont les composantes de la vitesse
transverse moyenne qu'impose l'utilisateur � l'aide des fichiers de donn�es. 

%---------------------------------------------------
\subsubsection{Intensit� et dur�e de vie des vortex}
%---------------------------------------------------
Il serait possible, � partir de l'�quation de transport de la vorticit�,
d'obtenir un mod�le d'�volution pour l'intensit� du vecteur tourbillon
$\omega_p$ associ� � chacun des vortex. En pratique, on pr�f�re utiliser un
mod�le simplifi� dans lequel la circulation des tourbillons ne d�pend que de la
postion de ces derniers � l'instant consid�r�. La circulation initiale de chaque
vortex est alors obtenue � partir de la relation : 
\begin{equation}\notag
|\Gamma_p| = 4 \sqrt{\frac{\pi\,S\,k}{3N\,[2ln(3)-3ln(2)]}}
\end{equation}
o� $S$ est la surface du plan d'entr�e, $N$ le nombre de vortex, et $k$
l'�nergie cin�tique turbulente au point o� se trouve le vortex � l'instant
consid�r�. Le signe de $\Gamma_p$ correspond au sens de rotation du vortex et
est tir� al�atoirement. 

Ce param�tre est celui qui contr�le l'intensit� des fluctuations. Sa d�pendance
en $k$ exprime que, plus l'�coulement est turbulent, plus les vortex sont
intenses. La valeur de $k$ est sp�cifi�e par
l'utilisateur. Elle peut �tre constante ou impos�e � partir de profils d'�nergie
cin�tique turbulente en entr�e. 

Pour �viter que des structures trop allong�es ne se d�veloppent au niveau de
l'entr�e, l'utilisateur doit �galement sp�cifier un temps limites $\tau_p$ au
bout duquel le vortex $p$ va �tre d�truit. Ce temps $\tau_p$ peut �tre pris
constant ou estim� au moyen de la relation : 
\begin{equation}\notag
\tau_p = \frac{5 C_{\mu}k^{\frac{3}{2}}}{\varepsilon\,\overline{U}}
\end{equation}

$\overline{U}$ et $\varepsilon$ repr�sentent respectivement la vitesse moyenne
principale et la dissipation turbulente au point o� est initialement g�n�r� le
vortex ($C_{\mu}=0,09$). 
\\
Lorsque le temps �coul� depuis la cr�ation du vortex $p$ est sup�rieur �
$\tau_p$, le vortex est d�truit et un nouveau vortex g�n�r� (sa position et le
signe de $\Gamma_p$ sont tir�s de fa�on al�atoire). 

%-------------------------------- 
\subsubsection{Taille des vortex}
%--------------------------------
La taille des vortex peut �tre prise constante, ou calcul�e � partir des
relations :
\begin{equation}\notag
\begin{array}{ccc}
\displaystyle
\sigma = \frac{C_{\mu}^{\frac{3}{4}}k^{\frac{3}{2}}}{\varepsilon} 
& \text{    ou    } &
\sigma = max[L_t,L_k]
\\
\end{array}
\end{equation}
avec:
\begin{equation}\notag
\begin{array}{ccc}
\displaystyle
L_t = \sqrt{\left( \frac{5 \nu k}{\varepsilon} \right)} 
& \text{    et    } & 
\displaystyle
L_k = 200\, \left(\frac{\nu^3}{\varepsilon}\right)^{\frac{1}{4}}
\end{array}
\end{equation}
o� $\nu$, $k$ et $\varepsilon$ sont la viscosit� dynamique, l'�nergie cin�tique
turbulente et la dissipation turbulente au point o� se trouve le vortex. 

Dans tous les cas, la taille du vortex doit �tre sup�rieure � la taille des
mailles en entr�e afin que le vortex soit effectivement simul�. 

%==================================
\subsection{Conditions aux limites}
%==================================
Le champ de vitesse g�n�r� � l'aide de la relation \ref{Base_Vortex_compvit} ne tient pas
compte {\em a priori} des conditions aux limites appliqu�es sur les bords du plan
d'entr�e. Pour obtenir des valeurs de la vitesse qui soient coh�rentes sur les
fronti�res du domaine d'entr�e, des ``vortex images'', pseudo-vortex situ�s en
dehors du domaine d'entr�e, sont g�n�r�s � des positions particuli�res et leur
champ de vitesse associ� est superpos� au champ pr�c�demment calcul�.\\
Seuls les cas d'une conduite rectangulaire et d'une conduite circulaire
permettent la g�n�ration de ces pseudo-vortex.
On distingue pour cela trois types de conditions aux limites. 

\begin{figure}[h]
\centerline{\includegraphics[height=6cm]{../Base/Vortex/Images/condlimite.pdf}}
\caption{\label{Base_Vortex_condli} Principe de g�n�ration des ``vortex images'' suivant le
type de conditions aux limites dans une conduite carr�e.} 
\end{figure}

%----------------------------------
\subsubsection{Condition de paroi}
%----------------------------------
On cr�e, pour chaque vortex $P$ contenu dans le plan d'entr�e, un vortex image
$P'$ identique � $P$ (\textit{i.e} de m�me caract�ristiques) et sym�trique de $P$
par rapport au
point $J$ ($J$ �tant la projection orthogonalement � la paroi du point $M$
correspondant au centre de la face o� l'on cherche � calculer la vitesse). La
figure \ref{Base_Vortex_condli} illustre la technique dans le cas d'une conduite
carr�e. Dans ce cas les coordonn�es du vortex situ� en $P'$ v�rifient
$(y_{p'}+y_{p})/2 = y_{J}$ et $(z_{p'}+ z_{p})/2 = z_{J}$. Le champ de vitesse
per�u depuis le point $M$ au niveau du point $J$ est nul, ce qui est bien
l'effet recherch�. 

%------------------------------------
\subsubsection{Condition de sym�trie}
%-------------------------------------
La technique est identique � celle utilis�e pour les conditions de paroi, mais
seule la composante pour la vitesse normale au bord est modifi�e dans ce cas. 

%---------------------------------------
\subsubsection{Condition de p�riodicit�}
%---------------------------------------
On cr�e pour chaque vortex, un vortex images $P'$ identique � $P$ mais translat�
d'une quantit� $L$ correspondant � la longueur qui s�pare les deux plans de la
section d'entr�e o� sont appliqu�es les conditions de p�riodicit�. Dans le cas
o� il y a deux directions de p�riodicit�, on cr�e deux vortex image.

%=============================================
\subsection{Composante de vitesse principale}
%=============================================
La m�thode des vortex ne g�n�rant pas de fluctuation $u$ de la vitesse dans la
direction principale, la derni�re composante est calcul�e � partir d'une
�quation de Langevin. Les coefficients de cette �quation sont d�termin�s par
identification des expressions obtenues pour les contraintes de Reynolds en
$R_{ij}-\varepsilon$. Dans le cas d'un �coulement en canal plan, cette technique
conduit � l'�quation : 
\begin{equation}\notag
\displaystyle
\frac{du}{dt} = - \frac{C_1}{2T} u + \left(\frac{2}{3}C_2-1\right)\frac{\partial
U}{\partial y} v + \sqrt{C_0\varepsilon} dW_i 
\end{equation}

avec $\displaystyle T=\frac{k}{\varepsilon}$, $C_1 = 1,8$, $C_2 = 0,6$,
$C_0=\frac{14}{15}$, et $dW_i$ une variable al�toire Gaussienne de variance
$\sqrt{dt}$. 

En pratique, l'�quation de Langevin n'am�liore pas vraiment les r�sultats. Elle
n'est utilis�e que dans le cas de conduites circulaires. 

%                      Code_Saturne version 1.3
%                      ------------------------
%
%     This file is part of the Code_Saturne Kernel, element of the
%     Code_Saturne CFD tool.
%
%     Copyright (C) 1998-2007 EDF S.A., France
%
%     contact: saturne-support@edf.fr
%
%     The Code_Saturne Kernel is free software; you can redistribute it
%     and/or modify it under the terms of the GNU General Public License
%     as published by the Free Software Foundation; either version 2 of
%     the License, or (at your option) any later version.
%
%     The Code_Saturne Kernel is distributed in the hope that it will be
%     useful, but WITHOUT ANY WARRANTY; without even the implied warranty
%     of MERCHANTABILITY or FITNESS FOR A PARTICULAR PURPOSE.  See the
%     GNU General Public License for more details.
%
%     You should have received a copy of the GNU General Public License
%     along with the Code_Saturne Kernel; if not, write to the
%     Free Software Foundation, Inc.,
%     51 Franklin St, Fifth Floor,
%     Boston, MA  02110-1301  USA
%
%-----------------------------------------------------------------------
%

%%%%%%%%%%%%%%%%%%%%%%%%%%%%%%%%%%
%%%%%%%%%%%%%%%%%%%%%%%%%%%%%%%%%%
\section{Mise en \oe uvre}
%%%%%%%%%%%%%%%%%%%%%%%%%%%%%%%%%%
%%%%%%%%%%%%%%%%%%%%%%%%%%%%%%%%%%
Le syst\`eme (\ref{Cfbl_Cfmsvl_eq_densite_finale_cfmsvl}) est r\'esolu par une m\'ethode
d'incr\'ement et r\'esidu en utilisant
une m\'ethode de Jacobi pour inverser le syst\`eme si le terme convectif
est implicite et en utilisant une m\'ethode de gradient conjugu\'e
si le terme convectif est explicite (qui est le cas par d�faut).

Attention, les valeurs du flux de masse $\rho\,\vect{w}\cdot\vect{S}$ et
de la viscosit\'e $\Delta\,t\,c^2\frac{S}{d}$ aux faces de
bord, qui sont calcul\'ees dans \fort{cfmsfl} et \fort{cfmsvs} respectivement,
sont modifi\'ees imm\'ediatement apr\`es l'appel \`a ces sous-programmes.
En effet, il est indispensable que la contribution de bord de
$\left(\rho\,\vect{w}-\Delta\,t\,(c^2)\,\gradv\,\rho\right)\cdot\vect{S}$
repr\'esente exactement $\vect{Q}_{ac}\cdot\vect{S}$.
Pour cela,
\begin{itemize}
\item imm\'ediatement apr\`es l'appel \`a
\fort{cfmsfl}, on remplace la contribution de bord de
$\rho\,\vect{w}\cdot\vect{S}$
par le flux de masse exact, $\vect{Q}_{ac}\cdot\vect{S}$,
d\'etermin\'e \`a partir des conditions aux limites,
\item puis, imm\'ediatement apr\`es l'appel \`a
\fort{cfmsvs}, on annule la viscosit\'e au bord $\Delta\,t\,(c^2)$ pour
\'eliminer la contribution de $-\Delta\,t\,(c^2)\,(\gradv\,\rho)\cdot\vect{S}$
(l'annulation de la viscosit\'e n'est pas probl\'ematique pour la matrice,
puisqu'elle porte sur des incr\'ements).
\end{itemize}

\bigskip

Une fois qu'on a obtenu $\rho^{n+1}$,
on peut actualiser le flux de masse acoustique
aux faces $(\vect{Q}_{ac}^{n+1})_{ij} \cdot \vect{S}_{ij}$,
qui servira pour la convection des autres variables~:
\begin{equation}\label{Cfbl_Cfmsvl_eq_flux_masse_acoustique_cfmsvl}
\displaystyle(\vect{Q}_{ac}^{n+1})_{ij}\cdot\vect{S}_{ij}=
-\left(\Delta t^n (c^2)^n \gradv(\rho^{n+1})\right)_{ij}\cdot\vect{S}_{ij}
+\left(\rho^{n+\frac{1}{2}} \vect{w}^n\right)_{ij}\cdot\vect{S}_{ij}\\
\end{equation}
Ce calcul de flux est r\'ealis\'e par \fort{cfbsc3}.
Si l'on a choisi l'algorithme standard, \'equation~(\ref{Cfbl_Cfmsvl_eq_densite_cfmsvl}),
on compl\`ete le flux dans \fort{cfmsvl} imm\'ediatement apr\`es l'appel
\`a \fort{cfbsc3}.
En effet, dans ce cas,
la convection est explicite ($\rho^{n+\frac{1}{2}}=\rho^{n}$,
obtenu en imposant \var{ICONV(ISCA(IRHO(IPHAS)))=0})
et le sous-programme \fort{cfbsc3},
qui calcule le flux de masse aux faces,
ne prend pas en compte la contribution du terme
$\rho^{n+\frac{1}{2}}\,\vect{w}^n\cdot\vect{S}$. On ajoute donc cette
contribution dans \fort{cfmsvl}, apr\`es l'appel \`a \fort{cfbsc3}.
Au bord, en particulier, c'est bien le flux de masse calcul\'e \`a partir
des conditions aux limites que l'on obtient.

On actualise la pression \`a la fin de l'\'etape, en utilisant la loi d'\'etat~:
\begin{equation}
\displaystyle P_i^{pred}=P(\rho_i^{n+1},\varepsilon_i^{n})
\end{equation}


%%%%%%%%%%%%%%%%%%%%%%%%%%%%%%%%%%
%%%%%%%%%%%%%%%%%%%%%%%%%%%%%%%%%%
\section{Points \`a traiter}
%%%%%%%%%%%%%%%%%%%%%%%%%%%%%%%%%%
%%%%%%%%%%%%%%%%%%%%%%%%%%%%%%%%%%
Le calcul du flux de masse au  bord n'est pas enti\`erement satisfaisant
si la convection est trait\'ee de mani\`ere implicite
(algorithme non standard, non test\'e,
associ\'e \`a l'\'equation~(\ref{Cfbl_Cfmsvl_eq_densite_bis_cfmsvl}),
correspondant au choix $\rho^{n+\frac{1}{2}}=\rho^{n+1}$ et
obtenu en imposant \var{ICONV(ISCA(IRHO(IPHAS)))=1}).
En effet, apr\`es \fort{cfmsfl}, il faut d\'eterminer la vitesse de
convection $\vect{w}^n$ pour qu'apparaisse
$\rho^{n+1} \vect{w}^n\cdot\vect{n}$
au cours de la r\'esolution par \fort{codits}. De ce fait, on doit d\'eduire
une valeur de $\vect{w}^n$ \`a partir de la valeur
du flux de masse. Au bord, en particulier, il faut
donc diviser le flux de masse
issu des conditions aux limites par la valeur de bord de $\rho^{n+1}$.
Or, lorsque des conditions de Neumann sont appliqu\'ees \`a la
masse volumique,
la valeur de $\rho^{n+1}$ au bord n'est pas connue avant la r\'esolution du
syst\`eme. On utilise donc, au lieu de la valeur de bord inconnue de
$\rho^{n+1}$ la valeur de bord prise au pas de temps
pr\'ec\'edent $\rho^{n}$. Cette approximation est susceptible
d'affecter la valeur du flux de masse au bord.

%                      Code_Saturne version 1.3
%                      ------------------------
%
%     This file is part of the Code_Saturne Kernel, element of the
%     Code_Saturne CFD tool.
%
%     Copyright (C) 1998-2007 EDF S.A., France
%
%     contact: saturne-support@edf.fr
%
%     The Code_Saturne Kernel is free software; you can redistribute it
%     and/or modify it under the terms of the GNU General Public License
%     as published by the Free Software Foundation; either version 2 of
%     the License, or (at your option) any later version.
%
%     The Code_Saturne Kernel is distributed in the hope that it will be
%     useful, but WITHOUT ANY WARRANTY; without even the implied warranty
%     of MERCHANTABILITY or FITNESS FOR A PARTICULAR PURPOSE.  See the
%     GNU General Public License for more details.
%
%     You should have received a copy of the GNU General Public License
%     along with the Code_Saturne Kernel; if not, write to the
%     Free Software Foundation, Inc.,
%     51 Franklin St, Fifth Floor,
%     Boston, MA  02110-1301  USA
%
%-----------------------------------------------------------------------
%


\programme{navsto}

\vspace{1cm}
On s'int\'eresse \`a la r\'esolution du syst\`eme d'\'equations de Navier-Stokes
tridimensionnelles monophasiques, \`a une pression, instationnaires, en
incompressible ou faiblement dilatable, bas\'ees sur une discr\'etisation
temporelle de type Euler implicite d'ordre 1 ou Crank-Nicolson d'ordre 2 et sur
une discr\'etisation spatiale  par volumes finis colocalis\'ee.


%%%%%%%%%%%%%%%%%%%%%%%%%%%%%%%%%%
%%%%%%%%%%%%%%%%%%%%%%%%%%%%%%%%%%
\section{Fonction}
%%%%%%%%%%%%%%%%%%%%%%%%%%%%%%%%%%
%%%%%%%%%%%%%%%%%%%%%%%%%%%%%%%%%%

  Dans ce sous-programme sont calcul\'ees, \`a un pas de temps donn\'e, les
variables vitesse et pression de ce probl\`eme en proc\'edant en
deux  \'etapes issues d'une d\'ecomposition des op\'erateurs (m\'ethode \`a
pas fractionnaires).\\
Les variables sont donc suppos\'ees connues \`a
l'instant ${t^n}$ et on cherche \`a les d\'eterminer \`a l'instant\footnote{La pression est suppos�e connue � l'instant $t^{n-1+\theta}$ et recherch�e en $t^{n+\theta}$, avec $\theta=1$ ou $1/2$ suivant le sch�ma en temps consid�r�.} ${t^{n+1}}$. Soit ${\Delta {t^n} ={t^{n+1}- {t^n}}}$ le pas de temps associ\'e. Dans un premier temps, on r\'ealise l'\'etape de
pr\'ediction de la vitesse en r\'esolvant l'\'equation de quantit\'e de
mouvement avec une pression explicite. Suit l'\'etape de correction de la
pression (ou projection de la vitesse) qui permet d'obtenir un champ de vitesse \`a divergence nulle.\\\\
Les \'equations en continu sont donc :
\begin{equation}
\left\{\begin{array}{l}
\displaystyle\frac{\partial}{\partial t}(\rho \vect{u}) + \dive(\rho\, \vect{u} \otimes \vect{u})
=\dive(\tens{\sigma}) + \vect{TS} - \tens{K}\,\vect{u}\\
\dive(\rho \vect{u}) = \Gamma
\end{array}\right.
\end{equation}

%(plus tard $\frac{\partial \rho}{\partial t} + \dive(\rho \vect{u}) = \Gamma$)



avec $\rho$ la masse volumique, $\vect{u}$ le champ de vitesse,
$[\,\vect{TS}-\tens{K}\,\vect{u}\,]$ les autres termes sources ($\tens{K}$~est un
tenseur diagonal positif par d\'efinition), $\tens{\sigma}$ le tenseur
de contraintes, $\tens{\tau}$ le tenseur des contraintes visqueuses, $\mu$ la
viscosit\'e dynamique (mol\'eculaire et \'eventuellement turbulente), $\kappa$
la viscosit� de
volume (usuellement nulle et n�glig�e dans le code et dans la suite du document,
sauf en compressible),
$\tens{D}$ le tenseur taux de d\'eformation\footnote{\`A ne pas confondre, malgr\'e la m\^eme notation $D$,
avec les flux diffusifs $\vect{D}_{\,ij}$ et $\vect{D}_{\,{b}_{ik}}$ d\'ecrits par la suite dans ce
sous-programme.}, $\Gamma$ le terme source de masse.
\begin{equation}
\left\{\begin{array}{l}
\tens{\sigma} = \tens{\tau} - P\tens{Id}  \\
\tens{\tau} = 2\,\mu\ \tens{D} +\ (\kappa\ - \frac{2}{3}\mu)\  tr({\tens{D}})\
\tens{Id}  \\
\tens{D} = \frac{1}{2}(\ggrad\vect{u}+\,^{t}\ggrad\vect{u})
\end{array}\right.
\end{equation}
 \\

On rappelle la d\'efinition des notations employ\'ees\footnote{en
utilisant la convention de sommation d'Einstein.}~:
\begin{equation}\notag
\left\{\begin{array}{lll}
\left[\ggrad{\vect{a}}\right]_{ij} &=& \partial_j a_i\\
\left[\dive(\tens{\sigma})\right]_i &=& \partial_j \sigma_{ij}\\
\left[\vect{a}\otimes\vect{b}\right]_{ij} &= &
a_i\,b_j\\
\end{array}\right.
\end{equation}
et donc :
\begin{equation}\notag
\begin{array}{lll}
\left[\dive(\vect{a}\otimes\vect{b})\right]_i &= &
\partial_j (a_i\,b_j)
\end{array}
\end{equation}

\minititre{Remarque}
Dans le cas de la prise en compte d'une masse volumique variable, l'�quation de continuit� s'�crit :
$$\frac{\partial \rho}{\partial t} + \dive{\,(\rho\,\vect{u})} = \Gamma  $$
Cette �quation n'est pas prise en compte dans l'�tape de projection (on continue � r�soudre
seulement
$\displaystyle \dive(\,{\rho\,\vect{u}}) = \Gamma$) alors que le terme
$\displaystyle \frac{\partial \rho}{\partial t}$ appara\^{\i}t lors de l'�tape de pr\'ediction de la vitesse
dans le sous-programme \fort{preduv}. Si ce terme joue un r�le sensible, l'algorithme compressible
de \CS\ (qui r�sout l'�quation compl�te) est alors sans doute plus adapt�.

%                      Code_Saturne version 1.3
%                      ------------------------
%
%     This file is part of the Code_Saturne Kernel, element of the
%     Code_Saturne CFD tool.
% 
%     Copyright (C) 1998-2007 EDF S.A., France
%
%     contact: saturne-support@edf.fr
% 
%     The Code_Saturne Kernel is free software; you can redistribute it
%     and/or modify it under the terms of the GNU General Public License
%     as published by the Free Software Foundation; either version 2 of
%     the License, or (at your option) any later version.
% 
%     The Code_Saturne Kernel is distributed in the hope that it will be
%     useful, but WITHOUT ANY WARRANTY; without even the implied warranty
%     of MERCHANTABILITY or FITNESS FOR A PARTICULAR PURPOSE.  See the
%     GNU General Public License for more details.
% 
%     You should have received a copy of the GNU General Public License
%     along with the Code_Saturne Kernel; if not, write to the
%     Free Software Foundation, Inc.,
%     51 Franklin St, Fifth Floor,
%     Boston, MA  02110-1301  USA
%
%-----------------------------------------------------------------------
%
%%%%%%%%%%%%%%%%%%%%%%%%%%%%%%%%%
%%%%%%%%%%%%%%%%%%%%%%%%%%%%%%%%%%
\section{Discr\'etisation}
%%%%%%%%%%%%%%%%%%%%%%%%%%%%%%%%%%
%%%%%%%%%%%%%%%%%%%%%%%%%%%%%%%%%%

Pour utiliser la m�thode, on se place tout d'abord dans un rep�re local d�fini
de mani�re � ce que le plan $(0yz)$, o� sont inject�s les vortex, soit confondu
avec le plan d'entr�e du calcul (voir figure \ref{Base_Vortex_entree}). 

\begin{figure}[h]
\centerline{\includegraphics[height=6cm]{../Base/Vortex/Images/entree.pdf}}
\caption{\label{Base_Vortex_entree} D�finiton des diff�rentes grandeurs dans le rep�re local
correspondant � l'entr�e d'une conduite de section carr�e.} 
\end{figure}

$u$, $v$ et $w$  sont les composantes de la vitesse fluctuante (principale et
transverse) dans ce plan, et
$\displaystyle \omega(y,z) = \frac{\partial w}{\partial y}-\frac{\partial v}{\partial z}$
la vorticit� dans la direction
normale au plan d'entr�e. $\overline{U}(y,z)$ repr�sente ici la vitesse
principale moyenne impos�e par l'utilisateur dans le plan d'entr�e. 

Chaque vortex $p$ va �tre caract�ris� par sa fonction de forme $\xi$ (identique
pour tous les vortex), sa
circulation $\Gamma_p$, son rayon $\sigma_p$ et les coordonn�es $(y_p,z_p)$ du
point $P$ o� est situ� le vortex dans le plan $(0yz)$. 

Pour cela, on suppose que la vorticit� g�n�r�e par un vortex $p$ au point $M$ de
coordonn�e $(y,z)$ s'�crit : 
\begin{equation}\notag
\omega_p(y,z)= \Gamma_p \, \xi_{\sigma_p}(r)
\end{equation}
o� $r = \sqrt{(y-y_p)^2+(z-z_p)^2}$ est la distance s�parant le point $M$ du point $P$.

Dans la m�thode implant�e, la fonction de forme est de type gaussienne modifi�e :
\begin{equation}\notag
\displaystyle
\xi_\sigma (r) = \frac{1}{2\pi \sigma^2} 
\left(2 e^{-\frac{r^2}{2\sigma^2}}-1\right) e^{-\frac{r^2}{2\sigma^2}}
\end{equation}

Le champ de vitesse induit par cette distribution de vorticit� s'obtient par
inversion des deux �quations de poisson suivantes qui sont d�duites de la
condition d'incompressibilit� dans la plan\footnote{\textit{i.e}
$\displaystyle \frac{\partial v}{\partial y}+\frac{\partial w}{\partial w} = 0$} :
\begin{equation}\notag
\begin{array}{lcr}
\displaystyle
\frac{\partial \omega}{\partial y} = \Delta w
&
\text{    et    }
&
\displaystyle
\frac{\partial \omega}{\partial y} = -\Delta v
\\
\end{array}
\end{equation}

Dans le cas g�n�ral, ce syst�me peut �tre int�gr� � l'aide de la formule de Biot et Savart.

Dans le cas d'une distribution de vorticit� de type gaussienne modifi�e, les
composantes de vitesse v�rifient : 
\begin{equation}\notag
\left\{
\begin{array}{c}
\displaystyle
v_p(y,x) = - \frac{1}{2\pi} \frac{(z-z_p)}{r^2}\left(1 -
e^{-\frac{r^2}{2\sigma^2}}\right)\,e^{-\frac{r^2}{2\sigma^2}} 
\\
\displaystyle
w_p(y,z) = \frac{1}{2\pi} \frac{(y-y_p)}{r^2}\left(1 -e^{-\frac{r^2}{2\sigma^2}}
\right)\,e^{-\frac{r^2}{2\sigma^2}} 
\end{array}
\right.
\end{equation}

Ces relations s'�tendent de fa�on imm�diate au cas de $N$ vortex distincts.
Le champ de vitesse induit par la distribution de vorticit� 
\begin{equation}
\omega(y,z) = \sum_{p=1}^N \Gamma_p \, \xi_{\sigma_p}(r)
\end{equation}
vaut au point $M$ :
\begin{equation}\notag
\begin{array}{lcr}
v(x,y) = \sum_{p=1}^N \Gamma_p\, v_p(y,z) 
&
\text{    et    }
&
w(y,z) = \sum_{p=1}^N \Gamma_p\, w_p(y,z)
\\
\label{Base_Vortex_compvit}
\end{array}
\end{equation}
%================================
\subsection{Param�tres physiques}
%================================

%-------------------------------
\subsubsection{Marche en temps}
%-------------------------------
La position initiale de chaque vortex est tir�e de mani�re al�atoire. On calcul
les d�placements successifs de chacun des vortex dans le plan d'entr�e par
int�gration explicite du champ de vitesse lagrangien : 
\begin{equation}\notag
\begin{array}{lcr}
\displaystyle
\frac{dy_p}{dt} = V(y,z)
&
\text{    et    }
&
\displaystyle
\frac{dz_p}{dt} = W(y,z)
\\
\end{array}
\end{equation}
Les vortex sont alors assimil�s � des particules ponctuelles qui sont convect�es
par le champ $(V,W)$. Ce champ peut �tre impos� par des tirages al�atoires ou
bien d�duit de la vitesse induite par les vortex dans le plan d'entr�e. Dans ce
cas $V(x,y) = \overline{V}(y,z) + v (y,z)$ et $W(y,z)= \overline{W}(y,z) +
w(y,z)$ o� $\overline{V}$ et $\overline{W}$ sont les composantes de la vitesse
transverse moyenne qu'impose l'utilisateur � l'aide des fichiers de donn�es. 

%---------------------------------------------------
\subsubsection{Intensit� et dur�e de vie des vortex}
%---------------------------------------------------
Il serait possible, � partir de l'�quation de transport de la vorticit�,
d'obtenir un mod�le d'�volution pour l'intensit� du vecteur tourbillon
$\omega_p$ associ� � chacun des vortex. En pratique, on pr�f�re utiliser un
mod�le simplifi� dans lequel la circulation des tourbillons ne d�pend que de la
postion de ces derniers � l'instant consid�r�. La circulation initiale de chaque
vortex est alors obtenue � partir de la relation : 
\begin{equation}\notag
|\Gamma_p| = 4 \sqrt{\frac{\pi\,S\,k}{3N\,[2ln(3)-3ln(2)]}}
\end{equation}
o� $S$ est la surface du plan d'entr�e, $N$ le nombre de vortex, et $k$
l'�nergie cin�tique turbulente au point o� se trouve le vortex � l'instant
consid�r�. Le signe de $\Gamma_p$ correspond au sens de rotation du vortex et
est tir� al�atoirement. 

Ce param�tre est celui qui contr�le l'intensit� des fluctuations. Sa d�pendance
en $k$ exprime que, plus l'�coulement est turbulent, plus les vortex sont
intenses. La valeur de $k$ est sp�cifi�e par
l'utilisateur. Elle peut �tre constante ou impos�e � partir de profils d'�nergie
cin�tique turbulente en entr�e. 

Pour �viter que des structures trop allong�es ne se d�veloppent au niveau de
l'entr�e, l'utilisateur doit �galement sp�cifier un temps limites $\tau_p$ au
bout duquel le vortex $p$ va �tre d�truit. Ce temps $\tau_p$ peut �tre pris
constant ou estim� au moyen de la relation : 
\begin{equation}\notag
\tau_p = \frac{5 C_{\mu}k^{\frac{3}{2}}}{\varepsilon\,\overline{U}}
\end{equation}

$\overline{U}$ et $\varepsilon$ repr�sentent respectivement la vitesse moyenne
principale et la dissipation turbulente au point o� est initialement g�n�r� le
vortex ($C_{\mu}=0,09$). 
\\
Lorsque le temps �coul� depuis la cr�ation du vortex $p$ est sup�rieur �
$\tau_p$, le vortex est d�truit et un nouveau vortex g�n�r� (sa position et le
signe de $\Gamma_p$ sont tir�s de fa�on al�atoire). 

%-------------------------------- 
\subsubsection{Taille des vortex}
%--------------------------------
La taille des vortex peut �tre prise constante, ou calcul�e � partir des
relations :
\begin{equation}\notag
\begin{array}{ccc}
\displaystyle
\sigma = \frac{C_{\mu}^{\frac{3}{4}}k^{\frac{3}{2}}}{\varepsilon} 
& \text{    ou    } &
\sigma = max[L_t,L_k]
\\
\end{array}
\end{equation}
avec:
\begin{equation}\notag
\begin{array}{ccc}
\displaystyle
L_t = \sqrt{\left( \frac{5 \nu k}{\varepsilon} \right)} 
& \text{    et    } & 
\displaystyle
L_k = 200\, \left(\frac{\nu^3}{\varepsilon}\right)^{\frac{1}{4}}
\end{array}
\end{equation}
o� $\nu$, $k$ et $\varepsilon$ sont la viscosit� dynamique, l'�nergie cin�tique
turbulente et la dissipation turbulente au point o� se trouve le vortex. 

Dans tous les cas, la taille du vortex doit �tre sup�rieure � la taille des
mailles en entr�e afin que le vortex soit effectivement simul�. 

%==================================
\subsection{Conditions aux limites}
%==================================
Le champ de vitesse g�n�r� � l'aide de la relation \ref{Base_Vortex_compvit} ne tient pas
compte {\em a priori} des conditions aux limites appliqu�es sur les bords du plan
d'entr�e. Pour obtenir des valeurs de la vitesse qui soient coh�rentes sur les
fronti�res du domaine d'entr�e, des ``vortex images'', pseudo-vortex situ�s en
dehors du domaine d'entr�e, sont g�n�r�s � des positions particuli�res et leur
champ de vitesse associ� est superpos� au champ pr�c�demment calcul�.\\
Seuls les cas d'une conduite rectangulaire et d'une conduite circulaire
permettent la g�n�ration de ces pseudo-vortex.
On distingue pour cela trois types de conditions aux limites. 

\begin{figure}[h]
\centerline{\includegraphics[height=6cm]{../Base/Vortex/Images/condlimite.pdf}}
\caption{\label{Base_Vortex_condli} Principe de g�n�ration des ``vortex images'' suivant le
type de conditions aux limites dans une conduite carr�e.} 
\end{figure}

%----------------------------------
\subsubsection{Condition de paroi}
%----------------------------------
On cr�e, pour chaque vortex $P$ contenu dans le plan d'entr�e, un vortex image
$P'$ identique � $P$ (\textit{i.e} de m�me caract�ristiques) et sym�trique de $P$
par rapport au
point $J$ ($J$ �tant la projection orthogonalement � la paroi du point $M$
correspondant au centre de la face o� l'on cherche � calculer la vitesse). La
figure \ref{Base_Vortex_condli} illustre la technique dans le cas d'une conduite
carr�e. Dans ce cas les coordonn�es du vortex situ� en $P'$ v�rifient
$(y_{p'}+y_{p})/2 = y_{J}$ et $(z_{p'}+ z_{p})/2 = z_{J}$. Le champ de vitesse
per�u depuis le point $M$ au niveau du point $J$ est nul, ce qui est bien
l'effet recherch�. 

%------------------------------------
\subsubsection{Condition de sym�trie}
%-------------------------------------
La technique est identique � celle utilis�e pour les conditions de paroi, mais
seule la composante pour la vitesse normale au bord est modifi�e dans ce cas. 

%---------------------------------------
\subsubsection{Condition de p�riodicit�}
%---------------------------------------
On cr�e pour chaque vortex, un vortex images $P'$ identique � $P$ mais translat�
d'une quantit� $L$ correspondant � la longueur qui s�pare les deux plans de la
section d'entr�e o� sont appliqu�es les conditions de p�riodicit�. Dans le cas
o� il y a deux directions de p�riodicit�, on cr�e deux vortex image.

%=============================================
\subsection{Composante de vitesse principale}
%=============================================
La m�thode des vortex ne g�n�rant pas de fluctuation $u$ de la vitesse dans la
direction principale, la derni�re composante est calcul�e � partir d'une
�quation de Langevin. Les coefficients de cette �quation sont d�termin�s par
identification des expressions obtenues pour les contraintes de Reynolds en
$R_{ij}-\varepsilon$. Dans le cas d'un �coulement en canal plan, cette technique
conduit � l'�quation : 
\begin{equation}\notag
\displaystyle
\frac{du}{dt} = - \frac{C_1}{2T} u + \left(\frac{2}{3}C_2-1\right)\frac{\partial
U}{\partial y} v + \sqrt{C_0\varepsilon} dW_i 
\end{equation}

avec $\displaystyle T=\frac{k}{\varepsilon}$, $C_1 = 1,8$, $C_2 = 0,6$,
$C_0=\frac{14}{15}$, et $dW_i$ une variable al�toire Gaussienne de variance
$\sqrt{dt}$. 

En pratique, l'�quation de Langevin n'am�liore pas vraiment les r�sultats. Elle
n'est utilis�e que dans le cas de conduites circulaires. 

%                      Code_Saturne version 1.3
%                      ------------------------
%
%     This file is part of the Code_Saturne Kernel, element of the
%     Code_Saturne CFD tool.
%
%     Copyright (C) 1998-2007 EDF S.A., France
%
%     contact: saturne-support@edf.fr
%
%     The Code_Saturne Kernel is free software; you can redistribute it
%     and/or modify it under the terms of the GNU General Public License
%     as published by the Free Software Foundation; either version 2 of
%     the License, or (at your option) any later version.
%
%     The Code_Saturne Kernel is distributed in the hope that it will be
%     useful, but WITHOUT ANY WARRANTY; without even the implied warranty
%     of MERCHANTABILITY or FITNESS FOR A PARTICULAR PURPOSE.  See the
%     GNU General Public License for more details.
%
%     You should have received a copy of the GNU General Public License
%     along with the Code_Saturne Kernel; if not, write to the
%     Free Software Foundation, Inc.,
%     51 Franklin St, Fifth Floor,
%     Boston, MA  02110-1301  USA
%
%-----------------------------------------------------------------------
%

%%%%%%%%%%%%%%%%%%%%%%%%%%%%%%%%%%
%%%%%%%%%%%%%%%%%%%%%%%%%%%%%%%%%%
\section{Mise en \oe uvre}
%%%%%%%%%%%%%%%%%%%%%%%%%%%%%%%%%%
%%%%%%%%%%%%%%%%%%%%%%%%%%%%%%%%%%
Le syst\`eme (\ref{Cfbl_Cfmsvl_eq_densite_finale_cfmsvl}) est r\'esolu par une m\'ethode
d'incr\'ement et r\'esidu en utilisant
une m\'ethode de Jacobi pour inverser le syst\`eme si le terme convectif
est implicite et en utilisant une m\'ethode de gradient conjugu\'e
si le terme convectif est explicite (qui est le cas par d�faut).

Attention, les valeurs du flux de masse $\rho\,\vect{w}\cdot\vect{S}$ et
de la viscosit\'e $\Delta\,t\,c^2\frac{S}{d}$ aux faces de
bord, qui sont calcul\'ees dans \fort{cfmsfl} et \fort{cfmsvs} respectivement,
sont modifi\'ees imm\'ediatement apr\`es l'appel \`a ces sous-programmes.
En effet, il est indispensable que la contribution de bord de
$\left(\rho\,\vect{w}-\Delta\,t\,(c^2)\,\gradv\,\rho\right)\cdot\vect{S}$
repr\'esente exactement $\vect{Q}_{ac}\cdot\vect{S}$.
Pour cela,
\begin{itemize}
\item imm\'ediatement apr\`es l'appel \`a
\fort{cfmsfl}, on remplace la contribution de bord de
$\rho\,\vect{w}\cdot\vect{S}$
par le flux de masse exact, $\vect{Q}_{ac}\cdot\vect{S}$,
d\'etermin\'e \`a partir des conditions aux limites,
\item puis, imm\'ediatement apr\`es l'appel \`a
\fort{cfmsvs}, on annule la viscosit\'e au bord $\Delta\,t\,(c^2)$ pour
\'eliminer la contribution de $-\Delta\,t\,(c^2)\,(\gradv\,\rho)\cdot\vect{S}$
(l'annulation de la viscosit\'e n'est pas probl\'ematique pour la matrice,
puisqu'elle porte sur des incr\'ements).
\end{itemize}

\bigskip

Une fois qu'on a obtenu $\rho^{n+1}$,
on peut actualiser le flux de masse acoustique
aux faces $(\vect{Q}_{ac}^{n+1})_{ij} \cdot \vect{S}_{ij}$,
qui servira pour la convection des autres variables~:
\begin{equation}\label{Cfbl_Cfmsvl_eq_flux_masse_acoustique_cfmsvl}
\displaystyle(\vect{Q}_{ac}^{n+1})_{ij}\cdot\vect{S}_{ij}=
-\left(\Delta t^n (c^2)^n \gradv(\rho^{n+1})\right)_{ij}\cdot\vect{S}_{ij}
+\left(\rho^{n+\frac{1}{2}} \vect{w}^n\right)_{ij}\cdot\vect{S}_{ij}\\
\end{equation}
Ce calcul de flux est r\'ealis\'e par \fort{cfbsc3}.
Si l'on a choisi l'algorithme standard, \'equation~(\ref{Cfbl_Cfmsvl_eq_densite_cfmsvl}),
on compl\`ete le flux dans \fort{cfmsvl} imm\'ediatement apr\`es l'appel
\`a \fort{cfbsc3}.
En effet, dans ce cas,
la convection est explicite ($\rho^{n+\frac{1}{2}}=\rho^{n}$,
obtenu en imposant \var{ICONV(ISCA(IRHO(IPHAS)))=0})
et le sous-programme \fort{cfbsc3},
qui calcule le flux de masse aux faces,
ne prend pas en compte la contribution du terme
$\rho^{n+\frac{1}{2}}\,\vect{w}^n\cdot\vect{S}$. On ajoute donc cette
contribution dans \fort{cfmsvl}, apr\`es l'appel \`a \fort{cfbsc3}.
Au bord, en particulier, c'est bien le flux de masse calcul\'e \`a partir
des conditions aux limites que l'on obtient.

On actualise la pression \`a la fin de l'\'etape, en utilisant la loi d'\'etat~:
\begin{equation}
\displaystyle P_i^{pred}=P(\rho_i^{n+1},\varepsilon_i^{n})
\end{equation}


%%%%%%%%%%%%%%%%%%%%%%%%%%%%%%%%%%
%%%%%%%%%%%%%%%%%%%%%%%%%%%%%%%%%%
\section{Points \`a traiter}
%%%%%%%%%%%%%%%%%%%%%%%%%%%%%%%%%%
%%%%%%%%%%%%%%%%%%%%%%%%%%%%%%%%%%
Le calcul du flux de masse au  bord n'est pas enti\`erement satisfaisant
si la convection est trait\'ee de mani\`ere implicite
(algorithme non standard, non test\'e,
associ\'e \`a l'\'equation~(\ref{Cfbl_Cfmsvl_eq_densite_bis_cfmsvl}),
correspondant au choix $\rho^{n+\frac{1}{2}}=\rho^{n+1}$ et
obtenu en imposant \var{ICONV(ISCA(IRHO(IPHAS)))=1}).
En effet, apr\`es \fort{cfmsfl}, il faut d\'eterminer la vitesse de
convection $\vect{w}^n$ pour qu'apparaisse
$\rho^{n+1} \vect{w}^n\cdot\vect{n}$
au cours de la r\'esolution par \fort{codits}. De ce fait, on doit d\'eduire
une valeur de $\vect{w}^n$ \`a partir de la valeur
du flux de masse. Au bord, en particulier, il faut
donc diviser le flux de masse
issu des conditions aux limites par la valeur de bord de $\rho^{n+1}$.
Or, lorsque des conditions de Neumann sont appliqu\'ees \`a la
masse volumique,
la valeur de $\rho^{n+1}$ au bord n'est pas connue avant la r\'esolution du
syst\`eme. On utilise donc, au lieu de la valeur de bord inconnue de
$\rho^{n+1}$ la valeur de bord prise au pas de temps
pr\'ec\'edent $\rho^{n}$. Cette approximation est susceptible
d'affecter la valeur du flux de masse au bord.

%                      Code_Saturne version 1.3
%                      ------------------------
%
%     This file is part of the Code_Saturne Kernel, element of the
%     Code_Saturne CFD tool.
%
%     Copyright (C) 1998-2007 EDF S.A., France
%
%     contact: saturne-support@edf.fr
%
%     The Code_Saturne Kernel is free software; you can redistribute it
%     and/or modify it under the terms of the GNU General Public License
%     as published by the Free Software Foundation; either version 2 of
%     the License, or (at your option) any later version.
%
%     The Code_Saturne Kernel is distributed in the hope that it will be
%     useful, but WITHOUT ANY WARRANTY; without even the implied warranty
%     of MERCHANTABILITY or FITNESS FOR A PARTICULAR PURPOSE.  See the
%     GNU General Public License for more details.
%
%     You should have received a copy of the GNU General Public License
%     along with the Code_Saturne Kernel; if not, write to the
%     Free Software Foundation, Inc.,
%     51 Franklin St, Fifth Floor,
%     Boston, MA  02110-1301  USA
%
%-----------------------------------------------------------------------
%


\programme{navsto}

\vspace{1cm}
On s'int\'eresse \`a la r\'esolution du syst\`eme d'\'equations de Navier-Stokes
tridimensionnelles monophasiques, \`a une pression, instationnaires, en
incompressible ou faiblement dilatable, bas\'ees sur une discr\'etisation
temporelle de type Euler implicite d'ordre 1 ou Crank-Nicolson d'ordre 2 et sur
une discr\'etisation spatiale  par volumes finis colocalis\'ee.


%%%%%%%%%%%%%%%%%%%%%%%%%%%%%%%%%%
%%%%%%%%%%%%%%%%%%%%%%%%%%%%%%%%%%
\section{Fonction}
%%%%%%%%%%%%%%%%%%%%%%%%%%%%%%%%%%
%%%%%%%%%%%%%%%%%%%%%%%%%%%%%%%%%%

  Dans ce sous-programme sont calcul\'ees, \`a un pas de temps donn\'e, les
variables vitesse et pression de ce probl\`eme en proc\'edant en
deux  \'etapes issues d'une d\'ecomposition des op\'erateurs (m\'ethode \`a
pas fractionnaires).\\
Les variables sont donc suppos\'ees connues \`a
l'instant ${t^n}$ et on cherche \`a les d\'eterminer \`a l'instant\footnote{La pression est suppos�e connue � l'instant $t^{n-1+\theta}$ et recherch�e en $t^{n+\theta}$, avec $\theta=1$ ou $1/2$ suivant le sch�ma en temps consid�r�.} ${t^{n+1}}$. Soit ${\Delta {t^n} ={t^{n+1}- {t^n}}}$ le pas de temps associ\'e. Dans un premier temps, on r\'ealise l'\'etape de
pr\'ediction de la vitesse en r\'esolvant l'\'equation de quantit\'e de
mouvement avec une pression explicite. Suit l'\'etape de correction de la
pression (ou projection de la vitesse) qui permet d'obtenir un champ de vitesse \`a divergence nulle.\\\\
Les \'equations en continu sont donc :
\begin{equation}
\left\{\begin{array}{l}
\displaystyle\frac{\partial}{\partial t}(\rho \vect{u}) + \dive(\rho\, \vect{u} \otimes \vect{u})
=\dive(\tens{\sigma}) + \vect{TS} - \tens{K}\,\vect{u}\\
\dive(\rho \vect{u}) = \Gamma
\end{array}\right.
\end{equation}

%(plus tard $\frac{\partial \rho}{\partial t} + \dive(\rho \vect{u}) = \Gamma$)



avec $\rho$ la masse volumique, $\vect{u}$ le champ de vitesse,
$[\,\vect{TS}-\tens{K}\,\vect{u}\,]$ les autres termes sources ($\tens{K}$~est un
tenseur diagonal positif par d\'efinition), $\tens{\sigma}$ le tenseur
de contraintes, $\tens{\tau}$ le tenseur des contraintes visqueuses, $\mu$ la
viscosit\'e dynamique (mol\'eculaire et \'eventuellement turbulente), $\kappa$
la viscosit� de
volume (usuellement nulle et n�glig�e dans le code et dans la suite du document,
sauf en compressible),
$\tens{D}$ le tenseur taux de d\'eformation\footnote{\`A ne pas confondre, malgr\'e la m\^eme notation $D$,
avec les flux diffusifs $\vect{D}_{\,ij}$ et $\vect{D}_{\,{b}_{ik}}$ d\'ecrits par la suite dans ce
sous-programme.}, $\Gamma$ le terme source de masse.
\begin{equation}
\left\{\begin{array}{l}
\tens{\sigma} = \tens{\tau} - P\tens{Id}  \\
\tens{\tau} = 2\,\mu\ \tens{D} +\ (\kappa\ - \frac{2}{3}\mu)\  tr({\tens{D}})\
\tens{Id}  \\
\tens{D} = \frac{1}{2}(\ggrad\vect{u}+\,^{t}\ggrad\vect{u})
\end{array}\right.
\end{equation}
 \\

On rappelle la d\'efinition des notations employ\'ees\footnote{en
utilisant la convention de sommation d'Einstein.}~:
\begin{equation}\notag
\left\{\begin{array}{lll}
\left[\ggrad{\vect{a}}\right]_{ij} &=& \partial_j a_i\\
\left[\dive(\tens{\sigma})\right]_i &=& \partial_j \sigma_{ij}\\
\left[\vect{a}\otimes\vect{b}\right]_{ij} &= &
a_i\,b_j\\
\end{array}\right.
\end{equation}
et donc :
\begin{equation}\notag
\begin{array}{lll}
\left[\dive(\vect{a}\otimes\vect{b})\right]_i &= &
\partial_j (a_i\,b_j)
\end{array}
\end{equation}

\minititre{Remarque}
Dans le cas de la prise en compte d'une masse volumique variable, l'�quation de continuit� s'�crit :
$$\frac{\partial \rho}{\partial t} + \dive{\,(\rho\,\vect{u})} = \Gamma  $$
Cette �quation n'est pas prise en compte dans l'�tape de projection (on continue � r�soudre
seulement
$\displaystyle \dive(\,{\rho\,\vect{u}}) = \Gamma$) alors que le terme
$\displaystyle \frac{\partial \rho}{\partial t}$ appara\^{\i}t lors de l'�tape de pr\'ediction de la vitesse
dans le sous-programme \fort{preduv}. Si ce terme joue un r�le sensible, l'algorithme compressible
de \CS\ (qui r�sout l'�quation compl�te) est alors sans doute plus adapt�.

%                      Code_Saturne version 1.3
%                      ------------------------
%
%     This file is part of the Code_Saturne Kernel, element of the
%     Code_Saturne CFD tool.
% 
%     Copyright (C) 1998-2007 EDF S.A., France
%
%     contact: saturne-support@edf.fr
% 
%     The Code_Saturne Kernel is free software; you can redistribute it
%     and/or modify it under the terms of the GNU General Public License
%     as published by the Free Software Foundation; either version 2 of
%     the License, or (at your option) any later version.
% 
%     The Code_Saturne Kernel is distributed in the hope that it will be
%     useful, but WITHOUT ANY WARRANTY; without even the implied warranty
%     of MERCHANTABILITY or FITNESS FOR A PARTICULAR PURPOSE.  See the
%     GNU General Public License for more details.
% 
%     You should have received a copy of the GNU General Public License
%     along with the Code_Saturne Kernel; if not, write to the
%     Free Software Foundation, Inc.,
%     51 Franklin St, Fifth Floor,
%     Boston, MA  02110-1301  USA
%
%-----------------------------------------------------------------------
%
%%%%%%%%%%%%%%%%%%%%%%%%%%%%%%%%%
%%%%%%%%%%%%%%%%%%%%%%%%%%%%%%%%%%
\section{Discr\'etisation}
%%%%%%%%%%%%%%%%%%%%%%%%%%%%%%%%%%
%%%%%%%%%%%%%%%%%%%%%%%%%%%%%%%%%%

Pour utiliser la m�thode, on se place tout d'abord dans un rep�re local d�fini
de mani�re � ce que le plan $(0yz)$, o� sont inject�s les vortex, soit confondu
avec le plan d'entr�e du calcul (voir figure \ref{Base_Vortex_entree}). 

\begin{figure}[h]
\centerline{\includegraphics[height=6cm]{../Base/Vortex/Images/entree.pdf}}
\caption{\label{Base_Vortex_entree} D�finiton des diff�rentes grandeurs dans le rep�re local
correspondant � l'entr�e d'une conduite de section carr�e.} 
\end{figure}

$u$, $v$ et $w$  sont les composantes de la vitesse fluctuante (principale et
transverse) dans ce plan, et
$\displaystyle \omega(y,z) = \frac{\partial w}{\partial y}-\frac{\partial v}{\partial z}$
la vorticit� dans la direction
normale au plan d'entr�e. $\overline{U}(y,z)$ repr�sente ici la vitesse
principale moyenne impos�e par l'utilisateur dans le plan d'entr�e. 

Chaque vortex $p$ va �tre caract�ris� par sa fonction de forme $\xi$ (identique
pour tous les vortex), sa
circulation $\Gamma_p$, son rayon $\sigma_p$ et les coordonn�es $(y_p,z_p)$ du
point $P$ o� est situ� le vortex dans le plan $(0yz)$. 

Pour cela, on suppose que la vorticit� g�n�r�e par un vortex $p$ au point $M$ de
coordonn�e $(y,z)$ s'�crit : 
\begin{equation}\notag
\omega_p(y,z)= \Gamma_p \, \xi_{\sigma_p}(r)
\end{equation}
o� $r = \sqrt{(y-y_p)^2+(z-z_p)^2}$ est la distance s�parant le point $M$ du point $P$.

Dans la m�thode implant�e, la fonction de forme est de type gaussienne modifi�e :
\begin{equation}\notag
\displaystyle
\xi_\sigma (r) = \frac{1}{2\pi \sigma^2} 
\left(2 e^{-\frac{r^2}{2\sigma^2}}-1\right) e^{-\frac{r^2}{2\sigma^2}}
\end{equation}

Le champ de vitesse induit par cette distribution de vorticit� s'obtient par
inversion des deux �quations de poisson suivantes qui sont d�duites de la
condition d'incompressibilit� dans la plan\footnote{\textit{i.e}
$\displaystyle \frac{\partial v}{\partial y}+\frac{\partial w}{\partial w} = 0$} :
\begin{equation}\notag
\begin{array}{lcr}
\displaystyle
\frac{\partial \omega}{\partial y} = \Delta w
&
\text{    et    }
&
\displaystyle
\frac{\partial \omega}{\partial y} = -\Delta v
\\
\end{array}
\end{equation}

Dans le cas g�n�ral, ce syst�me peut �tre int�gr� � l'aide de la formule de Biot et Savart.

Dans le cas d'une distribution de vorticit� de type gaussienne modifi�e, les
composantes de vitesse v�rifient : 
\begin{equation}\notag
\left\{
\begin{array}{c}
\displaystyle
v_p(y,x) = - \frac{1}{2\pi} \frac{(z-z_p)}{r^2}\left(1 -
e^{-\frac{r^2}{2\sigma^2}}\right)\,e^{-\frac{r^2}{2\sigma^2}} 
\\
\displaystyle
w_p(y,z) = \frac{1}{2\pi} \frac{(y-y_p)}{r^2}\left(1 -e^{-\frac{r^2}{2\sigma^2}}
\right)\,e^{-\frac{r^2}{2\sigma^2}} 
\end{array}
\right.
\end{equation}

Ces relations s'�tendent de fa�on imm�diate au cas de $N$ vortex distincts.
Le champ de vitesse induit par la distribution de vorticit� 
\begin{equation}
\omega(y,z) = \sum_{p=1}^N \Gamma_p \, \xi_{\sigma_p}(r)
\end{equation}
vaut au point $M$ :
\begin{equation}\notag
\begin{array}{lcr}
v(x,y) = \sum_{p=1}^N \Gamma_p\, v_p(y,z) 
&
\text{    et    }
&
w(y,z) = \sum_{p=1}^N \Gamma_p\, w_p(y,z)
\\
\label{Base_Vortex_compvit}
\end{array}
\end{equation}
%================================
\subsection{Param�tres physiques}
%================================

%-------------------------------
\subsubsection{Marche en temps}
%-------------------------------
La position initiale de chaque vortex est tir�e de mani�re al�atoire. On calcul
les d�placements successifs de chacun des vortex dans le plan d'entr�e par
int�gration explicite du champ de vitesse lagrangien : 
\begin{equation}\notag
\begin{array}{lcr}
\displaystyle
\frac{dy_p}{dt} = V(y,z)
&
\text{    et    }
&
\displaystyle
\frac{dz_p}{dt} = W(y,z)
\\
\end{array}
\end{equation}
Les vortex sont alors assimil�s � des particules ponctuelles qui sont convect�es
par le champ $(V,W)$. Ce champ peut �tre impos� par des tirages al�atoires ou
bien d�duit de la vitesse induite par les vortex dans le plan d'entr�e. Dans ce
cas $V(x,y) = \overline{V}(y,z) + v (y,z)$ et $W(y,z)= \overline{W}(y,z) +
w(y,z)$ o� $\overline{V}$ et $\overline{W}$ sont les composantes de la vitesse
transverse moyenne qu'impose l'utilisateur � l'aide des fichiers de donn�es. 

%---------------------------------------------------
\subsubsection{Intensit� et dur�e de vie des vortex}
%---------------------------------------------------
Il serait possible, � partir de l'�quation de transport de la vorticit�,
d'obtenir un mod�le d'�volution pour l'intensit� du vecteur tourbillon
$\omega_p$ associ� � chacun des vortex. En pratique, on pr�f�re utiliser un
mod�le simplifi� dans lequel la circulation des tourbillons ne d�pend que de la
postion de ces derniers � l'instant consid�r�. La circulation initiale de chaque
vortex est alors obtenue � partir de la relation : 
\begin{equation}\notag
|\Gamma_p| = 4 \sqrt{\frac{\pi\,S\,k}{3N\,[2ln(3)-3ln(2)]}}
\end{equation}
o� $S$ est la surface du plan d'entr�e, $N$ le nombre de vortex, et $k$
l'�nergie cin�tique turbulente au point o� se trouve le vortex � l'instant
consid�r�. Le signe de $\Gamma_p$ correspond au sens de rotation du vortex et
est tir� al�atoirement. 

Ce param�tre est celui qui contr�le l'intensit� des fluctuations. Sa d�pendance
en $k$ exprime que, plus l'�coulement est turbulent, plus les vortex sont
intenses. La valeur de $k$ est sp�cifi�e par
l'utilisateur. Elle peut �tre constante ou impos�e � partir de profils d'�nergie
cin�tique turbulente en entr�e. 

Pour �viter que des structures trop allong�es ne se d�veloppent au niveau de
l'entr�e, l'utilisateur doit �galement sp�cifier un temps limites $\tau_p$ au
bout duquel le vortex $p$ va �tre d�truit. Ce temps $\tau_p$ peut �tre pris
constant ou estim� au moyen de la relation : 
\begin{equation}\notag
\tau_p = \frac{5 C_{\mu}k^{\frac{3}{2}}}{\varepsilon\,\overline{U}}
\end{equation}

$\overline{U}$ et $\varepsilon$ repr�sentent respectivement la vitesse moyenne
principale et la dissipation turbulente au point o� est initialement g�n�r� le
vortex ($C_{\mu}=0,09$). 
\\
Lorsque le temps �coul� depuis la cr�ation du vortex $p$ est sup�rieur �
$\tau_p$, le vortex est d�truit et un nouveau vortex g�n�r� (sa position et le
signe de $\Gamma_p$ sont tir�s de fa�on al�atoire). 

%-------------------------------- 
\subsubsection{Taille des vortex}
%--------------------------------
La taille des vortex peut �tre prise constante, ou calcul�e � partir des
relations :
\begin{equation}\notag
\begin{array}{ccc}
\displaystyle
\sigma = \frac{C_{\mu}^{\frac{3}{4}}k^{\frac{3}{2}}}{\varepsilon} 
& \text{    ou    } &
\sigma = max[L_t,L_k]
\\
\end{array}
\end{equation}
avec:
\begin{equation}\notag
\begin{array}{ccc}
\displaystyle
L_t = \sqrt{\left( \frac{5 \nu k}{\varepsilon} \right)} 
& \text{    et    } & 
\displaystyle
L_k = 200\, \left(\frac{\nu^3}{\varepsilon}\right)^{\frac{1}{4}}
\end{array}
\end{equation}
o� $\nu$, $k$ et $\varepsilon$ sont la viscosit� dynamique, l'�nergie cin�tique
turbulente et la dissipation turbulente au point o� se trouve le vortex. 

Dans tous les cas, la taille du vortex doit �tre sup�rieure � la taille des
mailles en entr�e afin que le vortex soit effectivement simul�. 

%==================================
\subsection{Conditions aux limites}
%==================================
Le champ de vitesse g�n�r� � l'aide de la relation \ref{Base_Vortex_compvit} ne tient pas
compte {\em a priori} des conditions aux limites appliqu�es sur les bords du plan
d'entr�e. Pour obtenir des valeurs de la vitesse qui soient coh�rentes sur les
fronti�res du domaine d'entr�e, des ``vortex images'', pseudo-vortex situ�s en
dehors du domaine d'entr�e, sont g�n�r�s � des positions particuli�res et leur
champ de vitesse associ� est superpos� au champ pr�c�demment calcul�.\\
Seuls les cas d'une conduite rectangulaire et d'une conduite circulaire
permettent la g�n�ration de ces pseudo-vortex.
On distingue pour cela trois types de conditions aux limites. 

\begin{figure}[h]
\centerline{\includegraphics[height=6cm]{../Base/Vortex/Images/condlimite.pdf}}
\caption{\label{Base_Vortex_condli} Principe de g�n�ration des ``vortex images'' suivant le
type de conditions aux limites dans une conduite carr�e.} 
\end{figure}

%----------------------------------
\subsubsection{Condition de paroi}
%----------------------------------
On cr�e, pour chaque vortex $P$ contenu dans le plan d'entr�e, un vortex image
$P'$ identique � $P$ (\textit{i.e} de m�me caract�ristiques) et sym�trique de $P$
par rapport au
point $J$ ($J$ �tant la projection orthogonalement � la paroi du point $M$
correspondant au centre de la face o� l'on cherche � calculer la vitesse). La
figure \ref{Base_Vortex_condli} illustre la technique dans le cas d'une conduite
carr�e. Dans ce cas les coordonn�es du vortex situ� en $P'$ v�rifient
$(y_{p'}+y_{p})/2 = y_{J}$ et $(z_{p'}+ z_{p})/2 = z_{J}$. Le champ de vitesse
per�u depuis le point $M$ au niveau du point $J$ est nul, ce qui est bien
l'effet recherch�. 

%------------------------------------
\subsubsection{Condition de sym�trie}
%-------------------------------------
La technique est identique � celle utilis�e pour les conditions de paroi, mais
seule la composante pour la vitesse normale au bord est modifi�e dans ce cas. 

%---------------------------------------
\subsubsection{Condition de p�riodicit�}
%---------------------------------------
On cr�e pour chaque vortex, un vortex images $P'$ identique � $P$ mais translat�
d'une quantit� $L$ correspondant � la longueur qui s�pare les deux plans de la
section d'entr�e o� sont appliqu�es les conditions de p�riodicit�. Dans le cas
o� il y a deux directions de p�riodicit�, on cr�e deux vortex image.

%=============================================
\subsection{Composante de vitesse principale}
%=============================================
La m�thode des vortex ne g�n�rant pas de fluctuation $u$ de la vitesse dans la
direction principale, la derni�re composante est calcul�e � partir d'une
�quation de Langevin. Les coefficients de cette �quation sont d�termin�s par
identification des expressions obtenues pour les contraintes de Reynolds en
$R_{ij}-\varepsilon$. Dans le cas d'un �coulement en canal plan, cette technique
conduit � l'�quation : 
\begin{equation}\notag
\displaystyle
\frac{du}{dt} = - \frac{C_1}{2T} u + \left(\frac{2}{3}C_2-1\right)\frac{\partial
U}{\partial y} v + \sqrt{C_0\varepsilon} dW_i 
\end{equation}

avec $\displaystyle T=\frac{k}{\varepsilon}$, $C_1 = 1,8$, $C_2 = 0,6$,
$C_0=\frac{14}{15}$, et $dW_i$ une variable al�toire Gaussienne de variance
$\sqrt{dt}$. 

En pratique, l'�quation de Langevin n'am�liore pas vraiment les r�sultats. Elle
n'est utilis�e que dans le cas de conduites circulaires. 

%                      Code_Saturne version 1.3
%                      ------------------------
%
%     This file is part of the Code_Saturne Kernel, element of the
%     Code_Saturne CFD tool.
%
%     Copyright (C) 1998-2007 EDF S.A., France
%
%     contact: saturne-support@edf.fr
%
%     The Code_Saturne Kernel is free software; you can redistribute it
%     and/or modify it under the terms of the GNU General Public License
%     as published by the Free Software Foundation; either version 2 of
%     the License, or (at your option) any later version.
%
%     The Code_Saturne Kernel is distributed in the hope that it will be
%     useful, but WITHOUT ANY WARRANTY; without even the implied warranty
%     of MERCHANTABILITY or FITNESS FOR A PARTICULAR PURPOSE.  See the
%     GNU General Public License for more details.
%
%     You should have received a copy of the GNU General Public License
%     along with the Code_Saturne Kernel; if not, write to the
%     Free Software Foundation, Inc.,
%     51 Franklin St, Fifth Floor,
%     Boston, MA  02110-1301  USA
%
%-----------------------------------------------------------------------
%

%%%%%%%%%%%%%%%%%%%%%%%%%%%%%%%%%%
%%%%%%%%%%%%%%%%%%%%%%%%%%%%%%%%%%
\section{Mise en \oe uvre}
%%%%%%%%%%%%%%%%%%%%%%%%%%%%%%%%%%
%%%%%%%%%%%%%%%%%%%%%%%%%%%%%%%%%%
Le syst\`eme (\ref{Cfbl_Cfmsvl_eq_densite_finale_cfmsvl}) est r\'esolu par une m\'ethode
d'incr\'ement et r\'esidu en utilisant
une m\'ethode de Jacobi pour inverser le syst\`eme si le terme convectif
est implicite et en utilisant une m\'ethode de gradient conjugu\'e
si le terme convectif est explicite (qui est le cas par d�faut).

Attention, les valeurs du flux de masse $\rho\,\vect{w}\cdot\vect{S}$ et
de la viscosit\'e $\Delta\,t\,c^2\frac{S}{d}$ aux faces de
bord, qui sont calcul\'ees dans \fort{cfmsfl} et \fort{cfmsvs} respectivement,
sont modifi\'ees imm\'ediatement apr\`es l'appel \`a ces sous-programmes.
En effet, il est indispensable que la contribution de bord de
$\left(\rho\,\vect{w}-\Delta\,t\,(c^2)\,\gradv\,\rho\right)\cdot\vect{S}$
repr\'esente exactement $\vect{Q}_{ac}\cdot\vect{S}$.
Pour cela,
\begin{itemize}
\item imm\'ediatement apr\`es l'appel \`a
\fort{cfmsfl}, on remplace la contribution de bord de
$\rho\,\vect{w}\cdot\vect{S}$
par le flux de masse exact, $\vect{Q}_{ac}\cdot\vect{S}$,
d\'etermin\'e \`a partir des conditions aux limites,
\item puis, imm\'ediatement apr\`es l'appel \`a
\fort{cfmsvs}, on annule la viscosit\'e au bord $\Delta\,t\,(c^2)$ pour
\'eliminer la contribution de $-\Delta\,t\,(c^2)\,(\gradv\,\rho)\cdot\vect{S}$
(l'annulation de la viscosit\'e n'est pas probl\'ematique pour la matrice,
puisqu'elle porte sur des incr\'ements).
\end{itemize}

\bigskip

Une fois qu'on a obtenu $\rho^{n+1}$,
on peut actualiser le flux de masse acoustique
aux faces $(\vect{Q}_{ac}^{n+1})_{ij} \cdot \vect{S}_{ij}$,
qui servira pour la convection des autres variables~:
\begin{equation}\label{Cfbl_Cfmsvl_eq_flux_masse_acoustique_cfmsvl}
\displaystyle(\vect{Q}_{ac}^{n+1})_{ij}\cdot\vect{S}_{ij}=
-\left(\Delta t^n (c^2)^n \gradv(\rho^{n+1})\right)_{ij}\cdot\vect{S}_{ij}
+\left(\rho^{n+\frac{1}{2}} \vect{w}^n\right)_{ij}\cdot\vect{S}_{ij}\\
\end{equation}
Ce calcul de flux est r\'ealis\'e par \fort{cfbsc3}.
Si l'on a choisi l'algorithme standard, \'equation~(\ref{Cfbl_Cfmsvl_eq_densite_cfmsvl}),
on compl\`ete le flux dans \fort{cfmsvl} imm\'ediatement apr\`es l'appel
\`a \fort{cfbsc3}.
En effet, dans ce cas,
la convection est explicite ($\rho^{n+\frac{1}{2}}=\rho^{n}$,
obtenu en imposant \var{ICONV(ISCA(IRHO(IPHAS)))=0})
et le sous-programme \fort{cfbsc3},
qui calcule le flux de masse aux faces,
ne prend pas en compte la contribution du terme
$\rho^{n+\frac{1}{2}}\,\vect{w}^n\cdot\vect{S}$. On ajoute donc cette
contribution dans \fort{cfmsvl}, apr\`es l'appel \`a \fort{cfbsc3}.
Au bord, en particulier, c'est bien le flux de masse calcul\'e \`a partir
des conditions aux limites que l'on obtient.

On actualise la pression \`a la fin de l'\'etape, en utilisant la loi d'\'etat~:
\begin{equation}
\displaystyle P_i^{pred}=P(\rho_i^{n+1},\varepsilon_i^{n})
\end{equation}


%%%%%%%%%%%%%%%%%%%%%%%%%%%%%%%%%%
%%%%%%%%%%%%%%%%%%%%%%%%%%%%%%%%%%
\section{Points \`a traiter}
%%%%%%%%%%%%%%%%%%%%%%%%%%%%%%%%%%
%%%%%%%%%%%%%%%%%%%%%%%%%%%%%%%%%%
Le calcul du flux de masse au  bord n'est pas enti\`erement satisfaisant
si la convection est trait\'ee de mani\`ere implicite
(algorithme non standard, non test\'e,
associ\'e \`a l'\'equation~(\ref{Cfbl_Cfmsvl_eq_densite_bis_cfmsvl}),
correspondant au choix $\rho^{n+\frac{1}{2}}=\rho^{n+1}$ et
obtenu en imposant \var{ICONV(ISCA(IRHO(IPHAS)))=1}).
En effet, apr\`es \fort{cfmsfl}, il faut d\'eterminer la vitesse de
convection $\vect{w}^n$ pour qu'apparaisse
$\rho^{n+1} \vect{w}^n\cdot\vect{n}$
au cours de la r\'esolution par \fort{codits}. De ce fait, on doit d\'eduire
une valeur de $\vect{w}^n$ \`a partir de la valeur
du flux de masse. Au bord, en particulier, il faut
donc diviser le flux de masse
issu des conditions aux limites par la valeur de bord de $\rho^{n+1}$.
Or, lorsque des conditions de Neumann sont appliqu\'ees \`a la
masse volumique,
la valeur de $\rho^{n+1}$ au bord n'est pas connue avant la r\'esolution du
syst\`eme. On utilise donc, au lieu de la valeur de bord inconnue de
$\rho^{n+1}$ la valeur de bord prise au pas de temps
pr\'ec\'edent $\rho^{n}$. Cette approximation est susceptible
d'affecter la valeur du flux de masse au bord.

%                      Code_Saturne version 1.3
%                      ------------------------
%
%     This file is part of the Code_Saturne Kernel, element of the
%     Code_Saturne CFD tool.
%
%     Copyright (C) 1998-2007 EDF S.A., France
%
%     contact: saturne-support@edf.fr
%
%     The Code_Saturne Kernel is free software; you can redistribute it
%     and/or modify it under the terms of the GNU General Public License
%     as published by the Free Software Foundation; either version 2 of
%     the License, or (at your option) any later version.
%
%     The Code_Saturne Kernel is distributed in the hope that it will be
%     useful, but WITHOUT ANY WARRANTY; without even the implied warranty
%     of MERCHANTABILITY or FITNESS FOR A PARTICULAR PURPOSE.  See the
%     GNU General Public License for more details.
%
%     You should have received a copy of the GNU General Public License
%     along with the Code_Saturne Kernel; if not, write to the
%     Free Software Foundation, Inc.,
%     51 Franklin St, Fifth Floor,
%     Boston, MA  02110-1301  USA
%
%-----------------------------------------------------------------------
%


\programme{navsto}

\vspace{1cm}
On s'int\'eresse \`a la r\'esolution du syst\`eme d'\'equations de Navier-Stokes
tridimensionnelles monophasiques, \`a une pression, instationnaires, en
incompressible ou faiblement dilatable, bas\'ees sur une discr\'etisation
temporelle de type Euler implicite d'ordre 1 ou Crank-Nicolson d'ordre 2 et sur
une discr\'etisation spatiale  par volumes finis colocalis\'ee.


%%%%%%%%%%%%%%%%%%%%%%%%%%%%%%%%%%
%%%%%%%%%%%%%%%%%%%%%%%%%%%%%%%%%%
\section{Fonction}
%%%%%%%%%%%%%%%%%%%%%%%%%%%%%%%%%%
%%%%%%%%%%%%%%%%%%%%%%%%%%%%%%%%%%

  Dans ce sous-programme sont calcul\'ees, \`a un pas de temps donn\'e, les
variables vitesse et pression de ce probl\`eme en proc\'edant en
deux  \'etapes issues d'une d\'ecomposition des op\'erateurs (m\'ethode \`a
pas fractionnaires).\\
Les variables sont donc suppos\'ees connues \`a
l'instant ${t^n}$ et on cherche \`a les d\'eterminer \`a l'instant\footnote{La pression est suppos�e connue � l'instant $t^{n-1+\theta}$ et recherch�e en $t^{n+\theta}$, avec $\theta=1$ ou $1/2$ suivant le sch�ma en temps consid�r�.} ${t^{n+1}}$. Soit ${\Delta {t^n} ={t^{n+1}- {t^n}}}$ le pas de temps associ\'e. Dans un premier temps, on r\'ealise l'\'etape de
pr\'ediction de la vitesse en r\'esolvant l'\'equation de quantit\'e de
mouvement avec une pression explicite. Suit l'\'etape de correction de la
pression (ou projection de la vitesse) qui permet d'obtenir un champ de vitesse \`a divergence nulle.\\\\
Les \'equations en continu sont donc :
\begin{equation}
\left\{\begin{array}{l}
\displaystyle\frac{\partial}{\partial t}(\rho \vect{u}) + \dive(\rho\, \vect{u} \otimes \vect{u})
=\dive(\tens{\sigma}) + \vect{TS} - \tens{K}\,\vect{u}\\
\dive(\rho \vect{u}) = \Gamma
\end{array}\right.
\end{equation}

%(plus tard $\frac{\partial \rho}{\partial t} + \dive(\rho \vect{u}) = \Gamma$)



avec $\rho$ la masse volumique, $\vect{u}$ le champ de vitesse,
$[\,\vect{TS}-\tens{K}\,\vect{u}\,]$ les autres termes sources ($\tens{K}$~est un
tenseur diagonal positif par d\'efinition), $\tens{\sigma}$ le tenseur
de contraintes, $\tens{\tau}$ le tenseur des contraintes visqueuses, $\mu$ la
viscosit\'e dynamique (mol\'eculaire et \'eventuellement turbulente), $\kappa$
la viscosit� de
volume (usuellement nulle et n�glig�e dans le code et dans la suite du document,
sauf en compressible),
$\tens{D}$ le tenseur taux de d\'eformation\footnote{\`A ne pas confondre, malgr\'e la m\^eme notation $D$,
avec les flux diffusifs $\vect{D}_{\,ij}$ et $\vect{D}_{\,{b}_{ik}}$ d\'ecrits par la suite dans ce
sous-programme.}, $\Gamma$ le terme source de masse.
\begin{equation}
\left\{\begin{array}{l}
\tens{\sigma} = \tens{\tau} - P\tens{Id}  \\
\tens{\tau} = 2\,\mu\ \tens{D} +\ (\kappa\ - \frac{2}{3}\mu)\  tr({\tens{D}})\
\tens{Id}  \\
\tens{D} = \frac{1}{2}(\ggrad\vect{u}+\,^{t}\ggrad\vect{u})
\end{array}\right.
\end{equation}
 \\

On rappelle la d\'efinition des notations employ\'ees\footnote{en
utilisant la convention de sommation d'Einstein.}~:
\begin{equation}\notag
\left\{\begin{array}{lll}
\left[\ggrad{\vect{a}}\right]_{ij} &=& \partial_j a_i\\
\left[\dive(\tens{\sigma})\right]_i &=& \partial_j \sigma_{ij}\\
\left[\vect{a}\otimes\vect{b}\right]_{ij} &= &
a_i\,b_j\\
\end{array}\right.
\end{equation}
et donc :
\begin{equation}\notag
\begin{array}{lll}
\left[\dive(\vect{a}\otimes\vect{b})\right]_i &= &
\partial_j (a_i\,b_j)
\end{array}
\end{equation}

\minititre{Remarque}
Dans le cas de la prise en compte d'une masse volumique variable, l'�quation de continuit� s'�crit :
$$\frac{\partial \rho}{\partial t} + \dive{\,(\rho\,\vect{u})} = \Gamma  $$
Cette �quation n'est pas prise en compte dans l'�tape de projection (on continue � r�soudre
seulement
$\displaystyle \dive(\,{\rho\,\vect{u}}) = \Gamma$) alors que le terme
$\displaystyle \frac{\partial \rho}{\partial t}$ appara\^{\i}t lors de l'�tape de pr\'ediction de la vitesse
dans le sous-programme \fort{preduv}. Si ce terme joue un r�le sensible, l'algorithme compressible
de \CS\ (qui r�sout l'�quation compl�te) est alors sans doute plus adapt�.

%                      Code_Saturne version 1.3
%                      ------------------------
%
%     This file is part of the Code_Saturne Kernel, element of the
%     Code_Saturne CFD tool.
% 
%     Copyright (C) 1998-2007 EDF S.A., France
%
%     contact: saturne-support@edf.fr
% 
%     The Code_Saturne Kernel is free software; you can redistribute it
%     and/or modify it under the terms of the GNU General Public License
%     as published by the Free Software Foundation; either version 2 of
%     the License, or (at your option) any later version.
% 
%     The Code_Saturne Kernel is distributed in the hope that it will be
%     useful, but WITHOUT ANY WARRANTY; without even the implied warranty
%     of MERCHANTABILITY or FITNESS FOR A PARTICULAR PURPOSE.  See the
%     GNU General Public License for more details.
% 
%     You should have received a copy of the GNU General Public License
%     along with the Code_Saturne Kernel; if not, write to the
%     Free Software Foundation, Inc.,
%     51 Franklin St, Fifth Floor,
%     Boston, MA  02110-1301  USA
%
%-----------------------------------------------------------------------
%
%%%%%%%%%%%%%%%%%%%%%%%%%%%%%%%%%
%%%%%%%%%%%%%%%%%%%%%%%%%%%%%%%%%%
\section{Discr\'etisation}
%%%%%%%%%%%%%%%%%%%%%%%%%%%%%%%%%%
%%%%%%%%%%%%%%%%%%%%%%%%%%%%%%%%%%

Pour utiliser la m�thode, on se place tout d'abord dans un rep�re local d�fini
de mani�re � ce que le plan $(0yz)$, o� sont inject�s les vortex, soit confondu
avec le plan d'entr�e du calcul (voir figure \ref{Base_Vortex_entree}). 

\begin{figure}[h]
\centerline{\includegraphics[height=6cm]{../Base/Vortex/Images/entree.pdf}}
\caption{\label{Base_Vortex_entree} D�finiton des diff�rentes grandeurs dans le rep�re local
correspondant � l'entr�e d'une conduite de section carr�e.} 
\end{figure}

$u$, $v$ et $w$  sont les composantes de la vitesse fluctuante (principale et
transverse) dans ce plan, et
$\displaystyle \omega(y,z) = \frac{\partial w}{\partial y}-\frac{\partial v}{\partial z}$
la vorticit� dans la direction
normale au plan d'entr�e. $\overline{U}(y,z)$ repr�sente ici la vitesse
principale moyenne impos�e par l'utilisateur dans le plan d'entr�e. 

Chaque vortex $p$ va �tre caract�ris� par sa fonction de forme $\xi$ (identique
pour tous les vortex), sa
circulation $\Gamma_p$, son rayon $\sigma_p$ et les coordonn�es $(y_p,z_p)$ du
point $P$ o� est situ� le vortex dans le plan $(0yz)$. 

Pour cela, on suppose que la vorticit� g�n�r�e par un vortex $p$ au point $M$ de
coordonn�e $(y,z)$ s'�crit : 
\begin{equation}\notag
\omega_p(y,z)= \Gamma_p \, \xi_{\sigma_p}(r)
\end{equation}
o� $r = \sqrt{(y-y_p)^2+(z-z_p)^2}$ est la distance s�parant le point $M$ du point $P$.

Dans la m�thode implant�e, la fonction de forme est de type gaussienne modifi�e :
\begin{equation}\notag
\displaystyle
\xi_\sigma (r) = \frac{1}{2\pi \sigma^2} 
\left(2 e^{-\frac{r^2}{2\sigma^2}}-1\right) e^{-\frac{r^2}{2\sigma^2}}
\end{equation}

Le champ de vitesse induit par cette distribution de vorticit� s'obtient par
inversion des deux �quations de poisson suivantes qui sont d�duites de la
condition d'incompressibilit� dans la plan\footnote{\textit{i.e}
$\displaystyle \frac{\partial v}{\partial y}+\frac{\partial w}{\partial w} = 0$} :
\begin{equation}\notag
\begin{array}{lcr}
\displaystyle
\frac{\partial \omega}{\partial y} = \Delta w
&
\text{    et    }
&
\displaystyle
\frac{\partial \omega}{\partial y} = -\Delta v
\\
\end{array}
\end{equation}

Dans le cas g�n�ral, ce syst�me peut �tre int�gr� � l'aide de la formule de Biot et Savart.

Dans le cas d'une distribution de vorticit� de type gaussienne modifi�e, les
composantes de vitesse v�rifient : 
\begin{equation}\notag
\left\{
\begin{array}{c}
\displaystyle
v_p(y,x) = - \frac{1}{2\pi} \frac{(z-z_p)}{r^2}\left(1 -
e^{-\frac{r^2}{2\sigma^2}}\right)\,e^{-\frac{r^2}{2\sigma^2}} 
\\
\displaystyle
w_p(y,z) = \frac{1}{2\pi} \frac{(y-y_p)}{r^2}\left(1 -e^{-\frac{r^2}{2\sigma^2}}
\right)\,e^{-\frac{r^2}{2\sigma^2}} 
\end{array}
\right.
\end{equation}

Ces relations s'�tendent de fa�on imm�diate au cas de $N$ vortex distincts.
Le champ de vitesse induit par la distribution de vorticit� 
\begin{equation}
\omega(y,z) = \sum_{p=1}^N \Gamma_p \, \xi_{\sigma_p}(r)
\end{equation}
vaut au point $M$ :
\begin{equation}\notag
\begin{array}{lcr}
v(x,y) = \sum_{p=1}^N \Gamma_p\, v_p(y,z) 
&
\text{    et    }
&
w(y,z) = \sum_{p=1}^N \Gamma_p\, w_p(y,z)
\\
\label{Base_Vortex_compvit}
\end{array}
\end{equation}
%================================
\subsection{Param�tres physiques}
%================================

%-------------------------------
\subsubsection{Marche en temps}
%-------------------------------
La position initiale de chaque vortex est tir�e de mani�re al�atoire. On calcul
les d�placements successifs de chacun des vortex dans le plan d'entr�e par
int�gration explicite du champ de vitesse lagrangien : 
\begin{equation}\notag
\begin{array}{lcr}
\displaystyle
\frac{dy_p}{dt} = V(y,z)
&
\text{    et    }
&
\displaystyle
\frac{dz_p}{dt} = W(y,z)
\\
\end{array}
\end{equation}
Les vortex sont alors assimil�s � des particules ponctuelles qui sont convect�es
par le champ $(V,W)$. Ce champ peut �tre impos� par des tirages al�atoires ou
bien d�duit de la vitesse induite par les vortex dans le plan d'entr�e. Dans ce
cas $V(x,y) = \overline{V}(y,z) + v (y,z)$ et $W(y,z)= \overline{W}(y,z) +
w(y,z)$ o� $\overline{V}$ et $\overline{W}$ sont les composantes de la vitesse
transverse moyenne qu'impose l'utilisateur � l'aide des fichiers de donn�es. 

%---------------------------------------------------
\subsubsection{Intensit� et dur�e de vie des vortex}
%---------------------------------------------------
Il serait possible, � partir de l'�quation de transport de la vorticit�,
d'obtenir un mod�le d'�volution pour l'intensit� du vecteur tourbillon
$\omega_p$ associ� � chacun des vortex. En pratique, on pr�f�re utiliser un
mod�le simplifi� dans lequel la circulation des tourbillons ne d�pend que de la
postion de ces derniers � l'instant consid�r�. La circulation initiale de chaque
vortex est alors obtenue � partir de la relation : 
\begin{equation}\notag
|\Gamma_p| = 4 \sqrt{\frac{\pi\,S\,k}{3N\,[2ln(3)-3ln(2)]}}
\end{equation}
o� $S$ est la surface du plan d'entr�e, $N$ le nombre de vortex, et $k$
l'�nergie cin�tique turbulente au point o� se trouve le vortex � l'instant
consid�r�. Le signe de $\Gamma_p$ correspond au sens de rotation du vortex et
est tir� al�atoirement. 

Ce param�tre est celui qui contr�le l'intensit� des fluctuations. Sa d�pendance
en $k$ exprime que, plus l'�coulement est turbulent, plus les vortex sont
intenses. La valeur de $k$ est sp�cifi�e par
l'utilisateur. Elle peut �tre constante ou impos�e � partir de profils d'�nergie
cin�tique turbulente en entr�e. 

Pour �viter que des structures trop allong�es ne se d�veloppent au niveau de
l'entr�e, l'utilisateur doit �galement sp�cifier un temps limites $\tau_p$ au
bout duquel le vortex $p$ va �tre d�truit. Ce temps $\tau_p$ peut �tre pris
constant ou estim� au moyen de la relation : 
\begin{equation}\notag
\tau_p = \frac{5 C_{\mu}k^{\frac{3}{2}}}{\varepsilon\,\overline{U}}
\end{equation}

$\overline{U}$ et $\varepsilon$ repr�sentent respectivement la vitesse moyenne
principale et la dissipation turbulente au point o� est initialement g�n�r� le
vortex ($C_{\mu}=0,09$). 
\\
Lorsque le temps �coul� depuis la cr�ation du vortex $p$ est sup�rieur �
$\tau_p$, le vortex est d�truit et un nouveau vortex g�n�r� (sa position et le
signe de $\Gamma_p$ sont tir�s de fa�on al�atoire). 

%-------------------------------- 
\subsubsection{Taille des vortex}
%--------------------------------
La taille des vortex peut �tre prise constante, ou calcul�e � partir des
relations :
\begin{equation}\notag
\begin{array}{ccc}
\displaystyle
\sigma = \frac{C_{\mu}^{\frac{3}{4}}k^{\frac{3}{2}}}{\varepsilon} 
& \text{    ou    } &
\sigma = max[L_t,L_k]
\\
\end{array}
\end{equation}
avec:
\begin{equation}\notag
\begin{array}{ccc}
\displaystyle
L_t = \sqrt{\left( \frac{5 \nu k}{\varepsilon} \right)} 
& \text{    et    } & 
\displaystyle
L_k = 200\, \left(\frac{\nu^3}{\varepsilon}\right)^{\frac{1}{4}}
\end{array}
\end{equation}
o� $\nu$, $k$ et $\varepsilon$ sont la viscosit� dynamique, l'�nergie cin�tique
turbulente et la dissipation turbulente au point o� se trouve le vortex. 

Dans tous les cas, la taille du vortex doit �tre sup�rieure � la taille des
mailles en entr�e afin que le vortex soit effectivement simul�. 

%==================================
\subsection{Conditions aux limites}
%==================================
Le champ de vitesse g�n�r� � l'aide de la relation \ref{Base_Vortex_compvit} ne tient pas
compte {\em a priori} des conditions aux limites appliqu�es sur les bords du plan
d'entr�e. Pour obtenir des valeurs de la vitesse qui soient coh�rentes sur les
fronti�res du domaine d'entr�e, des ``vortex images'', pseudo-vortex situ�s en
dehors du domaine d'entr�e, sont g�n�r�s � des positions particuli�res et leur
champ de vitesse associ� est superpos� au champ pr�c�demment calcul�.\\
Seuls les cas d'une conduite rectangulaire et d'une conduite circulaire
permettent la g�n�ration de ces pseudo-vortex.
On distingue pour cela trois types de conditions aux limites. 

\begin{figure}[h]
\centerline{\includegraphics[height=6cm]{../Base/Vortex/Images/condlimite.pdf}}
\caption{\label{Base_Vortex_condli} Principe de g�n�ration des ``vortex images'' suivant le
type de conditions aux limites dans une conduite carr�e.} 
\end{figure}

%----------------------------------
\subsubsection{Condition de paroi}
%----------------------------------
On cr�e, pour chaque vortex $P$ contenu dans le plan d'entr�e, un vortex image
$P'$ identique � $P$ (\textit{i.e} de m�me caract�ristiques) et sym�trique de $P$
par rapport au
point $J$ ($J$ �tant la projection orthogonalement � la paroi du point $M$
correspondant au centre de la face o� l'on cherche � calculer la vitesse). La
figure \ref{Base_Vortex_condli} illustre la technique dans le cas d'une conduite
carr�e. Dans ce cas les coordonn�es du vortex situ� en $P'$ v�rifient
$(y_{p'}+y_{p})/2 = y_{J}$ et $(z_{p'}+ z_{p})/2 = z_{J}$. Le champ de vitesse
per�u depuis le point $M$ au niveau du point $J$ est nul, ce qui est bien
l'effet recherch�. 

%------------------------------------
\subsubsection{Condition de sym�trie}
%-------------------------------------
La technique est identique � celle utilis�e pour les conditions de paroi, mais
seule la composante pour la vitesse normale au bord est modifi�e dans ce cas. 

%---------------------------------------
\subsubsection{Condition de p�riodicit�}
%---------------------------------------
On cr�e pour chaque vortex, un vortex images $P'$ identique � $P$ mais translat�
d'une quantit� $L$ correspondant � la longueur qui s�pare les deux plans de la
section d'entr�e o� sont appliqu�es les conditions de p�riodicit�. Dans le cas
o� il y a deux directions de p�riodicit�, on cr�e deux vortex image.

%=============================================
\subsection{Composante de vitesse principale}
%=============================================
La m�thode des vortex ne g�n�rant pas de fluctuation $u$ de la vitesse dans la
direction principale, la derni�re composante est calcul�e � partir d'une
�quation de Langevin. Les coefficients de cette �quation sont d�termin�s par
identification des expressions obtenues pour les contraintes de Reynolds en
$R_{ij}-\varepsilon$. Dans le cas d'un �coulement en canal plan, cette technique
conduit � l'�quation : 
\begin{equation}\notag
\displaystyle
\frac{du}{dt} = - \frac{C_1}{2T} u + \left(\frac{2}{3}C_2-1\right)\frac{\partial
U}{\partial y} v + \sqrt{C_0\varepsilon} dW_i 
\end{equation}

avec $\displaystyle T=\frac{k}{\varepsilon}$, $C_1 = 1,8$, $C_2 = 0,6$,
$C_0=\frac{14}{15}$, et $dW_i$ une variable al�toire Gaussienne de variance
$\sqrt{dt}$. 

En pratique, l'�quation de Langevin n'am�liore pas vraiment les r�sultats. Elle
n'est utilis�e que dans le cas de conduites circulaires. 

%                      Code_Saturne version 1.3
%                      ------------------------
%
%     This file is part of the Code_Saturne Kernel, element of the
%     Code_Saturne CFD tool.
%
%     Copyright (C) 1998-2007 EDF S.A., France
%
%     contact: saturne-support@edf.fr
%
%     The Code_Saturne Kernel is free software; you can redistribute it
%     and/or modify it under the terms of the GNU General Public License
%     as published by the Free Software Foundation; either version 2 of
%     the License, or (at your option) any later version.
%
%     The Code_Saturne Kernel is distributed in the hope that it will be
%     useful, but WITHOUT ANY WARRANTY; without even the implied warranty
%     of MERCHANTABILITY or FITNESS FOR A PARTICULAR PURPOSE.  See the
%     GNU General Public License for more details.
%
%     You should have received a copy of the GNU General Public License
%     along with the Code_Saturne Kernel; if not, write to the
%     Free Software Foundation, Inc.,
%     51 Franklin St, Fifth Floor,
%     Boston, MA  02110-1301  USA
%
%-----------------------------------------------------------------------
%

%%%%%%%%%%%%%%%%%%%%%%%%%%%%%%%%%%
%%%%%%%%%%%%%%%%%%%%%%%%%%%%%%%%%%
\section{Mise en \oe uvre}
%%%%%%%%%%%%%%%%%%%%%%%%%%%%%%%%%%
%%%%%%%%%%%%%%%%%%%%%%%%%%%%%%%%%%
Le syst\`eme (\ref{Cfbl_Cfmsvl_eq_densite_finale_cfmsvl}) est r\'esolu par une m\'ethode
d'incr\'ement et r\'esidu en utilisant
une m\'ethode de Jacobi pour inverser le syst\`eme si le terme convectif
est implicite et en utilisant une m\'ethode de gradient conjugu\'e
si le terme convectif est explicite (qui est le cas par d�faut).

Attention, les valeurs du flux de masse $\rho\,\vect{w}\cdot\vect{S}$ et
de la viscosit\'e $\Delta\,t\,c^2\frac{S}{d}$ aux faces de
bord, qui sont calcul\'ees dans \fort{cfmsfl} et \fort{cfmsvs} respectivement,
sont modifi\'ees imm\'ediatement apr\`es l'appel \`a ces sous-programmes.
En effet, il est indispensable que la contribution de bord de
$\left(\rho\,\vect{w}-\Delta\,t\,(c^2)\,\gradv\,\rho\right)\cdot\vect{S}$
repr\'esente exactement $\vect{Q}_{ac}\cdot\vect{S}$.
Pour cela,
\begin{itemize}
\item imm\'ediatement apr\`es l'appel \`a
\fort{cfmsfl}, on remplace la contribution de bord de
$\rho\,\vect{w}\cdot\vect{S}$
par le flux de masse exact, $\vect{Q}_{ac}\cdot\vect{S}$,
d\'etermin\'e \`a partir des conditions aux limites,
\item puis, imm\'ediatement apr\`es l'appel \`a
\fort{cfmsvs}, on annule la viscosit\'e au bord $\Delta\,t\,(c^2)$ pour
\'eliminer la contribution de $-\Delta\,t\,(c^2)\,(\gradv\,\rho)\cdot\vect{S}$
(l'annulation de la viscosit\'e n'est pas probl\'ematique pour la matrice,
puisqu'elle porte sur des incr\'ements).
\end{itemize}

\bigskip

Une fois qu'on a obtenu $\rho^{n+1}$,
on peut actualiser le flux de masse acoustique
aux faces $(\vect{Q}_{ac}^{n+1})_{ij} \cdot \vect{S}_{ij}$,
qui servira pour la convection des autres variables~:
\begin{equation}\label{Cfbl_Cfmsvl_eq_flux_masse_acoustique_cfmsvl}
\displaystyle(\vect{Q}_{ac}^{n+1})_{ij}\cdot\vect{S}_{ij}=
-\left(\Delta t^n (c^2)^n \gradv(\rho^{n+1})\right)_{ij}\cdot\vect{S}_{ij}
+\left(\rho^{n+\frac{1}{2}} \vect{w}^n\right)_{ij}\cdot\vect{S}_{ij}\\
\end{equation}
Ce calcul de flux est r\'ealis\'e par \fort{cfbsc3}.
Si l'on a choisi l'algorithme standard, \'equation~(\ref{Cfbl_Cfmsvl_eq_densite_cfmsvl}),
on compl\`ete le flux dans \fort{cfmsvl} imm\'ediatement apr\`es l'appel
\`a \fort{cfbsc3}.
En effet, dans ce cas,
la convection est explicite ($\rho^{n+\frac{1}{2}}=\rho^{n}$,
obtenu en imposant \var{ICONV(ISCA(IRHO(IPHAS)))=0})
et le sous-programme \fort{cfbsc3},
qui calcule le flux de masse aux faces,
ne prend pas en compte la contribution du terme
$\rho^{n+\frac{1}{2}}\,\vect{w}^n\cdot\vect{S}$. On ajoute donc cette
contribution dans \fort{cfmsvl}, apr\`es l'appel \`a \fort{cfbsc3}.
Au bord, en particulier, c'est bien le flux de masse calcul\'e \`a partir
des conditions aux limites que l'on obtient.

On actualise la pression \`a la fin de l'\'etape, en utilisant la loi d'\'etat~:
\begin{equation}
\displaystyle P_i^{pred}=P(\rho_i^{n+1},\varepsilon_i^{n})
\end{equation}


%%%%%%%%%%%%%%%%%%%%%%%%%%%%%%%%%%
%%%%%%%%%%%%%%%%%%%%%%%%%%%%%%%%%%
\section{Points \`a traiter}
%%%%%%%%%%%%%%%%%%%%%%%%%%%%%%%%%%
%%%%%%%%%%%%%%%%%%%%%%%%%%%%%%%%%%
Le calcul du flux de masse au  bord n'est pas enti\`erement satisfaisant
si la convection est trait\'ee de mani\`ere implicite
(algorithme non standard, non test\'e,
associ\'e \`a l'\'equation~(\ref{Cfbl_Cfmsvl_eq_densite_bis_cfmsvl}),
correspondant au choix $\rho^{n+\frac{1}{2}}=\rho^{n+1}$ et
obtenu en imposant \var{ICONV(ISCA(IRHO(IPHAS)))=1}).
En effet, apr\`es \fort{cfmsfl}, il faut d\'eterminer la vitesse de
convection $\vect{w}^n$ pour qu'apparaisse
$\rho^{n+1} \vect{w}^n\cdot\vect{n}$
au cours de la r\'esolution par \fort{codits}. De ce fait, on doit d\'eduire
une valeur de $\vect{w}^n$ \`a partir de la valeur
du flux de masse. Au bord, en particulier, il faut
donc diviser le flux de masse
issu des conditions aux limites par la valeur de bord de $\rho^{n+1}$.
Or, lorsque des conditions de Neumann sont appliqu\'ees \`a la
masse volumique,
la valeur de $\rho^{n+1}$ au bord n'est pas connue avant la r\'esolution du
syst\`eme. On utilise donc, au lieu de la valeur de bord inconnue de
$\rho^{n+1}$ la valeur de bord prise au pas de temps
pr\'ec\'edent $\rho^{n}$. Cette approximation est susceptible
d'affecter la valeur du flux de masse au bord.

%                      Code_Saturne version 1.3
%                      ------------------------
%
%     This file is part of the Code_Saturne Kernel, element of the
%     Code_Saturne CFD tool.
%
%     Copyright (C) 1998-2007 EDF S.A., France
%
%     contact: saturne-support@edf.fr
%
%     The Code_Saturne Kernel is free software; you can redistribute it
%     and/or modify it under the terms of the GNU General Public License
%     as published by the Free Software Foundation; either version 2 of
%     the License, or (at your option) any later version.
%
%     The Code_Saturne Kernel is distributed in the hope that it will be
%     useful, but WITHOUT ANY WARRANTY; without even the implied warranty
%     of MERCHANTABILITY or FITNESS FOR A PARTICULAR PURPOSE.  See the
%     GNU General Public License for more details.
%
%     You should have received a copy of the GNU General Public License
%     along with the Code_Saturne Kernel; if not, write to the
%     Free Software Foundation, Inc.,
%     51 Franklin St, Fifth Floor,
%     Boston, MA  02110-1301  USA
%
%-----------------------------------------------------------------------
%


\programme{navsto}

\vspace{1cm}
On s'int\'eresse \`a la r\'esolution du syst\`eme d'\'equations de Navier-Stokes
tridimensionnelles monophasiques, \`a une pression, instationnaires, en
incompressible ou faiblement dilatable, bas\'ees sur une discr\'etisation
temporelle de type Euler implicite d'ordre 1 ou Crank-Nicolson d'ordre 2 et sur
une discr\'etisation spatiale  par volumes finis colocalis\'ee.


%%%%%%%%%%%%%%%%%%%%%%%%%%%%%%%%%%
%%%%%%%%%%%%%%%%%%%%%%%%%%%%%%%%%%
\section{Fonction}
%%%%%%%%%%%%%%%%%%%%%%%%%%%%%%%%%%
%%%%%%%%%%%%%%%%%%%%%%%%%%%%%%%%%%

  Dans ce sous-programme sont calcul\'ees, \`a un pas de temps donn\'e, les
variables vitesse et pression de ce probl\`eme en proc\'edant en
deux  \'etapes issues d'une d\'ecomposition des op\'erateurs (m\'ethode \`a
pas fractionnaires).\\
Les variables sont donc suppos\'ees connues \`a
l'instant ${t^n}$ et on cherche \`a les d\'eterminer \`a l'instant\footnote{La pression est suppos�e connue � l'instant $t^{n-1+\theta}$ et recherch�e en $t^{n+\theta}$, avec $\theta=1$ ou $1/2$ suivant le sch�ma en temps consid�r�.} ${t^{n+1}}$. Soit ${\Delta {t^n} ={t^{n+1}- {t^n}}}$ le pas de temps associ\'e. Dans un premier temps, on r\'ealise l'\'etape de
pr\'ediction de la vitesse en r\'esolvant l'\'equation de quantit\'e de
mouvement avec une pression explicite. Suit l'\'etape de correction de la
pression (ou projection de la vitesse) qui permet d'obtenir un champ de vitesse \`a divergence nulle.\\\\
Les \'equations en continu sont donc :
\begin{equation}
\left\{\begin{array}{l}
\displaystyle\frac{\partial}{\partial t}(\rho \vect{u}) + \dive(\rho\, \vect{u} \otimes \vect{u})
=\dive(\tens{\sigma}) + \vect{TS} - \tens{K}\,\vect{u}\\
\dive(\rho \vect{u}) = \Gamma
\end{array}\right.
\end{equation}

%(plus tard $\frac{\partial \rho}{\partial t} + \dive(\rho \vect{u}) = \Gamma$)



avec $\rho$ la masse volumique, $\vect{u}$ le champ de vitesse,
$[\,\vect{TS}-\tens{K}\,\vect{u}\,]$ les autres termes sources ($\tens{K}$~est un
tenseur diagonal positif par d\'efinition), $\tens{\sigma}$ le tenseur
de contraintes, $\tens{\tau}$ le tenseur des contraintes visqueuses, $\mu$ la
viscosit\'e dynamique (mol\'eculaire et \'eventuellement turbulente), $\kappa$
la viscosit� de
volume (usuellement nulle et n�glig�e dans le code et dans la suite du document,
sauf en compressible),
$\tens{D}$ le tenseur taux de d\'eformation\footnote{\`A ne pas confondre, malgr\'e la m\^eme notation $D$,
avec les flux diffusifs $\vect{D}_{\,ij}$ et $\vect{D}_{\,{b}_{ik}}$ d\'ecrits par la suite dans ce
sous-programme.}, $\Gamma$ le terme source de masse.
\begin{equation}
\left\{\begin{array}{l}
\tens{\sigma} = \tens{\tau} - P\tens{Id}  \\
\tens{\tau} = 2\,\mu\ \tens{D} +\ (\kappa\ - \frac{2}{3}\mu)\  tr({\tens{D}})\
\tens{Id}  \\
\tens{D} = \frac{1}{2}(\ggrad\vect{u}+\,^{t}\ggrad\vect{u})
\end{array}\right.
\end{equation}
 \\

On rappelle la d\'efinition des notations employ\'ees\footnote{en
utilisant la convention de sommation d'Einstein.}~:
\begin{equation}\notag
\left\{\begin{array}{lll}
\left[\ggrad{\vect{a}}\right]_{ij} &=& \partial_j a_i\\
\left[\dive(\tens{\sigma})\right]_i &=& \partial_j \sigma_{ij}\\
\left[\vect{a}\otimes\vect{b}\right]_{ij} &= &
a_i\,b_j\\
\end{array}\right.
\end{equation}
et donc :
\begin{equation}\notag
\begin{array}{lll}
\left[\dive(\vect{a}\otimes\vect{b})\right]_i &= &
\partial_j (a_i\,b_j)
\end{array}
\end{equation}

\minititre{Remarque}
Dans le cas de la prise en compte d'une masse volumique variable, l'�quation de continuit� s'�crit :
$$\frac{\partial \rho}{\partial t} + \dive{\,(\rho\,\vect{u})} = \Gamma  $$
Cette �quation n'est pas prise en compte dans l'�tape de projection (on continue � r�soudre
seulement
$\displaystyle \dive(\,{\rho\,\vect{u}}) = \Gamma$) alors que le terme
$\displaystyle \frac{\partial \rho}{\partial t}$ appara\^{\i}t lors de l'�tape de pr\'ediction de la vitesse
dans le sous-programme \fort{preduv}. Si ce terme joue un r�le sensible, l'algorithme compressible
de \CS\ (qui r�sout l'�quation compl�te) est alors sans doute plus adapt�.

%                      Code_Saturne version 1.3
%                      ------------------------
%
%     This file is part of the Code_Saturne Kernel, element of the
%     Code_Saturne CFD tool.
% 
%     Copyright (C) 1998-2007 EDF S.A., France
%
%     contact: saturne-support@edf.fr
% 
%     The Code_Saturne Kernel is free software; you can redistribute it
%     and/or modify it under the terms of the GNU General Public License
%     as published by the Free Software Foundation; either version 2 of
%     the License, or (at your option) any later version.
% 
%     The Code_Saturne Kernel is distributed in the hope that it will be
%     useful, but WITHOUT ANY WARRANTY; without even the implied warranty
%     of MERCHANTABILITY or FITNESS FOR A PARTICULAR PURPOSE.  See the
%     GNU General Public License for more details.
% 
%     You should have received a copy of the GNU General Public License
%     along with the Code_Saturne Kernel; if not, write to the
%     Free Software Foundation, Inc.,
%     51 Franklin St, Fifth Floor,
%     Boston, MA  02110-1301  USA
%
%-----------------------------------------------------------------------
%
%%%%%%%%%%%%%%%%%%%%%%%%%%%%%%%%%
%%%%%%%%%%%%%%%%%%%%%%%%%%%%%%%%%%
\section{Discr\'etisation}
%%%%%%%%%%%%%%%%%%%%%%%%%%%%%%%%%%
%%%%%%%%%%%%%%%%%%%%%%%%%%%%%%%%%%

Pour utiliser la m�thode, on se place tout d'abord dans un rep�re local d�fini
de mani�re � ce que le plan $(0yz)$, o� sont inject�s les vortex, soit confondu
avec le plan d'entr�e du calcul (voir figure \ref{Base_Vortex_entree}). 

\begin{figure}[h]
\centerline{\includegraphics[height=6cm]{../Base/Vortex/Images/entree.pdf}}
\caption{\label{Base_Vortex_entree} D�finiton des diff�rentes grandeurs dans le rep�re local
correspondant � l'entr�e d'une conduite de section carr�e.} 
\end{figure}

$u$, $v$ et $w$  sont les composantes de la vitesse fluctuante (principale et
transverse) dans ce plan, et
$\displaystyle \omega(y,z) = \frac{\partial w}{\partial y}-\frac{\partial v}{\partial z}$
la vorticit� dans la direction
normale au plan d'entr�e. $\overline{U}(y,z)$ repr�sente ici la vitesse
principale moyenne impos�e par l'utilisateur dans le plan d'entr�e. 

Chaque vortex $p$ va �tre caract�ris� par sa fonction de forme $\xi$ (identique
pour tous les vortex), sa
circulation $\Gamma_p$, son rayon $\sigma_p$ et les coordonn�es $(y_p,z_p)$ du
point $P$ o� est situ� le vortex dans le plan $(0yz)$. 

Pour cela, on suppose que la vorticit� g�n�r�e par un vortex $p$ au point $M$ de
coordonn�e $(y,z)$ s'�crit : 
\begin{equation}\notag
\omega_p(y,z)= \Gamma_p \, \xi_{\sigma_p}(r)
\end{equation}
o� $r = \sqrt{(y-y_p)^2+(z-z_p)^2}$ est la distance s�parant le point $M$ du point $P$.

Dans la m�thode implant�e, la fonction de forme est de type gaussienne modifi�e :
\begin{equation}\notag
\displaystyle
\xi_\sigma (r) = \frac{1}{2\pi \sigma^2} 
\left(2 e^{-\frac{r^2}{2\sigma^2}}-1\right) e^{-\frac{r^2}{2\sigma^2}}
\end{equation}

Le champ de vitesse induit par cette distribution de vorticit� s'obtient par
inversion des deux �quations de poisson suivantes qui sont d�duites de la
condition d'incompressibilit� dans la plan\footnote{\textit{i.e}
$\displaystyle \frac{\partial v}{\partial y}+\frac{\partial w}{\partial w} = 0$} :
\begin{equation}\notag
\begin{array}{lcr}
\displaystyle
\frac{\partial \omega}{\partial y} = \Delta w
&
\text{    et    }
&
\displaystyle
\frac{\partial \omega}{\partial y} = -\Delta v
\\
\end{array}
\end{equation}

Dans le cas g�n�ral, ce syst�me peut �tre int�gr� � l'aide de la formule de Biot et Savart.

Dans le cas d'une distribution de vorticit� de type gaussienne modifi�e, les
composantes de vitesse v�rifient : 
\begin{equation}\notag
\left\{
\begin{array}{c}
\displaystyle
v_p(y,x) = - \frac{1}{2\pi} \frac{(z-z_p)}{r^2}\left(1 -
e^{-\frac{r^2}{2\sigma^2}}\right)\,e^{-\frac{r^2}{2\sigma^2}} 
\\
\displaystyle
w_p(y,z) = \frac{1}{2\pi} \frac{(y-y_p)}{r^2}\left(1 -e^{-\frac{r^2}{2\sigma^2}}
\right)\,e^{-\frac{r^2}{2\sigma^2}} 
\end{array}
\right.
\end{equation}

Ces relations s'�tendent de fa�on imm�diate au cas de $N$ vortex distincts.
Le champ de vitesse induit par la distribution de vorticit� 
\begin{equation}
\omega(y,z) = \sum_{p=1}^N \Gamma_p \, \xi_{\sigma_p}(r)
\end{equation}
vaut au point $M$ :
\begin{equation}\notag
\begin{array}{lcr}
v(x,y) = \sum_{p=1}^N \Gamma_p\, v_p(y,z) 
&
\text{    et    }
&
w(y,z) = \sum_{p=1}^N \Gamma_p\, w_p(y,z)
\\
\label{Base_Vortex_compvit}
\end{array}
\end{equation}
%================================
\subsection{Param�tres physiques}
%================================

%-------------------------------
\subsubsection{Marche en temps}
%-------------------------------
La position initiale de chaque vortex est tir�e de mani�re al�atoire. On calcul
les d�placements successifs de chacun des vortex dans le plan d'entr�e par
int�gration explicite du champ de vitesse lagrangien : 
\begin{equation}\notag
\begin{array}{lcr}
\displaystyle
\frac{dy_p}{dt} = V(y,z)
&
\text{    et    }
&
\displaystyle
\frac{dz_p}{dt} = W(y,z)
\\
\end{array}
\end{equation}
Les vortex sont alors assimil�s � des particules ponctuelles qui sont convect�es
par le champ $(V,W)$. Ce champ peut �tre impos� par des tirages al�atoires ou
bien d�duit de la vitesse induite par les vortex dans le plan d'entr�e. Dans ce
cas $V(x,y) = \overline{V}(y,z) + v (y,z)$ et $W(y,z)= \overline{W}(y,z) +
w(y,z)$ o� $\overline{V}$ et $\overline{W}$ sont les composantes de la vitesse
transverse moyenne qu'impose l'utilisateur � l'aide des fichiers de donn�es. 

%---------------------------------------------------
\subsubsection{Intensit� et dur�e de vie des vortex}
%---------------------------------------------------
Il serait possible, � partir de l'�quation de transport de la vorticit�,
d'obtenir un mod�le d'�volution pour l'intensit� du vecteur tourbillon
$\omega_p$ associ� � chacun des vortex. En pratique, on pr�f�re utiliser un
mod�le simplifi� dans lequel la circulation des tourbillons ne d�pend que de la
postion de ces derniers � l'instant consid�r�. La circulation initiale de chaque
vortex est alors obtenue � partir de la relation : 
\begin{equation}\notag
|\Gamma_p| = 4 \sqrt{\frac{\pi\,S\,k}{3N\,[2ln(3)-3ln(2)]}}
\end{equation}
o� $S$ est la surface du plan d'entr�e, $N$ le nombre de vortex, et $k$
l'�nergie cin�tique turbulente au point o� se trouve le vortex � l'instant
consid�r�. Le signe de $\Gamma_p$ correspond au sens de rotation du vortex et
est tir� al�atoirement. 

Ce param�tre est celui qui contr�le l'intensit� des fluctuations. Sa d�pendance
en $k$ exprime que, plus l'�coulement est turbulent, plus les vortex sont
intenses. La valeur de $k$ est sp�cifi�e par
l'utilisateur. Elle peut �tre constante ou impos�e � partir de profils d'�nergie
cin�tique turbulente en entr�e. 

Pour �viter que des structures trop allong�es ne se d�veloppent au niveau de
l'entr�e, l'utilisateur doit �galement sp�cifier un temps limites $\tau_p$ au
bout duquel le vortex $p$ va �tre d�truit. Ce temps $\tau_p$ peut �tre pris
constant ou estim� au moyen de la relation : 
\begin{equation}\notag
\tau_p = \frac{5 C_{\mu}k^{\frac{3}{2}}}{\varepsilon\,\overline{U}}
\end{equation}

$\overline{U}$ et $\varepsilon$ repr�sentent respectivement la vitesse moyenne
principale et la dissipation turbulente au point o� est initialement g�n�r� le
vortex ($C_{\mu}=0,09$). 
\\
Lorsque le temps �coul� depuis la cr�ation du vortex $p$ est sup�rieur �
$\tau_p$, le vortex est d�truit et un nouveau vortex g�n�r� (sa position et le
signe de $\Gamma_p$ sont tir�s de fa�on al�atoire). 

%-------------------------------- 
\subsubsection{Taille des vortex}
%--------------------------------
La taille des vortex peut �tre prise constante, ou calcul�e � partir des
relations :
\begin{equation}\notag
\begin{array}{ccc}
\displaystyle
\sigma = \frac{C_{\mu}^{\frac{3}{4}}k^{\frac{3}{2}}}{\varepsilon} 
& \text{    ou    } &
\sigma = max[L_t,L_k]
\\
\end{array}
\end{equation}
avec:
\begin{equation}\notag
\begin{array}{ccc}
\displaystyle
L_t = \sqrt{\left( \frac{5 \nu k}{\varepsilon} \right)} 
& \text{    et    } & 
\displaystyle
L_k = 200\, \left(\frac{\nu^3}{\varepsilon}\right)^{\frac{1}{4}}
\end{array}
\end{equation}
o� $\nu$, $k$ et $\varepsilon$ sont la viscosit� dynamique, l'�nergie cin�tique
turbulente et la dissipation turbulente au point o� se trouve le vortex. 

Dans tous les cas, la taille du vortex doit �tre sup�rieure � la taille des
mailles en entr�e afin que le vortex soit effectivement simul�. 

%==================================
\subsection{Conditions aux limites}
%==================================
Le champ de vitesse g�n�r� � l'aide de la relation \ref{Base_Vortex_compvit} ne tient pas
compte {\em a priori} des conditions aux limites appliqu�es sur les bords du plan
d'entr�e. Pour obtenir des valeurs de la vitesse qui soient coh�rentes sur les
fronti�res du domaine d'entr�e, des ``vortex images'', pseudo-vortex situ�s en
dehors du domaine d'entr�e, sont g�n�r�s � des positions particuli�res et leur
champ de vitesse associ� est superpos� au champ pr�c�demment calcul�.\\
Seuls les cas d'une conduite rectangulaire et d'une conduite circulaire
permettent la g�n�ration de ces pseudo-vortex.
On distingue pour cela trois types de conditions aux limites. 

\begin{figure}[h]
\centerline{\includegraphics[height=6cm]{../Base/Vortex/Images/condlimite.pdf}}
\caption{\label{Base_Vortex_condli} Principe de g�n�ration des ``vortex images'' suivant le
type de conditions aux limites dans une conduite carr�e.} 
\end{figure}

%----------------------------------
\subsubsection{Condition de paroi}
%----------------------------------
On cr�e, pour chaque vortex $P$ contenu dans le plan d'entr�e, un vortex image
$P'$ identique � $P$ (\textit{i.e} de m�me caract�ristiques) et sym�trique de $P$
par rapport au
point $J$ ($J$ �tant la projection orthogonalement � la paroi du point $M$
correspondant au centre de la face o� l'on cherche � calculer la vitesse). La
figure \ref{Base_Vortex_condli} illustre la technique dans le cas d'une conduite
carr�e. Dans ce cas les coordonn�es du vortex situ� en $P'$ v�rifient
$(y_{p'}+y_{p})/2 = y_{J}$ et $(z_{p'}+ z_{p})/2 = z_{J}$. Le champ de vitesse
per�u depuis le point $M$ au niveau du point $J$ est nul, ce qui est bien
l'effet recherch�. 

%------------------------------------
\subsubsection{Condition de sym�trie}
%-------------------------------------
La technique est identique � celle utilis�e pour les conditions de paroi, mais
seule la composante pour la vitesse normale au bord est modifi�e dans ce cas. 

%---------------------------------------
\subsubsection{Condition de p�riodicit�}
%---------------------------------------
On cr�e pour chaque vortex, un vortex images $P'$ identique � $P$ mais translat�
d'une quantit� $L$ correspondant � la longueur qui s�pare les deux plans de la
section d'entr�e o� sont appliqu�es les conditions de p�riodicit�. Dans le cas
o� il y a deux directions de p�riodicit�, on cr�e deux vortex image.

%=============================================
\subsection{Composante de vitesse principale}
%=============================================
La m�thode des vortex ne g�n�rant pas de fluctuation $u$ de la vitesse dans la
direction principale, la derni�re composante est calcul�e � partir d'une
�quation de Langevin. Les coefficients de cette �quation sont d�termin�s par
identification des expressions obtenues pour les contraintes de Reynolds en
$R_{ij}-\varepsilon$. Dans le cas d'un �coulement en canal plan, cette technique
conduit � l'�quation : 
\begin{equation}\notag
\displaystyle
\frac{du}{dt} = - \frac{C_1}{2T} u + \left(\frac{2}{3}C_2-1\right)\frac{\partial
U}{\partial y} v + \sqrt{C_0\varepsilon} dW_i 
\end{equation}

avec $\displaystyle T=\frac{k}{\varepsilon}$, $C_1 = 1,8$, $C_2 = 0,6$,
$C_0=\frac{14}{15}$, et $dW_i$ une variable al�toire Gaussienne de variance
$\sqrt{dt}$. 

En pratique, l'�quation de Langevin n'am�liore pas vraiment les r�sultats. Elle
n'est utilis�e que dans le cas de conduites circulaires. 

%                      Code_Saturne version 1.3
%                      ------------------------
%
%     This file is part of the Code_Saturne Kernel, element of the
%     Code_Saturne CFD tool.
%
%     Copyright (C) 1998-2007 EDF S.A., France
%
%     contact: saturne-support@edf.fr
%
%     The Code_Saturne Kernel is free software; you can redistribute it
%     and/or modify it under the terms of the GNU General Public License
%     as published by the Free Software Foundation; either version 2 of
%     the License, or (at your option) any later version.
%
%     The Code_Saturne Kernel is distributed in the hope that it will be
%     useful, but WITHOUT ANY WARRANTY; without even the implied warranty
%     of MERCHANTABILITY or FITNESS FOR A PARTICULAR PURPOSE.  See the
%     GNU General Public License for more details.
%
%     You should have received a copy of the GNU General Public License
%     along with the Code_Saturne Kernel; if not, write to the
%     Free Software Foundation, Inc.,
%     51 Franklin St, Fifth Floor,
%     Boston, MA  02110-1301  USA
%
%-----------------------------------------------------------------------
%

%%%%%%%%%%%%%%%%%%%%%%%%%%%%%%%%%%
%%%%%%%%%%%%%%%%%%%%%%%%%%%%%%%%%%
\section{Mise en \oe uvre}
%%%%%%%%%%%%%%%%%%%%%%%%%%%%%%%%%%
%%%%%%%%%%%%%%%%%%%%%%%%%%%%%%%%%%
Le syst\`eme (\ref{Cfbl_Cfmsvl_eq_densite_finale_cfmsvl}) est r\'esolu par une m\'ethode
d'incr\'ement et r\'esidu en utilisant
une m\'ethode de Jacobi pour inverser le syst\`eme si le terme convectif
est implicite et en utilisant une m\'ethode de gradient conjugu\'e
si le terme convectif est explicite (qui est le cas par d�faut).

Attention, les valeurs du flux de masse $\rho\,\vect{w}\cdot\vect{S}$ et
de la viscosit\'e $\Delta\,t\,c^2\frac{S}{d}$ aux faces de
bord, qui sont calcul\'ees dans \fort{cfmsfl} et \fort{cfmsvs} respectivement,
sont modifi\'ees imm\'ediatement apr\`es l'appel \`a ces sous-programmes.
En effet, il est indispensable que la contribution de bord de
$\left(\rho\,\vect{w}-\Delta\,t\,(c^2)\,\gradv\,\rho\right)\cdot\vect{S}$
repr\'esente exactement $\vect{Q}_{ac}\cdot\vect{S}$.
Pour cela,
\begin{itemize}
\item imm\'ediatement apr\`es l'appel \`a
\fort{cfmsfl}, on remplace la contribution de bord de
$\rho\,\vect{w}\cdot\vect{S}$
par le flux de masse exact, $\vect{Q}_{ac}\cdot\vect{S}$,
d\'etermin\'e \`a partir des conditions aux limites,
\item puis, imm\'ediatement apr\`es l'appel \`a
\fort{cfmsvs}, on annule la viscosit\'e au bord $\Delta\,t\,(c^2)$ pour
\'eliminer la contribution de $-\Delta\,t\,(c^2)\,(\gradv\,\rho)\cdot\vect{S}$
(l'annulation de la viscosit\'e n'est pas probl\'ematique pour la matrice,
puisqu'elle porte sur des incr\'ements).
\end{itemize}

\bigskip

Une fois qu'on a obtenu $\rho^{n+1}$,
on peut actualiser le flux de masse acoustique
aux faces $(\vect{Q}_{ac}^{n+1})_{ij} \cdot \vect{S}_{ij}$,
qui servira pour la convection des autres variables~:
\begin{equation}\label{Cfbl_Cfmsvl_eq_flux_masse_acoustique_cfmsvl}
\displaystyle(\vect{Q}_{ac}^{n+1})_{ij}\cdot\vect{S}_{ij}=
-\left(\Delta t^n (c^2)^n \gradv(\rho^{n+1})\right)_{ij}\cdot\vect{S}_{ij}
+\left(\rho^{n+\frac{1}{2}} \vect{w}^n\right)_{ij}\cdot\vect{S}_{ij}\\
\end{equation}
Ce calcul de flux est r\'ealis\'e par \fort{cfbsc3}.
Si l'on a choisi l'algorithme standard, \'equation~(\ref{Cfbl_Cfmsvl_eq_densite_cfmsvl}),
on compl\`ete le flux dans \fort{cfmsvl} imm\'ediatement apr\`es l'appel
\`a \fort{cfbsc3}.
En effet, dans ce cas,
la convection est explicite ($\rho^{n+\frac{1}{2}}=\rho^{n}$,
obtenu en imposant \var{ICONV(ISCA(IRHO(IPHAS)))=0})
et le sous-programme \fort{cfbsc3},
qui calcule le flux de masse aux faces,
ne prend pas en compte la contribution du terme
$\rho^{n+\frac{1}{2}}\,\vect{w}^n\cdot\vect{S}$. On ajoute donc cette
contribution dans \fort{cfmsvl}, apr\`es l'appel \`a \fort{cfbsc3}.
Au bord, en particulier, c'est bien le flux de masse calcul\'e \`a partir
des conditions aux limites que l'on obtient.

On actualise la pression \`a la fin de l'\'etape, en utilisant la loi d'\'etat~:
\begin{equation}
\displaystyle P_i^{pred}=P(\rho_i^{n+1},\varepsilon_i^{n})
\end{equation}


%%%%%%%%%%%%%%%%%%%%%%%%%%%%%%%%%%
%%%%%%%%%%%%%%%%%%%%%%%%%%%%%%%%%%
\section{Points \`a traiter}
%%%%%%%%%%%%%%%%%%%%%%%%%%%%%%%%%%
%%%%%%%%%%%%%%%%%%%%%%%%%%%%%%%%%%
Le calcul du flux de masse au  bord n'est pas enti\`erement satisfaisant
si la convection est trait\'ee de mani\`ere implicite
(algorithme non standard, non test\'e,
associ\'e \`a l'\'equation~(\ref{Cfbl_Cfmsvl_eq_densite_bis_cfmsvl}),
correspondant au choix $\rho^{n+\frac{1}{2}}=\rho^{n+1}$ et
obtenu en imposant \var{ICONV(ISCA(IRHO(IPHAS)))=1}).
En effet, apr\`es \fort{cfmsfl}, il faut d\'eterminer la vitesse de
convection $\vect{w}^n$ pour qu'apparaisse
$\rho^{n+1} \vect{w}^n\cdot\vect{n}$
au cours de la r\'esolution par \fort{codits}. De ce fait, on doit d\'eduire
une valeur de $\vect{w}^n$ \`a partir de la valeur
du flux de masse. Au bord, en particulier, il faut
donc diviser le flux de masse
issu des conditions aux limites par la valeur de bord de $\rho^{n+1}$.
Or, lorsque des conditions de Neumann sont appliqu\'ees \`a la
masse volumique,
la valeur de $\rho^{n+1}$ au bord n'est pas connue avant la r\'esolution du
syst\`eme. On utilise donc, au lieu de la valeur de bord inconnue de
$\rho^{n+1}$ la valeur de bord prise au pas de temps
pr\'ec\'edent $\rho^{n}$. Cette approximation est susceptible
d'affecter la valeur du flux de masse au bord.

%                      Code_Saturne version 1.3
%                      ------------------------
%
%     This file is part of the Code_Saturne Kernel, element of the
%     Code_Saturne CFD tool.
%
%     Copyright (C) 1998-2007 EDF S.A., France
%
%     contact: saturne-support@edf.fr
%
%     The Code_Saturne Kernel is free software; you can redistribute it
%     and/or modify it under the terms of the GNU General Public License
%     as published by the Free Software Foundation; either version 2 of
%     the License, or (at your option) any later version.
%
%     The Code_Saturne Kernel is distributed in the hope that it will be
%     useful, but WITHOUT ANY WARRANTY; without even the implied warranty
%     of MERCHANTABILITY or FITNESS FOR A PARTICULAR PURPOSE.  See the
%     GNU General Public License for more details.
%
%     You should have received a copy of the GNU General Public License
%     along with the Code_Saturne Kernel; if not, write to the
%     Free Software Foundation, Inc.,
%     51 Franklin St, Fifth Floor,
%     Boston, MA  02110-1301  USA
%
%-----------------------------------------------------------------------
%


\programme{navsto}

\vspace{1cm}
On s'int\'eresse \`a la r\'esolution du syst\`eme d'\'equations de Navier-Stokes
tridimensionnelles monophasiques, \`a une pression, instationnaires, en
incompressible ou faiblement dilatable, bas\'ees sur une discr\'etisation
temporelle de type Euler implicite d'ordre 1 ou Crank-Nicolson d'ordre 2 et sur
une discr\'etisation spatiale  par volumes finis colocalis\'ee.


%%%%%%%%%%%%%%%%%%%%%%%%%%%%%%%%%%
%%%%%%%%%%%%%%%%%%%%%%%%%%%%%%%%%%
\section{Fonction}
%%%%%%%%%%%%%%%%%%%%%%%%%%%%%%%%%%
%%%%%%%%%%%%%%%%%%%%%%%%%%%%%%%%%%

  Dans ce sous-programme sont calcul\'ees, \`a un pas de temps donn\'e, les
variables vitesse et pression de ce probl\`eme en proc\'edant en
deux  \'etapes issues d'une d\'ecomposition des op\'erateurs (m\'ethode \`a
pas fractionnaires).\\
Les variables sont donc suppos\'ees connues \`a
l'instant ${t^n}$ et on cherche \`a les d\'eterminer \`a l'instant\footnote{La pression est suppos�e connue � l'instant $t^{n-1+\theta}$ et recherch�e en $t^{n+\theta}$, avec $\theta=1$ ou $1/2$ suivant le sch�ma en temps consid�r�.} ${t^{n+1}}$. Soit ${\Delta {t^n} ={t^{n+1}- {t^n}}}$ le pas de temps associ\'e. Dans un premier temps, on r\'ealise l'\'etape de
pr\'ediction de la vitesse en r\'esolvant l'\'equation de quantit\'e de
mouvement avec une pression explicite. Suit l'\'etape de correction de la
pression (ou projection de la vitesse) qui permet d'obtenir un champ de vitesse \`a divergence nulle.\\\\
Les \'equations en continu sont donc :
\begin{equation}
\left\{\begin{array}{l}
\displaystyle\frac{\partial}{\partial t}(\rho \vect{u}) + \dive(\rho\, \vect{u} \otimes \vect{u})
=\dive(\tens{\sigma}) + \vect{TS} - \tens{K}\,\vect{u}\\
\dive(\rho \vect{u}) = \Gamma
\end{array}\right.
\end{equation}

%(plus tard $\frac{\partial \rho}{\partial t} + \dive(\rho \vect{u}) = \Gamma$)



avec $\rho$ la masse volumique, $\vect{u}$ le champ de vitesse,
$[\,\vect{TS}-\tens{K}\,\vect{u}\,]$ les autres termes sources ($\tens{K}$~est un
tenseur diagonal positif par d\'efinition), $\tens{\sigma}$ le tenseur
de contraintes, $\tens{\tau}$ le tenseur des contraintes visqueuses, $\mu$ la
viscosit\'e dynamique (mol\'eculaire et \'eventuellement turbulente), $\kappa$
la viscosit� de
volume (usuellement nulle et n�glig�e dans le code et dans la suite du document,
sauf en compressible),
$\tens{D}$ le tenseur taux de d\'eformation\footnote{\`A ne pas confondre, malgr\'e la m\^eme notation $D$,
avec les flux diffusifs $\vect{D}_{\,ij}$ et $\vect{D}_{\,{b}_{ik}}$ d\'ecrits par la suite dans ce
sous-programme.}, $\Gamma$ le terme source de masse.
\begin{equation}
\left\{\begin{array}{l}
\tens{\sigma} = \tens{\tau} - P\tens{Id}  \\
\tens{\tau} = 2\,\mu\ \tens{D} +\ (\kappa\ - \frac{2}{3}\mu)\  tr({\tens{D}})\
\tens{Id}  \\
\tens{D} = \frac{1}{2}(\ggrad\vect{u}+\,^{t}\ggrad\vect{u})
\end{array}\right.
\end{equation}
 \\

On rappelle la d\'efinition des notations employ\'ees\footnote{en
utilisant la convention de sommation d'Einstein.}~:
\begin{equation}\notag
\left\{\begin{array}{lll}
\left[\ggrad{\vect{a}}\right]_{ij} &=& \partial_j a_i\\
\left[\dive(\tens{\sigma})\right]_i &=& \partial_j \sigma_{ij}\\
\left[\vect{a}\otimes\vect{b}\right]_{ij} &= &
a_i\,b_j\\
\end{array}\right.
\end{equation}
et donc :
\begin{equation}\notag
\begin{array}{lll}
\left[\dive(\vect{a}\otimes\vect{b})\right]_i &= &
\partial_j (a_i\,b_j)
\end{array}
\end{equation}

\minititre{Remarque}
Dans le cas de la prise en compte d'une masse volumique variable, l'�quation de continuit� s'�crit :
$$\frac{\partial \rho}{\partial t} + \dive{\,(\rho\,\vect{u})} = \Gamma  $$
Cette �quation n'est pas prise en compte dans l'�tape de projection (on continue � r�soudre
seulement
$\displaystyle \dive(\,{\rho\,\vect{u}}) = \Gamma$) alors que le terme
$\displaystyle \frac{\partial \rho}{\partial t}$ appara\^{\i}t lors de l'�tape de pr\'ediction de la vitesse
dans le sous-programme \fort{preduv}. Si ce terme joue un r�le sensible, l'algorithme compressible
de \CS\ (qui r�sout l'�quation compl�te) est alors sans doute plus adapt�.

%                      Code_Saturne version 1.3
%                      ------------------------
%
%     This file is part of the Code_Saturne Kernel, element of the
%     Code_Saturne CFD tool.
% 
%     Copyright (C) 1998-2007 EDF S.A., France
%
%     contact: saturne-support@edf.fr
% 
%     The Code_Saturne Kernel is free software; you can redistribute it
%     and/or modify it under the terms of the GNU General Public License
%     as published by the Free Software Foundation; either version 2 of
%     the License, or (at your option) any later version.
% 
%     The Code_Saturne Kernel is distributed in the hope that it will be
%     useful, but WITHOUT ANY WARRANTY; without even the implied warranty
%     of MERCHANTABILITY or FITNESS FOR A PARTICULAR PURPOSE.  See the
%     GNU General Public License for more details.
% 
%     You should have received a copy of the GNU General Public License
%     along with the Code_Saturne Kernel; if not, write to the
%     Free Software Foundation, Inc.,
%     51 Franklin St, Fifth Floor,
%     Boston, MA  02110-1301  USA
%
%-----------------------------------------------------------------------
%
%%%%%%%%%%%%%%%%%%%%%%%%%%%%%%%%%
%%%%%%%%%%%%%%%%%%%%%%%%%%%%%%%%%%
\section{Discr\'etisation}
%%%%%%%%%%%%%%%%%%%%%%%%%%%%%%%%%%
%%%%%%%%%%%%%%%%%%%%%%%%%%%%%%%%%%

Pour utiliser la m�thode, on se place tout d'abord dans un rep�re local d�fini
de mani�re � ce que le plan $(0yz)$, o� sont inject�s les vortex, soit confondu
avec le plan d'entr�e du calcul (voir figure \ref{Base_Vortex_entree}). 

\begin{figure}[h]
\centerline{\includegraphics[height=6cm]{../Base/Vortex/Images/entree.pdf}}
\caption{\label{Base_Vortex_entree} D�finiton des diff�rentes grandeurs dans le rep�re local
correspondant � l'entr�e d'une conduite de section carr�e.} 
\end{figure}

$u$, $v$ et $w$  sont les composantes de la vitesse fluctuante (principale et
transverse) dans ce plan, et
$\displaystyle \omega(y,z) = \frac{\partial w}{\partial y}-\frac{\partial v}{\partial z}$
la vorticit� dans la direction
normale au plan d'entr�e. $\overline{U}(y,z)$ repr�sente ici la vitesse
principale moyenne impos�e par l'utilisateur dans le plan d'entr�e. 

Chaque vortex $p$ va �tre caract�ris� par sa fonction de forme $\xi$ (identique
pour tous les vortex), sa
circulation $\Gamma_p$, son rayon $\sigma_p$ et les coordonn�es $(y_p,z_p)$ du
point $P$ o� est situ� le vortex dans le plan $(0yz)$. 

Pour cela, on suppose que la vorticit� g�n�r�e par un vortex $p$ au point $M$ de
coordonn�e $(y,z)$ s'�crit : 
\begin{equation}\notag
\omega_p(y,z)= \Gamma_p \, \xi_{\sigma_p}(r)
\end{equation}
o� $r = \sqrt{(y-y_p)^2+(z-z_p)^2}$ est la distance s�parant le point $M$ du point $P$.

Dans la m�thode implant�e, la fonction de forme est de type gaussienne modifi�e :
\begin{equation}\notag
\displaystyle
\xi_\sigma (r) = \frac{1}{2\pi \sigma^2} 
\left(2 e^{-\frac{r^2}{2\sigma^2}}-1\right) e^{-\frac{r^2}{2\sigma^2}}
\end{equation}

Le champ de vitesse induit par cette distribution de vorticit� s'obtient par
inversion des deux �quations de poisson suivantes qui sont d�duites de la
condition d'incompressibilit� dans la plan\footnote{\textit{i.e}
$\displaystyle \frac{\partial v}{\partial y}+\frac{\partial w}{\partial w} = 0$} :
\begin{equation}\notag
\begin{array}{lcr}
\displaystyle
\frac{\partial \omega}{\partial y} = \Delta w
&
\text{    et    }
&
\displaystyle
\frac{\partial \omega}{\partial y} = -\Delta v
\\
\end{array}
\end{equation}

Dans le cas g�n�ral, ce syst�me peut �tre int�gr� � l'aide de la formule de Biot et Savart.

Dans le cas d'une distribution de vorticit� de type gaussienne modifi�e, les
composantes de vitesse v�rifient : 
\begin{equation}\notag
\left\{
\begin{array}{c}
\displaystyle
v_p(y,x) = - \frac{1}{2\pi} \frac{(z-z_p)}{r^2}\left(1 -
e^{-\frac{r^2}{2\sigma^2}}\right)\,e^{-\frac{r^2}{2\sigma^2}} 
\\
\displaystyle
w_p(y,z) = \frac{1}{2\pi} \frac{(y-y_p)}{r^2}\left(1 -e^{-\frac{r^2}{2\sigma^2}}
\right)\,e^{-\frac{r^2}{2\sigma^2}} 
\end{array}
\right.
\end{equation}

Ces relations s'�tendent de fa�on imm�diate au cas de $N$ vortex distincts.
Le champ de vitesse induit par la distribution de vorticit� 
\begin{equation}
\omega(y,z) = \sum_{p=1}^N \Gamma_p \, \xi_{\sigma_p}(r)
\end{equation}
vaut au point $M$ :
\begin{equation}\notag
\begin{array}{lcr}
v(x,y) = \sum_{p=1}^N \Gamma_p\, v_p(y,z) 
&
\text{    et    }
&
w(y,z) = \sum_{p=1}^N \Gamma_p\, w_p(y,z)
\\
\label{Base_Vortex_compvit}
\end{array}
\end{equation}
%================================
\subsection{Param�tres physiques}
%================================

%-------------------------------
\subsubsection{Marche en temps}
%-------------------------------
La position initiale de chaque vortex est tir�e de mani�re al�atoire. On calcul
les d�placements successifs de chacun des vortex dans le plan d'entr�e par
int�gration explicite du champ de vitesse lagrangien : 
\begin{equation}\notag
\begin{array}{lcr}
\displaystyle
\frac{dy_p}{dt} = V(y,z)
&
\text{    et    }
&
\displaystyle
\frac{dz_p}{dt} = W(y,z)
\\
\end{array}
\end{equation}
Les vortex sont alors assimil�s � des particules ponctuelles qui sont convect�es
par le champ $(V,W)$. Ce champ peut �tre impos� par des tirages al�atoires ou
bien d�duit de la vitesse induite par les vortex dans le plan d'entr�e. Dans ce
cas $V(x,y) = \overline{V}(y,z) + v (y,z)$ et $W(y,z)= \overline{W}(y,z) +
w(y,z)$ o� $\overline{V}$ et $\overline{W}$ sont les composantes de la vitesse
transverse moyenne qu'impose l'utilisateur � l'aide des fichiers de donn�es. 

%---------------------------------------------------
\subsubsection{Intensit� et dur�e de vie des vortex}
%---------------------------------------------------
Il serait possible, � partir de l'�quation de transport de la vorticit�,
d'obtenir un mod�le d'�volution pour l'intensit� du vecteur tourbillon
$\omega_p$ associ� � chacun des vortex. En pratique, on pr�f�re utiliser un
mod�le simplifi� dans lequel la circulation des tourbillons ne d�pend que de la
postion de ces derniers � l'instant consid�r�. La circulation initiale de chaque
vortex est alors obtenue � partir de la relation : 
\begin{equation}\notag
|\Gamma_p| = 4 \sqrt{\frac{\pi\,S\,k}{3N\,[2ln(3)-3ln(2)]}}
\end{equation}
o� $S$ est la surface du plan d'entr�e, $N$ le nombre de vortex, et $k$
l'�nergie cin�tique turbulente au point o� se trouve le vortex � l'instant
consid�r�. Le signe de $\Gamma_p$ correspond au sens de rotation du vortex et
est tir� al�atoirement. 

Ce param�tre est celui qui contr�le l'intensit� des fluctuations. Sa d�pendance
en $k$ exprime que, plus l'�coulement est turbulent, plus les vortex sont
intenses. La valeur de $k$ est sp�cifi�e par
l'utilisateur. Elle peut �tre constante ou impos�e � partir de profils d'�nergie
cin�tique turbulente en entr�e. 

Pour �viter que des structures trop allong�es ne se d�veloppent au niveau de
l'entr�e, l'utilisateur doit �galement sp�cifier un temps limites $\tau_p$ au
bout duquel le vortex $p$ va �tre d�truit. Ce temps $\tau_p$ peut �tre pris
constant ou estim� au moyen de la relation : 
\begin{equation}\notag
\tau_p = \frac{5 C_{\mu}k^{\frac{3}{2}}}{\varepsilon\,\overline{U}}
\end{equation}

$\overline{U}$ et $\varepsilon$ repr�sentent respectivement la vitesse moyenne
principale et la dissipation turbulente au point o� est initialement g�n�r� le
vortex ($C_{\mu}=0,09$). 
\\
Lorsque le temps �coul� depuis la cr�ation du vortex $p$ est sup�rieur �
$\tau_p$, le vortex est d�truit et un nouveau vortex g�n�r� (sa position et le
signe de $\Gamma_p$ sont tir�s de fa�on al�atoire). 

%-------------------------------- 
\subsubsection{Taille des vortex}
%--------------------------------
La taille des vortex peut �tre prise constante, ou calcul�e � partir des
relations :
\begin{equation}\notag
\begin{array}{ccc}
\displaystyle
\sigma = \frac{C_{\mu}^{\frac{3}{4}}k^{\frac{3}{2}}}{\varepsilon} 
& \text{    ou    } &
\sigma = max[L_t,L_k]
\\
\end{array}
\end{equation}
avec:
\begin{equation}\notag
\begin{array}{ccc}
\displaystyle
L_t = \sqrt{\left( \frac{5 \nu k}{\varepsilon} \right)} 
& \text{    et    } & 
\displaystyle
L_k = 200\, \left(\frac{\nu^3}{\varepsilon}\right)^{\frac{1}{4}}
\end{array}
\end{equation}
o� $\nu$, $k$ et $\varepsilon$ sont la viscosit� dynamique, l'�nergie cin�tique
turbulente et la dissipation turbulente au point o� se trouve le vortex. 

Dans tous les cas, la taille du vortex doit �tre sup�rieure � la taille des
mailles en entr�e afin que le vortex soit effectivement simul�. 

%==================================
\subsection{Conditions aux limites}
%==================================
Le champ de vitesse g�n�r� � l'aide de la relation \ref{Base_Vortex_compvit} ne tient pas
compte {\em a priori} des conditions aux limites appliqu�es sur les bords du plan
d'entr�e. Pour obtenir des valeurs de la vitesse qui soient coh�rentes sur les
fronti�res du domaine d'entr�e, des ``vortex images'', pseudo-vortex situ�s en
dehors du domaine d'entr�e, sont g�n�r�s � des positions particuli�res et leur
champ de vitesse associ� est superpos� au champ pr�c�demment calcul�.\\
Seuls les cas d'une conduite rectangulaire et d'une conduite circulaire
permettent la g�n�ration de ces pseudo-vortex.
On distingue pour cela trois types de conditions aux limites. 

\begin{figure}[h]
\centerline{\includegraphics[height=6cm]{../Base/Vortex/Images/condlimite.pdf}}
\caption{\label{Base_Vortex_condli} Principe de g�n�ration des ``vortex images'' suivant le
type de conditions aux limites dans une conduite carr�e.} 
\end{figure}

%----------------------------------
\subsubsection{Condition de paroi}
%----------------------------------
On cr�e, pour chaque vortex $P$ contenu dans le plan d'entr�e, un vortex image
$P'$ identique � $P$ (\textit{i.e} de m�me caract�ristiques) et sym�trique de $P$
par rapport au
point $J$ ($J$ �tant la projection orthogonalement � la paroi du point $M$
correspondant au centre de la face o� l'on cherche � calculer la vitesse). La
figure \ref{Base_Vortex_condli} illustre la technique dans le cas d'une conduite
carr�e. Dans ce cas les coordonn�es du vortex situ� en $P'$ v�rifient
$(y_{p'}+y_{p})/2 = y_{J}$ et $(z_{p'}+ z_{p})/2 = z_{J}$. Le champ de vitesse
per�u depuis le point $M$ au niveau du point $J$ est nul, ce qui est bien
l'effet recherch�. 

%------------------------------------
\subsubsection{Condition de sym�trie}
%-------------------------------------
La technique est identique � celle utilis�e pour les conditions de paroi, mais
seule la composante pour la vitesse normale au bord est modifi�e dans ce cas. 

%---------------------------------------
\subsubsection{Condition de p�riodicit�}
%---------------------------------------
On cr�e pour chaque vortex, un vortex images $P'$ identique � $P$ mais translat�
d'une quantit� $L$ correspondant � la longueur qui s�pare les deux plans de la
section d'entr�e o� sont appliqu�es les conditions de p�riodicit�. Dans le cas
o� il y a deux directions de p�riodicit�, on cr�e deux vortex image.

%=============================================
\subsection{Composante de vitesse principale}
%=============================================
La m�thode des vortex ne g�n�rant pas de fluctuation $u$ de la vitesse dans la
direction principale, la derni�re composante est calcul�e � partir d'une
�quation de Langevin. Les coefficients de cette �quation sont d�termin�s par
identification des expressions obtenues pour les contraintes de Reynolds en
$R_{ij}-\varepsilon$. Dans le cas d'un �coulement en canal plan, cette technique
conduit � l'�quation : 
\begin{equation}\notag
\displaystyle
\frac{du}{dt} = - \frac{C_1}{2T} u + \left(\frac{2}{3}C_2-1\right)\frac{\partial
U}{\partial y} v + \sqrt{C_0\varepsilon} dW_i 
\end{equation}

avec $\displaystyle T=\frac{k}{\varepsilon}$, $C_1 = 1,8$, $C_2 = 0,6$,
$C_0=\frac{14}{15}$, et $dW_i$ une variable al�toire Gaussienne de variance
$\sqrt{dt}$. 

En pratique, l'�quation de Langevin n'am�liore pas vraiment les r�sultats. Elle
n'est utilis�e que dans le cas de conduites circulaires. 

%                      Code_Saturne version 1.3
%                      ------------------------
%
%     This file is part of the Code_Saturne Kernel, element of the
%     Code_Saturne CFD tool.
%
%     Copyright (C) 1998-2007 EDF S.A., France
%
%     contact: saturne-support@edf.fr
%
%     The Code_Saturne Kernel is free software; you can redistribute it
%     and/or modify it under the terms of the GNU General Public License
%     as published by the Free Software Foundation; either version 2 of
%     the License, or (at your option) any later version.
%
%     The Code_Saturne Kernel is distributed in the hope that it will be
%     useful, but WITHOUT ANY WARRANTY; without even the implied warranty
%     of MERCHANTABILITY or FITNESS FOR A PARTICULAR PURPOSE.  See the
%     GNU General Public License for more details.
%
%     You should have received a copy of the GNU General Public License
%     along with the Code_Saturne Kernel; if not, write to the
%     Free Software Foundation, Inc.,
%     51 Franklin St, Fifth Floor,
%     Boston, MA  02110-1301  USA
%
%-----------------------------------------------------------------------
%

%%%%%%%%%%%%%%%%%%%%%%%%%%%%%%%%%%
%%%%%%%%%%%%%%%%%%%%%%%%%%%%%%%%%%
\section{Mise en \oe uvre}
%%%%%%%%%%%%%%%%%%%%%%%%%%%%%%%%%%
%%%%%%%%%%%%%%%%%%%%%%%%%%%%%%%%%%
Le syst\`eme (\ref{Cfbl_Cfmsvl_eq_densite_finale_cfmsvl}) est r\'esolu par une m\'ethode
d'incr\'ement et r\'esidu en utilisant
une m\'ethode de Jacobi pour inverser le syst\`eme si le terme convectif
est implicite et en utilisant une m\'ethode de gradient conjugu\'e
si le terme convectif est explicite (qui est le cas par d�faut).

Attention, les valeurs du flux de masse $\rho\,\vect{w}\cdot\vect{S}$ et
de la viscosit\'e $\Delta\,t\,c^2\frac{S}{d}$ aux faces de
bord, qui sont calcul\'ees dans \fort{cfmsfl} et \fort{cfmsvs} respectivement,
sont modifi\'ees imm\'ediatement apr\`es l'appel \`a ces sous-programmes.
En effet, il est indispensable que la contribution de bord de
$\left(\rho\,\vect{w}-\Delta\,t\,(c^2)\,\gradv\,\rho\right)\cdot\vect{S}$
repr\'esente exactement $\vect{Q}_{ac}\cdot\vect{S}$.
Pour cela,
\begin{itemize}
\item imm\'ediatement apr\`es l'appel \`a
\fort{cfmsfl}, on remplace la contribution de bord de
$\rho\,\vect{w}\cdot\vect{S}$
par le flux de masse exact, $\vect{Q}_{ac}\cdot\vect{S}$,
d\'etermin\'e \`a partir des conditions aux limites,
\item puis, imm\'ediatement apr\`es l'appel \`a
\fort{cfmsvs}, on annule la viscosit\'e au bord $\Delta\,t\,(c^2)$ pour
\'eliminer la contribution de $-\Delta\,t\,(c^2)\,(\gradv\,\rho)\cdot\vect{S}$
(l'annulation de la viscosit\'e n'est pas probl\'ematique pour la matrice,
puisqu'elle porte sur des incr\'ements).
\end{itemize}

\bigskip

Une fois qu'on a obtenu $\rho^{n+1}$,
on peut actualiser le flux de masse acoustique
aux faces $(\vect{Q}_{ac}^{n+1})_{ij} \cdot \vect{S}_{ij}$,
qui servira pour la convection des autres variables~:
\begin{equation}\label{Cfbl_Cfmsvl_eq_flux_masse_acoustique_cfmsvl}
\displaystyle(\vect{Q}_{ac}^{n+1})_{ij}\cdot\vect{S}_{ij}=
-\left(\Delta t^n (c^2)^n \gradv(\rho^{n+1})\right)_{ij}\cdot\vect{S}_{ij}
+\left(\rho^{n+\frac{1}{2}} \vect{w}^n\right)_{ij}\cdot\vect{S}_{ij}\\
\end{equation}
Ce calcul de flux est r\'ealis\'e par \fort{cfbsc3}.
Si l'on a choisi l'algorithme standard, \'equation~(\ref{Cfbl_Cfmsvl_eq_densite_cfmsvl}),
on compl\`ete le flux dans \fort{cfmsvl} imm\'ediatement apr\`es l'appel
\`a \fort{cfbsc3}.
En effet, dans ce cas,
la convection est explicite ($\rho^{n+\frac{1}{2}}=\rho^{n}$,
obtenu en imposant \var{ICONV(ISCA(IRHO(IPHAS)))=0})
et le sous-programme \fort{cfbsc3},
qui calcule le flux de masse aux faces,
ne prend pas en compte la contribution du terme
$\rho^{n+\frac{1}{2}}\,\vect{w}^n\cdot\vect{S}$. On ajoute donc cette
contribution dans \fort{cfmsvl}, apr\`es l'appel \`a \fort{cfbsc3}.
Au bord, en particulier, c'est bien le flux de masse calcul\'e \`a partir
des conditions aux limites que l'on obtient.

On actualise la pression \`a la fin de l'\'etape, en utilisant la loi d'\'etat~:
\begin{equation}
\displaystyle P_i^{pred}=P(\rho_i^{n+1},\varepsilon_i^{n})
\end{equation}


%%%%%%%%%%%%%%%%%%%%%%%%%%%%%%%%%%
%%%%%%%%%%%%%%%%%%%%%%%%%%%%%%%%%%
\section{Points \`a traiter}
%%%%%%%%%%%%%%%%%%%%%%%%%%%%%%%%%%
%%%%%%%%%%%%%%%%%%%%%%%%%%%%%%%%%%
Le calcul du flux de masse au  bord n'est pas enti\`erement satisfaisant
si la convection est trait\'ee de mani\`ere implicite
(algorithme non standard, non test\'e,
associ\'e \`a l'\'equation~(\ref{Cfbl_Cfmsvl_eq_densite_bis_cfmsvl}),
correspondant au choix $\rho^{n+\frac{1}{2}}=\rho^{n+1}$ et
obtenu en imposant \var{ICONV(ISCA(IRHO(IPHAS)))=1}).
En effet, apr\`es \fort{cfmsfl}, il faut d\'eterminer la vitesse de
convection $\vect{w}^n$ pour qu'apparaisse
$\rho^{n+1} \vect{w}^n\cdot\vect{n}$
au cours de la r\'esolution par \fort{codits}. De ce fait, on doit d\'eduire
une valeur de $\vect{w}^n$ \`a partir de la valeur
du flux de masse. Au bord, en particulier, il faut
donc diviser le flux de masse
issu des conditions aux limites par la valeur de bord de $\rho^{n+1}$.
Or, lorsque des conditions de Neumann sont appliqu\'ees \`a la
masse volumique,
la valeur de $\rho^{n+1}$ au bord n'est pas connue avant la r\'esolution du
syst\`eme. On utilise donc, au lieu de la valeur de bord inconnue de
$\rho^{n+1}$ la valeur de bord prise au pas de temps
pr\'ec\'edent $\rho^{n}$. Cette approximation est susceptible
d'affecter la valeur du flux de masse au bord.

%                      Code_Saturne version 1.3
%                      ------------------------
%
%     This file is part of the Code_Saturne Kernel, element of the
%     Code_Saturne CFD tool.
%
%     Copyright (C) 1998-2007 EDF S.A., France
%
%     contact: saturne-support@edf.fr
%
%     The Code_Saturne Kernel is free software; you can redistribute it
%     and/or modify it under the terms of the GNU General Public License
%     as published by the Free Software Foundation; either version 2 of
%     the License, or (at your option) any later version.
%
%     The Code_Saturne Kernel is distributed in the hope that it will be
%     useful, but WITHOUT ANY WARRANTY; without even the implied warranty
%     of MERCHANTABILITY or FITNESS FOR A PARTICULAR PURPOSE.  See the
%     GNU General Public License for more details.
%
%     You should have received a copy of the GNU General Public License
%     along with the Code_Saturne Kernel; if not, write to the
%     Free Software Foundation, Inc.,
%     51 Franklin St, Fifth Floor,
%     Boston, MA  02110-1301  USA
%
%-----------------------------------------------------------------------
%


\programme{navsto}

\vspace{1cm}
On s'int\'eresse \`a la r\'esolution du syst\`eme d'\'equations de Navier-Stokes
tridimensionnelles monophasiques, \`a une pression, instationnaires, en
incompressible ou faiblement dilatable, bas\'ees sur une discr\'etisation
temporelle de type Euler implicite d'ordre 1 ou Crank-Nicolson d'ordre 2 et sur
une discr\'etisation spatiale  par volumes finis colocalis\'ee.


%%%%%%%%%%%%%%%%%%%%%%%%%%%%%%%%%%
%%%%%%%%%%%%%%%%%%%%%%%%%%%%%%%%%%
\section{Fonction}
%%%%%%%%%%%%%%%%%%%%%%%%%%%%%%%%%%
%%%%%%%%%%%%%%%%%%%%%%%%%%%%%%%%%%

  Dans ce sous-programme sont calcul\'ees, \`a un pas de temps donn\'e, les
variables vitesse et pression de ce probl\`eme en proc\'edant en
deux  \'etapes issues d'une d\'ecomposition des op\'erateurs (m\'ethode \`a
pas fractionnaires).\\
Les variables sont donc suppos\'ees connues \`a
l'instant ${t^n}$ et on cherche \`a les d\'eterminer \`a l'instant\footnote{La pression est suppos�e connue � l'instant $t^{n-1+\theta}$ et recherch�e en $t^{n+\theta}$, avec $\theta=1$ ou $1/2$ suivant le sch�ma en temps consid�r�.} ${t^{n+1}}$. Soit ${\Delta {t^n} ={t^{n+1}- {t^n}}}$ le pas de temps associ\'e. Dans un premier temps, on r\'ealise l'\'etape de
pr\'ediction de la vitesse en r\'esolvant l'\'equation de quantit\'e de
mouvement avec une pression explicite. Suit l'\'etape de correction de la
pression (ou projection de la vitesse) qui permet d'obtenir un champ de vitesse \`a divergence nulle.\\\\
Les \'equations en continu sont donc :
\begin{equation}
\left\{\begin{array}{l}
\displaystyle\frac{\partial}{\partial t}(\rho \vect{u}) + \dive(\rho\, \vect{u} \otimes \vect{u})
=\dive(\tens{\sigma}) + \vect{TS} - \tens{K}\,\vect{u}\\
\dive(\rho \vect{u}) = \Gamma
\end{array}\right.
\end{equation}

%(plus tard $\frac{\partial \rho}{\partial t} + \dive(\rho \vect{u}) = \Gamma$)



avec $\rho$ la masse volumique, $\vect{u}$ le champ de vitesse,
$[\,\vect{TS}-\tens{K}\,\vect{u}\,]$ les autres termes sources ($\tens{K}$~est un
tenseur diagonal positif par d\'efinition), $\tens{\sigma}$ le tenseur
de contraintes, $\tens{\tau}$ le tenseur des contraintes visqueuses, $\mu$ la
viscosit\'e dynamique (mol\'eculaire et \'eventuellement turbulente), $\kappa$
la viscosit� de
volume (usuellement nulle et n�glig�e dans le code et dans la suite du document,
sauf en compressible),
$\tens{D}$ le tenseur taux de d\'eformation\footnote{\`A ne pas confondre, malgr\'e la m\^eme notation $D$,
avec les flux diffusifs $\vect{D}_{\,ij}$ et $\vect{D}_{\,{b}_{ik}}$ d\'ecrits par la suite dans ce
sous-programme.}, $\Gamma$ le terme source de masse.
\begin{equation}
\left\{\begin{array}{l}
\tens{\sigma} = \tens{\tau} - P\tens{Id}  \\
\tens{\tau} = 2\,\mu\ \tens{D} +\ (\kappa\ - \frac{2}{3}\mu)\  tr({\tens{D}})\
\tens{Id}  \\
\tens{D} = \frac{1}{2}(\ggrad\vect{u}+\,^{t}\ggrad\vect{u})
\end{array}\right.
\end{equation}
 \\

On rappelle la d\'efinition des notations employ\'ees\footnote{en
utilisant la convention de sommation d'Einstein.}~:
\begin{equation}\notag
\left\{\begin{array}{lll}
\left[\ggrad{\vect{a}}\right]_{ij} &=& \partial_j a_i\\
\left[\dive(\tens{\sigma})\right]_i &=& \partial_j \sigma_{ij}\\
\left[\vect{a}\otimes\vect{b}\right]_{ij} &= &
a_i\,b_j\\
\end{array}\right.
\end{equation}
et donc :
\begin{equation}\notag
\begin{array}{lll}
\left[\dive(\vect{a}\otimes\vect{b})\right]_i &= &
\partial_j (a_i\,b_j)
\end{array}
\end{equation}

\minititre{Remarque}
Dans le cas de la prise en compte d'une masse volumique variable, l'�quation de continuit� s'�crit :
$$\frac{\partial \rho}{\partial t} + \dive{\,(\rho\,\vect{u})} = \Gamma  $$
Cette �quation n'est pas prise en compte dans l'�tape de projection (on continue � r�soudre
seulement
$\displaystyle \dive(\,{\rho\,\vect{u}}) = \Gamma$) alors que le terme
$\displaystyle \frac{\partial \rho}{\partial t}$ appara\^{\i}t lors de l'�tape de pr\'ediction de la vitesse
dans le sous-programme \fort{preduv}. Si ce terme joue un r�le sensible, l'algorithme compressible
de \CS\ (qui r�sout l'�quation compl�te) est alors sans doute plus adapt�.

%                      Code_Saturne version 1.3
%                      ------------------------
%
%     This file is part of the Code_Saturne Kernel, element of the
%     Code_Saturne CFD tool.
% 
%     Copyright (C) 1998-2007 EDF S.A., France
%
%     contact: saturne-support@edf.fr
% 
%     The Code_Saturne Kernel is free software; you can redistribute it
%     and/or modify it under the terms of the GNU General Public License
%     as published by the Free Software Foundation; either version 2 of
%     the License, or (at your option) any later version.
% 
%     The Code_Saturne Kernel is distributed in the hope that it will be
%     useful, but WITHOUT ANY WARRANTY; without even the implied warranty
%     of MERCHANTABILITY or FITNESS FOR A PARTICULAR PURPOSE.  See the
%     GNU General Public License for more details.
% 
%     You should have received a copy of the GNU General Public License
%     along with the Code_Saturne Kernel; if not, write to the
%     Free Software Foundation, Inc.,
%     51 Franklin St, Fifth Floor,
%     Boston, MA  02110-1301  USA
%
%-----------------------------------------------------------------------
%
%%%%%%%%%%%%%%%%%%%%%%%%%%%%%%%%%
%%%%%%%%%%%%%%%%%%%%%%%%%%%%%%%%%%
\section{Discr\'etisation}
%%%%%%%%%%%%%%%%%%%%%%%%%%%%%%%%%%
%%%%%%%%%%%%%%%%%%%%%%%%%%%%%%%%%%

Pour utiliser la m�thode, on se place tout d'abord dans un rep�re local d�fini
de mani�re � ce que le plan $(0yz)$, o� sont inject�s les vortex, soit confondu
avec le plan d'entr�e du calcul (voir figure \ref{Base_Vortex_entree}). 

\begin{figure}[h]
\centerline{\includegraphics[height=6cm]{../Base/Vortex/Images/entree.pdf}}
\caption{\label{Base_Vortex_entree} D�finiton des diff�rentes grandeurs dans le rep�re local
correspondant � l'entr�e d'une conduite de section carr�e.} 
\end{figure}

$u$, $v$ et $w$  sont les composantes de la vitesse fluctuante (principale et
transverse) dans ce plan, et
$\displaystyle \omega(y,z) = \frac{\partial w}{\partial y}-\frac{\partial v}{\partial z}$
la vorticit� dans la direction
normale au plan d'entr�e. $\overline{U}(y,z)$ repr�sente ici la vitesse
principale moyenne impos�e par l'utilisateur dans le plan d'entr�e. 

Chaque vortex $p$ va �tre caract�ris� par sa fonction de forme $\xi$ (identique
pour tous les vortex), sa
circulation $\Gamma_p$, son rayon $\sigma_p$ et les coordonn�es $(y_p,z_p)$ du
point $P$ o� est situ� le vortex dans le plan $(0yz)$. 

Pour cela, on suppose que la vorticit� g�n�r�e par un vortex $p$ au point $M$ de
coordonn�e $(y,z)$ s'�crit : 
\begin{equation}\notag
\omega_p(y,z)= \Gamma_p \, \xi_{\sigma_p}(r)
\end{equation}
o� $r = \sqrt{(y-y_p)^2+(z-z_p)^2}$ est la distance s�parant le point $M$ du point $P$.

Dans la m�thode implant�e, la fonction de forme est de type gaussienne modifi�e :
\begin{equation}\notag
\displaystyle
\xi_\sigma (r) = \frac{1}{2\pi \sigma^2} 
\left(2 e^{-\frac{r^2}{2\sigma^2}}-1\right) e^{-\frac{r^2}{2\sigma^2}}
\end{equation}

Le champ de vitesse induit par cette distribution de vorticit� s'obtient par
inversion des deux �quations de poisson suivantes qui sont d�duites de la
condition d'incompressibilit� dans la plan\footnote{\textit{i.e}
$\displaystyle \frac{\partial v}{\partial y}+\frac{\partial w}{\partial w} = 0$} :
\begin{equation}\notag
\begin{array}{lcr}
\displaystyle
\frac{\partial \omega}{\partial y} = \Delta w
&
\text{    et    }
&
\displaystyle
\frac{\partial \omega}{\partial y} = -\Delta v
\\
\end{array}
\end{equation}

Dans le cas g�n�ral, ce syst�me peut �tre int�gr� � l'aide de la formule de Biot et Savart.

Dans le cas d'une distribution de vorticit� de type gaussienne modifi�e, les
composantes de vitesse v�rifient : 
\begin{equation}\notag
\left\{
\begin{array}{c}
\displaystyle
v_p(y,x) = - \frac{1}{2\pi} \frac{(z-z_p)}{r^2}\left(1 -
e^{-\frac{r^2}{2\sigma^2}}\right)\,e^{-\frac{r^2}{2\sigma^2}} 
\\
\displaystyle
w_p(y,z) = \frac{1}{2\pi} \frac{(y-y_p)}{r^2}\left(1 -e^{-\frac{r^2}{2\sigma^2}}
\right)\,e^{-\frac{r^2}{2\sigma^2}} 
\end{array}
\right.
\end{equation}

Ces relations s'�tendent de fa�on imm�diate au cas de $N$ vortex distincts.
Le champ de vitesse induit par la distribution de vorticit� 
\begin{equation}
\omega(y,z) = \sum_{p=1}^N \Gamma_p \, \xi_{\sigma_p}(r)
\end{equation}
vaut au point $M$ :
\begin{equation}\notag
\begin{array}{lcr}
v(x,y) = \sum_{p=1}^N \Gamma_p\, v_p(y,z) 
&
\text{    et    }
&
w(y,z) = \sum_{p=1}^N \Gamma_p\, w_p(y,z)
\\
\label{Base_Vortex_compvit}
\end{array}
\end{equation}
%================================
\subsection{Param�tres physiques}
%================================

%-------------------------------
\subsubsection{Marche en temps}
%-------------------------------
La position initiale de chaque vortex est tir�e de mani�re al�atoire. On calcul
les d�placements successifs de chacun des vortex dans le plan d'entr�e par
int�gration explicite du champ de vitesse lagrangien : 
\begin{equation}\notag
\begin{array}{lcr}
\displaystyle
\frac{dy_p}{dt} = V(y,z)
&
\text{    et    }
&
\displaystyle
\frac{dz_p}{dt} = W(y,z)
\\
\end{array}
\end{equation}
Les vortex sont alors assimil�s � des particules ponctuelles qui sont convect�es
par le champ $(V,W)$. Ce champ peut �tre impos� par des tirages al�atoires ou
bien d�duit de la vitesse induite par les vortex dans le plan d'entr�e. Dans ce
cas $V(x,y) = \overline{V}(y,z) + v (y,z)$ et $W(y,z)= \overline{W}(y,z) +
w(y,z)$ o� $\overline{V}$ et $\overline{W}$ sont les composantes de la vitesse
transverse moyenne qu'impose l'utilisateur � l'aide des fichiers de donn�es. 

%---------------------------------------------------
\subsubsection{Intensit� et dur�e de vie des vortex}
%---------------------------------------------------
Il serait possible, � partir de l'�quation de transport de la vorticit�,
d'obtenir un mod�le d'�volution pour l'intensit� du vecteur tourbillon
$\omega_p$ associ� � chacun des vortex. En pratique, on pr�f�re utiliser un
mod�le simplifi� dans lequel la circulation des tourbillons ne d�pend que de la
postion de ces derniers � l'instant consid�r�. La circulation initiale de chaque
vortex est alors obtenue � partir de la relation : 
\begin{equation}\notag
|\Gamma_p| = 4 \sqrt{\frac{\pi\,S\,k}{3N\,[2ln(3)-3ln(2)]}}
\end{equation}
o� $S$ est la surface du plan d'entr�e, $N$ le nombre de vortex, et $k$
l'�nergie cin�tique turbulente au point o� se trouve le vortex � l'instant
consid�r�. Le signe de $\Gamma_p$ correspond au sens de rotation du vortex et
est tir� al�atoirement. 

Ce param�tre est celui qui contr�le l'intensit� des fluctuations. Sa d�pendance
en $k$ exprime que, plus l'�coulement est turbulent, plus les vortex sont
intenses. La valeur de $k$ est sp�cifi�e par
l'utilisateur. Elle peut �tre constante ou impos�e � partir de profils d'�nergie
cin�tique turbulente en entr�e. 

Pour �viter que des structures trop allong�es ne se d�veloppent au niveau de
l'entr�e, l'utilisateur doit �galement sp�cifier un temps limites $\tau_p$ au
bout duquel le vortex $p$ va �tre d�truit. Ce temps $\tau_p$ peut �tre pris
constant ou estim� au moyen de la relation : 
\begin{equation}\notag
\tau_p = \frac{5 C_{\mu}k^{\frac{3}{2}}}{\varepsilon\,\overline{U}}
\end{equation}

$\overline{U}$ et $\varepsilon$ repr�sentent respectivement la vitesse moyenne
principale et la dissipation turbulente au point o� est initialement g�n�r� le
vortex ($C_{\mu}=0,09$). 
\\
Lorsque le temps �coul� depuis la cr�ation du vortex $p$ est sup�rieur �
$\tau_p$, le vortex est d�truit et un nouveau vortex g�n�r� (sa position et le
signe de $\Gamma_p$ sont tir�s de fa�on al�atoire). 

%-------------------------------- 
\subsubsection{Taille des vortex}
%--------------------------------
La taille des vortex peut �tre prise constante, ou calcul�e � partir des
relations :
\begin{equation}\notag
\begin{array}{ccc}
\displaystyle
\sigma = \frac{C_{\mu}^{\frac{3}{4}}k^{\frac{3}{2}}}{\varepsilon} 
& \text{    ou    } &
\sigma = max[L_t,L_k]
\\
\end{array}
\end{equation}
avec:
\begin{equation}\notag
\begin{array}{ccc}
\displaystyle
L_t = \sqrt{\left( \frac{5 \nu k}{\varepsilon} \right)} 
& \text{    et    } & 
\displaystyle
L_k = 200\, \left(\frac{\nu^3}{\varepsilon}\right)^{\frac{1}{4}}
\end{array}
\end{equation}
o� $\nu$, $k$ et $\varepsilon$ sont la viscosit� dynamique, l'�nergie cin�tique
turbulente et la dissipation turbulente au point o� se trouve le vortex. 

Dans tous les cas, la taille du vortex doit �tre sup�rieure � la taille des
mailles en entr�e afin que le vortex soit effectivement simul�. 

%==================================
\subsection{Conditions aux limites}
%==================================
Le champ de vitesse g�n�r� � l'aide de la relation \ref{Base_Vortex_compvit} ne tient pas
compte {\em a priori} des conditions aux limites appliqu�es sur les bords du plan
d'entr�e. Pour obtenir des valeurs de la vitesse qui soient coh�rentes sur les
fronti�res du domaine d'entr�e, des ``vortex images'', pseudo-vortex situ�s en
dehors du domaine d'entr�e, sont g�n�r�s � des positions particuli�res et leur
champ de vitesse associ� est superpos� au champ pr�c�demment calcul�.\\
Seuls les cas d'une conduite rectangulaire et d'une conduite circulaire
permettent la g�n�ration de ces pseudo-vortex.
On distingue pour cela trois types de conditions aux limites. 

\begin{figure}[h]
\centerline{\includegraphics[height=6cm]{../Base/Vortex/Images/condlimite.pdf}}
\caption{\label{Base_Vortex_condli} Principe de g�n�ration des ``vortex images'' suivant le
type de conditions aux limites dans une conduite carr�e.} 
\end{figure}

%----------------------------------
\subsubsection{Condition de paroi}
%----------------------------------
On cr�e, pour chaque vortex $P$ contenu dans le plan d'entr�e, un vortex image
$P'$ identique � $P$ (\textit{i.e} de m�me caract�ristiques) et sym�trique de $P$
par rapport au
point $J$ ($J$ �tant la projection orthogonalement � la paroi du point $M$
correspondant au centre de la face o� l'on cherche � calculer la vitesse). La
figure \ref{Base_Vortex_condli} illustre la technique dans le cas d'une conduite
carr�e. Dans ce cas les coordonn�es du vortex situ� en $P'$ v�rifient
$(y_{p'}+y_{p})/2 = y_{J}$ et $(z_{p'}+ z_{p})/2 = z_{J}$. Le champ de vitesse
per�u depuis le point $M$ au niveau du point $J$ est nul, ce qui est bien
l'effet recherch�. 

%------------------------------------
\subsubsection{Condition de sym�trie}
%-------------------------------------
La technique est identique � celle utilis�e pour les conditions de paroi, mais
seule la composante pour la vitesse normale au bord est modifi�e dans ce cas. 

%---------------------------------------
\subsubsection{Condition de p�riodicit�}
%---------------------------------------
On cr�e pour chaque vortex, un vortex images $P'$ identique � $P$ mais translat�
d'une quantit� $L$ correspondant � la longueur qui s�pare les deux plans de la
section d'entr�e o� sont appliqu�es les conditions de p�riodicit�. Dans le cas
o� il y a deux directions de p�riodicit�, on cr�e deux vortex image.

%=============================================
\subsection{Composante de vitesse principale}
%=============================================
La m�thode des vortex ne g�n�rant pas de fluctuation $u$ de la vitesse dans la
direction principale, la derni�re composante est calcul�e � partir d'une
�quation de Langevin. Les coefficients de cette �quation sont d�termin�s par
identification des expressions obtenues pour les contraintes de Reynolds en
$R_{ij}-\varepsilon$. Dans le cas d'un �coulement en canal plan, cette technique
conduit � l'�quation : 
\begin{equation}\notag
\displaystyle
\frac{du}{dt} = - \frac{C_1}{2T} u + \left(\frac{2}{3}C_2-1\right)\frac{\partial
U}{\partial y} v + \sqrt{C_0\varepsilon} dW_i 
\end{equation}

avec $\displaystyle T=\frac{k}{\varepsilon}$, $C_1 = 1,8$, $C_2 = 0,6$,
$C_0=\frac{14}{15}$, et $dW_i$ une variable al�toire Gaussienne de variance
$\sqrt{dt}$. 

En pratique, l'�quation de Langevin n'am�liore pas vraiment les r�sultats. Elle
n'est utilis�e que dans le cas de conduites circulaires. 

%                      Code_Saturne version 1.3
%                      ------------------------
%
%     This file is part of the Code_Saturne Kernel, element of the
%     Code_Saturne CFD tool.
%
%     Copyright (C) 1998-2007 EDF S.A., France
%
%     contact: saturne-support@edf.fr
%
%     The Code_Saturne Kernel is free software; you can redistribute it
%     and/or modify it under the terms of the GNU General Public License
%     as published by the Free Software Foundation; either version 2 of
%     the License, or (at your option) any later version.
%
%     The Code_Saturne Kernel is distributed in the hope that it will be
%     useful, but WITHOUT ANY WARRANTY; without even the implied warranty
%     of MERCHANTABILITY or FITNESS FOR A PARTICULAR PURPOSE.  See the
%     GNU General Public License for more details.
%
%     You should have received a copy of the GNU General Public License
%     along with the Code_Saturne Kernel; if not, write to the
%     Free Software Foundation, Inc.,
%     51 Franklin St, Fifth Floor,
%     Boston, MA  02110-1301  USA
%
%-----------------------------------------------------------------------
%

%%%%%%%%%%%%%%%%%%%%%%%%%%%%%%%%%%
%%%%%%%%%%%%%%%%%%%%%%%%%%%%%%%%%%
\section{Mise en \oe uvre}
%%%%%%%%%%%%%%%%%%%%%%%%%%%%%%%%%%
%%%%%%%%%%%%%%%%%%%%%%%%%%%%%%%%%%
Le syst\`eme (\ref{Cfbl_Cfmsvl_eq_densite_finale_cfmsvl}) est r\'esolu par une m\'ethode
d'incr\'ement et r\'esidu en utilisant
une m\'ethode de Jacobi pour inverser le syst\`eme si le terme convectif
est implicite et en utilisant une m\'ethode de gradient conjugu\'e
si le terme convectif est explicite (qui est le cas par d�faut).

Attention, les valeurs du flux de masse $\rho\,\vect{w}\cdot\vect{S}$ et
de la viscosit\'e $\Delta\,t\,c^2\frac{S}{d}$ aux faces de
bord, qui sont calcul\'ees dans \fort{cfmsfl} et \fort{cfmsvs} respectivement,
sont modifi\'ees imm\'ediatement apr\`es l'appel \`a ces sous-programmes.
En effet, il est indispensable que la contribution de bord de
$\left(\rho\,\vect{w}-\Delta\,t\,(c^2)\,\gradv\,\rho\right)\cdot\vect{S}$
repr\'esente exactement $\vect{Q}_{ac}\cdot\vect{S}$.
Pour cela,
\begin{itemize}
\item imm\'ediatement apr\`es l'appel \`a
\fort{cfmsfl}, on remplace la contribution de bord de
$\rho\,\vect{w}\cdot\vect{S}$
par le flux de masse exact, $\vect{Q}_{ac}\cdot\vect{S}$,
d\'etermin\'e \`a partir des conditions aux limites,
\item puis, imm\'ediatement apr\`es l'appel \`a
\fort{cfmsvs}, on annule la viscosit\'e au bord $\Delta\,t\,(c^2)$ pour
\'eliminer la contribution de $-\Delta\,t\,(c^2)\,(\gradv\,\rho)\cdot\vect{S}$
(l'annulation de la viscosit\'e n'est pas probl\'ematique pour la matrice,
puisqu'elle porte sur des incr\'ements).
\end{itemize}

\bigskip

Une fois qu'on a obtenu $\rho^{n+1}$,
on peut actualiser le flux de masse acoustique
aux faces $(\vect{Q}_{ac}^{n+1})_{ij} \cdot \vect{S}_{ij}$,
qui servira pour la convection des autres variables~:
\begin{equation}\label{Cfbl_Cfmsvl_eq_flux_masse_acoustique_cfmsvl}
\displaystyle(\vect{Q}_{ac}^{n+1})_{ij}\cdot\vect{S}_{ij}=
-\left(\Delta t^n (c^2)^n \gradv(\rho^{n+1})\right)_{ij}\cdot\vect{S}_{ij}
+\left(\rho^{n+\frac{1}{2}} \vect{w}^n\right)_{ij}\cdot\vect{S}_{ij}\\
\end{equation}
Ce calcul de flux est r\'ealis\'e par \fort{cfbsc3}.
Si l'on a choisi l'algorithme standard, \'equation~(\ref{Cfbl_Cfmsvl_eq_densite_cfmsvl}),
on compl\`ete le flux dans \fort{cfmsvl} imm\'ediatement apr\`es l'appel
\`a \fort{cfbsc3}.
En effet, dans ce cas,
la convection est explicite ($\rho^{n+\frac{1}{2}}=\rho^{n}$,
obtenu en imposant \var{ICONV(ISCA(IRHO(IPHAS)))=0})
et le sous-programme \fort{cfbsc3},
qui calcule le flux de masse aux faces,
ne prend pas en compte la contribution du terme
$\rho^{n+\frac{1}{2}}\,\vect{w}^n\cdot\vect{S}$. On ajoute donc cette
contribution dans \fort{cfmsvl}, apr\`es l'appel \`a \fort{cfbsc3}.
Au bord, en particulier, c'est bien le flux de masse calcul\'e \`a partir
des conditions aux limites que l'on obtient.

On actualise la pression \`a la fin de l'\'etape, en utilisant la loi d'\'etat~:
\begin{equation}
\displaystyle P_i^{pred}=P(\rho_i^{n+1},\varepsilon_i^{n})
\end{equation}


%%%%%%%%%%%%%%%%%%%%%%%%%%%%%%%%%%
%%%%%%%%%%%%%%%%%%%%%%%%%%%%%%%%%%
\section{Points \`a traiter}
%%%%%%%%%%%%%%%%%%%%%%%%%%%%%%%%%%
%%%%%%%%%%%%%%%%%%%%%%%%%%%%%%%%%%
Le calcul du flux de masse au  bord n'est pas enti\`erement satisfaisant
si la convection est trait\'ee de mani\`ere implicite
(algorithme non standard, non test\'e,
associ\'e \`a l'\'equation~(\ref{Cfbl_Cfmsvl_eq_densite_bis_cfmsvl}),
correspondant au choix $\rho^{n+\frac{1}{2}}=\rho^{n+1}$ et
obtenu en imposant \var{ICONV(ISCA(IRHO(IPHAS)))=1}).
En effet, apr\`es \fort{cfmsfl}, il faut d\'eterminer la vitesse de
convection $\vect{w}^n$ pour qu'apparaisse
$\rho^{n+1} \vect{w}^n\cdot\vect{n}$
au cours de la r\'esolution par \fort{codits}. De ce fait, on doit d\'eduire
une valeur de $\vect{w}^n$ \`a partir de la valeur
du flux de masse. Au bord, en particulier, il faut
donc diviser le flux de masse
issu des conditions aux limites par la valeur de bord de $\rho^{n+1}$.
Or, lorsque des conditions de Neumann sont appliqu\'ees \`a la
masse volumique,
la valeur de $\rho^{n+1}$ au bord n'est pas connue avant la r\'esolution du
syst\`eme. On utilise donc, au lieu de la valeur de bord inconnue de
$\rho^{n+1}$ la valeur de bord prise au pas de temps
pr\'ec\'edent $\rho^{n}$. Cette approximation est susceptible
d'affecter la valeur du flux de masse au bord.

%                      Code_Saturne version 1.3
%                      ------------------------
%
%     This file is part of the Code_Saturne Kernel, element of the
%     Code_Saturne CFD tool.
%
%     Copyright (C) 1998-2007 EDF S.A., France
%
%     contact: saturne-support@edf.fr
%
%     The Code_Saturne Kernel is free software; you can redistribute it
%     and/or modify it under the terms of the GNU General Public License
%     as published by the Free Software Foundation; either version 2 of
%     the License, or (at your option) any later version.
%
%     The Code_Saturne Kernel is distributed in the hope that it will be
%     useful, but WITHOUT ANY WARRANTY; without even the implied warranty
%     of MERCHANTABILITY or FITNESS FOR A PARTICULAR PURPOSE.  See the
%     GNU General Public License for more details.
%
%     You should have received a copy of the GNU General Public License
%     along with the Code_Saturne Kernel; if not, write to the
%     Free Software Foundation, Inc.,
%     51 Franklin St, Fifth Floor,
%     Boston, MA  02110-1301  USA
%
%-----------------------------------------------------------------------
%


\programme{navsto}

\vspace{1cm}
On s'int\'eresse \`a la r\'esolution du syst\`eme d'\'equations de Navier-Stokes
tridimensionnelles monophasiques, \`a une pression, instationnaires, en
incompressible ou faiblement dilatable, bas\'ees sur une discr\'etisation
temporelle de type Euler implicite d'ordre 1 ou Crank-Nicolson d'ordre 2 et sur
une discr\'etisation spatiale  par volumes finis colocalis\'ee.


%%%%%%%%%%%%%%%%%%%%%%%%%%%%%%%%%%
%%%%%%%%%%%%%%%%%%%%%%%%%%%%%%%%%%
\section{Fonction}
%%%%%%%%%%%%%%%%%%%%%%%%%%%%%%%%%%
%%%%%%%%%%%%%%%%%%%%%%%%%%%%%%%%%%

  Dans ce sous-programme sont calcul\'ees, \`a un pas de temps donn\'e, les
variables vitesse et pression de ce probl\`eme en proc\'edant en
deux  \'etapes issues d'une d\'ecomposition des op\'erateurs (m\'ethode \`a
pas fractionnaires).\\
Les variables sont donc suppos\'ees connues \`a
l'instant ${t^n}$ et on cherche \`a les d\'eterminer \`a l'instant\footnote{La pression est suppos�e connue � l'instant $t^{n-1+\theta}$ et recherch�e en $t^{n+\theta}$, avec $\theta=1$ ou $1/2$ suivant le sch�ma en temps consid�r�.} ${t^{n+1}}$. Soit ${\Delta {t^n} ={t^{n+1}- {t^n}}}$ le pas de temps associ\'e. Dans un premier temps, on r\'ealise l'\'etape de
pr\'ediction de la vitesse en r\'esolvant l'\'equation de quantit\'e de
mouvement avec une pression explicite. Suit l'\'etape de correction de la
pression (ou projection de la vitesse) qui permet d'obtenir un champ de vitesse \`a divergence nulle.\\\\
Les \'equations en continu sont donc :
\begin{equation}
\left\{\begin{array}{l}
\displaystyle\frac{\partial}{\partial t}(\rho \vect{u}) + \dive(\rho\, \vect{u} \otimes \vect{u})
=\dive(\tens{\sigma}) + \vect{TS} - \tens{K}\,\vect{u}\\
\dive(\rho \vect{u}) = \Gamma
\end{array}\right.
\end{equation}

%(plus tard $\frac{\partial \rho}{\partial t} + \dive(\rho \vect{u}) = \Gamma$)



avec $\rho$ la masse volumique, $\vect{u}$ le champ de vitesse,
$[\,\vect{TS}-\tens{K}\,\vect{u}\,]$ les autres termes sources ($\tens{K}$~est un
tenseur diagonal positif par d\'efinition), $\tens{\sigma}$ le tenseur
de contraintes, $\tens{\tau}$ le tenseur des contraintes visqueuses, $\mu$ la
viscosit\'e dynamique (mol\'eculaire et \'eventuellement turbulente), $\kappa$
la viscosit� de
volume (usuellement nulle et n�glig�e dans le code et dans la suite du document,
sauf en compressible),
$\tens{D}$ le tenseur taux de d\'eformation\footnote{\`A ne pas confondre, malgr\'e la m\^eme notation $D$,
avec les flux diffusifs $\vect{D}_{\,ij}$ et $\vect{D}_{\,{b}_{ik}}$ d\'ecrits par la suite dans ce
sous-programme.}, $\Gamma$ le terme source de masse.
\begin{equation}
\left\{\begin{array}{l}
\tens{\sigma} = \tens{\tau} - P\tens{Id}  \\
\tens{\tau} = 2\,\mu\ \tens{D} +\ (\kappa\ - \frac{2}{3}\mu)\  tr({\tens{D}})\
\tens{Id}  \\
\tens{D} = \frac{1}{2}(\ggrad\vect{u}+\,^{t}\ggrad\vect{u})
\end{array}\right.
\end{equation}
 \\

On rappelle la d\'efinition des notations employ\'ees\footnote{en
utilisant la convention de sommation d'Einstein.}~:
\begin{equation}\notag
\left\{\begin{array}{lll}
\left[\ggrad{\vect{a}}\right]_{ij} &=& \partial_j a_i\\
\left[\dive(\tens{\sigma})\right]_i &=& \partial_j \sigma_{ij}\\
\left[\vect{a}\otimes\vect{b}\right]_{ij} &= &
a_i\,b_j\\
\end{array}\right.
\end{equation}
et donc :
\begin{equation}\notag
\begin{array}{lll}
\left[\dive(\vect{a}\otimes\vect{b})\right]_i &= &
\partial_j (a_i\,b_j)
\end{array}
\end{equation}

\minititre{Remarque}
Dans le cas de la prise en compte d'une masse volumique variable, l'�quation de continuit� s'�crit :
$$\frac{\partial \rho}{\partial t} + \dive{\,(\rho\,\vect{u})} = \Gamma  $$
Cette �quation n'est pas prise en compte dans l'�tape de projection (on continue � r�soudre
seulement
$\displaystyle \dive(\,{\rho\,\vect{u}}) = \Gamma$) alors que le terme
$\displaystyle \frac{\partial \rho}{\partial t}$ appara\^{\i}t lors de l'�tape de pr\'ediction de la vitesse
dans le sous-programme \fort{preduv}. Si ce terme joue un r�le sensible, l'algorithme compressible
de \CS\ (qui r�sout l'�quation compl�te) est alors sans doute plus adapt�.

%                      Code_Saturne version 1.3
%                      ------------------------
%
%     This file is part of the Code_Saturne Kernel, element of the
%     Code_Saturne CFD tool.
% 
%     Copyright (C) 1998-2007 EDF S.A., France
%
%     contact: saturne-support@edf.fr
% 
%     The Code_Saturne Kernel is free software; you can redistribute it
%     and/or modify it under the terms of the GNU General Public License
%     as published by the Free Software Foundation; either version 2 of
%     the License, or (at your option) any later version.
% 
%     The Code_Saturne Kernel is distributed in the hope that it will be
%     useful, but WITHOUT ANY WARRANTY; without even the implied warranty
%     of MERCHANTABILITY or FITNESS FOR A PARTICULAR PURPOSE.  See the
%     GNU General Public License for more details.
% 
%     You should have received a copy of the GNU General Public License
%     along with the Code_Saturne Kernel; if not, write to the
%     Free Software Foundation, Inc.,
%     51 Franklin St, Fifth Floor,
%     Boston, MA  02110-1301  USA
%
%-----------------------------------------------------------------------
%
%%%%%%%%%%%%%%%%%%%%%%%%%%%%%%%%%
%%%%%%%%%%%%%%%%%%%%%%%%%%%%%%%%%%
\section{Discr\'etisation}
%%%%%%%%%%%%%%%%%%%%%%%%%%%%%%%%%%
%%%%%%%%%%%%%%%%%%%%%%%%%%%%%%%%%%

Pour utiliser la m�thode, on se place tout d'abord dans un rep�re local d�fini
de mani�re � ce que le plan $(0yz)$, o� sont inject�s les vortex, soit confondu
avec le plan d'entr�e du calcul (voir figure \ref{Base_Vortex_entree}). 

\begin{figure}[h]
\centerline{\includegraphics[height=6cm]{../Base/Vortex/Images/entree.pdf}}
\caption{\label{Base_Vortex_entree} D�finiton des diff�rentes grandeurs dans le rep�re local
correspondant � l'entr�e d'une conduite de section carr�e.} 
\end{figure}

$u$, $v$ et $w$  sont les composantes de la vitesse fluctuante (principale et
transverse) dans ce plan, et
$\displaystyle \omega(y,z) = \frac{\partial w}{\partial y}-\frac{\partial v}{\partial z}$
la vorticit� dans la direction
normale au plan d'entr�e. $\overline{U}(y,z)$ repr�sente ici la vitesse
principale moyenne impos�e par l'utilisateur dans le plan d'entr�e. 

Chaque vortex $p$ va �tre caract�ris� par sa fonction de forme $\xi$ (identique
pour tous les vortex), sa
circulation $\Gamma_p$, son rayon $\sigma_p$ et les coordonn�es $(y_p,z_p)$ du
point $P$ o� est situ� le vortex dans le plan $(0yz)$. 

Pour cela, on suppose que la vorticit� g�n�r�e par un vortex $p$ au point $M$ de
coordonn�e $(y,z)$ s'�crit : 
\begin{equation}\notag
\omega_p(y,z)= \Gamma_p \, \xi_{\sigma_p}(r)
\end{equation}
o� $r = \sqrt{(y-y_p)^2+(z-z_p)^2}$ est la distance s�parant le point $M$ du point $P$.

Dans la m�thode implant�e, la fonction de forme est de type gaussienne modifi�e :
\begin{equation}\notag
\displaystyle
\xi_\sigma (r) = \frac{1}{2\pi \sigma^2} 
\left(2 e^{-\frac{r^2}{2\sigma^2}}-1\right) e^{-\frac{r^2}{2\sigma^2}}
\end{equation}

Le champ de vitesse induit par cette distribution de vorticit� s'obtient par
inversion des deux �quations de poisson suivantes qui sont d�duites de la
condition d'incompressibilit� dans la plan\footnote{\textit{i.e}
$\displaystyle \frac{\partial v}{\partial y}+\frac{\partial w}{\partial w} = 0$} :
\begin{equation}\notag
\begin{array}{lcr}
\displaystyle
\frac{\partial \omega}{\partial y} = \Delta w
&
\text{    et    }
&
\displaystyle
\frac{\partial \omega}{\partial y} = -\Delta v
\\
\end{array}
\end{equation}

Dans le cas g�n�ral, ce syst�me peut �tre int�gr� � l'aide de la formule de Biot et Savart.

Dans le cas d'une distribution de vorticit� de type gaussienne modifi�e, les
composantes de vitesse v�rifient : 
\begin{equation}\notag
\left\{
\begin{array}{c}
\displaystyle
v_p(y,x) = - \frac{1}{2\pi} \frac{(z-z_p)}{r^2}\left(1 -
e^{-\frac{r^2}{2\sigma^2}}\right)\,e^{-\frac{r^2}{2\sigma^2}} 
\\
\displaystyle
w_p(y,z) = \frac{1}{2\pi} \frac{(y-y_p)}{r^2}\left(1 -e^{-\frac{r^2}{2\sigma^2}}
\right)\,e^{-\frac{r^2}{2\sigma^2}} 
\end{array}
\right.
\end{equation}

Ces relations s'�tendent de fa�on imm�diate au cas de $N$ vortex distincts.
Le champ de vitesse induit par la distribution de vorticit� 
\begin{equation}
\omega(y,z) = \sum_{p=1}^N \Gamma_p \, \xi_{\sigma_p}(r)
\end{equation}
vaut au point $M$ :
\begin{equation}\notag
\begin{array}{lcr}
v(x,y) = \sum_{p=1}^N \Gamma_p\, v_p(y,z) 
&
\text{    et    }
&
w(y,z) = \sum_{p=1}^N \Gamma_p\, w_p(y,z)
\\
\label{Base_Vortex_compvit}
\end{array}
\end{equation}
%================================
\subsection{Param�tres physiques}
%================================

%-------------------------------
\subsubsection{Marche en temps}
%-------------------------------
La position initiale de chaque vortex est tir�e de mani�re al�atoire. On calcul
les d�placements successifs de chacun des vortex dans le plan d'entr�e par
int�gration explicite du champ de vitesse lagrangien : 
\begin{equation}\notag
\begin{array}{lcr}
\displaystyle
\frac{dy_p}{dt} = V(y,z)
&
\text{    et    }
&
\displaystyle
\frac{dz_p}{dt} = W(y,z)
\\
\end{array}
\end{equation}
Les vortex sont alors assimil�s � des particules ponctuelles qui sont convect�es
par le champ $(V,W)$. Ce champ peut �tre impos� par des tirages al�atoires ou
bien d�duit de la vitesse induite par les vortex dans le plan d'entr�e. Dans ce
cas $V(x,y) = \overline{V}(y,z) + v (y,z)$ et $W(y,z)= \overline{W}(y,z) +
w(y,z)$ o� $\overline{V}$ et $\overline{W}$ sont les composantes de la vitesse
transverse moyenne qu'impose l'utilisateur � l'aide des fichiers de donn�es. 

%---------------------------------------------------
\subsubsection{Intensit� et dur�e de vie des vortex}
%---------------------------------------------------
Il serait possible, � partir de l'�quation de transport de la vorticit�,
d'obtenir un mod�le d'�volution pour l'intensit� du vecteur tourbillon
$\omega_p$ associ� � chacun des vortex. En pratique, on pr�f�re utiliser un
mod�le simplifi� dans lequel la circulation des tourbillons ne d�pend que de la
postion de ces derniers � l'instant consid�r�. La circulation initiale de chaque
vortex est alors obtenue � partir de la relation : 
\begin{equation}\notag
|\Gamma_p| = 4 \sqrt{\frac{\pi\,S\,k}{3N\,[2ln(3)-3ln(2)]}}
\end{equation}
o� $S$ est la surface du plan d'entr�e, $N$ le nombre de vortex, et $k$
l'�nergie cin�tique turbulente au point o� se trouve le vortex � l'instant
consid�r�. Le signe de $\Gamma_p$ correspond au sens de rotation du vortex et
est tir� al�atoirement. 

Ce param�tre est celui qui contr�le l'intensit� des fluctuations. Sa d�pendance
en $k$ exprime que, plus l'�coulement est turbulent, plus les vortex sont
intenses. La valeur de $k$ est sp�cifi�e par
l'utilisateur. Elle peut �tre constante ou impos�e � partir de profils d'�nergie
cin�tique turbulente en entr�e. 

Pour �viter que des structures trop allong�es ne se d�veloppent au niveau de
l'entr�e, l'utilisateur doit �galement sp�cifier un temps limites $\tau_p$ au
bout duquel le vortex $p$ va �tre d�truit. Ce temps $\tau_p$ peut �tre pris
constant ou estim� au moyen de la relation : 
\begin{equation}\notag
\tau_p = \frac{5 C_{\mu}k^{\frac{3}{2}}}{\varepsilon\,\overline{U}}
\end{equation}

$\overline{U}$ et $\varepsilon$ repr�sentent respectivement la vitesse moyenne
principale et la dissipation turbulente au point o� est initialement g�n�r� le
vortex ($C_{\mu}=0,09$). 
\\
Lorsque le temps �coul� depuis la cr�ation du vortex $p$ est sup�rieur �
$\tau_p$, le vortex est d�truit et un nouveau vortex g�n�r� (sa position et le
signe de $\Gamma_p$ sont tir�s de fa�on al�atoire). 

%-------------------------------- 
\subsubsection{Taille des vortex}
%--------------------------------
La taille des vortex peut �tre prise constante, ou calcul�e � partir des
relations :
\begin{equation}\notag
\begin{array}{ccc}
\displaystyle
\sigma = \frac{C_{\mu}^{\frac{3}{4}}k^{\frac{3}{2}}}{\varepsilon} 
& \text{    ou    } &
\sigma = max[L_t,L_k]
\\
\end{array}
\end{equation}
avec:
\begin{equation}\notag
\begin{array}{ccc}
\displaystyle
L_t = \sqrt{\left( \frac{5 \nu k}{\varepsilon} \right)} 
& \text{    et    } & 
\displaystyle
L_k = 200\, \left(\frac{\nu^3}{\varepsilon}\right)^{\frac{1}{4}}
\end{array}
\end{equation}
o� $\nu$, $k$ et $\varepsilon$ sont la viscosit� dynamique, l'�nergie cin�tique
turbulente et la dissipation turbulente au point o� se trouve le vortex. 

Dans tous les cas, la taille du vortex doit �tre sup�rieure � la taille des
mailles en entr�e afin que le vortex soit effectivement simul�. 

%==================================
\subsection{Conditions aux limites}
%==================================
Le champ de vitesse g�n�r� � l'aide de la relation \ref{Base_Vortex_compvit} ne tient pas
compte {\em a priori} des conditions aux limites appliqu�es sur les bords du plan
d'entr�e. Pour obtenir des valeurs de la vitesse qui soient coh�rentes sur les
fronti�res du domaine d'entr�e, des ``vortex images'', pseudo-vortex situ�s en
dehors du domaine d'entr�e, sont g�n�r�s � des positions particuli�res et leur
champ de vitesse associ� est superpos� au champ pr�c�demment calcul�.\\
Seuls les cas d'une conduite rectangulaire et d'une conduite circulaire
permettent la g�n�ration de ces pseudo-vortex.
On distingue pour cela trois types de conditions aux limites. 

\begin{figure}[h]
\centerline{\includegraphics[height=6cm]{../Base/Vortex/Images/condlimite.pdf}}
\caption{\label{Base_Vortex_condli} Principe de g�n�ration des ``vortex images'' suivant le
type de conditions aux limites dans une conduite carr�e.} 
\end{figure}

%----------------------------------
\subsubsection{Condition de paroi}
%----------------------------------
On cr�e, pour chaque vortex $P$ contenu dans le plan d'entr�e, un vortex image
$P'$ identique � $P$ (\textit{i.e} de m�me caract�ristiques) et sym�trique de $P$
par rapport au
point $J$ ($J$ �tant la projection orthogonalement � la paroi du point $M$
correspondant au centre de la face o� l'on cherche � calculer la vitesse). La
figure \ref{Base_Vortex_condli} illustre la technique dans le cas d'une conduite
carr�e. Dans ce cas les coordonn�es du vortex situ� en $P'$ v�rifient
$(y_{p'}+y_{p})/2 = y_{J}$ et $(z_{p'}+ z_{p})/2 = z_{J}$. Le champ de vitesse
per�u depuis le point $M$ au niveau du point $J$ est nul, ce qui est bien
l'effet recherch�. 

%------------------------------------
\subsubsection{Condition de sym�trie}
%-------------------------------------
La technique est identique � celle utilis�e pour les conditions de paroi, mais
seule la composante pour la vitesse normale au bord est modifi�e dans ce cas. 

%---------------------------------------
\subsubsection{Condition de p�riodicit�}
%---------------------------------------
On cr�e pour chaque vortex, un vortex images $P'$ identique � $P$ mais translat�
d'une quantit� $L$ correspondant � la longueur qui s�pare les deux plans de la
section d'entr�e o� sont appliqu�es les conditions de p�riodicit�. Dans le cas
o� il y a deux directions de p�riodicit�, on cr�e deux vortex image.

%=============================================
\subsection{Composante de vitesse principale}
%=============================================
La m�thode des vortex ne g�n�rant pas de fluctuation $u$ de la vitesse dans la
direction principale, la derni�re composante est calcul�e � partir d'une
�quation de Langevin. Les coefficients de cette �quation sont d�termin�s par
identification des expressions obtenues pour les contraintes de Reynolds en
$R_{ij}-\varepsilon$. Dans le cas d'un �coulement en canal plan, cette technique
conduit � l'�quation : 
\begin{equation}\notag
\displaystyle
\frac{du}{dt} = - \frac{C_1}{2T} u + \left(\frac{2}{3}C_2-1\right)\frac{\partial
U}{\partial y} v + \sqrt{C_0\varepsilon} dW_i 
\end{equation}

avec $\displaystyle T=\frac{k}{\varepsilon}$, $C_1 = 1,8$, $C_2 = 0,6$,
$C_0=\frac{14}{15}$, et $dW_i$ une variable al�toire Gaussienne de variance
$\sqrt{dt}$. 

En pratique, l'�quation de Langevin n'am�liore pas vraiment les r�sultats. Elle
n'est utilis�e que dans le cas de conduites circulaires. 

%                      Code_Saturne version 1.3
%                      ------------------------
%
%     This file is part of the Code_Saturne Kernel, element of the
%     Code_Saturne CFD tool.
%
%     Copyright (C) 1998-2007 EDF S.A., France
%
%     contact: saturne-support@edf.fr
%
%     The Code_Saturne Kernel is free software; you can redistribute it
%     and/or modify it under the terms of the GNU General Public License
%     as published by the Free Software Foundation; either version 2 of
%     the License, or (at your option) any later version.
%
%     The Code_Saturne Kernel is distributed in the hope that it will be
%     useful, but WITHOUT ANY WARRANTY; without even the implied warranty
%     of MERCHANTABILITY or FITNESS FOR A PARTICULAR PURPOSE.  See the
%     GNU General Public License for more details.
%
%     You should have received a copy of the GNU General Public License
%     along with the Code_Saturne Kernel; if not, write to the
%     Free Software Foundation, Inc.,
%     51 Franklin St, Fifth Floor,
%     Boston, MA  02110-1301  USA
%
%-----------------------------------------------------------------------
%

%%%%%%%%%%%%%%%%%%%%%%%%%%%%%%%%%%
%%%%%%%%%%%%%%%%%%%%%%%%%%%%%%%%%%
\section{Mise en \oe uvre}
%%%%%%%%%%%%%%%%%%%%%%%%%%%%%%%%%%
%%%%%%%%%%%%%%%%%%%%%%%%%%%%%%%%%%
Le syst\`eme (\ref{Cfbl_Cfmsvl_eq_densite_finale_cfmsvl}) est r\'esolu par une m\'ethode
d'incr\'ement et r\'esidu en utilisant
une m\'ethode de Jacobi pour inverser le syst\`eme si le terme convectif
est implicite et en utilisant une m\'ethode de gradient conjugu\'e
si le terme convectif est explicite (qui est le cas par d�faut).

Attention, les valeurs du flux de masse $\rho\,\vect{w}\cdot\vect{S}$ et
de la viscosit\'e $\Delta\,t\,c^2\frac{S}{d}$ aux faces de
bord, qui sont calcul\'ees dans \fort{cfmsfl} et \fort{cfmsvs} respectivement,
sont modifi\'ees imm\'ediatement apr\`es l'appel \`a ces sous-programmes.
En effet, il est indispensable que la contribution de bord de
$\left(\rho\,\vect{w}-\Delta\,t\,(c^2)\,\gradv\,\rho\right)\cdot\vect{S}$
repr\'esente exactement $\vect{Q}_{ac}\cdot\vect{S}$.
Pour cela,
\begin{itemize}
\item imm\'ediatement apr\`es l'appel \`a
\fort{cfmsfl}, on remplace la contribution de bord de
$\rho\,\vect{w}\cdot\vect{S}$
par le flux de masse exact, $\vect{Q}_{ac}\cdot\vect{S}$,
d\'etermin\'e \`a partir des conditions aux limites,
\item puis, imm\'ediatement apr\`es l'appel \`a
\fort{cfmsvs}, on annule la viscosit\'e au bord $\Delta\,t\,(c^2)$ pour
\'eliminer la contribution de $-\Delta\,t\,(c^2)\,(\gradv\,\rho)\cdot\vect{S}$
(l'annulation de la viscosit\'e n'est pas probl\'ematique pour la matrice,
puisqu'elle porte sur des incr\'ements).
\end{itemize}

\bigskip

Une fois qu'on a obtenu $\rho^{n+1}$,
on peut actualiser le flux de masse acoustique
aux faces $(\vect{Q}_{ac}^{n+1})_{ij} \cdot \vect{S}_{ij}$,
qui servira pour la convection des autres variables~:
\begin{equation}\label{Cfbl_Cfmsvl_eq_flux_masse_acoustique_cfmsvl}
\displaystyle(\vect{Q}_{ac}^{n+1})_{ij}\cdot\vect{S}_{ij}=
-\left(\Delta t^n (c^2)^n \gradv(\rho^{n+1})\right)_{ij}\cdot\vect{S}_{ij}
+\left(\rho^{n+\frac{1}{2}} \vect{w}^n\right)_{ij}\cdot\vect{S}_{ij}\\
\end{equation}
Ce calcul de flux est r\'ealis\'e par \fort{cfbsc3}.
Si l'on a choisi l'algorithme standard, \'equation~(\ref{Cfbl_Cfmsvl_eq_densite_cfmsvl}),
on compl\`ete le flux dans \fort{cfmsvl} imm\'ediatement apr\`es l'appel
\`a \fort{cfbsc3}.
En effet, dans ce cas,
la convection est explicite ($\rho^{n+\frac{1}{2}}=\rho^{n}$,
obtenu en imposant \var{ICONV(ISCA(IRHO(IPHAS)))=0})
et le sous-programme \fort{cfbsc3},
qui calcule le flux de masse aux faces,
ne prend pas en compte la contribution du terme
$\rho^{n+\frac{1}{2}}\,\vect{w}^n\cdot\vect{S}$. On ajoute donc cette
contribution dans \fort{cfmsvl}, apr\`es l'appel \`a \fort{cfbsc3}.
Au bord, en particulier, c'est bien le flux de masse calcul\'e \`a partir
des conditions aux limites que l'on obtient.

On actualise la pression \`a la fin de l'\'etape, en utilisant la loi d'\'etat~:
\begin{equation}
\displaystyle P_i^{pred}=P(\rho_i^{n+1},\varepsilon_i^{n})
\end{equation}


%%%%%%%%%%%%%%%%%%%%%%%%%%%%%%%%%%
%%%%%%%%%%%%%%%%%%%%%%%%%%%%%%%%%%
\section{Points \`a traiter}
%%%%%%%%%%%%%%%%%%%%%%%%%%%%%%%%%%
%%%%%%%%%%%%%%%%%%%%%%%%%%%%%%%%%%
Le calcul du flux de masse au  bord n'est pas enti\`erement satisfaisant
si la convection est trait\'ee de mani\`ere implicite
(algorithme non standard, non test\'e,
associ\'e \`a l'\'equation~(\ref{Cfbl_Cfmsvl_eq_densite_bis_cfmsvl}),
correspondant au choix $\rho^{n+\frac{1}{2}}=\rho^{n+1}$ et
obtenu en imposant \var{ICONV(ISCA(IRHO(IPHAS)))=1}).
En effet, apr\`es \fort{cfmsfl}, il faut d\'eterminer la vitesse de
convection $\vect{w}^n$ pour qu'apparaisse
$\rho^{n+1} \vect{w}^n\cdot\vect{n}$
au cours de la r\'esolution par \fort{codits}. De ce fait, on doit d\'eduire
une valeur de $\vect{w}^n$ \`a partir de la valeur
du flux de masse. Au bord, en particulier, il faut
donc diviser le flux de masse
issu des conditions aux limites par la valeur de bord de $\rho^{n+1}$.
Or, lorsque des conditions de Neumann sont appliqu\'ees \`a la
masse volumique,
la valeur de $\rho^{n+1}$ au bord n'est pas connue avant la r\'esolution du
syst\`eme. On utilise donc, au lieu de la valeur de bord inconnue de
$\rho^{n+1}$ la valeur de bord prise au pas de temps
pr\'ec\'edent $\rho^{n}$. Cette approximation est susceptible
d'affecter la valeur du flux de masse au bord.

%                      Code_Saturne version 1.3
%                      ------------------------
%
%     This file is part of the Code_Saturne Kernel, element of the
%     Code_Saturne CFD tool.
%
%     Copyright (C) 1998-2007 EDF S.A., France
%
%     contact: saturne-support@edf.fr
%
%     The Code_Saturne Kernel is free software; you can redistribute it
%     and/or modify it under the terms of the GNU General Public License
%     as published by the Free Software Foundation; either version 2 of
%     the License, or (at your option) any later version.
%
%     The Code_Saturne Kernel is distributed in the hope that it will be
%     useful, but WITHOUT ANY WARRANTY; without even the implied warranty
%     of MERCHANTABILITY or FITNESS FOR A PARTICULAR PURPOSE.  See the
%     GNU General Public License for more details.
%
%     You should have received a copy of the GNU General Public License
%     along with the Code_Saturne Kernel; if not, write to the
%     Free Software Foundation, Inc.,
%     51 Franklin St, Fifth Floor,
%     Boston, MA  02110-1301  USA
%
%-----------------------------------------------------------------------
%


\programme{navsto}

\vspace{1cm}
On s'int\'eresse \`a la r\'esolution du syst\`eme d'\'equations de Navier-Stokes
tridimensionnelles monophasiques, \`a une pression, instationnaires, en
incompressible ou faiblement dilatable, bas\'ees sur une discr\'etisation
temporelle de type Euler implicite d'ordre 1 ou Crank-Nicolson d'ordre 2 et sur
une discr\'etisation spatiale  par volumes finis colocalis\'ee.


%%%%%%%%%%%%%%%%%%%%%%%%%%%%%%%%%%
%%%%%%%%%%%%%%%%%%%%%%%%%%%%%%%%%%
\section{Fonction}
%%%%%%%%%%%%%%%%%%%%%%%%%%%%%%%%%%
%%%%%%%%%%%%%%%%%%%%%%%%%%%%%%%%%%

  Dans ce sous-programme sont calcul\'ees, \`a un pas de temps donn\'e, les
variables vitesse et pression de ce probl\`eme en proc\'edant en
deux  \'etapes issues d'une d\'ecomposition des op\'erateurs (m\'ethode \`a
pas fractionnaires).\\
Les variables sont donc suppos\'ees connues \`a
l'instant ${t^n}$ et on cherche \`a les d\'eterminer \`a l'instant\footnote{La pression est suppos�e connue � l'instant $t^{n-1+\theta}$ et recherch�e en $t^{n+\theta}$, avec $\theta=1$ ou $1/2$ suivant le sch�ma en temps consid�r�.} ${t^{n+1}}$. Soit ${\Delta {t^n} ={t^{n+1}- {t^n}}}$ le pas de temps associ\'e. Dans un premier temps, on r\'ealise l'\'etape de
pr\'ediction de la vitesse en r\'esolvant l'\'equation de quantit\'e de
mouvement avec une pression explicite. Suit l'\'etape de correction de la
pression (ou projection de la vitesse) qui permet d'obtenir un champ de vitesse \`a divergence nulle.\\\\
Les \'equations en continu sont donc :
\begin{equation}
\left\{\begin{array}{l}
\displaystyle\frac{\partial}{\partial t}(\rho \vect{u}) + \dive(\rho\, \vect{u} \otimes \vect{u})
=\dive(\tens{\sigma}) + \vect{TS} - \tens{K}\,\vect{u}\\
\dive(\rho \vect{u}) = \Gamma
\end{array}\right.
\end{equation}

%(plus tard $\frac{\partial \rho}{\partial t} + \dive(\rho \vect{u}) = \Gamma$)



avec $\rho$ la masse volumique, $\vect{u}$ le champ de vitesse,
$[\,\vect{TS}-\tens{K}\,\vect{u}\,]$ les autres termes sources ($\tens{K}$~est un
tenseur diagonal positif par d\'efinition), $\tens{\sigma}$ le tenseur
de contraintes, $\tens{\tau}$ le tenseur des contraintes visqueuses, $\mu$ la
viscosit\'e dynamique (mol\'eculaire et \'eventuellement turbulente), $\kappa$
la viscosit� de
volume (usuellement nulle et n�glig�e dans le code et dans la suite du document,
sauf en compressible),
$\tens{D}$ le tenseur taux de d\'eformation\footnote{\`A ne pas confondre, malgr\'e la m\^eme notation $D$,
avec les flux diffusifs $\vect{D}_{\,ij}$ et $\vect{D}_{\,{b}_{ik}}$ d\'ecrits par la suite dans ce
sous-programme.}, $\Gamma$ le terme source de masse.
\begin{equation}
\left\{\begin{array}{l}
\tens{\sigma} = \tens{\tau} - P\tens{Id}  \\
\tens{\tau} = 2\,\mu\ \tens{D} +\ (\kappa\ - \frac{2}{3}\mu)\  tr({\tens{D}})\
\tens{Id}  \\
\tens{D} = \frac{1}{2}(\ggrad\vect{u}+\,^{t}\ggrad\vect{u})
\end{array}\right.
\end{equation}
 \\

On rappelle la d\'efinition des notations employ\'ees\footnote{en
utilisant la convention de sommation d'Einstein.}~:
\begin{equation}\notag
\left\{\begin{array}{lll}
\left[\ggrad{\vect{a}}\right]_{ij} &=& \partial_j a_i\\
\left[\dive(\tens{\sigma})\right]_i &=& \partial_j \sigma_{ij}\\
\left[\vect{a}\otimes\vect{b}\right]_{ij} &= &
a_i\,b_j\\
\end{array}\right.
\end{equation}
et donc :
\begin{equation}\notag
\begin{array}{lll}
\left[\dive(\vect{a}\otimes\vect{b})\right]_i &= &
\partial_j (a_i\,b_j)
\end{array}
\end{equation}

\minititre{Remarque}
Dans le cas de la prise en compte d'une masse volumique variable, l'�quation de continuit� s'�crit :
$$\frac{\partial \rho}{\partial t} + \dive{\,(\rho\,\vect{u})} = \Gamma  $$
Cette �quation n'est pas prise en compte dans l'�tape de projection (on continue � r�soudre
seulement
$\displaystyle \dive(\,{\rho\,\vect{u}}) = \Gamma$) alors que le terme
$\displaystyle \frac{\partial \rho}{\partial t}$ appara\^{\i}t lors de l'�tape de pr\'ediction de la vitesse
dans le sous-programme \fort{preduv}. Si ce terme joue un r�le sensible, l'algorithme compressible
de \CS\ (qui r�sout l'�quation compl�te) est alors sans doute plus adapt�.

%                      Code_Saturne version 1.3
%                      ------------------------
%
%     This file is part of the Code_Saturne Kernel, element of the
%     Code_Saturne CFD tool.
% 
%     Copyright (C) 1998-2007 EDF S.A., France
%
%     contact: saturne-support@edf.fr
% 
%     The Code_Saturne Kernel is free software; you can redistribute it
%     and/or modify it under the terms of the GNU General Public License
%     as published by the Free Software Foundation; either version 2 of
%     the License, or (at your option) any later version.
% 
%     The Code_Saturne Kernel is distributed in the hope that it will be
%     useful, but WITHOUT ANY WARRANTY; without even the implied warranty
%     of MERCHANTABILITY or FITNESS FOR A PARTICULAR PURPOSE.  See the
%     GNU General Public License for more details.
% 
%     You should have received a copy of the GNU General Public License
%     along with the Code_Saturne Kernel; if not, write to the
%     Free Software Foundation, Inc.,
%     51 Franklin St, Fifth Floor,
%     Boston, MA  02110-1301  USA
%
%-----------------------------------------------------------------------
%
%%%%%%%%%%%%%%%%%%%%%%%%%%%%%%%%%
%%%%%%%%%%%%%%%%%%%%%%%%%%%%%%%%%%
\section{Discr\'etisation}
%%%%%%%%%%%%%%%%%%%%%%%%%%%%%%%%%%
%%%%%%%%%%%%%%%%%%%%%%%%%%%%%%%%%%

Pour utiliser la m�thode, on se place tout d'abord dans un rep�re local d�fini
de mani�re � ce que le plan $(0yz)$, o� sont inject�s les vortex, soit confondu
avec le plan d'entr�e du calcul (voir figure \ref{Base_Vortex_entree}). 

\begin{figure}[h]
\centerline{\includegraphics[height=6cm]{../Base/Vortex/Images/entree.pdf}}
\caption{\label{Base_Vortex_entree} D�finiton des diff�rentes grandeurs dans le rep�re local
correspondant � l'entr�e d'une conduite de section carr�e.} 
\end{figure}

$u$, $v$ et $w$  sont les composantes de la vitesse fluctuante (principale et
transverse) dans ce plan, et
$\displaystyle \omega(y,z) = \frac{\partial w}{\partial y}-\frac{\partial v}{\partial z}$
la vorticit� dans la direction
normale au plan d'entr�e. $\overline{U}(y,z)$ repr�sente ici la vitesse
principale moyenne impos�e par l'utilisateur dans le plan d'entr�e. 

Chaque vortex $p$ va �tre caract�ris� par sa fonction de forme $\xi$ (identique
pour tous les vortex), sa
circulation $\Gamma_p$, son rayon $\sigma_p$ et les coordonn�es $(y_p,z_p)$ du
point $P$ o� est situ� le vortex dans le plan $(0yz)$. 

Pour cela, on suppose que la vorticit� g�n�r�e par un vortex $p$ au point $M$ de
coordonn�e $(y,z)$ s'�crit : 
\begin{equation}\notag
\omega_p(y,z)= \Gamma_p \, \xi_{\sigma_p}(r)
\end{equation}
o� $r = \sqrt{(y-y_p)^2+(z-z_p)^2}$ est la distance s�parant le point $M$ du point $P$.

Dans la m�thode implant�e, la fonction de forme est de type gaussienne modifi�e :
\begin{equation}\notag
\displaystyle
\xi_\sigma (r) = \frac{1}{2\pi \sigma^2} 
\left(2 e^{-\frac{r^2}{2\sigma^2}}-1\right) e^{-\frac{r^2}{2\sigma^2}}
\end{equation}

Le champ de vitesse induit par cette distribution de vorticit� s'obtient par
inversion des deux �quations de poisson suivantes qui sont d�duites de la
condition d'incompressibilit� dans la plan\footnote{\textit{i.e}
$\displaystyle \frac{\partial v}{\partial y}+\frac{\partial w}{\partial w} = 0$} :
\begin{equation}\notag
\begin{array}{lcr}
\displaystyle
\frac{\partial \omega}{\partial y} = \Delta w
&
\text{    et    }
&
\displaystyle
\frac{\partial \omega}{\partial y} = -\Delta v
\\
\end{array}
\end{equation}

Dans le cas g�n�ral, ce syst�me peut �tre int�gr� � l'aide de la formule de Biot et Savart.

Dans le cas d'une distribution de vorticit� de type gaussienne modifi�e, les
composantes de vitesse v�rifient : 
\begin{equation}\notag
\left\{
\begin{array}{c}
\displaystyle
v_p(y,x) = - \frac{1}{2\pi} \frac{(z-z_p)}{r^2}\left(1 -
e^{-\frac{r^2}{2\sigma^2}}\right)\,e^{-\frac{r^2}{2\sigma^2}} 
\\
\displaystyle
w_p(y,z) = \frac{1}{2\pi} \frac{(y-y_p)}{r^2}\left(1 -e^{-\frac{r^2}{2\sigma^2}}
\right)\,e^{-\frac{r^2}{2\sigma^2}} 
\end{array}
\right.
\end{equation}

Ces relations s'�tendent de fa�on imm�diate au cas de $N$ vortex distincts.
Le champ de vitesse induit par la distribution de vorticit� 
\begin{equation}
\omega(y,z) = \sum_{p=1}^N \Gamma_p \, \xi_{\sigma_p}(r)
\end{equation}
vaut au point $M$ :
\begin{equation}\notag
\begin{array}{lcr}
v(x,y) = \sum_{p=1}^N \Gamma_p\, v_p(y,z) 
&
\text{    et    }
&
w(y,z) = \sum_{p=1}^N \Gamma_p\, w_p(y,z)
\\
\label{Base_Vortex_compvit}
\end{array}
\end{equation}
%================================
\subsection{Param�tres physiques}
%================================

%-------------------------------
\subsubsection{Marche en temps}
%-------------------------------
La position initiale de chaque vortex est tir�e de mani�re al�atoire. On calcul
les d�placements successifs de chacun des vortex dans le plan d'entr�e par
int�gration explicite du champ de vitesse lagrangien : 
\begin{equation}\notag
\begin{array}{lcr}
\displaystyle
\frac{dy_p}{dt} = V(y,z)
&
\text{    et    }
&
\displaystyle
\frac{dz_p}{dt} = W(y,z)
\\
\end{array}
\end{equation}
Les vortex sont alors assimil�s � des particules ponctuelles qui sont convect�es
par le champ $(V,W)$. Ce champ peut �tre impos� par des tirages al�atoires ou
bien d�duit de la vitesse induite par les vortex dans le plan d'entr�e. Dans ce
cas $V(x,y) = \overline{V}(y,z) + v (y,z)$ et $W(y,z)= \overline{W}(y,z) +
w(y,z)$ o� $\overline{V}$ et $\overline{W}$ sont les composantes de la vitesse
transverse moyenne qu'impose l'utilisateur � l'aide des fichiers de donn�es. 

%---------------------------------------------------
\subsubsection{Intensit� et dur�e de vie des vortex}
%---------------------------------------------------
Il serait possible, � partir de l'�quation de transport de la vorticit�,
d'obtenir un mod�le d'�volution pour l'intensit� du vecteur tourbillon
$\omega_p$ associ� � chacun des vortex. En pratique, on pr�f�re utiliser un
mod�le simplifi� dans lequel la circulation des tourbillons ne d�pend que de la
postion de ces derniers � l'instant consid�r�. La circulation initiale de chaque
vortex est alors obtenue � partir de la relation : 
\begin{equation}\notag
|\Gamma_p| = 4 \sqrt{\frac{\pi\,S\,k}{3N\,[2ln(3)-3ln(2)]}}
\end{equation}
o� $S$ est la surface du plan d'entr�e, $N$ le nombre de vortex, et $k$
l'�nergie cin�tique turbulente au point o� se trouve le vortex � l'instant
consid�r�. Le signe de $\Gamma_p$ correspond au sens de rotation du vortex et
est tir� al�atoirement. 

Ce param�tre est celui qui contr�le l'intensit� des fluctuations. Sa d�pendance
en $k$ exprime que, plus l'�coulement est turbulent, plus les vortex sont
intenses. La valeur de $k$ est sp�cifi�e par
l'utilisateur. Elle peut �tre constante ou impos�e � partir de profils d'�nergie
cin�tique turbulente en entr�e. 

Pour �viter que des structures trop allong�es ne se d�veloppent au niveau de
l'entr�e, l'utilisateur doit �galement sp�cifier un temps limites $\tau_p$ au
bout duquel le vortex $p$ va �tre d�truit. Ce temps $\tau_p$ peut �tre pris
constant ou estim� au moyen de la relation : 
\begin{equation}\notag
\tau_p = \frac{5 C_{\mu}k^{\frac{3}{2}}}{\varepsilon\,\overline{U}}
\end{equation}

$\overline{U}$ et $\varepsilon$ repr�sentent respectivement la vitesse moyenne
principale et la dissipation turbulente au point o� est initialement g�n�r� le
vortex ($C_{\mu}=0,09$). 
\\
Lorsque le temps �coul� depuis la cr�ation du vortex $p$ est sup�rieur �
$\tau_p$, le vortex est d�truit et un nouveau vortex g�n�r� (sa position et le
signe de $\Gamma_p$ sont tir�s de fa�on al�atoire). 

%-------------------------------- 
\subsubsection{Taille des vortex}
%--------------------------------
La taille des vortex peut �tre prise constante, ou calcul�e � partir des
relations :
\begin{equation}\notag
\begin{array}{ccc}
\displaystyle
\sigma = \frac{C_{\mu}^{\frac{3}{4}}k^{\frac{3}{2}}}{\varepsilon} 
& \text{    ou    } &
\sigma = max[L_t,L_k]
\\
\end{array}
\end{equation}
avec:
\begin{equation}\notag
\begin{array}{ccc}
\displaystyle
L_t = \sqrt{\left( \frac{5 \nu k}{\varepsilon} \right)} 
& \text{    et    } & 
\displaystyle
L_k = 200\, \left(\frac{\nu^3}{\varepsilon}\right)^{\frac{1}{4}}
\end{array}
\end{equation}
o� $\nu$, $k$ et $\varepsilon$ sont la viscosit� dynamique, l'�nergie cin�tique
turbulente et la dissipation turbulente au point o� se trouve le vortex. 

Dans tous les cas, la taille du vortex doit �tre sup�rieure � la taille des
mailles en entr�e afin que le vortex soit effectivement simul�. 

%==================================
\subsection{Conditions aux limites}
%==================================
Le champ de vitesse g�n�r� � l'aide de la relation \ref{Base_Vortex_compvit} ne tient pas
compte {\em a priori} des conditions aux limites appliqu�es sur les bords du plan
d'entr�e. Pour obtenir des valeurs de la vitesse qui soient coh�rentes sur les
fronti�res du domaine d'entr�e, des ``vortex images'', pseudo-vortex situ�s en
dehors du domaine d'entr�e, sont g�n�r�s � des positions particuli�res et leur
champ de vitesse associ� est superpos� au champ pr�c�demment calcul�.\\
Seuls les cas d'une conduite rectangulaire et d'une conduite circulaire
permettent la g�n�ration de ces pseudo-vortex.
On distingue pour cela trois types de conditions aux limites. 

\begin{figure}[h]
\centerline{\includegraphics[height=6cm]{../Base/Vortex/Images/condlimite.pdf}}
\caption{\label{Base_Vortex_condli} Principe de g�n�ration des ``vortex images'' suivant le
type de conditions aux limites dans une conduite carr�e.} 
\end{figure}

%----------------------------------
\subsubsection{Condition de paroi}
%----------------------------------
On cr�e, pour chaque vortex $P$ contenu dans le plan d'entr�e, un vortex image
$P'$ identique � $P$ (\textit{i.e} de m�me caract�ristiques) et sym�trique de $P$
par rapport au
point $J$ ($J$ �tant la projection orthogonalement � la paroi du point $M$
correspondant au centre de la face o� l'on cherche � calculer la vitesse). La
figure \ref{Base_Vortex_condli} illustre la technique dans le cas d'une conduite
carr�e. Dans ce cas les coordonn�es du vortex situ� en $P'$ v�rifient
$(y_{p'}+y_{p})/2 = y_{J}$ et $(z_{p'}+ z_{p})/2 = z_{J}$. Le champ de vitesse
per�u depuis le point $M$ au niveau du point $J$ est nul, ce qui est bien
l'effet recherch�. 

%------------------------------------
\subsubsection{Condition de sym�trie}
%-------------------------------------
La technique est identique � celle utilis�e pour les conditions de paroi, mais
seule la composante pour la vitesse normale au bord est modifi�e dans ce cas. 

%---------------------------------------
\subsubsection{Condition de p�riodicit�}
%---------------------------------------
On cr�e pour chaque vortex, un vortex images $P'$ identique � $P$ mais translat�
d'une quantit� $L$ correspondant � la longueur qui s�pare les deux plans de la
section d'entr�e o� sont appliqu�es les conditions de p�riodicit�. Dans le cas
o� il y a deux directions de p�riodicit�, on cr�e deux vortex image.

%=============================================
\subsection{Composante de vitesse principale}
%=============================================
La m�thode des vortex ne g�n�rant pas de fluctuation $u$ de la vitesse dans la
direction principale, la derni�re composante est calcul�e � partir d'une
�quation de Langevin. Les coefficients de cette �quation sont d�termin�s par
identification des expressions obtenues pour les contraintes de Reynolds en
$R_{ij}-\varepsilon$. Dans le cas d'un �coulement en canal plan, cette technique
conduit � l'�quation : 
\begin{equation}\notag
\displaystyle
\frac{du}{dt} = - \frac{C_1}{2T} u + \left(\frac{2}{3}C_2-1\right)\frac{\partial
U}{\partial y} v + \sqrt{C_0\varepsilon} dW_i 
\end{equation}

avec $\displaystyle T=\frac{k}{\varepsilon}$, $C_1 = 1,8$, $C_2 = 0,6$,
$C_0=\frac{14}{15}$, et $dW_i$ une variable al�toire Gaussienne de variance
$\sqrt{dt}$. 

En pratique, l'�quation de Langevin n'am�liore pas vraiment les r�sultats. Elle
n'est utilis�e que dans le cas de conduites circulaires. 

%                      Code_Saturne version 1.3
%                      ------------------------
%
%     This file is part of the Code_Saturne Kernel, element of the
%     Code_Saturne CFD tool.
%
%     Copyright (C) 1998-2007 EDF S.A., France
%
%     contact: saturne-support@edf.fr
%
%     The Code_Saturne Kernel is free software; you can redistribute it
%     and/or modify it under the terms of the GNU General Public License
%     as published by the Free Software Foundation; either version 2 of
%     the License, or (at your option) any later version.
%
%     The Code_Saturne Kernel is distributed in the hope that it will be
%     useful, but WITHOUT ANY WARRANTY; without even the implied warranty
%     of MERCHANTABILITY or FITNESS FOR A PARTICULAR PURPOSE.  See the
%     GNU General Public License for more details.
%
%     You should have received a copy of the GNU General Public License
%     along with the Code_Saturne Kernel; if not, write to the
%     Free Software Foundation, Inc.,
%     51 Franklin St, Fifth Floor,
%     Boston, MA  02110-1301  USA
%
%-----------------------------------------------------------------------
%

%%%%%%%%%%%%%%%%%%%%%%%%%%%%%%%%%%
%%%%%%%%%%%%%%%%%%%%%%%%%%%%%%%%%%
\section{Mise en \oe uvre}
%%%%%%%%%%%%%%%%%%%%%%%%%%%%%%%%%%
%%%%%%%%%%%%%%%%%%%%%%%%%%%%%%%%%%
Le syst\`eme (\ref{Cfbl_Cfmsvl_eq_densite_finale_cfmsvl}) est r\'esolu par une m\'ethode
d'incr\'ement et r\'esidu en utilisant
une m\'ethode de Jacobi pour inverser le syst\`eme si le terme convectif
est implicite et en utilisant une m\'ethode de gradient conjugu\'e
si le terme convectif est explicite (qui est le cas par d�faut).

Attention, les valeurs du flux de masse $\rho\,\vect{w}\cdot\vect{S}$ et
de la viscosit\'e $\Delta\,t\,c^2\frac{S}{d}$ aux faces de
bord, qui sont calcul\'ees dans \fort{cfmsfl} et \fort{cfmsvs} respectivement,
sont modifi\'ees imm\'ediatement apr\`es l'appel \`a ces sous-programmes.
En effet, il est indispensable que la contribution de bord de
$\left(\rho\,\vect{w}-\Delta\,t\,(c^2)\,\gradv\,\rho\right)\cdot\vect{S}$
repr\'esente exactement $\vect{Q}_{ac}\cdot\vect{S}$.
Pour cela,
\begin{itemize}
\item imm\'ediatement apr\`es l'appel \`a
\fort{cfmsfl}, on remplace la contribution de bord de
$\rho\,\vect{w}\cdot\vect{S}$
par le flux de masse exact, $\vect{Q}_{ac}\cdot\vect{S}$,
d\'etermin\'e \`a partir des conditions aux limites,
\item puis, imm\'ediatement apr\`es l'appel \`a
\fort{cfmsvs}, on annule la viscosit\'e au bord $\Delta\,t\,(c^2)$ pour
\'eliminer la contribution de $-\Delta\,t\,(c^2)\,(\gradv\,\rho)\cdot\vect{S}$
(l'annulation de la viscosit\'e n'est pas probl\'ematique pour la matrice,
puisqu'elle porte sur des incr\'ements).
\end{itemize}

\bigskip

Une fois qu'on a obtenu $\rho^{n+1}$,
on peut actualiser le flux de masse acoustique
aux faces $(\vect{Q}_{ac}^{n+1})_{ij} \cdot \vect{S}_{ij}$,
qui servira pour la convection des autres variables~:
\begin{equation}\label{Cfbl_Cfmsvl_eq_flux_masse_acoustique_cfmsvl}
\displaystyle(\vect{Q}_{ac}^{n+1})_{ij}\cdot\vect{S}_{ij}=
-\left(\Delta t^n (c^2)^n \gradv(\rho^{n+1})\right)_{ij}\cdot\vect{S}_{ij}
+\left(\rho^{n+\frac{1}{2}} \vect{w}^n\right)_{ij}\cdot\vect{S}_{ij}\\
\end{equation}
Ce calcul de flux est r\'ealis\'e par \fort{cfbsc3}.
Si l'on a choisi l'algorithme standard, \'equation~(\ref{Cfbl_Cfmsvl_eq_densite_cfmsvl}),
on compl\`ete le flux dans \fort{cfmsvl} imm\'ediatement apr\`es l'appel
\`a \fort{cfbsc3}.
En effet, dans ce cas,
la convection est explicite ($\rho^{n+\frac{1}{2}}=\rho^{n}$,
obtenu en imposant \var{ICONV(ISCA(IRHO(IPHAS)))=0})
et le sous-programme \fort{cfbsc3},
qui calcule le flux de masse aux faces,
ne prend pas en compte la contribution du terme
$\rho^{n+\frac{1}{2}}\,\vect{w}^n\cdot\vect{S}$. On ajoute donc cette
contribution dans \fort{cfmsvl}, apr\`es l'appel \`a \fort{cfbsc3}.
Au bord, en particulier, c'est bien le flux de masse calcul\'e \`a partir
des conditions aux limites que l'on obtient.

On actualise la pression \`a la fin de l'\'etape, en utilisant la loi d'\'etat~:
\begin{equation}
\displaystyle P_i^{pred}=P(\rho_i^{n+1},\varepsilon_i^{n})
\end{equation}


%%%%%%%%%%%%%%%%%%%%%%%%%%%%%%%%%%
%%%%%%%%%%%%%%%%%%%%%%%%%%%%%%%%%%
\section{Points \`a traiter}
%%%%%%%%%%%%%%%%%%%%%%%%%%%%%%%%%%
%%%%%%%%%%%%%%%%%%%%%%%%%%%%%%%%%%
Le calcul du flux de masse au  bord n'est pas enti\`erement satisfaisant
si la convection est trait\'ee de mani\`ere implicite
(algorithme non standard, non test\'e,
associ\'e \`a l'\'equation~(\ref{Cfbl_Cfmsvl_eq_densite_bis_cfmsvl}),
correspondant au choix $\rho^{n+\frac{1}{2}}=\rho^{n+1}$ et
obtenu en imposant \var{ICONV(ISCA(IRHO(IPHAS)))=1}).
En effet, apr\`es \fort{cfmsfl}, il faut d\'eterminer la vitesse de
convection $\vect{w}^n$ pour qu'apparaisse
$\rho^{n+1} \vect{w}^n\cdot\vect{n}$
au cours de la r\'esolution par \fort{codits}. De ce fait, on doit d\'eduire
une valeur de $\vect{w}^n$ \`a partir de la valeur
du flux de masse. Au bord, en particulier, il faut
donc diviser le flux de masse
issu des conditions aux limites par la valeur de bord de $\rho^{n+1}$.
Or, lorsque des conditions de Neumann sont appliqu\'ees \`a la
masse volumique,
la valeur de $\rho^{n+1}$ au bord n'est pas connue avant la r\'esolution du
syst\`eme. On utilise donc, au lieu de la valeur de bord inconnue de
$\rho^{n+1}$ la valeur de bord prise au pas de temps
pr\'ec\'edent $\rho^{n}$. Cette approximation est susceptible
d'affecter la valeur du flux de masse au bord.

%                      Code_Saturne version 1.3
%                      ------------------------
%
%     This file is part of the Code_Saturne Kernel, element of the
%     Code_Saturne CFD tool.
%
%     Copyright (C) 1998-2007 EDF S.A., France
%
%     contact: saturne-support@edf.fr
%
%     The Code_Saturne Kernel is free software; you can redistribute it
%     and/or modify it under the terms of the GNU General Public License
%     as published by the Free Software Foundation; either version 2 of
%     the License, or (at your option) any later version.
%
%     The Code_Saturne Kernel is distributed in the hope that it will be
%     useful, but WITHOUT ANY WARRANTY; without even the implied warranty
%     of MERCHANTABILITY or FITNESS FOR A PARTICULAR PURPOSE.  See the
%     GNU General Public License for more details.
%
%     You should have received a copy of the GNU General Public License
%     along with the Code_Saturne Kernel; if not, write to the
%     Free Software Foundation, Inc.,
%     51 Franklin St, Fifth Floor,
%     Boston, MA  02110-1301  USA
%
%-----------------------------------------------------------------------
%


\programme{navsto}

\vspace{1cm}
On s'int\'eresse \`a la r\'esolution du syst\`eme d'\'equations de Navier-Stokes
tridimensionnelles monophasiques, \`a une pression, instationnaires, en
incompressible ou faiblement dilatable, bas\'ees sur une discr\'etisation
temporelle de type Euler implicite d'ordre 1 ou Crank-Nicolson d'ordre 2 et sur
une discr\'etisation spatiale  par volumes finis colocalis\'ee.


%%%%%%%%%%%%%%%%%%%%%%%%%%%%%%%%%%
%%%%%%%%%%%%%%%%%%%%%%%%%%%%%%%%%%
\section{Fonction}
%%%%%%%%%%%%%%%%%%%%%%%%%%%%%%%%%%
%%%%%%%%%%%%%%%%%%%%%%%%%%%%%%%%%%

  Dans ce sous-programme sont calcul\'ees, \`a un pas de temps donn\'e, les
variables vitesse et pression de ce probl\`eme en proc\'edant en
deux  \'etapes issues d'une d\'ecomposition des op\'erateurs (m\'ethode \`a
pas fractionnaires).\\
Les variables sont donc suppos\'ees connues \`a
l'instant ${t^n}$ et on cherche \`a les d\'eterminer \`a l'instant\footnote{La pression est suppos�e connue � l'instant $t^{n-1+\theta}$ et recherch�e en $t^{n+\theta}$, avec $\theta=1$ ou $1/2$ suivant le sch�ma en temps consid�r�.} ${t^{n+1}}$. Soit ${\Delta {t^n} ={t^{n+1}- {t^n}}}$ le pas de temps associ\'e. Dans un premier temps, on r\'ealise l'\'etape de
pr\'ediction de la vitesse en r\'esolvant l'\'equation de quantit\'e de
mouvement avec une pression explicite. Suit l'\'etape de correction de la
pression (ou projection de la vitesse) qui permet d'obtenir un champ de vitesse \`a divergence nulle.\\\\
Les \'equations en continu sont donc :
\begin{equation}
\left\{\begin{array}{l}
\displaystyle\frac{\partial}{\partial t}(\rho \vect{u}) + \dive(\rho\, \vect{u} \otimes \vect{u})
=\dive(\tens{\sigma}) + \vect{TS} - \tens{K}\,\vect{u}\\
\dive(\rho \vect{u}) = \Gamma
\end{array}\right.
\end{equation}

%(plus tard $\frac{\partial \rho}{\partial t} + \dive(\rho \vect{u}) = \Gamma$)



avec $\rho$ la masse volumique, $\vect{u}$ le champ de vitesse,
$[\,\vect{TS}-\tens{K}\,\vect{u}\,]$ les autres termes sources ($\tens{K}$~est un
tenseur diagonal positif par d\'efinition), $\tens{\sigma}$ le tenseur
de contraintes, $\tens{\tau}$ le tenseur des contraintes visqueuses, $\mu$ la
viscosit\'e dynamique (mol\'eculaire et \'eventuellement turbulente), $\kappa$
la viscosit� de
volume (usuellement nulle et n�glig�e dans le code et dans la suite du document,
sauf en compressible),
$\tens{D}$ le tenseur taux de d\'eformation\footnote{\`A ne pas confondre, malgr\'e la m\^eme notation $D$,
avec les flux diffusifs $\vect{D}_{\,ij}$ et $\vect{D}_{\,{b}_{ik}}$ d\'ecrits par la suite dans ce
sous-programme.}, $\Gamma$ le terme source de masse.
\begin{equation}
\left\{\begin{array}{l}
\tens{\sigma} = \tens{\tau} - P\tens{Id}  \\
\tens{\tau} = 2\,\mu\ \tens{D} +\ (\kappa\ - \frac{2}{3}\mu)\  tr({\tens{D}})\
\tens{Id}  \\
\tens{D} = \frac{1}{2}(\ggrad\vect{u}+\,^{t}\ggrad\vect{u})
\end{array}\right.
\end{equation}
 \\

On rappelle la d\'efinition des notations employ\'ees\footnote{en
utilisant la convention de sommation d'Einstein.}~:
\begin{equation}\notag
\left\{\begin{array}{lll}
\left[\ggrad{\vect{a}}\right]_{ij} &=& \partial_j a_i\\
\left[\dive(\tens{\sigma})\right]_i &=& \partial_j \sigma_{ij}\\
\left[\vect{a}\otimes\vect{b}\right]_{ij} &= &
a_i\,b_j\\
\end{array}\right.
\end{equation}
et donc :
\begin{equation}\notag
\begin{array}{lll}
\left[\dive(\vect{a}\otimes\vect{b})\right]_i &= &
\partial_j (a_i\,b_j)
\end{array}
\end{equation}

\minititre{Remarque}
Dans le cas de la prise en compte d'une masse volumique variable, l'�quation de continuit� s'�crit :
$$\frac{\partial \rho}{\partial t} + \dive{\,(\rho\,\vect{u})} = \Gamma  $$
Cette �quation n'est pas prise en compte dans l'�tape de projection (on continue � r�soudre
seulement
$\displaystyle \dive(\,{\rho\,\vect{u}}) = \Gamma$) alors que le terme
$\displaystyle \frac{\partial \rho}{\partial t}$ appara\^{\i}t lors de l'�tape de pr\'ediction de la vitesse
dans le sous-programme \fort{preduv}. Si ce terme joue un r�le sensible, l'algorithme compressible
de \CS\ (qui r�sout l'�quation compl�te) est alors sans doute plus adapt�.

%                      Code_Saturne version 1.3
%                      ------------------------
%
%     This file is part of the Code_Saturne Kernel, element of the
%     Code_Saturne CFD tool.
% 
%     Copyright (C) 1998-2007 EDF S.A., France
%
%     contact: saturne-support@edf.fr
% 
%     The Code_Saturne Kernel is free software; you can redistribute it
%     and/or modify it under the terms of the GNU General Public License
%     as published by the Free Software Foundation; either version 2 of
%     the License, or (at your option) any later version.
% 
%     The Code_Saturne Kernel is distributed in the hope that it will be
%     useful, but WITHOUT ANY WARRANTY; without even the implied warranty
%     of MERCHANTABILITY or FITNESS FOR A PARTICULAR PURPOSE.  See the
%     GNU General Public License for more details.
% 
%     You should have received a copy of the GNU General Public License
%     along with the Code_Saturne Kernel; if not, write to the
%     Free Software Foundation, Inc.,
%     51 Franklin St, Fifth Floor,
%     Boston, MA  02110-1301  USA
%
%-----------------------------------------------------------------------
%
%%%%%%%%%%%%%%%%%%%%%%%%%%%%%%%%%
%%%%%%%%%%%%%%%%%%%%%%%%%%%%%%%%%%
\section{Discr\'etisation}
%%%%%%%%%%%%%%%%%%%%%%%%%%%%%%%%%%
%%%%%%%%%%%%%%%%%%%%%%%%%%%%%%%%%%

Pour utiliser la m�thode, on se place tout d'abord dans un rep�re local d�fini
de mani�re � ce que le plan $(0yz)$, o� sont inject�s les vortex, soit confondu
avec le plan d'entr�e du calcul (voir figure \ref{Base_Vortex_entree}). 

\begin{figure}[h]
\centerline{\includegraphics[height=6cm]{../Base/Vortex/Images/entree.pdf}}
\caption{\label{Base_Vortex_entree} D�finiton des diff�rentes grandeurs dans le rep�re local
correspondant � l'entr�e d'une conduite de section carr�e.} 
\end{figure}

$u$, $v$ et $w$  sont les composantes de la vitesse fluctuante (principale et
transverse) dans ce plan, et
$\displaystyle \omega(y,z) = \frac{\partial w}{\partial y}-\frac{\partial v}{\partial z}$
la vorticit� dans la direction
normale au plan d'entr�e. $\overline{U}(y,z)$ repr�sente ici la vitesse
principale moyenne impos�e par l'utilisateur dans le plan d'entr�e. 

Chaque vortex $p$ va �tre caract�ris� par sa fonction de forme $\xi$ (identique
pour tous les vortex), sa
circulation $\Gamma_p$, son rayon $\sigma_p$ et les coordonn�es $(y_p,z_p)$ du
point $P$ o� est situ� le vortex dans le plan $(0yz)$. 

Pour cela, on suppose que la vorticit� g�n�r�e par un vortex $p$ au point $M$ de
coordonn�e $(y,z)$ s'�crit : 
\begin{equation}\notag
\omega_p(y,z)= \Gamma_p \, \xi_{\sigma_p}(r)
\end{equation}
o� $r = \sqrt{(y-y_p)^2+(z-z_p)^2}$ est la distance s�parant le point $M$ du point $P$.

Dans la m�thode implant�e, la fonction de forme est de type gaussienne modifi�e :
\begin{equation}\notag
\displaystyle
\xi_\sigma (r) = \frac{1}{2\pi \sigma^2} 
\left(2 e^{-\frac{r^2}{2\sigma^2}}-1\right) e^{-\frac{r^2}{2\sigma^2}}
\end{equation}

Le champ de vitesse induit par cette distribution de vorticit� s'obtient par
inversion des deux �quations de poisson suivantes qui sont d�duites de la
condition d'incompressibilit� dans la plan\footnote{\textit{i.e}
$\displaystyle \frac{\partial v}{\partial y}+\frac{\partial w}{\partial w} = 0$} :
\begin{equation}\notag
\begin{array}{lcr}
\displaystyle
\frac{\partial \omega}{\partial y} = \Delta w
&
\text{    et    }
&
\displaystyle
\frac{\partial \omega}{\partial y} = -\Delta v
\\
\end{array}
\end{equation}

Dans le cas g�n�ral, ce syst�me peut �tre int�gr� � l'aide de la formule de Biot et Savart.

Dans le cas d'une distribution de vorticit� de type gaussienne modifi�e, les
composantes de vitesse v�rifient : 
\begin{equation}\notag
\left\{
\begin{array}{c}
\displaystyle
v_p(y,x) = - \frac{1}{2\pi} \frac{(z-z_p)}{r^2}\left(1 -
e^{-\frac{r^2}{2\sigma^2}}\right)\,e^{-\frac{r^2}{2\sigma^2}} 
\\
\displaystyle
w_p(y,z) = \frac{1}{2\pi} \frac{(y-y_p)}{r^2}\left(1 -e^{-\frac{r^2}{2\sigma^2}}
\right)\,e^{-\frac{r^2}{2\sigma^2}} 
\end{array}
\right.
\end{equation}

Ces relations s'�tendent de fa�on imm�diate au cas de $N$ vortex distincts.
Le champ de vitesse induit par la distribution de vorticit� 
\begin{equation}
\omega(y,z) = \sum_{p=1}^N \Gamma_p \, \xi_{\sigma_p}(r)
\end{equation}
vaut au point $M$ :
\begin{equation}\notag
\begin{array}{lcr}
v(x,y) = \sum_{p=1}^N \Gamma_p\, v_p(y,z) 
&
\text{    et    }
&
w(y,z) = \sum_{p=1}^N \Gamma_p\, w_p(y,z)
\\
\label{Base_Vortex_compvit}
\end{array}
\end{equation}
%================================
\subsection{Param�tres physiques}
%================================

%-------------------------------
\subsubsection{Marche en temps}
%-------------------------------
La position initiale de chaque vortex est tir�e de mani�re al�atoire. On calcul
les d�placements successifs de chacun des vortex dans le plan d'entr�e par
int�gration explicite du champ de vitesse lagrangien : 
\begin{equation}\notag
\begin{array}{lcr}
\displaystyle
\frac{dy_p}{dt} = V(y,z)
&
\text{    et    }
&
\displaystyle
\frac{dz_p}{dt} = W(y,z)
\\
\end{array}
\end{equation}
Les vortex sont alors assimil�s � des particules ponctuelles qui sont convect�es
par le champ $(V,W)$. Ce champ peut �tre impos� par des tirages al�atoires ou
bien d�duit de la vitesse induite par les vortex dans le plan d'entr�e. Dans ce
cas $V(x,y) = \overline{V}(y,z) + v (y,z)$ et $W(y,z)= \overline{W}(y,z) +
w(y,z)$ o� $\overline{V}$ et $\overline{W}$ sont les composantes de la vitesse
transverse moyenne qu'impose l'utilisateur � l'aide des fichiers de donn�es. 

%---------------------------------------------------
\subsubsection{Intensit� et dur�e de vie des vortex}
%---------------------------------------------------
Il serait possible, � partir de l'�quation de transport de la vorticit�,
d'obtenir un mod�le d'�volution pour l'intensit� du vecteur tourbillon
$\omega_p$ associ� � chacun des vortex. En pratique, on pr�f�re utiliser un
mod�le simplifi� dans lequel la circulation des tourbillons ne d�pend que de la
postion de ces derniers � l'instant consid�r�. La circulation initiale de chaque
vortex est alors obtenue � partir de la relation : 
\begin{equation}\notag
|\Gamma_p| = 4 \sqrt{\frac{\pi\,S\,k}{3N\,[2ln(3)-3ln(2)]}}
\end{equation}
o� $S$ est la surface du plan d'entr�e, $N$ le nombre de vortex, et $k$
l'�nergie cin�tique turbulente au point o� se trouve le vortex � l'instant
consid�r�. Le signe de $\Gamma_p$ correspond au sens de rotation du vortex et
est tir� al�atoirement. 

Ce param�tre est celui qui contr�le l'intensit� des fluctuations. Sa d�pendance
en $k$ exprime que, plus l'�coulement est turbulent, plus les vortex sont
intenses. La valeur de $k$ est sp�cifi�e par
l'utilisateur. Elle peut �tre constante ou impos�e � partir de profils d'�nergie
cin�tique turbulente en entr�e. 

Pour �viter que des structures trop allong�es ne se d�veloppent au niveau de
l'entr�e, l'utilisateur doit �galement sp�cifier un temps limites $\tau_p$ au
bout duquel le vortex $p$ va �tre d�truit. Ce temps $\tau_p$ peut �tre pris
constant ou estim� au moyen de la relation : 
\begin{equation}\notag
\tau_p = \frac{5 C_{\mu}k^{\frac{3}{2}}}{\varepsilon\,\overline{U}}
\end{equation}

$\overline{U}$ et $\varepsilon$ repr�sentent respectivement la vitesse moyenne
principale et la dissipation turbulente au point o� est initialement g�n�r� le
vortex ($C_{\mu}=0,09$). 
\\
Lorsque le temps �coul� depuis la cr�ation du vortex $p$ est sup�rieur �
$\tau_p$, le vortex est d�truit et un nouveau vortex g�n�r� (sa position et le
signe de $\Gamma_p$ sont tir�s de fa�on al�atoire). 

%-------------------------------- 
\subsubsection{Taille des vortex}
%--------------------------------
La taille des vortex peut �tre prise constante, ou calcul�e � partir des
relations :
\begin{equation}\notag
\begin{array}{ccc}
\displaystyle
\sigma = \frac{C_{\mu}^{\frac{3}{4}}k^{\frac{3}{2}}}{\varepsilon} 
& \text{    ou    } &
\sigma = max[L_t,L_k]
\\
\end{array}
\end{equation}
avec:
\begin{equation}\notag
\begin{array}{ccc}
\displaystyle
L_t = \sqrt{\left( \frac{5 \nu k}{\varepsilon} \right)} 
& \text{    et    } & 
\displaystyle
L_k = 200\, \left(\frac{\nu^3}{\varepsilon}\right)^{\frac{1}{4}}
\end{array}
\end{equation}
o� $\nu$, $k$ et $\varepsilon$ sont la viscosit� dynamique, l'�nergie cin�tique
turbulente et la dissipation turbulente au point o� se trouve le vortex. 

Dans tous les cas, la taille du vortex doit �tre sup�rieure � la taille des
mailles en entr�e afin que le vortex soit effectivement simul�. 

%==================================
\subsection{Conditions aux limites}
%==================================
Le champ de vitesse g�n�r� � l'aide de la relation \ref{Base_Vortex_compvit} ne tient pas
compte {\em a priori} des conditions aux limites appliqu�es sur les bords du plan
d'entr�e. Pour obtenir des valeurs de la vitesse qui soient coh�rentes sur les
fronti�res du domaine d'entr�e, des ``vortex images'', pseudo-vortex situ�s en
dehors du domaine d'entr�e, sont g�n�r�s � des positions particuli�res et leur
champ de vitesse associ� est superpos� au champ pr�c�demment calcul�.\\
Seuls les cas d'une conduite rectangulaire et d'une conduite circulaire
permettent la g�n�ration de ces pseudo-vortex.
On distingue pour cela trois types de conditions aux limites. 

\begin{figure}[h]
\centerline{\includegraphics[height=6cm]{../Base/Vortex/Images/condlimite.pdf}}
\caption{\label{Base_Vortex_condli} Principe de g�n�ration des ``vortex images'' suivant le
type de conditions aux limites dans une conduite carr�e.} 
\end{figure}

%----------------------------------
\subsubsection{Condition de paroi}
%----------------------------------
On cr�e, pour chaque vortex $P$ contenu dans le plan d'entr�e, un vortex image
$P'$ identique � $P$ (\textit{i.e} de m�me caract�ristiques) et sym�trique de $P$
par rapport au
point $J$ ($J$ �tant la projection orthogonalement � la paroi du point $M$
correspondant au centre de la face o� l'on cherche � calculer la vitesse). La
figure \ref{Base_Vortex_condli} illustre la technique dans le cas d'une conduite
carr�e. Dans ce cas les coordonn�es du vortex situ� en $P'$ v�rifient
$(y_{p'}+y_{p})/2 = y_{J}$ et $(z_{p'}+ z_{p})/2 = z_{J}$. Le champ de vitesse
per�u depuis le point $M$ au niveau du point $J$ est nul, ce qui est bien
l'effet recherch�. 

%------------------------------------
\subsubsection{Condition de sym�trie}
%-------------------------------------
La technique est identique � celle utilis�e pour les conditions de paroi, mais
seule la composante pour la vitesse normale au bord est modifi�e dans ce cas. 

%---------------------------------------
\subsubsection{Condition de p�riodicit�}
%---------------------------------------
On cr�e pour chaque vortex, un vortex images $P'$ identique � $P$ mais translat�
d'une quantit� $L$ correspondant � la longueur qui s�pare les deux plans de la
section d'entr�e o� sont appliqu�es les conditions de p�riodicit�. Dans le cas
o� il y a deux directions de p�riodicit�, on cr�e deux vortex image.

%=============================================
\subsection{Composante de vitesse principale}
%=============================================
La m�thode des vortex ne g�n�rant pas de fluctuation $u$ de la vitesse dans la
direction principale, la derni�re composante est calcul�e � partir d'une
�quation de Langevin. Les coefficients de cette �quation sont d�termin�s par
identification des expressions obtenues pour les contraintes de Reynolds en
$R_{ij}-\varepsilon$. Dans le cas d'un �coulement en canal plan, cette technique
conduit � l'�quation : 
\begin{equation}\notag
\displaystyle
\frac{du}{dt} = - \frac{C_1}{2T} u + \left(\frac{2}{3}C_2-1\right)\frac{\partial
U}{\partial y} v + \sqrt{C_0\varepsilon} dW_i 
\end{equation}

avec $\displaystyle T=\frac{k}{\varepsilon}$, $C_1 = 1,8$, $C_2 = 0,6$,
$C_0=\frac{14}{15}$, et $dW_i$ une variable al�toire Gaussienne de variance
$\sqrt{dt}$. 

En pratique, l'�quation de Langevin n'am�liore pas vraiment les r�sultats. Elle
n'est utilis�e que dans le cas de conduites circulaires. 

%                      Code_Saturne version 1.3
%                      ------------------------
%
%     This file is part of the Code_Saturne Kernel, element of the
%     Code_Saturne CFD tool.
%
%     Copyright (C) 1998-2007 EDF S.A., France
%
%     contact: saturne-support@edf.fr
%
%     The Code_Saturne Kernel is free software; you can redistribute it
%     and/or modify it under the terms of the GNU General Public License
%     as published by the Free Software Foundation; either version 2 of
%     the License, or (at your option) any later version.
%
%     The Code_Saturne Kernel is distributed in the hope that it will be
%     useful, but WITHOUT ANY WARRANTY; without even the implied warranty
%     of MERCHANTABILITY or FITNESS FOR A PARTICULAR PURPOSE.  See the
%     GNU General Public License for more details.
%
%     You should have received a copy of the GNU General Public License
%     along with the Code_Saturne Kernel; if not, write to the
%     Free Software Foundation, Inc.,
%     51 Franklin St, Fifth Floor,
%     Boston, MA  02110-1301  USA
%
%-----------------------------------------------------------------------
%

%%%%%%%%%%%%%%%%%%%%%%%%%%%%%%%%%%
%%%%%%%%%%%%%%%%%%%%%%%%%%%%%%%%%%
\section{Mise en \oe uvre}
%%%%%%%%%%%%%%%%%%%%%%%%%%%%%%%%%%
%%%%%%%%%%%%%%%%%%%%%%%%%%%%%%%%%%
Le syst\`eme (\ref{Cfbl_Cfmsvl_eq_densite_finale_cfmsvl}) est r\'esolu par une m\'ethode
d'incr\'ement et r\'esidu en utilisant
une m\'ethode de Jacobi pour inverser le syst\`eme si le terme convectif
est implicite et en utilisant une m\'ethode de gradient conjugu\'e
si le terme convectif est explicite (qui est le cas par d�faut).

Attention, les valeurs du flux de masse $\rho\,\vect{w}\cdot\vect{S}$ et
de la viscosit\'e $\Delta\,t\,c^2\frac{S}{d}$ aux faces de
bord, qui sont calcul\'ees dans \fort{cfmsfl} et \fort{cfmsvs} respectivement,
sont modifi\'ees imm\'ediatement apr\`es l'appel \`a ces sous-programmes.
En effet, il est indispensable que la contribution de bord de
$\left(\rho\,\vect{w}-\Delta\,t\,(c^2)\,\gradv\,\rho\right)\cdot\vect{S}$
repr\'esente exactement $\vect{Q}_{ac}\cdot\vect{S}$.
Pour cela,
\begin{itemize}
\item imm\'ediatement apr\`es l'appel \`a
\fort{cfmsfl}, on remplace la contribution de bord de
$\rho\,\vect{w}\cdot\vect{S}$
par le flux de masse exact, $\vect{Q}_{ac}\cdot\vect{S}$,
d\'etermin\'e \`a partir des conditions aux limites,
\item puis, imm\'ediatement apr\`es l'appel \`a
\fort{cfmsvs}, on annule la viscosit\'e au bord $\Delta\,t\,(c^2)$ pour
\'eliminer la contribution de $-\Delta\,t\,(c^2)\,(\gradv\,\rho)\cdot\vect{S}$
(l'annulation de la viscosit\'e n'est pas probl\'ematique pour la matrice,
puisqu'elle porte sur des incr\'ements).
\end{itemize}

\bigskip

Une fois qu'on a obtenu $\rho^{n+1}$,
on peut actualiser le flux de masse acoustique
aux faces $(\vect{Q}_{ac}^{n+1})_{ij} \cdot \vect{S}_{ij}$,
qui servira pour la convection des autres variables~:
\begin{equation}\label{Cfbl_Cfmsvl_eq_flux_masse_acoustique_cfmsvl}
\displaystyle(\vect{Q}_{ac}^{n+1})_{ij}\cdot\vect{S}_{ij}=
-\left(\Delta t^n (c^2)^n \gradv(\rho^{n+1})\right)_{ij}\cdot\vect{S}_{ij}
+\left(\rho^{n+\frac{1}{2}} \vect{w}^n\right)_{ij}\cdot\vect{S}_{ij}\\
\end{equation}
Ce calcul de flux est r\'ealis\'e par \fort{cfbsc3}.
Si l'on a choisi l'algorithme standard, \'equation~(\ref{Cfbl_Cfmsvl_eq_densite_cfmsvl}),
on compl\`ete le flux dans \fort{cfmsvl} imm\'ediatement apr\`es l'appel
\`a \fort{cfbsc3}.
En effet, dans ce cas,
la convection est explicite ($\rho^{n+\frac{1}{2}}=\rho^{n}$,
obtenu en imposant \var{ICONV(ISCA(IRHO(IPHAS)))=0})
et le sous-programme \fort{cfbsc3},
qui calcule le flux de masse aux faces,
ne prend pas en compte la contribution du terme
$\rho^{n+\frac{1}{2}}\,\vect{w}^n\cdot\vect{S}$. On ajoute donc cette
contribution dans \fort{cfmsvl}, apr\`es l'appel \`a \fort{cfbsc3}.
Au bord, en particulier, c'est bien le flux de masse calcul\'e \`a partir
des conditions aux limites que l'on obtient.

On actualise la pression \`a la fin de l'\'etape, en utilisant la loi d'\'etat~:
\begin{equation}
\displaystyle P_i^{pred}=P(\rho_i^{n+1},\varepsilon_i^{n})
\end{equation}


%%%%%%%%%%%%%%%%%%%%%%%%%%%%%%%%%%
%%%%%%%%%%%%%%%%%%%%%%%%%%%%%%%%%%
\section{Points \`a traiter}
%%%%%%%%%%%%%%%%%%%%%%%%%%%%%%%%%%
%%%%%%%%%%%%%%%%%%%%%%%%%%%%%%%%%%
Le calcul du flux de masse au  bord n'est pas enti\`erement satisfaisant
si la convection est trait\'ee de mani\`ere implicite
(algorithme non standard, non test\'e,
associ\'e \`a l'\'equation~(\ref{Cfbl_Cfmsvl_eq_densite_bis_cfmsvl}),
correspondant au choix $\rho^{n+\frac{1}{2}}=\rho^{n+1}$ et
obtenu en imposant \var{ICONV(ISCA(IRHO(IPHAS)))=1}).
En effet, apr\`es \fort{cfmsfl}, il faut d\'eterminer la vitesse de
convection $\vect{w}^n$ pour qu'apparaisse
$\rho^{n+1} \vect{w}^n\cdot\vect{n}$
au cours de la r\'esolution par \fort{codits}. De ce fait, on doit d\'eduire
une valeur de $\vect{w}^n$ \`a partir de la valeur
du flux de masse. Au bord, en particulier, il faut
donc diviser le flux de masse
issu des conditions aux limites par la valeur de bord de $\rho^{n+1}$.
Or, lorsque des conditions de Neumann sont appliqu\'ees \`a la
masse volumique,
la valeur de $\rho^{n+1}$ au bord n'est pas connue avant la r\'esolution du
syst\`eme. On utilise donc, au lieu de la valeur de bord inconnue de
$\rho^{n+1}$ la valeur de bord prise au pas de temps
pr\'ec\'edent $\rho^{n}$. Cette approximation est susceptible
d'affecter la valeur du flux de masse au bord.

\passepage
\part{Module compressible}
%                      Code_Saturne version 1.3
%                      ------------------------
%
%     This file is part of the Code_Saturne Kernel, element of the
%     Code_Saturne CFD tool.
%
%     Copyright (C) 1998-2007 EDF S.A., France
%
%     contact: saturne-support@edf.fr
%
%     The Code_Saturne Kernel is free software; you can redistribute it
%     and/or modify it under the terms of the GNU General Public License
%     as published by the Free Software Foundation; either version 2 of
%     the License, or (at your option) any later version.
%
%     The Code_Saturne Kernel is distributed in the hope that it will be
%     useful, but WITHOUT ANY WARRANTY; without even the implied warranty
%     of MERCHANTABILITY or FITNESS FOR A PARTICULAR PURPOSE.  See the
%     GNU General Public License for more details.
%
%     You should have received a copy of the GNU General Public License
%     along with the Code_Saturne Kernel; if not, write to the
%     Free Software Foundation, Inc.,
%     51 Franklin St, Fifth Floor,
%     Boston, MA  02110-1301  USA
%
%-----------------------------------------------------------------------
%


\programme{navsto}

\vspace{1cm}
On s'int\'eresse \`a la r\'esolution du syst\`eme d'\'equations de Navier-Stokes
tridimensionnelles monophasiques, \`a une pression, instationnaires, en
incompressible ou faiblement dilatable, bas\'ees sur une discr\'etisation
temporelle de type Euler implicite d'ordre 1 ou Crank-Nicolson d'ordre 2 et sur
une discr\'etisation spatiale  par volumes finis colocalis\'ee.


%%%%%%%%%%%%%%%%%%%%%%%%%%%%%%%%%%
%%%%%%%%%%%%%%%%%%%%%%%%%%%%%%%%%%
\section{Fonction}
%%%%%%%%%%%%%%%%%%%%%%%%%%%%%%%%%%
%%%%%%%%%%%%%%%%%%%%%%%%%%%%%%%%%%

  Dans ce sous-programme sont calcul\'ees, \`a un pas de temps donn\'e, les
variables vitesse et pression de ce probl\`eme en proc\'edant en
deux  \'etapes issues d'une d\'ecomposition des op\'erateurs (m\'ethode \`a
pas fractionnaires).\\
Les variables sont donc suppos\'ees connues \`a
l'instant ${t^n}$ et on cherche \`a les d\'eterminer \`a l'instant\footnote{La pression est suppos�e connue � l'instant $t^{n-1+\theta}$ et recherch�e en $t^{n+\theta}$, avec $\theta=1$ ou $1/2$ suivant le sch�ma en temps consid�r�.} ${t^{n+1}}$. Soit ${\Delta {t^n} ={t^{n+1}- {t^n}}}$ le pas de temps associ\'e. Dans un premier temps, on r\'ealise l'\'etape de
pr\'ediction de la vitesse en r\'esolvant l'\'equation de quantit\'e de
mouvement avec une pression explicite. Suit l'\'etape de correction de la
pression (ou projection de la vitesse) qui permet d'obtenir un champ de vitesse \`a divergence nulle.\\\\
Les \'equations en continu sont donc :
\begin{equation}
\left\{\begin{array}{l}
\displaystyle\frac{\partial}{\partial t}(\rho \vect{u}) + \dive(\rho\, \vect{u} \otimes \vect{u})
=\dive(\tens{\sigma}) + \vect{TS} - \tens{K}\,\vect{u}\\
\dive(\rho \vect{u}) = \Gamma
\end{array}\right.
\end{equation}

%(plus tard $\frac{\partial \rho}{\partial t} + \dive(\rho \vect{u}) = \Gamma$)



avec $\rho$ la masse volumique, $\vect{u}$ le champ de vitesse,
$[\,\vect{TS}-\tens{K}\,\vect{u}\,]$ les autres termes sources ($\tens{K}$~est un
tenseur diagonal positif par d\'efinition), $\tens{\sigma}$ le tenseur
de contraintes, $\tens{\tau}$ le tenseur des contraintes visqueuses, $\mu$ la
viscosit\'e dynamique (mol\'eculaire et \'eventuellement turbulente), $\kappa$
la viscosit� de
volume (usuellement nulle et n�glig�e dans le code et dans la suite du document,
sauf en compressible),
$\tens{D}$ le tenseur taux de d\'eformation\footnote{\`A ne pas confondre, malgr\'e la m\^eme notation $D$,
avec les flux diffusifs $\vect{D}_{\,ij}$ et $\vect{D}_{\,{b}_{ik}}$ d\'ecrits par la suite dans ce
sous-programme.}, $\Gamma$ le terme source de masse.
\begin{equation}
\left\{\begin{array}{l}
\tens{\sigma} = \tens{\tau} - P\tens{Id}  \\
\tens{\tau} = 2\,\mu\ \tens{D} +\ (\kappa\ - \frac{2}{3}\mu)\  tr({\tens{D}})\
\tens{Id}  \\
\tens{D} = \frac{1}{2}(\ggrad\vect{u}+\,^{t}\ggrad\vect{u})
\end{array}\right.
\end{equation}
 \\

On rappelle la d\'efinition des notations employ\'ees\footnote{en
utilisant la convention de sommation d'Einstein.}~:
\begin{equation}\notag
\left\{\begin{array}{lll}
\left[\ggrad{\vect{a}}\right]_{ij} &=& \partial_j a_i\\
\left[\dive(\tens{\sigma})\right]_i &=& \partial_j \sigma_{ij}\\
\left[\vect{a}\otimes\vect{b}\right]_{ij} &= &
a_i\,b_j\\
\end{array}\right.
\end{equation}
et donc :
\begin{equation}\notag
\begin{array}{lll}
\left[\dive(\vect{a}\otimes\vect{b})\right]_i &= &
\partial_j (a_i\,b_j)
\end{array}
\end{equation}

\minititre{Remarque}
Dans le cas de la prise en compte d'une masse volumique variable, l'�quation de continuit� s'�crit :
$$\frac{\partial \rho}{\partial t} + \dive{\,(\rho\,\vect{u})} = \Gamma  $$
Cette �quation n'est pas prise en compte dans l'�tape de projection (on continue � r�soudre
seulement
$\displaystyle \dive(\,{\rho\,\vect{u}}) = \Gamma$) alors que le terme
$\displaystyle \frac{\partial \rho}{\partial t}$ appara\^{\i}t lors de l'�tape de pr\'ediction de la vitesse
dans le sous-programme \fort{preduv}. Si ce terme joue un r�le sensible, l'algorithme compressible
de \CS\ (qui r�sout l'�quation compl�te) est alors sans doute plus adapt�.

%                      Code_Saturne version 1.3
%                      ------------------------
%
%     This file is part of the Code_Saturne Kernel, element of the
%     Code_Saturne CFD tool.
% 
%     Copyright (C) 1998-2007 EDF S.A., France
%
%     contact: saturne-support@edf.fr
% 
%     The Code_Saturne Kernel is free software; you can redistribute it
%     and/or modify it under the terms of the GNU General Public License
%     as published by the Free Software Foundation; either version 2 of
%     the License, or (at your option) any later version.
% 
%     The Code_Saturne Kernel is distributed in the hope that it will be
%     useful, but WITHOUT ANY WARRANTY; without even the implied warranty
%     of MERCHANTABILITY or FITNESS FOR A PARTICULAR PURPOSE.  See the
%     GNU General Public License for more details.
% 
%     You should have received a copy of the GNU General Public License
%     along with the Code_Saturne Kernel; if not, write to the
%     Free Software Foundation, Inc.,
%     51 Franklin St, Fifth Floor,
%     Boston, MA  02110-1301  USA
%
%-----------------------------------------------------------------------
%
%%%%%%%%%%%%%%%%%%%%%%%%%%%%%%%%%
%%%%%%%%%%%%%%%%%%%%%%%%%%%%%%%%%%
\section{Discr\'etisation}
%%%%%%%%%%%%%%%%%%%%%%%%%%%%%%%%%%
%%%%%%%%%%%%%%%%%%%%%%%%%%%%%%%%%%

Pour utiliser la m�thode, on se place tout d'abord dans un rep�re local d�fini
de mani�re � ce que le plan $(0yz)$, o� sont inject�s les vortex, soit confondu
avec le plan d'entr�e du calcul (voir figure \ref{Base_Vortex_entree}). 

\begin{figure}[h]
\centerline{\includegraphics[height=6cm]{../Base/Vortex/Images/entree.pdf}}
\caption{\label{Base_Vortex_entree} D�finiton des diff�rentes grandeurs dans le rep�re local
correspondant � l'entr�e d'une conduite de section carr�e.} 
\end{figure}

$u$, $v$ et $w$  sont les composantes de la vitesse fluctuante (principale et
transverse) dans ce plan, et
$\displaystyle \omega(y,z) = \frac{\partial w}{\partial y}-\frac{\partial v}{\partial z}$
la vorticit� dans la direction
normale au plan d'entr�e. $\overline{U}(y,z)$ repr�sente ici la vitesse
principale moyenne impos�e par l'utilisateur dans le plan d'entr�e. 

Chaque vortex $p$ va �tre caract�ris� par sa fonction de forme $\xi$ (identique
pour tous les vortex), sa
circulation $\Gamma_p$, son rayon $\sigma_p$ et les coordonn�es $(y_p,z_p)$ du
point $P$ o� est situ� le vortex dans le plan $(0yz)$. 

Pour cela, on suppose que la vorticit� g�n�r�e par un vortex $p$ au point $M$ de
coordonn�e $(y,z)$ s'�crit : 
\begin{equation}\notag
\omega_p(y,z)= \Gamma_p \, \xi_{\sigma_p}(r)
\end{equation}
o� $r = \sqrt{(y-y_p)^2+(z-z_p)^2}$ est la distance s�parant le point $M$ du point $P$.

Dans la m�thode implant�e, la fonction de forme est de type gaussienne modifi�e :
\begin{equation}\notag
\displaystyle
\xi_\sigma (r) = \frac{1}{2\pi \sigma^2} 
\left(2 e^{-\frac{r^2}{2\sigma^2}}-1\right) e^{-\frac{r^2}{2\sigma^2}}
\end{equation}

Le champ de vitesse induit par cette distribution de vorticit� s'obtient par
inversion des deux �quations de poisson suivantes qui sont d�duites de la
condition d'incompressibilit� dans la plan\footnote{\textit{i.e}
$\displaystyle \frac{\partial v}{\partial y}+\frac{\partial w}{\partial w} = 0$} :
\begin{equation}\notag
\begin{array}{lcr}
\displaystyle
\frac{\partial \omega}{\partial y} = \Delta w
&
\text{    et    }
&
\displaystyle
\frac{\partial \omega}{\partial y} = -\Delta v
\\
\end{array}
\end{equation}

Dans le cas g�n�ral, ce syst�me peut �tre int�gr� � l'aide de la formule de Biot et Savart.

Dans le cas d'une distribution de vorticit� de type gaussienne modifi�e, les
composantes de vitesse v�rifient : 
\begin{equation}\notag
\left\{
\begin{array}{c}
\displaystyle
v_p(y,x) = - \frac{1}{2\pi} \frac{(z-z_p)}{r^2}\left(1 -
e^{-\frac{r^2}{2\sigma^2}}\right)\,e^{-\frac{r^2}{2\sigma^2}} 
\\
\displaystyle
w_p(y,z) = \frac{1}{2\pi} \frac{(y-y_p)}{r^2}\left(1 -e^{-\frac{r^2}{2\sigma^2}}
\right)\,e^{-\frac{r^2}{2\sigma^2}} 
\end{array}
\right.
\end{equation}

Ces relations s'�tendent de fa�on imm�diate au cas de $N$ vortex distincts.
Le champ de vitesse induit par la distribution de vorticit� 
\begin{equation}
\omega(y,z) = \sum_{p=1}^N \Gamma_p \, \xi_{\sigma_p}(r)
\end{equation}
vaut au point $M$ :
\begin{equation}\notag
\begin{array}{lcr}
v(x,y) = \sum_{p=1}^N \Gamma_p\, v_p(y,z) 
&
\text{    et    }
&
w(y,z) = \sum_{p=1}^N \Gamma_p\, w_p(y,z)
\\
\label{Base_Vortex_compvit}
\end{array}
\end{equation}
%================================
\subsection{Param�tres physiques}
%================================

%-------------------------------
\subsubsection{Marche en temps}
%-------------------------------
La position initiale de chaque vortex est tir�e de mani�re al�atoire. On calcul
les d�placements successifs de chacun des vortex dans le plan d'entr�e par
int�gration explicite du champ de vitesse lagrangien : 
\begin{equation}\notag
\begin{array}{lcr}
\displaystyle
\frac{dy_p}{dt} = V(y,z)
&
\text{    et    }
&
\displaystyle
\frac{dz_p}{dt} = W(y,z)
\\
\end{array}
\end{equation}
Les vortex sont alors assimil�s � des particules ponctuelles qui sont convect�es
par le champ $(V,W)$. Ce champ peut �tre impos� par des tirages al�atoires ou
bien d�duit de la vitesse induite par les vortex dans le plan d'entr�e. Dans ce
cas $V(x,y) = \overline{V}(y,z) + v (y,z)$ et $W(y,z)= \overline{W}(y,z) +
w(y,z)$ o� $\overline{V}$ et $\overline{W}$ sont les composantes de la vitesse
transverse moyenne qu'impose l'utilisateur � l'aide des fichiers de donn�es. 

%---------------------------------------------------
\subsubsection{Intensit� et dur�e de vie des vortex}
%---------------------------------------------------
Il serait possible, � partir de l'�quation de transport de la vorticit�,
d'obtenir un mod�le d'�volution pour l'intensit� du vecteur tourbillon
$\omega_p$ associ� � chacun des vortex. En pratique, on pr�f�re utiliser un
mod�le simplifi� dans lequel la circulation des tourbillons ne d�pend que de la
postion de ces derniers � l'instant consid�r�. La circulation initiale de chaque
vortex est alors obtenue � partir de la relation : 
\begin{equation}\notag
|\Gamma_p| = 4 \sqrt{\frac{\pi\,S\,k}{3N\,[2ln(3)-3ln(2)]}}
\end{equation}
o� $S$ est la surface du plan d'entr�e, $N$ le nombre de vortex, et $k$
l'�nergie cin�tique turbulente au point o� se trouve le vortex � l'instant
consid�r�. Le signe de $\Gamma_p$ correspond au sens de rotation du vortex et
est tir� al�atoirement. 

Ce param�tre est celui qui contr�le l'intensit� des fluctuations. Sa d�pendance
en $k$ exprime que, plus l'�coulement est turbulent, plus les vortex sont
intenses. La valeur de $k$ est sp�cifi�e par
l'utilisateur. Elle peut �tre constante ou impos�e � partir de profils d'�nergie
cin�tique turbulente en entr�e. 

Pour �viter que des structures trop allong�es ne se d�veloppent au niveau de
l'entr�e, l'utilisateur doit �galement sp�cifier un temps limites $\tau_p$ au
bout duquel le vortex $p$ va �tre d�truit. Ce temps $\tau_p$ peut �tre pris
constant ou estim� au moyen de la relation : 
\begin{equation}\notag
\tau_p = \frac{5 C_{\mu}k^{\frac{3}{2}}}{\varepsilon\,\overline{U}}
\end{equation}

$\overline{U}$ et $\varepsilon$ repr�sentent respectivement la vitesse moyenne
principale et la dissipation turbulente au point o� est initialement g�n�r� le
vortex ($C_{\mu}=0,09$). 
\\
Lorsque le temps �coul� depuis la cr�ation du vortex $p$ est sup�rieur �
$\tau_p$, le vortex est d�truit et un nouveau vortex g�n�r� (sa position et le
signe de $\Gamma_p$ sont tir�s de fa�on al�atoire). 

%-------------------------------- 
\subsubsection{Taille des vortex}
%--------------------------------
La taille des vortex peut �tre prise constante, ou calcul�e � partir des
relations :
\begin{equation}\notag
\begin{array}{ccc}
\displaystyle
\sigma = \frac{C_{\mu}^{\frac{3}{4}}k^{\frac{3}{2}}}{\varepsilon} 
& \text{    ou    } &
\sigma = max[L_t,L_k]
\\
\end{array}
\end{equation}
avec:
\begin{equation}\notag
\begin{array}{ccc}
\displaystyle
L_t = \sqrt{\left( \frac{5 \nu k}{\varepsilon} \right)} 
& \text{    et    } & 
\displaystyle
L_k = 200\, \left(\frac{\nu^3}{\varepsilon}\right)^{\frac{1}{4}}
\end{array}
\end{equation}
o� $\nu$, $k$ et $\varepsilon$ sont la viscosit� dynamique, l'�nergie cin�tique
turbulente et la dissipation turbulente au point o� se trouve le vortex. 

Dans tous les cas, la taille du vortex doit �tre sup�rieure � la taille des
mailles en entr�e afin que le vortex soit effectivement simul�. 

%==================================
\subsection{Conditions aux limites}
%==================================
Le champ de vitesse g�n�r� � l'aide de la relation \ref{Base_Vortex_compvit} ne tient pas
compte {\em a priori} des conditions aux limites appliqu�es sur les bords du plan
d'entr�e. Pour obtenir des valeurs de la vitesse qui soient coh�rentes sur les
fronti�res du domaine d'entr�e, des ``vortex images'', pseudo-vortex situ�s en
dehors du domaine d'entr�e, sont g�n�r�s � des positions particuli�res et leur
champ de vitesse associ� est superpos� au champ pr�c�demment calcul�.\\
Seuls les cas d'une conduite rectangulaire et d'une conduite circulaire
permettent la g�n�ration de ces pseudo-vortex.
On distingue pour cela trois types de conditions aux limites. 

\begin{figure}[h]
\centerline{\includegraphics[height=6cm]{../Base/Vortex/Images/condlimite.pdf}}
\caption{\label{Base_Vortex_condli} Principe de g�n�ration des ``vortex images'' suivant le
type de conditions aux limites dans une conduite carr�e.} 
\end{figure}

%----------------------------------
\subsubsection{Condition de paroi}
%----------------------------------
On cr�e, pour chaque vortex $P$ contenu dans le plan d'entr�e, un vortex image
$P'$ identique � $P$ (\textit{i.e} de m�me caract�ristiques) et sym�trique de $P$
par rapport au
point $J$ ($J$ �tant la projection orthogonalement � la paroi du point $M$
correspondant au centre de la face o� l'on cherche � calculer la vitesse). La
figure \ref{Base_Vortex_condli} illustre la technique dans le cas d'une conduite
carr�e. Dans ce cas les coordonn�es du vortex situ� en $P'$ v�rifient
$(y_{p'}+y_{p})/2 = y_{J}$ et $(z_{p'}+ z_{p})/2 = z_{J}$. Le champ de vitesse
per�u depuis le point $M$ au niveau du point $J$ est nul, ce qui est bien
l'effet recherch�. 

%------------------------------------
\subsubsection{Condition de sym�trie}
%-------------------------------------
La technique est identique � celle utilis�e pour les conditions de paroi, mais
seule la composante pour la vitesse normale au bord est modifi�e dans ce cas. 

%---------------------------------------
\subsubsection{Condition de p�riodicit�}
%---------------------------------------
On cr�e pour chaque vortex, un vortex images $P'$ identique � $P$ mais translat�
d'une quantit� $L$ correspondant � la longueur qui s�pare les deux plans de la
section d'entr�e o� sont appliqu�es les conditions de p�riodicit�. Dans le cas
o� il y a deux directions de p�riodicit�, on cr�e deux vortex image.

%=============================================
\subsection{Composante de vitesse principale}
%=============================================
La m�thode des vortex ne g�n�rant pas de fluctuation $u$ de la vitesse dans la
direction principale, la derni�re composante est calcul�e � partir d'une
�quation de Langevin. Les coefficients de cette �quation sont d�termin�s par
identification des expressions obtenues pour les contraintes de Reynolds en
$R_{ij}-\varepsilon$. Dans le cas d'un �coulement en canal plan, cette technique
conduit � l'�quation : 
\begin{equation}\notag
\displaystyle
\frac{du}{dt} = - \frac{C_1}{2T} u + \left(\frac{2}{3}C_2-1\right)\frac{\partial
U}{\partial y} v + \sqrt{C_0\varepsilon} dW_i 
\end{equation}

avec $\displaystyle T=\frac{k}{\varepsilon}$, $C_1 = 1,8$, $C_2 = 0,6$,
$C_0=\frac{14}{15}$, et $dW_i$ une variable al�toire Gaussienne de variance
$\sqrt{dt}$. 

En pratique, l'�quation de Langevin n'am�liore pas vraiment les r�sultats. Elle
n'est utilis�e que dans le cas de conduites circulaires. 

%                      Code_Saturne version 1.3
%                      ------------------------
%
%     This file is part of the Code_Saturne Kernel, element of the
%     Code_Saturne CFD tool.
%
%     Copyright (C) 1998-2007 EDF S.A., France
%
%     contact: saturne-support@edf.fr
%
%     The Code_Saturne Kernel is free software; you can redistribute it
%     and/or modify it under the terms of the GNU General Public License
%     as published by the Free Software Foundation; either version 2 of
%     the License, or (at your option) any later version.
%
%     The Code_Saturne Kernel is distributed in the hope that it will be
%     useful, but WITHOUT ANY WARRANTY; without even the implied warranty
%     of MERCHANTABILITY or FITNESS FOR A PARTICULAR PURPOSE.  See the
%     GNU General Public License for more details.
%
%     You should have received a copy of the GNU General Public License
%     along with the Code_Saturne Kernel; if not, write to the
%     Free Software Foundation, Inc.,
%     51 Franklin St, Fifth Floor,
%     Boston, MA  02110-1301  USA
%
%-----------------------------------------------------------------------
%

%%%%%%%%%%%%%%%%%%%%%%%%%%%%%%%%%%
%%%%%%%%%%%%%%%%%%%%%%%%%%%%%%%%%%
\section{Mise en \oe uvre}
%%%%%%%%%%%%%%%%%%%%%%%%%%%%%%%%%%
%%%%%%%%%%%%%%%%%%%%%%%%%%%%%%%%%%
Le syst\`eme (\ref{Cfbl_Cfmsvl_eq_densite_finale_cfmsvl}) est r\'esolu par une m\'ethode
d'incr\'ement et r\'esidu en utilisant
une m\'ethode de Jacobi pour inverser le syst\`eme si le terme convectif
est implicite et en utilisant une m\'ethode de gradient conjugu\'e
si le terme convectif est explicite (qui est le cas par d�faut).

Attention, les valeurs du flux de masse $\rho\,\vect{w}\cdot\vect{S}$ et
de la viscosit\'e $\Delta\,t\,c^2\frac{S}{d}$ aux faces de
bord, qui sont calcul\'ees dans \fort{cfmsfl} et \fort{cfmsvs} respectivement,
sont modifi\'ees imm\'ediatement apr\`es l'appel \`a ces sous-programmes.
En effet, il est indispensable que la contribution de bord de
$\left(\rho\,\vect{w}-\Delta\,t\,(c^2)\,\gradv\,\rho\right)\cdot\vect{S}$
repr\'esente exactement $\vect{Q}_{ac}\cdot\vect{S}$.
Pour cela,
\begin{itemize}
\item imm\'ediatement apr\`es l'appel \`a
\fort{cfmsfl}, on remplace la contribution de bord de
$\rho\,\vect{w}\cdot\vect{S}$
par le flux de masse exact, $\vect{Q}_{ac}\cdot\vect{S}$,
d\'etermin\'e \`a partir des conditions aux limites,
\item puis, imm\'ediatement apr\`es l'appel \`a
\fort{cfmsvs}, on annule la viscosit\'e au bord $\Delta\,t\,(c^2)$ pour
\'eliminer la contribution de $-\Delta\,t\,(c^2)\,(\gradv\,\rho)\cdot\vect{S}$
(l'annulation de la viscosit\'e n'est pas probl\'ematique pour la matrice,
puisqu'elle porte sur des incr\'ements).
\end{itemize}

\bigskip

Une fois qu'on a obtenu $\rho^{n+1}$,
on peut actualiser le flux de masse acoustique
aux faces $(\vect{Q}_{ac}^{n+1})_{ij} \cdot \vect{S}_{ij}$,
qui servira pour la convection des autres variables~:
\begin{equation}\label{Cfbl_Cfmsvl_eq_flux_masse_acoustique_cfmsvl}
\displaystyle(\vect{Q}_{ac}^{n+1})_{ij}\cdot\vect{S}_{ij}=
-\left(\Delta t^n (c^2)^n \gradv(\rho^{n+1})\right)_{ij}\cdot\vect{S}_{ij}
+\left(\rho^{n+\frac{1}{2}} \vect{w}^n\right)_{ij}\cdot\vect{S}_{ij}\\
\end{equation}
Ce calcul de flux est r\'ealis\'e par \fort{cfbsc3}.
Si l'on a choisi l'algorithme standard, \'equation~(\ref{Cfbl_Cfmsvl_eq_densite_cfmsvl}),
on compl\`ete le flux dans \fort{cfmsvl} imm\'ediatement apr\`es l'appel
\`a \fort{cfbsc3}.
En effet, dans ce cas,
la convection est explicite ($\rho^{n+\frac{1}{2}}=\rho^{n}$,
obtenu en imposant \var{ICONV(ISCA(IRHO(IPHAS)))=0})
et le sous-programme \fort{cfbsc3},
qui calcule le flux de masse aux faces,
ne prend pas en compte la contribution du terme
$\rho^{n+\frac{1}{2}}\,\vect{w}^n\cdot\vect{S}$. On ajoute donc cette
contribution dans \fort{cfmsvl}, apr\`es l'appel \`a \fort{cfbsc3}.
Au bord, en particulier, c'est bien le flux de masse calcul\'e \`a partir
des conditions aux limites que l'on obtient.

On actualise la pression \`a la fin de l'\'etape, en utilisant la loi d'\'etat~:
\begin{equation}
\displaystyle P_i^{pred}=P(\rho_i^{n+1},\varepsilon_i^{n})
\end{equation}


%%%%%%%%%%%%%%%%%%%%%%%%%%%%%%%%%%
%%%%%%%%%%%%%%%%%%%%%%%%%%%%%%%%%%
\section{Points \`a traiter}
%%%%%%%%%%%%%%%%%%%%%%%%%%%%%%%%%%
%%%%%%%%%%%%%%%%%%%%%%%%%%%%%%%%%%
Le calcul du flux de masse au  bord n'est pas enti\`erement satisfaisant
si la convection est trait\'ee de mani\`ere implicite
(algorithme non standard, non test\'e,
associ\'e \`a l'\'equation~(\ref{Cfbl_Cfmsvl_eq_densite_bis_cfmsvl}),
correspondant au choix $\rho^{n+\frac{1}{2}}=\rho^{n+1}$ et
obtenu en imposant \var{ICONV(ISCA(IRHO(IPHAS)))=1}).
En effet, apr\`es \fort{cfmsfl}, il faut d\'eterminer la vitesse de
convection $\vect{w}^n$ pour qu'apparaisse
$\rho^{n+1} \vect{w}^n\cdot\vect{n}$
au cours de la r\'esolution par \fort{codits}. De ce fait, on doit d\'eduire
une valeur de $\vect{w}^n$ \`a partir de la valeur
du flux de masse. Au bord, en particulier, il faut
donc diviser le flux de masse
issu des conditions aux limites par la valeur de bord de $\rho^{n+1}$.
Or, lorsque des conditions de Neumann sont appliqu\'ees \`a la
masse volumique,
la valeur de $\rho^{n+1}$ au bord n'est pas connue avant la r\'esolution du
syst\`eme. On utilise donc, au lieu de la valeur de bord inconnue de
$\rho^{n+1}$ la valeur de bord prise au pas de temps
pr\'ec\'edent $\rho^{n}$. Cette approximation est susceptible
d'affecter la valeur du flux de masse au bord.

%                      Code_Saturne version 1.3
%                      ------------------------
%
%     This file is part of the Code_Saturne Kernel, element of the
%     Code_Saturne CFD tool.
%
%     Copyright (C) 1998-2007 EDF S.A., France
%
%     contact: saturne-support@edf.fr
%
%     The Code_Saturne Kernel is free software; you can redistribute it
%     and/or modify it under the terms of the GNU General Public License
%     as published by the Free Software Foundation; either version 2 of
%     the License, or (at your option) any later version.
%
%     The Code_Saturne Kernel is distributed in the hope that it will be
%     useful, but WITHOUT ANY WARRANTY; without even the implied warranty
%     of MERCHANTABILITY or FITNESS FOR A PARTICULAR PURPOSE.  See the
%     GNU General Public License for more details.
%
%     You should have received a copy of the GNU General Public License
%     along with the Code_Saturne Kernel; if not, write to the
%     Free Software Foundation, Inc.,
%     51 Franklin St, Fifth Floor,
%     Boston, MA  02110-1301  USA
%
%-----------------------------------------------------------------------
%


\programme{navsto}

\vspace{1cm}
On s'int\'eresse \`a la r\'esolution du syst\`eme d'\'equations de Navier-Stokes
tridimensionnelles monophasiques, \`a une pression, instationnaires, en
incompressible ou faiblement dilatable, bas\'ees sur une discr\'etisation
temporelle de type Euler implicite d'ordre 1 ou Crank-Nicolson d'ordre 2 et sur
une discr\'etisation spatiale  par volumes finis colocalis\'ee.


%%%%%%%%%%%%%%%%%%%%%%%%%%%%%%%%%%
%%%%%%%%%%%%%%%%%%%%%%%%%%%%%%%%%%
\section{Fonction}
%%%%%%%%%%%%%%%%%%%%%%%%%%%%%%%%%%
%%%%%%%%%%%%%%%%%%%%%%%%%%%%%%%%%%

  Dans ce sous-programme sont calcul\'ees, \`a un pas de temps donn\'e, les
variables vitesse et pression de ce probl\`eme en proc\'edant en
deux  \'etapes issues d'une d\'ecomposition des op\'erateurs (m\'ethode \`a
pas fractionnaires).\\
Les variables sont donc suppos\'ees connues \`a
l'instant ${t^n}$ et on cherche \`a les d\'eterminer \`a l'instant\footnote{La pression est suppos�e connue � l'instant $t^{n-1+\theta}$ et recherch�e en $t^{n+\theta}$, avec $\theta=1$ ou $1/2$ suivant le sch�ma en temps consid�r�.} ${t^{n+1}}$. Soit ${\Delta {t^n} ={t^{n+1}- {t^n}}}$ le pas de temps associ\'e. Dans un premier temps, on r\'ealise l'\'etape de
pr\'ediction de la vitesse en r\'esolvant l'\'equation de quantit\'e de
mouvement avec une pression explicite. Suit l'\'etape de correction de la
pression (ou projection de la vitesse) qui permet d'obtenir un champ de vitesse \`a divergence nulle.\\\\
Les \'equations en continu sont donc :
\begin{equation}
\left\{\begin{array}{l}
\displaystyle\frac{\partial}{\partial t}(\rho \vect{u}) + \dive(\rho\, \vect{u} \otimes \vect{u})
=\dive(\tens{\sigma}) + \vect{TS} - \tens{K}\,\vect{u}\\
\dive(\rho \vect{u}) = \Gamma
\end{array}\right.
\end{equation}

%(plus tard $\frac{\partial \rho}{\partial t} + \dive(\rho \vect{u}) = \Gamma$)



avec $\rho$ la masse volumique, $\vect{u}$ le champ de vitesse,
$[\,\vect{TS}-\tens{K}\,\vect{u}\,]$ les autres termes sources ($\tens{K}$~est un
tenseur diagonal positif par d\'efinition), $\tens{\sigma}$ le tenseur
de contraintes, $\tens{\tau}$ le tenseur des contraintes visqueuses, $\mu$ la
viscosit\'e dynamique (mol\'eculaire et \'eventuellement turbulente), $\kappa$
la viscosit� de
volume (usuellement nulle et n�glig�e dans le code et dans la suite du document,
sauf en compressible),
$\tens{D}$ le tenseur taux de d\'eformation\footnote{\`A ne pas confondre, malgr\'e la m\^eme notation $D$,
avec les flux diffusifs $\vect{D}_{\,ij}$ et $\vect{D}_{\,{b}_{ik}}$ d\'ecrits par la suite dans ce
sous-programme.}, $\Gamma$ le terme source de masse.
\begin{equation}
\left\{\begin{array}{l}
\tens{\sigma} = \tens{\tau} - P\tens{Id}  \\
\tens{\tau} = 2\,\mu\ \tens{D} +\ (\kappa\ - \frac{2}{3}\mu)\  tr({\tens{D}})\
\tens{Id}  \\
\tens{D} = \frac{1}{2}(\ggrad\vect{u}+\,^{t}\ggrad\vect{u})
\end{array}\right.
\end{equation}
 \\

On rappelle la d\'efinition des notations employ\'ees\footnote{en
utilisant la convention de sommation d'Einstein.}~:
\begin{equation}\notag
\left\{\begin{array}{lll}
\left[\ggrad{\vect{a}}\right]_{ij} &=& \partial_j a_i\\
\left[\dive(\tens{\sigma})\right]_i &=& \partial_j \sigma_{ij}\\
\left[\vect{a}\otimes\vect{b}\right]_{ij} &= &
a_i\,b_j\\
\end{array}\right.
\end{equation}
et donc :
\begin{equation}\notag
\begin{array}{lll}
\left[\dive(\vect{a}\otimes\vect{b})\right]_i &= &
\partial_j (a_i\,b_j)
\end{array}
\end{equation}

\minititre{Remarque}
Dans le cas de la prise en compte d'une masse volumique variable, l'�quation de continuit� s'�crit :
$$\frac{\partial \rho}{\partial t} + \dive{\,(\rho\,\vect{u})} = \Gamma  $$
Cette �quation n'est pas prise en compte dans l'�tape de projection (on continue � r�soudre
seulement
$\displaystyle \dive(\,{\rho\,\vect{u}}) = \Gamma$) alors que le terme
$\displaystyle \frac{\partial \rho}{\partial t}$ appara\^{\i}t lors de l'�tape de pr\'ediction de la vitesse
dans le sous-programme \fort{preduv}. Si ce terme joue un r�le sensible, l'algorithme compressible
de \CS\ (qui r�sout l'�quation compl�te) est alors sans doute plus adapt�.

%                      Code_Saturne version 1.3
%                      ------------------------
%
%     This file is part of the Code_Saturne Kernel, element of the
%     Code_Saturne CFD tool.
% 
%     Copyright (C) 1998-2007 EDF S.A., France
%
%     contact: saturne-support@edf.fr
% 
%     The Code_Saturne Kernel is free software; you can redistribute it
%     and/or modify it under the terms of the GNU General Public License
%     as published by the Free Software Foundation; either version 2 of
%     the License, or (at your option) any later version.
% 
%     The Code_Saturne Kernel is distributed in the hope that it will be
%     useful, but WITHOUT ANY WARRANTY; without even the implied warranty
%     of MERCHANTABILITY or FITNESS FOR A PARTICULAR PURPOSE.  See the
%     GNU General Public License for more details.
% 
%     You should have received a copy of the GNU General Public License
%     along with the Code_Saturne Kernel; if not, write to the
%     Free Software Foundation, Inc.,
%     51 Franklin St, Fifth Floor,
%     Boston, MA  02110-1301  USA
%
%-----------------------------------------------------------------------
%
%%%%%%%%%%%%%%%%%%%%%%%%%%%%%%%%%
%%%%%%%%%%%%%%%%%%%%%%%%%%%%%%%%%%
\section{Discr\'etisation}
%%%%%%%%%%%%%%%%%%%%%%%%%%%%%%%%%%
%%%%%%%%%%%%%%%%%%%%%%%%%%%%%%%%%%

Pour utiliser la m�thode, on se place tout d'abord dans un rep�re local d�fini
de mani�re � ce que le plan $(0yz)$, o� sont inject�s les vortex, soit confondu
avec le plan d'entr�e du calcul (voir figure \ref{Base_Vortex_entree}). 

\begin{figure}[h]
\centerline{\includegraphics[height=6cm]{../Base/Vortex/Images/entree.pdf}}
\caption{\label{Base_Vortex_entree} D�finiton des diff�rentes grandeurs dans le rep�re local
correspondant � l'entr�e d'une conduite de section carr�e.} 
\end{figure}

$u$, $v$ et $w$  sont les composantes de la vitesse fluctuante (principale et
transverse) dans ce plan, et
$\displaystyle \omega(y,z) = \frac{\partial w}{\partial y}-\frac{\partial v}{\partial z}$
la vorticit� dans la direction
normale au plan d'entr�e. $\overline{U}(y,z)$ repr�sente ici la vitesse
principale moyenne impos�e par l'utilisateur dans le plan d'entr�e. 

Chaque vortex $p$ va �tre caract�ris� par sa fonction de forme $\xi$ (identique
pour tous les vortex), sa
circulation $\Gamma_p$, son rayon $\sigma_p$ et les coordonn�es $(y_p,z_p)$ du
point $P$ o� est situ� le vortex dans le plan $(0yz)$. 

Pour cela, on suppose que la vorticit� g�n�r�e par un vortex $p$ au point $M$ de
coordonn�e $(y,z)$ s'�crit : 
\begin{equation}\notag
\omega_p(y,z)= \Gamma_p \, \xi_{\sigma_p}(r)
\end{equation}
o� $r = \sqrt{(y-y_p)^2+(z-z_p)^2}$ est la distance s�parant le point $M$ du point $P$.

Dans la m�thode implant�e, la fonction de forme est de type gaussienne modifi�e :
\begin{equation}\notag
\displaystyle
\xi_\sigma (r) = \frac{1}{2\pi \sigma^2} 
\left(2 e^{-\frac{r^2}{2\sigma^2}}-1\right) e^{-\frac{r^2}{2\sigma^2}}
\end{equation}

Le champ de vitesse induit par cette distribution de vorticit� s'obtient par
inversion des deux �quations de poisson suivantes qui sont d�duites de la
condition d'incompressibilit� dans la plan\footnote{\textit{i.e}
$\displaystyle \frac{\partial v}{\partial y}+\frac{\partial w}{\partial w} = 0$} :
\begin{equation}\notag
\begin{array}{lcr}
\displaystyle
\frac{\partial \omega}{\partial y} = \Delta w
&
\text{    et    }
&
\displaystyle
\frac{\partial \omega}{\partial y} = -\Delta v
\\
\end{array}
\end{equation}

Dans le cas g�n�ral, ce syst�me peut �tre int�gr� � l'aide de la formule de Biot et Savart.

Dans le cas d'une distribution de vorticit� de type gaussienne modifi�e, les
composantes de vitesse v�rifient : 
\begin{equation}\notag
\left\{
\begin{array}{c}
\displaystyle
v_p(y,x) = - \frac{1}{2\pi} \frac{(z-z_p)}{r^2}\left(1 -
e^{-\frac{r^2}{2\sigma^2}}\right)\,e^{-\frac{r^2}{2\sigma^2}} 
\\
\displaystyle
w_p(y,z) = \frac{1}{2\pi} \frac{(y-y_p)}{r^2}\left(1 -e^{-\frac{r^2}{2\sigma^2}}
\right)\,e^{-\frac{r^2}{2\sigma^2}} 
\end{array}
\right.
\end{equation}

Ces relations s'�tendent de fa�on imm�diate au cas de $N$ vortex distincts.
Le champ de vitesse induit par la distribution de vorticit� 
\begin{equation}
\omega(y,z) = \sum_{p=1}^N \Gamma_p \, \xi_{\sigma_p}(r)
\end{equation}
vaut au point $M$ :
\begin{equation}\notag
\begin{array}{lcr}
v(x,y) = \sum_{p=1}^N \Gamma_p\, v_p(y,z) 
&
\text{    et    }
&
w(y,z) = \sum_{p=1}^N \Gamma_p\, w_p(y,z)
\\
\label{Base_Vortex_compvit}
\end{array}
\end{equation}
%================================
\subsection{Param�tres physiques}
%================================

%-------------------------------
\subsubsection{Marche en temps}
%-------------------------------
La position initiale de chaque vortex est tir�e de mani�re al�atoire. On calcul
les d�placements successifs de chacun des vortex dans le plan d'entr�e par
int�gration explicite du champ de vitesse lagrangien : 
\begin{equation}\notag
\begin{array}{lcr}
\displaystyle
\frac{dy_p}{dt} = V(y,z)
&
\text{    et    }
&
\displaystyle
\frac{dz_p}{dt} = W(y,z)
\\
\end{array}
\end{equation}
Les vortex sont alors assimil�s � des particules ponctuelles qui sont convect�es
par le champ $(V,W)$. Ce champ peut �tre impos� par des tirages al�atoires ou
bien d�duit de la vitesse induite par les vortex dans le plan d'entr�e. Dans ce
cas $V(x,y) = \overline{V}(y,z) + v (y,z)$ et $W(y,z)= \overline{W}(y,z) +
w(y,z)$ o� $\overline{V}$ et $\overline{W}$ sont les composantes de la vitesse
transverse moyenne qu'impose l'utilisateur � l'aide des fichiers de donn�es. 

%---------------------------------------------------
\subsubsection{Intensit� et dur�e de vie des vortex}
%---------------------------------------------------
Il serait possible, � partir de l'�quation de transport de la vorticit�,
d'obtenir un mod�le d'�volution pour l'intensit� du vecteur tourbillon
$\omega_p$ associ� � chacun des vortex. En pratique, on pr�f�re utiliser un
mod�le simplifi� dans lequel la circulation des tourbillons ne d�pend que de la
postion de ces derniers � l'instant consid�r�. La circulation initiale de chaque
vortex est alors obtenue � partir de la relation : 
\begin{equation}\notag
|\Gamma_p| = 4 \sqrt{\frac{\pi\,S\,k}{3N\,[2ln(3)-3ln(2)]}}
\end{equation}
o� $S$ est la surface du plan d'entr�e, $N$ le nombre de vortex, et $k$
l'�nergie cin�tique turbulente au point o� se trouve le vortex � l'instant
consid�r�. Le signe de $\Gamma_p$ correspond au sens de rotation du vortex et
est tir� al�atoirement. 

Ce param�tre est celui qui contr�le l'intensit� des fluctuations. Sa d�pendance
en $k$ exprime que, plus l'�coulement est turbulent, plus les vortex sont
intenses. La valeur de $k$ est sp�cifi�e par
l'utilisateur. Elle peut �tre constante ou impos�e � partir de profils d'�nergie
cin�tique turbulente en entr�e. 

Pour �viter que des structures trop allong�es ne se d�veloppent au niveau de
l'entr�e, l'utilisateur doit �galement sp�cifier un temps limites $\tau_p$ au
bout duquel le vortex $p$ va �tre d�truit. Ce temps $\tau_p$ peut �tre pris
constant ou estim� au moyen de la relation : 
\begin{equation}\notag
\tau_p = \frac{5 C_{\mu}k^{\frac{3}{2}}}{\varepsilon\,\overline{U}}
\end{equation}

$\overline{U}$ et $\varepsilon$ repr�sentent respectivement la vitesse moyenne
principale et la dissipation turbulente au point o� est initialement g�n�r� le
vortex ($C_{\mu}=0,09$). 
\\
Lorsque le temps �coul� depuis la cr�ation du vortex $p$ est sup�rieur �
$\tau_p$, le vortex est d�truit et un nouveau vortex g�n�r� (sa position et le
signe de $\Gamma_p$ sont tir�s de fa�on al�atoire). 

%-------------------------------- 
\subsubsection{Taille des vortex}
%--------------------------------
La taille des vortex peut �tre prise constante, ou calcul�e � partir des
relations :
\begin{equation}\notag
\begin{array}{ccc}
\displaystyle
\sigma = \frac{C_{\mu}^{\frac{3}{4}}k^{\frac{3}{2}}}{\varepsilon} 
& \text{    ou    } &
\sigma = max[L_t,L_k]
\\
\end{array}
\end{equation}
avec:
\begin{equation}\notag
\begin{array}{ccc}
\displaystyle
L_t = \sqrt{\left( \frac{5 \nu k}{\varepsilon} \right)} 
& \text{    et    } & 
\displaystyle
L_k = 200\, \left(\frac{\nu^3}{\varepsilon}\right)^{\frac{1}{4}}
\end{array}
\end{equation}
o� $\nu$, $k$ et $\varepsilon$ sont la viscosit� dynamique, l'�nergie cin�tique
turbulente et la dissipation turbulente au point o� se trouve le vortex. 

Dans tous les cas, la taille du vortex doit �tre sup�rieure � la taille des
mailles en entr�e afin que le vortex soit effectivement simul�. 

%==================================
\subsection{Conditions aux limites}
%==================================
Le champ de vitesse g�n�r� � l'aide de la relation \ref{Base_Vortex_compvit} ne tient pas
compte {\em a priori} des conditions aux limites appliqu�es sur les bords du plan
d'entr�e. Pour obtenir des valeurs de la vitesse qui soient coh�rentes sur les
fronti�res du domaine d'entr�e, des ``vortex images'', pseudo-vortex situ�s en
dehors du domaine d'entr�e, sont g�n�r�s � des positions particuli�res et leur
champ de vitesse associ� est superpos� au champ pr�c�demment calcul�.\\
Seuls les cas d'une conduite rectangulaire et d'une conduite circulaire
permettent la g�n�ration de ces pseudo-vortex.
On distingue pour cela trois types de conditions aux limites. 

\begin{figure}[h]
\centerline{\includegraphics[height=6cm]{../Base/Vortex/Images/condlimite.pdf}}
\caption{\label{Base_Vortex_condli} Principe de g�n�ration des ``vortex images'' suivant le
type de conditions aux limites dans une conduite carr�e.} 
\end{figure}

%----------------------------------
\subsubsection{Condition de paroi}
%----------------------------------
On cr�e, pour chaque vortex $P$ contenu dans le plan d'entr�e, un vortex image
$P'$ identique � $P$ (\textit{i.e} de m�me caract�ristiques) et sym�trique de $P$
par rapport au
point $J$ ($J$ �tant la projection orthogonalement � la paroi du point $M$
correspondant au centre de la face o� l'on cherche � calculer la vitesse). La
figure \ref{Base_Vortex_condli} illustre la technique dans le cas d'une conduite
carr�e. Dans ce cas les coordonn�es du vortex situ� en $P'$ v�rifient
$(y_{p'}+y_{p})/2 = y_{J}$ et $(z_{p'}+ z_{p})/2 = z_{J}$. Le champ de vitesse
per�u depuis le point $M$ au niveau du point $J$ est nul, ce qui est bien
l'effet recherch�. 

%------------------------------------
\subsubsection{Condition de sym�trie}
%-------------------------------------
La technique est identique � celle utilis�e pour les conditions de paroi, mais
seule la composante pour la vitesse normale au bord est modifi�e dans ce cas. 

%---------------------------------------
\subsubsection{Condition de p�riodicit�}
%---------------------------------------
On cr�e pour chaque vortex, un vortex images $P'$ identique � $P$ mais translat�
d'une quantit� $L$ correspondant � la longueur qui s�pare les deux plans de la
section d'entr�e o� sont appliqu�es les conditions de p�riodicit�. Dans le cas
o� il y a deux directions de p�riodicit�, on cr�e deux vortex image.

%=============================================
\subsection{Composante de vitesse principale}
%=============================================
La m�thode des vortex ne g�n�rant pas de fluctuation $u$ de la vitesse dans la
direction principale, la derni�re composante est calcul�e � partir d'une
�quation de Langevin. Les coefficients de cette �quation sont d�termin�s par
identification des expressions obtenues pour les contraintes de Reynolds en
$R_{ij}-\varepsilon$. Dans le cas d'un �coulement en canal plan, cette technique
conduit � l'�quation : 
\begin{equation}\notag
\displaystyle
\frac{du}{dt} = - \frac{C_1}{2T} u + \left(\frac{2}{3}C_2-1\right)\frac{\partial
U}{\partial y} v + \sqrt{C_0\varepsilon} dW_i 
\end{equation}

avec $\displaystyle T=\frac{k}{\varepsilon}$, $C_1 = 1,8$, $C_2 = 0,6$,
$C_0=\frac{14}{15}$, et $dW_i$ une variable al�toire Gaussienne de variance
$\sqrt{dt}$. 

En pratique, l'�quation de Langevin n'am�liore pas vraiment les r�sultats. Elle
n'est utilis�e que dans le cas de conduites circulaires. 

%                      Code_Saturne version 1.3
%                      ------------------------
%
%     This file is part of the Code_Saturne Kernel, element of the
%     Code_Saturne CFD tool.
%
%     Copyright (C) 1998-2007 EDF S.A., France
%
%     contact: saturne-support@edf.fr
%
%     The Code_Saturne Kernel is free software; you can redistribute it
%     and/or modify it under the terms of the GNU General Public License
%     as published by the Free Software Foundation; either version 2 of
%     the License, or (at your option) any later version.
%
%     The Code_Saturne Kernel is distributed in the hope that it will be
%     useful, but WITHOUT ANY WARRANTY; without even the implied warranty
%     of MERCHANTABILITY or FITNESS FOR A PARTICULAR PURPOSE.  See the
%     GNU General Public License for more details.
%
%     You should have received a copy of the GNU General Public License
%     along with the Code_Saturne Kernel; if not, write to the
%     Free Software Foundation, Inc.,
%     51 Franklin St, Fifth Floor,
%     Boston, MA  02110-1301  USA
%
%-----------------------------------------------------------------------
%

%%%%%%%%%%%%%%%%%%%%%%%%%%%%%%%%%%
%%%%%%%%%%%%%%%%%%%%%%%%%%%%%%%%%%
\section{Mise en \oe uvre}
%%%%%%%%%%%%%%%%%%%%%%%%%%%%%%%%%%
%%%%%%%%%%%%%%%%%%%%%%%%%%%%%%%%%%
Le syst\`eme (\ref{Cfbl_Cfmsvl_eq_densite_finale_cfmsvl}) est r\'esolu par une m\'ethode
d'incr\'ement et r\'esidu en utilisant
une m\'ethode de Jacobi pour inverser le syst\`eme si le terme convectif
est implicite et en utilisant une m\'ethode de gradient conjugu\'e
si le terme convectif est explicite (qui est le cas par d�faut).

Attention, les valeurs du flux de masse $\rho\,\vect{w}\cdot\vect{S}$ et
de la viscosit\'e $\Delta\,t\,c^2\frac{S}{d}$ aux faces de
bord, qui sont calcul\'ees dans \fort{cfmsfl} et \fort{cfmsvs} respectivement,
sont modifi\'ees imm\'ediatement apr\`es l'appel \`a ces sous-programmes.
En effet, il est indispensable que la contribution de bord de
$\left(\rho\,\vect{w}-\Delta\,t\,(c^2)\,\gradv\,\rho\right)\cdot\vect{S}$
repr\'esente exactement $\vect{Q}_{ac}\cdot\vect{S}$.
Pour cela,
\begin{itemize}
\item imm\'ediatement apr\`es l'appel \`a
\fort{cfmsfl}, on remplace la contribution de bord de
$\rho\,\vect{w}\cdot\vect{S}$
par le flux de masse exact, $\vect{Q}_{ac}\cdot\vect{S}$,
d\'etermin\'e \`a partir des conditions aux limites,
\item puis, imm\'ediatement apr\`es l'appel \`a
\fort{cfmsvs}, on annule la viscosit\'e au bord $\Delta\,t\,(c^2)$ pour
\'eliminer la contribution de $-\Delta\,t\,(c^2)\,(\gradv\,\rho)\cdot\vect{S}$
(l'annulation de la viscosit\'e n'est pas probl\'ematique pour la matrice,
puisqu'elle porte sur des incr\'ements).
\end{itemize}

\bigskip

Une fois qu'on a obtenu $\rho^{n+1}$,
on peut actualiser le flux de masse acoustique
aux faces $(\vect{Q}_{ac}^{n+1})_{ij} \cdot \vect{S}_{ij}$,
qui servira pour la convection des autres variables~:
\begin{equation}\label{Cfbl_Cfmsvl_eq_flux_masse_acoustique_cfmsvl}
\displaystyle(\vect{Q}_{ac}^{n+1})_{ij}\cdot\vect{S}_{ij}=
-\left(\Delta t^n (c^2)^n \gradv(\rho^{n+1})\right)_{ij}\cdot\vect{S}_{ij}
+\left(\rho^{n+\frac{1}{2}} \vect{w}^n\right)_{ij}\cdot\vect{S}_{ij}\\
\end{equation}
Ce calcul de flux est r\'ealis\'e par \fort{cfbsc3}.
Si l'on a choisi l'algorithme standard, \'equation~(\ref{Cfbl_Cfmsvl_eq_densite_cfmsvl}),
on compl\`ete le flux dans \fort{cfmsvl} imm\'ediatement apr\`es l'appel
\`a \fort{cfbsc3}.
En effet, dans ce cas,
la convection est explicite ($\rho^{n+\frac{1}{2}}=\rho^{n}$,
obtenu en imposant \var{ICONV(ISCA(IRHO(IPHAS)))=0})
et le sous-programme \fort{cfbsc3},
qui calcule le flux de masse aux faces,
ne prend pas en compte la contribution du terme
$\rho^{n+\frac{1}{2}}\,\vect{w}^n\cdot\vect{S}$. On ajoute donc cette
contribution dans \fort{cfmsvl}, apr\`es l'appel \`a \fort{cfbsc3}.
Au bord, en particulier, c'est bien le flux de masse calcul\'e \`a partir
des conditions aux limites que l'on obtient.

On actualise la pression \`a la fin de l'\'etape, en utilisant la loi d'\'etat~:
\begin{equation}
\displaystyle P_i^{pred}=P(\rho_i^{n+1},\varepsilon_i^{n})
\end{equation}


%%%%%%%%%%%%%%%%%%%%%%%%%%%%%%%%%%
%%%%%%%%%%%%%%%%%%%%%%%%%%%%%%%%%%
\section{Points \`a traiter}
%%%%%%%%%%%%%%%%%%%%%%%%%%%%%%%%%%
%%%%%%%%%%%%%%%%%%%%%%%%%%%%%%%%%%
Le calcul du flux de masse au  bord n'est pas enti\`erement satisfaisant
si la convection est trait\'ee de mani\`ere implicite
(algorithme non standard, non test\'e,
associ\'e \`a l'\'equation~(\ref{Cfbl_Cfmsvl_eq_densite_bis_cfmsvl}),
correspondant au choix $\rho^{n+\frac{1}{2}}=\rho^{n+1}$ et
obtenu en imposant \var{ICONV(ISCA(IRHO(IPHAS)))=1}).
En effet, apr\`es \fort{cfmsfl}, il faut d\'eterminer la vitesse de
convection $\vect{w}^n$ pour qu'apparaisse
$\rho^{n+1} \vect{w}^n\cdot\vect{n}$
au cours de la r\'esolution par \fort{codits}. De ce fait, on doit d\'eduire
une valeur de $\vect{w}^n$ \`a partir de la valeur
du flux de masse. Au bord, en particulier, il faut
donc diviser le flux de masse
issu des conditions aux limites par la valeur de bord de $\rho^{n+1}$.
Or, lorsque des conditions de Neumann sont appliqu\'ees \`a la
masse volumique,
la valeur de $\rho^{n+1}$ au bord n'est pas connue avant la r\'esolution du
syst\`eme. On utilise donc, au lieu de la valeur de bord inconnue de
$\rho^{n+1}$ la valeur de bord prise au pas de temps
pr\'ec\'edent $\rho^{n}$. Cette approximation est susceptible
d'affecter la valeur du flux de masse au bord.

%                      Code_Saturne version 1.3
%                      ------------------------
%
%     This file is part of the Code_Saturne Kernel, element of the
%     Code_Saturne CFD tool.
%
%     Copyright (C) 1998-2007 EDF S.A., France
%
%     contact: saturne-support@edf.fr
%
%     The Code_Saturne Kernel is free software; you can redistribute it
%     and/or modify it under the terms of the GNU General Public License
%     as published by the Free Software Foundation; either version 2 of
%     the License, or (at your option) any later version.
%
%     The Code_Saturne Kernel is distributed in the hope that it will be
%     useful, but WITHOUT ANY WARRANTY; without even the implied warranty
%     of MERCHANTABILITY or FITNESS FOR A PARTICULAR PURPOSE.  See the
%     GNU General Public License for more details.
%
%     You should have received a copy of the GNU General Public License
%     along with the Code_Saturne Kernel; if not, write to the
%     Free Software Foundation, Inc.,
%     51 Franklin St, Fifth Floor,
%     Boston, MA  02110-1301  USA
%
%-----------------------------------------------------------------------
%


\programme{navsto}

\vspace{1cm}
On s'int\'eresse \`a la r\'esolution du syst\`eme d'\'equations de Navier-Stokes
tridimensionnelles monophasiques, \`a une pression, instationnaires, en
incompressible ou faiblement dilatable, bas\'ees sur une discr\'etisation
temporelle de type Euler implicite d'ordre 1 ou Crank-Nicolson d'ordre 2 et sur
une discr\'etisation spatiale  par volumes finis colocalis\'ee.


%%%%%%%%%%%%%%%%%%%%%%%%%%%%%%%%%%
%%%%%%%%%%%%%%%%%%%%%%%%%%%%%%%%%%
\section{Fonction}
%%%%%%%%%%%%%%%%%%%%%%%%%%%%%%%%%%
%%%%%%%%%%%%%%%%%%%%%%%%%%%%%%%%%%

  Dans ce sous-programme sont calcul\'ees, \`a un pas de temps donn\'e, les
variables vitesse et pression de ce probl\`eme en proc\'edant en
deux  \'etapes issues d'une d\'ecomposition des op\'erateurs (m\'ethode \`a
pas fractionnaires).\\
Les variables sont donc suppos\'ees connues \`a
l'instant ${t^n}$ et on cherche \`a les d\'eterminer \`a l'instant\footnote{La pression est suppos�e connue � l'instant $t^{n-1+\theta}$ et recherch�e en $t^{n+\theta}$, avec $\theta=1$ ou $1/2$ suivant le sch�ma en temps consid�r�.} ${t^{n+1}}$. Soit ${\Delta {t^n} ={t^{n+1}- {t^n}}}$ le pas de temps associ\'e. Dans un premier temps, on r\'ealise l'\'etape de
pr\'ediction de la vitesse en r\'esolvant l'\'equation de quantit\'e de
mouvement avec une pression explicite. Suit l'\'etape de correction de la
pression (ou projection de la vitesse) qui permet d'obtenir un champ de vitesse \`a divergence nulle.\\\\
Les \'equations en continu sont donc :
\begin{equation}
\left\{\begin{array}{l}
\displaystyle\frac{\partial}{\partial t}(\rho \vect{u}) + \dive(\rho\, \vect{u} \otimes \vect{u})
=\dive(\tens{\sigma}) + \vect{TS} - \tens{K}\,\vect{u}\\
\dive(\rho \vect{u}) = \Gamma
\end{array}\right.
\end{equation}

%(plus tard $\frac{\partial \rho}{\partial t} + \dive(\rho \vect{u}) = \Gamma$)



avec $\rho$ la masse volumique, $\vect{u}$ le champ de vitesse,
$[\,\vect{TS}-\tens{K}\,\vect{u}\,]$ les autres termes sources ($\tens{K}$~est un
tenseur diagonal positif par d\'efinition), $\tens{\sigma}$ le tenseur
de contraintes, $\tens{\tau}$ le tenseur des contraintes visqueuses, $\mu$ la
viscosit\'e dynamique (mol\'eculaire et \'eventuellement turbulente), $\kappa$
la viscosit� de
volume (usuellement nulle et n�glig�e dans le code et dans la suite du document,
sauf en compressible),
$\tens{D}$ le tenseur taux de d\'eformation\footnote{\`A ne pas confondre, malgr\'e la m\^eme notation $D$,
avec les flux diffusifs $\vect{D}_{\,ij}$ et $\vect{D}_{\,{b}_{ik}}$ d\'ecrits par la suite dans ce
sous-programme.}, $\Gamma$ le terme source de masse.
\begin{equation}
\left\{\begin{array}{l}
\tens{\sigma} = \tens{\tau} - P\tens{Id}  \\
\tens{\tau} = 2\,\mu\ \tens{D} +\ (\kappa\ - \frac{2}{3}\mu)\  tr({\tens{D}})\
\tens{Id}  \\
\tens{D} = \frac{1}{2}(\ggrad\vect{u}+\,^{t}\ggrad\vect{u})
\end{array}\right.
\end{equation}
 \\

On rappelle la d\'efinition des notations employ\'ees\footnote{en
utilisant la convention de sommation d'Einstein.}~:
\begin{equation}\notag
\left\{\begin{array}{lll}
\left[\ggrad{\vect{a}}\right]_{ij} &=& \partial_j a_i\\
\left[\dive(\tens{\sigma})\right]_i &=& \partial_j \sigma_{ij}\\
\left[\vect{a}\otimes\vect{b}\right]_{ij} &= &
a_i\,b_j\\
\end{array}\right.
\end{equation}
et donc :
\begin{equation}\notag
\begin{array}{lll}
\left[\dive(\vect{a}\otimes\vect{b})\right]_i &= &
\partial_j (a_i\,b_j)
\end{array}
\end{equation}

\minititre{Remarque}
Dans le cas de la prise en compte d'une masse volumique variable, l'�quation de continuit� s'�crit :
$$\frac{\partial \rho}{\partial t} + \dive{\,(\rho\,\vect{u})} = \Gamma  $$
Cette �quation n'est pas prise en compte dans l'�tape de projection (on continue � r�soudre
seulement
$\displaystyle \dive(\,{\rho\,\vect{u}}) = \Gamma$) alors que le terme
$\displaystyle \frac{\partial \rho}{\partial t}$ appara\^{\i}t lors de l'�tape de pr\'ediction de la vitesse
dans le sous-programme \fort{preduv}. Si ce terme joue un r�le sensible, l'algorithme compressible
de \CS\ (qui r�sout l'�quation compl�te) est alors sans doute plus adapt�.

%                      Code_Saturne version 1.3
%                      ------------------------
%
%     This file is part of the Code_Saturne Kernel, element of the
%     Code_Saturne CFD tool.
% 
%     Copyright (C) 1998-2007 EDF S.A., France
%
%     contact: saturne-support@edf.fr
% 
%     The Code_Saturne Kernel is free software; you can redistribute it
%     and/or modify it under the terms of the GNU General Public License
%     as published by the Free Software Foundation; either version 2 of
%     the License, or (at your option) any later version.
% 
%     The Code_Saturne Kernel is distributed in the hope that it will be
%     useful, but WITHOUT ANY WARRANTY; without even the implied warranty
%     of MERCHANTABILITY or FITNESS FOR A PARTICULAR PURPOSE.  See the
%     GNU General Public License for more details.
% 
%     You should have received a copy of the GNU General Public License
%     along with the Code_Saturne Kernel; if not, write to the
%     Free Software Foundation, Inc.,
%     51 Franklin St, Fifth Floor,
%     Boston, MA  02110-1301  USA
%
%-----------------------------------------------------------------------
%
%%%%%%%%%%%%%%%%%%%%%%%%%%%%%%%%%
%%%%%%%%%%%%%%%%%%%%%%%%%%%%%%%%%%
\section{Discr\'etisation}
%%%%%%%%%%%%%%%%%%%%%%%%%%%%%%%%%%
%%%%%%%%%%%%%%%%%%%%%%%%%%%%%%%%%%

Pour utiliser la m�thode, on se place tout d'abord dans un rep�re local d�fini
de mani�re � ce que le plan $(0yz)$, o� sont inject�s les vortex, soit confondu
avec le plan d'entr�e du calcul (voir figure \ref{Base_Vortex_entree}). 

\begin{figure}[h]
\centerline{\includegraphics[height=6cm]{../Base/Vortex/Images/entree.pdf}}
\caption{\label{Base_Vortex_entree} D�finiton des diff�rentes grandeurs dans le rep�re local
correspondant � l'entr�e d'une conduite de section carr�e.} 
\end{figure}

$u$, $v$ et $w$  sont les composantes de la vitesse fluctuante (principale et
transverse) dans ce plan, et
$\displaystyle \omega(y,z) = \frac{\partial w}{\partial y}-\frac{\partial v}{\partial z}$
la vorticit� dans la direction
normale au plan d'entr�e. $\overline{U}(y,z)$ repr�sente ici la vitesse
principale moyenne impos�e par l'utilisateur dans le plan d'entr�e. 

Chaque vortex $p$ va �tre caract�ris� par sa fonction de forme $\xi$ (identique
pour tous les vortex), sa
circulation $\Gamma_p$, son rayon $\sigma_p$ et les coordonn�es $(y_p,z_p)$ du
point $P$ o� est situ� le vortex dans le plan $(0yz)$. 

Pour cela, on suppose que la vorticit� g�n�r�e par un vortex $p$ au point $M$ de
coordonn�e $(y,z)$ s'�crit : 
\begin{equation}\notag
\omega_p(y,z)= \Gamma_p \, \xi_{\sigma_p}(r)
\end{equation}
o� $r = \sqrt{(y-y_p)^2+(z-z_p)^2}$ est la distance s�parant le point $M$ du point $P$.

Dans la m�thode implant�e, la fonction de forme est de type gaussienne modifi�e :
\begin{equation}\notag
\displaystyle
\xi_\sigma (r) = \frac{1}{2\pi \sigma^2} 
\left(2 e^{-\frac{r^2}{2\sigma^2}}-1\right) e^{-\frac{r^2}{2\sigma^2}}
\end{equation}

Le champ de vitesse induit par cette distribution de vorticit� s'obtient par
inversion des deux �quations de poisson suivantes qui sont d�duites de la
condition d'incompressibilit� dans la plan\footnote{\textit{i.e}
$\displaystyle \frac{\partial v}{\partial y}+\frac{\partial w}{\partial w} = 0$} :
\begin{equation}\notag
\begin{array}{lcr}
\displaystyle
\frac{\partial \omega}{\partial y} = \Delta w
&
\text{    et    }
&
\displaystyle
\frac{\partial \omega}{\partial y} = -\Delta v
\\
\end{array}
\end{equation}

Dans le cas g�n�ral, ce syst�me peut �tre int�gr� � l'aide de la formule de Biot et Savart.

Dans le cas d'une distribution de vorticit� de type gaussienne modifi�e, les
composantes de vitesse v�rifient : 
\begin{equation}\notag
\left\{
\begin{array}{c}
\displaystyle
v_p(y,x) = - \frac{1}{2\pi} \frac{(z-z_p)}{r^2}\left(1 -
e^{-\frac{r^2}{2\sigma^2}}\right)\,e^{-\frac{r^2}{2\sigma^2}} 
\\
\displaystyle
w_p(y,z) = \frac{1}{2\pi} \frac{(y-y_p)}{r^2}\left(1 -e^{-\frac{r^2}{2\sigma^2}}
\right)\,e^{-\frac{r^2}{2\sigma^2}} 
\end{array}
\right.
\end{equation}

Ces relations s'�tendent de fa�on imm�diate au cas de $N$ vortex distincts.
Le champ de vitesse induit par la distribution de vorticit� 
\begin{equation}
\omega(y,z) = \sum_{p=1}^N \Gamma_p \, \xi_{\sigma_p}(r)
\end{equation}
vaut au point $M$ :
\begin{equation}\notag
\begin{array}{lcr}
v(x,y) = \sum_{p=1}^N \Gamma_p\, v_p(y,z) 
&
\text{    et    }
&
w(y,z) = \sum_{p=1}^N \Gamma_p\, w_p(y,z)
\\
\label{Base_Vortex_compvit}
\end{array}
\end{equation}
%================================
\subsection{Param�tres physiques}
%================================

%-------------------------------
\subsubsection{Marche en temps}
%-------------------------------
La position initiale de chaque vortex est tir�e de mani�re al�atoire. On calcul
les d�placements successifs de chacun des vortex dans le plan d'entr�e par
int�gration explicite du champ de vitesse lagrangien : 
\begin{equation}\notag
\begin{array}{lcr}
\displaystyle
\frac{dy_p}{dt} = V(y,z)
&
\text{    et    }
&
\displaystyle
\frac{dz_p}{dt} = W(y,z)
\\
\end{array}
\end{equation}
Les vortex sont alors assimil�s � des particules ponctuelles qui sont convect�es
par le champ $(V,W)$. Ce champ peut �tre impos� par des tirages al�atoires ou
bien d�duit de la vitesse induite par les vortex dans le plan d'entr�e. Dans ce
cas $V(x,y) = \overline{V}(y,z) + v (y,z)$ et $W(y,z)= \overline{W}(y,z) +
w(y,z)$ o� $\overline{V}$ et $\overline{W}$ sont les composantes de la vitesse
transverse moyenne qu'impose l'utilisateur � l'aide des fichiers de donn�es. 

%---------------------------------------------------
\subsubsection{Intensit� et dur�e de vie des vortex}
%---------------------------------------------------
Il serait possible, � partir de l'�quation de transport de la vorticit�,
d'obtenir un mod�le d'�volution pour l'intensit� du vecteur tourbillon
$\omega_p$ associ� � chacun des vortex. En pratique, on pr�f�re utiliser un
mod�le simplifi� dans lequel la circulation des tourbillons ne d�pend que de la
postion de ces derniers � l'instant consid�r�. La circulation initiale de chaque
vortex est alors obtenue � partir de la relation : 
\begin{equation}\notag
|\Gamma_p| = 4 \sqrt{\frac{\pi\,S\,k}{3N\,[2ln(3)-3ln(2)]}}
\end{equation}
o� $S$ est la surface du plan d'entr�e, $N$ le nombre de vortex, et $k$
l'�nergie cin�tique turbulente au point o� se trouve le vortex � l'instant
consid�r�. Le signe de $\Gamma_p$ correspond au sens de rotation du vortex et
est tir� al�atoirement. 

Ce param�tre est celui qui contr�le l'intensit� des fluctuations. Sa d�pendance
en $k$ exprime que, plus l'�coulement est turbulent, plus les vortex sont
intenses. La valeur de $k$ est sp�cifi�e par
l'utilisateur. Elle peut �tre constante ou impos�e � partir de profils d'�nergie
cin�tique turbulente en entr�e. 

Pour �viter que des structures trop allong�es ne se d�veloppent au niveau de
l'entr�e, l'utilisateur doit �galement sp�cifier un temps limites $\tau_p$ au
bout duquel le vortex $p$ va �tre d�truit. Ce temps $\tau_p$ peut �tre pris
constant ou estim� au moyen de la relation : 
\begin{equation}\notag
\tau_p = \frac{5 C_{\mu}k^{\frac{3}{2}}}{\varepsilon\,\overline{U}}
\end{equation}

$\overline{U}$ et $\varepsilon$ repr�sentent respectivement la vitesse moyenne
principale et la dissipation turbulente au point o� est initialement g�n�r� le
vortex ($C_{\mu}=0,09$). 
\\
Lorsque le temps �coul� depuis la cr�ation du vortex $p$ est sup�rieur �
$\tau_p$, le vortex est d�truit et un nouveau vortex g�n�r� (sa position et le
signe de $\Gamma_p$ sont tir�s de fa�on al�atoire). 

%-------------------------------- 
\subsubsection{Taille des vortex}
%--------------------------------
La taille des vortex peut �tre prise constante, ou calcul�e � partir des
relations :
\begin{equation}\notag
\begin{array}{ccc}
\displaystyle
\sigma = \frac{C_{\mu}^{\frac{3}{4}}k^{\frac{3}{2}}}{\varepsilon} 
& \text{    ou    } &
\sigma = max[L_t,L_k]
\\
\end{array}
\end{equation}
avec:
\begin{equation}\notag
\begin{array}{ccc}
\displaystyle
L_t = \sqrt{\left( \frac{5 \nu k}{\varepsilon} \right)} 
& \text{    et    } & 
\displaystyle
L_k = 200\, \left(\frac{\nu^3}{\varepsilon}\right)^{\frac{1}{4}}
\end{array}
\end{equation}
o� $\nu$, $k$ et $\varepsilon$ sont la viscosit� dynamique, l'�nergie cin�tique
turbulente et la dissipation turbulente au point o� se trouve le vortex. 

Dans tous les cas, la taille du vortex doit �tre sup�rieure � la taille des
mailles en entr�e afin que le vortex soit effectivement simul�. 

%==================================
\subsection{Conditions aux limites}
%==================================
Le champ de vitesse g�n�r� � l'aide de la relation \ref{Base_Vortex_compvit} ne tient pas
compte {\em a priori} des conditions aux limites appliqu�es sur les bords du plan
d'entr�e. Pour obtenir des valeurs de la vitesse qui soient coh�rentes sur les
fronti�res du domaine d'entr�e, des ``vortex images'', pseudo-vortex situ�s en
dehors du domaine d'entr�e, sont g�n�r�s � des positions particuli�res et leur
champ de vitesse associ� est superpos� au champ pr�c�demment calcul�.\\
Seuls les cas d'une conduite rectangulaire et d'une conduite circulaire
permettent la g�n�ration de ces pseudo-vortex.
On distingue pour cela trois types de conditions aux limites. 

\begin{figure}[h]
\centerline{\includegraphics[height=6cm]{../Base/Vortex/Images/condlimite.pdf}}
\caption{\label{Base_Vortex_condli} Principe de g�n�ration des ``vortex images'' suivant le
type de conditions aux limites dans une conduite carr�e.} 
\end{figure}

%----------------------------------
\subsubsection{Condition de paroi}
%----------------------------------
On cr�e, pour chaque vortex $P$ contenu dans le plan d'entr�e, un vortex image
$P'$ identique � $P$ (\textit{i.e} de m�me caract�ristiques) et sym�trique de $P$
par rapport au
point $J$ ($J$ �tant la projection orthogonalement � la paroi du point $M$
correspondant au centre de la face o� l'on cherche � calculer la vitesse). La
figure \ref{Base_Vortex_condli} illustre la technique dans le cas d'une conduite
carr�e. Dans ce cas les coordonn�es du vortex situ� en $P'$ v�rifient
$(y_{p'}+y_{p})/2 = y_{J}$ et $(z_{p'}+ z_{p})/2 = z_{J}$. Le champ de vitesse
per�u depuis le point $M$ au niveau du point $J$ est nul, ce qui est bien
l'effet recherch�. 

%------------------------------------
\subsubsection{Condition de sym�trie}
%-------------------------------------
La technique est identique � celle utilis�e pour les conditions de paroi, mais
seule la composante pour la vitesse normale au bord est modifi�e dans ce cas. 

%---------------------------------------
\subsubsection{Condition de p�riodicit�}
%---------------------------------------
On cr�e pour chaque vortex, un vortex images $P'$ identique � $P$ mais translat�
d'une quantit� $L$ correspondant � la longueur qui s�pare les deux plans de la
section d'entr�e o� sont appliqu�es les conditions de p�riodicit�. Dans le cas
o� il y a deux directions de p�riodicit�, on cr�e deux vortex image.

%=============================================
\subsection{Composante de vitesse principale}
%=============================================
La m�thode des vortex ne g�n�rant pas de fluctuation $u$ de la vitesse dans la
direction principale, la derni�re composante est calcul�e � partir d'une
�quation de Langevin. Les coefficients de cette �quation sont d�termin�s par
identification des expressions obtenues pour les contraintes de Reynolds en
$R_{ij}-\varepsilon$. Dans le cas d'un �coulement en canal plan, cette technique
conduit � l'�quation : 
\begin{equation}\notag
\displaystyle
\frac{du}{dt} = - \frac{C_1}{2T} u + \left(\frac{2}{3}C_2-1\right)\frac{\partial
U}{\partial y} v + \sqrt{C_0\varepsilon} dW_i 
\end{equation}

avec $\displaystyle T=\frac{k}{\varepsilon}$, $C_1 = 1,8$, $C_2 = 0,6$,
$C_0=\frac{14}{15}$, et $dW_i$ une variable al�toire Gaussienne de variance
$\sqrt{dt}$. 

En pratique, l'�quation de Langevin n'am�liore pas vraiment les r�sultats. Elle
n'est utilis�e que dans le cas de conduites circulaires. 

%                      Code_Saturne version 1.3
%                      ------------------------
%
%     This file is part of the Code_Saturne Kernel, element of the
%     Code_Saturne CFD tool.
%
%     Copyright (C) 1998-2007 EDF S.A., France
%
%     contact: saturne-support@edf.fr
%
%     The Code_Saturne Kernel is free software; you can redistribute it
%     and/or modify it under the terms of the GNU General Public License
%     as published by the Free Software Foundation; either version 2 of
%     the License, or (at your option) any later version.
%
%     The Code_Saturne Kernel is distributed in the hope that it will be
%     useful, but WITHOUT ANY WARRANTY; without even the implied warranty
%     of MERCHANTABILITY or FITNESS FOR A PARTICULAR PURPOSE.  See the
%     GNU General Public License for more details.
%
%     You should have received a copy of the GNU General Public License
%     along with the Code_Saturne Kernel; if not, write to the
%     Free Software Foundation, Inc.,
%     51 Franklin St, Fifth Floor,
%     Boston, MA  02110-1301  USA
%
%-----------------------------------------------------------------------
%

%%%%%%%%%%%%%%%%%%%%%%%%%%%%%%%%%%
%%%%%%%%%%%%%%%%%%%%%%%%%%%%%%%%%%
\section{Mise en \oe uvre}
%%%%%%%%%%%%%%%%%%%%%%%%%%%%%%%%%%
%%%%%%%%%%%%%%%%%%%%%%%%%%%%%%%%%%
Le syst\`eme (\ref{Cfbl_Cfmsvl_eq_densite_finale_cfmsvl}) est r\'esolu par une m\'ethode
d'incr\'ement et r\'esidu en utilisant
une m\'ethode de Jacobi pour inverser le syst\`eme si le terme convectif
est implicite et en utilisant une m\'ethode de gradient conjugu\'e
si le terme convectif est explicite (qui est le cas par d�faut).

Attention, les valeurs du flux de masse $\rho\,\vect{w}\cdot\vect{S}$ et
de la viscosit\'e $\Delta\,t\,c^2\frac{S}{d}$ aux faces de
bord, qui sont calcul\'ees dans \fort{cfmsfl} et \fort{cfmsvs} respectivement,
sont modifi\'ees imm\'ediatement apr\`es l'appel \`a ces sous-programmes.
En effet, il est indispensable que la contribution de bord de
$\left(\rho\,\vect{w}-\Delta\,t\,(c^2)\,\gradv\,\rho\right)\cdot\vect{S}$
repr\'esente exactement $\vect{Q}_{ac}\cdot\vect{S}$.
Pour cela,
\begin{itemize}
\item imm\'ediatement apr\`es l'appel \`a
\fort{cfmsfl}, on remplace la contribution de bord de
$\rho\,\vect{w}\cdot\vect{S}$
par le flux de masse exact, $\vect{Q}_{ac}\cdot\vect{S}$,
d\'etermin\'e \`a partir des conditions aux limites,
\item puis, imm\'ediatement apr\`es l'appel \`a
\fort{cfmsvs}, on annule la viscosit\'e au bord $\Delta\,t\,(c^2)$ pour
\'eliminer la contribution de $-\Delta\,t\,(c^2)\,(\gradv\,\rho)\cdot\vect{S}$
(l'annulation de la viscosit\'e n'est pas probl\'ematique pour la matrice,
puisqu'elle porte sur des incr\'ements).
\end{itemize}

\bigskip

Une fois qu'on a obtenu $\rho^{n+1}$,
on peut actualiser le flux de masse acoustique
aux faces $(\vect{Q}_{ac}^{n+1})_{ij} \cdot \vect{S}_{ij}$,
qui servira pour la convection des autres variables~:
\begin{equation}\label{Cfbl_Cfmsvl_eq_flux_masse_acoustique_cfmsvl}
\displaystyle(\vect{Q}_{ac}^{n+1})_{ij}\cdot\vect{S}_{ij}=
-\left(\Delta t^n (c^2)^n \gradv(\rho^{n+1})\right)_{ij}\cdot\vect{S}_{ij}
+\left(\rho^{n+\frac{1}{2}} \vect{w}^n\right)_{ij}\cdot\vect{S}_{ij}\\
\end{equation}
Ce calcul de flux est r\'ealis\'e par \fort{cfbsc3}.
Si l'on a choisi l'algorithme standard, \'equation~(\ref{Cfbl_Cfmsvl_eq_densite_cfmsvl}),
on compl\`ete le flux dans \fort{cfmsvl} imm\'ediatement apr\`es l'appel
\`a \fort{cfbsc3}.
En effet, dans ce cas,
la convection est explicite ($\rho^{n+\frac{1}{2}}=\rho^{n}$,
obtenu en imposant \var{ICONV(ISCA(IRHO(IPHAS)))=0})
et le sous-programme \fort{cfbsc3},
qui calcule le flux de masse aux faces,
ne prend pas en compte la contribution du terme
$\rho^{n+\frac{1}{2}}\,\vect{w}^n\cdot\vect{S}$. On ajoute donc cette
contribution dans \fort{cfmsvl}, apr\`es l'appel \`a \fort{cfbsc3}.
Au bord, en particulier, c'est bien le flux de masse calcul\'e \`a partir
des conditions aux limites que l'on obtient.

On actualise la pression \`a la fin de l'\'etape, en utilisant la loi d'\'etat~:
\begin{equation}
\displaystyle P_i^{pred}=P(\rho_i^{n+1},\varepsilon_i^{n})
\end{equation}


%%%%%%%%%%%%%%%%%%%%%%%%%%%%%%%%%%
%%%%%%%%%%%%%%%%%%%%%%%%%%%%%%%%%%
\section{Points \`a traiter}
%%%%%%%%%%%%%%%%%%%%%%%%%%%%%%%%%%
%%%%%%%%%%%%%%%%%%%%%%%%%%%%%%%%%%
Le calcul du flux de masse au  bord n'est pas enti\`erement satisfaisant
si la convection est trait\'ee de mani\`ere implicite
(algorithme non standard, non test\'e,
associ\'e \`a l'\'equation~(\ref{Cfbl_Cfmsvl_eq_densite_bis_cfmsvl}),
correspondant au choix $\rho^{n+\frac{1}{2}}=\rho^{n+1}$ et
obtenu en imposant \var{ICONV(ISCA(IRHO(IPHAS)))=1}).
En effet, apr\`es \fort{cfmsfl}, il faut d\'eterminer la vitesse de
convection $\vect{w}^n$ pour qu'apparaisse
$\rho^{n+1} \vect{w}^n\cdot\vect{n}$
au cours de la r\'esolution par \fort{codits}. De ce fait, on doit d\'eduire
une valeur de $\vect{w}^n$ \`a partir de la valeur
du flux de masse. Au bord, en particulier, il faut
donc diviser le flux de masse
issu des conditions aux limites par la valeur de bord de $\rho^{n+1}$.
Or, lorsque des conditions de Neumann sont appliqu\'ees \`a la
masse volumique,
la valeur de $\rho^{n+1}$ au bord n'est pas connue avant la r\'esolution du
syst\`eme. On utilise donc, au lieu de la valeur de bord inconnue de
$\rho^{n+1}$ la valeur de bord prise au pas de temps
pr\'ec\'edent $\rho^{n}$. Cette approximation est susceptible
d'affecter la valeur du flux de masse au bord.

%                      Code_Saturne version 1.3
%                      ------------------------
%
%     This file is part of the Code_Saturne Kernel, element of the
%     Code_Saturne CFD tool.
%
%     Copyright (C) 1998-2007 EDF S.A., France
%
%     contact: saturne-support@edf.fr
%
%     The Code_Saturne Kernel is free software; you can redistribute it
%     and/or modify it under the terms of the GNU General Public License
%     as published by the Free Software Foundation; either version 2 of
%     the License, or (at your option) any later version.
%
%     The Code_Saturne Kernel is distributed in the hope that it will be
%     useful, but WITHOUT ANY WARRANTY; without even the implied warranty
%     of MERCHANTABILITY or FITNESS FOR A PARTICULAR PURPOSE.  See the
%     GNU General Public License for more details.
%
%     You should have received a copy of the GNU General Public License
%     along with the Code_Saturne Kernel; if not, write to the
%     Free Software Foundation, Inc.,
%     51 Franklin St, Fifth Floor,
%     Boston, MA  02110-1301  USA
%
%-----------------------------------------------------------------------
%


\programme{navsto}

\vspace{1cm}
On s'int\'eresse \`a la r\'esolution du syst\`eme d'\'equations de Navier-Stokes
tridimensionnelles monophasiques, \`a une pression, instationnaires, en
incompressible ou faiblement dilatable, bas\'ees sur une discr\'etisation
temporelle de type Euler implicite d'ordre 1 ou Crank-Nicolson d'ordre 2 et sur
une discr\'etisation spatiale  par volumes finis colocalis\'ee.


%%%%%%%%%%%%%%%%%%%%%%%%%%%%%%%%%%
%%%%%%%%%%%%%%%%%%%%%%%%%%%%%%%%%%
\section{Fonction}
%%%%%%%%%%%%%%%%%%%%%%%%%%%%%%%%%%
%%%%%%%%%%%%%%%%%%%%%%%%%%%%%%%%%%

  Dans ce sous-programme sont calcul\'ees, \`a un pas de temps donn\'e, les
variables vitesse et pression de ce probl\`eme en proc\'edant en
deux  \'etapes issues d'une d\'ecomposition des op\'erateurs (m\'ethode \`a
pas fractionnaires).\\
Les variables sont donc suppos\'ees connues \`a
l'instant ${t^n}$ et on cherche \`a les d\'eterminer \`a l'instant\footnote{La pression est suppos�e connue � l'instant $t^{n-1+\theta}$ et recherch�e en $t^{n+\theta}$, avec $\theta=1$ ou $1/2$ suivant le sch�ma en temps consid�r�.} ${t^{n+1}}$. Soit ${\Delta {t^n} ={t^{n+1}- {t^n}}}$ le pas de temps associ\'e. Dans un premier temps, on r\'ealise l'\'etape de
pr\'ediction de la vitesse en r\'esolvant l'\'equation de quantit\'e de
mouvement avec une pression explicite. Suit l'\'etape de correction de la
pression (ou projection de la vitesse) qui permet d'obtenir un champ de vitesse \`a divergence nulle.\\\\
Les \'equations en continu sont donc :
\begin{equation}
\left\{\begin{array}{l}
\displaystyle\frac{\partial}{\partial t}(\rho \vect{u}) + \dive(\rho\, \vect{u} \otimes \vect{u})
=\dive(\tens{\sigma}) + \vect{TS} - \tens{K}\,\vect{u}\\
\dive(\rho \vect{u}) = \Gamma
\end{array}\right.
\end{equation}

%(plus tard $\frac{\partial \rho}{\partial t} + \dive(\rho \vect{u}) = \Gamma$)



avec $\rho$ la masse volumique, $\vect{u}$ le champ de vitesse,
$[\,\vect{TS}-\tens{K}\,\vect{u}\,]$ les autres termes sources ($\tens{K}$~est un
tenseur diagonal positif par d\'efinition), $\tens{\sigma}$ le tenseur
de contraintes, $\tens{\tau}$ le tenseur des contraintes visqueuses, $\mu$ la
viscosit\'e dynamique (mol\'eculaire et \'eventuellement turbulente), $\kappa$
la viscosit� de
volume (usuellement nulle et n�glig�e dans le code et dans la suite du document,
sauf en compressible),
$\tens{D}$ le tenseur taux de d\'eformation\footnote{\`A ne pas confondre, malgr\'e la m\^eme notation $D$,
avec les flux diffusifs $\vect{D}_{\,ij}$ et $\vect{D}_{\,{b}_{ik}}$ d\'ecrits par la suite dans ce
sous-programme.}, $\Gamma$ le terme source de masse.
\begin{equation}
\left\{\begin{array}{l}
\tens{\sigma} = \tens{\tau} - P\tens{Id}  \\
\tens{\tau} = 2\,\mu\ \tens{D} +\ (\kappa\ - \frac{2}{3}\mu)\  tr({\tens{D}})\
\tens{Id}  \\
\tens{D} = \frac{1}{2}(\ggrad\vect{u}+\,^{t}\ggrad\vect{u})
\end{array}\right.
\end{equation}
 \\

On rappelle la d\'efinition des notations employ\'ees\footnote{en
utilisant la convention de sommation d'Einstein.}~:
\begin{equation}\notag
\left\{\begin{array}{lll}
\left[\ggrad{\vect{a}}\right]_{ij} &=& \partial_j a_i\\
\left[\dive(\tens{\sigma})\right]_i &=& \partial_j \sigma_{ij}\\
\left[\vect{a}\otimes\vect{b}\right]_{ij} &= &
a_i\,b_j\\
\end{array}\right.
\end{equation}
et donc :
\begin{equation}\notag
\begin{array}{lll}
\left[\dive(\vect{a}\otimes\vect{b})\right]_i &= &
\partial_j (a_i\,b_j)
\end{array}
\end{equation}

\minititre{Remarque}
Dans le cas de la prise en compte d'une masse volumique variable, l'�quation de continuit� s'�crit :
$$\frac{\partial \rho}{\partial t} + \dive{\,(\rho\,\vect{u})} = \Gamma  $$
Cette �quation n'est pas prise en compte dans l'�tape de projection (on continue � r�soudre
seulement
$\displaystyle \dive(\,{\rho\,\vect{u}}) = \Gamma$) alors que le terme
$\displaystyle \frac{\partial \rho}{\partial t}$ appara\^{\i}t lors de l'�tape de pr\'ediction de la vitesse
dans le sous-programme \fort{preduv}. Si ce terme joue un r�le sensible, l'algorithme compressible
de \CS\ (qui r�sout l'�quation compl�te) est alors sans doute plus adapt�.

%                      Code_Saturne version 1.3
%                      ------------------------
%
%     This file is part of the Code_Saturne Kernel, element of the
%     Code_Saturne CFD tool.
% 
%     Copyright (C) 1998-2007 EDF S.A., France
%
%     contact: saturne-support@edf.fr
% 
%     The Code_Saturne Kernel is free software; you can redistribute it
%     and/or modify it under the terms of the GNU General Public License
%     as published by the Free Software Foundation; either version 2 of
%     the License, or (at your option) any later version.
% 
%     The Code_Saturne Kernel is distributed in the hope that it will be
%     useful, but WITHOUT ANY WARRANTY; without even the implied warranty
%     of MERCHANTABILITY or FITNESS FOR A PARTICULAR PURPOSE.  See the
%     GNU General Public License for more details.
% 
%     You should have received a copy of the GNU General Public License
%     along with the Code_Saturne Kernel; if not, write to the
%     Free Software Foundation, Inc.,
%     51 Franklin St, Fifth Floor,
%     Boston, MA  02110-1301  USA
%
%-----------------------------------------------------------------------
%
%%%%%%%%%%%%%%%%%%%%%%%%%%%%%%%%%
%%%%%%%%%%%%%%%%%%%%%%%%%%%%%%%%%%
\section{Discr\'etisation}
%%%%%%%%%%%%%%%%%%%%%%%%%%%%%%%%%%
%%%%%%%%%%%%%%%%%%%%%%%%%%%%%%%%%%

Pour utiliser la m�thode, on se place tout d'abord dans un rep�re local d�fini
de mani�re � ce que le plan $(0yz)$, o� sont inject�s les vortex, soit confondu
avec le plan d'entr�e du calcul (voir figure \ref{Base_Vortex_entree}). 

\begin{figure}[h]
\centerline{\includegraphics[height=6cm]{../Base/Vortex/Images/entree.pdf}}
\caption{\label{Base_Vortex_entree} D�finiton des diff�rentes grandeurs dans le rep�re local
correspondant � l'entr�e d'une conduite de section carr�e.} 
\end{figure}

$u$, $v$ et $w$  sont les composantes de la vitesse fluctuante (principale et
transverse) dans ce plan, et
$\displaystyle \omega(y,z) = \frac{\partial w}{\partial y}-\frac{\partial v}{\partial z}$
la vorticit� dans la direction
normale au plan d'entr�e. $\overline{U}(y,z)$ repr�sente ici la vitesse
principale moyenne impos�e par l'utilisateur dans le plan d'entr�e. 

Chaque vortex $p$ va �tre caract�ris� par sa fonction de forme $\xi$ (identique
pour tous les vortex), sa
circulation $\Gamma_p$, son rayon $\sigma_p$ et les coordonn�es $(y_p,z_p)$ du
point $P$ o� est situ� le vortex dans le plan $(0yz)$. 

Pour cela, on suppose que la vorticit� g�n�r�e par un vortex $p$ au point $M$ de
coordonn�e $(y,z)$ s'�crit : 
\begin{equation}\notag
\omega_p(y,z)= \Gamma_p \, \xi_{\sigma_p}(r)
\end{equation}
o� $r = \sqrt{(y-y_p)^2+(z-z_p)^2}$ est la distance s�parant le point $M$ du point $P$.

Dans la m�thode implant�e, la fonction de forme est de type gaussienne modifi�e :
\begin{equation}\notag
\displaystyle
\xi_\sigma (r) = \frac{1}{2\pi \sigma^2} 
\left(2 e^{-\frac{r^2}{2\sigma^2}}-1\right) e^{-\frac{r^2}{2\sigma^2}}
\end{equation}

Le champ de vitesse induit par cette distribution de vorticit� s'obtient par
inversion des deux �quations de poisson suivantes qui sont d�duites de la
condition d'incompressibilit� dans la plan\footnote{\textit{i.e}
$\displaystyle \frac{\partial v}{\partial y}+\frac{\partial w}{\partial w} = 0$} :
\begin{equation}\notag
\begin{array}{lcr}
\displaystyle
\frac{\partial \omega}{\partial y} = \Delta w
&
\text{    et    }
&
\displaystyle
\frac{\partial \omega}{\partial y} = -\Delta v
\\
\end{array}
\end{equation}

Dans le cas g�n�ral, ce syst�me peut �tre int�gr� � l'aide de la formule de Biot et Savart.

Dans le cas d'une distribution de vorticit� de type gaussienne modifi�e, les
composantes de vitesse v�rifient : 
\begin{equation}\notag
\left\{
\begin{array}{c}
\displaystyle
v_p(y,x) = - \frac{1}{2\pi} \frac{(z-z_p)}{r^2}\left(1 -
e^{-\frac{r^2}{2\sigma^2}}\right)\,e^{-\frac{r^2}{2\sigma^2}} 
\\
\displaystyle
w_p(y,z) = \frac{1}{2\pi} \frac{(y-y_p)}{r^2}\left(1 -e^{-\frac{r^2}{2\sigma^2}}
\right)\,e^{-\frac{r^2}{2\sigma^2}} 
\end{array}
\right.
\end{equation}

Ces relations s'�tendent de fa�on imm�diate au cas de $N$ vortex distincts.
Le champ de vitesse induit par la distribution de vorticit� 
\begin{equation}
\omega(y,z) = \sum_{p=1}^N \Gamma_p \, \xi_{\sigma_p}(r)
\end{equation}
vaut au point $M$ :
\begin{equation}\notag
\begin{array}{lcr}
v(x,y) = \sum_{p=1}^N \Gamma_p\, v_p(y,z) 
&
\text{    et    }
&
w(y,z) = \sum_{p=1}^N \Gamma_p\, w_p(y,z)
\\
\label{Base_Vortex_compvit}
\end{array}
\end{equation}
%================================
\subsection{Param�tres physiques}
%================================

%-------------------------------
\subsubsection{Marche en temps}
%-------------------------------
La position initiale de chaque vortex est tir�e de mani�re al�atoire. On calcul
les d�placements successifs de chacun des vortex dans le plan d'entr�e par
int�gration explicite du champ de vitesse lagrangien : 
\begin{equation}\notag
\begin{array}{lcr}
\displaystyle
\frac{dy_p}{dt} = V(y,z)
&
\text{    et    }
&
\displaystyle
\frac{dz_p}{dt} = W(y,z)
\\
\end{array}
\end{equation}
Les vortex sont alors assimil�s � des particules ponctuelles qui sont convect�es
par le champ $(V,W)$. Ce champ peut �tre impos� par des tirages al�atoires ou
bien d�duit de la vitesse induite par les vortex dans le plan d'entr�e. Dans ce
cas $V(x,y) = \overline{V}(y,z) + v (y,z)$ et $W(y,z)= \overline{W}(y,z) +
w(y,z)$ o� $\overline{V}$ et $\overline{W}$ sont les composantes de la vitesse
transverse moyenne qu'impose l'utilisateur � l'aide des fichiers de donn�es. 

%---------------------------------------------------
\subsubsection{Intensit� et dur�e de vie des vortex}
%---------------------------------------------------
Il serait possible, � partir de l'�quation de transport de la vorticit�,
d'obtenir un mod�le d'�volution pour l'intensit� du vecteur tourbillon
$\omega_p$ associ� � chacun des vortex. En pratique, on pr�f�re utiliser un
mod�le simplifi� dans lequel la circulation des tourbillons ne d�pend que de la
postion de ces derniers � l'instant consid�r�. La circulation initiale de chaque
vortex est alors obtenue � partir de la relation : 
\begin{equation}\notag
|\Gamma_p| = 4 \sqrt{\frac{\pi\,S\,k}{3N\,[2ln(3)-3ln(2)]}}
\end{equation}
o� $S$ est la surface du plan d'entr�e, $N$ le nombre de vortex, et $k$
l'�nergie cin�tique turbulente au point o� se trouve le vortex � l'instant
consid�r�. Le signe de $\Gamma_p$ correspond au sens de rotation du vortex et
est tir� al�atoirement. 

Ce param�tre est celui qui contr�le l'intensit� des fluctuations. Sa d�pendance
en $k$ exprime que, plus l'�coulement est turbulent, plus les vortex sont
intenses. La valeur de $k$ est sp�cifi�e par
l'utilisateur. Elle peut �tre constante ou impos�e � partir de profils d'�nergie
cin�tique turbulente en entr�e. 

Pour �viter que des structures trop allong�es ne se d�veloppent au niveau de
l'entr�e, l'utilisateur doit �galement sp�cifier un temps limites $\tau_p$ au
bout duquel le vortex $p$ va �tre d�truit. Ce temps $\tau_p$ peut �tre pris
constant ou estim� au moyen de la relation : 
\begin{equation}\notag
\tau_p = \frac{5 C_{\mu}k^{\frac{3}{2}}}{\varepsilon\,\overline{U}}
\end{equation}

$\overline{U}$ et $\varepsilon$ repr�sentent respectivement la vitesse moyenne
principale et la dissipation turbulente au point o� est initialement g�n�r� le
vortex ($C_{\mu}=0,09$). 
\\
Lorsque le temps �coul� depuis la cr�ation du vortex $p$ est sup�rieur �
$\tau_p$, le vortex est d�truit et un nouveau vortex g�n�r� (sa position et le
signe de $\Gamma_p$ sont tir�s de fa�on al�atoire). 

%-------------------------------- 
\subsubsection{Taille des vortex}
%--------------------------------
La taille des vortex peut �tre prise constante, ou calcul�e � partir des
relations :
\begin{equation}\notag
\begin{array}{ccc}
\displaystyle
\sigma = \frac{C_{\mu}^{\frac{3}{4}}k^{\frac{3}{2}}}{\varepsilon} 
& \text{    ou    } &
\sigma = max[L_t,L_k]
\\
\end{array}
\end{equation}
avec:
\begin{equation}\notag
\begin{array}{ccc}
\displaystyle
L_t = \sqrt{\left( \frac{5 \nu k}{\varepsilon} \right)} 
& \text{    et    } & 
\displaystyle
L_k = 200\, \left(\frac{\nu^3}{\varepsilon}\right)^{\frac{1}{4}}
\end{array}
\end{equation}
o� $\nu$, $k$ et $\varepsilon$ sont la viscosit� dynamique, l'�nergie cin�tique
turbulente et la dissipation turbulente au point o� se trouve le vortex. 

Dans tous les cas, la taille du vortex doit �tre sup�rieure � la taille des
mailles en entr�e afin que le vortex soit effectivement simul�. 

%==================================
\subsection{Conditions aux limites}
%==================================
Le champ de vitesse g�n�r� � l'aide de la relation \ref{Base_Vortex_compvit} ne tient pas
compte {\em a priori} des conditions aux limites appliqu�es sur les bords du plan
d'entr�e. Pour obtenir des valeurs de la vitesse qui soient coh�rentes sur les
fronti�res du domaine d'entr�e, des ``vortex images'', pseudo-vortex situ�s en
dehors du domaine d'entr�e, sont g�n�r�s � des positions particuli�res et leur
champ de vitesse associ� est superpos� au champ pr�c�demment calcul�.\\
Seuls les cas d'une conduite rectangulaire et d'une conduite circulaire
permettent la g�n�ration de ces pseudo-vortex.
On distingue pour cela trois types de conditions aux limites. 

\begin{figure}[h]
\centerline{\includegraphics[height=6cm]{../Base/Vortex/Images/condlimite.pdf}}
\caption{\label{Base_Vortex_condli} Principe de g�n�ration des ``vortex images'' suivant le
type de conditions aux limites dans une conduite carr�e.} 
\end{figure}

%----------------------------------
\subsubsection{Condition de paroi}
%----------------------------------
On cr�e, pour chaque vortex $P$ contenu dans le plan d'entr�e, un vortex image
$P'$ identique � $P$ (\textit{i.e} de m�me caract�ristiques) et sym�trique de $P$
par rapport au
point $J$ ($J$ �tant la projection orthogonalement � la paroi du point $M$
correspondant au centre de la face o� l'on cherche � calculer la vitesse). La
figure \ref{Base_Vortex_condli} illustre la technique dans le cas d'une conduite
carr�e. Dans ce cas les coordonn�es du vortex situ� en $P'$ v�rifient
$(y_{p'}+y_{p})/2 = y_{J}$ et $(z_{p'}+ z_{p})/2 = z_{J}$. Le champ de vitesse
per�u depuis le point $M$ au niveau du point $J$ est nul, ce qui est bien
l'effet recherch�. 

%------------------------------------
\subsubsection{Condition de sym�trie}
%-------------------------------------
La technique est identique � celle utilis�e pour les conditions de paroi, mais
seule la composante pour la vitesse normale au bord est modifi�e dans ce cas. 

%---------------------------------------
\subsubsection{Condition de p�riodicit�}
%---------------------------------------
On cr�e pour chaque vortex, un vortex images $P'$ identique � $P$ mais translat�
d'une quantit� $L$ correspondant � la longueur qui s�pare les deux plans de la
section d'entr�e o� sont appliqu�es les conditions de p�riodicit�. Dans le cas
o� il y a deux directions de p�riodicit�, on cr�e deux vortex image.

%=============================================
\subsection{Composante de vitesse principale}
%=============================================
La m�thode des vortex ne g�n�rant pas de fluctuation $u$ de la vitesse dans la
direction principale, la derni�re composante est calcul�e � partir d'une
�quation de Langevin. Les coefficients de cette �quation sont d�termin�s par
identification des expressions obtenues pour les contraintes de Reynolds en
$R_{ij}-\varepsilon$. Dans le cas d'un �coulement en canal plan, cette technique
conduit � l'�quation : 
\begin{equation}\notag
\displaystyle
\frac{du}{dt} = - \frac{C_1}{2T} u + \left(\frac{2}{3}C_2-1\right)\frac{\partial
U}{\partial y} v + \sqrt{C_0\varepsilon} dW_i 
\end{equation}

avec $\displaystyle T=\frac{k}{\varepsilon}$, $C_1 = 1,8$, $C_2 = 0,6$,
$C_0=\frac{14}{15}$, et $dW_i$ une variable al�toire Gaussienne de variance
$\sqrt{dt}$. 

En pratique, l'�quation de Langevin n'am�liore pas vraiment les r�sultats. Elle
n'est utilis�e que dans le cas de conduites circulaires. 

%                      Code_Saturne version 1.3
%                      ------------------------
%
%     This file is part of the Code_Saturne Kernel, element of the
%     Code_Saturne CFD tool.
%
%     Copyright (C) 1998-2007 EDF S.A., France
%
%     contact: saturne-support@edf.fr
%
%     The Code_Saturne Kernel is free software; you can redistribute it
%     and/or modify it under the terms of the GNU General Public License
%     as published by the Free Software Foundation; either version 2 of
%     the License, or (at your option) any later version.
%
%     The Code_Saturne Kernel is distributed in the hope that it will be
%     useful, but WITHOUT ANY WARRANTY; without even the implied warranty
%     of MERCHANTABILITY or FITNESS FOR A PARTICULAR PURPOSE.  See the
%     GNU General Public License for more details.
%
%     You should have received a copy of the GNU General Public License
%     along with the Code_Saturne Kernel; if not, write to the
%     Free Software Foundation, Inc.,
%     51 Franklin St, Fifth Floor,
%     Boston, MA  02110-1301  USA
%
%-----------------------------------------------------------------------
%

%%%%%%%%%%%%%%%%%%%%%%%%%%%%%%%%%%
%%%%%%%%%%%%%%%%%%%%%%%%%%%%%%%%%%
\section{Mise en \oe uvre}
%%%%%%%%%%%%%%%%%%%%%%%%%%%%%%%%%%
%%%%%%%%%%%%%%%%%%%%%%%%%%%%%%%%%%
Le syst\`eme (\ref{Cfbl_Cfmsvl_eq_densite_finale_cfmsvl}) est r\'esolu par une m\'ethode
d'incr\'ement et r\'esidu en utilisant
une m\'ethode de Jacobi pour inverser le syst\`eme si le terme convectif
est implicite et en utilisant une m\'ethode de gradient conjugu\'e
si le terme convectif est explicite (qui est le cas par d�faut).

Attention, les valeurs du flux de masse $\rho\,\vect{w}\cdot\vect{S}$ et
de la viscosit\'e $\Delta\,t\,c^2\frac{S}{d}$ aux faces de
bord, qui sont calcul\'ees dans \fort{cfmsfl} et \fort{cfmsvs} respectivement,
sont modifi\'ees imm\'ediatement apr\`es l'appel \`a ces sous-programmes.
En effet, il est indispensable que la contribution de bord de
$\left(\rho\,\vect{w}-\Delta\,t\,(c^2)\,\gradv\,\rho\right)\cdot\vect{S}$
repr\'esente exactement $\vect{Q}_{ac}\cdot\vect{S}$.
Pour cela,
\begin{itemize}
\item imm\'ediatement apr\`es l'appel \`a
\fort{cfmsfl}, on remplace la contribution de bord de
$\rho\,\vect{w}\cdot\vect{S}$
par le flux de masse exact, $\vect{Q}_{ac}\cdot\vect{S}$,
d\'etermin\'e \`a partir des conditions aux limites,
\item puis, imm\'ediatement apr\`es l'appel \`a
\fort{cfmsvs}, on annule la viscosit\'e au bord $\Delta\,t\,(c^2)$ pour
\'eliminer la contribution de $-\Delta\,t\,(c^2)\,(\gradv\,\rho)\cdot\vect{S}$
(l'annulation de la viscosit\'e n'est pas probl\'ematique pour la matrice,
puisqu'elle porte sur des incr\'ements).
\end{itemize}

\bigskip

Une fois qu'on a obtenu $\rho^{n+1}$,
on peut actualiser le flux de masse acoustique
aux faces $(\vect{Q}_{ac}^{n+1})_{ij} \cdot \vect{S}_{ij}$,
qui servira pour la convection des autres variables~:
\begin{equation}\label{Cfbl_Cfmsvl_eq_flux_masse_acoustique_cfmsvl}
\displaystyle(\vect{Q}_{ac}^{n+1})_{ij}\cdot\vect{S}_{ij}=
-\left(\Delta t^n (c^2)^n \gradv(\rho^{n+1})\right)_{ij}\cdot\vect{S}_{ij}
+\left(\rho^{n+\frac{1}{2}} \vect{w}^n\right)_{ij}\cdot\vect{S}_{ij}\\
\end{equation}
Ce calcul de flux est r\'ealis\'e par \fort{cfbsc3}.
Si l'on a choisi l'algorithme standard, \'equation~(\ref{Cfbl_Cfmsvl_eq_densite_cfmsvl}),
on compl\`ete le flux dans \fort{cfmsvl} imm\'ediatement apr\`es l'appel
\`a \fort{cfbsc3}.
En effet, dans ce cas,
la convection est explicite ($\rho^{n+\frac{1}{2}}=\rho^{n}$,
obtenu en imposant \var{ICONV(ISCA(IRHO(IPHAS)))=0})
et le sous-programme \fort{cfbsc3},
qui calcule le flux de masse aux faces,
ne prend pas en compte la contribution du terme
$\rho^{n+\frac{1}{2}}\,\vect{w}^n\cdot\vect{S}$. On ajoute donc cette
contribution dans \fort{cfmsvl}, apr\`es l'appel \`a \fort{cfbsc3}.
Au bord, en particulier, c'est bien le flux de masse calcul\'e \`a partir
des conditions aux limites que l'on obtient.

On actualise la pression \`a la fin de l'\'etape, en utilisant la loi d'\'etat~:
\begin{equation}
\displaystyle P_i^{pred}=P(\rho_i^{n+1},\varepsilon_i^{n})
\end{equation}


%%%%%%%%%%%%%%%%%%%%%%%%%%%%%%%%%%
%%%%%%%%%%%%%%%%%%%%%%%%%%%%%%%%%%
\section{Points \`a traiter}
%%%%%%%%%%%%%%%%%%%%%%%%%%%%%%%%%%
%%%%%%%%%%%%%%%%%%%%%%%%%%%%%%%%%%
Le calcul du flux de masse au  bord n'est pas enti\`erement satisfaisant
si la convection est trait\'ee de mani\`ere implicite
(algorithme non standard, non test\'e,
associ\'e \`a l'\'equation~(\ref{Cfbl_Cfmsvl_eq_densite_bis_cfmsvl}),
correspondant au choix $\rho^{n+\frac{1}{2}}=\rho^{n+1}$ et
obtenu en imposant \var{ICONV(ISCA(IRHO(IPHAS)))=1}).
En effet, apr\`es \fort{cfmsfl}, il faut d\'eterminer la vitesse de
convection $\vect{w}^n$ pour qu'apparaisse
$\rho^{n+1} \vect{w}^n\cdot\vect{n}$
au cours de la r\'esolution par \fort{codits}. De ce fait, on doit d\'eduire
une valeur de $\vect{w}^n$ \`a partir de la valeur
du flux de masse. Au bord, en particulier, il faut
donc diviser le flux de masse
issu des conditions aux limites par la valeur de bord de $\rho^{n+1}$.
Or, lorsque des conditions de Neumann sont appliqu\'ees \`a la
masse volumique,
la valeur de $\rho^{n+1}$ au bord n'est pas connue avant la r\'esolution du
syst\`eme. On utilise donc, au lieu de la valeur de bord inconnue de
$\rho^{n+1}$ la valeur de bord prise au pas de temps
pr\'ec\'edent $\rho^{n}$. Cette approximation est susceptible
d'affecter la valeur du flux de masse au bord.

%                      Code_Saturne version 1.3
%                      ------------------------
%
%     This file is part of the Code_Saturne Kernel, element of the
%     Code_Saturne CFD tool.
%
%     Copyright (C) 1998-2007 EDF S.A., France
%
%     contact: saturne-support@edf.fr
%
%     The Code_Saturne Kernel is free software; you can redistribute it
%     and/or modify it under the terms of the GNU General Public License
%     as published by the Free Software Foundation; either version 2 of
%     the License, or (at your option) any later version.
%
%     The Code_Saturne Kernel is distributed in the hope that it will be
%     useful, but WITHOUT ANY WARRANTY; without even the implied warranty
%     of MERCHANTABILITY or FITNESS FOR A PARTICULAR PURPOSE.  See the
%     GNU General Public License for more details.
%
%     You should have received a copy of the GNU General Public License
%     along with the Code_Saturne Kernel; if not, write to the
%     Free Software Foundation, Inc.,
%     51 Franklin St, Fifth Floor,
%     Boston, MA  02110-1301  USA
%
%-----------------------------------------------------------------------
%


\programme{navsto}

\vspace{1cm}
On s'int\'eresse \`a la r\'esolution du syst\`eme d'\'equations de Navier-Stokes
tridimensionnelles monophasiques, \`a une pression, instationnaires, en
incompressible ou faiblement dilatable, bas\'ees sur une discr\'etisation
temporelle de type Euler implicite d'ordre 1 ou Crank-Nicolson d'ordre 2 et sur
une discr\'etisation spatiale  par volumes finis colocalis\'ee.


%%%%%%%%%%%%%%%%%%%%%%%%%%%%%%%%%%
%%%%%%%%%%%%%%%%%%%%%%%%%%%%%%%%%%
\section{Fonction}
%%%%%%%%%%%%%%%%%%%%%%%%%%%%%%%%%%
%%%%%%%%%%%%%%%%%%%%%%%%%%%%%%%%%%

  Dans ce sous-programme sont calcul\'ees, \`a un pas de temps donn\'e, les
variables vitesse et pression de ce probl\`eme en proc\'edant en
deux  \'etapes issues d'une d\'ecomposition des op\'erateurs (m\'ethode \`a
pas fractionnaires).\\
Les variables sont donc suppos\'ees connues \`a
l'instant ${t^n}$ et on cherche \`a les d\'eterminer \`a l'instant\footnote{La pression est suppos�e connue � l'instant $t^{n-1+\theta}$ et recherch�e en $t^{n+\theta}$, avec $\theta=1$ ou $1/2$ suivant le sch�ma en temps consid�r�.} ${t^{n+1}}$. Soit ${\Delta {t^n} ={t^{n+1}- {t^n}}}$ le pas de temps associ\'e. Dans un premier temps, on r\'ealise l'\'etape de
pr\'ediction de la vitesse en r\'esolvant l'\'equation de quantit\'e de
mouvement avec une pression explicite. Suit l'\'etape de correction de la
pression (ou projection de la vitesse) qui permet d'obtenir un champ de vitesse \`a divergence nulle.\\\\
Les \'equations en continu sont donc :
\begin{equation}
\left\{\begin{array}{l}
\displaystyle\frac{\partial}{\partial t}(\rho \vect{u}) + \dive(\rho\, \vect{u} \otimes \vect{u})
=\dive(\tens{\sigma}) + \vect{TS} - \tens{K}\,\vect{u}\\
\dive(\rho \vect{u}) = \Gamma
\end{array}\right.
\end{equation}

%(plus tard $\frac{\partial \rho}{\partial t} + \dive(\rho \vect{u}) = \Gamma$)



avec $\rho$ la masse volumique, $\vect{u}$ le champ de vitesse,
$[\,\vect{TS}-\tens{K}\,\vect{u}\,]$ les autres termes sources ($\tens{K}$~est un
tenseur diagonal positif par d\'efinition), $\tens{\sigma}$ le tenseur
de contraintes, $\tens{\tau}$ le tenseur des contraintes visqueuses, $\mu$ la
viscosit\'e dynamique (mol\'eculaire et \'eventuellement turbulente), $\kappa$
la viscosit� de
volume (usuellement nulle et n�glig�e dans le code et dans la suite du document,
sauf en compressible),
$\tens{D}$ le tenseur taux de d\'eformation\footnote{\`A ne pas confondre, malgr\'e la m\^eme notation $D$,
avec les flux diffusifs $\vect{D}_{\,ij}$ et $\vect{D}_{\,{b}_{ik}}$ d\'ecrits par la suite dans ce
sous-programme.}, $\Gamma$ le terme source de masse.
\begin{equation}
\left\{\begin{array}{l}
\tens{\sigma} = \tens{\tau} - P\tens{Id}  \\
\tens{\tau} = 2\,\mu\ \tens{D} +\ (\kappa\ - \frac{2}{3}\mu)\  tr({\tens{D}})\
\tens{Id}  \\
\tens{D} = \frac{1}{2}(\ggrad\vect{u}+\,^{t}\ggrad\vect{u})
\end{array}\right.
\end{equation}
 \\

On rappelle la d\'efinition des notations employ\'ees\footnote{en
utilisant la convention de sommation d'Einstein.}~:
\begin{equation}\notag
\left\{\begin{array}{lll}
\left[\ggrad{\vect{a}}\right]_{ij} &=& \partial_j a_i\\
\left[\dive(\tens{\sigma})\right]_i &=& \partial_j \sigma_{ij}\\
\left[\vect{a}\otimes\vect{b}\right]_{ij} &= &
a_i\,b_j\\
\end{array}\right.
\end{equation}
et donc :
\begin{equation}\notag
\begin{array}{lll}
\left[\dive(\vect{a}\otimes\vect{b})\right]_i &= &
\partial_j (a_i\,b_j)
\end{array}
\end{equation}

\minititre{Remarque}
Dans le cas de la prise en compte d'une masse volumique variable, l'�quation de continuit� s'�crit :
$$\frac{\partial \rho}{\partial t} + \dive{\,(\rho\,\vect{u})} = \Gamma  $$
Cette �quation n'est pas prise en compte dans l'�tape de projection (on continue � r�soudre
seulement
$\displaystyle \dive(\,{\rho\,\vect{u}}) = \Gamma$) alors que le terme
$\displaystyle \frac{\partial \rho}{\partial t}$ appara\^{\i}t lors de l'�tape de pr\'ediction de la vitesse
dans le sous-programme \fort{preduv}. Si ce terme joue un r�le sensible, l'algorithme compressible
de \CS\ (qui r�sout l'�quation compl�te) est alors sans doute plus adapt�.

%                      Code_Saturne version 1.3
%                      ------------------------
%
%     This file is part of the Code_Saturne Kernel, element of the
%     Code_Saturne CFD tool.
% 
%     Copyright (C) 1998-2007 EDF S.A., France
%
%     contact: saturne-support@edf.fr
% 
%     The Code_Saturne Kernel is free software; you can redistribute it
%     and/or modify it under the terms of the GNU General Public License
%     as published by the Free Software Foundation; either version 2 of
%     the License, or (at your option) any later version.
% 
%     The Code_Saturne Kernel is distributed in the hope that it will be
%     useful, but WITHOUT ANY WARRANTY; without even the implied warranty
%     of MERCHANTABILITY or FITNESS FOR A PARTICULAR PURPOSE.  See the
%     GNU General Public License for more details.
% 
%     You should have received a copy of the GNU General Public License
%     along with the Code_Saturne Kernel; if not, write to the
%     Free Software Foundation, Inc.,
%     51 Franklin St, Fifth Floor,
%     Boston, MA  02110-1301  USA
%
%-----------------------------------------------------------------------
%
%%%%%%%%%%%%%%%%%%%%%%%%%%%%%%%%%
%%%%%%%%%%%%%%%%%%%%%%%%%%%%%%%%%%
\section{Discr\'etisation}
%%%%%%%%%%%%%%%%%%%%%%%%%%%%%%%%%%
%%%%%%%%%%%%%%%%%%%%%%%%%%%%%%%%%%

Pour utiliser la m�thode, on se place tout d'abord dans un rep�re local d�fini
de mani�re � ce que le plan $(0yz)$, o� sont inject�s les vortex, soit confondu
avec le plan d'entr�e du calcul (voir figure \ref{Base_Vortex_entree}). 

\begin{figure}[h]
\centerline{\includegraphics[height=6cm]{../Base/Vortex/Images/entree.pdf}}
\caption{\label{Base_Vortex_entree} D�finiton des diff�rentes grandeurs dans le rep�re local
correspondant � l'entr�e d'une conduite de section carr�e.} 
\end{figure}

$u$, $v$ et $w$  sont les composantes de la vitesse fluctuante (principale et
transverse) dans ce plan, et
$\displaystyle \omega(y,z) = \frac{\partial w}{\partial y}-\frac{\partial v}{\partial z}$
la vorticit� dans la direction
normale au plan d'entr�e. $\overline{U}(y,z)$ repr�sente ici la vitesse
principale moyenne impos�e par l'utilisateur dans le plan d'entr�e. 

Chaque vortex $p$ va �tre caract�ris� par sa fonction de forme $\xi$ (identique
pour tous les vortex), sa
circulation $\Gamma_p$, son rayon $\sigma_p$ et les coordonn�es $(y_p,z_p)$ du
point $P$ o� est situ� le vortex dans le plan $(0yz)$. 

Pour cela, on suppose que la vorticit� g�n�r�e par un vortex $p$ au point $M$ de
coordonn�e $(y,z)$ s'�crit : 
\begin{equation}\notag
\omega_p(y,z)= \Gamma_p \, \xi_{\sigma_p}(r)
\end{equation}
o� $r = \sqrt{(y-y_p)^2+(z-z_p)^2}$ est la distance s�parant le point $M$ du point $P$.

Dans la m�thode implant�e, la fonction de forme est de type gaussienne modifi�e :
\begin{equation}\notag
\displaystyle
\xi_\sigma (r) = \frac{1}{2\pi \sigma^2} 
\left(2 e^{-\frac{r^2}{2\sigma^2}}-1\right) e^{-\frac{r^2}{2\sigma^2}}
\end{equation}

Le champ de vitesse induit par cette distribution de vorticit� s'obtient par
inversion des deux �quations de poisson suivantes qui sont d�duites de la
condition d'incompressibilit� dans la plan\footnote{\textit{i.e}
$\displaystyle \frac{\partial v}{\partial y}+\frac{\partial w}{\partial w} = 0$} :
\begin{equation}\notag
\begin{array}{lcr}
\displaystyle
\frac{\partial \omega}{\partial y} = \Delta w
&
\text{    et    }
&
\displaystyle
\frac{\partial \omega}{\partial y} = -\Delta v
\\
\end{array}
\end{equation}

Dans le cas g�n�ral, ce syst�me peut �tre int�gr� � l'aide de la formule de Biot et Savart.

Dans le cas d'une distribution de vorticit� de type gaussienne modifi�e, les
composantes de vitesse v�rifient : 
\begin{equation}\notag
\left\{
\begin{array}{c}
\displaystyle
v_p(y,x) = - \frac{1}{2\pi} \frac{(z-z_p)}{r^2}\left(1 -
e^{-\frac{r^2}{2\sigma^2}}\right)\,e^{-\frac{r^2}{2\sigma^2}} 
\\
\displaystyle
w_p(y,z) = \frac{1}{2\pi} \frac{(y-y_p)}{r^2}\left(1 -e^{-\frac{r^2}{2\sigma^2}}
\right)\,e^{-\frac{r^2}{2\sigma^2}} 
\end{array}
\right.
\end{equation}

Ces relations s'�tendent de fa�on imm�diate au cas de $N$ vortex distincts.
Le champ de vitesse induit par la distribution de vorticit� 
\begin{equation}
\omega(y,z) = \sum_{p=1}^N \Gamma_p \, \xi_{\sigma_p}(r)
\end{equation}
vaut au point $M$ :
\begin{equation}\notag
\begin{array}{lcr}
v(x,y) = \sum_{p=1}^N \Gamma_p\, v_p(y,z) 
&
\text{    et    }
&
w(y,z) = \sum_{p=1}^N \Gamma_p\, w_p(y,z)
\\
\label{Base_Vortex_compvit}
\end{array}
\end{equation}
%================================
\subsection{Param�tres physiques}
%================================

%-------------------------------
\subsubsection{Marche en temps}
%-------------------------------
La position initiale de chaque vortex est tir�e de mani�re al�atoire. On calcul
les d�placements successifs de chacun des vortex dans le plan d'entr�e par
int�gration explicite du champ de vitesse lagrangien : 
\begin{equation}\notag
\begin{array}{lcr}
\displaystyle
\frac{dy_p}{dt} = V(y,z)
&
\text{    et    }
&
\displaystyle
\frac{dz_p}{dt} = W(y,z)
\\
\end{array}
\end{equation}
Les vortex sont alors assimil�s � des particules ponctuelles qui sont convect�es
par le champ $(V,W)$. Ce champ peut �tre impos� par des tirages al�atoires ou
bien d�duit de la vitesse induite par les vortex dans le plan d'entr�e. Dans ce
cas $V(x,y) = \overline{V}(y,z) + v (y,z)$ et $W(y,z)= \overline{W}(y,z) +
w(y,z)$ o� $\overline{V}$ et $\overline{W}$ sont les composantes de la vitesse
transverse moyenne qu'impose l'utilisateur � l'aide des fichiers de donn�es. 

%---------------------------------------------------
\subsubsection{Intensit� et dur�e de vie des vortex}
%---------------------------------------------------
Il serait possible, � partir de l'�quation de transport de la vorticit�,
d'obtenir un mod�le d'�volution pour l'intensit� du vecteur tourbillon
$\omega_p$ associ� � chacun des vortex. En pratique, on pr�f�re utiliser un
mod�le simplifi� dans lequel la circulation des tourbillons ne d�pend que de la
postion de ces derniers � l'instant consid�r�. La circulation initiale de chaque
vortex est alors obtenue � partir de la relation : 
\begin{equation}\notag
|\Gamma_p| = 4 \sqrt{\frac{\pi\,S\,k}{3N\,[2ln(3)-3ln(2)]}}
\end{equation}
o� $S$ est la surface du plan d'entr�e, $N$ le nombre de vortex, et $k$
l'�nergie cin�tique turbulente au point o� se trouve le vortex � l'instant
consid�r�. Le signe de $\Gamma_p$ correspond au sens de rotation du vortex et
est tir� al�atoirement. 

Ce param�tre est celui qui contr�le l'intensit� des fluctuations. Sa d�pendance
en $k$ exprime que, plus l'�coulement est turbulent, plus les vortex sont
intenses. La valeur de $k$ est sp�cifi�e par
l'utilisateur. Elle peut �tre constante ou impos�e � partir de profils d'�nergie
cin�tique turbulente en entr�e. 

Pour �viter que des structures trop allong�es ne se d�veloppent au niveau de
l'entr�e, l'utilisateur doit �galement sp�cifier un temps limites $\tau_p$ au
bout duquel le vortex $p$ va �tre d�truit. Ce temps $\tau_p$ peut �tre pris
constant ou estim� au moyen de la relation : 
\begin{equation}\notag
\tau_p = \frac{5 C_{\mu}k^{\frac{3}{2}}}{\varepsilon\,\overline{U}}
\end{equation}

$\overline{U}$ et $\varepsilon$ repr�sentent respectivement la vitesse moyenne
principale et la dissipation turbulente au point o� est initialement g�n�r� le
vortex ($C_{\mu}=0,09$). 
\\
Lorsque le temps �coul� depuis la cr�ation du vortex $p$ est sup�rieur �
$\tau_p$, le vortex est d�truit et un nouveau vortex g�n�r� (sa position et le
signe de $\Gamma_p$ sont tir�s de fa�on al�atoire). 

%-------------------------------- 
\subsubsection{Taille des vortex}
%--------------------------------
La taille des vortex peut �tre prise constante, ou calcul�e � partir des
relations :
\begin{equation}\notag
\begin{array}{ccc}
\displaystyle
\sigma = \frac{C_{\mu}^{\frac{3}{4}}k^{\frac{3}{2}}}{\varepsilon} 
& \text{    ou    } &
\sigma = max[L_t,L_k]
\\
\end{array}
\end{equation}
avec:
\begin{equation}\notag
\begin{array}{ccc}
\displaystyle
L_t = \sqrt{\left( \frac{5 \nu k}{\varepsilon} \right)} 
& \text{    et    } & 
\displaystyle
L_k = 200\, \left(\frac{\nu^3}{\varepsilon}\right)^{\frac{1}{4}}
\end{array}
\end{equation}
o� $\nu$, $k$ et $\varepsilon$ sont la viscosit� dynamique, l'�nergie cin�tique
turbulente et la dissipation turbulente au point o� se trouve le vortex. 

Dans tous les cas, la taille du vortex doit �tre sup�rieure � la taille des
mailles en entr�e afin que le vortex soit effectivement simul�. 

%==================================
\subsection{Conditions aux limites}
%==================================
Le champ de vitesse g�n�r� � l'aide de la relation \ref{Base_Vortex_compvit} ne tient pas
compte {\em a priori} des conditions aux limites appliqu�es sur les bords du plan
d'entr�e. Pour obtenir des valeurs de la vitesse qui soient coh�rentes sur les
fronti�res du domaine d'entr�e, des ``vortex images'', pseudo-vortex situ�s en
dehors du domaine d'entr�e, sont g�n�r�s � des positions particuli�res et leur
champ de vitesse associ� est superpos� au champ pr�c�demment calcul�.\\
Seuls les cas d'une conduite rectangulaire et d'une conduite circulaire
permettent la g�n�ration de ces pseudo-vortex.
On distingue pour cela trois types de conditions aux limites. 

\begin{figure}[h]
\centerline{\includegraphics[height=6cm]{../Base/Vortex/Images/condlimite.pdf}}
\caption{\label{Base_Vortex_condli} Principe de g�n�ration des ``vortex images'' suivant le
type de conditions aux limites dans une conduite carr�e.} 
\end{figure}

%----------------------------------
\subsubsection{Condition de paroi}
%----------------------------------
On cr�e, pour chaque vortex $P$ contenu dans le plan d'entr�e, un vortex image
$P'$ identique � $P$ (\textit{i.e} de m�me caract�ristiques) et sym�trique de $P$
par rapport au
point $J$ ($J$ �tant la projection orthogonalement � la paroi du point $M$
correspondant au centre de la face o� l'on cherche � calculer la vitesse). La
figure \ref{Base_Vortex_condli} illustre la technique dans le cas d'une conduite
carr�e. Dans ce cas les coordonn�es du vortex situ� en $P'$ v�rifient
$(y_{p'}+y_{p})/2 = y_{J}$ et $(z_{p'}+ z_{p})/2 = z_{J}$. Le champ de vitesse
per�u depuis le point $M$ au niveau du point $J$ est nul, ce qui est bien
l'effet recherch�. 

%------------------------------------
\subsubsection{Condition de sym�trie}
%-------------------------------------
La technique est identique � celle utilis�e pour les conditions de paroi, mais
seule la composante pour la vitesse normale au bord est modifi�e dans ce cas. 

%---------------------------------------
\subsubsection{Condition de p�riodicit�}
%---------------------------------------
On cr�e pour chaque vortex, un vortex images $P'$ identique � $P$ mais translat�
d'une quantit� $L$ correspondant � la longueur qui s�pare les deux plans de la
section d'entr�e o� sont appliqu�es les conditions de p�riodicit�. Dans le cas
o� il y a deux directions de p�riodicit�, on cr�e deux vortex image.

%=============================================
\subsection{Composante de vitesse principale}
%=============================================
La m�thode des vortex ne g�n�rant pas de fluctuation $u$ de la vitesse dans la
direction principale, la derni�re composante est calcul�e � partir d'une
�quation de Langevin. Les coefficients de cette �quation sont d�termin�s par
identification des expressions obtenues pour les contraintes de Reynolds en
$R_{ij}-\varepsilon$. Dans le cas d'un �coulement en canal plan, cette technique
conduit � l'�quation : 
\begin{equation}\notag
\displaystyle
\frac{du}{dt} = - \frac{C_1}{2T} u + \left(\frac{2}{3}C_2-1\right)\frac{\partial
U}{\partial y} v + \sqrt{C_0\varepsilon} dW_i 
\end{equation}

avec $\displaystyle T=\frac{k}{\varepsilon}$, $C_1 = 1,8$, $C_2 = 0,6$,
$C_0=\frac{14}{15}$, et $dW_i$ une variable al�toire Gaussienne de variance
$\sqrt{dt}$. 

En pratique, l'�quation de Langevin n'am�liore pas vraiment les r�sultats. Elle
n'est utilis�e que dans le cas de conduites circulaires. 

%                      Code_Saturne version 1.3
%                      ------------------------
%
%     This file is part of the Code_Saturne Kernel, element of the
%     Code_Saturne CFD tool.
%
%     Copyright (C) 1998-2007 EDF S.A., France
%
%     contact: saturne-support@edf.fr
%
%     The Code_Saturne Kernel is free software; you can redistribute it
%     and/or modify it under the terms of the GNU General Public License
%     as published by the Free Software Foundation; either version 2 of
%     the License, or (at your option) any later version.
%
%     The Code_Saturne Kernel is distributed in the hope that it will be
%     useful, but WITHOUT ANY WARRANTY; without even the implied warranty
%     of MERCHANTABILITY or FITNESS FOR A PARTICULAR PURPOSE.  See the
%     GNU General Public License for more details.
%
%     You should have received a copy of the GNU General Public License
%     along with the Code_Saturne Kernel; if not, write to the
%     Free Software Foundation, Inc.,
%     51 Franklin St, Fifth Floor,
%     Boston, MA  02110-1301  USA
%
%-----------------------------------------------------------------------
%

%%%%%%%%%%%%%%%%%%%%%%%%%%%%%%%%%%
%%%%%%%%%%%%%%%%%%%%%%%%%%%%%%%%%%
\section{Mise en \oe uvre}
%%%%%%%%%%%%%%%%%%%%%%%%%%%%%%%%%%
%%%%%%%%%%%%%%%%%%%%%%%%%%%%%%%%%%
Le syst\`eme (\ref{Cfbl_Cfmsvl_eq_densite_finale_cfmsvl}) est r\'esolu par une m\'ethode
d'incr\'ement et r\'esidu en utilisant
une m\'ethode de Jacobi pour inverser le syst\`eme si le terme convectif
est implicite et en utilisant une m\'ethode de gradient conjugu\'e
si le terme convectif est explicite (qui est le cas par d�faut).

Attention, les valeurs du flux de masse $\rho\,\vect{w}\cdot\vect{S}$ et
de la viscosit\'e $\Delta\,t\,c^2\frac{S}{d}$ aux faces de
bord, qui sont calcul\'ees dans \fort{cfmsfl} et \fort{cfmsvs} respectivement,
sont modifi\'ees imm\'ediatement apr\`es l'appel \`a ces sous-programmes.
En effet, il est indispensable que la contribution de bord de
$\left(\rho\,\vect{w}-\Delta\,t\,(c^2)\,\gradv\,\rho\right)\cdot\vect{S}$
repr\'esente exactement $\vect{Q}_{ac}\cdot\vect{S}$.
Pour cela,
\begin{itemize}
\item imm\'ediatement apr\`es l'appel \`a
\fort{cfmsfl}, on remplace la contribution de bord de
$\rho\,\vect{w}\cdot\vect{S}$
par le flux de masse exact, $\vect{Q}_{ac}\cdot\vect{S}$,
d\'etermin\'e \`a partir des conditions aux limites,
\item puis, imm\'ediatement apr\`es l'appel \`a
\fort{cfmsvs}, on annule la viscosit\'e au bord $\Delta\,t\,(c^2)$ pour
\'eliminer la contribution de $-\Delta\,t\,(c^2)\,(\gradv\,\rho)\cdot\vect{S}$
(l'annulation de la viscosit\'e n'est pas probl\'ematique pour la matrice,
puisqu'elle porte sur des incr\'ements).
\end{itemize}

\bigskip

Une fois qu'on a obtenu $\rho^{n+1}$,
on peut actualiser le flux de masse acoustique
aux faces $(\vect{Q}_{ac}^{n+1})_{ij} \cdot \vect{S}_{ij}$,
qui servira pour la convection des autres variables~:
\begin{equation}\label{Cfbl_Cfmsvl_eq_flux_masse_acoustique_cfmsvl}
\displaystyle(\vect{Q}_{ac}^{n+1})_{ij}\cdot\vect{S}_{ij}=
-\left(\Delta t^n (c^2)^n \gradv(\rho^{n+1})\right)_{ij}\cdot\vect{S}_{ij}
+\left(\rho^{n+\frac{1}{2}} \vect{w}^n\right)_{ij}\cdot\vect{S}_{ij}\\
\end{equation}
Ce calcul de flux est r\'ealis\'e par \fort{cfbsc3}.
Si l'on a choisi l'algorithme standard, \'equation~(\ref{Cfbl_Cfmsvl_eq_densite_cfmsvl}),
on compl\`ete le flux dans \fort{cfmsvl} imm\'ediatement apr\`es l'appel
\`a \fort{cfbsc3}.
En effet, dans ce cas,
la convection est explicite ($\rho^{n+\frac{1}{2}}=\rho^{n}$,
obtenu en imposant \var{ICONV(ISCA(IRHO(IPHAS)))=0})
et le sous-programme \fort{cfbsc3},
qui calcule le flux de masse aux faces,
ne prend pas en compte la contribution du terme
$\rho^{n+\frac{1}{2}}\,\vect{w}^n\cdot\vect{S}$. On ajoute donc cette
contribution dans \fort{cfmsvl}, apr\`es l'appel \`a \fort{cfbsc3}.
Au bord, en particulier, c'est bien le flux de masse calcul\'e \`a partir
des conditions aux limites que l'on obtient.

On actualise la pression \`a la fin de l'\'etape, en utilisant la loi d'\'etat~:
\begin{equation}
\displaystyle P_i^{pred}=P(\rho_i^{n+1},\varepsilon_i^{n})
\end{equation}


%%%%%%%%%%%%%%%%%%%%%%%%%%%%%%%%%%
%%%%%%%%%%%%%%%%%%%%%%%%%%%%%%%%%%
\section{Points \`a traiter}
%%%%%%%%%%%%%%%%%%%%%%%%%%%%%%%%%%
%%%%%%%%%%%%%%%%%%%%%%%%%%%%%%%%%%
Le calcul du flux de masse au  bord n'est pas enti\`erement satisfaisant
si la convection est trait\'ee de mani\`ere implicite
(algorithme non standard, non test\'e,
associ\'e \`a l'\'equation~(\ref{Cfbl_Cfmsvl_eq_densite_bis_cfmsvl}),
correspondant au choix $\rho^{n+\frac{1}{2}}=\rho^{n+1}$ et
obtenu en imposant \var{ICONV(ISCA(IRHO(IPHAS)))=1}).
En effet, apr\`es \fort{cfmsfl}, il faut d\'eterminer la vitesse de
convection $\vect{w}^n$ pour qu'apparaisse
$\rho^{n+1} \vect{w}^n\cdot\vect{n}$
au cours de la r\'esolution par \fort{codits}. De ce fait, on doit d\'eduire
une valeur de $\vect{w}^n$ \`a partir de la valeur
du flux de masse. Au bord, en particulier, il faut
donc diviser le flux de masse
issu des conditions aux limites par la valeur de bord de $\rho^{n+1}$.
Or, lorsque des conditions de Neumann sont appliqu\'ees \`a la
masse volumique,
la valeur de $\rho^{n+1}$ au bord n'est pas connue avant la r\'esolution du
syst\`eme. On utilise donc, au lieu de la valeur de bord inconnue de
$\rho^{n+1}$ la valeur de bord prise au pas de temps
pr\'ec\'edent $\rho^{n}$. Cette approximation est susceptible
d'affecter la valeur du flux de masse au bord.

\passepage
\part{Module �lectrique}
%                      Code_Saturne version 1.3
%                      ------------------------
%
%     This file is part of the Code_Saturne Kernel, element of the
%     Code_Saturne CFD tool.
%
%     Copyright (C) 1998-2007 EDF S.A., France
%
%     contact: saturne-support@edf.fr
%
%     The Code_Saturne Kernel is free software; you can redistribute it
%     and/or modify it under the terms of the GNU General Public License
%     as published by the Free Software Foundation; either version 2 of
%     the License, or (at your option) any later version.
%
%     The Code_Saturne Kernel is distributed in the hope that it will be
%     useful, but WITHOUT ANY WARRANTY; without even the implied warranty
%     of MERCHANTABILITY or FITNESS FOR A PARTICULAR PURPOSE.  See the
%     GNU General Public License for more details.
%
%     You should have received a copy of the GNU General Public License
%     along with the Code_Saturne Kernel; if not, write to the
%     Free Software Foundation, Inc.,
%     51 Franklin St, Fifth Floor,
%     Boston, MA  02110-1301  USA
%
%-----------------------------------------------------------------------
%


\programme{navsto}

\vspace{1cm}
On s'int\'eresse \`a la r\'esolution du syst\`eme d'\'equations de Navier-Stokes
tridimensionnelles monophasiques, \`a une pression, instationnaires, en
incompressible ou faiblement dilatable, bas\'ees sur une discr\'etisation
temporelle de type Euler implicite d'ordre 1 ou Crank-Nicolson d'ordre 2 et sur
une discr\'etisation spatiale  par volumes finis colocalis\'ee.


%%%%%%%%%%%%%%%%%%%%%%%%%%%%%%%%%%
%%%%%%%%%%%%%%%%%%%%%%%%%%%%%%%%%%
\section{Fonction}
%%%%%%%%%%%%%%%%%%%%%%%%%%%%%%%%%%
%%%%%%%%%%%%%%%%%%%%%%%%%%%%%%%%%%

  Dans ce sous-programme sont calcul\'ees, \`a un pas de temps donn\'e, les
variables vitesse et pression de ce probl\`eme en proc\'edant en
deux  \'etapes issues d'une d\'ecomposition des op\'erateurs (m\'ethode \`a
pas fractionnaires).\\
Les variables sont donc suppos\'ees connues \`a
l'instant ${t^n}$ et on cherche \`a les d\'eterminer \`a l'instant\footnote{La pression est suppos�e connue � l'instant $t^{n-1+\theta}$ et recherch�e en $t^{n+\theta}$, avec $\theta=1$ ou $1/2$ suivant le sch�ma en temps consid�r�.} ${t^{n+1}}$. Soit ${\Delta {t^n} ={t^{n+1}- {t^n}}}$ le pas de temps associ\'e. Dans un premier temps, on r\'ealise l'\'etape de
pr\'ediction de la vitesse en r\'esolvant l'\'equation de quantit\'e de
mouvement avec une pression explicite. Suit l'\'etape de correction de la
pression (ou projection de la vitesse) qui permet d'obtenir un champ de vitesse \`a divergence nulle.\\\\
Les \'equations en continu sont donc :
\begin{equation}
\left\{\begin{array}{l}
\displaystyle\frac{\partial}{\partial t}(\rho \vect{u}) + \dive(\rho\, \vect{u} \otimes \vect{u})
=\dive(\tens{\sigma}) + \vect{TS} - \tens{K}\,\vect{u}\\
\dive(\rho \vect{u}) = \Gamma
\end{array}\right.
\end{equation}

%(plus tard $\frac{\partial \rho}{\partial t} + \dive(\rho \vect{u}) = \Gamma$)



avec $\rho$ la masse volumique, $\vect{u}$ le champ de vitesse,
$[\,\vect{TS}-\tens{K}\,\vect{u}\,]$ les autres termes sources ($\tens{K}$~est un
tenseur diagonal positif par d\'efinition), $\tens{\sigma}$ le tenseur
de contraintes, $\tens{\tau}$ le tenseur des contraintes visqueuses, $\mu$ la
viscosit\'e dynamique (mol\'eculaire et \'eventuellement turbulente), $\kappa$
la viscosit� de
volume (usuellement nulle et n�glig�e dans le code et dans la suite du document,
sauf en compressible),
$\tens{D}$ le tenseur taux de d\'eformation\footnote{\`A ne pas confondre, malgr\'e la m\^eme notation $D$,
avec les flux diffusifs $\vect{D}_{\,ij}$ et $\vect{D}_{\,{b}_{ik}}$ d\'ecrits par la suite dans ce
sous-programme.}, $\Gamma$ le terme source de masse.
\begin{equation}
\left\{\begin{array}{l}
\tens{\sigma} = \tens{\tau} - P\tens{Id}  \\
\tens{\tau} = 2\,\mu\ \tens{D} +\ (\kappa\ - \frac{2}{3}\mu)\  tr({\tens{D}})\
\tens{Id}  \\
\tens{D} = \frac{1}{2}(\ggrad\vect{u}+\,^{t}\ggrad\vect{u})
\end{array}\right.
\end{equation}
 \\

On rappelle la d\'efinition des notations employ\'ees\footnote{en
utilisant la convention de sommation d'Einstein.}~:
\begin{equation}\notag
\left\{\begin{array}{lll}
\left[\ggrad{\vect{a}}\right]_{ij} &=& \partial_j a_i\\
\left[\dive(\tens{\sigma})\right]_i &=& \partial_j \sigma_{ij}\\
\left[\vect{a}\otimes\vect{b}\right]_{ij} &= &
a_i\,b_j\\
\end{array}\right.
\end{equation}
et donc :
\begin{equation}\notag
\begin{array}{lll}
\left[\dive(\vect{a}\otimes\vect{b})\right]_i &= &
\partial_j (a_i\,b_j)
\end{array}
\end{equation}

\minititre{Remarque}
Dans le cas de la prise en compte d'une masse volumique variable, l'�quation de continuit� s'�crit :
$$\frac{\partial \rho}{\partial t} + \dive{\,(\rho\,\vect{u})} = \Gamma  $$
Cette �quation n'est pas prise en compte dans l'�tape de projection (on continue � r�soudre
seulement
$\displaystyle \dive(\,{\rho\,\vect{u}}) = \Gamma$) alors que le terme
$\displaystyle \frac{\partial \rho}{\partial t}$ appara\^{\i}t lors de l'�tape de pr\'ediction de la vitesse
dans le sous-programme \fort{preduv}. Si ce terme joue un r�le sensible, l'algorithme compressible
de \CS\ (qui r�sout l'�quation compl�te) est alors sans doute plus adapt�.

%                      Code_Saturne version 1.3
%                      ------------------------
%
%     This file is part of the Code_Saturne Kernel, element of the
%     Code_Saturne CFD tool.
% 
%     Copyright (C) 1998-2007 EDF S.A., France
%
%     contact: saturne-support@edf.fr
% 
%     The Code_Saturne Kernel is free software; you can redistribute it
%     and/or modify it under the terms of the GNU General Public License
%     as published by the Free Software Foundation; either version 2 of
%     the License, or (at your option) any later version.
% 
%     The Code_Saturne Kernel is distributed in the hope that it will be
%     useful, but WITHOUT ANY WARRANTY; without even the implied warranty
%     of MERCHANTABILITY or FITNESS FOR A PARTICULAR PURPOSE.  See the
%     GNU General Public License for more details.
% 
%     You should have received a copy of the GNU General Public License
%     along with the Code_Saturne Kernel; if not, write to the
%     Free Software Foundation, Inc.,
%     51 Franklin St, Fifth Floor,
%     Boston, MA  02110-1301  USA
%
%-----------------------------------------------------------------------
%
%%%%%%%%%%%%%%%%%%%%%%%%%%%%%%%%%
%%%%%%%%%%%%%%%%%%%%%%%%%%%%%%%%%%
\section{Discr\'etisation}
%%%%%%%%%%%%%%%%%%%%%%%%%%%%%%%%%%
%%%%%%%%%%%%%%%%%%%%%%%%%%%%%%%%%%

Pour utiliser la m�thode, on se place tout d'abord dans un rep�re local d�fini
de mani�re � ce que le plan $(0yz)$, o� sont inject�s les vortex, soit confondu
avec le plan d'entr�e du calcul (voir figure \ref{Base_Vortex_entree}). 

\begin{figure}[h]
\centerline{\includegraphics[height=6cm]{../Base/Vortex/Images/entree.pdf}}
\caption{\label{Base_Vortex_entree} D�finiton des diff�rentes grandeurs dans le rep�re local
correspondant � l'entr�e d'une conduite de section carr�e.} 
\end{figure}

$u$, $v$ et $w$  sont les composantes de la vitesse fluctuante (principale et
transverse) dans ce plan, et
$\displaystyle \omega(y,z) = \frac{\partial w}{\partial y}-\frac{\partial v}{\partial z}$
la vorticit� dans la direction
normale au plan d'entr�e. $\overline{U}(y,z)$ repr�sente ici la vitesse
principale moyenne impos�e par l'utilisateur dans le plan d'entr�e. 

Chaque vortex $p$ va �tre caract�ris� par sa fonction de forme $\xi$ (identique
pour tous les vortex), sa
circulation $\Gamma_p$, son rayon $\sigma_p$ et les coordonn�es $(y_p,z_p)$ du
point $P$ o� est situ� le vortex dans le plan $(0yz)$. 

Pour cela, on suppose que la vorticit� g�n�r�e par un vortex $p$ au point $M$ de
coordonn�e $(y,z)$ s'�crit : 
\begin{equation}\notag
\omega_p(y,z)= \Gamma_p \, \xi_{\sigma_p}(r)
\end{equation}
o� $r = \sqrt{(y-y_p)^2+(z-z_p)^2}$ est la distance s�parant le point $M$ du point $P$.

Dans la m�thode implant�e, la fonction de forme est de type gaussienne modifi�e :
\begin{equation}\notag
\displaystyle
\xi_\sigma (r) = \frac{1}{2\pi \sigma^2} 
\left(2 e^{-\frac{r^2}{2\sigma^2}}-1\right) e^{-\frac{r^2}{2\sigma^2}}
\end{equation}

Le champ de vitesse induit par cette distribution de vorticit� s'obtient par
inversion des deux �quations de poisson suivantes qui sont d�duites de la
condition d'incompressibilit� dans la plan\footnote{\textit{i.e}
$\displaystyle \frac{\partial v}{\partial y}+\frac{\partial w}{\partial w} = 0$} :
\begin{equation}\notag
\begin{array}{lcr}
\displaystyle
\frac{\partial \omega}{\partial y} = \Delta w
&
\text{    et    }
&
\displaystyle
\frac{\partial \omega}{\partial y} = -\Delta v
\\
\end{array}
\end{equation}

Dans le cas g�n�ral, ce syst�me peut �tre int�gr� � l'aide de la formule de Biot et Savart.

Dans le cas d'une distribution de vorticit� de type gaussienne modifi�e, les
composantes de vitesse v�rifient : 
\begin{equation}\notag
\left\{
\begin{array}{c}
\displaystyle
v_p(y,x) = - \frac{1}{2\pi} \frac{(z-z_p)}{r^2}\left(1 -
e^{-\frac{r^2}{2\sigma^2}}\right)\,e^{-\frac{r^2}{2\sigma^2}} 
\\
\displaystyle
w_p(y,z) = \frac{1}{2\pi} \frac{(y-y_p)}{r^2}\left(1 -e^{-\frac{r^2}{2\sigma^2}}
\right)\,e^{-\frac{r^2}{2\sigma^2}} 
\end{array}
\right.
\end{equation}

Ces relations s'�tendent de fa�on imm�diate au cas de $N$ vortex distincts.
Le champ de vitesse induit par la distribution de vorticit� 
\begin{equation}
\omega(y,z) = \sum_{p=1}^N \Gamma_p \, \xi_{\sigma_p}(r)
\end{equation}
vaut au point $M$ :
\begin{equation}\notag
\begin{array}{lcr}
v(x,y) = \sum_{p=1}^N \Gamma_p\, v_p(y,z) 
&
\text{    et    }
&
w(y,z) = \sum_{p=1}^N \Gamma_p\, w_p(y,z)
\\
\label{Base_Vortex_compvit}
\end{array}
\end{equation}
%================================
\subsection{Param�tres physiques}
%================================

%-------------------------------
\subsubsection{Marche en temps}
%-------------------------------
La position initiale de chaque vortex est tir�e de mani�re al�atoire. On calcul
les d�placements successifs de chacun des vortex dans le plan d'entr�e par
int�gration explicite du champ de vitesse lagrangien : 
\begin{equation}\notag
\begin{array}{lcr}
\displaystyle
\frac{dy_p}{dt} = V(y,z)
&
\text{    et    }
&
\displaystyle
\frac{dz_p}{dt} = W(y,z)
\\
\end{array}
\end{equation}
Les vortex sont alors assimil�s � des particules ponctuelles qui sont convect�es
par le champ $(V,W)$. Ce champ peut �tre impos� par des tirages al�atoires ou
bien d�duit de la vitesse induite par les vortex dans le plan d'entr�e. Dans ce
cas $V(x,y) = \overline{V}(y,z) + v (y,z)$ et $W(y,z)= \overline{W}(y,z) +
w(y,z)$ o� $\overline{V}$ et $\overline{W}$ sont les composantes de la vitesse
transverse moyenne qu'impose l'utilisateur � l'aide des fichiers de donn�es. 

%---------------------------------------------------
\subsubsection{Intensit� et dur�e de vie des vortex}
%---------------------------------------------------
Il serait possible, � partir de l'�quation de transport de la vorticit�,
d'obtenir un mod�le d'�volution pour l'intensit� du vecteur tourbillon
$\omega_p$ associ� � chacun des vortex. En pratique, on pr�f�re utiliser un
mod�le simplifi� dans lequel la circulation des tourbillons ne d�pend que de la
postion de ces derniers � l'instant consid�r�. La circulation initiale de chaque
vortex est alors obtenue � partir de la relation : 
\begin{equation}\notag
|\Gamma_p| = 4 \sqrt{\frac{\pi\,S\,k}{3N\,[2ln(3)-3ln(2)]}}
\end{equation}
o� $S$ est la surface du plan d'entr�e, $N$ le nombre de vortex, et $k$
l'�nergie cin�tique turbulente au point o� se trouve le vortex � l'instant
consid�r�. Le signe de $\Gamma_p$ correspond au sens de rotation du vortex et
est tir� al�atoirement. 

Ce param�tre est celui qui contr�le l'intensit� des fluctuations. Sa d�pendance
en $k$ exprime que, plus l'�coulement est turbulent, plus les vortex sont
intenses. La valeur de $k$ est sp�cifi�e par
l'utilisateur. Elle peut �tre constante ou impos�e � partir de profils d'�nergie
cin�tique turbulente en entr�e. 

Pour �viter que des structures trop allong�es ne se d�veloppent au niveau de
l'entr�e, l'utilisateur doit �galement sp�cifier un temps limites $\tau_p$ au
bout duquel le vortex $p$ va �tre d�truit. Ce temps $\tau_p$ peut �tre pris
constant ou estim� au moyen de la relation : 
\begin{equation}\notag
\tau_p = \frac{5 C_{\mu}k^{\frac{3}{2}}}{\varepsilon\,\overline{U}}
\end{equation}

$\overline{U}$ et $\varepsilon$ repr�sentent respectivement la vitesse moyenne
principale et la dissipation turbulente au point o� est initialement g�n�r� le
vortex ($C_{\mu}=0,09$). 
\\
Lorsque le temps �coul� depuis la cr�ation du vortex $p$ est sup�rieur �
$\tau_p$, le vortex est d�truit et un nouveau vortex g�n�r� (sa position et le
signe de $\Gamma_p$ sont tir�s de fa�on al�atoire). 

%-------------------------------- 
\subsubsection{Taille des vortex}
%--------------------------------
La taille des vortex peut �tre prise constante, ou calcul�e � partir des
relations :
\begin{equation}\notag
\begin{array}{ccc}
\displaystyle
\sigma = \frac{C_{\mu}^{\frac{3}{4}}k^{\frac{3}{2}}}{\varepsilon} 
& \text{    ou    } &
\sigma = max[L_t,L_k]
\\
\end{array}
\end{equation}
avec:
\begin{equation}\notag
\begin{array}{ccc}
\displaystyle
L_t = \sqrt{\left( \frac{5 \nu k}{\varepsilon} \right)} 
& \text{    et    } & 
\displaystyle
L_k = 200\, \left(\frac{\nu^3}{\varepsilon}\right)^{\frac{1}{4}}
\end{array}
\end{equation}
o� $\nu$, $k$ et $\varepsilon$ sont la viscosit� dynamique, l'�nergie cin�tique
turbulente et la dissipation turbulente au point o� se trouve le vortex. 

Dans tous les cas, la taille du vortex doit �tre sup�rieure � la taille des
mailles en entr�e afin que le vortex soit effectivement simul�. 

%==================================
\subsection{Conditions aux limites}
%==================================
Le champ de vitesse g�n�r� � l'aide de la relation \ref{Base_Vortex_compvit} ne tient pas
compte {\em a priori} des conditions aux limites appliqu�es sur les bords du plan
d'entr�e. Pour obtenir des valeurs de la vitesse qui soient coh�rentes sur les
fronti�res du domaine d'entr�e, des ``vortex images'', pseudo-vortex situ�s en
dehors du domaine d'entr�e, sont g�n�r�s � des positions particuli�res et leur
champ de vitesse associ� est superpos� au champ pr�c�demment calcul�.\\
Seuls les cas d'une conduite rectangulaire et d'une conduite circulaire
permettent la g�n�ration de ces pseudo-vortex.
On distingue pour cela trois types de conditions aux limites. 

\begin{figure}[h]
\centerline{\includegraphics[height=6cm]{../Base/Vortex/Images/condlimite.pdf}}
\caption{\label{Base_Vortex_condli} Principe de g�n�ration des ``vortex images'' suivant le
type de conditions aux limites dans une conduite carr�e.} 
\end{figure}

%----------------------------------
\subsubsection{Condition de paroi}
%----------------------------------
On cr�e, pour chaque vortex $P$ contenu dans le plan d'entr�e, un vortex image
$P'$ identique � $P$ (\textit{i.e} de m�me caract�ristiques) et sym�trique de $P$
par rapport au
point $J$ ($J$ �tant la projection orthogonalement � la paroi du point $M$
correspondant au centre de la face o� l'on cherche � calculer la vitesse). La
figure \ref{Base_Vortex_condli} illustre la technique dans le cas d'une conduite
carr�e. Dans ce cas les coordonn�es du vortex situ� en $P'$ v�rifient
$(y_{p'}+y_{p})/2 = y_{J}$ et $(z_{p'}+ z_{p})/2 = z_{J}$. Le champ de vitesse
per�u depuis le point $M$ au niveau du point $J$ est nul, ce qui est bien
l'effet recherch�. 

%------------------------------------
\subsubsection{Condition de sym�trie}
%-------------------------------------
La technique est identique � celle utilis�e pour les conditions de paroi, mais
seule la composante pour la vitesse normale au bord est modifi�e dans ce cas. 

%---------------------------------------
\subsubsection{Condition de p�riodicit�}
%---------------------------------------
On cr�e pour chaque vortex, un vortex images $P'$ identique � $P$ mais translat�
d'une quantit� $L$ correspondant � la longueur qui s�pare les deux plans de la
section d'entr�e o� sont appliqu�es les conditions de p�riodicit�. Dans le cas
o� il y a deux directions de p�riodicit�, on cr�e deux vortex image.

%=============================================
\subsection{Composante de vitesse principale}
%=============================================
La m�thode des vortex ne g�n�rant pas de fluctuation $u$ de la vitesse dans la
direction principale, la derni�re composante est calcul�e � partir d'une
�quation de Langevin. Les coefficients de cette �quation sont d�termin�s par
identification des expressions obtenues pour les contraintes de Reynolds en
$R_{ij}-\varepsilon$. Dans le cas d'un �coulement en canal plan, cette technique
conduit � l'�quation : 
\begin{equation}\notag
\displaystyle
\frac{du}{dt} = - \frac{C_1}{2T} u + \left(\frac{2}{3}C_2-1\right)\frac{\partial
U}{\partial y} v + \sqrt{C_0\varepsilon} dW_i 
\end{equation}

avec $\displaystyle T=\frac{k}{\varepsilon}$, $C_1 = 1,8$, $C_2 = 0,6$,
$C_0=\frac{14}{15}$, et $dW_i$ une variable al�toire Gaussienne de variance
$\sqrt{dt}$. 

En pratique, l'�quation de Langevin n'am�liore pas vraiment les r�sultats. Elle
n'est utilis�e que dans le cas de conduites circulaires. 

%                      Code_Saturne version 1.3
%                      ------------------------
%
%     This file is part of the Code_Saturne Kernel, element of the
%     Code_Saturne CFD tool.
%
%     Copyright (C) 1998-2007 EDF S.A., France
%
%     contact: saturne-support@edf.fr
%
%     The Code_Saturne Kernel is free software; you can redistribute it
%     and/or modify it under the terms of the GNU General Public License
%     as published by the Free Software Foundation; either version 2 of
%     the License, or (at your option) any later version.
%
%     The Code_Saturne Kernel is distributed in the hope that it will be
%     useful, but WITHOUT ANY WARRANTY; without even the implied warranty
%     of MERCHANTABILITY or FITNESS FOR A PARTICULAR PURPOSE.  See the
%     GNU General Public License for more details.
%
%     You should have received a copy of the GNU General Public License
%     along with the Code_Saturne Kernel; if not, write to the
%     Free Software Foundation, Inc.,
%     51 Franklin St, Fifth Floor,
%     Boston, MA  02110-1301  USA
%
%-----------------------------------------------------------------------
%

%%%%%%%%%%%%%%%%%%%%%%%%%%%%%%%%%%
%%%%%%%%%%%%%%%%%%%%%%%%%%%%%%%%%%
\section{Mise en \oe uvre}
%%%%%%%%%%%%%%%%%%%%%%%%%%%%%%%%%%
%%%%%%%%%%%%%%%%%%%%%%%%%%%%%%%%%%
Le syst\`eme (\ref{Cfbl_Cfmsvl_eq_densite_finale_cfmsvl}) est r\'esolu par une m\'ethode
d'incr\'ement et r\'esidu en utilisant
une m\'ethode de Jacobi pour inverser le syst\`eme si le terme convectif
est implicite et en utilisant une m\'ethode de gradient conjugu\'e
si le terme convectif est explicite (qui est le cas par d�faut).

Attention, les valeurs du flux de masse $\rho\,\vect{w}\cdot\vect{S}$ et
de la viscosit\'e $\Delta\,t\,c^2\frac{S}{d}$ aux faces de
bord, qui sont calcul\'ees dans \fort{cfmsfl} et \fort{cfmsvs} respectivement,
sont modifi\'ees imm\'ediatement apr\`es l'appel \`a ces sous-programmes.
En effet, il est indispensable que la contribution de bord de
$\left(\rho\,\vect{w}-\Delta\,t\,(c^2)\,\gradv\,\rho\right)\cdot\vect{S}$
repr\'esente exactement $\vect{Q}_{ac}\cdot\vect{S}$.
Pour cela,
\begin{itemize}
\item imm\'ediatement apr\`es l'appel \`a
\fort{cfmsfl}, on remplace la contribution de bord de
$\rho\,\vect{w}\cdot\vect{S}$
par le flux de masse exact, $\vect{Q}_{ac}\cdot\vect{S}$,
d\'etermin\'e \`a partir des conditions aux limites,
\item puis, imm\'ediatement apr\`es l'appel \`a
\fort{cfmsvs}, on annule la viscosit\'e au bord $\Delta\,t\,(c^2)$ pour
\'eliminer la contribution de $-\Delta\,t\,(c^2)\,(\gradv\,\rho)\cdot\vect{S}$
(l'annulation de la viscosit\'e n'est pas probl\'ematique pour la matrice,
puisqu'elle porte sur des incr\'ements).
\end{itemize}

\bigskip

Une fois qu'on a obtenu $\rho^{n+1}$,
on peut actualiser le flux de masse acoustique
aux faces $(\vect{Q}_{ac}^{n+1})_{ij} \cdot \vect{S}_{ij}$,
qui servira pour la convection des autres variables~:
\begin{equation}\label{Cfbl_Cfmsvl_eq_flux_masse_acoustique_cfmsvl}
\displaystyle(\vect{Q}_{ac}^{n+1})_{ij}\cdot\vect{S}_{ij}=
-\left(\Delta t^n (c^2)^n \gradv(\rho^{n+1})\right)_{ij}\cdot\vect{S}_{ij}
+\left(\rho^{n+\frac{1}{2}} \vect{w}^n\right)_{ij}\cdot\vect{S}_{ij}\\
\end{equation}
Ce calcul de flux est r\'ealis\'e par \fort{cfbsc3}.
Si l'on a choisi l'algorithme standard, \'equation~(\ref{Cfbl_Cfmsvl_eq_densite_cfmsvl}),
on compl\`ete le flux dans \fort{cfmsvl} imm\'ediatement apr\`es l'appel
\`a \fort{cfbsc3}.
En effet, dans ce cas,
la convection est explicite ($\rho^{n+\frac{1}{2}}=\rho^{n}$,
obtenu en imposant \var{ICONV(ISCA(IRHO(IPHAS)))=0})
et le sous-programme \fort{cfbsc3},
qui calcule le flux de masse aux faces,
ne prend pas en compte la contribution du terme
$\rho^{n+\frac{1}{2}}\,\vect{w}^n\cdot\vect{S}$. On ajoute donc cette
contribution dans \fort{cfmsvl}, apr\`es l'appel \`a \fort{cfbsc3}.
Au bord, en particulier, c'est bien le flux de masse calcul\'e \`a partir
des conditions aux limites que l'on obtient.

On actualise la pression \`a la fin de l'\'etape, en utilisant la loi d'\'etat~:
\begin{equation}
\displaystyle P_i^{pred}=P(\rho_i^{n+1},\varepsilon_i^{n})
\end{equation}


%%%%%%%%%%%%%%%%%%%%%%%%%%%%%%%%%%
%%%%%%%%%%%%%%%%%%%%%%%%%%%%%%%%%%
\section{Points \`a traiter}
%%%%%%%%%%%%%%%%%%%%%%%%%%%%%%%%%%
%%%%%%%%%%%%%%%%%%%%%%%%%%%%%%%%%%
Le calcul du flux de masse au  bord n'est pas enti\`erement satisfaisant
si la convection est trait\'ee de mani\`ere implicite
(algorithme non standard, non test\'e,
associ\'e \`a l'\'equation~(\ref{Cfbl_Cfmsvl_eq_densite_bis_cfmsvl}),
correspondant au choix $\rho^{n+\frac{1}{2}}=\rho^{n+1}$ et
obtenu en imposant \var{ICONV(ISCA(IRHO(IPHAS)))=1}).
En effet, apr\`es \fort{cfmsfl}, il faut d\'eterminer la vitesse de
convection $\vect{w}^n$ pour qu'apparaisse
$\rho^{n+1} \vect{w}^n\cdot\vect{n}$
au cours de la r\'esolution par \fort{codits}. De ce fait, on doit d\'eduire
une valeur de $\vect{w}^n$ \`a partir de la valeur
du flux de masse. Au bord, en particulier, il faut
donc diviser le flux de masse
issu des conditions aux limites par la valeur de bord de $\rho^{n+1}$.
Or, lorsque des conditions de Neumann sont appliqu\'ees \`a la
masse volumique,
la valeur de $\rho^{n+1}$ au bord n'est pas connue avant la r\'esolution du
syst\`eme. On utilise donc, au lieu de la valeur de bord inconnue de
$\rho^{n+1}$ la valeur de bord prise au pas de temps
pr\'ec\'edent $\rho^{n}$. Cette approximation est susceptible
d'affecter la valeur du flux de masse au bord.

\passepage
\part{Module combustion}
%                      Code_Saturne version 1.3
%                      ------------------------
%
%     This file is part of the Code_Saturne Kernel, element of the
%     Code_Saturne CFD tool.
%
%     Copyright (C) 1998-2007 EDF S.A., France
%
%     contact: saturne-support@edf.fr
%
%     The Code_Saturne Kernel is free software; you can redistribute it
%     and/or modify it under the terms of the GNU General Public License
%     as published by the Free Software Foundation; either version 2 of
%     the License, or (at your option) any later version.
%
%     The Code_Saturne Kernel is distributed in the hope that it will be
%     useful, but WITHOUT ANY WARRANTY; without even the implied warranty
%     of MERCHANTABILITY or FITNESS FOR A PARTICULAR PURPOSE.  See the
%     GNU General Public License for more details.
%
%     You should have received a copy of the GNU General Public License
%     along with the Code_Saturne Kernel; if not, write to the
%     Free Software Foundation, Inc.,
%     51 Franklin St, Fifth Floor,
%     Boston, MA  02110-1301  USA
%
%-----------------------------------------------------------------------
%


\programme{navsto}

\vspace{1cm}
On s'int\'eresse \`a la r\'esolution du syst\`eme d'\'equations de Navier-Stokes
tridimensionnelles monophasiques, \`a une pression, instationnaires, en
incompressible ou faiblement dilatable, bas\'ees sur une discr\'etisation
temporelle de type Euler implicite d'ordre 1 ou Crank-Nicolson d'ordre 2 et sur
une discr\'etisation spatiale  par volumes finis colocalis\'ee.


%%%%%%%%%%%%%%%%%%%%%%%%%%%%%%%%%%
%%%%%%%%%%%%%%%%%%%%%%%%%%%%%%%%%%
\section{Fonction}
%%%%%%%%%%%%%%%%%%%%%%%%%%%%%%%%%%
%%%%%%%%%%%%%%%%%%%%%%%%%%%%%%%%%%

  Dans ce sous-programme sont calcul\'ees, \`a un pas de temps donn\'e, les
variables vitesse et pression de ce probl\`eme en proc\'edant en
deux  \'etapes issues d'une d\'ecomposition des op\'erateurs (m\'ethode \`a
pas fractionnaires).\\
Les variables sont donc suppos\'ees connues \`a
l'instant ${t^n}$ et on cherche \`a les d\'eterminer \`a l'instant\footnote{La pression est suppos�e connue � l'instant $t^{n-1+\theta}$ et recherch�e en $t^{n+\theta}$, avec $\theta=1$ ou $1/2$ suivant le sch�ma en temps consid�r�.} ${t^{n+1}}$. Soit ${\Delta {t^n} ={t^{n+1}- {t^n}}}$ le pas de temps associ\'e. Dans un premier temps, on r\'ealise l'\'etape de
pr\'ediction de la vitesse en r\'esolvant l'\'equation de quantit\'e de
mouvement avec une pression explicite. Suit l'\'etape de correction de la
pression (ou projection de la vitesse) qui permet d'obtenir un champ de vitesse \`a divergence nulle.\\\\
Les \'equations en continu sont donc :
\begin{equation}
\left\{\begin{array}{l}
\displaystyle\frac{\partial}{\partial t}(\rho \vect{u}) + \dive(\rho\, \vect{u} \otimes \vect{u})
=\dive(\tens{\sigma}) + \vect{TS} - \tens{K}\,\vect{u}\\
\dive(\rho \vect{u}) = \Gamma
\end{array}\right.
\end{equation}

%(plus tard $\frac{\partial \rho}{\partial t} + \dive(\rho \vect{u}) = \Gamma$)



avec $\rho$ la masse volumique, $\vect{u}$ le champ de vitesse,
$[\,\vect{TS}-\tens{K}\,\vect{u}\,]$ les autres termes sources ($\tens{K}$~est un
tenseur diagonal positif par d\'efinition), $\tens{\sigma}$ le tenseur
de contraintes, $\tens{\tau}$ le tenseur des contraintes visqueuses, $\mu$ la
viscosit\'e dynamique (mol\'eculaire et \'eventuellement turbulente), $\kappa$
la viscosit� de
volume (usuellement nulle et n�glig�e dans le code et dans la suite du document,
sauf en compressible),
$\tens{D}$ le tenseur taux de d\'eformation\footnote{\`A ne pas confondre, malgr\'e la m\^eme notation $D$,
avec les flux diffusifs $\vect{D}_{\,ij}$ et $\vect{D}_{\,{b}_{ik}}$ d\'ecrits par la suite dans ce
sous-programme.}, $\Gamma$ le terme source de masse.
\begin{equation}
\left\{\begin{array}{l}
\tens{\sigma} = \tens{\tau} - P\tens{Id}  \\
\tens{\tau} = 2\,\mu\ \tens{D} +\ (\kappa\ - \frac{2}{3}\mu)\  tr({\tens{D}})\
\tens{Id}  \\
\tens{D} = \frac{1}{2}(\ggrad\vect{u}+\,^{t}\ggrad\vect{u})
\end{array}\right.
\end{equation}
 \\

On rappelle la d\'efinition des notations employ\'ees\footnote{en
utilisant la convention de sommation d'Einstein.}~:
\begin{equation}\notag
\left\{\begin{array}{lll}
\left[\ggrad{\vect{a}}\right]_{ij} &=& \partial_j a_i\\
\left[\dive(\tens{\sigma})\right]_i &=& \partial_j \sigma_{ij}\\
\left[\vect{a}\otimes\vect{b}\right]_{ij} &= &
a_i\,b_j\\
\end{array}\right.
\end{equation}
et donc :
\begin{equation}\notag
\begin{array}{lll}
\left[\dive(\vect{a}\otimes\vect{b})\right]_i &= &
\partial_j (a_i\,b_j)
\end{array}
\end{equation}

\minititre{Remarque}
Dans le cas de la prise en compte d'une masse volumique variable, l'�quation de continuit� s'�crit :
$$\frac{\partial \rho}{\partial t} + \dive{\,(\rho\,\vect{u})} = \Gamma  $$
Cette �quation n'est pas prise en compte dans l'�tape de projection (on continue � r�soudre
seulement
$\displaystyle \dive(\,{\rho\,\vect{u}}) = \Gamma$) alors que le terme
$\displaystyle \frac{\partial \rho}{\partial t}$ appara\^{\i}t lors de l'�tape de pr\'ediction de la vitesse
dans le sous-programme \fort{preduv}. Si ce terme joue un r�le sensible, l'algorithme compressible
de \CS\ (qui r�sout l'�quation compl�te) est alors sans doute plus adapt�.

%
%%%%%%%%%%%%%%%%%%%%%%%%%%%%%%%%%%%%%%%%%%%%%%%%%%%%%%%%%%%%%%%%%%%%%%

%

%%%%%%%%%%%%%%%%%%%%%%%%%%%%%%%%%%%%%%%%%%%%%%%%%%%%%%%%%%%%%%%%%%%%%%

%
%%%%%%%%%%%%%%%%%%%%%%%%%%%%%%%%%%%%%%%%%%%%%%%%%%%%%%%%%%%%%%%%%%%%%%
% FIN DU DOCUMENT
\end{document}
%
%%%%%%%%%%%%%%%%%%%%%%%%%%%%%%%%%%%%%%%%%%%%%%%%%%%%%%%%%%%%%%%%%%%%%%

%%% Local Variables: 
%%% mode: latex
%%% TeX-master: None
%%% End: 

